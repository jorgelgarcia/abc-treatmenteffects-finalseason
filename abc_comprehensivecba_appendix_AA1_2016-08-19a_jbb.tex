%Input preamble
\documentclass[11pt]{article}

% colors
\usepackage[table]{xcolor}
\definecolor{maroon}{RGB}{153,0,18}
\definecolor{lime}{RGB}{190,213,88}
\definecolor{sand}{RGB}{217,202,179}
\definecolor{fire}{RGB}{144,50,61}
\definecolor{brick}{RGB}{94,11,21}
\definecolor{olive}{RGB}{117,109,84}
\definecolor{lavpink}{RGB}{172,123,132}
\definecolor{darkpurp}{RGB}{49,10,49}
\definecolor{salmon}{RGB}{204,90,113}
\definecolor{mauve}{RGB}{94,73,85}
\definecolor{greyblue}{RGB}{125,132,145}
\definecolor{greypurp}{RGB}{68,56,80}
\definecolor{brightpurp}{RGB}{96,20,255}

% packages (please add in alphabetical order)
\usepackage{adjustbox}
\usepackage{amsfonts}
\usepackage{amsmath}
\usepackage{amssymb}
\usepackage{array}
\usepackage{bm}
\usepackage{booktabs}
\usepackage{caption}
\usepackage{epstopdf}
\usepackage{float}
\usepackage[margin=1in]{geometry}
\usepackage{graphicx}
\usepackage[colorlinks=true, linkcolor=brightpurp, citecolor=brightpurp, urlcolor=salmon]{hyperref}
\usepackage{lipsum}
\usepackage{longtable}
\usepackage{mathtools}
\usepackage{multirow}
\usepackage{natbib}
\usepackage{rotating}
\usepackage{setspace}
\usepackage{subcaption}
%\usepackage{threeparttable}
\usepackage{threeparttablex}
\usepackage{xr}
\usepackage[printwatermark]{xwatermark}


\newcolumntype{L}[1]{>{\raggedright\let\newline\\\arraybackslash\hspace{0pt}}m{#1}}
\newcolumntype{C}[1]{>{\centering\let\newline\\\arraybackslash\hspace{0pt}}m{#1}}
\newcolumntype{R}[1]{>{\raggedleft\let\newline\\\arraybackslash\hspace{0pt}}m{#1}}

% commands
\newcommand{\mr}{\multirow}
\newcommand{\mc}{\multicolumn}

%Other parameters
\newcommand{\noutcomes}{95}
\newcommand{\treatsubsabc}{$75\%$}
\newcommand{\treatsubscarec}{$74\%$}
\newcommand{\treatsubscaref}{$63\%$}

%Counts
%Males
\newcommand{\positivem}{$79\%$}
\newcommand{\positivesm}{$37\%$}

%Females
\newcommand{\positivef}{$73\%$}
\newcommand{\positivesf}{$35\%$}

%Counts, control substitution
%Males
\newcommand{\positivecsnm}{$58\%$}
\newcommand{\positivescsnm}{$25\%$}

\newcommand{\positivecsam}{$74\%$}
\newcommand{\positivescsam}{$38\%$}

%Females
%% no alternative
\newcommand{\positivecsnf}{$83\%$}
\newcommand{\positivescsnf}{$46\%$}

%% alternative
\newcommand{\positivecsaf}{$73\%$}
\newcommand{\positivescsaf}{$23\%$}

%Pooled

%Effects
%Males

%Females
\newcommand{\hsgradf}{$7$}
\newcommand{\yearsedf}{$1.2$}



%Pooled

%CBA
%IRR
%Males
\newcommand{\irrm}{$15\%$}
\newcommand{\irrsem}{$13\%$}

%Females
\newcommand{\irrf}{$10\%$}
\newcommand{\irrsef}{$12\%$}

%Pooled
\newcommand{\irrp}{$13\%$}
\newcommand{\irrsep}{$11\%$}

%BC
%Males
\newcommand{\bcm}{$7.88$}
\newcommand{\bcsem}{$8.06$}

%Females
\newcommand{\bcf}{$2.30$}
\newcommand{\bcsef}{$1.56$}

%Pooled
\newcommand{\bcp}{$4.35$}
\newcommand{\bcsep}{$2.57$}

%NPV streams
%Pooled
\newcommand{\parincomenpvp}{$\$115,026$}

\externaldocument{abc_comprehensivecba_2016-08-21b_jlg.tex}
\pagenumbering{roman}

\begin{document}
\title{\Large \textbf{Appendix (AA-1): \\ The Long-Term Economic and Social Benefits of an Influential Early Childhood Intervention}}

\author{
Jorge Luis Garc\'{i}a\\
The University of Chicago \and
James J. Heckman \\
American Bar Foundation \\
The University of Chicago \and
Andr\'{e}s Hojman\\
The University of Chicago \and
Duncan Ermini Leaf \\
University of Southern California \and
Mar\'{i}a Jos\'{e} Prados \\
University of Southern California \and
Joshua Shea \\
The University of Chicago \and
Jake C. Torcasso \\
The University of Chicago}
\date{First Draft: January 5, 2016\\ This Draft: \today}
\maketitle
\thispagestyle{empty}

\clearpage

%\pagebreak

\pagebreak
\doublespacing

%Input Appendices
\pagenumbering{arabic}
\begin{appendices}
\setcounter{figure}{0}  \renewcommand{\thefigure}{A.\arabic{figure}}
\setcounter{table}{0}   \renewcommand{\thetable}{A.\arabic{table}}

For both ABC and CARE, FPGC was open to families from 7:45 a.m. to 5:30 p.m., 5 days per week and 50 weeks per year.\footnote{\citet{Ramey_Collier_etal_1976_CarolinaAbecedarianProject, Ramey_etal_1985_Project-CARE_TiECSE}.} Subjects were offered free transportation to and from the center. A driver and second adult staffed each vehicle (one van and two station wagons) equipped with child safety seats.\footnote{\citet{Ramey_Campbell_1979_SR,abc2014-2015interviews}.} Data provided by FPGC indicate that approximately 65\% of treated ABC families utilized the free transportation.\footnote{\citet{Barnett_Masse_2002_benefitcost}.} At FPGC, ABC and CARE treatment-group subjects received breakfast, lunch, and a snack planned by a nutritionist.\footnote{ \citet{Haskins_1985_CD, Bryant_et_al_1987_Carolina_Approach_TIECSE, Ramey-et-al_1977_Intro-to-ABC}.}

To promote trust in FPGC within the sample families, staff were recruited from the local community.\footnote{\citet{Ramey-et-al_1977_Intro-to-ABC, Bryant_et_al_1987_Carolina_Approach_TIECSE, Feagans_1996_Childrens-Talk,abc2014-2015interviews}.} Infant and toddler caregivers and preschool teachers demonstrated varied educational backgrounds ranging from high school graduation to master's degrees. Their average professional working experience with young children was 7 years.\footnote{\citet{Ramey_McGinness_etal_1982_Abecedarianapproach, Ramey_etal_1985_Project-CARE_TiECSE, Wasik_Ramey_etal_1990_CD}.} All classroom staff participated in extensive training and were closely observed by FPGC's academic staff, as part of a broad variety of ongoing clinical and social research related to early childhood education, psychology, and health. In ABC, child-caregiver ratios varied by age: 3:1 for infants up to 13 to 15 months of age; 4:1 for toddlers up to 36 months; and 5:1 or 6:1 for children aged 3 to 5 years, depending on cohort size.\footnote{\citet{Ramey-et-al_1977_Intro-to-ABC,Ramey_Campbell_1979_SR,Ramey_McGinness_etal_1982_Abecedarianapproach}.} Child-caregiver ratios were similar in CARE.\footnote{\citet{Burchinal_Campbell_etal_1997_CD, Ramey_etal_1985_Project-CARE_TiECSE}.}




\end{appendices}

%References
\renewcommand{\refname}{Appendix References}
\clearpage
\singlespace
\bibliographystyle{chicago}
\bibliography{heckman}

\end{document} 