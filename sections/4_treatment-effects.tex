This section presents our estimates of treatment effects by gender. We categorize outcomes and present estimates of treatment effects pooled within each category. Treatment effects for individual outcome variables are listed in Appendix~\ref{appendix:results}.

\begin{table}[!htpb]
\begin{threeparttable}
\caption{Combining Functions and Exact Non-Parametric Tests} \label{table:massiveall}
\centering
\begin{tabular}{l c c c c}
\toprule
 & Average & \% $ >0 $ & \% $ >0 $ , Significant & \citet{Rosenbaum_2005_Distribution_JRSS} \\
 & Effect Size & Treatment Effect & Treatment Effect & $ p $ -value \\
\midrule
\textbf{IQ} & & & & \\
\quad Females &  \textbf{    0.719} & \textbf{  100.000} & \textbf{  100.000} & \textbf{ 0.046} \\
\quad Males &  \textbf{    0.664} & \textbf{  100.000} & \textbf{   85.714} & \textbf{ 0.045} \\
\midrule
\textbf{Achievement} & & & & \\
\quad Females &  \textbf{    0.672} & \textbf{  100.000} & \textbf{  100.000} & \textbf{ 0.046} \\
\quad Males &  \textbf{    0.235} & \textbf{  100.000} & 40.000 & \textbf{ 0.086} \\
\midrule
\textbf{Social-emotional} & & & & \\
\quad Females &  \textbf{    0.385} & \textbf{   92.857} & \textbf{   71.429} & 0.235 \\
\quad Males &      0.059 & \textbf{   50.000} & \textbf{   21.429} & 0.147 \\
\midrule
\textbf{Parental Income} & & & & \\
\quad Females & \textbf{    0.283} & \textbf{  100.000} & \textbf{   37.500}  & \textbf{ 0.086} \\
\quad Males &  \textbf{    0.157} & \textbf{  100.000} & \textbf{   25.000}  & 0.147 \\
\midrule
\textbf{Parenting} & & & & \\
\quad Females &  \textbf{    0.274} & \textbf{  100.000} & \textbf{   100.000} & 0.602 \\
\quad Males &      0.060 & \textbf{   80.000} &     0.000 & 0.147 \\
\midrule
\textbf{Education} & & & & \\
\quad Females &  \textbf{    0.356} & \textbf{   83.333} & \textbf{   66.667} & \textbf{ 0.000} \\
\quad Males &  \textbf{    0.174} & \textbf{   83.333} & \textbf{   16.667} & 0.235 \\
\midrule
\textbf{Employment} & & & & \\
\quad Females &      0.200 & \textbf{  100.000} & \textbf{   50.000} & 0.151 \\
\quad Males &  \textbf{    0.438} & \textbf{  100.000} & \textbf{  100.000} & \textbf{ 0.022} \\
\midrule
\textbf{Crime} & & & & \\
\quad Females &  \textbf{    0.242} & \textbf{  100.000} & \textbf{   100.000} & 0.715 \\
\quad Males &     -0.093 & \textbf{   33.333} &     0.000 & 0.812 \\
\midrule
\textbf{Risky Behavior} & & & & \\
\quad Females &      0.099 & \textbf{  100.000} & \textbf{   0.000} & 0.469 \\
\quad Males &      0.011 & \textbf{   25.000} & \textbf{   25.000} & \textbf{ 0.086} \\
\midrule
\textbf{Health} & & & & \\
\quad Females &      0.060 & \textbf{   68.750} & \textbf{   6.250} & \textbf{0.046} \\
\quad Males &      0.061 & \textbf{   73.333} & \textbf{   420.000} & \textbf{0.000} \\
\bottomrule
\end{tabular}
% This file generated by: /scripts/abccare/genderdifferences/abccare-gdiff-raw-rosenbaum-table-big.do

\begin{tablenotes}
\scriptsize
\item \textbf{Note:} This table displays summaries of treatment effects by outcome category and gender. Each of the panels contains statistics calculated using outcomes grouped by category. The average effect size is calculated by averaging over the effect size of the outcomes in the outcome category. The effect sizes of the individual outcomes are calculated by dividing the treatment-control mean difference by the standard deviation of the control group. We present bootstrapped $p$-values. For the proportion of outcomes that are positive and significant, we do a ``double bootstrap'' procedure. The null hypothesis for the average effect sizes is that they are 0. The null hypothesis for the proportion of outcomes that are (significantly) positive is that they are (10\%) 50\%. Bolded statistics are significant at the 10\% level. The \citet{Rosenbaum_2005_Distribution_JRSS} $p$-value tests the null of equality of treatment and control distributions in each category. For computational simplicity, we approximate the exact $p$-values using asymptotic $p$-values. \citet{Rosenbaum_2005_Distribution_JRSS} presents several simulation exercises showing that the validity of this approximation. Statistics significant at the $0.10$ level are bolded.
\end{tablenotes}
\end{threeparttable}
\textbf{[(Regarding Row ``Parental Income'' and ``Parenting'') Please note that treatment effect for females is larger than for males.]}
\end{table}

Table~\ref{table:massiveall} aggregates treatment effects across all ages and within categories. The benefits of treatment are noticeable for both males and females. Benefits appear across the life cycle and across multiple outcomes. Participants in ABC/CARE benefit in terms of both cognitive and socio-emotional skills. They also benefit in terms of scores on achievement tests, which help measure both cognitive and non-cognitive skills.\footnote{\citet{Almlund_Duckworth_etal_2011_ecoval}.} These estimates reveal a clear female advantage in the program's effect on skill development.

ABC/CARE offered full day child care for participants and thus facilitated maternal employment and education. The program has a sizable effect on ``education.''\footnote{In Appendix~\ref{appendix:results}, we show the effects on mother's employment individually. Table~\ref{table:abccare_rslt_female_cat4} shows that the effect on females is large across ages compared to those who stayed at home. This is also seen for males (Table~\ref{table:abccare_rslt_male_cat4}), although the effect on males does not resist adjustments for multiple hypothesis testing (Table~\ref{table:abccare_rslt_male_cat4_sd}). The female results are robust (Table~\ref{table:abccare_rslt_female_cat4_sd}). This goes against the findings of \citet{Havnes_Mogstad_2011_JPE}, which find that subsidized child care does not increase maternal labor supply, but is consistent with several other studies finding an increase in maternal labor supply as a result of subsidized child care \citep{Bauernschuster_Schlotter_2015_JPE,Bettendorf_etal_2015_LE,Geyer_etal_2015_LE,Brilli_etal_2016_REH}.} The effect size is 0.356 for females --- about twice that for males, 0.174. The program enhanced parental income.\footnote{The Home Observation for the Measurement of the Environment (HOME) was collected on the ABC/CARE subjects annually until age 5 and at age 8. To administer it, a trained researcher visited the homes to observe how the mother and child interacted, using a rubric of items that capture different dimensions of parent-child interactions. Up to 3 years of age, the HOME score measures maternal warmth, absence of punishment, organization of the environment, provision of appropriate toys, maternal involvement with child, and opportunity for variety. Up to 5 years of age, the HOME score measures stimulation through toys and experiences, stimulation of mature behavior, physical and language environment, avoidance of restriction, pride and affection, masculine stimulation, and independence from parental control. At 8 years of age, the HOME score measures organization of a stable environment, developmental stimulation, quality of the language environment, need gratification, fostering maturity, emotional climate, breadth of experience, aspects of physical environment, and play materials \citep{Bradley-Caldwell_1977_AJMD}.} The program enhanced parenting as measured by HOME scores for children and their mothers between ages 0 and 8. The effect size on parenting for boys is smaller than that for girls (0.06 vs. 0.274), in part due to the fact that the HOME measurement depends on punishment and boys are more likely to be punished than girls. The families of boys scored lower in this dimension than the families of girls.

In outcomes like education, employment, crime, risky behavior (which includes, for example, drug use) there are also sizable treatment effects. A companion paper, \citet{Garcia_Heckman_Leaf_etal_2017_Comp_CBA_Unpublished}, finds that monetized versions of these treatment effects translate into a benefit/cost ratio of $7.3$. This estimate accounts for the costs of implementing the program, including the welfare loss generated by taxing society in order to fund the program.\footnote{One advantage of the benefit/cost analysis is that it intrinsically accounts for extreme values. For example, individual crimes are weighed by their social cost instead of the average crime being weighed.}

Consistent with the results in Table~\ref{table:summary}, the results in Table~\ref{table:massiveall} show that females generally benefit more from treatment compared to males. In 7 out of the 10 categories that we consider, the within-category average effect size is larger for females. In 9 out of 10 categories, the proportion of positive effects is at least as large for males compared to that of females. Gender comparisons based on the \citet{Rosenbaum_2005_Distribution_JRSS} $p$-values are somewhat less clear-cut but are in the same direction. We next drill down on gender differences.


