

In this paper, we focus on testing treatment effects within categories of outcomes or across all available outcomes. This is a powerful, comprehensive tool that allows us to summarize the impact of ABC/CARE and avoid problems such a cherry-picking statistically significant results. It has the downside of obscuring the interpretation of the magnitudes of the treatment effects in each of the outcomes of interest. To be as inclusive as possible, we categorize our outcomes and obtain a principal component per category. This principal component is a ``latent'' variable describing each category for which we can calculate treatment effects.

Table~\ref{table:treatfactors} displays the results. Column (1) is the parameter in Equation~\eqref{eq:meandiff}, which is identified by random assignment to treatment. Column (2) displays the same parameter controlling for baseline variable and accounting for attrition. The procedures to select the control variables and to account for attrition are in Appendices~\ref{} and~\ref{}. Columns (3) to (6) display estimates for the parameters in Equations~\eqref{eq:cont1} and~\eqref{eq:cont2}. These are relevant when explaining gender differences so we delay discussing them to Section~\ref{sec:gender-differences}.

\begin{sidewaystable}[!htpb]
\begin{threeparttable}
\caption{Treatment Effects on Latent Outcomes \textbf{[JLG: This table is wrong.]}} \label{table:treatfactors}
\centering 
\begin{tabularx}{22.5cm}{XcX}
&   \begin{tabular}{cccccccccc}
  \toprule
   Category & Age & (1) & (2) & (3) & (4) & (5) & (6) \\

    \midrule
     \multicolumn{8}{c}{\textbf{\emph{Females}}} \\
    %cat 3
   \mc{1}{l}{\small{Parental Income Latent}} & \mc{1}{c}{\small{1.5 to 21}} & \mc{1}{c}{\small{0.286}} & \mc{1}{c}{\small{0.195}} & \mc{1}{c}{\small{0.578}} & \mc{1}{c}{\small{0.635}} & \mc{1}{c}{\small{0.226}} & \mc{1}{c}{\small{0.297}} \\  

     &  & \mc{1}{c}{\small{\textbf{(0.021)}}} & \mc{1}{c}{\small{\textbf{(0.037)}}} & \mc{1}{c}{\small{\textbf{(0.080)}}} & \mc{1}{c}{\small{(0.143)}} & \mc{1}{c}{\small{\textbf{(0.030)}}} & \mc{1}{c}{\small{(0.197)}} \\  

    \mc{1}{l}{\small{Education Latent}} & \mc{1}{c}{\small{21 to 30}} & \mc{1}{c}{\small{0.561}} & \mc{1}{c}{\small{0.499}} & \mc{1}{c}{\small{0.632}} & \mc{1}{c}{\small{0.726}} & \mc{1}{c}{\small{0.431}} & \mc{1}{c}{\small{0.312}} \\  

     &  & \mc{1}{c}{\small{\textbf{(0.004)}}} & \mc{1}{c}{\small{\textbf{(0.005)}}} & \mc{1}{c}{\small{\textbf{(0.010)}}} & \mc{1}{c}{\small{\textbf{(0.028)}}} & \mc{1}{c}{\small{\textbf{(0.012)}}} & \mc{1}{c}{\small{(0.195)}} \\  

    \mc{1}{l}{\small{Employment Latent}} & \mc{1}{c}{\small{21 to 30}} & \mc{1}{c}{\small{0.434}} & \mc{1}{c}{\small{0.064}} & \mc{1}{c}{\small{0.793}} & \mc{1}{c}{\small{0.998}} & \mc{1}{c}{\small{-0.064}} & \mc{1}{c}{\small{0.244}} \\  

     &  & \mc{1}{c}{\small{\textbf{(0.010)}}} & \mc{1}{c}{\small{\textbf{(0.041)}}} & \mc{1}{c}{\small{\textbf{(0.013)}}} & \mc{1}{c}{\small{\textbf{(0.031)}}} & \mc{1}{c}{\small{\textbf{(0.064)}}} & \mc{1}{c}{\small{(0.233)}} \\  

    \mc{1}{l}{\small{Crime Latent}} & \mc{1}{c}{\small{30 to Mid-30s}} & \mc{1}{c}{\small{-0.239}} & \mc{1}{c}{\small{-0.304}} & \mc{1}{c}{\small{-0.764}} & \mc{1}{c}{\small{-0.725}} & \mc{1}{c}{\small{-0.108}} & \mc{1}{c}{\small{-0.070}} \\  

     &  & \mc{1}{c}{\small{\textbf{(0.009)}}} & \mc{1}{c}{\small{\textbf{(0.008)}}} & \mc{1}{c}{\small{\textbf{(0.010)}}} & \mc{1}{c}{\small{(0.114)}} & \mc{1}{c}{\small{\textbf{(0.020)}}} & \mc{1}{c}{\small{(0.242)}} \\  

    \mc{1}{l}{\small{Hypertension Latent}} & \mc{1}{c}{\small{Mid-30s}} & \mc{1}{c}{\small{-0.061}} & \mc{1}{c}{\small{0.001}} & \mc{1}{c}{\small{0.121}} & \mc{1}{c}{\small{0.178}} & \mc{1}{c}{\small{-0.017}} & \mc{1}{c}{\small{-0.177}} \\  

     &  & \mc{1}{c}{\small{\textbf{(0.042)}}} & \mc{1}{c}{\small{\textbf{(0.051)}}} & \mc{1}{c}{\small{\textbf{(0.063)}}} & \mc{1}{c}{\small{(0.655)}} & \mc{1}{c}{\small{\textbf{(0.049)}}} & \mc{1}{c}{\small{(0.309)}} \\  
     
     
     
     
     
     
     
     
     
     
     
     
     
\midrule     
    \multicolumn{8}{c}{\textbf{\emph{Males}}} \\ 
    %cat 3 

 \mc{1}{l}{\small{Parental Income Latent}} & \mc{1}{c}{\small{1.5 to 21}} & \mc{1}{c}{\small{-0.078}} & \mc{1}{c}{\small{-0.222}} & \mc{1}{c}{\small{1.127}} & \mc{1}{c}{\small{0.363}} & \mc{1}{c}{\small{-0.271}} & \mc{1}{c}{\small{-0.122}} \\  

     &  & \mc{1}{c}{\small{\textbf{(0.057)}}} & \mc{1}{c}{\small{\textbf{(0.077)}}} & \mc{1}{c}{\small{\textbf{(0.094)}}} & \mc{1}{c}{\small{(0.307)}} & \mc{1}{c}{\small{\textbf{(0.079)}}} & \mc{1}{c}{\small{(0.607)}} \\  

    \mc{1}{l}{\small{Education Latent}} & \mc{1}{c}{\small{21 to 30}} & \mc{1}{c}{\small{0.344}} & \mc{1}{c}{\small{0.564}} & \mc{1}{c}{\small{1.020}} & \mc{1}{c}{\small{0.222}} & \mc{1}{c}{\small{0.485}} & \mc{1}{c}{\small{0.375}} \\  

     &  & \mc{1}{c}{\small{\textbf{(0.013)}}} & \mc{1}{c}{\small{\textbf{(0.004)}}} & \mc{1}{c}{\small{\textbf{(0.000)}}} & \mc{1}{c}{\small{(0.299)}} & \mc{1}{c}{\small{\textbf{(0.004)}}} & \mc{1}{c}{\small{\textbf{(0.083)}}} \\  

    \mc{1}{l}{\small{Employment Latent}} & \mc{1}{c}{\small{21 to 30}} & \mc{1}{c}{\small{0.501}} & \mc{1}{c}{\small{0.331}} & \mc{1}{c}{\small{-0.470}} & \mc{1}{c}{\small{0.098}} & \mc{1}{c}{\small{0.517}} & \mc{1}{c}{\small{0.693}} \\  

     &  & \mc{1}{c}{\small{\textbf{(0.007)}}} & \mc{1}{c}{\small{\textbf{(0.019)}}} & \mc{1}{c}{\small{\textbf{(0.001)}}} & \mc{1}{c}{\small{(0.414)}} & \mc{1}{c}{\small{\textbf{(0.012)}}} & \mc{1}{c}{\small{\textbf{(0.065)}}} \\  

    \mc{1}{l}{\small{Crime Latent}} & \mc{1}{c}{\small{30 to Mid-30s}} & \mc{1}{c}{\small{0.192}} & \mc{1}{c}{\small{0.333}} & \mc{1}{c}{\small{0.729}} & \mc{1}{c}{\small{0.648}} & \mc{1}{c}{\small{0.318}} & \mc{1}{c}{\small{0.224}} \\  

     &  & \mc{1}{c}{\small{\textbf{(0.066)}}} & \mc{1}{c}{\small{\textbf{(0.072)}}} & \mc{1}{c}{\small{\textbf{(0.084)}}} & \mc{1}{c}{\small{(0.945)}} & \mc{1}{c}{\small{\textbf{(0.073)}}} & \mc{1}{c}{\small{(0.682)}} \\  

    \mc{1}{l}{\small{Hypertension Latent}} & \mc{1}{c}{\small{Mid-30s}} & \mc{1}{c}{\small{-0.643}} & \mc{1}{c}{\small{-0.920}} & \mc{1}{c}{\small{0.146}} & \mc{1}{c}{\small{-0.025}} & \mc{1}{c}{\small{-1.315}} & \mc{1}{c}{\small{-1.140}} \\  

     &  & \mc{1}{c}{\small{\textbf{(0.004)}}} & \mc{1}{c}{\small{\textbf{(0.002)}}} & \mc{1}{c}{\small{\textbf{(0.097)}}} & \mc{1}{c}{\small{(0.446)}} & \mc{1}{c}{\small{\textbf{(0.000)}}} & \mc{1}{c}{\small{\textbf{(0.002)}}} \\  
     \bottomrule
    \end{tabular}  & 
\end{tabularx}
\begin{tablenotes}
\footnotesize
\item \textbf{Note:} This table shows the treatment effects for ``latent'' outcomes, constructed from various related outcomes to the latent. Each column present estimates for the following parameters: \textbf{(1)} $\mathbb{E} \big[ \bm{Y}^1 - \bm{Y}^0 | W = 1]$; {\textbf{(2)} $\mathbb{E} \big[ \bm{Y}^1 - \bm{Y}^0 | \bm{B} \big]$}; {\textbf{(3)} $\mathbb{E} \big[ \bm{Y}^1 | \bm{B}, D=1 \big] - \mathbb{E} \big[ \bm{Y}^0 | \bm{B}, V=0, D=0 \big]$}; {\textbf{(4)} $\mathbb{E} \big[ \bm{Y}^1 - \bm{Y}^0 | \bm{B}, V=0 \big] $}; {\textbf{(5)} $\mathbb{E} \big[ \bm{Y}^1 | \bm{B}, D=1 \big] - \mathbb{E} \big[ \bm{Y}^0 | \bm{B}, V=1, D = 0 \big]$}; {\textbf{(6)} $\mathbb{E} \big[ \bm{Y}^1 - \bm{Y}^0 | \bm{B}, V=1 \big]$}. We account for the following background variables ($\bm{B}$): Apgar scores at minutes 1 and 5 and the high-risk index. We define the high-risk index in Appendix~\ref{appendix:background} and explain how we choose the control variables in Appendix~\ref{appendix:bvariables}. Inference is based on non-parametric, one-sided $p$-values from the empirical bootstrap distribution. We highlight point estimates significant at the $10\%$ level.
\end{tablenotes}
\end{threeparttable}
\end{sidewaystable}

The results in  Columns (1) and (2) of Table~\ref{table:treatfactors} indicate that accounting for baseline controls and attrition has little effect on the estimates of the treatment effects. More importantly, they reflect that ABC/CARE has substantial market and non-market benefits across the life-cycle. Recall that this inference is valid for all individuals in the population for whom $\bm{B} \in \mathcal{B}_0$ (i.e.,\ are at considerable socio-economic disadvantage). The latents for each category have an in-sample mean of $0$ and standard deviation of $1$. The latent capturing several measures of parental income for female increases $1/4$ of a standard deviation as a consequence of treatment. Thus ABC/CARE benefits not only treated girls but also their parents (i.e.,\ there are intergenerational benefits due to the childcare component of the program that propagate over the parents' life cycle). The benefits in education, employment, crime reduction, and hypertension reduction at least as sizable. For boys, there are benefits across all of the categories although smaller in magnitude. We expand on these gender differences in Section~\ref{sec:gender-differences}, after presenting results and tests on the aggregated treatment effects.

\textbf{Estimates of Combining Functions and Non-Parametric Exact Tests.} 

\begin{table}[!htpb]
\begin{threeparttable}
\caption{Combining Functions and Non-Parametric, Exact Tests} \label{table:massiveall}
\centering
\begin{tabularx}{16.75cm}{XcX}
& 
\begin{tabular}{lcc}
\toprule
& \mc{2}{c}{Treat. vs. Control}   \\
\cmidrule(lr){2-3}
 & Females  & Males \\
 \midrule
IQ &     0.719 &     0.664 \\  
\quad \% Positive &   100.000 &   100.000 \\  
\quad \% Significant &    85.714 &    85.714 \\  
\quad $p$-value &     \textbf{0.046} &     \textbf{0.045} \\  
 \midrule
Achievement &     0.672 &     0.235 \\  
\quad \% Positive &   100.000 &   100.000 \\  
\quad \% Significant &    80.000 &     0.000 \\  
\quad $p$-value &     \textbf{0.046} &     \textbf{0.086} \\  
 \midrule
Social-emotional &     0.385 &     0.059 \\  
\quad \% Positive &    92.857 &    50.000 \\  
\quad \% Significant &    57.143 &     7.143 \\  
\quad $p$-value &     0.235 &     0.147 \\  
 \midrule
Mother Labor &     0.228 &     0.445 \\  
\quad \% Positive &   100.000 &   100.000 \\  
\quad \% Significant &    20.000 &    20.000 \\  
\quad $p$-value &     \textbf{0.086} &     \textbf{0.001} \\  
 \midrule
Parenting  &     0.274 &     0.060 \\  
\quad \% Positive &   100.000 &    80.000 \\  
\quad \% Significant &    20.000 &     0.000 \\  
\quad $p$-value &     0.602 &     0.147 \\  
 \midrule
Education &     0.356 &     0.174 \\  
\quad \% Positive &    83.333 &    83.333 \\  
\quad \% Significant &    33.333 &     0.000 \\  
\quad $p$-value &     \textbf{0.000} &     0.235 \\  
 \midrule
Employment &     0.200 &     0.438 \\  
\quad \% Positive &   100.000 &   100.000 \\  
\quad \% Significant &     0.000 &     0.000 \\  
\quad $p$-value &     0.151 &     \textbf{0.022} \\  
 \midrule
Crime &     0.242 &    -0.093 \\  
\quad \% Positive &   100.000 &    33.333 \\  
\quad \% Significant &     0.000 &     0.000 \\  
\quad $p$-value &     0.715 &     0.812 \\  
 \midrule
Risk &     0.099 &     0.011 \\  
\quad \% Positive &   100.000 &    25.000 \\  
\quad \% Significant &     0.000 &    25.000 \\  
\quad $p$-value &     0.469 &     \textbf{0.086} \\  
 \midrule
Health &     0.060 &     0.061 \\  
\quad \% Positive &    68.750 &    73.333 \\  
\quad \% Significant &     6.250 &    13.333 \\  
\quad $p$-value &     \textbf{0.046} &     \textbf{0.000} \\  
\bottomrule
\end{tabular}

 & 
\end{tabularx}
\begin{tablenotes}
\footnotesize
\item \textbf{Note:} This table displays summaries of treatment effects by outcome category and gender. Each of the panels contains statistics calculated using outcomes grouped by category. The average effect size (Avg.\ Effect) is calculated by averaging over the effect size of the outcomes in the outcome category. The effect sizes of the individual outcomes are calculated by dividing the coefficient by the standard deviation of the control group. The $p$-values are asymptotic following \citet{Rosenbaum_2005_Distribution_JRSS}. The null hypothesis is that the control and treatment distributions, within gender, are equal. A $p$-value less than 0.1 (bolded) indicates that the distributions are significantly different. 
\end{tablenotes}
\end{threeparttable}
\end{table}




We consider a total of 126 outcomes reported in Appendix~\ref{appendix:gdiff-tes}. 

