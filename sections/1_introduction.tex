This paper studies two influential early childhood programs offered as an educational form of childcare. The programs are the Carolina Abecedarian Project (ABC) and its almost identical sister program, the Carolina Approach to Responsive Education (CARE), both evaluated by the method of random assignment---henceforth ABC/CARE. While certain categories of outcomes of ABC/CARE have been thoroughly assessed, our paper is the first to aggregate and summarize all of the available outcomes to evaluate the program, including, for example, adulthood administrative criminal records.

ABC/CARE was conducted in Chapel Hill, North Carolina for a sample of children born between 1972 and 1980. It was one of the pioneer programs that focused intensively on child-led learning and is a template for many current and proposed early childhood programs.\footnote{Programs inspired by ABC/CARE have been (and are currently being) launched around the world. \citet{Sparling_2010_Highlights} and \citet{Ramey_Ramey_Lanzi_2014_Interventions} list numerous programs based on the ABC/CARE approach. The programs are: IHDP---eight different cities around the U.S. \citep{Spiker-etal_1997_Helping}; Early Head Start and Head Start in the U.S. \citep{Schneider_McDonald-eds_2007_Scale-Up_Vol-1}; John's Hopkins Cerebral Palsy Study in the U.S. \citep{Sparling_2010_Highlights}; Classroom Literacy Interventions and Outcomes (CLIO) study in the U.S. \citep{Sparling_2010_Highlights}; Massachusetts Family Child Care Study \citep{Collins_etal_2010_Massachusetts-Study}; Healthy Child Manitoba Evaluation \citep{Healthy_Child_Manitoba_2015_Starting-Early}; Abecedarian Approach within an Innovative Implementation Framework \citep{Jensen_Nielsen_2016_ABC-Programme-Pilot}; and Building a Bridge into Preschool in Remote Northern Territory Communities in Australia \citep{UMonash_Dataset_2015_URL}. Educare programs are also based on ABC/CARE \citep{Educare_2014_Research_Agenda,Yazejian_Bryant_2012_Educare}.} It started at 8 weeks of age and continued through age 5. Treatment and control children were followed through their mid 30s, with data collected on multiple dimensions of human development. As a result of this intensive data collection, there are over 100 outcomes that we could use to evaluate the program.

Given the difficulty of individually analyzing all of the available outcomes to evaluate the program and the need to avoid cherry-picking outcomes with significant treatment effects, we propose parametric and non-parametric tests to aggregate and summarize treatment effects. Table~\ref{table:summary} previews our approach. Panel (a) displays our summaries for early-childhood outcomes (birth to age 5). We offer four treatment-control comparisons across these outcomes by gender: (i) the average effect size; (i) the percentage of outcomes for which there is a positive treatment effect; (iii) the percentage of outcomes for which there is a positive and significant ($\alpha=10\%$) effect;\footnote{We perform inference at the 10\% level throughout this paper due to the sample size. However, we also display the $p$-values so that the reader can qualify the conclusions at different significance levels.} and (iv) the $p$-value resulting from a non-parametric comparison of the joint distribution of outcomes \citep{Rosenbaum_2005_Distribution_JRSS}. Panels (b) and (c) are analogous for school age (ages 6 to 18) and adulthood (21 to 35), respectively, and Panel (d) includes all of the outcomes in Panels (a) through (c).


\begin{table}[!htpb]
\begin{threeparttable}
\caption{Summary of Treatment-Control Comparisons by Gender \textbf{[AZ: With more bootstraps, the significance  of the first three statistics and the proportion of significant effects will change.]}} \label{table:summary}
\centering 
\begin{tabularx}{16.5cm}{XcX}
& % matrix: COMBINE file: raw-rosenbaum-table.tex  10 May 2018 09:03:15

\begin{tabular}{lcc} 
\toprule
& \mc{2}{c}{Treat. vs. Control}  \\
\cmidrule(lr){2-3} 
 & Females  & Males \\
 \midrule
Early Childhood &     0.392 &     0.259 \\  
\quad \% Positive &    95.833 &    75.000 \\  
\quad \% Significant &    45.833 &    29.167 \\  
\quad $p$-value &     \textbf{0.086} &     \textbf{0.086} \\  
 \midrule
School Age &     0.531 &     0.297 \\  
\quad \% Positive &   100.000 &   100.000 \\  
\quad \% Significant &    55.556 &    11.111 \\  
\quad $p$-value &     \textbf{0.022} &     0.147 \\  
 \midrule
Adult &     0.291 &     0.227 \\  
\quad \% Positive &    88.889 &    72.222 \\  
\quad \% Significant &    33.333 &    16.667 \\  
\quad $p$-value &     \textbf{0.010} &     \textbf{0.086} \\  
 \midrule
All &     0.381 &     0.254 \\  
\quad \% Positive &    94.118 &    78.431 \\  
\quad \% Significant &    43.137 &    21.569 \\  
\quad $p$-value &     \textbf{0.010} &     \textbf{0.086} \\  
\bottomrule
\end{tabular}
 & 
\end{tabularx}
\begin{tablenotes}
\footnotesize
\item \textbf{Note:} This table displays summaries of treatment effects by age and gender. Each of the panels contains statistics calculated using outcomes measured at the indicated ages. Early childhood includes outcomes measured before age 6, school age includes outcomes measured between age 6 and 18, and adult includes outcomes measured between 21 and 35. All (panel d) is a combination of all the outcomes in panels (a) to (c). The average effect size is calculated by averaging over the effect sizes of the outcomes in the age category. The effect sizes of the individual outcomes are calculated by dividing the coefficient by the standard deviation of the control group. We test these three statistics bootstrapped $p$-values. For the proportion of outcomes that are positive and significant, we do a ``double bootstrap'' procedure. The null hypothesis for the effect sizes is that they are 0. The null hypothesis for the proportion of outcomes that are (significantly) positive is that they are (10\%) 50\%. We explain these statistics and the inference in more detail below. Bolded statistics are significant at the 10\% level. The \citet{Rosenbaum_2005_Distribution_JRSS} $p$-value originates from a test where the null is a common joint distribution of the variables in each category. A $p$-value less than $0.10$ (bolded) indicates that the distributions are significantly different at the 10\% level. More details on our inference procedure are in Section~\ref{sec:parameters}.
\end{tablenotes}
\end{threeparttable}
\end{table}

The first finding in Table~\ref{table:summary} is that randomized assignment to ABC/CARE benefited both females and males across the life cycle. Across the life cycle, the average effect size for females (males) is $0.381$ ($0.254)$, the percentage of positive treatment effects is $94\%$ ($78\%$), and the percentage of significant treatment effects at the $10\%$ level is $63\%$ ($41\%$). We provide the $p$-values from a non-parametric test contrasting the joint distribution of the outcomes. The conclusion from these comparisons aligns with the conclusions from the other measures, even in magnitude if the size of the $p$-value is taken as a measure of magnitude \citep{Fisher_1935_Inference_JRSS}. The $p$-values and proportions do not directly explain the magnitudes of the effect, although we describe and discuss the magnitudes in more detail below.\footnote{For example, assignment to treatment increased college graduation for males by 17 percentage points, with a standard error of 5 percentage points (the college graduation rate for the control group is 12\%). For females, assignment to treatment increased high school graduation by 25 percentage points, with a standard error of 1 percentage point (the high school graduation rate for the control group is 53\% points). These results are the basis of the forecasts in \citet{Garcia_Heckman_Leaf_etal_2017_Comp_CBA_Unpublished}: The authors document that education is actually the main intermediate input when forecasting benefits across the life cycle.}

The second finding in Table~\ref{table:summary} is that females benefit more than males. This holds across the life cycle and across all of the measures that we employ to summarize the treatment effects. To explain the gendered treatment effects, we document that control-group girls grew up in less favorable environments if compared to control-group boys using baseline measures (e.g.,\ less father presence, lower maternal education and IQ). Girls in the control group who stayed at home ($27\%$ of the control-group girls), were taken care of in a disadvantaged environment. Girls who went to preschools other than ABC/CARE ($73\%$ of the control-group girls), likely went to lower-quality preschools because their families were more resource constrained if compared to their male counterparts. Thus, girls benefited more from treatment because they would have otherwise grown up in more disadvantaged environments if compared to boys.\footnote{Documentation in \citet{Burchinal_etal_1989_CD_Daycare-Pre-K-Dev} shows that the quality of the available alternatives was of lower quality than the treatment offered through ABC/CARE. We supplement this documentation with historical records showing that even the alternatives that followed state and federal standards of the era were of lower quality than ABC/CARE as measured by concrete measures such as staff-child ratios. We provide more detail on the quality of ABC/CARE and the alternative options in Section~\ref{sec:data} and Appendix~\ref{appendix:background}.}

Although there was no significant difference in the take-up of preschool alternatives between parents of control-group girls and parents of control-group boys, there is a subtle, important difference in the within-gender take-up of alternatives. Within girls, take-up of alternatives does not differ significantly by socio-economic disadvantage, although we show later that more disadvantaged control-group parents of girls select home care. Within boys, the relatively advantaged boys stayed at home more often and did not attend the lower-quality preschool alternatives available when ABC/CARE was implemented. These results are verified when we estimate parameters that explicitly account for control substitution. That is, when we compare treatment to alternative preschools and treatment to staying at home. These parameters are not identified by randomization into treatment and require stronger assumptions that we discuss below. Boys benefit the most from ABC/CARE when compared to alternative preschools, given their relatively better home environments. Girls benefit the most from ABC/CARE when compared to home-based care, given that those who stay at home are more disadvantaged. The difference in this benefit is less than the difference for boys, which is consistent with the stronger presence of selection for boys.

ABC/CARE greatly benefits children by providing an enriching environment to complement their home environments. This sounds a cautionary note for advocates of early childhood programs in any form: Quality matters and low-quality programs could cause harm. Dissecting the counterfactual brings our results to a modern context in which many children are enrolled in preschool, and in which low-quality center-based care is abundant, especially for children from socio-economic disadvantaged backgrounds.\footnote{Approximately 60\% of preschool-age children regularly receive non-parental childcare \citep{FIFCFS_2009_Wellbeing_REPORT}. At age 4, the child care break down for disadvantaged children, which compose 13\% of all children in the US \citep{USCB_2014_CoverageReport}, was the following. Head Start: 29\%; Non-Head Start center-based: 26\%; relative care: 15\%; non-relative care: 5\%; parental care: 25\%. While the quality of Head Start centers and family day care arrangements is constant across disadvantaged and advantaged children, the quality of the rest of arrangements is considerably and consistently lower for disadvantaged children \citep{FIFCFS_2009_Wellbeing_REPORT}.}

The rest of our analysis unfolds in the following way. Section~\ref{sec:data} describes the experimental data we analyze and its special features. It documents the quality level of the alternative childcare that a considerable proportion of the control-group subjects attended. Section~\ref{sec:parameters} defines the treatment effects we estimate and how we summarize them. Section~\ref{sec:treatment-effects} reports the treatment effects overall and by gender and establishes the existence of sharp gender differences for many categories of outcomes. Section~\ref{sec:gender-differences} discusses the sources of these differences. Section~\ref{sec:conclusion} concludes.
