This paper uses data from a randomized controlled trial of two influential and intensive early childhood programs that enriched the early lives of disadvantaged children. The programs are the Carolina Abecedarian Project (ABC) and its almost identical sister program, the Carolina Approach to Responsive Education (CARE). These programs were conducted in Chapel Hill, North Carolina for a sample of children born between 1972 and 1980. In this paper, we refer to the combined programs by the acronym ABC/CARE. It was one of the pioneer programs that focused intensively on child-led learning and is a template for many current and proposed early interventions.\footnote{Programs inspired by ABC/CARE have been (and are currently being) launched around the world. \citet{Sparling_2010_Highlights} and \citet{Ramey_Ramey_Lanzi_2014_Interventions} list numerous programs based on the ABC/CARE approach. The programs are: IHDP---eight different cities around the U.S. \citep{Spiker-etal_1997_Helping}; Early Head Start and Head Start in the U.S. \citep{Schneider_McDonald-eds_2007_Scale-Up_Vol-1}; John's Hopkins Cerebral Palsy Study in the U.S. \citep{Sparling_2010_Highlights}; Classroom Literacy Interventions and Outcomes (CLIO) study in the U.S. \citep{Sparling_2010_Highlights}; Massachusetts Family Child Care Study \citep{Collins_etal_2010_Massachusetts-Study}; Healthy Child Manitoba Evaluation \citep{Healthy_Child_Manitoba_2015_Starting-Early}; Abecedarian Approach within an Innovative Implementation Framework \citep{Jensen_Nielsen_2016_ABC-Programme-Pilot}; and Building a Bridge into Preschool in Remote Northern Territory Communities in Australia \citep{UMonash_Dataset_2015_URL}. Educare programs are also based on ABC/CARE \citep{Educare_2014_Research_Agenda,Yazejian_Bryant_2012_Educare}.} It started at 8 weeks of age and continued through age 5. Treatment and control subjects were followed through their mid 30s, with data collected on multiple dimensions of human development \citep{Ramey_Campbell_1991_childreninpoverty}.

The program benefits both males and females in several outcomes that we have measures for across the life cycle.\footnote{Previous studies presenting treatment effects of ABC and CARE include \citet{Ramey_etal_1985_Project-CARE_TiECSE,Clarke_Campbell_1998_ABC_Comparison_ECRQ,Campbell_Pungello_etal_2001_DP,Campbell_Ramey_etal_2002_ADS,Campbell_Wasik_etal_2008_ECRQ,Campbell_Conti_etal_2014_EarlyChildhoodInvestments}. Only \citet{Campbell_Conti_etal_2014_EarlyChildhoodInvestments} divide the treatment effects by gender, however, to focus on the health effects, with men having many more positive effects especially in cardiovascular and metabolic conditions. This is consistent with the results we report below.} However, there are substantial differences in how males and females benefit from the program across domains. The program promotes the employment and health of males, while it enhances the cognition and educational attainment of girls. While all the treatment-group subjects attended the center-based care, the experiences of the control-group subjects varied with 75\% enrolling in alternative center-based options. Historical documentation, records, and evidence from knowledgeable individuals indicate that although the available alternative centers mostly followed state and federal standards, they were of considerably lower quality than the ABC/CARE program.\footnote{We document this in Section~\ref{sec:data} and Appendix~\ref{app:control-subbb}.} We analyze the marginal benefit of ABC/CARE relative to this lower-quality center-based care accouting for selection into the alternatives using matching on observables. Boys in ABC/CARE benefit relatively more than girls from participation in high-quality center-based care compared to lower-quality center-based care.

Our analysis responds to recent claims about the harm caused by enrolling children in center-based infant care and preschool. \citet{Baker_Gruber_etal_2008_JPE} show that participants in childcare manifest adverse behavioral outcomes. \citet{Kottelenberg_Lehrer_2014_Gender-Effects} show that this result is localized to boy. The program analyzed by \citet{Baker_Gruber_etal_2008_JPE} and \citet{Kottelenberg_Lehrer_2014_Gender-Effects} is of relatively low quality compared to ABC/CARE.\footnote{The program they analyze was a modest payment (between 5 and 7 Canadian dollars a day starting in 1997) in the form of a voucher. The voucher component itself implies a reduction in the quality of the received program given an information problem by parents. The full cost of the program was \$44 (2014) a day \citep{Baker_etal_2005_Universal_Childcare_NBER}. This is less than two thirds of the cost of ABC/CARE, which was close to \$75 (2014) a day \citep{Garcia_Heckman_Leaf_etal_2017_Comp_CBA_Unpublished}.}

Randomization allows us to identify the parameter for the effect of the program relative to the control group, which consists of children who stayed at home or went to an alternative childcare arrangement. The alternative childcare arrangements were considerably worse in quality if compared to ABC/CARE \citep{}. High-quality childcare greatly benefits boys relative to lower-quality childcare. The benefits are smaller when comparing high-quality childcare to staying at home. For girls, both comparisons obtain similar results. Randomization does not identify provide identification of the parameters allowing us to draw these conclusions. We discuss the econometric methods allowing us to do so.

There is no contradiction between the claim that lower-quality programs impair child development, while high-quality programs do not. This analysis sounds a cautionary note for advocates of early childcare in any form: quality matters and low-quality programs can cause harm. Dissecting the counterfactual further helps bring the results to a modern context in which many more children are enrolled in preschool programs: in 2016, an estimated 42\% of 3-year-olds, 66\% of 4-year-olds, and 86\% of 5-year-olds were enrolled in preschool programs \citep{NCES_2017_Education_IES}. \textbf{[JLG: we need a sharper statistic so that our quality statement matters.]}

\textbf{[JLG: introduce new table of summary of results.]}

\textbf{[JLG: extensively place CBA.]}


The rest of our analysis unfolds in the following way. Section~\ref{sec:data} describes the experimental data we analyze and its special features. It documents the quality level of the alternative childcare that a considerable proportion of the control-group subjects attended. Section~\ref{sec:parameters} defines the treatment effects we estimate and how we summarize them. Section~\ref{sec:treatment-effects} reports the treatment effects overall and by gender and establishes the existence of sharp gender differences for many categories of outcomes. Section~\ref{sec:gender-differences} discusses the sources of these differences. Section~\ref{sec:conclusion} concludes.

%ee \citet{Beeghly-etal_2017_IMHJ,Dayton_2017_IMHJ,Iruka_2017_IMHJ,Schore_2017_IMHJ} for recent findings on the topic of different development of males and females early in life. 