Differences in gender are central features of economic and social life. This paper investigates how participation in enriched early childhood programs differentially affects the lives of disadvantaged boys and girls, and whether it promotes or reduces gender gaps. In addition, we study the effect of these gender gaps considering different levels of quality. We show that, on many outcomes, boys are more susceptible to lower-quality alternatives to enriched early childhood education, which contributes to the observed reversals in boy-girl gender gaps. 

Many studies across disciplines point to the greater vulnerability of boys to adverse life conditions from a combination of psychological, biological, and social factors.\footnote{See \citet{Golding_Fitzgerald_2017_IMHJ} for a review of recent work.} As a group, girls mature earlier, are more resilient to adversity, and perform better in a variety of life tasks.\footnote{See \citet{Schore_2017_IMHJ} for an informative overview of this literature. Economists have also contributed to this literature. See, e.g., \citet{Bertrand_Pan_2013_AEJAE}, \citet{Kottelenberg_Lehrer_2014_Gender-Effects}, and \citet{Autor-etal_2015_Family-Disadvantage}.} Less is known about effective strategies for reducing or compensating for the vulnerability of boys to disadvantage.

Many studies have shown the benefits of early-life interventions for improving the skills of children, especially those from disadvantaged families \citep{Currie_2011_AER,Elango_Hojman_etal_2016_Early-Edu}. Although several of these studies report effects by gender, most do not. Some examples include \citet{Bernal_Keane_2011_JoLE,Cascio_Schanzenbach_2013_ImpactsExpandingAccess,Bitler_et_al_2014_Head_Start_Unpublished,Kline_Walters_2016_QJE}. There are some exceptions \citep{Heckman_Moon_etal_2010_QE,Campbell_Conti_etal_2014_EarlyChildhoodInvestments,Garcia_Heckman_Leaf_etal_2017_Comp_CBA_Unpublished}. Pooling males and females ignores potentially large differences in treatment effects.\footnote{We survey the literature more extensively in Appendix~\ref{appendix:gdiff-survey}.}

This paper investigates this issue using data from a randomized controlled trial of a prototypical intensive early childhood program that enriched the early lives of disadvantaged children. The data come from the Carolina Abecedarian Project (ABC) and its almost identical sister program, the Carolina Approach to Responsive Education (CARE). These programs were conducted in Chapel Hill, North Carolina for a sample of children born between 1972 and 1980. In this paper, we refer to the combined programs by the acronym ABC/CARE. It was one of the pioneer programs that focused intensively on child-led learning and is a template for many current and proposed early interventions.\footnote{Programs inspired by ABC/CARE have been (and are currently being) launched around the world. \citet{Sparling_2010_Highlights} and \citet{Ramey_Ramey_Lanzi_2014_Interventions} list numerous programs based on the ABC/CARE approach. The programs are: IHDP---eight different cities around the U.S. \citep{Spiker-etal_1997_Helping}; Early Head Start and Head Start in the U.S. \citep{Schneider_McDonald-eds_2007_Scale-Up_Vol-1}; John's Hopkins Cerebral Palsy Study in the U.S. \citep{Sparling_2010_Highlights}; Classroom Literacy Interventions and Outcomes (CLIO) study in the U.S. \citep{Sparling_2010_Highlights}; Massachusetts Family Child Care Study \citep{Collins_etal_2010_Massachusetts-Study}; Healthy Child Manitoba Evaluation \citep{Healthy_Child_Manitoba_2015_Starting-Early}; Abecedarian Approach within an Innovative Implementation Framework \citep{Jensen_Nielsen_2016_ABC-Programme-Pilot}; and Building a Bridge into Preschool in Remote Northern Territory Communities in Australia \citep{UMonash_Dataset_2015_URL}. Educare programs are also based on ABC/CARE \citep{Educare_2014_Research_Agenda,Yazejian_Bryant_2012_Educare}.} It started at 8 weeks of age and continued through age 5. Treatment and control subjects were followed through their mid 30s, with data collected on multiple dimensions of human development \citep{Ramey_Campbell_1991_childreninpoverty}. 

There are positive impacts of the program across the life cycle for both genders.\footnote{Previous studies presenting treatment effects of ABC and CARE include \citet{Ramey_etal_1985_Project-CARE_TiECSE,Clarke_Campbell_1998_ABC_Comparison_ECRQ,Campbell_Pungello_etal_2001_DP,Campbell_Ramey_etal_2002_ADS,Campbell_Wasik_etal_2008_ECRQ,Campbell_Conti_etal_2014_EarlyChildhoodInvestments}.} However, there are substantial differences in impacts by gender across domains. The program promotes the labor income, employment, and health of males and reduces their participation in crime, while it enhances the cognition, achievement, and educational attainment of girls. While all the treatment-group subjects attended the center-based care, the experiences of the control-group subjects varied with 75\% enrolling in alternative center-based options. Historical documentation, records, and evidence from knowledgeable individuals indicate that although the available alternative centers followed state and federal standards, they were of considerably lower quality than the ABC/CARE program.\footnote{We document this in Appendix~\ref{app:control-subbb}.} We analyze the marginal benefit of ABC/CARE relative to this lower-quality center-based care. Boys in ABC/CARE benefit relatively more than girls from participation in high-quality center-based care compared to lower-quality center-based care.

Our analysis responds to recent claims about the harm caused by enrolling children in center-based infant care and preschool. In an influential analysis, \citet{Baker_Gruber_etal_2008_JPE} show that participants in childcare manifest adverse behavioral outcomes. \citet{Kottelenberg_Lehrer_2014_Gender-Effects} find a similar result but localized to boys.  The program analyzed by \citet{Baker_Gruber_Milligan_2015_Noncog_Defects} and \citet{Kottelenberg_Lehrer_2014_Gender-Effects} is of relatively low quality compared to the enriched program we analyze.\footnote{The program they analyze was a modest payment (between 5 and 7 Canadian dollars a day starting in 1997) in the form of a voucher. The voucher component itself implies a reduction in the quality of the received program given an information problem by parents. The full cost of the program was \$44 (2014) a day \citep{Baker_etal_2005_Universal_Childcare_NBER}. This is less than two thirds of the cost of ABC/CARE, which was close to \$75 (2014) a day \citep{Garcia_Heckman_Leaf_etal_2017_Comp_CBA_Unpublished}.} However, not all childcare programs are alike. We produce evidence that high-quality childcare greatly benefits boys relative to lower-quality childcare. Staying at home is a better option for them, especially if the family environment is relatively advantaged. This effect is not found for girls. There is no contradiction between the claim that lower-quality programs impair child development, while high-quality programs do not. This analysis sounds a cautionary note for advocates of early childcare in any form: quality matters and low-quality programs can cause harm.

Unlike previous studies, we estimate treatment effects comparing the treatment group to different control groups: subjects who received home care or care in centers of lower quality then ABC/CARE. We find that home care is beneficial for boys compared to low-quality center childcare.

To preview our analysis, we present gender differences in outcomes in Table~\ref{tab:proportion-table-ranksign}. We report the proportion of outcomes, by category, for which males outperform females (we have multiple outcome measures in each category which we explain in greater detail below). These proportions are invariant to the scales used for individual measures. Under the null hypothesis of no difference in treatment effects by gender, the proportion of outcomes favoring any gender should be 50\%. We test this hypothesis for the control group and the treatment group separately to determine whether there is a baseline (control) gender difference and whether treatment affects this gender difference.

Pooling the two control groups, males have higher IQs, employment, enhanced parental labor income (through subsidized childcare), and are more likely to participate in criminal activity than females. They do better on an aggregate over all categories. Treatment reverses the gap between males and females in the pooled control group for achievement. All achievement measures favor males in the pooled control group, but favor females in the treatment group. Education is another outcome category for which treatment reverses the gender gap between treatment and a pooled control group. Males have higher educational attainment in the pooled control group with 66.7\% of the educational outcomes favoring males, although the result is not statistically significant. In the treatment group, however, only 11.1\% of the educational outcomes favor males. Treatment reverses the gap for an aggregate across all outcome categories and narrows the gap for employment.

\begin{table}[H]
\centering
\caption{Summary of Proportion of Outcomes Males $>$ Females}
\label{tab:proportion-table-ranksign}
\begin{threeparttable}
\begin{tabular}{l c c c c}
\toprule
Category & \# Outcomes & \mc{2}{c}{Proportion} & Difference \\
\cmidrule(lr){3-4} \cmidrule(lr){5-5}
            &                       & Control & Treatment & Treatment $- $ Control \\
\midrule
IQ & 15 & 0.733 & 0.533 & -0.200 \\
Achievement & 12 & \textbf{0.833} & \textbf{0.000} & \textbf{-0.833} \\
Social-Emotional & 22 & 0.455 & 0.364 & \textbf{-0.091} \\
Parenting & 7 & 0.571 & 0.286 & \textbf{-0.286} \\
Parental Labor Income & 15 & 0.600 & 0.733 & 0.133 \\
Education & 9 & 0.667 & \textbf{0.111} & \textbf{-0.556} \\
Employment & 4 & \textbf{1.000} & \textbf{0.750} & \textbf{-0.250} \\
Crime & 4 & \textbf{0.000} & \textbf{0.000} & \textbf{0.000} \\
Risky Behavior & 5 & 0.400 & \textbf{0.200} & \textbf{-0.200} \\
Health & 22 & 0.545 & 0.545 & 0.000 \\
Mental Health & 11 & \textbf{0.818} & 0.545 & \textbf{-0.273} \\
\midrule
All & 126 & 0.611 & 0.413 & \textbf{-0.198} \\
\bottomrule
\end{tabular}
% This file generated by: abccare-cba/scripts/abccare/genderdifferences/abccare-gdiff-gaps-ranksign.do

\begin{tablenotes}
\footnotesize
\item Note: This table summarizes comparison of gender gaps across outcome categories by different groups. A bold proportion for the treatment or control group indicates that the proportion is statistically different than 50\% at the 10\% level. A bold difference between treatment and control indicates that the Wilcoxon signed-rank test comparing the control and treatment proportions, over 1,000 bootstraps, yields a $p$-value less than or equal to 0.10. The variables for each outcome category are listed in Appendix~\ref{appendix:gdiff-tes}. The inference procedure is described in detail in Appendix~\ref{little-gdiff-proportions}. In summary, we bootstrap with replacement 1,000 times. For bootstrap $b \in [1, \ldots, 1,000]$ we compute the proportion of variables in each outcome category for which the males do better than the females. We do this separately for the treatment and control group, estimating the pair of proportions $(P_{T,b}, P_{C,b})$. Over the distribution of the pairs, we compute the signed-rank test. It compares the empirical distribution of the proportions in the treatment group to the empirical distribution of the proportions in the control group. We repeat this procedure for each outcome category.
\end{tablenotes}
\end{threeparttable}
\end{table}

Table~\ref{tab:proportion-table} summarizes the gender gaps reported in Table~\ref{tab:proportion-table-ranksign}, as well as those for different childcare environments for control-group children. In the control group, the proportion of outcomes for which males do better than females is higher than 50\% for most of the categories, although not all are statistically significant. Exceptions include social-emotional skills, risky behavior, and crime, in which females surpass males.\footnote{Females commit fewer crimes and report less risky behavior in comparison to males.} Treatment reverses the gaps for achievement, education, employment, parental income, and overall outcomes. For controls who stay at home, females have slightly better health outcomes than males. The males who attend lower-quality childcare do not outperform females on any of the outcome categories associated with cognition, education, and induced parental labor income (through subsidy of childcare). These outcomes are concentrated relatively early in the life of a child, indicating an early-life disadvantage that enriched early childcare programs partially correct.

\begin{table}[H]
\centering
\caption{Summary of Proportion of Outcomes Males $>$ Females by Home Status}
\label{tab:proportion-table}
\begin{threeparttable}
\begin{footnotesize}
\begin{tabular}{l | c |c |c| c}
\toprule
& (1) & (2) & (3) & (4) \\
Category & Control Group  &  Control Group &  Control Group &  Treatment \\
	&				&	Stay at Home		& Alternative Preschool &  Group \\
\midrule  
IQ 								& \checkmark &  \checkmark* & $\times$&\checkmark \\
Achievement						& \checkmark &  \checkmark* &$\times$ & $\times$* \\
Parenting							& \checkmark&  \checkmark* &$\times$ & \checkmark \\
Social-emotional					& $\times$&$=$ &$\times$* &$\times$* \\
Education							& \checkmark&\checkmark & $=$ &$\times$* \\
Employment						&  \checkmark* &  \checkmark* &  \checkmark* &$\times$* \\
Parental income					&  \checkmark* &\checkmark & \checkmark & $\times$\\
Health 							& \checkmark &$\times$ &\checkmark &  \checkmark* \\
Crime							&  \checkmark* &  $=$ & \checkmark* &  \checkmark* \\
\midrule
All								&  \checkmark* &\checkmark*&  $\times$ & $\times$\\
\bottomrule
\end{tabular}
\end{footnotesize}
\begin{tablenotes}
\footnotesize
\item Note: This table summarizes comparison of gender gaps across outcome categories by different groups. A checkmark indicates that the proportion of outcomes in the corresponding category is larger than 50\%, meaning that males outperform females. A checkmark with an asterisk indicates that the proportion is significantly larger than 50\% at the 10\% level. A cross indicates that the proportion of outcomes is smaller than 50\%. A cross with an asterisk indicates that the proportion is significantly smaller than 50\% at the 10\% level. An equals sign indicates that the proportion is exactly 50\%. Column (1) is the difference between males and females in the full control group.  Column (2) is the difference between males and females in the control group only considering those who stayed at home. Column (3) is the difference between males and females in the control group only considering those who attended alternative childcare. Column (4) is the difference between males and females in the treatment group.
\end{tablenotes}
\end{threeparttable}
\end{table}

This paper unfolds in the following way. Section~\ref{sec:data} describes the experimental data we analyze and its special features. It documents the quality level of the alternative childcare that a considerable proportion of the control-group subjects attended. Section~\ref{sec:parameters} defines the treatment effects we estimate and how we summarize them. Section~\ref{sec:treatment-effects} reports the treatment effects overall and by gender and establishes the existence of sharp gender differences for many categories of outcomes. Section~\ref{sec:gender-differences} discusses the sources of these differences. Section~\ref{sec:conclusion} concludes.


%ee \citet{Beeghly-etal_2017_IMHJ,Dayton_2017_IMHJ,Iruka_2017_IMHJ,Schore_2017_IMHJ} for recent findings on the topic of different development of males and females early in life. 