\documentclass[11pt]{article}

\usepackage{geometry}
\usepackage{lipsum}                    
\usepackage{xargs}                      
\usepackage[pdftex,dvipsnames]{xcolor}  
\usepackage[colorinlistoftodos,prependcaption,textsize=tiny]{todonotes}
\usepackage{titlesec}

\newcommandx{\unsure}[2][1=]{\todo[linecolor=red,backgroundcolor=red!25,bordercolor=red,#1]{#2}}
\newcommandx{\change}[2][1=]{\todo[linecolor=blue,backgroundcolor=blue!25,bordercolor=blue,#1]{#2}}
\newcommandx{\info}[2][1=]{\todo[linecolor=OliveGreen,backgroundcolor=OliveGreen!25,bordercolor=OliveGreen,#1]{#2}}
\newcommandx{\improvement}[2][1=]{\todo[linecolor=Plum,backgroundcolor=Plum!25,bordercolor=Plum,#1]{#2}}
\newcommandx{\thiswillnotshow}[2][1=]{\todo[disable,#1]{#2}}

\titleformat*{\section}{\normalsize\bfseries}
\titleformat*{\subsection}{\normalsize\bfseries}
\titleformat*{\subsubsection}{\normalsize\bfseries}
\titleformat*{\paragraph}{\normalsize\bfseries}
\titleformat*{\subparagraph}{\normalsize\bfseries}


\linespread{1.5}

\begin{document}

\title{Collated Referee Reports}
\author{Notes and Action Items}
\maketitle

\section*{Common Concerns}
\begin{enumerate}
	\item Unobservable characteristics in selection into alternative preschool/home care 
	\item Description of alternative preschool/home care and quality more generally
	\item Too much focus on direction of treatment effects instead of magnitude of treatment effects
	\item Description of CBA results vs. description of treatment effects
	\item Too little reference to other studies
	\item Description of treatment effects too broad
	\item Generalizability of results due to sample size and sample disadvantage
\end{enumerate}

\section*{Reviewer \#2}

This is an interesting paper with potentially important findings regarding the longer-term effects of an innovative early childhood intervention. While based on a relatively small sample, randomized treatment assignment (program entry) permits credible causal inference regarding the program�s effectiveness. While the paper makes a valuable contribution to the literature, the paper has some shortcomings and I see significant scope for further improvements.
\begin{enumerate}
\item The findings in the paper apply to the population of eligible households who signed up to participate in the experiment. Are there systematic differences between eligible households who signed up and did not sign up? It would be useful to give the reader a sense of whether the participating households are representative of the eligible population. \textbf{All those who were offered ABC/CARE took it up, as is written in footnote 24.} \change{Add to main text.}
\item Strictly speaking, only the pooled control-treatment outcome comparisons exploit the randomized treatment assignment. Assuming no other selection issues related to nonrandom dropout, I consider those estimates to be the most credible. Several other main findings in this study instead rely on strong conditional independence assumptions for identification. They involve a comparison of treatment group outcomes with outcomes for two different control groups: those where the child stayed at home and those where the child entered a lower-quality childcare center. For example, among the major findings is that ``Low-quality childcare arrangements are detrimental for boys'' and ``Home care is beneficial for boys compared to low- quality center childcare''. While not always clearly discussed in the paper, these conclusions are presumably based on taking the difference between columns 6 and 4 in Table 4 \unsure{Check and clarify.}, representing treatment effects that control for differences in observed characteristics. An obvious challenge to establishing these causal effects is that the childcare choice among control group members is endogenous. As the authors acknowledge (for example on page 11), those who opt for alternative care are less economically disadvantaged at baseline compared to children who stay at home. So the reliability of these findings in the paper depends on the extent to which the adjusted differences in outcomes reflect selection on unobservables. One would imagine, for example, that parent(s) of more difficult or demanding boys may be more inclined to send them to childcare instead of keeping them at home. So boys in childcare may differ systematically from boys staying at home. How confident can we be that the included controls adequately control for selection into childcare? How do mothers and households who keep boys at home compare to those who choose childcare, and how does this compare for girls? There is no discussion in the paper of any potential role of selection on unobservables. \change{Add more description and discuss limitations more.} \info{Common concern 1}
\item Related to the previous point, it is unclear whether one could attribute a difference in outcomes solely to a child attending a lower-quality childcare center. Presumably the ``treatment'' also involves changes in other inputs as the decision to send a child to childcare involves monetary costs (while ABC/CARE and home care presumably are free) \change{Clarify.}, and may coincide with increased maternal employment. So what is estimated here is the effect of using lower-quality childcare center combined with changes in parental income and time inputs into child development, which could be compensatory or complimentary. Are the negative effects for boys due to lower quality childcare, or do they capture income effects? \unsure{No structural model.} Does the choice perhaps reflect the absence of good alternative informal childcare by the parent(s), grandparents and friends, relative to those who opt for home care?\unsure{Can look into this more.} What exactly counts as �home care�? Does it include informal childcare arrangements? \change{Clarify description.} \info{Common concern 2} These issues deserve more discussion in the paper. Do you have access to data on the time mothers spent with their children, and on time spent with other family, friends and siblings? \textbf{No.} Perhaps more importantly, is there any evidence on gender differences in what constitutes ``staying at home'' or ``lower-quality childcare''? \change{Describe better - make table.} 
\item Related to the previous comment, there is little discussion in the paper of the effect of the intervention on parental income. As the estimated effects look very large, they should be discussed in more detail. Does labor income include zero income values for those with non- working parents? \unsure{Check.} Presumably the large effect then captures changes in maternal employment, thus explaining why the parental income effects are so much bigger relative to the alternative of the child staying at home as compared to alternative center childcare. So one of the channels through which ABC/CARE affects child outcomes is through allowing increased maternal employment? \textbf{This is presumably one aspect, although we cannot distinguish the program's direct effect from the income effect.} \unsure{Check with Jorge.} How does this relate to estimated effects of maternal employment uncovered in the literature? \unsure{Check.}
\item Similar to the treatment of childcare choice among control group members, in your analysis you treat the presence of the father as exogenous, while it seems more like an endogenous outcome to me. \textbf{We control for the presence of the father before randomization.} \change{Clarify this in text.}
\item Nonrandom attrition. On page 9 you discuss that dropout is related to the health of the child and to mobility of families. Both seem highly related to potential outcomes, so this seems to generate a classic case of attrition-induced selection bias. How can we be confident that controlling for observables is sufficient? The estimates in Tables 4 and 5 clearly show that adjusting for covariates has meaningful impacts on treatment effect estimates, but they may still reflect differences in unobservables. Dropping 22 subjects out of a total sample of 121 is obviously not trivial, so it seems to me that the reliability of your approach for controlling for nonrandom attrition deserves more discussion in the main body of the paper. \unsure{Add more discussion from appendix and specifically address this concern.}
\item Most of the analysis in the paper focuses on the direction rather than the magnitude of treatment effects, \info{Common concern 3} and all outcomes are treated as equally important. While I understand your motivation for doing so, clearly some outcomes matter more than others. In my view it would be useful to add a brief discussion of weighting, or relative importance of these outcomes in your benefit (rates of return) analysis. \change{Add to description of CBA results.} \info{Common concern 4} Furthermore, it appears from Tables 4 and 5 that the magnitudes of several estimated effects are very large (almost unrealistically so?), which at a minimum deserves more discussion in the paper. \unsure{Check and explain.} Relatedly, the results in Tables 4 and 5 are limited to average effects, while it may be useful to focus on extreme negative (and perhaps positive) outcomes. \unsure{Discuss distributions for some vars? Ask Jorge.}
\item It was not clear from the paper whether the authors investigated whether treatment effects vary with the presence of a grandparent or sibling, an analysis which could provide additional insights. \unsure{See if this is possible.}
\end{enumerate}

\textbf{Additional comments}
\begin{enumerate}
\item Page 1. It would be useful to add a sentence briefly summarizing earlier findings about differential gender benefits of early-life interventions (among studies that report these). \change{Add this.} \info{Common concern 5}
\item Page 2. Please clarify here that what is meant with ``high-quality childcare'' is childcare provided by the ABC/CARE program, and ``low quality childcare'' is any alternative childcare arrangement used by control group members. It seems more appropriate to call these ``lower-quality childcare'' - can one really say these were low-quality? \info{Common concern 3}
\item Page 2. The statement ``staying at home is a better option for them [boys], especially if the family environment is relatively advantaged (e.g. the father is present)'' appears at odds with the findings reported later in the paper. I presume you meant when the father is absent? \unsure{Check and correct.} \info{Common concern 6}
\item Page 3. How do the findings of the earlier studies listed in footnote 8 compare to those in this paper? \change{Add description.}
\item Page 4. The use of a 10\% significance level is unconventional in empirical economic research. Given the small size of your sample it may be reasonable, and perhaps is a convention in analyses based on combining functions, but in my view deserves a brief explanation. \change{Add explanation.}
\item Page 15. Explain what is meant with ``accounts for model pretesting''. Does this include the formation of the outcome ``blocks''? \change{Check and clarify.}
\item Page 15. Does the test based on the 10\% significance level of treatment effects on outcomes assume independence across outcomes within and across blocks? \textbf{Yes. The stepdown results are in the appendix.} \change{Clarify.}
\item Page 17. You list a finding of increased college graduation for females, but in table 5 this result is missing and you state that you could not estimate the treatment effect for that outcome for women. \change{Correct.} \info{Common concern 6}
\item Page 20. ``Male results are stronger than female results''. Here and elsewhere in the paper the use of ``results'' is often confusing � do you mean outcomes or treatment effects? \change{Clarify.} \info{Common concern 6}
\item Pages 20-22. The statement on page 20 that benefits from ABC/CARE are largely driven by its effects on males' seem inconsistent with the statement on page 22 that ``females benefit more from the program than do males''. Is this apparent contradiction due to the difference between direction and magnitude of treatment effects? \change{Clarify.} \info{Common concern 4 and 6}
\item Page 22. It appears that the results for the ``proportions equal 10\%'' test based on the pooled control group are missing in the paper? \change{Check and correct.}
\end{enumerate}

\section*{Reviewer \#3}

This manuscript considers how a host of treatment outcomes differ between disadvantaged boys and girls who were subjects of two high-quality, random-assignment child care interventions. The manuscript finds that the number of significant socially beneficial treatment effects was higher for girls than boys and that the gender differences principally stem from a smaller number of significant treatment effects for boys when the relevant comparison is home care. The manuscript makes a methodological contribution through its use and statistical analysis of combining functions to count and statistically compare the numbers of significant socially beneficial treatment effects. As a methodological exercise, the manuscript is impressive. However, although the manuscript applies these methods to an important child care experiment, its substantive contribution is less clear. \change{Justify.} First, comparisons of the counts of significant effects may not be especially valuable in this context. There is little consideration of the magnitudes of effects. \info{Common concern 3} Also, many of the outcomes are highly-gendered adult behaviors, such as employment and arrests, where we might not have expectations of comparable effects. Even if child care had similar effects on boys' and girls' early cognitive, emotional and behavioral outcomes, these might not translate into comparable adult outcomes. \unsure{Compare with estimates in national datasets?} Second, the manuscript does not offer theories or explanations of how effects might have occurred in different domains; it focuses instead on the summary measures. \unsure{Discuss different domains.} The main area where it does dig deeper---gender differences in the benefits of the child care intervention relative to either home care or alternative care---doesn't lead to a convincing result, because the manuscript doesn't establish the comparability or representativeness of the comparison groups. \unsure{To be generalized to population?} \info{Common concern 7} The manuscript makes numerous statements about the inferiority of alternative child care for boys, but it hasn't made a strong case. \change{Add to description of alternative care.} \info{Common concerns 2 and 6}

\subsection*{Gender Differences}

The manuscript convincingly documents that there are many differences in outcomes from high-quality child care between boys and girls. The text documents treatment effects for many adult outcomes and shows that women have better treatment outcomes for education and employment, while men have better treatment outcomes for drug use, blood pressure, and hypertension. Some of the gender differences in health outcomes had previously been reported by Campbell et al. (2014) and other outcomes have been reported before. 

The manuscript glosses over differences in the patterns of results. For example, it summarizes the results in Table 6 by writing, ``The general pattern is that male results are stronger than female results in the control group. The pattern is generally reversed in the treatment group.'' Well, yes, but it's still the case that men's treatment effects were better than women's treatment effects for many outcomes (age 5 IQ, college graduation, labor income, and all four health measures) and indistinguishable for a few others. The manuscript formalizes this approach with its combining functions and statistically confirms that there are more significant positive treatment effects for girls than boys. However, these analyses continue to show that there are many positive treatment effects for boys. \change{Tighten description.} \info{Common concern 6}

The manuscript finds differences across many domains. But it's not clear that this is helpful. For example, how do we compare that girls had significantly better home scores at age 1.5 while boys didn't but boys had better labor incomes at 30 while girls didn't. The analysis seems worthwhile within domains of closely related measures but not across domains. More discussion and analysis of Figure 4 (and similar comparisons) would be a better way to go. \change{Implement.} \info{Common concern 6}

One piece of evidence that the counts may not be that valuable come from the reports of benefit-cost ratios that the authors calculated in their companion paper which indicate that the value of the benefits for boys is approximately four times that of girls. \change{Add to description.} \info{Common concern 4}

\subsection*{Alternative Child Care}

The focus of the manuscript turns to the type of control. The numbers of significantly positive treatment effects for girls are mostly similar regardless of the type of control situation, but, there are more differences in the counts for boys when the control is alternative child care rather than home care. Again, however, there are exceptions in the patterns. The treatment for boys is stronger for the home care comparison group for health but stronger for the alternative care group for cognitive skills, parenting, education/employment/income. There is no substantive discussion about why these different results should occur. \change{Improve description.} \info{Common concern 6}

The argument about the inferiority of alternative care seems to stem mainly from the higher count of positive treatment effects for the alternative care control than the home care control. However, these counts are similar to the counts for girls under each of the control situations. So this doesn't seem to provide strong evidence of gender differences in the harms associated with alternative care. Rather it looks like there is something unique about the home care that the boys received. \unsure{Check and explain.}

\subsection*{Selection into Home Care and Alternative Care}

Alternative care was selected by the parents and not randomly assigned. The manuscript uses matching to address the selection issues. However, it provides very little information (none in the body of the manuscript) about how much the resulting matched samples look like the treatment sample or each other. \unsure{Add this?} The manuscript is drawing conclusions about the impacts of alternative care but it hasn't shown that the matched alternative care sample for boys looks like either the home care or treatment samples.

\subsection*{Generalizability}

 The children in this analysis come from extremely disadvantaged backgrounds. The mothers had an average IQ of 84; their average age at baseline was just under 20; and fathers were absent from three-quarters of the homes. Lack of maternal relatives (potential other caregivers) entered into the selection criteria. The initial characteristics of the children should be discussed more in the body of the manuscript for readers (especially those outside the U.S.) who are not familiar with the interventions. \unsure{Check and add.}
 
A particular concern is that with such a modest initial sample and with the gender and control condition cuts that were made to the sample, the authors are trying to generalize from some extremely small cells. This is especially true of the home care cells. If Table A.3 is a guide, the boys' home care group is not only very small (9 children) but has a relatively high income, a high proportion of fathers present, older mothers, and more siblings. This again raises concerns that the differences in the treatment effects reflect something unusual in the boys home care sample. \info{Common concern 7}
 
 \subsection*{Minor Points}
 \begin{itemize}
\item P. 17, manuscript reports that treatment increases girls' college completion by 13 percentage points, but the corresponding figures are dropped from Table 5. \change{Correct.}
\item Table A.5, is ``Mother Works'' a simple dummy variable? I can't determine how you arrived at the percentages listed in the table if it is. \change{Check and clarify.}
 \end{itemize}

\section*{Reviewer \#4}

\subsection*{Summary}

This paper estimates the pooled impacts of two ``model'' randomized early childhood interventions---the Carolina Abecedarian Project (ABC) and the Carolina Approach to Responsive Education (CARE)---separately by gender. Because there are so many outcomes---indeed, nearly as many outcomes as participants (126 vs. 156)---inference focuses on the proportion of positive (and separately, proportion of statistically significant) treatment effects within outcome categories (IQ, education, etc.). In addition to the basic treatment-control comparison, which is based on the randomization of access to (and participation in) the intervention, the authors present comparisons of outcomes in the treatment group to those of subsets of the control group defined by exposure to center-based care under age five, adjusting for selection on observables (only) \unsure{Only??} through matching. Thus, they attempt to assess the impacts of the ``high quality'' ABC/CARE intervention relative to two counterfactuals: home care and the ``low quality'' center-based care available to the control group.

\subsection*{Comments}

Impact evaluations of early childhood programs depend on the counterfactual; in turn, the policy relevance of findings from a particular intervention depend on whether that counterfactual still stands. The fraction of children today in formal childcare or early education settings is much higher than it was in the late 1970s and early 1980s, when the ABC and CARE interventions were carried out. I therefore really like the fact that the authors are trying to understand the importance of the counterfactual to their findings. \change{Add this justification to studying counterfactual.} That the importance may vary by gender is also interesting, and I think novel. I also applaud the authors for producing a very readable and generally clear paper; the graphical representations of the impacts are particularly helpful. 

However, I have three major concerns about the paper as it currently stands. I will elaborate on each of these concerns in the remainder of this report, then offer a few more minor comments.
\begin{description}
\item[Major concern 1]  Participation in center-based care in the control group is not randomized. As the paper is currently written, it seems like the treatment-control/home care and treatment-control/center care inferences are as credible as the treatment-full control group inferences. These inferences are not created equally. The treatment-full control group inference is based on randomization; sample sizes are small, but absent selective attrition, randomization should address both selection on observables and selection on unobservables. By contrast, the treatment-control/home care and treatment-control/center care comparisons only adjust for selection on observables. I think that this needs to be made very clear early on in the paper (well before page 14, where it is first mentioned in the current draft), and appropriate caveats on conclusions drawn from these inferences given throughout the text. \change{Qualify these results more and justify matching more.} \info{Common concern 1}
\item[Major concern 2] There is too much focus on the sign and statistical significance of effects, and too little on magnitudes. In part, this is an artifact of the statistics (and corresponding inference) that the authors have chosen to highlight:  proportions of positive effects within categories. Given that the interest is in gender differences in treatment effects, shouldn't magnitudes matter as well?  I think that there is an easy solution to this:  instead of focusing on proportions, why not focus on standardized treatment effects within categories, as in the Kling, Liebman, and Katz (2007) evaluation of MTO or the Finkelstein, et al. (2012) evaluation of the Oregon Health Insurance Experiment?  Alternatively, monetization of the effects (e.g., to calculate cost-benefit ratios) would provide an omnibus magnitude, but it looks like you've already done this in another paper. \change{Discuss these methods.} \info{Common concerns 3 and 5}
\item[Major concern 3:]  There is inadequate reconciliation with existing literature. \info{Common concern 5} There is (to me) a glaring lack of reconciliation with several key papers. First, and most importantly, Anderson (2008) estimates the impacts of a number of model early interventions---including ABC---correcting for multiple inference. This paper is not even cited, despite the fact that it is well known in this literature (cited 590 times in Google Scholar at the time this report was written). The authors really must cite this paper, and describe what they do differently from it; if gender differences in ABC have already been explored, what is the contribution? Do the authors arrive at different conclusions? \textbf{We use different data and factors. He does not explore gender differences compared to different counterfactuals.} If so, why? Second, given that gender differences in ABC have indeed previously been documented, the attempt to estimate effects relative to specific counterfactuals (home care, center-based care) becomes a more important aspect of the paper's potential contribution. With this in mind, I think that the authors must really reconcile with Kline and Walters (2016) as well as Feller et al. (2016), who do the same thing (but in different ways) in the context of the Head Start Impact Study. Are they taking a similar approach? If not, does taking the approach of, for example, Kline and Walters (2016), matter for the conclusions?  \textbf{Both use HSIS, which had a multi-site randomization. This allows for the use weakly exogenous instruments or stratification.} Approach aside, the authors seem to be doing a similar dissection of counterfactual care options in their companion paper (Garcia et al. 2017). Is the innovation of the current paper simply that you're doing it by gender?  Why is this important?  \change{Address previous studies.}
\end{description}

\textbf{Minor comments}
\begin{enumerate}
\item I would recommend being clearer about what is meant by early education ``quality'' in this paper. Is it simply the cost of the program or in resources (``structural quality'')?  Is it something about interactions between caregivers and children (``process quality'')? \textbf{Yes to both.} I don't think that any process quality measures are available for the ABC or CARE interventions, but discussions of quality in early intervention have evolved, and the text should reflect this. \change{Clarify this.} \info{Common concern 2}
\item Figure 1 is troubling, in that it suggests that there is a lack of balance in observables across the control and treatment groups despite randomization. I think it would be helpful to see more discussion of the validity of randomized variation among non-attriters in the main text. (Relatedly, I don't think that dropouts post-randomization are not a problem, as seems to be implied in footnote 9.) \unsure{Check and explain.}
\item Likewise, it would be helpful to see differences in observable characteristics between control children who attended and did not attend alternative childcare. \change{Add table.} \info{Common concern 1} 
\item Presenting estimates by gender and family background, as is done in section 5, doesn't seem advisable given the very small sample sizes. \change{Check and delete}
\item Needing clarification:
\begin{enumerate}
		\item 1st full para., p. 2:  what do ``differentially promotes'' and ``differentially enhances'' mean exactly?  The language seems to imply that both sexes are positively affected by the intervention on all dimensions stated, but this does not appear to be the case (particularly with the step-down p-values) in Tables 4 and 5. \unsure{Check and clarify}
		\item Last full paragraph, p. 3:  if ``all achievement measures favor males in the control group,'' why isn't the proportion favoring males in the control group 1.0 in Table 1? \unsure{Check}
		\item P. 6:  It is not clear how the home visitation component can be safely ignored. Was there additional randomization of home visitation within the initial treatment group? \textbf{Yes.} \change{Add a little to footnote 14.}
		\item P. 9:  ``Dropouts are evenly balanced across treatments and controls?''  In terms of numbers or observable characteristics? \textbf{Number.} \change{Explain.}
		\item P. 16:  how exactly does column (2) adjust for ``the differences in attrition''? \change{Check and explain.}
		\item Continuing paragraph, top of p. 17:  college graduation estimates are referenced but not given in the table. \change{Check and correct.}
		\item I would be clear in the discussion on pp. 16-17 that you are using the bootstrapped p-values, not the step-down p-values. \change{Clarify}
		\item Pp. 21-22:  How is it that the benefit/cost ratio of ABC/CARE is so much larger for males than for females? This seems contradictory to the following statement that ``females benefit more from the program than males.''  Is that not all outcomes are created equal (in terms of social value)?  This gets back to my comments about the importance of magnitudes. \change{Clarify} \info{Common concern 4}
		\item P. 26:  Another seeming contradiction: estimated treatment effects do not ``appear similar across genders comparing treatment to staying at home full time''; at least it doesn't appear that way in Figures 3 and 4. \change{Check and correct.} 
\end{enumerate}
\end{enumerate}

\end{document}
