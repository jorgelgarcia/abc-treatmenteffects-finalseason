\noindent This paper estimates gender differences in the long-term benefits of a widely implemented prototypical early childhood program that targeted disadvantaged families. The program promoted different aspects of child development for different genders. There are differences in intergenerational educational attainment and employment. We account for control substitution. Males benefit more from high quality treatment in comparison to those who stay at home in disadvantaged environments. Females are less sensitive to low quality home environments. Although both genders benefit, females outperform males on many dimensions. Finally, we reconcile this analysis by concluding that the treatment compensated for early-life deficits that males experience relative to females. This compensation helps the males experience long-lasting benefits of the program into adulthood. \textbf{[Word Count: 116]} \textbf{[JJH: Both programs contribute -- do we eliminate or increase the male-female gaps by crime?] [JJH: This is confusing -- what is being said? (a) How different are female controls from male controls? (b) Who benefits most in terms of value added?]} 