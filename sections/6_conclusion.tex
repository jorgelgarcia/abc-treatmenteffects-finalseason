\textbf{[JJH: I will rewrite when we resolve. Section 5 is a mess. The paper falls apart in Figure 3.] [JLG: I believe that the first two paragraphs much better summarize the paper, and the third paragraph is not relevant anymore and the evidence is not as strong as to support that claim.]}

This paper examines gender differences in the impacts of treatment of an influential early childhood program targeted to disadvantaged children. Upon the impracticality of analyzing the dozens of outcomes available to evaluate the program one at the time and the need to avoid cherry-picking outcomes with significant treatment effects, we propose parametric and non-parametric tests to aggregate and summarize treatment effects. We document that girls benefit more than boys. 

The source of the gender difference is worse control-group conditions for girls resulting from more severe socio-economic disadvantage at baseline. Fathers of sons are more likely to support spouses than fathers of daughters. Boys are more advantaged, with the more advantaged boys being more likely to stay at home. This explains why the effects are larger for males in comparison to alternative care than in comparison to staying at home. For girls, the difference in advantage between the different counterfactual scenarios is not as stark. The effects are larger in general for girls in comparison to staying at home than in comparison to alternative care, but the statistical tests are not as robust as for boys.

Our analysis sounds a cautionary note about the value of early childcare programs. In the past decade, many politicians and pundits have warmly embraced early childhood programs as solutions for reducing inequality and promoting social mobility. Little attention has been paid to the quality of those programs. \cite{Garcia_Heckman_Leaf_etal_2017_Comp_CBA_Unpublished} show that they have a high economic rate of return. Even though we find that females benefit more in terms of effect sizes and in the number of statistically significant treatment effects than do males, after weighing for the social benefits of reduced crime and improved health, education, and employment, the social return is higher for males than females.