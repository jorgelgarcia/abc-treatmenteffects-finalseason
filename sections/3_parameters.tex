Random assignment to treatment does not guarantee that conventional treatment effect estimators answer policy-relevant questions. In this paper, we define and estimate three parameters that address different policy questions.

Let $W=1$ indicate that the parents referred to the program participate in the randomization protocol. $W=0$ indicates otherwise. $R$ indicates randomization into the treatment group ($R = 1$) or to the control group ($R = 0$). $D$ indicates attending the program, i.e., $D = R$ implies compliance with the initial randomization protocol.

Individuals are eligible to participate in the program if baseline background variables $\bm{B}\in\mathcal{B}_0$. $\mathcal{B}_0$ is the set of scores on the risk index that determines program eligibility. Because all of the eligible people given the option to participate choose to do so $(W=1\text{, and } D=R)$, we can safely interpret the treatment effects generated by the experiment as average treatment effects for the population for which $\bm{B}\in\mathcal{B}_0$ and not just treatment effects for the treated.

Let $\bm{Y}^1_a$ be the outcome vector at age $a$ for the treated. $\bm{Y}^0_a$ is the age-$a$ outcome vector for the controls. In principle, life-cycle outcomes for the treated and controls can depend on the exposures to various alternative preschools at each age. It would be desirable to estimate treatment effects for each possible exposure but our samples are too small to make credible estimates for very detailed levels of exposure. All treatment-group children have the same exposure to the ABC/CARE treatment and no exposure to alternative center-based care. 

We simplify the analysis of the controls by creating two categories. ``$H$'' indicates that the control-group child is in home care throughout the entire length of the program. ``$C$'' indicates that a control-group child is in alternative center childcare for any amount of time.\footnote{This assumption is consistent with Figure~\ref{fig:proportion-alt-pre}. Once parents decide to enroll their children in alternative childcare arrangements, the children stay enrolled up to age 5.} We test the sensitivity of our estimates to the choice of different categorizations in our empirical analysis in Appendix~\ref{appendix:vsensitivity-controlsub}.

We thus compress a complex reality into two counterfactual outcome states at age $a$ for control-group subjects:
\begin{align*}
\bm{Y}_{a,H}^0 \quad &: \quad \textbf{ Subject received home care exclusively} \\
\bm{Y}_{a,C}^0 \quad &: \quad \textbf{ Subject received some alternative childcare}.
\end{align*}

We define $V$ as a dummy variable indicating participation by control-group children in an alternative childcare. $V=0$ denotes staying at home. The outcome when a child is in control status is
\begin{equation}
\bm{Y}^0_a : = \left( 1 - V \right) \bm{Y}^0_{a,H} + \left( V \right) \bm{Y}^0_{a,C}. \label{eq:meandiff}
\end{equation}

One parameter of interest addresses the question: what is the effect of the program as implemented? This is the effect of the program compared to the next best alternative as perceived by the parents (or the relevant decision maker) and is defined by
\begin{equation}\label{eq:effect}
\bm{\Delta}_a := \mathbb{E} \left[ \bm{Y}^1_a -  \bm{Y}^0_a | W =1 \right] = \mathbb{E} \left[\bm{Y}^1_a - \bm{Y}^0_a | \bm{B} \in \mathcal{B}_0 \right],
\end{equation}
where the second equality follows because everyone who was eligible elected to participate in the program. For the sample of eligible people, this parameter addresses the effectiveness of the program relative to the quality of all alternatives available when the program was implemented, including staying at home. It is the parameter intended to be estimated by Local Average Treatment Effects (LATE; \citealp{Imbens_Angrist_1994_Econometrica}).

It is fruitful to assess the effectiveness of the program with respect to a counterfactual world in which the child stays at home full time. The associated causal parameter for those who would choose to keep the child at home is:
\begin{equation}\label{eq:influenza}
\bm{\Delta}_a \left(V = 0 \right) : =   \mathbb{E} \left[ \bm{Y}^1_a - \bm{Y}^0_a | V = 0, W = 1 \right] := \mathbb{E} \left[\bm{Y}^1_{a} - \bm{Y}^0_{a,H} | V = 0, \bm{B} \in \mathcal{B}_0 \right].\footnote{Appendix~\ref{appendix:vsensitivity-controlsub} displays results with alternative definitions of $V$ (i.e., different thresholds define if a child attended alternative childcare). The results are robust to the various definitions. What matters is whether any out-of-home child care is being used ($V>0$), and not the specific value of $V$.}
\end{equation}
It is also useful to assess the average effectiveness of a program relative to attendance in an alternative childcare center for those who would choose an alternative:
\begin{equation}\label{eq:smallpox}
\bm{\Delta}_a \left( V =1 \right) : =   \mathbb{E} \left[ \bm{Y}^1_a - \bm{Y}^0_a | V = 1, W = 1 \right] := \mathbb{E} \left[\bm{Y}^1_a - \bm{Y}^0_{a,C} | V = 1, \bm{B} \in \mathcal{B}_0 \right].
\end{equation}

Random assignment to treatment does not directly identify \eqref{eq:influenza} or \eqref{eq:smallpox}. Econometric methods are required to identify these parameters. We primarily rely on matching to control for selection into home or low-quality childcare by the control group. Our procedure assumes that the observed characteristics are sufficient to describe the selection into alternative center-based arrangements. We report results from alternative strategies, including instrumental variables and control functions, in Appendix~\ref{appendix:amethodology}. The results from these alternative strategies are consistent with our main results. We characterize the determinants of choices and our strategy for controlling for selection into home care (``$H$'') and alternative center-based care (``$C$'') when we discuss the empirical results in Section~\ref{sec:treatment-effects}.

\subsection{Summarizing Multiple Treatment Effects}\label{sec:combining-functions}

ABC/CARE has rich longitudinal data on multiple outcomes over multiple periods of the life cycle. Summarizing these effects in an interpretable way is challenging.\footnote{Appendix~\ref{appendix:results} presents step-down $p$-values for the blocks of outcomes that are used in the cost-benefit analysis of \citet{Garcia_Heckman_Leaf_etal_2017_Comp_CBA_Unpublished}. We follow the algorithm in \citet{Romano_Wolf_2016_pval_SaPL}.} Simpler, more digestible summary measures are useful for understanding our main findings. To construct these, we use combining functions that count the proportion of treatment effects that are positive by different categories of outcomes. In a companion paper \citep{Garcia_Heckman_Leaf_etal_2017_Comp_CBA_Unpublished}, we monetize outcomes and estimate rates of return and cost/benefit rates. This requires making an additional layer of assumptions to extrapolate lifetime benefits, which we avoid making in this paper. This cost/benefit analysis gives a weighted summary of the treatment effects, weighing by cost or benefit of the effect to society. The method described below does not weigh the individual effects by this or any other measure of ``importance.''

Consider a block of $N_l$ outcomes indexed by set $Q_l = \{1,\dots,N_l\}$. Let $j \in Q_l$ be a particular outcome within block $l$. Associated with it is a mean treatment effect
\begin{equation}
\Delta_{j,a} : = \mathbb{E} \left[ Y^1_{j,a} - Y^0_{j,a} | \bm{B} \in \mathcal{B}_0 \right], j \in Q_l.
\end{equation}

We assume that outcomes can be ordered so that $\Delta_{j,a} >0$ is beneficial.\footnote{All but 5\% of the outcomes we study can be ranked in this fashion. See Appendix~\ref{appendix:results} for a discussion.} We summarize the estimated effects of the program on outcomes within the block by the count of positive impacts within block $l$:
\begin{equation}
C_l = \sum^{N_l}_{j=1} 1 (\hat{\Delta}_{j,a} >0).
\end{equation}
The proportion of beneficial outcomes in block $l$ is $C_l / N_l$.\footnote{In our empirical application we consider all the outcomes as a block, and then different blocks grouped according to common categories---e.g., skills, health, crime.}

Let $\mathcal{L}$ be the set of blocks. Under the null hypothesis of no treatment effects for all $j \in Q_l, l \in \mathcal{L}$, and assuming the validity of asymptotic approximations, $C_l / N_l$ should be centered around $\frac{1}{2}$. We bootstrap to obtain $p$-values for the null for each block and over all blocks. This procedure accounts for dependence in unobservables across outcomes. It also accounts for model pretesting. Bootstrapping allows us to account for dependence across outcomes (within blocks) in a general way. We adjust for pretesting by estimating a series of alternative models and computing the standard errors that account for doing so.

We also count the beneficial treatment effects that are statistically significant in the sets of outcomes across each of the groups indexed by the set $Q_l$. Using a 10\% significance level, on average 10\% of all outcomes should be ``significant'' at the 10\% level even if there is no treatment effect of the program. We provide evidence against both null hypotheses.\footnote{In this case, we perform a ``double bootstrap'' procedure to first determine significant treatment effects at $10\%$ level and then calculate the standard error of the count.} Combining counts across all blocks enables us to avoid (i) arbitrarily picking outcomes that have statistically significant effects---``cherry picking''; or (ii) arbitrarily selecting blocks of outcomes to correct the $p$-values when accounting for multiple hypothesis testing.\footnote{We present $p$-values for these hypotheses and a number of combining functions by outcome category in Appendix~\ref{appendix:results}.}$^{\text{,}}$\footnote{In Appendix~\ref{appendix:results} we present yet another alternative. We calculate a ``latent'' outcome out of the set of outcomes within a block and perform inference on this latent outcome. This analysis also points to beneficial effects of the program.} 