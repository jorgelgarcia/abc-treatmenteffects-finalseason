To find a set of early, school-age, and adult mediators, we start by focusing on two life-cycle outcomes: labor income and savings in crime costs, both of which are in net present value of 2014 dollars discounted to birth of the subjects assuming a 3\% discount rate.\footnote{These outcomes are estimated following the methods described in \citet{Garcia_etal_2016_Comp_CBA_Unpublished}. These outcomes allow us to analyze the mediation beyond the ages observed at data collection. The inference for the following analysis accounts for the construction of the net present value of income, which uses an auxiliary sample to predict unobserved earnings. We bootstrap over the auxiliary sample in addition to bootstrapping over the ABC/CARE sample. The auxiliary data for crime is the complete population of individuals who committed crimes in North Carolina.}

We first estimate the effect of the early mediators, $\bm{\theta^E}$, on the school-age mediators, $\bm{\theta^S}$. In this case, $\bm{\theta^E}$ and $\bm{\theta^S}$ are vectors containing cognitive, non-cognitive, and parenting skills. For each $\theta^E_s$ in $\bm{\theta^S}$, we calculate

\begin{equation}
	\theta^S_s = \alpha_0 +\bm{ \alpha} \bm{\theta^E} + \varepsilon.
\end{equation}

Then, we estimate the effect of the school-age mediators on the adult mediators, $\bm{\theta^A}$. For each $\theta^A_a$ in $\bm{\theta^A}$, we calculate:

\begin{equation}
	\theta^A_a = \mu_0 + \bm{\mu} \bm{\theta^S} + \epsilon.
\end{equation}

Finally, we estimate the effect of the later mediators on the outcome of interest, $Y$:

\begin{equation}
	Y = \gamma_0 +\bm{\gamma} \bm{\theta^A} + \nu. 
\end{equation}

All of the parameters in these equations are estimated in the full sample of ABC/CARE. That is, we impose that the technology is the same across genders. We then split the sample by gender to calculate the treatment effect and decompose it based on the inputs of the above equations. To get the proportions reported below, we multiply this treatment effect by the estimates of the corresponding parameters from the above equations, and divide by the total treatment effect. This is a standard Laspeyres decomposition.

A pattern that emerges when considering the early and school-age skills as mediators is that cognitive skills tend to mediate for females more so than for males, and that the reverse is true for non-cognitive and parenting skills.

\begin{figure}[H]
\begin{center}
\caption{Early Skills on School-age Non-cognitive Skills}
\label{fig:earlyskills-ncog}
	\includegraphics[width=\textwidth]{../output/mediation/ncogfactor-nch-1-1-1}
\end{center}
\raggedright
Note: 
\end{figure}


\begin{figure}[H]
\begin{center}
\caption{Early Skills on School-age Retention}
\label{fig:earlyskills-retention}
	\includegraphics[width=\textwidth]{../output/mediation/never_ret-nch-1-1-1}
\end{center}
\raggedright
Note: 
\end{figure}


\begin{figure}[H]
\begin{center}
\caption{School-age Skills on Years of Education}
\label{fig:schoolskills-years}
	\includegraphics[width=\textwidth]{../output/mediation/years_30y-ncr-1-1-1}
\end{center}
\raggedright
Note: 
\end{figure}


\begin{figure}[H]
\begin{center}
\caption{School-age Skills on College Graduation}
\label{fig:schoolskills-univ}
	\includegraphics[width=\textwidth]{../output/mediation/si30y_univ_comp-ncr-1-1-1}
\end{center}
\raggedright
Note: 
\end{figure}

\begin{figure}[H]
\begin{center}
\caption{High School Graduation on Lifetime Income}
\label{fig:hs-npvincome}
	\includegraphics[width=\textwidth]{../output/mediation/income-s-1-1-1}
\end{center}
\raggedright
Note: 
\end{figure}

\begin{figure}[H]
\begin{center}
\caption{High School Graduation on Lifetime Crime Savings}
\label{fig:hs-npvcrime}
	\includegraphics[width=\textwidth]{../output/mediation/crime-s-1-1-1}
\end{center}
\raggedright
Note: 
\end{figure}
