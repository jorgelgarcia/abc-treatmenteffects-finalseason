%Input preamble
\documentclass[11pt]{article}

% colors
\usepackage[table]{xcolor}
\definecolor{maroon}{RGB}{153,0,18}
\definecolor{lime}{RGB}{190,213,88}
\definecolor{sand}{RGB}{217,202,179}
\definecolor{fire}{RGB}{144,50,61}
\definecolor{brick}{RGB}{94,11,21}
\definecolor{olive}{RGB}{117,109,84}
\definecolor{lavpink}{RGB}{172,123,132}
\definecolor{darkpurp}{RGB}{49,10,49}
\definecolor{salmon}{RGB}{204,90,113}
\definecolor{mauve}{RGB}{94,73,85}
\definecolor{greyblue}{RGB}{125,132,145}
\definecolor{greypurp}{RGB}{68,56,80}
\definecolor{brightpurp}{RGB}{96,20,255}

% packages (please add in alphabetical order)
\usepackage{adjustbox}
\usepackage{amsfonts}
\usepackage{amsmath}
\usepackage{amssymb}
\usepackage{array}
\usepackage{bm}
\usepackage{booktabs}
\usepackage{caption}
\usepackage{epstopdf}
\usepackage{float}
\usepackage[margin=1in]{geometry}
\usepackage{graphicx}
\usepackage[colorlinks=true, linkcolor=brightpurp, citecolor=brightpurp, urlcolor=salmon]{hyperref}
\usepackage{lipsum}
\usepackage{longtable}
\usepackage{mathtools}
\usepackage{multirow}
\usepackage{natbib}
\usepackage{rotating}
\usepackage{setspace}
\usepackage{subcaption}
%\usepackage{threeparttable}
\usepackage{threeparttablex}
\usepackage{xr}
\usepackage[printwatermark]{xwatermark}


\newcolumntype{L}[1]{>{\raggedright\let\newline\\\arraybackslash\hspace{0pt}}m{#1}}
\newcolumntype{C}[1]{>{\centering\let\newline\\\arraybackslash\hspace{0pt}}m{#1}}
\newcolumntype{R}[1]{>{\raggedleft\let\newline\\\arraybackslash\hspace{0pt}}m{#1}}

% commands
\newcommand{\mr}{\multirow}
\newcommand{\mc}{\multicolumn}

%Other parameters
\newcommand{\noutcomes}{95}
\newcommand{\treatsubsabc}{$75\%$}
\newcommand{\treatsubscarec}{$74\%$}
\newcommand{\treatsubscaref}{$63\%$}

%Counts
%Males
\newcommand{\positivem}{$79\%$}
\newcommand{\positivesm}{$37\%$}

%Females
\newcommand{\positivef}{$73\%$}
\newcommand{\positivesf}{$35\%$}

%Counts, control substitution
%Males
\newcommand{\positivecsnm}{$58\%$}
\newcommand{\positivescsnm}{$25\%$}

\newcommand{\positivecsam}{$74\%$}
\newcommand{\positivescsam}{$38\%$}

%Females
%% no alternative
\newcommand{\positivecsnf}{$83\%$}
\newcommand{\positivescsnf}{$46\%$}

%% alternative
\newcommand{\positivecsaf}{$73\%$}
\newcommand{\positivescsaf}{$23\%$}

%Pooled

%Effects
%Males

%Females
\newcommand{\hsgradf}{$7$}
\newcommand{\yearsedf}{$1.2$}



%Pooled

%CBA
%IRR
%Males
\newcommand{\irrm}{$15\%$}
\newcommand{\irrsem}{$13\%$}

%Females
\newcommand{\irrf}{$10\%$}
\newcommand{\irrsef}{$12\%$}

%Pooled
\newcommand{\irrp}{$13\%$}
\newcommand{\irrsep}{$11\%$}

%BC
%Males
\newcommand{\bcm}{$7.88$}
\newcommand{\bcsem}{$8.06$}

%Females
\newcommand{\bcf}{$2.30$}
\newcommand{\bcsef}{$1.56$}

%Pooled
\newcommand{\bcp}{$4.35$}
\newcommand{\bcsep}{$2.57$}

%NPV streams
%Pooled
\newcommand{\parincomenpvp}{$\$115,026$}

\externaldocument{abc_comprehensivecba}
\pagenumbering{roman}

\begin{document}
\title{\Large \textbf{ABC CARE: Heterogeneous Treatment Effects \\ Rough Approximation}}
\maketitle

\newgeometry{left=.1in,right=.1in,top=.1in,bottom=.1in}
\begin{sidewaystable}[H] 
\begin{threeparttable}
\caption{Treatment Effects and the High Risk Index}
\centering 
\scriptsize
\begin{tabular}{l|cccccc|cccccc} \toprule
 & (1) & (2) & (3) & (4) & (5) & (6) & (7) & (8) & (9) & (10) & (11) & (12) \\
 & \multicolumn{6}{c}{\textbf{{Males}}} & \multicolumn{6}{c}{\textbf{{Females}}} \\
 & IQ (5) & IQ (5) & Educ (30) & Educ (30) & Employed (30) & Employed (30)  & IQ (5) & IQ (5) & Educ (30) & Educ (30) & Employed (30) & Employed (30) \\ \\ \midrule
R & 7.697*** & 4.990 & 0.525 & -0.137 & 0.119 & -0.083 & 4.921 & -9.523 & 2.143*** & -2.200 & 0.131 & -0.856** \\
 & (2.802) & (10.006) & (0.508) & (2.065) & (0.104) & (0.404) & (3.217) & (11.859) & (0.624) & (2.348) & (0.105) & (0.386) \\
HRI  &  & -0.486 &  & -0.030 &  & -0.029* &  & -0.655* &  & -0.126* &  & -0.029** \\
 &  & (0.342) &  & (0.081) &  & (0.016) &  & (0.353) &  & (0.069) &  & (0.011) \\
R*HRI &  & 0.119 &  & 0.034 &  & 0.010 &  & 0.628 &  & 0.206* &  & 0.046** \\
 &  & (0.486) &  & (0.102) &  & (0.020) &  & (0.544) &  & (0.108) &  & (0.018) \\
Cons & 93.446*** & 103.281*** & 12.903*** & 13.489*** & 0.710*** & 1.288*** & 95.629*** & 110.590*** & 11.757*** & 14.595*** & 0.703*** & 1.349*** \\
 & (1.954) & (7.178) & (0.370) & (1.637) & (0.076) & (0.320) & (2.185) & (8.362) & (0.418) & (1.610) & (0.070) & (0.265) \\ \\ \midrule
Observations & 72 & 72 & 66 & 66 & 66 & 66 & 65 & 65 & 67 & 66 & 67 & 66 \\
$R^2$ & 0.097 & 0.137 & 0.016 & 0.019 & 0.020 & 0.106 & 0.036 & 0.087 & 0.154 & 0.214 & 0.023 & 0.132 \\ \bottomrule
\end{tabular}

\begin{tablenotes}
\footnotesize
\item Note: This table presents results from regressing the variables listed by column on an indicator of randomization to treatment in ABC/CARE, the high risk index (HRI), the interaction of these two variables, and a constant. IQ (5) is the Wechsler Preschool and Primary Scale of Intelligence at Age 5. Educ (30) is years of education at age 30, and Employed (30) is an indicator for being employed at age 30. HRI: Eligibility to ABC/CARE was determined by a score of 11 or more on a weighted 13-factor High-risk Index (HRI). These included various measures of household-level disadvantages. Asymptotic standard errors are in parentheses. $^{***}$: $p$-value $< .01$. $^{**}$: $p$-value $< .05$. $^{*}$: $p$-value $< .10$. 
\end{tablenotes}
\end{threeparttable}
\end{sidewaystable}


\end{document} 