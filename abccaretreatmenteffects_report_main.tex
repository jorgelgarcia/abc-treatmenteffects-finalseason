%Input preamble
\documentclass[11pt]{article}

% colors
\usepackage[table]{xcolor}
\definecolor{maroon}{RGB}{153,0,18}
\definecolor{lime}{RGB}{190,213,88}
\definecolor{sand}{RGB}{217,202,179}
\definecolor{fire}{RGB}{144,50,61}
\definecolor{brick}{RGB}{94,11,21}
\definecolor{olive}{RGB}{117,109,84}
\definecolor{lavpink}{RGB}{172,123,132}
\definecolor{darkpurp}{RGB}{49,10,49}
\definecolor{salmon}{RGB}{204,90,113}
\definecolor{mauve}{RGB}{94,73,85}
\definecolor{greyblue}{RGB}{125,132,145}
\definecolor{greypurp}{RGB}{68,56,80}
\definecolor{brightpurp}{RGB}{96,20,255}

% packages (please add in alphabetical order)
\usepackage{adjustbox}
\usepackage{amsfonts}
\usepackage{amsmath}
\usepackage{amssymb}
\usepackage{array}
\usepackage{bm}
\usepackage{booktabs}
\usepackage{caption}
\usepackage{epstopdf}
\usepackage{float}
\usepackage[margin=1in]{geometry}
\usepackage{graphicx}
\usepackage[colorlinks=true, linkcolor=brightpurp, citecolor=brightpurp, urlcolor=salmon]{hyperref}
\usepackage{lipsum}
\usepackage{longtable}
\usepackage{mathtools}
\usepackage{multirow}
\usepackage{natbib}
\usepackage{rotating}
\usepackage{setspace}
\usepackage{subcaption}
%\usepackage{threeparttable}
\usepackage{threeparttablex}
\usepackage{xr}
\usepackage[printwatermark]{xwatermark}


\newcolumntype{L}[1]{>{\raggedright\let\newline\\\arraybackslash\hspace{0pt}}m{#1}}
\newcolumntype{C}[1]{>{\centering\let\newline\\\arraybackslash\hspace{0pt}}m{#1}}
\newcolumntype{R}[1]{>{\raggedleft\let\newline\\\arraybackslash\hspace{0pt}}m{#1}}

% commands
\newcommand{\mr}{\multirow}
\newcommand{\mc}{\multicolumn}


% \externaldocument{abccaretreatmenteffects_report_appendix}

\begin{document}
\title{\Large \textbf{Analyzing the Short and Long-term Effects of Early Childhood Education on Multiple Dimensions of Human Development}\thanks{This research was supported in part by the American Bar Foundation; the Pritzker Children's Initiative, the
Buffett Early Childhood Fund, NIH grants NICHD R37HD065072, NICHD R01HD54702, and NIA R24AG048081, an
anonymous funder, Successful Pathways from School to Work, an initiative of the University of Chicago's Committee
on Education funded by the Hymen Milgrom Supporting Organization, and the Human Capital and Economic
Opportunity Global Working Group, an initiative of the Center for the Economics of Human Development, affiliated with
the Becker Friedman Institute for Research in Economics, and funded by the Institute for New Economic Thinking. The
views expressed in this paper are solely those of the authors and do not necessarily represent those of the funders or
the official views of the National Institutes of Health. For helpful comments, we thank St\'{e}phane Bonhomme, Steven Durlauf, and Azeem Shaikh. For information on the implementation of the Carolina Abecedarian Project and assistance in data acquisition, we thank Peg Burchinal, Carrie Bynum, Frances Campbell, and Elizabeth Gunn. For information on childcare in North Carolina, we thank Richard Clifford and Sue Russell. For exceptional research assistance, we thank Thomas Choi. Lastly, we thank Sylvi Kuperman for sharing detailed and careful descriptions of the Carolina Abecedarian Project, as well as the federal and state childcare policies while the program was active. This information helped us improve estimates of the costs of the program and its alternatives.}}

\author{
Jorge Luis Garc\'{i}a\\
The University of Chicago \and
James J. Heckman \\
American Bar Foundation \\
The University of Chicago \and
Andr\'{e}s Hojman\\
The University of Chicago \and
Yu Kyung Koh \\ 
The University of Chicago \and
Joshua Shea \\
The University of Chicago \and
Anna Ziff \\ 
The University of Chicago}
\date{First Draft: January 5, 2016\\ This Draft: \today}
\maketitle

\singlespacing
\pagebreak
\tableofcontents
\listoffigures
\listoftables
\pagebreak

\section{Background} \label{section:bakcground}
\subsection{Overview}

\noindent The Carolina Abecedarian Project (ABC) and the Carolina Approach to Responsive Education (CARE) were both designed and implemented by full-time researchers and professional staff at the Frank Porter Graham Center (FPGC) of the University of North Carolina in Chapel Hill. The programs recruited very disadvantaged children from the semi-rural communities surrounding the area.\\

\noindent In both studies, the sample sizes were small: ABC recruited 122 children over four cohorts, while CARE recruited 65 children over two cohorts. ABC had two phases. The first phase began at birth and ended up at age 5. It randomly assigned children into either treatment or control. The treatment group received: (i) center-based childcare; (ii) breakfast, lunch, an afternoon snack, iron-fortified formula for the first 15 months of life, and a monthly supply of diapers; and (iii) medical care from licenses nurses supervises by a pediatrician, frequent health check-ups, and hospital referrals in case of need of serious medical treatment.The control group only received the formula and the diapers. The second phase began at age 5 and ended up at age 8. It randomly assigned the 95 children who remained in the program at age 5 into treatment or control, independently of their status in the second randomization. The treatment consisted of home visits targeting both children and parents.\\ 

\noindent The design and implementation of CARE was closely related to the first phase of ABC. An additional objective of CARE was to study the importance of improving the home environment on child development. Therefore, CARE extended the design of ABC to include a family education program. The program had a single phase, from birth up to age 5. Children were randomly assigned to either of three experimental groups: (i) control---23 children; (ii) family education---25 children; and (iii) family education and center-based childcare---17 children. The first group received no treatment at all. The second group received home visits that helped the parents to solve everyday problems that could interfere with the way in which they interacted with their children.\\

\noindent Data availability is vast and very similar in both programs. In the first 8 years of life, yearly data on cognitive and socio-emotional skills, home environment, and family structure and economic conditions were collected. After age 8, the collection of data was scattered. Information on cognitive and socio-emotional skills, educaton, and family economic conditions was collected at ages 12, 15, 21, and 30.\footnote{At age 30, information on cognitive skills is unavailable both in ABC and CARE.} In addition, we have two elements that are unique in the literature: administrative criminal records and a full medical sweep at age 34. These rich sources of data allow us to study the long-term effects on several aspects of human development.

\subsection{Eligibility Criteria and Populations Served}

\noindent ABC recruited four cohorts of children, who were born in the years 1973 to 1977. CARE recruited two cohorts of children, who were born in 1978 and 1979. The recruitment process was identical. Potential families were referred by local social service agencies and hospitals at the beginning of the mothers' last trimester of pregnancy. Eligibility for inclusion was determined by a score of 11 or more on a weighted 13-factor High-Risk Index.\footnote{The HRI was based on 13 weighted variables. (i) maternal educational levels —years of education followed by weight in parentheses: ≤6 (8); 7 (7); 8 (6); 9 (3); 10 (2) 11 (1); ≥12 (0)— (ii); paternal educational levels —weights identical to scheme for mother’s education; (iii) family income —income bracket in current dollars followed by weight in parentheses: ≤1,000 (8); 1,001-2000 (7); 2,001-3,000 (6); 3,001-4,000 (5); 4,001-5,000 (4); ≥ 5,001 (0); (iv) father’s absence in the household for reasons other than health or death (weight = 3); (v) absence of maternal relatives in the area (weight = 3); (vi) siblings of school age one or more grades behind age- appropriate level, or with equivalently low scores on school-administered achievement tests (weight = 3); (vii) payments received from welfare agencies within past 3 years (weight = 3); (viii) record of father’s work indicates unstable or unskilled and semi-skilled labor (weight = 3); (ix) record of mother’s or father’s IQ indicate scores of 90 or below (weight = 3); (x) records of sibling’s IQ indicates scores of 90 or below (weight = 3); (xi) relevant social agencies in the community indicate the family is in need of assistance; (xii) one or more members of the family has sought counseling or professional help in the past 3 years (weight = 1); and (xiii) special circumstances not included in any of the above that are likely contributors to cultural or social disadvantage (weight = 1).} Table~\ref{table:} compares the ABC and CARE children with respect to the most representative elements in the HRI. The children in CARE were relatively less disadvantaged: their mothers were significantly more educated and worked more before pregnancy; their parents had more income total income.\\

\noindent \textbf{[JLG: ABC vs. CARE Table]}.\\

\noindent Neither ABC nor CARE considered race as an eligibility factor. However, $98\%$ and $90\%$ of their participants, respectively, were African-American. Mean maternal age in ABC (CARE) when the target child was born was 19.9 (21) years, and approximately half of the mothers of both the treatment and control groups were 19 or younger.\footnote{Maternal age at target child’s birth ranges from 13 to 43 years of age. One-third were 17 or younger.} Mean maternal IQ for both treatment and control groups was approximately 85 (87 in CARE), one standard deviation below the national mean. Only $25\%$ of the ABC children lived with both biological parents, and more than $50\%$ lived with extended families in multi-generational households ($61\%$ of treated children and $56\%$ of control children).\\

\begin{figure}[H]
\caption{Family Environment Baseline Characteristics, ABC and CARE}  \label{figure:baselineabccare}
    \centering
\begin{subfigure}{.5\textwidth}
  \centering
  \subcaption{Average Maternal Age}
  \includegraphics[height=2.3in]{output/abccarepsid_m_age0pool.eps}
\end{subfigure}%
\begin{subfigure}{.5\textwidth}
  \centering
  \subcaption{Average Maternal Education} 
  \includegraphics[height=2.3in]{output/abccarepsid_m_edu0pool.eps}
\end{subfigure}

\begin{subfigure}{.5\textwidth}
  \centering
  \subcaption{Proportion of Households with Father at Home}
  \includegraphics[height=2.3in]{output/abccarepsid_f_home0pool.eps}
\end{subfigure}%
\begin{subfigure}{.5\textwidth}
  \centering
  \subcaption{Median Parental Income in 1,000 of 2014 USD}
  \includegraphics[height=2.3in]{output/abccarepsid_p_inc0pool.eps}
\end{subfigure}
\floatfoot{
\footnotesize
\noindent  Note: These panels plot four variable characteristics---mother's age, mother's education, an indicator of father at home, and parental income (in thousands of 2014 USD). In each panel, the first bar shows the national-level for a cohort born in the same years as the ABC and CARE individuals (1973-1979), obtained from the Panel Study of Income Dynamics (PSID). The second bar uses this same information but restricted to black individuals. The third and fourth bars plot the same variables for ABC and CARE, pooling the treatment and control groups. We display the sample size for each calculation in parentheses next to the horizontal axis labels.
}
\end{figure}

\noindent To better understand the socio-economic status of the ABC participant families, we construct two comparison groups using the Panel Study of Income Dynamics (PSID), a nationally representative cohort of the children born in the same years as the ABC and CARE children (1973-1979), and a similar cohort restricted to black children. We show a comparison in Figure~\ref{figure:baselineabccare}. Compared to both nationally representative groups, ABC children were, on average, significantly more disadvantaged; they were born to younger, less educated mothers, most of whom were raising their children without the support of a father, and the total parental labor income was extremely low due to the father not being at home and the low human capital of mothers. The participants of CARE were also disadvantaged compared to nationally representative groups with respect to basic household socio-demographic characteristics.\\

\subsection{Randomization Protocol and Compromises}

\subsubsection{ABC}

\noindent The first-phase randomization was paired at the family level. This implied jointly randomizing a pair of siblings and a pair of twins to either treatment or control.\footnote{The two siblings in the pair were close enough in age as for both of them to be eligible to the program.} Although we know that the pairing was based on HRI, maternal characteristics, number of siblings, and gender, we do not knows the original pairs. In total, 120 were part of the study. As discussed below, our calculations indicate that 112 families participated of the initial randomization, while 8 families became part of the program as replacement for the families that withdrew.\\

\noindent There were some compromises to first-phase randomization. One control children participated of the program and three children in the treatment group did not participate, by decision of their families. We have information on these children and their families from birth and up the end of elementary school. Four families moved out out from the area before any data on them was collected. Only data during their adulthood is available. Two children in the control group were swapped to treatment at the request of local authorities, because they were considered at high-risk of being developmentally delayed. Other two children in the control group were dropped because they were diagnosed with developmental delay, which made them ineligible for the program. Most of the data for these last four control children is available from birth and up to elementary school. Finally, four children died before age 5, two in the control group and two in the treatment group.\footnote{The two control children died from at 3 and 18 months from cardiomyopathy and seizure disorder and a cardiac arrest, respectively. One child in the control group died at 3 months old from crib, while the other died in a pedestrian accident at 50 months old \citet{}.} Based on this, we classify 14 children as cases of attrition. Table~\ref{table:} compares their baseline characteristics with the baseline characteristics of the children who complied to the first-phase randomization.\\

\noindent \textbf{[JLG: Compliers vs. Attriters Table]}\\

\noindent The children who stopped participating of the program before they were 6 months old were replaced. These children were: (i) two who died; (ii) one who withdrew; and (iii) one who was diagnosed with developmental delay.\footnote{Three replacements are documented in \citet{•}. One is documented in correspondence with the program officers, which is available from the authors of this document upon request. The rest are implied by the number of children who participated of the randomization in each cohort.} This gave a final first-phase sample of 111 children, 53 treatment children and 58 control children.\\

\noindent At the moment of the second-phase randomization, 3 children in the control group of the first phase and 3 children in the treatment group of the first phase stopped being followed. One child in the control group and 8 children in the treatment group of the first phase did not participate of the second phase but later agreed to participate of the data collections during adulthood. This yielded a sample of 96 children, who were randomized to treatment (49) and control (47), independently of their status in the second phase. Three children in the control group did not comply to the randomization protocol by not participating, while all children in the control group complied to their treatment status. Figure~\ref{fig:abc-flow} illustrates the randomization protocol for the two phases of the program.

\begin{center}
	\begin{figure}[H]
		\caption{Randomization Protocol and Treatment Compliance, ABC} \label{fig:abc-flow}
		\centering
		\includegraphics[width=.9\columnwidth]{output/Diagram_ABC.pdf}
\floatfoot{
\tiny
\noindent
Sources: \cite{Ramey_Collier_etal_1976_CarolinaAbecedarianProject, Ramey_Smith_1977_AJMD,Ramey_Campbell_1979_SR,Ramey_Campbell_1984_AJMD}, internal documents of the program, and our own calculations using the data. Note: The variable R represents randomization into treatment [R=1] or control [R=0] groups. After the original randomization, some children died or withdrew the program early in life and were replaced. R also includes those replacements. Arrows pointing outside the diagram indicate children that left the study permanently. The variable P represents participation in the preschool-age program [P=1] or not participation on it [P=0]. The variable SR represents randomization into the school-age program [SR=1], or out of it [SR=0]. Some children were not randomized at school age [SR=No]. We use the term temporary attrited for children that did not participate in the study at school age, but were later interviewed in the age-21 followup.
}
	\end{figure}
\end{center}

\noindent While there was balance in observed, baseline characteristics between the children in the treatment and control groups in the two randomization phases (see Table~\ref{table:}), the explained compromises pose a methodological challenge that we assess when calculating the effects the program for all the outcomes and across all the ages for which we have data. We develop this further in Section~\ref{section:methodology}.\\

\noindent \textbf{[JLG: balance in observed characteristics for all both phases.]}\\

\subsubsection{CARE}

\noindent The randomization protocol in CARE was analogous to the first-phase randomization protocol in ABC. Importantly, it had no major compromises. Of the initial 65 families, 23 were randomized to control, 25 to family education treatment, and 17 to family education treatment and center-based childcare treatment. Two families in the family education treatment group had twins. As in ABC, the twin children were jointly randomized. After randomization, the family of one child initially assigned to the family education and center-based childcare treatment refused its status and did not participate of the program. Within the first six months of treatment, a child assigned to this same group left the study due to family reallocation. Finally, one child who was assigned to the family education treatment group died.\\

\noindent 


\subsection{Treatment Substitution}

\subsection{Program Description and Content}

\section{Methodology} \label{section:methodology}

%References
\clearpage
\singlespace
\bibliographystyle{chicago}
\bibliography{heckman}

\end{document} 