%Input preamble
\documentclass[11pt]{article}

% colors
\usepackage[table]{xcolor}
\definecolor{maroon}{RGB}{153,0,18}
\definecolor{lime}{RGB}{190,213,88}
\definecolor{sand}{RGB}{217,202,179}
\definecolor{fire}{RGB}{144,50,61}
\definecolor{brick}{RGB}{94,11,21}
\definecolor{olive}{RGB}{117,109,84}
\definecolor{lavpink}{RGB}{172,123,132}
\definecolor{darkpurp}{RGB}{49,10,49}
\definecolor{salmon}{RGB}{204,90,113}
\definecolor{mauve}{RGB}{94,73,85}
\definecolor{greyblue}{RGB}{125,132,145}
\definecolor{greypurp}{RGB}{68,56,80}
\definecolor{brightpurp}{RGB}{96,20,255}

% packages (please add in alphabetical order)
\usepackage{adjustbox}
\usepackage{amsfonts}
\usepackage{amsmath}
\usepackage{amssymb}
\usepackage{array}
\usepackage{bm}
\usepackage{booktabs}
\usepackage{caption}
\usepackage{epstopdf}
\usepackage{float}
\usepackage[margin=1in]{geometry}
\usepackage{graphicx}
\usepackage[colorlinks=true, linkcolor=brightpurp, citecolor=brightpurp, urlcolor=salmon]{hyperref}
\usepackage{lipsum}
\usepackage{longtable}
\usepackage{mathtools}
\usepackage{multirow}
\usepackage{natbib}
\usepackage{rotating}
\usepackage{setspace}
\usepackage{subcaption}
%\usepackage{threeparttable}
\usepackage{threeparttablex}
\usepackage{xr}
\usepackage[printwatermark]{xwatermark}


\newcolumntype{L}[1]{>{\raggedright\let\newline\\\arraybackslash\hspace{0pt}}m{#1}}
\newcolumntype{C}[1]{>{\centering\let\newline\\\arraybackslash\hspace{0pt}}m{#1}}
\newcolumntype{R}[1]{>{\raggedleft\let\newline\\\arraybackslash\hspace{0pt}}m{#1}}

% commands
\newcommand{\mr}{\multirow}
\newcommand{\mc}{\multicolumn}


% \externaldocument{abccaretreatmenteffects_report_appendix}

\begin{document}
\title{\Large \textbf{Analyzing the Long-term Effects of Social Experiments on Multiple Dimensions of Human Development}\thanks{This research was supported in part by the American Bar Foundation; the Pritzker Children's Initiative, the
Buffett Early Childhood Fund, NIH grants NICHD R37HD065072, NICHD R01HD54702, and NIA R24AG048081, an
anonymous funder, Successful Pathways from School to Work, an initiative of the University of Chicago's Committee
on Education funded by the Hymen Milgrom Supporting Organization, and the Human Capital and Economic
Opportunity Global Working Group, an initiative of the Center for the Economics of Human Development, affiliated with
the Becker Friedman Institute for Research in Economics, and funded by the Institute for New Economic Thinking. The
views expressed in this paper are solely those of the authors and do not necessarily represent those of the funders or
the official views of the National Institutes of Health. For helpful comments, we thank St\'{e}phane Bonhomme, Steven Durlauf, and Azeem Shaikh. For information on the implementation of the Carolina Abecedarian Project and assistance in data acquisition, we thank Peg Burchinal, Carrie Bynum, Frances Campbell, and Elizabeth Gunn. For information on childcare in North Carolina, we thank Richard Clifford and Sue Russell. For exceptional research assistance, we thank Thomas Choi. Lastly, we thank Sylvi Kuperman for sharing detailed and careful descriptions of the Carolina Abecedarian Project, as well as the federal and state childcare policies while the program was active. This information helped us improve estimates of the costs of the program and its alternatives.}}

\author{
Jorge Luis Garc\'{i}a\\
The University of Chicago \and
James J. Heckman \\
American Bar Foundation \\
The University of Chicago \and
Andr\'{e}s Hojman\\
The University of Chicago \and
Yu Kyung Koh \\ 
The University of Chicago \and
Joshua Shea \\
The University of Chicago \and
Anna Ziff \\ 
The University of Chicago}
\date{First Draft: January 5, 2016\\ This Draft: \today}
\maketitle

\singlespacing
\pagebreak
\tableofcontents
\listoffigures
\listoftables
\pagebreak

\section{Background}
\subsection{Overview}

\noindent The Carolina Abecedarian Project (ABC) and the Carolina Approach to Responsive Education (CARE) were both designed and implemented by full-time researchers and professional staff at the Frank Porter Graham Center (FPGC) of the University of North Carolina in Chapel Hill. The programs recruited very disadvantaged children from the semi-rural communities surrounding the area.\\

\noindent In both studies, the sample sizes were small: ABC recruited 116 children over four cohorts, while CARE recruited 65 children over two cohorts. ABC had two phases. The first phase began at birth and ended up at age 5. It randomly assigned 116 children into either treatment or control. The treatment group received: (i) center-based childcare; (ii) breakfast, lunch, an afternoon snack, iron-fortified formula for the first 15 months of life, and a monthly supply of diapers; and (iii) medical care from licenses nurses supervises by a pediatrician, frequent health check-ups, and hospital referrals in case of need of serious medical treatment.The control group only received the formula and the diapers. The second phase began at age 5 and ended up at age 8. It randomly assigned the 95 children who remained in the program at age 5 into treatment or control, independently of their status in the second randomization. The treatment consisted of home visits targeting both children and parents.\\ 

\noindent The design and implementation of CARE was closely related to the first stage of ABC. An additional objective of CARE was to study the importance of improving the home environment on child development. Therefore, CARE extended the design of ABC to include a family education program. The program had a single stage, from birth up to age 5. Children were randomly assigned to either of three experimental groups: (i) control---23 children; (ii) family education---25 children; and (iii) family education and center-based childcare---17 children. The first group received no treatment at all. The second group received home visits that helped the parents to solve everyday problems that could interfere with the way in which they interacted with their children.\\

\noindent Data availability is vast and very similar in both programs. In the first 8 years of life, yearly data on cognitive and socio-emotional skills, home environment, and family structure and economic conditions were collected. After age 8, the collection of data was scattered. Information on cognitive and socio-emotional skills, educaton, and family economic conditions was collected at ages 12, 15, 21, and 30.\footnote{At age 30, information on cognitive skills is unavailable both in ABC and CARE.} In addition, we have two elements that are unique in the literature: administrative criminal records and a full medical sweep at age 34. These rich sources of data allow us to study the long-term effects on several aspects of human development.

\subsection{Eligibility Criteria and Populations Served}

\noindent ABC recruited four cohorts of children, who were born in the years 1973 to 1977. CARE recruited two cohorts of children, who were born in 1978 and 1979. The recruitment process was identical. Potential families were referred by local social service agencies and hospitals at the beginning of the mothers' last trimester of pregnancy. Eligibility for inclusion was determined by a score of 11 or more on a weighted 13-factor High-Risk Index.\footnote{The HRI was based on 13 weighted variables. (i) maternal educational levels —years of education followed by weight in parentheses: ≤6 (8); 7 (7); 8 (6); 9 (3); 10 (2) 11 (1); ≥12 (0)— (ii); paternal educational levels —weights identical to scheme for mother’s education; (iii) family income —income bracket in current dollars followed by weight in parentheses: ≤1,000 (8); 1,001-2000 (7); 2,001-3,000 (6); 3,001-4,000 (5); 4,001-5,000 (4); ≥ 5,001 (0); (iv) father’s absence in the household for reasons other than health or death (weight = 3); (v) absence of maternal relatives in the area (weight = 3); (vi) siblings of school age one or more grades behind age- appropriate level, or with equivalently low scores on school-administered achievement tests (weight = 3); (vii) payments received from welfare agencies within past 3 years (weight = 3); (viii) record of father’s work indicates unstable or unskilled and semi-skilled labor (weight = 3); (ix) record of mother’s or father’s IQ indicate scores of 90 or below (weight = 3); (x) records of sibling’s IQ indicates scores of 90 or below (weight = 3); (xi) relevant social agencies in the community indicate the family is in need of assistance; (xii) one or more members of the family has sought counseling or professional help in the past 3 years (weight = 1); and (xiii) special circumstances not included in any of the above that are likely contributors to cultural or social disadvantage (weight = 1).} Table~\ref{table:} compares the ABC and CARE children with respect to the most representative elements in the HRI. The children in CARE were relatively less disadvantaged: their mothers were significantly more educated and worked more before pregnancy; their parents had more income total income.\\

\noindent \textbf{[Table: ABC vs. CARE]}.\\

\noindent Neither ABC nor CARE considered race as an eligibility factor. However, $98\%$ and $90\%$ of their participants, respectively, were African-American. Mean maternal age in ABC (CARE) when the target child was born was 19.9 (21) years, and approximately half of the mothers of both the treatment and control groups were 19 or younger.\footnote{Maternal age at target child’s birth ranges from 13 to 43 years of age. One-third were 17 or younger.} Mean maternal IQ for both treatment and control groups was approximately 85 (87 in CARE), one standard deviation below the national mean. Only $25\%$ of the ABC children lived with both biological parents, and more than $50\%$ lived with extended families in multi-generational households ($61\%$ of treated children and $56\%$ of control children).\\

\noindent To better understand the socio-economic status of the ABC participant families, we construct two comparison groups using the Panel Study of Income Dynamics (PSID), a nationally representative cohort of the children born in the same years as the ABC and CARE children (1973-1979) and a similar cohort restricted to black children. Compared to both nationally representative groups, ABC children were, on average, significantly more disadvantaged; they were born to younger, less educated mothers, most of whom were raising their children without the support of a father, and the total parental labor income was extremely low due to the father not being at home and the low human capital of mothers. The participants of CARE were also disadvantaged compared to nationally representative groups with respect to basic household socio-demographic characteristics.

\subsection{Randomization Protocol and Compromises}

\subsection{Treatment Substitution}

\subsection{Program Description and Content}

%References
\clearpage
\singlespace
\bibliographystyle{chicago}
\bibliography{heckman}

\end{document} 