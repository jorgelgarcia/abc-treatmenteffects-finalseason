\begin{sidewaystable}[H]
\captionsetup{singlelinecheck=false,justification=centering}
\caption{Treatment Effects of Center-based Childcare on Females, Counts  \label{tab:counts_female}}

  \begin{threeparttable}
  \begin{tabular}{ccccccccc} \toprule

     & \footnotesize{(1)} & \footnotesize{(2)} & \footnotesize{(3)} & \footnotesize{(4)} & \footnotesize{(5)} & \footnotesize{(6)} & \footnotesize{(7)} & \footnotesize{(8)} \\  

     &  &  & \mc{3}{c}{\footnotesize{$P=0$}} & \mc{3}{c}{\footnotesize{$P=1$}} \\ 
    \cmidrule(lr){4-6} \cmidrule(lr){7-9} 

    \footnotesize{Variable} & \footnotesize{ITT} & \footnotesize{ITT$|X,W$} & \footnotesize{ITT} & \footnotesize{ITT$|X,W$} & \footnotesize{KE$|X,W$} & \footnotesize{ITT} & \footnotesize{ITT$|X,W$} & \footnotesize{KE$|X,W$} \\ 
    \midrule 

    \\[0.1cm]
    \mc{1}{l}{\footnotesize{\% Pos. TE}} & \mc{1}{c}{\footnotesize{79}} & \mc{1}{c}{\footnotesize{78}} & \mc{1}{c}{\footnotesize{81}} & \mc{1}{c}{\footnotesize{80}} & \mc{1}{c}{\footnotesize{81}} & \mc{1}{c}{\footnotesize{73}} & \mc{1}{c}{\footnotesize{72}} & \mc{1}{c}{\footnotesize{70}} \\  

    \mc{1}{l}{\footnotesize{$H_0$: $\le$ 25\%}} & \mc{1}{c}{\footnotesize{\textbf{(0.000)}}} & \mc{1}{c}{\footnotesize{\textbf{(0.000)}}} & \mc{1}{c}{\footnotesize{\textbf{(0.000)}}} & \mc{1}{c}{\footnotesize{\textbf{(0.000)}}} & \mc{1}{c}{\footnotesize{\textbf{(0.000)}}} & \mc{1}{c}{\footnotesize{\textbf{(0.000)}}} & \mc{1}{c}{\footnotesize{\textbf{(0.000)}}} & \mc{1}{c}{\footnotesize{\textbf{(0.000)}}} \\  

    \mc{1}{l}{\footnotesize{$H_0$: $\le$ 50\%}} & \mc{1}{c}{\footnotesize{\textbf{(0.000)}}} & \mc{1}{c}{\footnotesize{\textbf{(0.000)}}} & \mc{1}{c}{\footnotesize{\textbf{(0.000)}}} & \mc{1}{c}{\footnotesize{\textbf{(0.000)}}} & \mc{1}{c}{\footnotesize{\textbf{(0.000)}}} & \mc{1}{c}{\footnotesize{\textbf{(0.020)}}} & \mc{1}{c}{\footnotesize{\textbf{(0.000)}}} & \mc{1}{c}{\footnotesize{\textbf{(0.020)}}} \\  

    \mc{1}{l}{\footnotesize{$H_0$: $\le$ 75\%}} & \mc{1}{c}{\footnotesize{(0.333)}} & \mc{1}{c}{\footnotesize{(0.353)}} & \mc{1}{c}{\footnotesize{(0.216)}} & \mc{1}{c}{\footnotesize{(0.235)}} & \mc{1}{c}{\footnotesize{(0.294)}} & \mc{1}{c}{\footnotesize{(0.627)}} & \mc{1}{c}{\footnotesize{(0.667)}} & \mc{1}{c}{\footnotesize{(0.725)}} \\ 
    \midrule

    \\[0.1cm]
    \mc{1}{l}{\footnotesize{\% Pos. TE $|$ 10\% Significance}} & \mc{1}{c}{\footnotesize{37}} & \mc{1}{c}{\footnotesize{37}} & \mc{1}{c}{\footnotesize{49}} & \mc{1}{c}{\footnotesize{48}} & \mc{1}{c}{\footnotesize{45}} & \mc{1}{c}{\footnotesize{23}} & \mc{1}{c}{\footnotesize{27}} & \mc{1}{c}{\footnotesize{20}} \\  

    \mc{1}{l}{\footnotesize{$H_0$: $\le$ 25\%}} & \mc{1}{c}{\footnotesize{(0.118)}} & \mc{1}{c}{\footnotesize{\textbf{(0.078)}}} & \mc{1}{c}{\footnotesize{\textbf{(0.000)}}} & \mc{1}{c}{\footnotesize{\textbf{(0.000)}}} & \mc{1}{c}{\footnotesize{\textbf{(0.039)}}} & \mc{1}{c}{\footnotesize{(0.588)}} & \mc{1}{c}{\footnotesize{(0.373)}} & \mc{1}{c}{\footnotesize{(0.725)}} \\  

    \mc{1}{l}{\footnotesize{$H_0$: $\le$ 50\%}} & \mc{1}{c}{\footnotesize{(0.882)}} & \mc{1}{c}{\footnotesize{(0.922)}} & \mc{1}{c}{\footnotesize{(0.549)}} & \mc{1}{c}{\footnotesize{(0.588)}} & \mc{1}{c}{\footnotesize{(0.725)}} & \mc{1}{c}{\footnotesize{(1.000)}} & \mc{1}{c}{\footnotesize{(0.980)}} & \mc{1}{c}{\footnotesize{(1.000)}} \\  

    \mc{1}{l}{\footnotesize{$H_0$: $\le$ 75\%}} & \mc{1}{c}{\footnotesize{(1.000)}} & \mc{1}{c}{\footnotesize{(1.000)}} & \mc{1}{c}{\footnotesize{(1.000)}} & \mc{1}{c}{\footnotesize{(1.000)}} & \mc{1}{c}{\footnotesize{(1.000)}} & \mc{1}{c}{\footnotesize{(1.000)}} & \mc{1}{c}{\footnotesize{(1.000)}} & \mc{1}{c}{\footnotesize{(1.000)}} \\ 
    \toprule
  \end{tabular}
    \begin{tablenotes}
    \footnotesize
    \item 
Note: This table displays the percentage of the 95 outcomes for which we estimate positive
treatment effects. For outcomes where a negative treatment effect is beneficial to the subjects
(e.g. prevalence of diabetes), we reverse the signs of treatment effects so that all beneficial 
effects have positive signs.
Column (1) corresponds to the ITT, without accounting for any controls.
Column (2) corresponds to the ITT conditioning on vector of controls, $X$, consisting of Apgar scores 1 minute and 5 minutes after birth, the HRI index, maternal IQ,
an indicator for having a grandmother residing in the same county, and an index for the number
of relatives living in the same household. We also apply IPW weights, $W$, to account for attrition.
Columns (3)--(4) are analogous to columns (1)--(2), but we restrict the control sample to subjects
who did not enroll in any alternative care.
Column (5) corresponds to the matching estimate, where we use the Mahalanobis metric and Epanechnikov kernel
to match on controls $X$ listed above, and restrict the control sample to subjects who did not enroll
in any alternative care. Additionally, we apply IPW weights, $W$.
Columns (6)--(8) are analogous to Columns (3)--(5), except we restrict the control sample to subjects
who did enroll in alternative care. 
Numbers in parentheses represent the $p$-value from a single hypothesis test, and are obtained from 
the empirical bootstrap distribution generated by 200 resamples of the original data. 
Bold $p$-values indicate significance at the 10\% level. Blank point estimates indicate that
we are unable to obtain estimates due to a lack of support in the data. 

    \end{tablenotes}
  \end{threeparttable}
\end{sidewaystable}