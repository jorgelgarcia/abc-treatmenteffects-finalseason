\begin{table}[H]
\captionsetup{singlelinecheck=false,justification=centering}
\caption{CARE Average Treatment Effects, Males and Females \\ Vitamin D Deficiency \label{tab:ate_pooled_apx14}}

  \begin{threeparttable}
  \begin{tabular}{cccccccccc}
  \hline\hline

     &  & \scriptsize{(1)} & \scriptsize{(2)} & \scriptsize{(3)} & \scriptsize{(4)} & \scriptsize{(5)} & \scriptsize{(6)} & \scriptsize{(7)} & \scriptsize{(8)} \\  

     &  &  &  & \mc{3}{c}{\scriptsize{$P=0$}} & \mc{3}{c}{\scriptsize{$P=1$}} \\ 
    \cmidrule(lr){5-7} \cmidrule(lr){8-10} 

    \scriptsize{Variable} & \scriptsize{Age} & \scriptsize{ITT} & \scriptsize{ITT$|X,W$} & \scriptsize{ITT} & \scriptsize{ITT$|X,W$} & \scriptsize{KE$|X,W$} & \scriptsize{ITT} & \scriptsize{ITT$|X,W$} & \scriptsize{KE$|X,W$} \\ 
    \hline  

    \mc{1}{l}{\scriptsize{Vitamin D Deficiency}} & \mc{1}{c}{\scriptsize{Mid-30s}} & \mc{1}{c}{\scriptsize{-0.076}} & \mc{1}{c}{\scriptsize{0.116}} & \mc{1}{c}{\scriptsize{-0.167}} & \mc{1}{c}{\scriptsize{-0.127}} &  &  & \mc{1}{c}{\scriptsize{0.318}} &  \\  

     &  & \mc{1}{c}{\scriptsize{(0.353)}} & \mc{1}{c}{\scriptsize{(0.745)}} & \mc{1}{c}{\scriptsize{\textbf{(0.020)}}} & \mc{1}{c}{\scriptsize{(0.275)}} &  &  & \mc{1}{c}{\scriptsize{(0.863)}} &  \\ 
    \hline  

    \\[0.1cm]
    \mc{2}{l}{\scriptsize{\% of Pos. TE ($H_0$: $\le$ 25\% $|$ 10\% Significance)}} & \mc{1}{c}{\scriptsize{0}} & \mc{1}{c}{\scriptsize{0}} & \mc{1}{c}{\scriptsize{100}} & \mc{1}{c}{\scriptsize{0}} &  &  & \mc{1}{c}{\scriptsize{0}} &  \\  

     &  & \mc{1}{c}{\scriptsize{(0.275)}} & \mc{1}{c}{\scriptsize{(0.922)}} & \mc{1}{c}{\scriptsize{\textbf{(0.000)}}} & \mc{1}{c}{\scriptsize{(0.216)}} &  &  & \mc{1}{c}{\scriptsize{(0.902)}} &  \\  

    \mc{2}{l}{\scriptsize{\% of Pos. TE ($H_0$: $\le$ 50\% $|$ 10\% Significance)}} & \mc{1}{c}{\scriptsize{0}} & \mc{1}{c}{\scriptsize{0}} & \mc{1}{c}{\scriptsize{100}} & \mc{1}{c}{\scriptsize{0}} &  &  & \mc{1}{c}{\scriptsize{0}} &  \\  

     &  & \mc{1}{c}{\scriptsize{(0.902)}} & \mc{1}{c}{\scriptsize{(0.922)}} & \mc{1}{c}{\scriptsize{\textbf{(0.000)}}} & \mc{1}{c}{\scriptsize{(0.667)}} &  &  & \mc{1}{c}{\scriptsize{(0.902)}} &  \\  

    \mc{2}{l}{\scriptsize{\% of Pos. TE ($H_0$: $\le$ 75\% $|$ 10\% Significance)}} & \mc{1}{c}{\scriptsize{0}} & \mc{1}{c}{\scriptsize{0}} & \mc{1}{c}{\scriptsize{100}} & \mc{1}{c}{\scriptsize{0}} &  &  & \mc{1}{c}{\scriptsize{0}} &  \\  

     &  & \mc{1}{c}{\scriptsize{(0.902)}} & \mc{1}{c}{\scriptsize{(0.922)}} & \mc{1}{c}{\scriptsize{\textbf{(0.000)}}} & \mc{1}{c}{\scriptsize{(0.667)}} &  &  & \mc{1}{c}{\scriptsize{(0.902)}} &  \\  

  \hline\hline
  \end{tabular}
    \begin{tablenotes}
    \scriptsize
    \item 
Note: This table displays various estimates of the treatment effect of CARE's family education program.
Column (1) displays the ITT, without accounting for any controls.
Column (2) displays the ITT conditioning on vector of controls, $X$, consisting of Apgar scores 1 minute and 5 minutes after birth, an indicator for the subject 
being born prematurely, an indicator for the mother being married at baseline, an indicator for
teenage pregnancy of the mother, and an indicator for being born in the fall. We also apply IPW weights, $W$, to account for attrition.
Columns (3)--(4) are analogous to columns (1)--(2), but we restrict the control sample to subjects
who did not enroll in any alternative care.
Column (5) displys the matching estimate, where we use the Mahalanobis metric and Epanechnikov kernel
to match on controls $X$ listed above, and restrict the control sample to subjects who did not enroll
in any alternative care. Additionally, we apply IPW weights, $W$.
Columns (6)--(8) are analogous to Columns (3)--(5), except we restrict the control sample to subejcts
who did enroll in alternative care. The final three pairs of rows display the proportion of treatment effects in the table that are 
socially positive. The first row in each pair displays the percentage of treatment effects, and the
second row presents the inference. 
Numbers in parentheses represent the $p$-value from a single hypothesis test, and are obtained from 
the empirical bootstrap distribution generated by 200 resamples of the original data. 
Bold $p$-values indicate significance at the 10\% level.
Blank point estimates indicate that we are unable to obtain estimates due to a lack of support in the data. 

    \end{tablenotes}
  \end{threeparttable}

\end{table}