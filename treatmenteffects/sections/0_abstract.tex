\noindent This paper studies the treatment effects of the Carolina Abecedarian Project and the Carolina Approach to Responsive Education (ABC/CARE). ABC/CARE was a randomized control trial with treatment consisting of high-quality center-based preschool shortly after birth until age 5. We present treatment effects by gender to clearly present the relationship between gender differences and early childhood education's effect on those differences. In addition to conventional treatment effects, we account for control substitution by considering two counterfactual scenarios: the control-group subjects either stayed at home or attended alternative preschool. We find that males benefit more from treatment in comparison to those who stay at home while females are more resilient to their early environment. We present a complementary aggregate estimate to our companion paper, \citet{Garcia_Heckman_Leaf_etal_2017_Comp_CBA_Unpublished}, that estimates the proportion of total outcomes that are positive (and significant). Although both genders have large proportions, females outperform males. Finally, we reconcile this analysis by concluding that the treatment compensated for early-life deficits that males experience relative to females. This compensation helps the males experience long-lasting benefits of the program into adulthood.