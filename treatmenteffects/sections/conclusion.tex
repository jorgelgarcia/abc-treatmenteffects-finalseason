When considering the early and school-age skills mediating educational attainment, we find that non-cognitive skills are significant mediators for both males and females. Given the importance of educational attainment in mediating other adult outcomes, the fact that non-cognitive skills are mediators for it highlights the dynamic importance of the non-cognitive skills. 
The proportion of the treatment effect explained by the skills do not differ by gender. Instead, the treatment effects for years of education are much larger for females than for males. 

Gender differences also appear when considering educational attainment as a mediator for labor income. At age 30, years of education is a negative mediator of labor income for females. This indicates that more educated women are staying at home, i.e., although treatment may increase years of education, that in turn does not necessarily lead to a full-time career by age 30. However, when considering the lifetime income, we find that the educational attainment has the opposite effect. In this case, educational attainment for females positively mediates labor income. It is striking to compare these results to those of the males, in which educational attainment is not a significant mediator income for neither the age-30 nor the lifetime income. Although females may exit the labor force during their child-bearing years, they re-enter and recover income through their increased educational attainment. 

Finally, we consider crime savings to explore another way in which educational attainment can mediate the adult treatment effects. For females, years of schooling mediate a large proportion of the crime savings, as is consistent with the findings in \citet{Heckman_Pinto_etal_2013_PerryFactor} for subjects in the Perry Preschool Project (Perry). There are two justifications for this similarity. One reason is that the females in ABC/CARE commit similar crimes to the Perry subjects. Relative to the ABC/CARE males, ABC/CARE female subjects commit more non-violent crimes than violent ones. Another reason is that the males begin committing crimes earlier, in turn precluding further educational attainment.

These results reveal the importance of non-cognitive skills, not only on later mediators like educational attainment, but also on adult outcomes of interest. Although there are not gender differences on the importance of these skills on educational attainment, there are gender differences on educational attainment's mediation of labor income and crime savings.  
