\subsection{Possible Explanations for Gender Differences}

The benefit/cost ratio and internal rate of return calculations both indicate that males and females benefit \emph{differently} from the program compared to the alternatives ``$H$'' and ``$C$''. There are two complementary stories that help to explain this difference. First, gender differences could exist as a consequence of the outcomes monetized, and not because of the particular counterfactuals that we estimate. Males have relatively high benefits from the outcomes that we are able to monetize. Labor income and crime are prime examples of this. Females are less likely to work than males. While all males supply labor in our sample at age 30, not all females do. We are not able to quantify household production benefits for either males or females. This is an important omission for females who decide to stay at home instead of work. For example, we lack data on their children's outcomes.

ABC/CARE has treatment effects on crime for females for a number of categories (see Appendix~\ref{appendix:results}). However, males are much more likely to commit crimes that are more costly to the victims, to the criminal justice system, and to society \citep{Cohen-Bowles_2010_Estimating-Cost-Crime,Gregg_etal_2015_SocialRealities_BOOK}. ABC/CARE also has treatment effects on crime for females for a number of categories (see Appendix~\ref{appendix:results}). However, males commit crimes that are much more expensive to society. These two categories are examples of why the magnitudes of the gains are much higher for males than they are for females.

For health, there are also substantial gender differences. Both males and females have substantial gains: males benefit on more standard measures of physical health, and females benefit on a set of mental health measures (see Appendix~\ref{appendix:results}). We quantify both components (see Section~\ref{section:health} and Appendix~\ref{appendix:health}).

There is a second factor at work. There are substantial differences between males and females in one counterfactual: treatment vs. alternative preschools. The estimated treatment effects are very similar across genders for treatment compared to those staying at home full time. Males benefit much more from treatment relative to alternative preschools compared to their benefits from treatment relative to staying at home. This result is consistent with findings noted elsewhere: (i) stark gender differences resulting from attending low quality childcare \citep{Kottelenberg-Lehrer_2014_Gender-Effects,Baker_Gruber_Milligan_2015_Noncog_Defects}; and (ii) females are less sensitive to uncertain environments (see, e.g., \citealp{Autor-etal_2015_Family-Disadvantage}).

Our evidence does not indicate that the program has no benefits for females. When compared to staying at home, there is a gain of $4.93$ dollars per each dollar invested. When we decompose the net-present value for each of the components that we monetize, we find substantial benefits for females across a variety of categories, including health and crime. For males, the magnitudes are noticeably increased when comparing outcomes from treatment to outcomes from attending alternative preschools (see Figure~\ref{fig:npvsgender}).