We set up a common methodology to be used for both programs. The specific skill measurements, outcome variables, and background characteristics vary by program depending on the available data. We estimate all parameters separately for males and females, allowing us to compare the technologies of the two groups.\footnote{The estimation of separate technologies by gender is also done in \citet{Heckman_Pinto_etal_2013_PerryFactor}.} 
%\footnote{See Table~\ref{tab:outcomes-skills-factors} for an overview of the variables that we use for each program.} 

We begin by defining the outcome equation, suppressing subscripts by individual, gender, and program for expositional simplicity. Let $D$ indicate treatment status. We can fix individuals to the treatment ($D=1$) or control ($D=0$) group. That is, for some outcome, $Y_1$ is realized when fixed to the treatment group and $Y_0$ when fixed to the control group. Using a Roy model,\footnote{See \citet{Roy_1951_OEP} and \citet{Heckman_Honore_1990_Econometrica}.} we write the counterfactual outcomes

\begin{equation}
\label{eq:potential-outcome}
	Y = DY_1 + (1-D)Y_0.
\end{equation}

\textbf{[After IPW is implemented, a brief description will go here.]} 

We can test the extent to which the treatment affects $Y$ by estimating the average treatment effect. We condition on the vector $\bm{X}$ indexed by $\mathcal{X}$ and comprised of variables not influenced by the treatment. For each $d \in \{0,1\}$, we define the relationship between the outcome and treatment:

\begin{equation}
\label{eq:y-no-mediator}
	Y_d = \tilde{\delta}_d + \sum_{j = 1}^{|\mathcal{X}|} \tilde{\beta}_d^j X^j  + \tilde{\varepsilon}_d, 
\end{equation}

\noindent where $\tilde{\varepsilon}_d$ is assumed to be mean-zero and independent of $Y_d$ conditional on $\bm{X}$. 

It is possible that there are variables that mediate the treatment's effect on $Y$. Without including these variables, the estimates of the treatment effect will be biased. Let $\bm{W}$ be a vector of mediators indexed by the set $\mathcal{W}$. Unlike the pre-program characteristics that comprise $\bm{X}$, the elements of $\bm{W}$ are affected by treatment and can be written in the counterfactual framework of Equation~\eqref{eq:potential-outcome}. The outcome equation, \eqref{eq:y-no-mediator}, can be rewritten to include these mediators:

\begin{equation}
\label{eq:y-mediator}
	Y_d = \delta_d +  \sum_{k=1}^{|\mathcal{W}|} \alpha_d^k W_d^k + \sum_{j = 1}^{|\mathcal{X}|} \beta_d^j X^j  + \varepsilon_d,
\end{equation}

\noindent where $\varepsilon_d$ is assumed to be mean-zero and independent of $Y_d$ conditional on $\bm{X}$ and $\bm{W}$. 

%We test that $\varepsilon_0$ is equal to $\varepsilon_1$ and continue with a common $\varepsilon$.

Although the data are plentiful, it is possible that there are unmeasured mediators that are both affected by $D$ and affect $Y$. Following \citet{Heckman_Pinto_etal_2013_PerryFactor} we partition $\mathcal{W}$ into measured skills indexed by $\mathcal{W}^M$ and unmeasured skills indexed by $\mathcal{W}^U$. 

\begin{align}
\label{eq:y-unmeasured}
	Y_d &= \delta_d + \sum_{k=1}^{|\mathcal{W}^M|} \alpha_d^k W_d^k + \sum_{k=1}^{|\mathcal{W}^U|} \alpha_d^k W_d^k  + \sum_{j = 1}^{|\mathcal{X}|} \beta_d^j X^j  + \varepsilon_d  \nonumber \\
	 &=  \delta_d  + \sum_{k=1}^{|\mathcal{W}^M|} \alpha_d^k W_d^k  + \sum_{k=1}^{|\mathcal{W}^U|} \alpha_d^k \Big( W_d^k + \mathbb{E}[W_d^k] - \mathbb{E}[W_d^k] \Big) + \sum_{j = 1}^{|\mathcal{X}|} \beta_d^j X^j  + \varepsilon_d  \nonumber \\
	 &= \underbrace{\bigg[ \delta_d + \sum_{k=1}^{|\mathcal{W}^U|} \alpha_d^k \mathbb{E}[W_d^k] \bigg]}_{\kappa_d} + \sum_{k=1}^{|\mathcal{W}^M|} \alpha_d^k W_d^k  + \sum_{j = 1}^{|\mathcal{X}|} \beta_d^j X^j  + \underbrace{\bigg[ \varepsilon_d + \sum_{k=1}^{|\mathcal{W}^U|} \alpha_d^k \Big( W_d^k - \mathbb{E}[W_d^k] \Big) \bigg]}_{\upsilon_d} \nonumber \\
	 &= \kappa_d + \sum_{k=1}^{|\mathcal{W}^M|} \alpha_d^k W_d^k  + \sum_{j = 1}^{|\mathcal{X}|} \beta_d^j X^j + \upsilon_d.
\end{align}

%\begin{enumerate}
%\item State updated ATE
%\item Apply tests that increments in unobservables are independent of increments of observables
%\item Write production function of later outputs 

After testing for conditions laid out in \citet{Heckman_Pinto_etal_2013_PerryFactor}, Equation~\eqref{eq:y-unmeasured} can be rewritten as

\begin{equation}
\label{eq:y-simplified}
Y_d = \kappa_d + \sum_{k=1}^{|\mathcal{W}^M|} \alpha^k W_d^k  + \sum_{j = 1}^{|\mathcal{X}|} \beta^j X^j + \upsilon_d
\end{equation}

\noindent and Equation~\eqref{eq:potential-outcome} can be rewritten as

\begin{align}
\label{eq:y-simplified}
Y &= \big[ D\kappa_1 + (1-D)\kappa_0 \big] + \sum_{k=1}^{|\mathcal{W}^M|} \alpha^k \underbrace{\big [ DW_1^k + (1-D)W_0^k \big]}_{W^k}  + \sum_{j = 1}^{|\mathcal{X}|} \beta^j X^j + \underbrace{\big[ D\upsilon_1 + (1-D)\upsilon_0 \big]}_{\upsilon} \nonumber \\
&= \kappa_0 + D(\kappa_1 - \kappa_0) + \sum_{k=1}^{|\mathcal{W}^M|} \alpha^k W^k + \sum_{j = 1}^{|\mathcal{X}|} \beta^j X^j + \upsilon.
\end{align}

In addition to partitioning the mediators into measured and unmeasured mediators, we also partition them by age following \citet{Conti_etal_2016_LongTermHealth}. We extend this work by including non-health outcomes and introducing more stages into the dynamic framework.

We partition $\bm{W} = [ \bm{W_E}, \bm{W_M}, \bm{W_L} ]$ into early (before age 5, during treatment), medium (between the ages of 5.5 and 18), and late (after 18) mediators. We set up static and dynamic mediation with these sets of mediators. Using mediators from these age categories allows us to analyze their effects statically and dynamically on medium and late outcomes. Following the steps detailed above, we define the relationship between early, medium, and late mediators:

\begin{align}
W_M &= \gamma_{M,0} + D(\gamma_{M,1} - \gamma_{M,0}) + \sum_{k=1}^{|\mathcal{W}^M_E|} \lambda_M^k W^k + \sum_{j = 1}^{|\mathcal{X}|} \phi_M^j X^j + \epsilon_M \label{eq:med-mediator}\\
W_L &= \gamma_{L,0} + D(\gamma_{L,1} - \gamma_{L,0}) + \sum_{k=1}^{|\mathcal{W}^M_E|} \lambda_L^k W^k + \sum_{k=1}^{|\mathcal{W}^M_M|} \psi_L^k W^k + \sum_{j = 1}^{|\mathcal{X}|} \phi_L^j X^j + \epsilon_L \label{eq:late-mediator}.
\end{align}

The outcome equation is then

\begin{equation}
\label{y-age-mediators}
Y = \kappa_0 + D(\kappa_1 - \kappa_0) + \sum_{k=1}^{|\mathcal{W}^M_E|} \alpha^k W^k + \sum_{k=1}^{|\mathcal{W}^M_M|} \alpha^k W^k+ \sum_{k=1}^{|\mathcal{W}^M_L|} \alpha^k W^k + \sum_{j = 1}^{|\mathcal{X}|} \beta^j X^j + \upsilon.
\end{equation}

\textbf{[The dynamic equation will go here.]}

We begin by testing the mediators through static mediation described in Equations~\ref{eq:late-mediator} and~\ref{eq:y-simplified}.
