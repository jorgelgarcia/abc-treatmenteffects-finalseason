To find a set of early, school-age, and adult mediators, we start by focusing on two life-cycle outcomes: labor income and savings in crime costs, both of which are in net present value of 2014 dollars discounted to birth of the subjects assuming a 3\% discount rate.\footnote{These outcomes are estimated following the methods described in \citet{Garcia_etal_2016_Comp_CBA_Unpublished}. These outcomes allow us to analyze the mediation beyond the ages observed at data collection. The inference for the following analysis accounts for the construction of the net present value of income, which uses an auxiliary sample to predict unobserved earnings. We bootstrap over the auxiliary sample in addition to bootstrapping over the ABC/CARE sample. The auxiliary data for crime is the complete population of individuals who committed crimes in North Carolina.}

We first estimate the effect of the early mediators, $\bm{\theta}$, on the later mediators, $M$. In this case, $\bm{\theta}$ is a vector containing cognitive and non-cognitive skills and $M$ is years of education.

\begin{equation}
	M = \alpha_0 +\bm{ \alpha} \bm{\theta} + \varepsilon.
\end{equation}

Then, we estimate the effect of the early mediators on the outcome of interest, $Y$:

\begin{equation}
	Y = \mu_0 + \bm{\mu} \bm{\theta} + \epsilon.
\end{equation}

Finally, we estimate the effect of the later mediators on the outcome of interest:

\begin{equation}
	Y = \gamma_0 + \gamma M + \nu. 
\end{equation}

All of the parameters in these equations are estimated in the full sample of ABC/CARE. That is, we impose that the technology is the same across genders. We then split the sample by gender to calculate the treatment effect and decompose it based on the inputs of the above equations. To get the proportions reported below, we multiply this treatment effect by the estimates of the corresponding parameters from the above equations, and divide by the total treatment effect. This is a standard Laspeyres decomposition.

We investigate the effect of early (before 5 years) and school-age (5 to 18 years) skills on adult educational attainment. Figure~\ref{fig:task-years} displays the proportion of the treatment effect explained by IQ (cognitive) and task orientation (non-cognitive) skills. Both skills significantly mediate females' educational attainment, but only non-cognitive skills significantly mediate that of males. Further, the magnitudes of the proportions are similar indicating that non-cognitive skills mediate educational attainment equally for males and females in the sample. However, as we will show below, it is the effect of these skills on later mediators that determine the later-life difference in outcomes.

\begin{figure}[H]
\begin{center}
\caption{Early and School-Age Skills Mediating Years of Education at 30}
\label{fig:task-years}
	\includegraphics[width=\textwidth]{../output/years_30y-taskCBIIBR-noc}
\end{center}
\raggedright
Note: This plot shows the proportion of the treatment effect on years of education explained by the mediators. The cognitive factor combines verbal IQ at 5 and 8 years. The non-cognitive factor combines measures of task orientation from the Infant Behavior Record at 6 months, 1 year, and 1.5 years and the Classroom Behavior Inventory at 6 and 8 years. Years of education is measured at 30 years. The technology is imposed to be the same between males and females. We calculate the $p$-values using 100 bootstraps. The treatment effect, by gender, on years of education at 30 years and the one-sided $p$-value are reported under each bar.
\end{figure}

Once we establish this relationship between earlier mediators and years of education, we can consider it as an adult mediator. We test this by reporting the proportion that years of education mediates labor income at age 30. As reported in Figure~\ref{fig:years-income30}, we find that years of education significantly and negatively mediates labor income for females. This indicates that although the treatment effect on years of education is higher for females than for males, this increased educational attainment has a negative effect on labor income at age 30. 

\begin{figure}[H]
\begin{center}
\caption{Years of Education Mediating Labor Income at 30}
\label{fig:years-income30}
	\includegraphics[width=\textwidth]{../output/si30y_inc_labor-years-male-noc}
\end{center}
\raggedright
Note: This plot shows the proportion of the treatment effect on labor income at age 30 explained by the mediators. Both years of education and labor income are measured at 30 years. Labor income is in 2014 US dollars. The technology is imposed to be the same between males and females. We calculate the $p$-values using 100 bootstraps. The treatment effect, by gender, on labor income at 30 years and the one-sided $p$-value are reported under each bar.
\end{figure}

Labor income measured at one age can be a non-representative and misleading measure, especially considering that it might capture a point in time when females exit the workforce to rear children. We use the projected lifetime labor income calculated in \citet{Garcia_etal_2016_Comp_CBA_Unpublished} to set up the same mediation as in Figure~\ref{fig:years-income30}. We find that the effect of educational reverses for females with years of education significantly and positively mediating the effect. This measure of income includes projected income received later in adulthood, perhaps after the deficit from exiting the work force to raise a family. The proportion is similar for males across income variables. This further supports the hypothesis of women staying at home during their child-bearing years even though they re-enter the workforce afterwards.

\begin{figure}[H]
\begin{center}
\caption{Years of Education Mediating Lifetime Labor Income}
\label{fig:years-incomenpv}
	\includegraphics[width=\textwidth]{../output/income-years-male-noc}
\end{center}
\raggedright
Note: This plot shows the proportion of the treatment effect on projected lifetime labor income explained by the mediators. Lifetime labor income is the net present value of the lifetime labor income projected using the method described in \citet{Garcia_etal_2016_Comp_CBA_Unpublished}. Years of education is measured at age 30. The technology is imposed to be the same between males and females. We calculate the $p$-values using 100 bootstraps. The treatment effect, by gender, on lifetime labor income and the one-sided $p$-value are reported under each bar.
\end{figure}


We also consider the mediation of early and school-age skills on the net present value of labor income (Figure~\ref{fig:skills-incomenpv}). Both cognitive and non-cognitive skills significantly and positively mediate income for females, with non-cognitive skills mediating a larger proportion that cognitive skills. The direct effect of skills on lifetime labor income are also strong in comparison to males, for which neither are significant. Dynamic analysis will help further understand the direct and indirect effect of skills. 

\begin{figure}[H]
\begin{center}
\caption{Early and School-Age Skills Mediating Lifetime Labor Income}
\label{fig:skills-incomenpv}
	\includegraphics[width=\textwidth]{../output/income-skills-male-noc}
\end{center}
\raggedright
Note: This plot shows the proportion of the treatment effect on projected lifetime labor income explained by skills. Lifetime labor income is the net present value of the lifetime labor income projected using the method described in \citet{Garcia_etal_2016_Comp_CBA_Unpublished}. Years of education is measured at age 30. The technology is imposed to be the same between males and females. We calculate the $p$-values using 100 bootstraps. The treatment effect, by gender, on years of education at 30 and the one-sided $p$-value are reported under each bar.
\end{figure}

Finally, we consider another lifetime projected outcome: savings due to decreased crime. We find that educational attainment does not mediate a large proportion of crime savings for males, while it non-significantly mediates more of the effect for females. This is a result of females tending to commit less violent crimes than males as well as females committing crimes later than males.

\begin{figure}[H]
\begin{center}
\caption{Years of Education Mediating Lifetime Crime Savings}
\label{fig:years-crime}
	\includegraphics[width=\textwidth]{../output/crime-years-male}
\end{center}
\raggedright
Note: This plot shows the proportion of the treatment effect on projected crime savings explained by the mediators. Crime savings are calculated and projected to age 50 using the methods described in \citet{Garcia_etal_2016_Comp_CBA_Unpublished}. Years of education is measured at age 30. The technology is imposed to be the same between males and females. We calculate the $p$-values using 100 bootstraps. This specification controls for mother's years of education. The treatment effect, by gender, on lifetime crime savings and the one-sided $p$-value are reported under each bar.
\end{figure}

