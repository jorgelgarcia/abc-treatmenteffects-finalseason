When considering gender differences in the modern economy, the wage gap between men and women is an oft-cited number. In the fourth quarter of 2016, white women earned 81.1\% as much as white men and black women earned 92.1\% as much as black men. This reveals a gap not only by gender, but also by race: Black (Hispanic) women only earn 67.8\% (62.0\%) as white men and black (Hispanic) men earn 73.6\% (71.4\%) as much as white men.\footnote{\citet{USDPTL_2017_Wage_News-Release}.} 

Many studies have shown the life-cycle benefits of early education for children from disadvantaged families.\footnote{\citet{Elango_Hojman_etal_2016_Early-Edu}.} Several of these studies separate analysis by gender and find that males and females benefit differently from early childhood education. For example, \citet{Heckman_Moon_etal_2010_QE} and \citet{Garcia_etal_2016_Comp_CBA_Unpublished}, both of which analyze randomized controlled trials with long-term data follow-ups, find that females tend to have more positive effects in education outcomes while males tend to have more positive effects in labor market and health outcomes. Other studies analyzing programs with shorter-term data also find gender differences in early skills and academic outcomes.\footnote{\citet{Deming_2009_AEJAE} and \citet{Ou_Reynolds_2010_Mechanisms_CYSR} are examples of this.} 

These gender results are variable across studies, however, obfuscating the conclusions.\footnote{\citet{Magnuson_Kelchen_Duncan_etal_2016_ECRQ} use effect sizes and program characteristics from 23 evaluations of early-life interventions to understand the association between program characteristics and effect sizes by gender. This meta-analysis finds that over the programs used, the most pronounced difference in treatment effects between males and females can be found for outcomes related to schooling, e.g. special education and grade retention. However, most of the programs do not have non-cognitive measures and the varied structure of the evaluations makes the conclusions for gender differences suggestive.} Even within the same program, different approaches to analyzing treatment effects can result in seemingly contradictory conclusions. Using data from the Carolina Abecedarian Project and the Carolina Approach to Responsive Education (ABC/CARE), \citet{Garcia_etal_2016_Comp_CBA_Unpublished} calculate a higher lifetime benefit-cost ratio for males (11.10) than for females (2.45), with these ratios including life-cycle projections of health, crime, and income. According to this analysis, the monetary returns of the program from a social perspective are driven more by males than females, making it appear that males benefit more from the program than do females. However, when looking at the treatment effects unweighted by monetary amounts, there are more positive treatment effects for females for certain categories (see Section~\ref{sec:gdiff}). Unlike the cost-benefit analysis, this aggregate result makes it appear that females benefit more from the program in skills, labor market outcomes, and crime outcomes. Although both of these measures aggregate across outcomes, they lead to seemingly contradictory conclusions on the differential effect of early childhood education.

Understanding the mechanisms of the treatment effects complements the above analyses by explaining and understanding the later-life gender differences seen in both approaches. We address the contradiction by focusing on how early childhood education affects the skill formation process of males and females differently, with these skills in turn affecting other outcomes.\footnote{There is also evidence that the development of males and females differs at this age such that they experience early childhood interventions differently. See \citet{Beeghly-etal_2017_IMHJ,Dayton_2017_IMHJ,Iruka_2017_IMHJ,Schore_2017_IMHJ} for recent findings on the topic of different development of males and females early in life. We consider these findings complementary to our own.}

After describing the data in Section~\ref{sec:data}, we clarify the gender differences in Section~\ref{sec:gdiff}. In that section, we emphasize that gender differences do not strongly appear when aggregating the outcomes. Rather, they appear when considering the outcomes grouped by broad category, such as education or crime outcomes. 
%We then set up a method for understanding the mechanisms through which these later-life gender differences form in Section~\ref{sec:methodology}. 
Once establishing these general gender differences, we explore in depth the gender differences seen in measures of parenting quality in order to explain an important interaction between the inputs of education and the family environment (Section~\ref{sec:parenting}).
We present the results of static mediation analyses that use early, school-age, and adult mediators in Section~\ref{sec:results}. Section~\ref{sec:conclusion} concludes.