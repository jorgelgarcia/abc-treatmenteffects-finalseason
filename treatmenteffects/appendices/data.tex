
\subsection{Description of Measurement Instruments}

\begin{ThreePartTable}
 \begin{TableNotes}
    \footnotesize
    \item Non-cognitive measures were coded such that a higher score is more desirable. The BSID (IBR version) is an exception, where neither extreme of the scale is desirable. Higher scores in parenting subdomains represent a higher level of the specified field.
    \item[1] Data for general reading, math, and achievement scores at 12 uears were the averages of Woodcock Johnson and California Achievement Test scores at 12y. The instruments have been separated as seen above. The score for knowledge at 12 years is the Woodcock Johnson score.
    \item[2] The total HOME score scales are divided into three age groups: 1) 0.5y, 1.5y, 2.5y 2) 3.5y, 4,5y 3) 8y.  
  \end{TableNotes}  
  \begin{longtable}{c c c c} 
  \caption{Standardized Scales Measuring Child and Parent Outcomes} \label{test} \\
      \toprule
        \textbf{Domain} & \textbf{Instrument} & \textbf{Subdomains} & \textbf{Scale} \\
      \midrule
      \endhead 
      %\multicolumn{4}{r}{{Continued on next page}} \\
      \midrule
      \endfoot
      \bottomrule
      \insertTableNotes         
      \endlastfoot
        & Stanford-Binet Intelligence Scale & IQ & 40 to 160 \\
        & \citet{Terman_1916_BOOKmeasurementofintelligence} & & \\
        & & & \\
        \textbf{Cognitive} & Woodcock-Johnson & total & N/A \\
        & \citet{Woodcock_1977_PsychEd_TechReport} & reading & \\
        & & math & \\
        & & knowledge & \\
        & & & \\
        & California Achievement Test & total & N/A \\
        & \citet{Tiegs_Clark_1963_BOOKCaliforniaAchievementTests} & reading & \\
        & & math & \\
        & & & \\
        & General Achievement$^1$  & total & N/A \\
        & & reading & \\
        & & math & \\
        & & knowledge & \\
      \midrule
        & Bayley Scales of Infant & task orientation & 4 to 36 \\         
        &  Development (IBR version) & activity level & 3 to 23  \\
        & \citet{Bayley_1969_BSID-Manual} & sociability & 2 to 14  \\
        & & & \\
        & Classroom Behavior Inventory & extrovert & 5 to 25\\
        & \citet{SRI_1972_Schaefer-Behav-Inv} & creative & \\
        & & considerate & \\
        & & task oriented  & \\
        & & verbal intelligence &\\         
        & & distractible & 3 to 15 \\
        & & dependent & \\
        & & hostility & \\
        & & introvert &  \\
        & & & \\
        \textbf{Socioemotional} & Walker Problem Behavior & acting out & 0 to 26  \\
        & Identificaion Checklist & bad peer relations & 0 to 25\\
        & \citet{Walker_1970_Walker-Problem-Behavior} & immature & 0 to 19 \\
        & & distractible & 0 to 14  \\
        & & withdrawn & \\
        & & & \\
        & EASI Temperament Survey & general emotionality & 5 to 25 \\
        & \citet{Buss-Plomin_1975_BOOK_Temperament-Theory} & fear & \\
        & & anger & \\
        & & tempo &\\
        & & vigor &\\
        & & sociability &\\
        & & control & \\
        & & decisive & \\
        & & sensation & \\
        & & persistence & \\
        & & & \\
        & Kohn and Rosman & confident/friendly & 1 to 6 \\
        & \citet{Kohn.Rosman_1972_DP}& anxious/withdrawn & \\
        & & attention/cooperative & \\
        & & distractible/disruptive &\\
      \midrule
        & Observation for Measurement & total$^2$ & 0 to 43 \\ 
        & of Environment (HOME) & & 0 to 89 \\
        & \citet{Bradley-Caldwell_1977_AJMD} & & 0 to 85 \\
        & & maternal warmth & 0 to 10\\
        & & maternal involvement & 0 to 6\\
        & & physical environment & 0 to 16 \\
        & & developmental stimulation & 0 to 15\\
        \textbf{Parenting}& & fostering independence & \\
        & & breadth of experience & 0 to 10 \\
        & & language environment & \\
        & & emotional climate & 0 to 6 \\
        & & organization of environment & \\
        & & play materials & 0 to 5 \\
        & & need gratify of restriction & 0 to 3 \\
        & & & \\
        & Parenting Attitudes Research Inventory & authoritarian & 25 to 100 \\
        & \citet{Schaefer-Bell_1958_CD}& hostility & 15 to 60 \\
        & & democratic & \\
     \bottomrule
  \end{longtable}
\end{ThreePartTable}


\begin{ThreePartTable}
 \begin{TableNotes}
        \footnotesize
        \item[1] One entry in each fear and control with reverse coded score of 26, origin undetermined.
     \end{TableNotes}
  \begin{longtable}{c c c c c c}
    \caption{Descriptive Statistics of Subdomains} \\
       \toprule
      \textbf{Instrument} & \textbf{Subdomains} & \textbf{Age} & \textbf{Data} & \textbf{Mean} & \textbf{Standard}\\
        & & \textbf{(Years)} & \textbf{Range} & & \textbf{Deviation} \\
      \midrule
      \endhead 
      %\multicolumn{6}{r}{{Continued on next page}} \\
      \midrule
      \endfoot
      \bottomrule
      \insertTableNotes         
      \endlastfoot
        Stanford-Binet & IQ & 2 & 66 to 125 & 91.4 & 11.9 \\
        & & 3 & 53 to 127 & 93.5 & 15.3 \\
        & & 3.5 & 62 to 126 & 98.2 & 12.4 \\
        & & 4 & 64 to 143 & 95.2 & 13.4 \\
        & & 4.5 & 49 to 119 & 96.3 & 12.8 \\
        & & 5 & 68 to 132 & 97.5 & 12.7 \\
        & & 12 & 64 to 130 & 93.5 & 11.4 \\
        & & 15 & 56 to 129 & 93 & 12.1 \\
      \midrule
        Woodcock-Johnson & total & 12 & 54 to 121 & 91.4 & 12.2 \\
        & reading & 12 & 56 to 123 & 89.5 & 12.5 \\
        & math & 12 & 53 to 121 & 90.5 & 13.3 \\
        & knowledge & 12 & 49 to 127 & 91.5 & 13 \\
      \midrule
        California Achievement Test & total & 12 & 70 to 116 & 93.2 & 11.2 \\
        & reading & 12 & 65 to 125 & 92.9 & 12 \\
        & math & 12 & 65 to 118 & 92 & 12.4 \\
      \midrule
        General Achievement & total & 12 & 79 to 117 & 95.3 & 10.1 \\
        & reading & 12 & 56 to 123 & 91.3 & 12 \\
        & math & 12 & 59 to 121 & 91.6 & 12.3 \\
        & knowledge & 12 & 49 to 127 & 91.5 & 13 \\
      \midrule
        BSID (IBR Version) & task orientation & 0.5 & 8 to 33 & 19.6 & 4.3 \\
        & & 1 & 11 to 30 & 21.4 & 3.5 \\
        & & 1.5 & 10 to 31 & 22 & 4.1 \\
        & & 2 & 13 to 35 & 22.4 & 3.7 \\
        & activity level & 0.5 & 5 to 20 & 12 & 3.2 \\
        & & 1 & 5 to 21 & 13.4 & 3.5 \\
        & & 1.5 & 6 to 21 & 14.9 & 3.6 \\
        & & 2 & 4 to 23 & 14.7 & 4.1 \\
        & sociability & 0.5 & 5 to 14 & 10.3 & 1.4 \\
        & & 1 & 7 to 14 & 11 & 1.4 \\
        & & 1.5 & 6 to 14 & 11.2 & 1.7 \\
        & & 2 & 5 to 14 & 10.6 & 1.6 \\
      \midrule
        CBI & extrovert & 5.5 & 7 to 24 & 18.5 & 3.5 \\
        & & 12 & 8 to 25 & 17.7 & 4.2 \\
        & creative & 5.5 & 7 to 24 & 15.5 & 3.7 \\
        & & 12 & 5 to 24 & 13 & 3.8 \\
        & considerate & 5.5 & 8 to 25 & 17.2 & 4.1 \\
        & & 12 & 5 to 25 & 16.3 & 5 \\
        & task oriented & 5.5 & 5 to 24 & 15.6 & 4.6 \\
        & & 12 & 5 to 25 & 12.8 & 5.2 \\
        & verbal intelligance & 5.5 & 5 to 24 & 14 & 3.9 \\
        & & 12 & 5 to 21 & 11.7 & 3.7 \\
        & distractible & 5.5 & 3 to 15 & 9.1 & 2.5 \\
        & & 12 & 3 to 14 & 8.6 & 3 \\
        & dependent & 5.5 & 4 to 15 & 11.2 & 2.4 \\
        & & 12 & 3 to 15 & 11.1 & 2.7 \\
        & hostility & 5.5 & 3 to 15 & 10.5 & 3.1 \\
        & & 12 & 3 to 15 & 9.9 & 3.4 \\
        & introvert & 5.5 & 5 to 15 & 11.6 & 2.5 \\
        & & 12 & 3 to 15 & 10.9 & 2.8 \\
      \midrule
        Walker & acting out & 8 & 0 to 26 & 19.7 & 6.8 \\
        & bad peer relations & 8 & 15 to 25 & 23.2 & 2.9 \\
        & immature & 8 & 5 to 19 & 17.3 & 2.9 \\
        & distractible & 8 & 1 to 14 & 9.4 & 3.6 \\
        & withdrawn & 8 & 4 to 14 & 13.2 & 1.9 \\
      \midrule
        EASI$^1$ & general emotionality & 8 & 6 to 23 & 14.1 & 3.9 \\
        & fear & 8 & 8 to 26 & 15.7 & 3.9 \\
        & anger & 8 & 5 to 24 & 15 & 4 \\
        & tempo & 8 & 5 to 25 & 13 & 4.3 \\
        & vigor & 8 & 5 to 22 & 11.6 & 3.5 \\
        & sociability & 8 & 9 to 25 & 19.5 & 2.9 \\
        & control & 8 & 10 to 26 & 18.5 & 3.2 \\
        & decisive & 8 & 6 to 22 & 14.9 & 3 \\
        & sensation & 8 & 5 to 20 & 13.8 & 3.2 \\
        & persistence & 8 & 7 to 25 & 16.6 & 3.4 \\
      \midrule
        Kohn and Rosman & confident/friendly & 2 & 1 to 5 & 3.7 & 0.9 \\
        & & 7 & 3 to 6 & 4.9 & 0.7 \\
        & anxious/withdrawn & 2 & 2 to 6 & 4.8 & 0.9 \\
        & & 7 & 3 to 6 & 5 & 0.5 \\
        & attention/cooperative & 2 & 1 to 6 & 3.3 & 0.9 \\
        & & 7 & 2 to 6 & 4.6 & 0.9 \\
        & distractible/disruptive & 2 & 2 to 6 & 3.8 & 0.9 \\
        & & 7 & 2 to 6 & 4.7 & 0.8 \\
      \midrule
        HOME & total & 0.5 & 15 to 38 & 27.7 & 5.3 \\
        & & 1.5 & 15 to 41 & 29.5 & 5.9 \\
        & & 2.5 & 14 to 43 & 29.8 & 5.9 \\
        & & 3.5 & 30 to 73 & 55.7 & 9.8 \\
        & & 4.5 & 31 to 73 & 58.6 & 9.5 \\
        & & 8 & 49 to 82 & 66.6 & 7.8 \\
        & maternal warmth & 0.5 & 1 to 10 & 7 & 2.1 \\
        & & 1.5 & 2 to 10 & 7.4 & 2.1 \\
        & & 2.5 & 1 to 10 & 7.3 & 1.8 \\
        & maternal involvement & 0.5 & 0 to 6 & 3.4 & 1.5 \\
        & & 1.5 & 0 to 6 & 3.3 & 1.6 \\
        & & 2.5 & 0 to 6 & 3.3 & 1.6 \\
        & physical environment & 8 & 8 to 16 & 12.7 & 1.8 \\
        & developmental stimulation & 8 & 2 to 14 & 8.2 & 2.9 \\
        & fostering independence & 8 & 9 to 15 & 13.8 & 1.3 \\
        & breadth of experience & 8 & 4 to 10 & 7.3 & 1.4 \\
        & language environment & 8 & 5 to 10 & 8.2 & 1.4 \\
        & emotional climate & 8 & 3 to 6 & 5.4 & 0.8 \\
        & orgnization of environment & 8 & 1 to 6 & 4.3 & 1.2 \\
        & play materials & 8 & 1 to 5 & 3.9 & 1 \\
        & need gratify of restriction & 8 & 1 to 3 & 2.8 & 0.5 \\
      \midrule
        PARI & authoritarian & 0.5 & 35 to 71 & 55 & 7.3\\
        & & 1.5 & 36 to 72 & 54& 7.2 \\
        & hostility & 0.5 & 25 to 52 & 38.9 & 5.9 \\
        & & 1.5 & 24 to 55 & 40.2 & 6.2 \\
        & democratic & 0.5 & 33 to 59 & 45.8& 5.6\\
        & & 1.5 & 35 to 58 & 46.4 & 4.7 \\
  \end{longtable}
\end{ThreePartTable}

\subsection{ABC/CARE Measures}

  \subsubsection{Stanford-Binet Intelligence Scale}

  The Stanford-Binet Intelligence Scales (commonly known as IQ) measures five main verbal and non-verbal factors of cognitive ability: fluid reasoning, knowledge, quantitative reasoning, visual-spatial processing, and working memory. IQ measured at ages 2 years, 3 years, 3.5 years, 4 years, 4.5 years, and 5 years were used to determine early cognitive abilities, and IQ measured at 12 years and 15 years were used to determine late cognitive abilities. There are observations in both CARE and ABC for IQ at all ages except 15 years old, where there are only observations in ABC.

  \subsubsection{Woodcock-Johnson}

  The Woodcock-Johnson Tests of Cognitive Abilities consist of a battery of intelligence tests. Reading, math, and total achievement at 12 years were determined using the mean of the Woodcock-Johnson and California Achievement Test (see below) scores. Knowledge at 12 years is simply the Woodcock-Johnson score at 12 years. There are observations in both CARE and ABC for Woodcock-Johnson at 12 years in reading, math, knowledge, and total achievement.  

  \subsubsection{California Achievement Test}

  The California Achievement Tests measure basic academic skills in reading, language, spelling, and mathematics. There are no observations in CARE, only in ABC for the California Achievement Test scores in reading, math, and total achievement. Please refer to the section on Woodcock-Johnson on how the test was incorporated in the measure of reading, math, knowledge, and achievement at 12 years old. Interestingly enough, general achievement at 12 years has no observations in ABC, only in CARE. On the other hand, reading and math have observations in both CARE and ABC.

  \subsubsection{Bayley Scales of Infant Development (IBR Version)}

  The Bayley Scales of Infant Development measure the mental, motor, and behavior of infants. Specifically, the Infant Behavior Record was used to measure the activity level, sociability, and task orientation of infants at ages 0.5 years, 1 year, 1.5 years, and 2 years. Scores at either ends of the extreme are not desirable, thus a median score indicates that the infant demostrates a normal level of the subdomain. There are observations in both CARE and ABC except for sociability at age 2, where there are only observations in ABC. Though cooperation was measured at all stated ages, there were too few observations and the subdomain was dropped.

  \subsubsection{Classroom Behavior Inventory}

  The Classroom Behavior Inventory measures a child's classroom behavior based on ratings given by teachers in the following subdomains: hostility, distractibility, introversion, extroversion, dependence, creativity, verbal intelligence, task orientation, and consideration at ages 5.5 years and 12 years. CBI scores for hostility, distractibility, introvert and dependence were reverse coded so that a higher score is more desirable. There are observations for all subdomains in CARE and ABC. 

  \subsubsection{Walker Problem Behavior Identification Checklist}

  The Walker Problem Behavior Identification Checklist identifies behavioral problems in children based on the following subdomains measured at 8 years: acting out, distractibility, immaturity, bad peer relationships, and withdrawn. All Walker scores were recoded so that higher scores indicate good behavior (in other words, less of the indicated problem in the subdomain). There are observations in all subdomains in both CARE and ABC. 

  \subsubsection{Emotionality/Activity/Sociability/Impulsitivity Temperament Survey}

  The EASI Temperament Survey measures subjects in the four temperaments at 8 years: emotionality (general, fear, anger), activity (tempo, vigor), sociability (sociability), and impulsitivity (control, decisive, sensation, and persistence). EASI scores for general emotionality, fear, anger, tempo, vigor, control, decisive, sensation, persistence were reverse coded so that a higher score indicates more calm, less fear, less anger, more slowed down, less vigorous, more control, more decisive, less thrill-seeking, and more persistent. There are observations in all subdomains for both CARE and ABC. 

  \subsubsection{Kohn and Rosman Test Behavior Inventory}

  The Kohn-Rosman Tests are completed by teachers and measure four main behavioral subdomains: confidently/friendly, anxious/withdrawn, attentive/cooperative, and distractible/disruptive at 2 years and 7 years. Kohn-Rosman scores for anxious/withdrawn and distractible/disruptive were reverse coded so that a higher score is more desirable behavior. Note that the mean score was used for standardization. There are observations in all subdomains for both CARE and ABC. 

  \subsubsection{Parenting Attitudes Research Instrument}

  The Parental Attitudes Research Instrument measures parental attitudes towards child-rearing in three main subdomains: authoritarian control, hostility, and democratic attitude when the child is 0.5 years and 1.5 years. New PARI factor scores based on the direct sum of subscales using UNC documentation were computed since the original PARI scores ranged from pari$\_$auth: 11 to 63, pari$\_$demo: 43 to 79, and pari$\_$hostl: 31 to 73 with no explanation. Higher scores indicate higher levels of the attitude. There are observations in all subdomains for both CARE and ABC. 

  \subsubsection{Home Observation for Measurement of the Environment Inventory}

  The Home Observation for the Measurement of the Environment Inventory measures the quality and quantity of household stimulation and support for the child. The scale measures the following subdomains in the child's home environment: total score, maternal involvement, maternal warmth, breadth of experience, development stimulation, emotional climate, fosterin independence, language environment, need gratify of restriction, organization of environment, physical environment, and play materials. Measures for maternal involvement and maternal warmth are available at 0.5 years, 1.5 years, and 2 years. Measures of the total score are available at 0.5 years, 1.5 years, 2.5 years, 3.5 years, 4.5 years, and 8 years. The remaining subdomains are all available at endpoint, 8 years. There are observations in both CARE and ABC. However, there are about half the number of CARE observations for the endpoint measurements compared to previous CARE observations. 

\subsection{Heat Maps of ABC/CARE Measures}

The following heat maps reflect the pairwise correlation coefficients amongst variables, or the strength of the linear relationship between two variables. Red represents a coefficient of 1 and blue represents a coefficient of -1. All observations that were non-empty for both variables are used in the pairwise correlation. The significance level of each correlation coefficient is also calculated, with the null hypothesis being that the correlation coefficient is not significantly different from zero. In other words, no significant linear relationship exists. The largest circle represents significance at the .05 level, the medium circle represents significance at the .1 level, and the smallest circle represents insignificance. Heat maps with both lower and upper triangular entries are split into male (lower triangular graph) and female (upper triangular graph). Exclusively lower triangular heat maps consider the entire range of available observations. All reverse coded subdomains in non-cognitive instruments are relabeled in the following heat maps to reflect what a higher score represents. 

\subsubsection{Noncognitive Component of School Achievement}

  The following heat maps depict the pairwise correlation between the residualized school achievement (after removing the effect from early and medium IQ), which are assumed to be reflective of noncognitive abilities, against factor variables of early and late socio-emotional measure, parental attitutes, and home environment.

    \begin{figure}[H]
      \centering
      \caption{Heat map with $p$-values of late non-cognitive achievement component against early and late factor variables}
      \label{fig:lateres_ach-factor}
      \begin{subfigure}{0.8\textwidth}
        \centering
        \caption{All available observations}          
        \includegraphics[width=\textwidth]{../output/Skills_Correlation/abccare/heat-map-sig-lateres_ach-factor.eps}
        \label{fig:lateres_ach-factor-nogender}
      \end{subfigure}

      \begin{subfigure}{0.8\textwidth}
        \centering
        \caption{Separated by gender}
        \includegraphics[width=\textwidth]{../output/Skills_Correlation/abccare/heat-map-sig-genders-lateres_ach-factor.eps}
        \label{fig:lateres_ach-factor-gender}
      \end{subfigure}
    \end{figure}

    \begin{figure}[H]  
      \centering
      \caption{Heat map with $p$-values of late non-cognitive achievement component against early and late factor variables} 
      \label{fig:earlyres_ach-factor}
      \begin{subfigure}{0.85\textwidth}
        \centering
        \caption{All available observations}        
        \includegraphics[width=\textwidth]{../output/Skills_Correlation/abccare/heat-map-sig-earlyres_ach-factor.eps}
        \label{fig:earlyres_ach-factor-nogender}
      \end{subfigure}

      \begin{subfigure}{0.85\textwidth} 
        \centering
        \caption{Separated by gender}
        \includegraphics[width=\textwidth]{../output/Skills_Correlation/abccare/heat-map-sig-genders-earlyres_ach-factor.eps}
        \label{fig:earlyres_ach-factor-gender}
      \end{subfigure}
    \end{figure}

  \subsubsection{Cognitive and Socioemotional Skills}

    \begin{figure}[H]   
      \centering
      \caption{Heat map with $p$-values of early and late cognitive skills}
      \label{fig:cog}
      \begin{subfigure}{0.775\textwidth}
        \centering
        \caption{All available observations}
        \includegraphics[width=\textwidth]{../output/Skills_Correlation/abccare/heat-map-sig-cognitiveskills.eps}
        \label{fig:cog-nogender}
      \end{subfigure}

      \begin{subfigure}{0.775\textwidth}
        \centering
        \caption{Separated by gender}       
        \includegraphics[width=\textwidth]{../output/Skills_Correlation/abccare/heat-map-sig-genders-cognitiveskills.eps}
        \label{fig:cog-gender}
      \end{subfigure}
    \end{figure}

    \begin{figure}[H] 
      \centering
      \caption{Heat map with $p$-values of IBR measures of task orientation, activity levels, and sociability over time from 0.5 years to 2y}         
      \includegraphics[width=\textwidth]{../output/Skills_Correlation/abccare/heat-map-sig-ibr.eps}
      \label{fig:ibr}
    \end{figure}

    \begin{figure}[H] 
      \centering
      \caption{Heat map with $p$-values of non-cognitive skills from different tests}    
      \label{fig:socio}  
      \begin{subfigure}{0.85\textwidth}  
        \centering
        \caption{Early non-cognitive skills} 
        \includegraphics[width=\textwidth]{../output/Skills_Correlation/abccare/heat-map-sig-earlynoncog.eps}
        \label{fig:earlysocio}
      \end{subfigure}

      \begin{subfigure}{0.85\textwidth} 
        \centering
        \caption{Late non-cognitive skills}        
        \includegraphics[width=\textwidth]{../output/Skills_Correlation/abccare/heat-map-sig-latenoncog.eps}
        \label{fig:latesocio}
      \end{subfigure}
    \end{figure}

    \begin{figure}[H]
      \centering
      \caption{Heat map with $p$-values of early non-cognitive instruments against early IQ}
      \label{fig:earlysocio-earlyiq}
      \begin{subfigure}{0.85\textwidth}  
        \centering
        \caption{Early CBI and early IQ}       
        \includegraphics[width=\textwidth]{../output/Skills_Correlation/abccare/heat-map-sig-earlycbi-earlycog.eps}
        \label{fig:earlycbi-earlyiq}
      \end{subfigure}

      \begin{subfigure}{0.85\textwidth} 
        \centering
        \caption{Early IBR and early IQ}         
        \includegraphics[width=\textwidth]{../output/Skills_Correlation/abccare/heat-map-sig-earlyibr-earlycog.eps}
        \label{fig:ibr-earlyiq}
      \end{subfigure}
    \end{figure}
    \clearpage
    \begin{figure}[H]
      \ContinuedFloat \centering
      \begin{subfigure}{0.85\textwidth}  
        \centering
        \caption{Early Kohn-Rosman and early IQ}
        \includegraphics[width=\textwidth]{../output/Skills_Correlation/abccare/heat-map-sig-earlykohnrosman-earlycog.eps}
        \label{fig:earlykr-earlyiq}
      \end{subfigure}
    \end{figure}

    \begin{figure}[H]
      \centering
      \caption{Heat map with $p$-values of late non-cognitive instruments against late cognitive}
      \label{fig:latesocio-latecog}
      \begin{subfigure}{0.85\textwidth}     
        \centering
        \caption{Late CBI and late cognitive skills, all available observations}
        \includegraphics[width=\textwidth]{../output/Skills_Correlation/abccare/heat-map-sig-latecbi-latecog.eps}
        \label{fig:latecbi-latecog-nogender}
      \end{subfigure}

      \begin{subfigure}{0.85\textwidth}    
        \centering
        \caption{Late CBI and late cognitive skills, separated by gender}
        \includegraphics[width=\textwidth]{../output/Skills_Correlation/abccare/heat-map-sig-genders-latecbi-latecog.eps}
        \label{fig:latecbi-latecog-gender}
      \end{subfigure}
    \end{figure}
    \clearpage
    \begin{figure}[H]
      \ContinuedFloat \centering    
      \begin{subfigure}{0.85\textwidth} 
        \centering
        \caption{Walker and late cognitive skills, all available observations}
        \includegraphics[width=\textwidth]{../output/Skills_Correlation/abccare/heat-map-sig-walker-latecog.eps}  
        \label{fig:walker-latecog-nogender}
      \end{subfigure}

      \begin{subfigure}{0.85\textwidth} 
        \centering
        \caption{Walker and late cognitive skills, separated by gender}
        \includegraphics[width=\textwidth]{../output/Skills_Correlation/abccare/heat-map-sig-genders-walker-latecog.eps}
        \label{fig:walker-latecog-gender}
      \end{subfigure}
    \end{figure}
    \clearpage
    \begin{figure}[H]
      \ContinuedFloat \centering
      \begin{subfigure}{0.85\textwidth}  
        \centering
        \caption{Late Kohn-Rosman and late cognitive skills, all available observations}         
        \includegraphics[width=\textwidth]{../output/Skills_Correlation/abccare/heat-map-sig-latekohnrosman-latecog.eps}
        \label{fig:latekr-latecog-nogender}
      \end{subfigure}

      \begin{subfigure}{0.85\textwidth}  
        \centering
        \caption{Late Kohn-Rosman and late cognitive skills, separated by gender}         
        \includegraphics[width=\textwidth]{../output/Skills_Correlation/abccare/heat-map-sig-genders-latekohnrosman-latecog.eps}
        \label{fig:latekr-latecog-gender}
      \end{subfigure}
    \end{figure}
    \clearpage
    \begin{figure}[H]
      \ContinuedFloat \centering
      \begin{subfigure}{0.85\textwidth}  
        \centering
        \caption{EASI and late cognitive skills, all available observations}
        \includegraphics[width=\textwidth]{../output/Skills_Correlation/abccare/heat-map-sig-easi-latecog.eps}
        \label{fig:easi-latecog-nogender}
      \end{subfigure}

      \begin{subfigure}{0.85\textwidth}  
        \centering
        \caption{EASI and late cognitive skills, separated by gender}
        \includegraphics[width=\textwidth]{../output/Skills_Correlation/abccare/heat-map-sig-genders-easi-latecog.eps}
        \label{fig:easi-latecog-gender}
      \end{subfigure}
    \end{figure}

    \begin{figure}[H]
      \centering
      \caption{Heat map with $p$-values of early non-cognitive instruments against late cognitive skills}
      \label{fig:earlysocio-latecog}
      \begin{subfigure}{0.85\textwidth}  
        \centering
        \caption{Early CBI and late cognitive skills, all available observations}        
        \includegraphics[width=\textwidth]{../output/Skills_Correlation/abccare/heat-map-sig-earlycbi-latecog.eps}
        \label{fig:earlycbi-latecog-nogender} 
      \end{subfigure}

      \begin{subfigure}{0.85\textwidth} 
        \centering
        \caption{Early CBI and late cognitive skills, separated by gender}        
        \includegraphics[width=\textwidth]{../output/Skills_Correlation/abccare/heat-map-sig-genders-earlycbi-latecog.eps}
        \label{fig:earlycbi-latecog-gender} 
      \end{subfigure}
    \end{figure}
    \clearpage
    \begin{figure}[H]
      \ContinuedFloat \centering
      \begin{subfigure}{0.85\textwidth}
        \centering
        \caption{Early IBR and late cognitive skills, all available observations}         
        \includegraphics[width=\textwidth]{../output/Skills_Correlation/abccare/heat-map-sig-earlyibr-latecog.eps}
        \label{fig:earlyibr-latecog-nogender}
      \end{subfigure} 

      \begin{subfigure}{0.85\textwidth} 
        \centering
        \caption{Early IBR and late cognitive skills, separated by gender}         
        \includegraphics[width=\textwidth]{../output/Skills_Correlation/abccare/heat-map-sig-genders-earlyibr-latecog.eps}
        \label{fig:earlyibr-latecog-gender}
      \end{subfigure}
    \end{figure}
    \clearpage
    \begin{figure}[H]
      \ContinuedFloat \centering
      \begin{subfigure}{0.85\textwidth}  
        \centering
        \caption{Early Kohn-Rosman and late cognitive skills, all available observations}        
        \includegraphics[width=\textwidth]{../output/Skills_Correlation/abccare/heat-map-sig-earlykohnrosman-latecog.eps}
        \label{fig:earlykohnrosman-latecog-nogender}
      \end{subfigure} 

      \begin{subfigure}{0.85\textwidth} 
        \centering
        \caption{Early Kohn-Rosman and late cognitive skills, separated by gender}        
        \includegraphics[width=\textwidth]{../output/Skills_Correlation/abccare/heat-map-sig-genders-earlykohnrosman-latecog.eps}
        \label{fig:earlykohnrosman-latecog-gender}
      \end{subfigure} 
    \end{figure}

    \begin{figure}[H]
      \centering
      \caption{Heat map with $p$-values of early cognitive skills against late non-cognitive instruments}
      \begin{subfigure}{0.85\textwidth} 
        \centering
        \caption{Walker and early IQ, all available observations}          
        \includegraphics[width=\textwidth]{../output/Skills_Correlation/abccare/heat-map-sig-earlycog-walker.eps}
        \label{fig:earlycog-walker-nogender}
      \end{subfigure} 

      \begin{subfigure}{0.85\textwidth} 
        \centering
        \caption{Walker and early IQ, separated by gender}          
        \includegraphics[width=\textwidth]{../output/Skills_Correlation/abccare/heat-map-sig-genders-earlycog-walker.eps}
        \label{fig:earlycog-walker-gender}
      \end{subfigure}
    \end{figure}
    \clearpage 
    \begin{figure}[H]
      \ContinuedFloat \centering
      \begin{subfigure}{0.85\textwidth}  
        \centering
        \caption{Late CBI and early IQ}         
        \includegraphics[width=\textwidth]{../output/Skills_Correlation/abccare/heat-map-sig-earlycog-latecbi.eps}
        \label{fig:earlycog-latecbi}
      \end{subfigure} 

      \begin{subfigure}{0.85\textwidth} 
        \centering
        \caption{Late Kohn-Rosman and early IQ}          
        \includegraphics[width=\textwidth]{../output/Skills_Correlation/abccare/heat-map-sig-earlycog-latekohnrosman.eps}
        \label{fig:earlycog-latekohnrosman}
      \end{subfigure} 
    \end{figure}
    \clearpage
    \begin{figure}[H]
      \ContinuedFloat \centering
      \begin{subfigure}{0.85\textwidth} 
        \centering
        \caption{EASI and early IQ}         
        \includegraphics[width=\textwidth]{../output/Skills_Correlation/abccare/heat-map-sig-earlycog-easi.eps}
        \label{fig:earlycog-easi}
      \end{subfigure} 
    \end{figure}

  \subsubsection{Parental Behavior and Home Environment}

  The following heat maps depict the pairwise correlations amongst and between measures of parental attitudes (PARI) and home environment (HOME). 

    \begin{figure}[H]  
      \centering
      \caption{Heat map with $p$-values of early maternal warmth and involvement from 0.5y to 2.5y}  
      \label{fig:homemother}
      \begin{subfigure}{0.85\textwidth} 
        \centering
        \caption{All available observations}
        \includegraphics[width=\textwidth]{../output/Skills_Correlation/abccare/heat-map-sig-homemother.eps}
        \label{fig:homemother-nogender}
      \end{subfigure}

      \begin{subfigure}{0.85\textwidth} 
        \centering
        \caption{Separated by gender}        
        \includegraphics[width=\textwidth]{../output/Skills_Correlation/abccare/heat-map-sig-genders-homemother.eps}
        \label{fig:homemother-gender}
      \end{subfigure}
    \end{figure}

    \begin{figure}[H]
      \centering
      \caption{Heat map with $p$-values of PARI (0.5y to 1.5y) against early home environment (0.5y to 4.5y)}
      \label{fig:pari-earlyhome}
      \begin{subfigure}{0.85\textwidth}   
        \centering
        \caption{All available observations}
        \includegraphics[width=\textwidth]{../output/Skills_Correlation/abccare/heat-map-sig-pari-earlyhome.eps}
        \label{fig:pari-earlyhome-nogender}
      \end{subfigure}

      \begin{subfigure}{0.85\textwidth}   
        \centering
        \caption{Separated by gender}
        \includegraphics[width=\textwidth]{../output/Skills_Correlation/abccare/heat-map-sig-genders-pari-earlyhome.eps}
        \label{fig:pari-earlyhome-gender}
      \end{subfigure}
    \end{figure}

    \begin{figure}[H]
      \centering
      \caption{Heat map with $p$-values of PARI (0.5y to 1.5y) against late home environment (8y)}
      \label{fig:pari-latehome}
      \begin{subfigure}{0.85\textwidth} 
        \centering
        \caption{All available observations} 
        \includegraphics[width=\textwidth]{../output/Skills_Correlation/abccare/heat-map-sig-pari-latehome.eps}
        \label{fig:pari-latehome-nogender}
      \end{subfigure}

      \begin{subfigure}{0.85\textwidth} 
        \centering
        \caption{Separated by gender} 
        \includegraphics[width=\textwidth]{../output/Skills_Correlation/abccare/heat-map-sig-genders-pari-latehome.eps}
        \label{fig:pari-latehome-gender}
      \end{subfigure}
    \end{figure}

  \subsubsection{Parental Behavior and Home Environment vs. Cog/Non-Cog Skills}

    \begin{figure}[H] 
      \centering
      \caption{Heat map with $p$-values of maternal involvement at 2.5y against Walker}
      \label{fig:mhome-walker}
      \begin{subfigure}{0.775\textwidth} 
        \centering
        \caption{All available observations}
        \includegraphics[width=\textwidth]{../output/Skills_Correlation/abccare/heat-map-sig-homemother-walker.eps}
        \label{fig:mhome-walker-nogender}
      \end{subfigure}

      \begin{subfigure}{0.775\textwidth} 
        \centering
        \caption{Separated by gender} 
        \includegraphics[width=\textwidth]{../output/Skills_Correlation/abccare/heat-map-sig-genders-homemother-walker.eps}
        \label{fig:mhome-walker-gender}
      \end{subfigure}
    \end{figure}

    \begin{figure}[H]
      \centering
      \caption{Heat map with $p$-values of PARI (0.5y to 1.5y), maternal involvement (2.5y), against late IQ}   
      \label{fig:pari-mhome-lateiq}    
      \begin{subfigure}{0.85\textwidth} 
        \centering
        \caption{All available observations}       
        \includegraphics[width=\textwidth]{../output/Skills_Correlation/abccare/heat-map-sig-pari-homemother-lateiq.eps}
        \label{fig:pari-mhome-lateiq-nogender}
      \end{subfigure}

      \begin{subfigure}{0.85\textwidth} 
        \centering
        \caption{Separated by gender}        
        \includegraphics[width=\textwidth]{../output/Skills_Correlation/abccare/heat-map-sig-genders-pari-homemother-lateiq.eps}
        \label{fig:pari-mhome-lateiq-gender}
      \end{subfigure}
    \end{figure}

    \begin{figure}[H]
      \centering
      \caption{Heat map with $p$-values of maternal involvement at 2.5y against late CBI at 12y}
      \label{fig:mhome-latecbi}
      \begin{subfigure}{0.85\textwidth}   
        \centering
        \caption{All available observations}
        \includegraphics[width=\textwidth]{../output/Skills_Correlation/abccare/heat-map-sig-homemother-latecbi.eps}
        \label{fig:mhome-latecbi-nogender}
      \end{subfigure}

      \begin{subfigure}{0.85\textwidth}    
        \centering
        \caption{Separated by gender}    
        \includegraphics[width=\textwidth]{../output/Skills_Correlation/abccare/heat-map-sig-genders-homemother-latecbi.eps}
        \label{fig:mhome-latecbi-gender}
      \end{subfigure}
    \end{figure}

    \begin{figure}[H]
      \centering
      \caption{Heat map with $p$-values of early CBI at 5.5y against late HOME at 8y} 
      \label{fig:earlycbi-latehome}
      \begin{subfigure}{0.85\textwidth}    
        \centering
        \caption{All available observations}       
        \includegraphics[width=\textwidth]{../output/Skills_Correlation/abccare/heat-map-sig-earlycbi-latehome.eps}
        \label{fig:earlycbi-latehome-nogender}
      \end{subfigure}

      \begin{subfigure}{0.85\textwidth}   
        \centering
        \caption{Separated by gender}       
        \includegraphics[width=\textwidth]{../output/Skills_Correlation/abccare/heat-map-sig-genders-earlycbi-latehome.eps}
        \label{fig:earlycbi-latehome-gender}
      \end{subfigure}
    \end{figure}

    \begin{figure}[H]   
      \centering
      \caption{Heat map with $p$-values of PARI (0.5y to 1.5y) against late CBI (12y)}    
      \includegraphics[width=\textwidth]{../output/Skills_Correlation/abccare/heat-map-sig-pari-latecbi.eps}
      \label{fig:pari-latecbi}
    \end{figure}

