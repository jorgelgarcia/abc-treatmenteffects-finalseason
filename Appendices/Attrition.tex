\section{Accounting for Attrition} \label{appendix:attrition}

\noindent We use a standard inverse probability weighting (IPW) scheme based on a logit model that predicts attrition using baseline characteristics.\footnote{\citet{Horvitz_Thompson_1952_JASA}.} Let $R = 1$ if the child is randomized to treatment, and $R = 0$ otherwise. Similarly, let $A = 1$ denote the case where we observe outcome $Y$, and $A = 0$ otherwise. $X,W,Z$ are background variables. $X$ could be affected by treatment and $W,Z$ are pre-treatment. $Z$ is used as part of the randomization protocol and $Z$ is not.

\noindent We assume $A$ is independent of $Y$ conditional on $Z$, $R$, $X$, and $W$. More formally, we invoke

\begin{assumption} \label{ass:attr}
	\begin{align*}
		A \independent Y | W, Z, X, R.
	\end{align*}
\end{assumption}

\noindent Let $Y_{r}$ represent outcome $Y$ when $R$ is fixed to take the value $r$. Based on Assumption~\ref{ass:attr}, we identify $\mathbb{E}[Y_r]$ as follows:

\begin{align} \label{eq:case2}
\mathbb{E}[Y_r] & = \int y f_{Y_r|Z}(y) f_Z(z) dydz \\ \nonumber
	           & = \int y f_{Y|Z,R=r}(y) f_Z(z) dydz \\ \nonumber
				& = \int y f_{Y|R=r,W,Z,X}(y) f_{W,X|R=r}(w,x) f_Z(z) dydxdwdz \\ \nonumber
				& = \int y f_{Y|R=r,W,Z,X,A=1}(y) f_{W,X|R=r,Z}(w,x) f_Z(z) dydxdwdz
\end{align}
\noindent where each component of the last expression in \eqref{eq:case2} is straightforward to recover from the data. Using Bayes' Theorem, we can write an equivalent expression to make the IPW scheme explicit. That is, we apply Bayes' Theorem to $f_{W,X|R=r,Z}(w,x)$ and $f_Z(z)$ to obtain

\begin{equation*}
f_{W,X|R=r,Z}(w,x) = \frac{f_{W,X|R=r,Z,A=1}(w,x) P(A=1|R=r,Z)}{P(A=1|R=r,W,Z,X)}
\end{equation*}
and
\begin{equation*}
	f_Z(z) = \frac{f_{Z|R=r,A=1}(z) P(R=r,A=1)}{P(R=r,A=1|Z)}.
\end{equation*}
\noindent Substituting these expressions into \eqref{eq:case2}, we obtain

\begin{align*} \label{eq:case2ipw}
\mathbb{E}[Y_r] & = \int y f_{Y,X,W,Z|R=r,A=1}(y,x,w,z) \frac{P(R=r,A=1) P(A=1|R=r,Z)}{P(R=r,A=1|Z) P(A=1|R=r,W,Z,X)} dydxdwdz \\
	            & = \int y f_{Y,X,W,Z|R=r,A=1}(y,x,w,z) \frac{P(R=r,A=1)}{P(R=r|Z) P(A=1|R=r,W,Z,X)} dydxdwdz. \\
\end{align*}

\noindent Assumption \ref{ass:attr} generalizes the matching assumption of \citet{Campbell_Conti_etal_2014_EarlyChildhoodInvestments}. It conditions not only on pre-program variables but on fully observed post-treatment variables, $X$. This enables us to account  for two types of selection:  (i) selection into treatment; and (ii) selection into item response. The corresponding sample estimator for $\mathbb{E}[Y_r]$ is

\begin{align*}
\sum_{i \in \mathcal{I}} y a_{i} w_{i,r} \mathbf{1}(r_i = r)
\end{align*}
\noindent where $\mathcal{I}$ indexes the individuals in the sample, $a_i$ indicates whether we observe $Y$ for individual $i$, and

\begin{align*}
	w_{i,r} = \frac{1}{\pi_r(z_i) \alpha(r_i,w_i,z_i,x_i)} \frac{1}{\sum_k{\frac{\mathbf{1}(r_i = r) \mathbf{1}(a_i = 1)}{\pi_r(z_k)\alpha(r_k,w_k,z_k,x_k)}}},
\end{align*}
\noindent with $\pi_r(z) := P(R=r|Z=z)$ and $\alpha(r,w,z,x) := P(A=1|R=r,W=w,Z=z,X=x)$. The weight $\pi_r$ corrects for selection into treatment based on pre-program variables $Z$. The weight $\alpha$ corrects for item non-response based on $R, W, Z, X$. \\
