\section{Accounting for Attrition} \label{appendix:attrition}

\noindent We use a standard inverse probability weighting (IPW) scheme based on a logit model that predicts attrition using baseline characteristics.\footnote{\citet{Horvitz_Thompson_1952_JASA}.} Recall that $R = 1$ if the child is randomized to treatment, and $R = 0$ otherwise. Similarly, let $A = 1$ denote the case where we observe a generic outcome $Y$, and $A = 0$ otherwise. As in the main text, $\bm{B}$ represents background (pre-treatment) variables and $\bm{X}$ variables that could be affected by treatment and that we use to predict $Y$. We abstract from dynamics in $Y$ and $\bm{X}$ for simplicity, but the procedure could be easily generalized.\\

\noindent We assume $A$ is independent of $Y$ conditional on $\bm{X}$ and $\bm{B}$. More formally, we invoke


\begin{assumption} \label{ass:attr}
	\begin{align*}
		A \independent Y | \bm{X}, \bm{B}, R.
	\end{align*}
\end{assumption}

\noindent Let $Y_{r}$ represent outcome $Y$ when $R$ is fixed to take the value $r$. Based on Assumption~\ref{ass:attr}, we identify $\mathbb{E}[Y_r]$ as follows:

\begin{align} \label{eq:case2}
\mathbb{E}[Y_r] & = \int \int y f_{ Y_ r| \bm{B} } (y) f_{\bm{B}} (b) dydb \\ \nonumber
	           & = \int \int y f_{Y| \bm{B}, R=r}(y) f_{\bm{B}} (b) dydb \\ \nonumber
	           & = \int \int \int y f_{Y|R=r,\bm{X}, \bm{B}} (y) f_{\bm{X} | R=r} (x) f_{\bm{B}} (b) dydxdb \\ \nonumber
				& = \int \int \int y f_{Y|R=r,\bm{X}, \bm{B}, A=1} (y) f_{\bm{X} | R=r, \bm{B} }(x) f_{\bm{B}} (b) dydxdb
\end{align}

\noindent where each component of the last expression in \eqref{eq:case2} is straightforward to recover from the data. Using Bayes' Theorem, we can write an equivalent expression to make the IPW scheme explicit. That is, we apply Bayes' Theorem to $f_{\bm{X} | R=r, \bm{B} }(x)$ and $f_{\bm{B}} (b)$ to obtain

\begin{equation*}
f_{\bm{X}|R=r,\bm{B}}(x) = \frac{f_{\bm{X}|R=r,\bm{B},A=1}(x) P(A=1|R=r,\bm{B})}{P(A=1|R=r,\bm{X},\bm{B})}
\end{equation*}
and
\begin{equation*}
	f_{\bm{B}} (b) = \frac{f_{\bm{B}|R=r,A=1} (x) P(R=r,A=1)}{P(R=r,A=1|\bm{B})}.
\end{equation*}
\noindent Substituting these expressions into \eqref{eq:case2}, we obtain

\begin{align*} \label{eq:case2ipw}
\mathbb{E}[Y_r] & = \int \int \int y f_{Y,\bm{X},\bm{B}|R=r,A=1}(y,x,b) \frac{P(R=r,A=1) P(A=1|R=r,\bm{B})}{P(R=r,A=1|\bm{B}) P(A=1|R=r,\bm{X},\bm{B})} dydxdb \\
	            & = \int \int  \int y f_{Y,\bm{X},\bm{B}|R=r,A=1}(y,x,b) \frac{P(R=r,A=1)}{P(R=r|\bm{B}) P(A=1|R=r,\bm{X},\bm{B})} dydxdb. \\
\end{align*}

\noindent Assumption \ref{ass:attr} generalizes the matching assumption of \citet{Campbell_Conti_etal_2014_EarlyChildhoodInvestments}. It conditions not only on pre-program variables but on fully observed post-treatment variables, $\bm{X}$. This enables us to account for two types of selection:  (i) selection into treatment; and (ii) selection into item response. The corresponding sample estimator for $\mathbb{E}[Y_r]$ is

\begin{align*}
\sum_{i \in \mathcal{I}} y \alpha_{i} \beta_{i,r} \mathbf{1}(r_i = r)
\end{align*}
\noindent where $\mathcal{I}$ indexes the individuals in the sample, $\alpha_i$ indicates whether we observe $Y$ for individual $i$, and

\begin{align*}
	\beta_{i,r} = \frac{1}{\pi_r(x_i) \alpha(r_i,x_i,b_i)} \frac{1}{\sum_k{\frac{\mathbf{1}(r_i = r) \mathbf{1}(\alpha_i = 1)}{\pi_r(x_k)\alpha(r_k,x_k,b_k)}}},
\end{align*}
\noindent with $\pi_r(x) := P(R=r|\bm{B}=b)$ and $\alpha(r,x,b) := P(A=1|R=r,\bm{X}=x,\bm{B}=b)$. The weight $\pi_r$ corrects for selection into treatment based on pre-program variables $\bm{B}$. The weight $\alpha_{i}$ corrects for item non-response based on $R, \bm{X}, \bm{B}$.\\

\subsection{Variables to Construct IPW}

\noindent We follow the procedure in Section~\ref{appendix:bvariables} to choose the variables with which we construct the weights. Tables~\ref{table:attpooled} lists the variables in which we account for attrition, together with the variables we use to construct the IPW scheme. We jointly estimate these weights for males and females. Section~\ref{appendix:methodology} explains and justifies the cases in which we account for attrition.

\newgeometry{top=.1in, bottom=.1in, left=.1in, right=.1in}
\begin{sidewaystable}[H]
\begin{threeparttable}
\caption{Variables Used to Create IPW Scheme, Estimates Pooling Females and Males}
\label{table:attpooled}
\centering
% CONTENT CREATED ON SPREADSHEET, TREAT AS A .CSV (TAB DELIMITED)
% CAN BE COPIED INTO A SPREADSHEET PROGRAM (EXCEL, LIBRECALC) FOR EDITING
\tiny
\begin{tabular}{l r r l l l}
\toprule											
Variable	&	Age	&	Obs.	&	(1)	&	(2)	&	(3)	\\
\midrule											
Parental Labor Income	&	1.5	&	112	& Mother's Age at Baseline	& APGAR 1 min.	&  Cohort \\
 Parental Labor Income	&	2.5	&	112	&	Mother's Age at Baseline	& APGAR 1 min.	&  Cohort \\
 Parental Labor Income	&	3.5	&	110	&	Mother's Age at Baseline	& APGAR 1 min.	&  Cohort \\
Parental Labor Income	&	4.5	&	131	&	High Risk Index (HRI)	& APGAR 1 min.	&  Cohort \\
 Parental Labor Income	&	8	&	87	&	High Risk Index (HRI)	& APGAR 1 min.	&  Cohort \\
 Parental Labor Income	&	12	&	108	&	High Risk Index (HRI)	& APGAR 1 min.	&  Cohort \\
 Parental Labor Income	&	15	&	92	&	APGAR 5 min. & 	Premature at Birth & Number of Siblings, Baseline \\
 Parental Labor Income	&	21	&	112	&	High Risk Index (HRI)	& APGAR 1 min.	&  Cohort \\
\\
Subject Labor Income	&	21	&	131	& High Risk Index (HRI) & APGAR 1 min. & Cohort \\
 Subject Labor Income	&	30	&	132	& Mother's Years of Education, Baseline & APGAR 5 min. & Cohort	\\
 Subject Public Transfer Income	&	21	&	105	& High Risk Index (HRI) & APGAR 1 min. & Cohort \\
 Subject Public Transfer Income	&	30	&	133	& Mother's Years of Education, Baseline & APGAR 5 min. & Cohort	\\
 \\
 Self-reported Health	&	Mid-30s	&	92	&	APGAR 1 min. & APGAR 5 min. & Premature at Birth	\\
 Self-reported Drug User	&	Mid-30s	&	89	&	APGAR 1 min. & APGAR 5 min. & Premature at Birth	\\
 Systolic Blood Pressure (mm Hg)	&	Mid-30s	&	90	&	APGAR 1 min. & Premature at Birth & Number of Siblings, Baseline	\\
 Diastolic Blood Pressure (mm Hg)	&	Mid-30s	 &	90	&	APGAR 1 min. & Premature at Birth & Number of Siblings, Baseline	\\
 
 Prehypertension, Sys. B.P. $>$ 120 or Dys. B.P. $>$ 80	&	Mid-30s	&	90	&	APGAR 1 min. & Premature at Birth & Number of Siblings, Baseline	\\
 Hypertension, Sys. B.P. $>$ 140 or Dys. B.P. $>$ 90	&	Mid-30s	&	90	&	APGAR 1 min. & Premature at Birth & Number of Siblings, Baseline	\\
 High-Density Lipoprotein (HDL) Cholesterol (mg/dL)	&	Mid-30s	&	93	&	APGAR 1 min. & Premature at Birth & Number of Siblings, Baseline	\\
 Dyslipidemia (HDL $<$ 40 mg/dL)	&	Mid-30s	&	93	&	APGAR 1 min. & Premature at Birth & Number of Siblings, Baseline	\\
 Hemoglobin Level (\%)	&	Mid-30s	&	92	&	APGAR 1 min. & Premature at Birth & Number of Siblings, Baseline	\\
 Prediabetes, Hemoglobin $>$ 5.7\%	&	Mid-30s	&	92	&	APGAR 1 min. & Premature at Birth & Number of Siblings, Baseline	\\
 Diabetes, Hemoglobin $>$ 6.5\%	&	Mid-30s	&	92	&	APGAR 1 min. & Premature at Birth & Number of Siblings, Baseline	\\
 Vitamin D Deficiency ($<$ 20 ng/mL)	&	Mid-30s	&	93	&	APGAR 1 min. & Premature at Birth & Number of Siblings, Baseline	\\
Measured BMI	&	Mid-30s	&	88	&	APGAR 1 min. & APGAR 5 min. & Premature at Birth \\
 Obesity (BMI $>$ 30)	&	Mid-30s	&	90	&	APGAR 1 min. & APGAR 5 min. & Premature at Birth \\
 Severe Obesity (BMI $>$ 35)	&	Mid-30s	&	91	&	APGAR 1 min. & APGAR 5 min. & Premature at Birth \\
 Waist-hip Ratio	&	Mid-30s	&	84	& APGAR 1 min. & Premature at Birth & Number of Siblings, Baseline \\
 Abdominal Obesity	&	Mid-30s	&	84	& APGAR 1 min. & Premature at Birth & Number of Siblings, Baseline \\
 Framingham Risk Score	&	Mid-30s	&	88	& APGAR 1 min. & Premature at Birth & Number of Siblings, Baseline \\

\bottomrule											
\end{tabular}
\begin{tablenotes}
\footnotesize
\item Note: This table provides a list of the variables we use to construct the IPW scheme to account for attrition when calculating treatment effects pooling females and males. The procedure to select these variables is described in Appendix~\ref{appendix:bvariables}.
\end{tablenotes}
\end{threeparttable}
\end{sidewaystable}

\restoregeometry
\doublespacing 