\section{Accounting for Attrition} \label{appendix:attrition}

\noindent We use a standard inverse probability weighting (IPW) scheme based on a logit model that predicts attrition using baseline characteristics.\footnote{\citet{Horvitz_Thompson_1952_JASA}.} Recall that $R = 1$ if the child is randomized to treatment, and $R = 0$ otherwise. Similarly, let $A = 1$ denote the case where we observe a generic outcome $Y$, and $A = 0$ otherwise. As in the main text, $\mathcal{B}$ represents background (pre-treatment) variables and $X$ variables that could be affected by treatment and that we use to predict $Y$. We abstract from dynamics in $Y$ and $X$ for simplicity, but the procedure could be easily generalized.\\

\noindent We assume $A$ is independent of $Y$ conditional on $\mathcal{B}$ and $X$. More formally, we invoke


\begin{assumption} \label{ass:attr}
	\begin{align*}
		A \independent Y | \mathcal{B}, X, R.
	\end{align*}
\end{assumption}

\noindent Let $Y_{r}$ represent outcome $Y$ when $R$ is fixed to take the value $r$. Based on Assumption~\ref{ass:attr}, we identify $\mathbb{E}[Y_r]$ as follows:

\begin{align} \label{eq:case2}
\mathbb{E}[Y_r] & = \int \int y f_{Y_r|X}(y) f_X(x) dydx \\ \nonumber
	           & = \int \int y f_{Y|X,R=r}(y) f_X(x) dydx \\ \nonumber
				& = \int \int \int y f_{Y|R=r,\mathcal{B},X}(y) f_{\mathcal{B}|R=r}(b) f_X(x) dydbdx \\ \nonumber
				& = \int \int \int y f_{Y|R=r,\mathcal{B},X,A=1}(y) f_{\mathcal{B}|R=r,X}(b) f_X dydbdx
\end{align}
\noindent where each component of the last expression in \eqref{eq:case2} is straightforward to recover from the data. Using Bayes' Theorem, we can write an equivalent expression to make the IPW scheme explicit. That is, we apply Bayes' Theorem to $f_{\mathcal{B}|R=r,X}(b)$ and $f_X(x)$ to obtain

\begin{equation*}
f_{\mathcal{B}|R=r,X}(b) = \frac{f_{\mathcal{B}|R=r,X,A=1}(b) P(A=1|R=r,X)}{P(A=1|R=r,\mathcal{B},X)}
\end{equation*}
and
\begin{equation*}
	f_X(x) = \frac{f_{X|R=r,A=1}(x) P(R=r,A=1)}{P(R=r,A=1|X)}.
\end{equation*}
\noindent Substituting these expressions into \eqref{eq:case2}, we obtain

\begin{align*} \label{eq:case2ipw}
\mathbb{E}[Y_r] & = \int \int \int y f_{Y,\mathcal{B},X|R=r,A=1}(y,b.x) \frac{P(R=r,A=1) P(A=1|R=r,X)}{P(R=r,A=1|X) P(A=1|R=r,\mathcal{B},X)} dydbdx \\
	            & = \int \int  \int y f_{Y,\mathcal{B},X|R=r,A=1}(y,b,x) \frac{P(R=r,A=1)}{P(R=r|X) P(A=1|R=r,\mathcal{B},X)} dydbdx. \\
\end{align*}

\noindent Assumption \ref{ass:attr} generalizes the matching assumption of \citet{Campbell_Conti_etal_2014_EarlyChildhoodInvestments}. It conditions not only on pre-program variables but on fully observed post-treatment variables, $X$. This enables us to account for two types of selection:  (i) selection into treatment; and (ii) selection into item response. The corresponding sample estimator for $\mathbb{E}[Y_r]$ is

\begin{align*}
\sum_{i \in \mathcal{I}} y \alpha_{i} \beta_{i,r} \mathbf{1}(r_i = r)
\end{align*}
\noindent where $\mathcal{I}$ indexes the individuals in the sample, $a_i$ indicates whether we observe $Y$ for individual $i$, and

\begin{align*}
	\beta_{i,r} = \frac{1}{\pi_r(x_i) \alpha(r_i,b_i,x_i)} \frac{1}{\sum_k{\frac{\mathbf{1}(r_i = r) \mathbf{1}(a_i = 1)}{\pi_r(x_k)\alpha(r_k,b_k,x_k)}}},
\end{align*}
\noindent with $\pi_r(x) := P(R=r|X=x)$ and $\alpha(r,b,x) := P(A=1|R=r,\mathcal{B}=b,X=x)$. The weight $\pi_r$ corrects for selection into treatment based on pre-program variables $X$. The weight $\alpha_{i}$ corrects for item non-response based on $R, \mathcal{B}, X$.\\

\subsection{Variables to Construct IPW}

\noindent We follow the procedure in Section~\ref{appendix:bvariables} to choose the variables with which we construct the weights. Tables~\ref{table:attpooled}-\ref{table:attfemales} list the variables in which we account fro attrition, together with the variables we use to construct the IPW scheme. We provide different lists for estimates pooling females and males and estimates by gender. Section~\ref{appendix:methodology} explains the cases in which we account for attrition.

\newgeometry{top=.1in, bottom=.1in, left=.1in, right=.1in}
\begin{sidewaystable}[H] 
\begin{threeparttable}
\caption{Variables Used to Create IPW Scheme, Estimates Pooling Females and Males}
\label{table:attpooled}
\centering 
% CONTENT CREATED ON SPREADSHEET, TREAT AS A .CSV (TAB DELIMITED)
% CAN BE COPIED INTO A SPREADSHEET PROGRAM (EXCEL, LIBRECALC) FOR EDITING
\scriptsize
\begin{tabular}{l r r l l l}
\toprule											
Partially Observed Outcomes	&	Age	&	N	&	\multicolumn{3}{c}{Variables Used to Produce IPW}	\\
\midrule	
IQ Score 				& 6.5 	& 126   & High Risk Index (HRI)	& APGAR 1 min.	&  Cohort \\
IQ Score 				& 7 	& 118   & High Risk Index (HRI)	& APGAR 5 min.	&  Cohort \\
IQ Score 				& 8 	& 125   & High Risk Index (HRI)	& APGAR 1 min.	&  Cohort \\
\\										
Achievement Score 		& 5.5	& 105 	& High Risk Index (HRI)	& APGAR 1 min.	&  Cohort \\ 
Achievement Score 		& 6		& 124 	& High Risk Index (HRI)	& APGAR 1 min.	&  Cohort \\ 
Achievement Score 		& 6.5	& 89 	& High Risk Index (HRI)	& APGAR 1 min.	&  Cohort \\ 
Achievement Score 		& 7		& 90 	& High Risk Index (HRI)	& APGAR 1 min.	&  Cohort \\ 
Achievement Score 		& 7.5	& 121 	& High Risk Index (HRI)	& APGAR 1 min.	&  Cohort \\ 
Achievement Score 		& 8		& 123 	& High Risk Index (HRI)	& APGAR 1 min.	&  Cohort \\ 
Achievement Score 		& 8.5	& 122 	& High Risk Index (HRI)	& APGAR 5 min.	&  Cohort \\ 
\\
Parental Labor Income	&	1.5	&	112	& Mother's Age at Baseline	& APGAR 1 min.	&  Cohort \\
 Parental Labor Income	&	2.5	&	112	&	Mother's Age at Baseline	& APGAR 1 min.	&  Cohort \\
 Parental Labor Income	&	3.5	&	110	&	Mother's Age at Baseline	& APGAR 1 min.	&  Cohort \\
 Parental Labor Income	&	8	&	87	&	High Risk Index (HRI)	& APGAR 1 min.	&  Cohort \\
 Parental Labor Income	&	12	&	108	&	High Risk Index (HRI)	& APGAR 1 min.	&  Cohort \\
 Parental Labor Income	&	15	&	92	&	APGAR 5 min. & 	Premature at Birth & Number of Siblings, Baseline \\
 Parental Labor Income	&	21	&	73	&	High Risk Index (HRI)	& APGAR 1 min.	&  Cohort \\
\\
HOME Score				& 8		& 	100 & 	High Risk Index (HRI)	& APGAR 1 min.	&  Cohort \\
\\
Father at Home			& 8		& 	116 & 	High Risk Index (HRI)	& APGAR 1 min.	&  Cohort \\
\\
Subject Public Transfer Income	&	21	&	105	& High Risk Index (HRI) & APGAR 1 min. & Cohort \\
 \\
Total Felony Arrests		& Mid-30s	& 115 &	APGAR 1 min. & APGAR 5 min. & Cohort	\\
Total Misdemeanor Arrests	& Mid-30s	& 115 &	APGAR 1 min. & APGAR 5 min. & Cohort	\\
 \\
 Self-reported Health	&	Mid-30s	&	92	&	APGAR 1 min. & APGAR 5 min. & Premature at Birth	\\
 Self-reported Drug User	&	Mid-30s	&	89	&	APGAR 1 min. & APGAR 5 min. & Premature at Birth	\\
 Systolic Blood Pressure (mm Hg)	&	Mid-30s	&	90	&	APGAR 1 min. & Premature at Birth & Number of Siblings, Baseline	\\
 Diastolic Blood Pressure (mm Hg)	&	Mid-30s	 &	90	&	APGAR 1 min. & Premature at Birth & Number of Siblings, Baseline	\\
 Prehypertension, Sys. B.P. $>$ 120 or Dys. B.P. $>$ 80	&	Mid-30s	&	90	&	APGAR 1 min. & Premature at Birth & Number of Siblings, Baseline	\\
 Hypertension, Sys. B.P. $>$ 140 or Dys. B.P. $>$ 90	&	Mid-30s	&	90	&	APGAR 1 min. & Premature at Birth & Number of Siblings, Baseline	\\
 High-Density Lipoprotein (HDL) Cholesterol (mg/dL)	&	Mid-30s	&	93	&	APGAR 1 min. & Premature at Birth & Number of Siblings, Baseline	\\
 Dyslipidemia (HDL $<$ 40 mg/dL)	&	Mid-30s	&	93	&	APGAR 1 min. & Premature at Birth & Number of Siblings, Baseline	\\
 Hemoglobin Level (\%)	&	Mid-30s	&	92	&	APGAR 1 min. & Premature at Birth & Number of Siblings, Baseline	\\
 Prediabetes, Hemoglobin $>$ 5.7\%	&	Mid-30s	&	92	&	APGAR 1 min. & Premature at Birth & Number of Siblings, Baseline	\\
 Diabetes, Hemoglobin $>$ 6.5\%	&	Mid-30s	&	92	&	APGAR 1 min. & Premature at Birth & Number of Siblings, Baseline	\\
 Vitamin D Deficiency ($<$ 20 ng/mL)	&	Mid-30s	&	93	&	APGAR 1 min. & Premature at Birth & Number of Siblings, Baseline	\\
 Measured BMI	&	Mid-30s	&	88	&	APGAR 1 min. & APGAR 5 min. & Premature at Birth \\
 Obesity (BMI $>$ 30)	&	Mid-30s	&	90	&	APGAR 1 min. & APGAR 5 min. & Premature at Birth \\
 Severe Obesity (BMI $>$ 35)	&	Mid-30s	&	91	&	APGAR 1 min. & APGAR 5 min. & Premature at Birth \\
 Waist-hip Ratio	&	Mid-30s	&	84	& APGAR 1 min. & Premature at Birth & Number of Siblings, Baseline \\
 Abdominal Obesity	&	Mid-30s	&	84	& APGAR 1 min. & Premature at Birth & Number of Siblings, Baseline \\
 Framingham Risk Score	&	Mid-30s	&	88	& APGAR 1 min. & Premature at Birth & Number of Siblings, Baseline \\
 Brief Symptom Survey (BSI) Score & Mid-30s & 92 &	APGAR 1 min. & APGAR 5 min. & Premature at Birth	\\
\bottomrule											
\end{tabular}
\begin{tablenotes}
\footnotesize
\item Note: This table provides a list of the variables we use to construct the IPW scheme to account for attrition when calculating treatment effects pooling females and males. The procedure to select these variables is described in Appendix~\ref{appendix:bvariables}.
\end{tablenotes}
\end{threeparttable}
\end{sidewaystable}

\begin{sidewaystable}[H] 
\begin{threeparttable}
\caption{Variables Used to Create IPW Scheme, Estimates for Females}
\label{table:attfemales}
\centering 
% CONTENT CREATED ON SPREADSHEET, TREAT AS A .CSV (TAB DELIMITED)
% CAN BE COPIED INTO A SPREADSHEET PROGRAM (EXCEL, LIBRECALC) FOR EDITING
\tiny
\begin{tabular}{l r r l l l}
\toprule										
Variable	&	Age	&	Obs.	&	(1)	&	(2)	&	(3)	\\
\midrule											
Parental Labor Income	&	2	&	33	&	Subject Birth Year	&	Mother Works, age 5	&	Father works before pregnancy	\\
Parental Labor Income	&	3	&	33	&	Subject Birth Year	&	Mother Works, age 5	&	Father works before pregnancy	\\
Parental Labor Income	&	4	&	31	&	Subject Birth Year	&	Father works before pregnancy	&	Siblings in Household, age 5	\\
Parental Labor Income	&	5	&	44	&	Mother Works before Pregnant	&	Mother Works, age 5	&	Father works before pregnancy	\\
Parental Labor Income	&	9	&	40	&	Mother Works, age 5	&	HRI 2: No Maternal Relatives	&	Father works before pregnancy	\\
Parental Labor Income	&	12	&	45	&	Subject Birth Year	&	Mother Works, age 5	&	Father works before pregnancy	\\
Parental Labor Income	&	15	&	47	&	Mother Works, age 5	&	HRI 2: No Maternal Relatives	&	Father works before pregnancy	\\
\\
Subject Labor Income	&	21	&	51	&	Mother's WAIS Performance IQ	&	Mother's Age at entry	&	ASR t-score: Attention Deficit and Hyperactivity	\\
Subject Labor Income	&	30	&	53	&	HRI 10: Other special circumstances	&	ASR t-score: Substance Abuse	&	Years of Education	\\
Subject Public Transfer Income	&	21	&	49	&	Mother's WAIS Comprehension	&	Number of cigarettes smoked per day	&	ASR t-score: Attention Deficit and Hyperactivity	\\
Subject Public Transfer Income	&	30	&	53	&	HRI 10: Other special circumstances	&	ASR t-score: Substance Abuse	&	Years of Education	\\
\\
Self-reported Health	&	Mid-30s	&	40	&	Length at birth	&	ASR t-score: Substance Abuse	&	Years of Education	\\
Self-reported Drug User	&	Mid-30s	&	40	&	Length at birth	&	ASR t-score: Substance Abuse	&	Years of Education	\\
Systolic Blood Pressure (mm Hg)	&	Mid-30s	&	40	&	Length at birth	&	ASR t-score: Substance Abuse	&	Years of Education	\\
Diastolic Blood Pressure (mm Hg)	&	Mid-30s	&	40	&	Length at birth	&	ASR t-score: Substance Abuse	&	Years of Education	\\
Prehypertension, Sys. B.P. $>$ 120 or Dys. B.P. $>$ 80	&	Mid-30s	&	40	&	Length at birth	&	ASR t-score: Substance Abuse	&	Years of Education	\\
Hypertension, Sys. B.P. $>$ 140 or Dys. B.P. $>$ 90	&	Mid-30s	&	40	&	Length at birth	&	ASR t-score: Substance Abuse	&	Years of Education	\\
High-Density Lipoprotein (HDL) Cholesterol (mg/dL)	&	Mid-30s	&	40	&	Length at birth	&	ASR t-score: Substance Abuse	&	Years of Education	\\
Dyslipidemia (HDL $<$ 40 mg/dL)	&	Mid-30s	&	40	&	Length at birth	&	ASR t-score: Substance Abuse	&	Years of Education	\\
Hemoglobin Level (\%)	&	Mid-30s	&	39	&	Length at birth	&	ASR t-score: Substance Abuse	&	Years of Education	\\
Prediabetes, Hemoglobin $>$ 5.7\%	&	Mid-30s	&	39	&	Length at birth	&	ASR t-score: Substance Abuse	&	Years of Education	\\
Diabetes, Hemoglobin $>$ 6.5\%	&	Mid-30s	&	39	&	Length at birth	&	ASR t-score: Substance Abuse	&	Years of Education	\\
Vitamin D Deficiency ($<$ 20 ng/mL)	&	Mid-30s	&	40	&	Length at birth	&	ASR t-score: Substance Abuse	&	Years of Education	\\
Measured BMI	&	Mid-30s	&	40	&	Length at birth	&	ASR t-score: Substance Abuse	&	Years of Education	\\
Obesity (BMI $>$ 30)	&	Mid-30s	&	40	&	Length at birth	&	ASR t-score: Substance Abuse	&	Years of Education	\\
Severe Obesity (BMI $>$ 35)	&	Mid-30s	&	40	&	Length at birth	&	ASR t-score: Substance Abuse	&	Years of Education	\\
Waist-hip Ratio	&	Mid-30s	&	37	&	Length at birth	&	ASR t-score: Substance Abuse	&	Years of Education	\\
Abdominal Obesity	&	Mid-30s	&	37	&	Length at birth	&	ASR t-score: Substance Abuse	&	Years of Education	\\
Framingham Risk Score	&	Mid-30s	&	39	&	Length at birth	&	ASR t-score: Substance Abuse	&	Years of Education	\\
\bottomrule																																	
\end{tabular}
\begin{tablenotes}
\footnotesize
\item Note: This table provides a list of the variables we use to construct the IPW scheme to account for attrition when calculating treatment effects for females. The procedure to select these variables is described in Appendix~\ref{appendix:bvariables}.
\end{tablenotes}
\end{threeparttable}
\end{sidewaystable}

\begin{sidewaystable}[H] 
\begin{threeparttable}
\caption{Variables Used to Create IPW Scheme, Estimates for Males}
\label{table:attfemales}
\centering 
% CONTENT CREATED ON SPREADSHEET, TREAT AS A .CSV (TAB DELIMITED)
% CAN BE COPIED INTO A SPREADSHEET PROGRAM (EXCEL, LIBRECALC) FOR EDITING
\tiny
\begin{tabular}{l r r l l l}
\hline\hline											
Variable	&	Age	&	Obs.	&	(1)	&	(2)	&	(3)	\\
\hline											
Parent Labor Income	&	2	&	39	&	Subject Birth Year	&	Mother Works, age 3	&	Siblings in Household, age 3	\\
Parent Labor Income	&	3	&	39	&	Subject Birth Year	&	Mother Works, age 3	&	Father works before pregnancy	\\
Parent Labor Income	&	4	&	38	&	Subject Birth Year	&	Mother Works, age 4	&	Siblings in Household, age 3	\\
Parent Labor Income	&	5	&	42	&	Mother Works, age 3	&	Mother Works, age 4	&	Siblings in Household, age 5	\\
Parent Labor Income	&	9	&	42	&	Mother Works, age 3	&	Father home, age 5	&	Siblings in Household, age 5	\\
Parent Labor Income	&	12	&	46	&	Subject Birth Year	&	Mother Works, age 4	&	Father home, age 5	\\
Parent Labor Income	&	15	&	43	&	Mother Works, age 4	&	Father works before pregnancy	&	Siblings in Household, age 5	\\
\\
Subject Labor Income	&	21	&	45	&	HRI 10: Other special circumstances	&	Gestational Age $\leq$ 36	&	ASR t-score: Substance Abuse	\\
Subject Labor Income	&	30	&	47	&	Father's Age at entry	&	Mother Married at entry	&	ASR t-score: Substance Abuse	\\
Subject Public Transfer Income	&	21	&	48	&	Father's Age at entry	&	Gestational Age $\leq$ 36	&	Years of Education	\\
Subject Public Transfer Income	&	30	&	48	&	Mother's WAIS Performance IQ	&	ASR t-score: Antisocial Personality Problems	&	Had physical exam for illness in last 2 years	\\
\\
Self-reported Health	&	Mid-30s	&	30	&	HRI 2: No Maternal Relatives	&	ASR t-score: Substance Abuse	&	ASR t-score: Antisocial Personality Problems	\\
Self-reported Drug User	&	Mid-30s	&	27	&	Gestational Age $\leq$ 36	&	ASR t-score: Substance Abuse	&	Had physical exam for illness in last 2 years	\\
Systolic Blood Pressure (mm Hg)	&	Mid-30s	&	28	&	Gestational Age $\leq$ 36	&	ASR t-score: Substance Abuse	&	Had physical exam for illness in last 2 years	\\
Diastolic Blood Pressure (mm Hg)	&	Mid-30s	&	28	&	Gestational Age $\leq$ 36	&	ASR t-score: Substance Abuse	&	Had physical exam for illness in last 2 years	\\
Prehypertension, Sys. B.P. $>$ 120 or Dys. B.P. $>$ 80	&	Mid-30s	&	28	&	Gestational Age $\leq$ 36	&	ASR t-score: Substance Abuse	&	Had physical exam for illness in last 2 years	\\
Hypertension, Sys. B.P. $>$ 140 or Dys. B.P. $>$ 90	&	Mid-30s	&	28	&	Gestational Age $\leq$ 36	&	ASR t-score: Substance Abuse	&	Had physical exam for illness in last 2 years	\\
High-Density Lipoprotein (HDL) Cholesterol (mg/dL)	&	Mid-30s	&	31	&	Gestational Age $\leq$ 36	&	ASR t-score: Substance Abuse	&	Had physical exam for illness in last 2 years	\\
Dyslipidemia (HDL $<$ 40 mg/dL)	&	Mid-30s	&	31	&	Gestational Age $\leq$ 36	&	ASR t-score: Substance Abuse	&	Had physical exam for illness in last 2 years	\\
Hemoglobin Level (\%)	&	Mid-30s	&	31	&	Gestational Age $\leq$ 36	&	ASR t-score: Substance Abuse	&	Had physical exam for illness in last 2 years	\\
Prediabetes, Hemoglobin $>$ 5.7\%	&	Mid-30s	&	31	&	Gestational Age $\leq$ 36	&	ASR t-score: Substance Abuse	&	Had physical exam for illness in last 2 years	\\
Diabetes, Hemoglobin $>$ 6.5\%	&	Mid-30s	&	31	&	Gestational Age $\leq$ 36	&	ASR t-score: Substance Abuse	&	Had physical exam for illness in last 2 years	\\
Vitamin D Deficiency ($<$ 20 ng/mL)	&	Mid-30s	&	31	&	Gestational Age $\leq$ 36	&	ASR t-score: Substance Abuse	&	Had physical exam for illness in last 2 years	\\
Measured BMI	&	Mid-30s	&	26	&	Gestational Age $\leq$ 36	&	ASR t-score: Substance Abuse	&	Had physical exam for illness in last 2 years	\\
Obesity (BMI $>$ 30)	&	Mid-30s	&	28	&	Father works before pregnancy	&	ASR t-score: Substance Abuse	&	ASR t-score: Attention Deficit and Hyperactivity	\\
Severe Obesity (BMI $>$ 35)	&	Mid-30s	&	29	&	HRI 2: No Maternal Relatives	&	Mother's WAIS Comprehension	&	Number of times smoked marijuana	\\
Waist-hip Ratio	&	Mid-30s	&	25	&	Gestational Age $\leq$ 36	&	ASR t-score: Substance Abuse	&	Had physical exam for illness in last 2 years	\\
Abdominal Obesity	&	Mid-30s	&	25	&	Gestational Age $\leq$ 36	&	ASR t-score: Substance Abuse	&	Had physical exam for illness in last 2 years	\\
Framingham Risk Score	&	Mid-30s	&	27	&	Gestational Age $\leq$ 36	&	ASR t-score: Substance Abuse	&	Had physical exam for illness in last 2 years	\\
\hline\hline																							
\end{tabular}
\begin{tablenotes}
\footnotesize
\item Note: This table provides a list of the variables we use to construct the IPW scheme to account for attrition when calculating treatment effects for males.  The procedure to select these variables is described in Appendix~\ref{appendix:bvariables}.
\end{tablenotes}
\end{threeparttable}
\end{sidewaystable}
\restoregeometry
\doublespacing