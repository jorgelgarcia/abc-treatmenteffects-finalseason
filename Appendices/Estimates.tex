\section{Using Our Estimates to Understand Recent Benefit/Cost Analyses}\label{appendix:bcaestimates}

We use our analysis to examine the empirical foundations of the approach to benefit/cost analysis taken in a prototypical study of \citet{Kline_Walters_2016_QJE}. They use data from the Head Start Impact Study (HSIS) and report a benefit/cost ratio between $1.50$ and $1.84$.\footnote{HSIS is a one-year-long randomized evaluation of Head Start.} Their analysis proceeds in three steps: (i) calculate the treatment effect on IQ around age 4.5 and 5\footnote{An index based on the Peabody Picture Vocabulary and Woodcock Johnson III Tests.}; (ii) monetize this gain using the return to the IQ measured between ages 5 and 7 in terms of net present value of labor income at age 27 using the analysis of \citet{Chetty_Friedman_etal_2011_QJoE}.\footnote{The \citet{Chetty_Friedman_etal_2011_QJoE} return is based on Stanford Achievement Tests.}$^,$\footnote{For this comparison exercise, we interpret the earnings estimated in \citet{Chetty_Friedman_etal_2011_QJoE} to be equivalent to labor income.} Calculations from \citet{Chetty_Friedman_etal_2011_QJoE} indicate that a 1 standard deviation gain in IQ at age 5 implies a $13.1\%$ increase in the net present value of labor income through age 27;\footnote{This is based on combining information from Project Star and administrative data at age 27.} and (iii) calculate the benefit/cost ratio based on this gain and their own calculations of the program's cost.\footnote{Their calculation assigns the net present value of labor income through age 27 of $\$385,907.17$ to the control-group participants, as estimated by  \citet{Chetty_Friedman_etal_2011_QJoE}.}$^,$\footnote{All money values that we provide in this section are in 2014 USD. We discount the value provided by \citet{Chetty_Friedman_etal_2011_QJoE} to the age of birth of the children in our sample (first cohort).}


\begin{table}[!htbp]
\begin{threeparttable}
\caption{Alternative Cost-benefit Analyses Calculations}
\label{table:comparing}
\centering
\footnotesize

\begin{tabular}{cllcc}
\toprule
Age & \mc{1}{c}{NPV Source} & Component & \citet{Kline-Walters_2016_QJE} & Authors' Method \\
& & & Method & \\
\midrule
\multirow{2}{*}{27} & \cite{Chetty_Friedman_etal_2010_HowDoesYour} & Earnings & 1.04 (s.e. 0.36) &  \\
& ABC/CARE-calculated & Earnings & 1.36 (s.e. 0.04) &  0.14 (s.e. 0.05)\\
\midrule
\multirow{2}{*}{34} & ABC/CARE-calculated & Earnings & 0.45 (s.e. 0.04) & 0.45 (s.e. 0.17) \\
& ABC/CARE-calculated & All & 0.88 (s.e. 0.04) &  0.88 (s.e. 0.34) \\
\midrule
\multirow{2}{*}{Life-cycle} &  ABC/CARE-calculated & Earnings & 1.58 (s.e. 0.07) & 1.58 (s.e. 0.60) \\
& ABC/CARE-calculated & All & 4.58 (s.e. 0.25) & 5.63 (s.e. 2.15) \\
\bottomrule
\end{tabular}

\begin{tablenotes}
\footnotesize
\item Note: This table displays benefit/cost ratios based on the methodology in \citet{Kline_Walters_2016_QJE} and based on our own methodology. Age: age at which we stop calculating the net-present value. NPV Source: source where we obtain the net present value. Component: item used to compute net present value (all refers to the net present value of all the components). \citet{Kline_Walters_2016_QJE} Method: estimate based on these authors methodology. Author's Method: estimates based on our methodology. Standard errors are based on the empirical bootstrap distribution.
\end{tablenotes}
\end{threeparttable}
\end{table}

To analyze how our estimates compare to those based on the method in \citet{Kline_Walters_2016_QJE}, we display a series of exercises in the fourth column of Table~\ref{table:comparing}. For purposes of comparison, the fifth column of Table~\ref{table:comparing} shows the analogous estimates based on our own method.

In the first exercise, we use both the ``return to IQ'' and the net-present value of labor income at age 27 reported in  \citet{Chetty_Friedman_etal_2011_QJoE}. In the second exercise, we perform a similar exercise but we use our own estimate of the net-present value of labor income at age 27.\footnote{This allows us to compute our own ``return to WPPSI'' and impute it to the treatment-group individuals.} The remaining exercises are similar, but (i) increase the age range over which we calculate the net-present value of labor income; or (ii) consider the value of all the lifetime components we analyze throughout the paper. The more inclusive the benefits measured and the longer the horizon over which they are measured, the greater the benefit/cost ration.

Our methodology provides a more accurate estimate of the net-present value (and the return to IQ) of the components. We better quantify the effects of the analyzed experiment by considering the whole life-cycle; we also better approximate the statistical uncertainty of our estimates by considering both the sampling error in the experimental and auxiliary samples and the prediction error due to the interpolation and extrapolation. Proceeding in this fashion enables us to an extensive sensitivity analysis of each of the components we monetize. 