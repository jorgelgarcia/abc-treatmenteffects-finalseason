\section{{Expanding the Analyzed Outcomes}} \label{appendix:moreoutcomes}

\noindent In this appendix, we expand the analysis in Section~\ref{section:results} by considering an additional set of outcomes. Instead of considering \noutcomes, we consider \noutcomesexpp when pooling the sample and \noutcomesexpm\ and \noutcomesexpf\ when for males and females, respectively.\footnote{We fail to estimate some treatment effects because in some cases the treatment indicator and the controls entirely determine the outcome. For the \noutcomes\ in Section~\ref{section:results}, we do not have this problem.} The difference between this set of outcomes and the set we present in Section~\ref{section:results} is that (i) the outcomes in Section~\ref{section:results} contain outcomes we monetize when computing the cost-benefit analysis; and (ii) it is not entirely clear if treatment should have an effect on the outcomes we add in this appendix. The \noutcomes\ outcomes in this appendix include the \noutcomes\ outcomes in Section~\ref{section:results}.\\

\noindent To illustrate the difference between the outcomes in this appendix and the outcomes in Section~\ref{section:results}, consider two outcomes related to employment. In Section~\ref{section:results}, we include employment at age 30. In this appendix, we include both employment at age 30 and job satisfaction at age 30. Job satisfaction is a subjective variable, and it is not as clear what the effect of treatment on outcome is.\\

\noindent Table~\ref{table:abccare_rslt_pooled_counts_all} presents the results for the pooled sample, and Table~\ref{table:abccare_rslt_male_counts_all} and \ref{table:abccare_rslt_female_counts_all} present the results for males and females. It is still the case both that (i) we find a majority of positive treatment effects; and (ii) more than 10\% of these positive treatment effects are significant at the 10\%. It is true that the results weaken if compared to those we consider in Section~\ref{section:results}, but this is expected given how we construct the lists of outcomes. Appendix~\ref{appendix:morebycat} present the results by categories.

	\begin{table}[H]
     \caption{Combining Functions, Pooled Sample} 
     \label{table:abccare_rslt_pooled_counts_all}
	
  \begin{tabular}{ccccccccc}
  \toprule
     & \scriptsize{(1)} & \scriptsize{(2)} & \scriptsize{(3)} & \scriptsize{(4)} & \scriptsize{(5)} & \scriptsize{(6)} & \scriptsize{(7)} & \scriptsize{(8)} \\ 
    \midrule

    \mc{1}{l}{\scriptsize{\% Pos. TE}} & \mc{1}{c}{\scriptsize{60}} & \mc{1}{c}{\scriptsize{62}} & \mc{1}{c}{\scriptsize{60}} & \mc{1}{c}{\scriptsize{65}} & \mc{1}{c}{\scriptsize{63}} & \mc{1}{c}{\scriptsize{59}} & \mc{1}{c}{\scriptsize{59}} & \mc{1}{c}{\scriptsize{61}} \\  

     & \mc{1}{c}{\scriptsize{\textbf{(0.013)}}} & \mc{1}{c}{\scriptsize{\textbf{(0.000)}}} & \mc{1}{c}{\scriptsize{\textbf{(0.013)}}} & \mc{1}{c}{\scriptsize{\textbf{(0.000)}}} & \mc{1}{c}{\scriptsize{\textbf{(0.000)}}} & \mc{1}{c}{\scriptsize{\textbf{(0.013)}}} & \mc{1}{c}{\scriptsize{\textbf{(0.013)}}} & \mc{1}{c}{\scriptsize{\textbf{(0.013)}}} \\  

    \mc{1}{l}{\scriptsize{\% Pos. TE $|$ 10\% Significance}} & \mc{1}{c}{\scriptsize{28}} & \mc{1}{c}{\scriptsize{23}} & \mc{1}{c}{\scriptsize{19}} & \mc{1}{c}{\scriptsize{16}} & \mc{1}{c}{\scriptsize{21}} & \mc{1}{c}{\scriptsize{24}} & \mc{1}{c}{\scriptsize{21}} & \mc{1}{c}{\scriptsize{27}} \\  

     & \mc{1}{c}{\scriptsize{\textbf{(0.000)}}} & \mc{1}{c}{\scriptsize{\textbf{(0.000)}}} & \mc{1}{c}{\scriptsize{\textbf{(0.026)}}} & \mc{1}{c}{\scriptsize{\textbf{(0.092)}}} & \mc{1}{c}{\scriptsize{\textbf{(0.026)}}} & \mc{1}{c}{\scriptsize{\textbf{(0.000)}}} & \mc{1}{c}{\scriptsize{\textbf{(0.000)}}} & \mc{1}{c}{\scriptsize{\textbf{(0.000)}}} \\  

  \bottomrule
  \end{tabular}

	\end{table}  

	\begin{table}[H]
     \caption{Combining Functions, Male Sample} 
     \label{table:abccare_rslt_male_counts_all}
	
  \begin{tabular}{ccccccccc}
  \toprule
     & \scriptsize{(1)} & \scriptsize{(2)} & \scriptsize{(3)} & \scriptsize{(4)} & \scriptsize{(5)} & \scriptsize{(6)} & \scriptsize{(7)} & \scriptsize{(8)} \\ 
    \midrule

    \mc{1}{l}{\scriptsize{\% Pos. TE}} & \mc{1}{c}{\scriptsize{52}} & \mc{1}{c}{\scriptsize{54}} & \mc{1}{c}{\scriptsize{39}} & \mc{1}{c}{\scriptsize{46}} & \mc{1}{c}{\scriptsize{40}} & \mc{1}{c}{\scriptsize{55}} & \mc{1}{c}{\scriptsize{57}} & \mc{1}{c}{\scriptsize{56}} \\  

     & \mc{1}{c}{\scriptsize{(0.289)}} & \mc{1}{c}{\scriptsize{(0.184)}} & \mc{1}{c}{\scriptsize{(0.987)}} & \mc{1}{c}{\scriptsize{(0.671)}} & \mc{1}{c}{\scriptsize{(0.961)}} & \mc{1}{c}{\scriptsize{(0.158)}} & \mc{1}{c}{\scriptsize{\textbf{(0.079)}}} & \mc{1}{c}{\scriptsize{(0.171)}} \\  

    \mc{1}{l}{\scriptsize{\% Pos. TE $|$ 10\% Significance}} & \mc{1}{c}{\scriptsize{15}} & \mc{1}{c}{\scriptsize{16}} & \mc{1}{c}{\scriptsize{9}} & \mc{1}{c}{\scriptsize{8}} & \mc{1}{c}{\scriptsize{8}} & \mc{1}{c}{\scriptsize{17}} & \mc{1}{c}{\scriptsize{14}} & \mc{1}{c}{\scriptsize{16}} \\  

     & \mc{1}{c}{\scriptsize{(0.132)}} & \mc{1}{c}{\scriptsize{\textbf{(0.053)}}} & \mc{1}{c}{\scriptsize{(0.539)}} & \mc{1}{c}{\scriptsize{(0.618)}} & \mc{1}{c}{\scriptsize{(0.724)}} & \mc{1}{c}{\scriptsize{\textbf{(0.026)}}} & \mc{1}{c}{\scriptsize{(0.118)}} & \mc{1}{c}{\scriptsize{\textbf{(0.053)}}} \\  

  \bottomrule
  \end{tabular}
	\end{table}  

	\begin{table}[H]
     \caption{Combining Functions, Female Sample} 
     \label{table:abccare_rslt_female_counts_all}
	
  \begin{tabular}{ccccccccc}
  \toprule
     & \scriptsize{(1)} & \scriptsize{(2)} & \scriptsize{(3)} & \scriptsize{(4)} & \scriptsize{(5)} & \scriptsize{(6)} & \scriptsize{(7)} & \scriptsize{(8)} \\ 
    \midrule

     & \scriptsize{(1)} & \scriptsize{(2)} & \scriptsize{(3)} & \scriptsize{(4)} & \scriptsize{(5)} & \scriptsize{(6)} & \scriptsize{(7)} & \scriptsize{(8)} \\ 
    \hline  

    \mc{1}{l}{\scriptsize{\% Pos. TE}} & \mc{1}{c}{\scriptsize{67}} & \mc{1}{c}{\scriptsize{68}} & \mc{1}{c}{\scriptsize{70}} & \mc{1}{c}{\scriptsize{72}} & \mc{1}{c}{\scriptsize{71}} & \mc{1}{c}{\scriptsize{62}} & \mc{1}{c}{\scriptsize{61}} & \mc{1}{c}{\scriptsize{63}} \\  

     & \mc{1}{c}{\scriptsize{\textbf{(0.000)}}} & \mc{1}{c}{\scriptsize{\textbf{(0.000)}}} & \mc{1}{c}{\scriptsize{\textbf{(0.000)}}} & \mc{1}{c}{\scriptsize{\textbf{(0.000)}}} & \mc{1}{c}{\scriptsize{\textbf{(0.000)}}} & \mc{1}{c}{\scriptsize{\textbf{(0.013)}}} & \mc{1}{c}{\scriptsize{\textbf{(0.000)}}} & \mc{1}{c}{\scriptsize{\textbf{(0.000)}}} \\  

    \mc{1}{l}{\scriptsize{\% Pos. TE $|$ 10\% Significance}} & \mc{1}{c}{\scriptsize{31}} & \mc{1}{c}{\scriptsize{25}} & \mc{1}{c}{\scriptsize{31}} & \mc{1}{c}{\scriptsize{25}} & \mc{1}{c}{\scriptsize{33}} & \mc{1}{c}{\scriptsize{23}} & \mc{1}{c}{\scriptsize{18}} & \mc{1}{c}{\scriptsize{26}} \\  

     & \mc{1}{c}{\scriptsize{\textbf{(0.000)}}} & \mc{1}{c}{\scriptsize{\textbf{(0.000)}}} & \mc{1}{c}{\scriptsize{\textbf{(0.000)}}} & \mc{1}{c}{\scriptsize{\textbf{(0.000)}}} & \mc{1}{c}{\scriptsize{\textbf{(0.000)}}} & \mc{1}{c}{\scriptsize{\textbf{(0.000)}}} & \mc{1}{c}{\scriptsize{\textbf{(0.053)}}} & \mc{1}{c}{\scriptsize{\textbf{(0.000)}}} \\  

  \bottomrule
  \end{tabular}
	\end{table}  

\section{{Outcomes by Category}} \label{appendix:morebycat}


	\begin{sidewaystable}[H]
     \caption{Combining Functions by Category, Pooled Sample} 
     \label{table:abccare_rslt_pooled_counts_n50a100_all}
	  \begin{tabular}{cccccccccc}
  \toprule

    \scriptsize{Category} & \scriptsize{(1)} & \scriptsize{(2)} & \scriptsize{(3)} & \scriptsize{(4)} & \scriptsize{(5)} & \scriptsize{(6)} & \scriptsize{(7)} & \scriptsize{(8)} &  \\ 
    \midrule  

    \mc{1}{l}{\scriptsize{Cognitive Skills}} & \mc{1}{c}{\scriptsize{100}} & \mc{1}{c}{\scriptsize{96}} & \mc{1}{c}{\scriptsize{100}} & \mc{1}{c}{\scriptsize{96}} & \mc{1}{c}{\scriptsize{100}} & \mc{1}{c}{\scriptsize{96}} & \mc{1}{c}{\scriptsize{96}} & \mc{1}{c}{\scriptsize{96}} & \mc{1}{c}{\scriptsize{25}} \\  

     & \mc{1}{c}{\scriptsize{\textbf{(0.000)}}} & \mc{1}{c}{\scriptsize{\textbf{(0.000)}}} & \mc{1}{c}{\scriptsize{\textbf{(0.000)}}} & \mc{1}{c}{\scriptsize{\textbf{(0.000)}}} & \mc{1}{c}{\scriptsize{\textbf{(0.000)}}} & \mc{1}{c}{\scriptsize{\textbf{(0.000)}}} & \mc{1}{c}{\scriptsize{\textbf{(0.000)}}} & \mc{1}{c}{\scriptsize{\textbf{(0.000)}}} &  \\  

    \mc{1}{l}{\scriptsize{Noncognitive Skills}} & \mc{1}{c}{\scriptsize{49}} & \mc{1}{c}{\scriptsize{54}} & \mc{1}{c}{\scriptsize{49}} & \mc{1}{c}{\scriptsize{58}} & \mc{1}{c}{\scriptsize{60}} & \mc{1}{c}{\scriptsize{50}} & \mc{1}{c}{\scriptsize{52}} & \mc{1}{c}{\scriptsize{54}} & \mc{1}{c}{\scriptsize{117}} \\  

     & \mc{1}{c}{\scriptsize{(0.539)}} & \mc{1}{c}{\scriptsize{(0.316)}} & \mc{1}{c}{\scriptsize{(0.513)}} & \mc{1}{c}{\scriptsize{(0.184)}} & \mc{1}{c}{\scriptsize{(0.145)}} & \mc{1}{c}{\scriptsize{(0.461)}} & \mc{1}{c}{\scriptsize{(0.395)}} & \mc{1}{c}{\scriptsize{(0.329)}} &  \\  

    \mc{1}{l}{\scriptsize{Mother's Employment, Education, and Income}} & \mc{1}{c}{\scriptsize{50}} & \mc{1}{c}{\scriptsize{50}} & \mc{1}{c}{\scriptsize{25}} & \mc{1}{c}{\scriptsize{75}} & \mc{1}{c}{\scriptsize{50}} & \mc{1}{c}{\scriptsize{50}} & \mc{1}{c}{\scriptsize{50}} & \mc{1}{c}{\scriptsize{25}} & \mc{1}{c}{\scriptsize{4}} \\  

     & \mc{1}{c}{\scriptsize{(0.671)}} & \mc{1}{c}{\scriptsize{(0.237)}} & \mc{1}{c}{\scriptsize{(0.987)}} & \mc{1}{c}{\scriptsize{\textbf{(0.039)}}} & \mc{1}{c}{\scriptsize{(0.671)}} & \mc{1}{c}{\scriptsize{(0.566)}} & \mc{1}{c}{\scriptsize{(0.263)}} & \mc{1}{c}{\scriptsize{(0.987)}} &  \\  

    \mc{1}{l}{\scriptsize{Childhood Household Environment}} & \mc{1}{c}{\scriptsize{67}} & \mc{1}{c}{\scriptsize{80}} & \mc{1}{c}{\scriptsize{53}} & \mc{1}{c}{\scriptsize{53}} & \mc{1}{c}{\scriptsize{60}} & \mc{1}{c}{\scriptsize{80}} & \mc{1}{c}{\scriptsize{60}} & \mc{1}{c}{\scriptsize{93}} & \mc{1}{c}{\scriptsize{15}} \\  

     & \mc{1}{c}{\scriptsize{(0.105)}} & \mc{1}{c}{\scriptsize{\textbf{(0.013)}}} & \mc{1}{c}{\scriptsize{(0.132)}} & \mc{1}{c}{\scriptsize{(0.474)}} & \mc{1}{c}{\scriptsize{\textbf{(0.079)}}} & \mc{1}{c}{\scriptsize{\textbf{(0.026)}}} & \mc{1}{c}{\scriptsize{(0.224)}} & \mc{1}{c}{\scriptsize{\textbf{(0.000)}}} &  \\  

    \mc{1}{l}{\scriptsize{Adult Household Environment}} & \mc{1}{c}{\scriptsize{78}} & \mc{1}{c}{\scriptsize{67}} & \mc{1}{c}{\scriptsize{56}} & \mc{1}{c}{\scriptsize{56}} & \mc{1}{c}{\scriptsize{44}} & \mc{1}{c}{\scriptsize{78}} & \mc{1}{c}{\scriptsize{67}} & \mc{1}{c}{\scriptsize{67}} & \mc{1}{c}{\scriptsize{9}} \\  

     & \mc{1}{c}{\scriptsize{\textbf{(0.000)}}} & \mc{1}{c}{\scriptsize{\textbf{(0.066)}}} & \mc{1}{c}{\scriptsize{(0.342)}} & \mc{1}{c}{\scriptsize{(0.368)}} & \mc{1}{c}{\scriptsize{(0.684)}} & \mc{1}{c}{\scriptsize{\textbf{(0.000)}}} & \mc{1}{c}{\scriptsize{\textbf{(0.026)}}} & \mc{1}{c}{\scriptsize{\textbf{(0.026)}}} &  \\  

    \mc{1}{l}{\scriptsize{Education, Employment, Income}} & \mc{1}{c}{\scriptsize{54}} & \mc{1}{c}{\scriptsize{54}} & \mc{1}{c}{\scriptsize{59}} & \mc{1}{c}{\scriptsize{71}} & \mc{1}{c}{\scriptsize{57}} & \mc{1}{c}{\scriptsize{54}} & \mc{1}{c}{\scriptsize{54}} & \mc{1}{c}{\scriptsize{61}} & \mc{1}{c}{\scriptsize{28}} \\  

     & \mc{1}{c}{\scriptsize{(0.395)}} & \mc{1}{c}{\scriptsize{(0.382)}} & \mc{1}{c}{\scriptsize{(0.263)}} & \mc{1}{c}{\scriptsize{\textbf{(0.066)}}} & \mc{1}{c}{\scriptsize{(0.368)}} & \mc{1}{c}{\scriptsize{(0.329)}} & \mc{1}{c}{\scriptsize{(0.342)}} & \mc{1}{c}{\scriptsize{(0.171)}} &  \\  

    \mc{1}{l}{\scriptsize{Crime}} & \mc{1}{c}{\scriptsize{33}} & \mc{1}{c}{\scriptsize{33}} & \mc{1}{c}{\scriptsize{67}} & \mc{1}{c}{\scriptsize{33}} & \mc{1}{c}{\scriptsize{33}} & \mc{1}{c}{\scriptsize{33}} & \mc{1}{c}{\scriptsize{33}} & \mc{1}{c}{\scriptsize{33}} & \mc{1}{c}{\scriptsize{3}} \\  

     & \mc{1}{c}{\scriptsize{(0.513)}} & \mc{1}{c}{\scriptsize{(0.868)}} & \mc{1}{c}{\scriptsize{(0.513)}} & \mc{1}{c}{\scriptsize{(0.882)}} & \mc{1}{c}{\scriptsize{(0.908)}} & \mc{1}{c}{\scriptsize{(0.461)}} & \mc{1}{c}{\scriptsize{(0.816)}} & \mc{1}{c}{\scriptsize{(0.842)}} &  \\  

    \mc{1}{l}{\scriptsize{Childhood Health}} & \mc{1}{c}{\scriptsize{71}} & \mc{1}{c}{\scriptsize{71}} & \mc{1}{c}{\scriptsize{57}} & \mc{1}{c}{\scriptsize{57}} & \mc{1}{c}{\scriptsize{57}} & \mc{1}{c}{\scriptsize{71}} & \mc{1}{c}{\scriptsize{64}} & \mc{1}{c}{\scriptsize{79}} & \mc{1}{c}{\scriptsize{14}} \\  

     & \mc{1}{c}{\scriptsize{\textbf{(0.013)}}} & \mc{1}{c}{\scriptsize{\textbf{(0.053)}}} & \mc{1}{c}{\scriptsize{(0.237)}} & \mc{1}{c}{\scriptsize{(0.237)}} & \mc{1}{c}{\scriptsize{(0.224)}} & \mc{1}{c}{\scriptsize{\textbf{(0.013)}}} & \mc{1}{c}{\scriptsize{\textbf{(0.053)}}} & \mc{1}{c}{\scriptsize{\textbf{(0.000)}}} &  \\  

    \mc{1}{l}{\scriptsize{Adult Health}} & \mc{1}{c}{\scriptsize{63}} & \mc{1}{c}{\scriptsize{60}} & \mc{1}{c}{\scriptsize{66}} & \mc{1}{c}{\scriptsize{61}} & \mc{1}{c}{\scriptsize{62}} & \mc{1}{c}{\scriptsize{54}} & \mc{1}{c}{\scriptsize{51}} & \mc{1}{c}{\scriptsize{51}} & \mc{1}{c}{\scriptsize{86}} \\  

     & \mc{1}{c}{\scriptsize{\textbf{(0.013)}}} & \mc{1}{c}{\scriptsize{\textbf{(0.013)}}} & \mc{1}{c}{\scriptsize{\textbf{(0.000)}}} & \mc{1}{c}{\scriptsize{\textbf{(0.013)}}} & \mc{1}{c}{\scriptsize{\textbf{(0.026)}}} & \mc{1}{c}{\scriptsize{(0.237)}} & \mc{1}{c}{\scriptsize{(0.447)}} & \mc{1}{c}{\scriptsize{(0.395)}} &  \\  

    \mc{1}{l}{\scriptsize{Mental Health}} & \mc{1}{c}{\scriptsize{64}} & \mc{1}{c}{\scriptsize{68}} & \mc{1}{c}{\scriptsize{62}} & \mc{1}{c}{\scriptsize{73}} & \mc{1}{c}{\scriptsize{64}} & \mc{1}{c}{\scriptsize{64}} & \mc{1}{c}{\scriptsize{75}} & \mc{1}{c}{\scriptsize{68}} & \mc{1}{c}{\scriptsize{56}} \\  

     & \mc{1}{c}{\scriptsize{(0.158)}} & \mc{1}{c}{\scriptsize{\textbf{(0.026)}}} & \mc{1}{c}{\scriptsize{(0.303)}} & \mc{1}{c}{\scriptsize{\textbf{(0.013)}}} & \mc{1}{c}{\scriptsize{(0.237)}} & \mc{1}{c}{\scriptsize{(0.105)}} & \mc{1}{c}{\scriptsize{\textbf{(0.000)}}} & \mc{1}{c}{\scriptsize{\textbf{(0.026)}}} &  \\  

  \bottomrule
  \end{tabular}

	\end{sidewaystable}   

	\begin{sidewaystable}[H]
     \caption{Combining Functions by Category $|$ 10\% Significance, Pooled Sample} 
     \label{table:abccare_rslt_pooled_counts_n10a10_all}
	  \begin{tabular}{cccccccccc}
  \toprule

    \scriptsize{Category} & \scriptsize{(1)} & \scriptsize{(2)} & \scriptsize{(3)} & \scriptsize{(4)} & \scriptsize{(5)} & \scriptsize{(6)} & \scriptsize{(7)} & \scriptsize{(8)} &  \\ 
    \midrule  

    \mc{1}{l}{\scriptsize{Cognitive Skills}} & \mc{1}{c}{\scriptsize{92}} & \mc{1}{c}{\scriptsize{88}} & \mc{1}{c}{\scriptsize{48}} & \mc{1}{c}{\scriptsize{48}} & \mc{1}{c}{\scriptsize{48}} & \mc{1}{c}{\scriptsize{88}} & \mc{1}{c}{\scriptsize{68}} & \mc{1}{c}{\scriptsize{88}} & \mc{1}{c}{\scriptsize{25}} \\  

     & \mc{1}{c}{\scriptsize{\textbf{(0.000)}}} & \mc{1}{c}{\scriptsize{\textbf{(0.000)}}} & \mc{1}{c}{\scriptsize{\textbf{(0.013)}}} & \mc{1}{c}{\scriptsize{\textbf{(0.000)}}} & \mc{1}{c}{\scriptsize{\textbf{(0.013)}}} & \mc{1}{c}{\scriptsize{\textbf{(0.000)}}} & \mc{1}{c}{\scriptsize{\textbf{(0.000)}}} & \mc{1}{c}{\scriptsize{\textbf{(0.000)}}} &  \\  

    \mc{1}{l}{\scriptsize{Noncognitive Skills}} & \mc{1}{c}{\scriptsize{20}} & \mc{1}{c}{\scriptsize{18}} & \mc{1}{c}{\scriptsize{11}} & \mc{1}{c}{\scriptsize{8}} & \mc{1}{c}{\scriptsize{11}} & \mc{1}{c}{\scriptsize{16}} & \mc{1}{c}{\scriptsize{15}} & \mc{1}{c}{\scriptsize{21}} & \mc{1}{c}{\scriptsize{117}} \\  

     & \mc{1}{c}{\scriptsize{\textbf{(0.053)}}} & \mc{1}{c}{\scriptsize{\textbf{(0.079)}}} & \mc{1}{c}{\scriptsize{(0.303)}} & \mc{1}{c}{\scriptsize{(0.618)}} & \mc{1}{c}{\scriptsize{(0.303)}} & \mc{1}{c}{\scriptsize{(0.105)}} & \mc{1}{c}{\scriptsize{(0.118)}} & \mc{1}{c}{\scriptsize{\textbf{(0.026)}}} &  \\  

    \mc{1}{l}{\scriptsize{Mother's Employment, Education, and Income}} & \mc{1}{c}{\scriptsize{0}} & \mc{1}{c}{\scriptsize{25}} & \mc{1}{c}{\scriptsize{25}} & \mc{1}{c}{\scriptsize{0}} & \mc{1}{c}{\scriptsize{25}} & \mc{1}{c}{\scriptsize{0}} & \mc{1}{c}{\scriptsize{25}} & \mc{1}{c}{\scriptsize{0}} & \mc{1}{c}{\scriptsize{4}} \\  

     & \mc{1}{c}{\scriptsize{(1.000)}} & \mc{1}{c}{\scriptsize{(0.368)}} & \mc{1}{c}{\scriptsize{\textbf{(0.079)}}} & \mc{1}{c}{\scriptsize{(0.395)}} & \mc{1}{c}{\scriptsize{\textbf{(0.053)}}} & \mc{1}{c}{\scriptsize{(1.000)}} & \mc{1}{c}{\scriptsize{\textbf{(0.039)}}} & \mc{1}{c}{\scriptsize{(0.461)}} &  \\  

    \mc{1}{l}{\scriptsize{Childhood Household Environment}} & \mc{1}{c}{\scriptsize{40}} & \mc{1}{c}{\scriptsize{7}} & \mc{1}{c}{\scriptsize{40}} & \mc{1}{c}{\scriptsize{27}} & \mc{1}{c}{\scriptsize{47}} & \mc{1}{c}{\scriptsize{27}} & \mc{1}{c}{\scriptsize{27}} & \mc{1}{c}{\scriptsize{40}} & \mc{1}{c}{\scriptsize{15}} \\  

     & \mc{1}{c}{\scriptsize{\textbf{(0.000)}}} & \mc{1}{c}{\scriptsize{(0.579)}} & \mc{1}{c}{\scriptsize{\textbf{(0.000)}}} & \mc{1}{c}{\scriptsize{\textbf{(0.039)}}} & \mc{1}{c}{\scriptsize{\textbf{(0.000)}}} & \mc{1}{c}{\scriptsize{\textbf{(0.092)}}} & \mc{1}{c}{\scriptsize{\textbf{(0.079)}}} & \mc{1}{c}{\scriptsize{\textbf{(0.053)}}} &  \\  

    \mc{1}{l}{\scriptsize{Adult Household Environment}} & \mc{1}{c}{\scriptsize{11}} & \mc{1}{c}{\scriptsize{22}} & \mc{1}{c}{\scriptsize{22}} & \mc{1}{c}{\scriptsize{11}} & \mc{1}{c}{\scriptsize{22}} & \mc{1}{c}{\scriptsize{11}} & \mc{1}{c}{\scriptsize{0}} & \mc{1}{c}{\scriptsize{0}} & \mc{1}{c}{\scriptsize{9}} \\  

     & \mc{1}{c}{\scriptsize{(0.434)}} & \mc{1}{c}{\scriptsize{\textbf{(0.039)}}} & \mc{1}{c}{\scriptsize{(0.211)}} & \mc{1}{c}{\scriptsize{(0.224)}} & \mc{1}{c}{\scriptsize{(0.197)}} & \mc{1}{c}{\scriptsize{(0.355)}} & \mc{1}{c}{\scriptsize{(1.000)}} & \mc{1}{c}{\scriptsize{(0.645)}} &  \\  

    \mc{1}{l}{\scriptsize{Education, Employment, Income}} & \mc{1}{c}{\scriptsize{21}} & \mc{1}{c}{\scriptsize{18}} & \mc{1}{c}{\scriptsize{22}} & \mc{1}{c}{\scriptsize{14}} & \mc{1}{c}{\scriptsize{29}} & \mc{1}{c}{\scriptsize{21}} & \mc{1}{c}{\scriptsize{14}} & \mc{1}{c}{\scriptsize{21}} & \mc{1}{c}{\scriptsize{28}} \\  

     & \mc{1}{c}{\scriptsize{\textbf{(0.026)}}} & \mc{1}{c}{\scriptsize{(0.145)}} & \mc{1}{c}{\scriptsize{\textbf{(0.079)}}} & \mc{1}{c}{\scriptsize{(0.132)}} & \mc{1}{c}{\scriptsize{\textbf{(0.079)}}} & \mc{1}{c}{\scriptsize{\textbf{(0.039)}}} & \mc{1}{c}{\scriptsize{(0.250)}} & \mc{1}{c}{\scriptsize{(0.132)}} &  \\  

    \mc{1}{l}{\scriptsize{Crime}} & \mc{1}{c}{\scriptsize{33}} & \mc{1}{c}{\scriptsize{0}} & \mc{1}{c}{\scriptsize{33}} & \mc{1}{c}{\scriptsize{0}} & \mc{1}{c}{\scriptsize{0}} & \mc{1}{c}{\scriptsize{0}} & \mc{1}{c}{\scriptsize{0}} & \mc{1}{c}{\scriptsize{0}} & \mc{1}{c}{\scriptsize{3}} \\  

     & \mc{1}{c}{\scriptsize{\textbf{(0.039)}}} & \mc{1}{c}{\scriptsize{(0.303)}} & \mc{1}{c}{\scriptsize{(0.250)}} & \mc{1}{c}{\scriptsize{(1.000)}} & \mc{1}{c}{\scriptsize{(1.000)}} & \mc{1}{c}{\scriptsize{(0.316)}} & \mc{1}{c}{\scriptsize{(1.000)}} & \mc{1}{c}{\scriptsize{(0.303)}} &  \\  

    \mc{1}{l}{\scriptsize{Childhood Health}} & \mc{1}{c}{\scriptsize{43}} & \mc{1}{c}{\scriptsize{36}} & \mc{1}{c}{\scriptsize{36}} & \mc{1}{c}{\scriptsize{21}} & \mc{1}{c}{\scriptsize{43}} & \mc{1}{c}{\scriptsize{36}} & \mc{1}{c}{\scriptsize{36}} & \mc{1}{c}{\scriptsize{36}} & \mc{1}{c}{\scriptsize{14}} \\  

     & \mc{1}{c}{\scriptsize{\textbf{(0.000)}}} & \mc{1}{c}{\scriptsize{\textbf{(0.013)}}} & \mc{1}{c}{\scriptsize{\textbf{(0.066)}}} & \mc{1}{c}{\scriptsize{(0.132)}} & \mc{1}{c}{\scriptsize{\textbf{(0.000)}}} & \mc{1}{c}{\scriptsize{\textbf{(0.000)}}} & \mc{1}{c}{\scriptsize{\textbf{(0.013)}}} & \mc{1}{c}{\scriptsize{\textbf{(0.000)}}} &  \\  

    \mc{1}{l}{\scriptsize{Adult Health}} & \mc{1}{c}{\scriptsize{23}} & \mc{1}{c}{\scriptsize{14}} & \mc{1}{c}{\scriptsize{15}} & \mc{1}{c}{\scriptsize{15}} & \mc{1}{c}{\scriptsize{18}} & \mc{1}{c}{\scriptsize{17}} & \mc{1}{c}{\scriptsize{13}} & \mc{1}{c}{\scriptsize{16}} & \mc{1}{c}{\scriptsize{86}} \\  

     & \mc{1}{c}{\scriptsize{\textbf{(0.000)}}} & \mc{1}{c}{\scriptsize{(0.211)}} & \mc{1}{c}{\scriptsize{(0.171)}} & \mc{1}{c}{\scriptsize{(0.118)}} & \mc{1}{c}{\scriptsize{\textbf{(0.066)}}} & \mc{1}{c}{\scriptsize{\textbf{(0.066)}}} & \mc{1}{c}{\scriptsize{(0.250)}} & \mc{1}{c}{\scriptsize{\textbf{(0.079)}}} &  \\  

    \mc{1}{l}{\scriptsize{Mental Health}} & \mc{1}{c}{\scriptsize{27}} & \mc{1}{c}{\scriptsize{27}} & \mc{1}{c}{\scriptsize{14}} & \mc{1}{c}{\scriptsize{20}} & \mc{1}{c}{\scriptsize{18}} & \mc{1}{c}{\scriptsize{25}} & \mc{1}{c}{\scriptsize{29}} & \mc{1}{c}{\scriptsize{34}} & \mc{1}{c}{\scriptsize{56}} \\  

     & \mc{1}{c}{\scriptsize{\textbf{(0.079)}}} & \mc{1}{c}{\scriptsize{(0.132)}} & \mc{1}{c}{\scriptsize{(0.289)}} & \mc{1}{c}{\scriptsize{(0.211)}} & \mc{1}{c}{\scriptsize{(0.224)}} & \mc{1}{c}{\scriptsize{(0.132)}} & \mc{1}{c}{\scriptsize{\textbf{(0.092)}}} & \mc{1}{c}{\scriptsize{\textbf{(0.000)}}} &  \\  

  \bottomrule
  \end{tabular}
	\end{sidewaystable}   

	\begin{sidewaystable}[H]
     \caption{Combining Functions by Category, Male Sample} 
     \label{table:abccare_rslt_male_counts_n50a100_all}
	  \begin{tabular}{cccccccccc}
  \toprule

    \scriptsize{Category} & \scriptsize{(1)} & \scriptsize{(2)} & \scriptsize{(3)} & \scriptsize{(4)} & \scriptsize{(5)} & \scriptsize{(6)} & \scriptsize{(7)} & \scriptsize{(8)} &  \\ 
    \midrule  

    \mc{1}{l}{\scriptsize{Cognitive Skills}} & \mc{1}{c}{\scriptsize{96}} & \mc{1}{c}{\scriptsize{84}} & \mc{1}{c}{\scriptsize{64}} & \mc{1}{c}{\scriptsize{60}} & \mc{1}{c}{\scriptsize{52}} & \mc{1}{c}{\scriptsize{96}} & \mc{1}{c}{\scriptsize{84}} & \mc{1}{c}{\scriptsize{84}} & \mc{1}{c}{\scriptsize{25}} \\  

     & \mc{1}{c}{\scriptsize{\textbf{(0.000)}}} & \mc{1}{c}{\scriptsize{\textbf{(0.000)}}} & \mc{1}{c}{\scriptsize{(0.303)}} & \mc{1}{c}{\scriptsize{(0.382)}} & \mc{1}{c}{\scriptsize{(0.474)}} & \mc{1}{c}{\scriptsize{\textbf{(0.000)}}} & \mc{1}{c}{\scriptsize{\textbf{(0.000)}}} & \mc{1}{c}{\scriptsize{\textbf{(0.000)}}} &  \\  

    \mc{1}{l}{\scriptsize{Noncognitive Skills}} & \mc{1}{c}{\scriptsize{42}} & \mc{1}{c}{\scriptsize{45}} & \mc{1}{c}{\scriptsize{28}} & \mc{1}{c}{\scriptsize{35}} & \mc{1}{c}{\scriptsize{34}} & \mc{1}{c}{\scriptsize{48}} & \mc{1}{c}{\scriptsize{50}} & \mc{1}{c}{\scriptsize{50}} & \mc{1}{c}{\scriptsize{117}} \\  

     & \mc{1}{c}{\scriptsize{(0.908)}} & \mc{1}{c}{\scriptsize{(0.645)}} & \mc{1}{c}{\scriptsize{(1.000)}} & \mc{1}{c}{\scriptsize{(0.934)}} & \mc{1}{c}{\scriptsize{(0.987)}} & \mc{1}{c}{\scriptsize{(0.579)}} & \mc{1}{c}{\scriptsize{(0.461)}} & \mc{1}{c}{\scriptsize{(0.539)}} &  \\  

    \mc{1}{l}{\scriptsize{Mother's Employment, Education, and Income}} & \mc{1}{c}{\scriptsize{50}} & \mc{1}{c}{\scriptsize{50}} & \mc{1}{c}{\scriptsize{50}} & \mc{1}{c}{\scriptsize{100}} & \mc{1}{c}{\scriptsize{50}} & \mc{1}{c}{\scriptsize{25}} & \mc{1}{c}{\scriptsize{50}} & \mc{1}{c}{\scriptsize{50}} & \mc{1}{c}{\scriptsize{4}} \\  

     & \mc{1}{c}{\scriptsize{(0.513)}} & \mc{1}{c}{\scriptsize{(0.658)}} & \mc{1}{c}{\scriptsize{(0.368)}} & \mc{1}{c}{\scriptsize{\textbf{(0.000)}}} & \mc{1}{c}{\scriptsize{(0.368)}} & \mc{1}{c}{\scriptsize{(0.961)}} & \mc{1}{c}{\scriptsize{(0.605)}} & \mc{1}{c}{\scriptsize{(0.632)}} &  \\  

    \mc{1}{l}{\scriptsize{Childhood Household Environment}} & \mc{1}{c}{\scriptsize{60}} & \mc{1}{c}{\scriptsize{87}} & \mc{1}{c}{\scriptsize{47}} & \mc{1}{c}{\scriptsize{53}} & \mc{1}{c}{\scriptsize{47}} & \mc{1}{c}{\scriptsize{71}} & \mc{1}{c}{\scriptsize{87}} & \mc{1}{c}{\scriptsize{80}} & \mc{1}{c}{\scriptsize{15}} \\  

     & \mc{1}{c}{\scriptsize{(0.329)}} & \mc{1}{c}{\scriptsize{\textbf{(0.000)}}} & \mc{1}{c}{\scriptsize{(0.605)}} & \mc{1}{c}{\scriptsize{(0.421)}} & \mc{1}{c}{\scriptsize{(0.579)}} & \mc{1}{c}{\scriptsize{(0.184)}} & \mc{1}{c}{\scriptsize{\textbf{(0.000)}}} & \mc{1}{c}{\scriptsize{\textbf{(0.039)}}} &  \\  

    \mc{1}{l}{\scriptsize{Adult Household Environment}} & \mc{1}{c}{\scriptsize{56}} & \mc{1}{c}{\scriptsize{44}} & \mc{1}{c}{\scriptsize{38}} & \mc{1}{c}{\scriptsize{33}} & \mc{1}{c}{\scriptsize{22}} & \mc{1}{c}{\scriptsize{56}} & \mc{1}{c}{\scriptsize{56}} & \mc{1}{c}{\scriptsize{44}} & \mc{1}{c}{\scriptsize{9}} \\  

     & \mc{1}{c}{\scriptsize{(0.395)}} & \mc{1}{c}{\scriptsize{(0.724)}} & \mc{1}{c}{\scriptsize{(0.618)}} & \mc{1}{c}{\scriptsize{(0.803)}} & \mc{1}{c}{\scriptsize{(1.000)}} & \mc{1}{c}{\scriptsize{(0.276)}} & \mc{1}{c}{\scriptsize{(0.237)}} & \mc{1}{c}{\scriptsize{(0.592)}} &  \\  

    \mc{1}{l}{\scriptsize{Education, Employment, Income}} & \mc{1}{c}{\scriptsize{32}} & \mc{1}{c}{\scriptsize{36}} & \mc{1}{c}{\scriptsize{32}} & \mc{1}{c}{\scriptsize{36}} & \mc{1}{c}{\scriptsize{36}} & \mc{1}{c}{\scriptsize{43}} & \mc{1}{c}{\scriptsize{32}} & \mc{1}{c}{\scriptsize{36}} & \mc{1}{c}{\scriptsize{28}} \\  

     & \mc{1}{c}{\scriptsize{(0.987)}} & \mc{1}{c}{\scriptsize{(0.934)}} & \mc{1}{c}{\scriptsize{(0.974)}} & \mc{1}{c}{\scriptsize{(0.934)}} & \mc{1}{c}{\scriptsize{(0.947)}} & \mc{1}{c}{\scriptsize{(0.750)}} & \mc{1}{c}{\scriptsize{(0.974)}} & \mc{1}{c}{\scriptsize{(0.947)}} &  \\  

    \mc{1}{l}{\scriptsize{Crime}} & \mc{1}{c}{\scriptsize{33}} & \mc{1}{c}{\scriptsize{33}} & \mc{1}{c}{\scriptsize{33}} & \mc{1}{c}{\scriptsize{0}} & \mc{1}{c}{\scriptsize{33}} & \mc{1}{c}{\scriptsize{33}} & \mc{1}{c}{\scriptsize{33}} & \mc{1}{c}{\scriptsize{33}} & \mc{1}{c}{\scriptsize{3}} \\  

     & \mc{1}{c}{\scriptsize{(0.882)}} & \mc{1}{c}{\scriptsize{(0.724)}} & \mc{1}{c}{\scriptsize{(0.513)}} & \mc{1}{c}{\scriptsize{(0.987)}} & \mc{1}{c}{\scriptsize{(0.987)}} & \mc{1}{c}{\scriptsize{(0.500)}} & \mc{1}{c}{\scriptsize{(0.789)}} & \mc{1}{c}{\scriptsize{(0.921)}} &  \\  

    \mc{1}{l}{\scriptsize{Childhood Health}} & \mc{1}{c}{\scriptsize{79}} & \mc{1}{c}{\scriptsize{79}} & \mc{1}{c}{\scriptsize{77}} & \mc{1}{c}{\scriptsize{69}} & \mc{1}{c}{\scriptsize{77}} & \mc{1}{c}{\scriptsize{86}} & \mc{1}{c}{\scriptsize{79}} & \mc{1}{c}{\scriptsize{86}} & \mc{1}{c}{\scriptsize{14}} \\  

     & \mc{1}{c}{\scriptsize{\textbf{(0.000)}}} & \mc{1}{c}{\scriptsize{\textbf{(0.039)}}} & \mc{1}{c}{\scriptsize{\textbf{(0.000)}}} & \mc{1}{c}{\scriptsize{(0.132)}} & \mc{1}{c}{\scriptsize{\textbf{(0.000)}}} & \mc{1}{c}{\scriptsize{\textbf{(0.000)}}} & \mc{1}{c}{\scriptsize{\textbf{(0.026)}}} & \mc{1}{c}{\scriptsize{\textbf{(0.000)}}} &  \\  

    \mc{1}{l}{\scriptsize{Adult Health}} & \mc{1}{c}{\scriptsize{68}} & \mc{1}{c}{\scriptsize{62}} & \mc{1}{c}{\scriptsize{58}} & \mc{1}{c}{\scriptsize{62}} & \mc{1}{c}{\scriptsize{58}} & \mc{1}{c}{\scriptsize{59}} & \mc{1}{c}{\scriptsize{62}} & \mc{1}{c}{\scriptsize{59}} & \mc{1}{c}{\scriptsize{74}} \\  

     & \mc{1}{c}{\scriptsize{\textbf{(0.000)}}} & \mc{1}{c}{\scriptsize{\textbf{(0.013)}}} & \mc{1}{c}{\scriptsize{(0.237)}} & \mc{1}{c}{\scriptsize{\textbf{(0.066)}}} & \mc{1}{c}{\scriptsize{(0.237)}} & \mc{1}{c}{\scriptsize{\textbf{(0.039)}}} & \mc{1}{c}{\scriptsize{\textbf{(0.000)}}} & \mc{1}{c}{\scriptsize{\textbf{(0.039)}}} &  \\  

    \mc{1}{l}{\scriptsize{Mental Health}} & \mc{1}{c}{\scriptsize{41}} & \mc{1}{c}{\scriptsize{52}} & \mc{1}{c}{\scriptsize{17}} & \mc{1}{c}{\scriptsize{37}} & \mc{1}{c}{\scriptsize{22}} & \mc{1}{c}{\scriptsize{46}} & \mc{1}{c}{\scriptsize{56}} & \mc{1}{c}{\scriptsize{52}} & \mc{1}{c}{\scriptsize{54}} \\  

     & \mc{1}{c}{\scriptsize{(0.750)}} & \mc{1}{c}{\scriptsize{(0.487)}} & \mc{1}{c}{\scriptsize{(1.000)}} & \mc{1}{c}{\scriptsize{(0.789)}} & \mc{1}{c}{\scriptsize{(1.000)}} & \mc{1}{c}{\scriptsize{(0.645)}} & \mc{1}{c}{\scriptsize{(0.408)}} & \mc{1}{c}{\scriptsize{(0.553)}} &  \\  

  \bottomrule
  \end{tabular}
	\end{sidewaystable}   

	\begin{sidewaystable}[H]
     \caption{Combining Functions by Category $|$ 10\% Significance, Male Sample} 
     \label{table:abccare_rslt_male_counts_n10a10_all}
	  \begin{tabular}{cccccccccc}
  \toprule

    \scriptsize{Category} & \scriptsize{(1)} & \scriptsize{(2)} & \scriptsize{(3)} & \scriptsize{(4)} & \scriptsize{(5)} & \scriptsize{(6)} & \scriptsize{(7)} & \scriptsize{(8)} &  \\ 
    \midrule  

    \mc{1}{l}{\scriptsize{Cognitive Skills}} & \mc{1}{c}{\scriptsize{48}} & \mc{1}{c}{\scriptsize{48}} & \mc{1}{c}{\scriptsize{24}} & \mc{1}{c}{\scriptsize{24}} & \mc{1}{c}{\scriptsize{24}} & \mc{1}{c}{\scriptsize{60}} & \mc{1}{c}{\scriptsize{44}} & \mc{1}{c}{\scriptsize{56}} & \mc{1}{c}{\scriptsize{25}} \\  

     & \mc{1}{c}{\scriptsize{\textbf{(0.039)}}} & \mc{1}{c}{\scriptsize{\textbf{(0.013)}}} & \mc{1}{c}{\scriptsize{(0.237)}} & \mc{1}{c}{\scriptsize{(0.158)}} & \mc{1}{c}{\scriptsize{(0.197)}} & \mc{1}{c}{\scriptsize{\textbf{(0.000)}}} & \mc{1}{c}{\scriptsize{\textbf{(0.000)}}} & \mc{1}{c}{\scriptsize{\textbf{(0.000)}}} &  \\  

    \mc{1}{l}{\scriptsize{Noncognitive Skills}} & \mc{1}{c}{\scriptsize{9}} & \mc{1}{c}{\scriptsize{8}} & \mc{1}{c}{\scriptsize{5}} & \mc{1}{c}{\scriptsize{4}} & \mc{1}{c}{\scriptsize{5}} & \mc{1}{c}{\scriptsize{10}} & \mc{1}{c}{\scriptsize{9}} & \mc{1}{c}{\scriptsize{12}} & \mc{1}{c}{\scriptsize{117}} \\  

     & \mc{1}{c}{\scriptsize{(0.487)}} & \mc{1}{c}{\scriptsize{(0.684)}} & \mc{1}{c}{\scriptsize{(0.908)}} & \mc{1}{c}{\scriptsize{(0.987)}} & \mc{1}{c}{\scriptsize{(0.895)}} & \mc{1}{c}{\scriptsize{(0.342)}} & \mc{1}{c}{\scriptsize{(0.447)}} & \mc{1}{c}{\scriptsize{(0.316)}} &  \\  

    \mc{1}{l}{\scriptsize{Mother's Employment, Education, and Income}} & \mc{1}{c}{\scriptsize{0}} & \mc{1}{c}{\scriptsize{0}} & \mc{1}{c}{\scriptsize{25}} & \mc{1}{c}{\scriptsize{25}} & \mc{1}{c}{\scriptsize{25}} & \mc{1}{c}{\scriptsize{0}} & \mc{1}{c}{\scriptsize{0}} & \mc{1}{c}{\scriptsize{0}} & \mc{1}{c}{\scriptsize{4}} \\  

     & \mc{1}{c}{\scriptsize{(0.987)}} & \mc{1}{c}{\scriptsize{(0.987)}} & \mc{1}{c}{\scriptsize{(0.250)}} & \mc{1}{c}{\scriptsize{(0.237)}} & \mc{1}{c}{\scriptsize{(0.237)}} & \mc{1}{c}{\scriptsize{(0.974)}} & \mc{1}{c}{\scriptsize{(0.974)}} & \mc{1}{c}{\scriptsize{(0.974)}} &  \\  

    \mc{1}{l}{\scriptsize{Childhood Household Environment}} & \mc{1}{c}{\scriptsize{7}} & \mc{1}{c}{\scriptsize{13}} & \mc{1}{c}{\scriptsize{7}} & \mc{1}{c}{\scriptsize{13}} & \mc{1}{c}{\scriptsize{7}} & \mc{1}{c}{\scriptsize{14}} & \mc{1}{c}{\scriptsize{20}} & \mc{1}{c}{\scriptsize{27}} & \mc{1}{c}{\scriptsize{15}} \\  

     & \mc{1}{c}{\scriptsize{(0.658)}} & \mc{1}{c}{\scriptsize{(0.434)}} & \mc{1}{c}{\scriptsize{(0.697)}} & \mc{1}{c}{\scriptsize{(0.316)}} & \mc{1}{c}{\scriptsize{(0.618)}} & \mc{1}{c}{\scriptsize{(0.224)}} & \mc{1}{c}{\scriptsize{(0.250)}} & \mc{1}{c}{\scriptsize{(0.197)}} &  \\  

    \mc{1}{l}{\scriptsize{Adult Household Environment}} & \mc{1}{c}{\scriptsize{0}} & \mc{1}{c}{\scriptsize{0}} & \mc{1}{c}{\scriptsize{12}} & \mc{1}{c}{\scriptsize{0}} & \mc{1}{c}{\scriptsize{11}} & \mc{1}{c}{\scriptsize{0}} & \mc{1}{c}{\scriptsize{0}} & \mc{1}{c}{\scriptsize{0}} & \mc{1}{c}{\scriptsize{9}} \\  

     & \mc{1}{c}{\scriptsize{(1.000)}} & \mc{1}{c}{\scriptsize{(1.000)}} & \mc{1}{c}{\scriptsize{(0.197)}} & \mc{1}{c}{\scriptsize{(0.697)}} & \mc{1}{c}{\scriptsize{(0.408)}} & \mc{1}{c}{\scriptsize{(0.618)}} & \mc{1}{c}{\scriptsize{(1.000)}} & \mc{1}{c}{\scriptsize{(1.000)}} &  \\  

    \mc{1}{l}{\scriptsize{Education, Employment, Income}} & \mc{1}{c}{\scriptsize{11}} & \mc{1}{c}{\scriptsize{7}} & \mc{1}{c}{\scriptsize{0}} & \mc{1}{c}{\scriptsize{4}} & \mc{1}{c}{\scriptsize{0}} & \mc{1}{c}{\scriptsize{11}} & \mc{1}{c}{\scriptsize{4}} & \mc{1}{c}{\scriptsize{7}} & \mc{1}{c}{\scriptsize{28}} \\  

     & \mc{1}{c}{\scriptsize{(0.408)}} & \mc{1}{c}{\scriptsize{(0.632)}} & \mc{1}{c}{\scriptsize{(0.882)}} & \mc{1}{c}{\scriptsize{(1.000)}} & \mc{1}{c}{\scriptsize{(1.000)}} & \mc{1}{c}{\scriptsize{(0.421)}} & \mc{1}{c}{\scriptsize{(0.908)}} & \mc{1}{c}{\scriptsize{(0.592)}} &  \\  

    \mc{1}{l}{\scriptsize{Crime}} & \mc{1}{c}{\scriptsize{0}} & \mc{1}{c}{\scriptsize{0}} & \mc{1}{c}{\scriptsize{0}} & \mc{1}{c}{\scriptsize{0}} & \mc{1}{c}{\scriptsize{0}} & \mc{1}{c}{\scriptsize{0}} & \mc{1}{c}{\scriptsize{0}} & \mc{1}{c}{\scriptsize{0}} & \mc{1}{c}{\scriptsize{3}} \\  

     & \mc{1}{c}{\scriptsize{(1.000)}} & \mc{1}{c}{\scriptsize{(1.000)}} & \mc{1}{c}{\scriptsize{(0.987)}} & \mc{1}{c}{\scriptsize{(0.987)}} & \mc{1}{c}{\scriptsize{(0.987)}} & \mc{1}{c}{\scriptsize{(0.276)}} & \mc{1}{c}{\scriptsize{(1.000)}} & \mc{1}{c}{\scriptsize{(0.316)}} &  \\  

    \mc{1}{l}{\scriptsize{Childhood Health}} & \mc{1}{c}{\scriptsize{29}} & \mc{1}{c}{\scriptsize{29}} & \mc{1}{c}{\scriptsize{46}} & \mc{1}{c}{\scriptsize{31}} & \mc{1}{c}{\scriptsize{31}} & \mc{1}{c}{\scriptsize{21}} & \mc{1}{c}{\scriptsize{14}} & \mc{1}{c}{\scriptsize{21}} & \mc{1}{c}{\scriptsize{14}} \\  

     & \mc{1}{c}{\scriptsize{(0.184)}} & \mc{1}{c}{\scriptsize{(0.145)}} & \mc{1}{c}{\scriptsize{\textbf{(0.000)}}} & \mc{1}{c}{\scriptsize{\textbf{(0.066)}}} & \mc{1}{c}{\scriptsize{\textbf{(0.079)}}} & \mc{1}{c}{\scriptsize{(0.276)}} & \mc{1}{c}{\scriptsize{(0.276)}} & \mc{1}{c}{\scriptsize{(0.263)}} &  \\  

    \mc{1}{l}{\scriptsize{Adult Health}} & \mc{1}{c}{\scriptsize{27}} & \mc{1}{c}{\scriptsize{31}} & \mc{1}{c}{\scriptsize{12}} & \mc{1}{c}{\scriptsize{14}} & \mc{1}{c}{\scriptsize{10}} & \mc{1}{c}{\scriptsize{31}} & \mc{1}{c}{\scriptsize{27}} & \mc{1}{c}{\scriptsize{26}} & \mc{1}{c}{\scriptsize{74}} \\  

     & \mc{1}{c}{\scriptsize{\textbf{(0.000)}}} & \mc{1}{c}{\scriptsize{\textbf{(0.000)}}} & \mc{1}{c}{\scriptsize{(0.368)}} & \mc{1}{c}{\scriptsize{(0.224)}} & \mc{1}{c}{\scriptsize{(0.461)}} & \mc{1}{c}{\scriptsize{\textbf{(0.000)}}} & \mc{1}{c}{\scriptsize{\textbf{(0.000)}}} & \mc{1}{c}{\scriptsize{\textbf{(0.000)}}} &  \\  

    \mc{1}{l}{\scriptsize{Mental Health}} & \mc{1}{c}{\scriptsize{0}} & \mc{1}{c}{\scriptsize{7}} & \mc{1}{c}{\scriptsize{0}} & \mc{1}{c}{\scriptsize{0}} & \mc{1}{c}{\scriptsize{0}} & \mc{1}{c}{\scriptsize{2}} & \mc{1}{c}{\scriptsize{2}} & \mc{1}{c}{\scriptsize{2}} & \mc{1}{c}{\scriptsize{54}} \\  

     & \mc{1}{c}{\scriptsize{(1.000)}} & \mc{1}{c}{\scriptsize{(0.474)}} & \mc{1}{c}{\scriptsize{(1.000)}} & \mc{1}{c}{\scriptsize{(1.000)}} & \mc{1}{c}{\scriptsize{(1.000)}} & \mc{1}{c}{\scriptsize{(1.000)}} & \mc{1}{c}{\scriptsize{(1.000)}} & \mc{1}{c}{\scriptsize{(1.000)}} &  \\  

  \bottomrule
  \end{tabular}
	\end{sidewaystable}   

	\begin{sidewaystable}[H]
     \caption{Combining Functions by Category, Female Sample} 
     \label{table:abccare_rslt_female_counts_n50a100_all}
	  \begin{tabular}{cccccccccc}
  \toprule

    \scriptsize{Category} & \scriptsize{(1)} & \scriptsize{(2)} & \scriptsize{(3)} & \scriptsize{(4)} & \scriptsize{(5)} & \scriptsize{(6)} & \scriptsize{(7)} & \scriptsize{(8)} &  \\ 
    \midrule  

    \mc{1}{l}{\scriptsize{Cognitive Skills}} & \mc{1}{c}{\scriptsize{100}} & \mc{1}{c}{\scriptsize{96}} & \mc{1}{c}{\scriptsize{100}} & \mc{1}{c}{\scriptsize{100}} & \mc{1}{c}{\scriptsize{100}} & \mc{1}{c}{\scriptsize{100}} & \mc{1}{c}{\scriptsize{92}} & \mc{1}{c}{\scriptsize{100}} & \mc{1}{c}{\scriptsize{25}} \\  

     & \mc{1}{c}{\scriptsize{\textbf{(0.000)}}} & \mc{1}{c}{\scriptsize{\textbf{(0.000)}}} & \mc{1}{c}{\scriptsize{\textbf{(0.000)}}} & \mc{1}{c}{\scriptsize{\textbf{(0.000)}}} & \mc{1}{c}{\scriptsize{\textbf{(0.000)}}} & \mc{1}{c}{\scriptsize{\textbf{(0.000)}}} & \mc{1}{c}{\scriptsize{\textbf{(0.000)}}} & \mc{1}{c}{\scriptsize{\textbf{(0.000)}}} &  \\  

    \mc{1}{l}{\scriptsize{Noncognitive Skills}} & \mc{1}{c}{\scriptsize{70}} & \mc{1}{c}{\scriptsize{70}} & \mc{1}{c}{\scriptsize{74}} & \mc{1}{c}{\scriptsize{74}} & \mc{1}{c}{\scriptsize{77}} & \mc{1}{c}{\scriptsize{57}} & \mc{1}{c}{\scriptsize{61}} & \mc{1}{c}{\scriptsize{66}} & \mc{1}{c}{\scriptsize{117}} \\  

     & \mc{1}{c}{\scriptsize{\textbf{(0.000)}}} & \mc{1}{c}{\scriptsize{\textbf{(0.000)}}} & \mc{1}{c}{\scriptsize{\textbf{(0.000)}}} & \mc{1}{c}{\scriptsize{\textbf{(0.000)}}} & \mc{1}{c}{\scriptsize{\textbf{(0.000)}}} & \mc{1}{c}{\scriptsize{(0.237)}} & \mc{1}{c}{\scriptsize{(0.105)}} & \mc{1}{c}{\scriptsize{\textbf{(0.013)}}} &  \\  

    \mc{1}{l}{\scriptsize{Mother's Employment, Education, and Income}} & \mc{1}{c}{\scriptsize{50}} & \mc{1}{c}{\scriptsize{25}} & \mc{1}{c}{\scriptsize{50}} & \mc{1}{c}{\scriptsize{25}} & \mc{1}{c}{\scriptsize{50}} & \mc{1}{c}{\scriptsize{50}} & \mc{1}{c}{\scriptsize{25}} & \mc{1}{c}{\scriptsize{50}} & \mc{1}{c}{\scriptsize{4}} \\  

     & \mc{1}{c}{\scriptsize{(0.842)}} & \mc{1}{c}{\scriptsize{(0.974)}} & \mc{1}{c}{\scriptsize{(0.724)}} & \mc{1}{c}{\scriptsize{(0.750)}} & \mc{1}{c}{\scriptsize{(0.645)}} & \mc{1}{c}{\scriptsize{(0.224)}} & \mc{1}{c}{\scriptsize{(0.803)}} & \mc{1}{c}{\scriptsize{(0.250)}} &  \\  

    \mc{1}{l}{\scriptsize{Childhood Household Environment}} & \mc{1}{c}{\scriptsize{73}} & \mc{1}{c}{\scriptsize{80}} & \mc{1}{c}{\scriptsize{60}} & \mc{1}{c}{\scriptsize{67}} & \mc{1}{c}{\scriptsize{60}} & \mc{1}{c}{\scriptsize{64}} & \mc{1}{c}{\scriptsize{64}} & \mc{1}{c}{\scriptsize{79}} & \mc{1}{c}{\scriptsize{15}} \\  

     & \mc{1}{c}{\scriptsize{\textbf{(0.039)}}} & \mc{1}{c}{\scriptsize{\textbf{(0.026)}}} & \mc{1}{c}{\scriptsize{\textbf{(0.066)}}} & \mc{1}{c}{\scriptsize{\textbf{(0.053)}}} & \mc{1}{c}{\scriptsize{(0.132)}} & \mc{1}{c}{\scriptsize{(0.263)}} & \mc{1}{c}{\scriptsize{(0.145)}} & \mc{1}{c}{\scriptsize{\textbf{(0.053)}}} &  \\  

    \mc{1}{l}{\scriptsize{Adult Household Environment}} & \mc{1}{c}{\scriptsize{78}} & \mc{1}{c}{\scriptsize{67}} & \mc{1}{c}{\scriptsize{89}} & \mc{1}{c}{\scriptsize{78}} & \mc{1}{c}{\scriptsize{89}} & \mc{1}{c}{\scriptsize{89}} & \mc{1}{c}{\scriptsize{56}} & \mc{1}{c}{\scriptsize{56}} & \mc{1}{c}{\scriptsize{9}} \\  

     & \mc{1}{c}{\scriptsize{\textbf{(0.000)}}} & \mc{1}{c}{\scriptsize{\textbf{(0.092)}}} & \mc{1}{c}{\scriptsize{\textbf{(0.000)}}} & \mc{1}{c}{\scriptsize{\textbf{(0.053)}}} & \mc{1}{c}{\scriptsize{\textbf{(0.000)}}} & \mc{1}{c}{\scriptsize{\textbf{(0.000)}}} & \mc{1}{c}{\scriptsize{(0.303)}} & \mc{1}{c}{\scriptsize{(0.289)}} &  \\  

    \mc{1}{l}{\scriptsize{Education, Employment, Income}} & \mc{1}{c}{\scriptsize{71}} & \mc{1}{c}{\scriptsize{89}} & \mc{1}{c}{\scriptsize{74}} & \mc{1}{c}{\scriptsize{78}} & \mc{1}{c}{\scriptsize{74}} & \mc{1}{c}{\scriptsize{71}} & \mc{1}{c}{\scriptsize{75}} & \mc{1}{c}{\scriptsize{71}} & \mc{1}{c}{\scriptsize{28}} \\  

     & \mc{1}{c}{\scriptsize{\textbf{(0.013)}}} & \mc{1}{c}{\scriptsize{\textbf{(0.000)}}} & \mc{1}{c}{\scriptsize{\textbf{(0.000)}}} & \mc{1}{c}{\scriptsize{\textbf{(0.000)}}} & \mc{1}{c}{\scriptsize{\textbf{(0.000)}}} & \mc{1}{c}{\scriptsize{\textbf{(0.026)}}} & \mc{1}{c}{\scriptsize{\textbf{(0.000)}}} & \mc{1}{c}{\scriptsize{\textbf{(0.000)}}} &  \\  

    \mc{1}{l}{\scriptsize{Crime}} & \mc{1}{c}{\scriptsize{100}} & \mc{1}{c}{\scriptsize{100}} & \mc{1}{c}{\scriptsize{100}} & \mc{1}{c}{\scriptsize{100}} & \mc{1}{c}{\scriptsize{100}} & \mc{1}{c}{\scriptsize{100}} & \mc{1}{c}{\scriptsize{100}} & \mc{1}{c}{\scriptsize{67}} & \mc{1}{c}{\scriptsize{3}} \\  

     & \mc{1}{c}{\scriptsize{\textbf{(0.000)}}} & \mc{1}{c}{\scriptsize{\textbf{(0.000)}}} & \mc{1}{c}{\scriptsize{\textbf{(0.000)}}} & \mc{1}{c}{\scriptsize{\textbf{(0.000)}}} & \mc{1}{c}{\scriptsize{\textbf{(0.000)}}} & \mc{1}{c}{\scriptsize{\textbf{(0.000)}}} & \mc{1}{c}{\scriptsize{\textbf{(0.000)}}} & \mc{1}{c}{\scriptsize{(0.421)}} &  \\  

    \mc{1}{l}{\scriptsize{Childhood Health}} & \mc{1}{c}{\scriptsize{64}} & \mc{1}{c}{\scriptsize{50}} & \mc{1}{c}{\scriptsize{57}} & \mc{1}{c}{\scriptsize{50}} & \mc{1}{c}{\scriptsize{50}} & \mc{1}{c}{\scriptsize{64}} & \mc{1}{c}{\scriptsize{43}} & \mc{1}{c}{\scriptsize{50}} & \mc{1}{c}{\scriptsize{14}} \\  

     & \mc{1}{c}{\scriptsize{(0.145)}} & \mc{1}{c}{\scriptsize{(0.447)}} & \mc{1}{c}{\scriptsize{(0.382)}} & \mc{1}{c}{\scriptsize{(0.553)}} & \mc{1}{c}{\scriptsize{(0.566)}} & \mc{1}{c}{\scriptsize{(0.145)}} & \mc{1}{c}{\scriptsize{(0.737)}} & \mc{1}{c}{\scriptsize{(0.474)}} &  \\  

    \mc{1}{l}{\scriptsize{Adult Health}} & \mc{1}{c}{\scriptsize{45}} & \mc{1}{c}{\scriptsize{44}} & \mc{1}{c}{\scriptsize{50}} & \mc{1}{c}{\scriptsize{52}} & \mc{1}{c}{\scriptsize{51}} & \mc{1}{c}{\scriptsize{38}} & \mc{1}{c}{\scriptsize{34}} & \mc{1}{c}{\scriptsize{33}} & \mc{1}{c}{\scriptsize{84}} \\  

     & \mc{1}{c}{\scriptsize{(0.750)}} & \mc{1}{c}{\scriptsize{(0.882)}} & \mc{1}{c}{\scriptsize{(0.500)}} & \mc{1}{c}{\scriptsize{(0.329)}} & \mc{1}{c}{\scriptsize{(0.474)}} & \mc{1}{c}{\scriptsize{(0.987)}} & \mc{1}{c}{\scriptsize{(1.000)}} & \mc{1}{c}{\scriptsize{(1.000)}} &  \\  

    \mc{1}{l}{\scriptsize{Mental Health}} & \mc{1}{c}{\scriptsize{73}} & \mc{1}{c}{\scriptsize{82}} & \mc{1}{c}{\scriptsize{73}} & \mc{1}{c}{\scriptsize{87}} & \mc{1}{c}{\scriptsize{76}} & \mc{1}{c}{\scriptsize{77}} & \mc{1}{c}{\scriptsize{84}} & \mc{1}{c}{\scriptsize{79}} & \mc{1}{c}{\scriptsize{56}} \\  

     & \mc{1}{c}{\scriptsize{\textbf{(0.013)}}} & \mc{1}{c}{\scriptsize{\textbf{(0.000)}}} & \mc{1}{c}{\scriptsize{\textbf{(0.079)}}} & \mc{1}{c}{\scriptsize{\textbf{(0.000)}}} & \mc{1}{c}{\scriptsize{\textbf{(0.000)}}} & \mc{1}{c}{\scriptsize{\textbf{(0.000)}}} & \mc{1}{c}{\scriptsize{\textbf{(0.000)}}} & \mc{1}{c}{\scriptsize{\textbf{(0.000)}}} &  \\  

  \bottomrule
  \end{tabular}
	\end{sidewaystable}   

	\begin{sidewaystable}[H]
     \caption{Combining Functions by Category $|$ 10\% Significance, Female Sample} 
     \label{table:abccare_rslt_female_counts_n10a10_all}
	  \begin{tabular}{cccccccccc}
  \toprule

    \scriptsize{Category} & \scriptsize{(1)} & \scriptsize{(2)} & \scriptsize{(3)} & \scriptsize{(4)} & \scriptsize{(5)} & \scriptsize{(6)} & \scriptsize{(7)} & \scriptsize{(8)} &  \\ 
    \midrule  

    \mc{1}{l}{\scriptsize{Cognitive Skills}} & \mc{1}{c}{\scriptsize{92}} & \mc{1}{c}{\scriptsize{64}} & \mc{1}{c}{\scriptsize{92}} & \mc{1}{c}{\scriptsize{92}} & \mc{1}{c}{\scriptsize{88}} & \mc{1}{c}{\scriptsize{88}} & \mc{1}{c}{\scriptsize{28}} & \mc{1}{c}{\scriptsize{80}} & \mc{1}{c}{\scriptsize{25}} \\  

     & \mc{1}{c}{\scriptsize{\textbf{(0.000)}}} & \mc{1}{c}{\scriptsize{\textbf{(0.000)}}} & \mc{1}{c}{\scriptsize{\textbf{(0.000)}}} & \mc{1}{c}{\scriptsize{\textbf{(0.000)}}} & \mc{1}{c}{\scriptsize{\textbf{(0.000)}}} & \mc{1}{c}{\scriptsize{\textbf{(0.000)}}} & \mc{1}{c}{\scriptsize{(0.184)}} & \mc{1}{c}{\scriptsize{\textbf{(0.000)}}} &  \\  

    \mc{1}{l}{\scriptsize{Noncognitive Skills}} & \mc{1}{c}{\scriptsize{24}} & \mc{1}{c}{\scriptsize{17}} & \mc{1}{c}{\scriptsize{23}} & \mc{1}{c}{\scriptsize{15}} & \mc{1}{c}{\scriptsize{27}} & \mc{1}{c}{\scriptsize{15}} & \mc{1}{c}{\scriptsize{11}} & \mc{1}{c}{\scriptsize{16}} & \mc{1}{c}{\scriptsize{117}} \\  

     & \mc{1}{c}{\scriptsize{\textbf{(0.039)}}} & \mc{1}{c}{\scriptsize{\textbf{(0.092)}}} & \mc{1}{c}{\scriptsize{\textbf{(0.066)}}} & \mc{1}{c}{\scriptsize{(0.211)}} & \mc{1}{c}{\scriptsize{\textbf{(0.039)}}} & \mc{1}{c}{\scriptsize{(0.211)}} & \mc{1}{c}{\scriptsize{(0.395)}} & \mc{1}{c}{\scriptsize{(0.158)}} &  \\  

    \mc{1}{l}{\scriptsize{Mother's Employment, Education, and Income}} & \mc{1}{c}{\scriptsize{25}} & \mc{1}{c}{\scriptsize{25}} & \mc{1}{c}{\scriptsize{25}} & \mc{1}{c}{\scriptsize{25}} & \mc{1}{c}{\scriptsize{0}} & \mc{1}{c}{\scriptsize{25}} & \mc{1}{c}{\scriptsize{25}} & \mc{1}{c}{\scriptsize{25}} & \mc{1}{c}{\scriptsize{4}} \\  

     & \mc{1}{c}{\scriptsize{(0.158)}} & \mc{1}{c}{\scriptsize{\textbf{(0.039)}}} & \mc{1}{c}{\scriptsize{(0.211)}} & \mc{1}{c}{\scriptsize{(0.132)}} & \mc{1}{c}{\scriptsize{(0.632)}} & \mc{1}{c}{\scriptsize{(0.197)}} & \mc{1}{c}{\scriptsize{(0.276)}} & \mc{1}{c}{\scriptsize{(0.250)}} &  \\  

    \mc{1}{l}{\scriptsize{Childhood Household Environment}} & \mc{1}{c}{\scriptsize{27}} & \mc{1}{c}{\scriptsize{7}} & \mc{1}{c}{\scriptsize{47}} & \mc{1}{c}{\scriptsize{47}} & \mc{1}{c}{\scriptsize{47}} & \mc{1}{c}{\scriptsize{0}} & \mc{1}{c}{\scriptsize{7}} & \mc{1}{c}{\scriptsize{7}} & \mc{1}{c}{\scriptsize{15}} \\  

     & \mc{1}{c}{\scriptsize{\textbf{(0.079)}}} & \mc{1}{c}{\scriptsize{(0.724)}} & \mc{1}{c}{\scriptsize{\textbf{(0.000)}}} & \mc{1}{c}{\scriptsize{\textbf{(0.000)}}} & \mc{1}{c}{\scriptsize{\textbf{(0.000)}}} & \mc{1}{c}{\scriptsize{(0.908)}} & \mc{1}{c}{\scriptsize{(0.408)}} & \mc{1}{c}{\scriptsize{(0.395)}} &  \\  

    \mc{1}{l}{\scriptsize{Adult Household Environment}} & \mc{1}{c}{\scriptsize{22}} & \mc{1}{c}{\scriptsize{22}} & \mc{1}{c}{\scriptsize{11}} & \mc{1}{c}{\scriptsize{22}} & \mc{1}{c}{\scriptsize{22}} & \mc{1}{c}{\scriptsize{0}} & \mc{1}{c}{\scriptsize{0}} & \mc{1}{c}{\scriptsize{11}} & \mc{1}{c}{\scriptsize{9}} \\  

     & \mc{1}{c}{\scriptsize{(0.118)}} & \mc{1}{c}{\scriptsize{(0.118)}} & \mc{1}{c}{\scriptsize{(0.395)}} & \mc{1}{c}{\scriptsize{(0.211)}} & \mc{1}{c}{\scriptsize{(0.224)}} & \mc{1}{c}{\scriptsize{(0.816)}} & \mc{1}{c}{\scriptsize{(1.000)}} & \mc{1}{c}{\scriptsize{(0.421)}} &  \\  

    \mc{1}{l}{\scriptsize{Education, Employment, Income}} & \mc{1}{c}{\scriptsize{36}} & \mc{1}{c}{\scriptsize{29}} & \mc{1}{c}{\scriptsize{44}} & \mc{1}{c}{\scriptsize{26}} & \mc{1}{c}{\scriptsize{56}} & \mc{1}{c}{\scriptsize{29}} & \mc{1}{c}{\scriptsize{21}} & \mc{1}{c}{\scriptsize{32}} & \mc{1}{c}{\scriptsize{28}} \\  

     & \mc{1}{c}{\scriptsize{\textbf{(0.013)}}} & \mc{1}{c}{\scriptsize{\textbf{(0.039)}}} & \mc{1}{c}{\scriptsize{\textbf{(0.013)}}} & \mc{1}{c}{\scriptsize{(0.105)}} & \mc{1}{c}{\scriptsize{\textbf{(0.000)}}} & \mc{1}{c}{\scriptsize{\textbf{(0.013)}}} & \mc{1}{c}{\scriptsize{\textbf{(0.092)}}} & \mc{1}{c}{\scriptsize{\textbf{(0.053)}}} &  \\  

    \mc{1}{l}{\scriptsize{Crime}} & \mc{1}{c}{\scriptsize{100}} & \mc{1}{c}{\scriptsize{33}} & \mc{1}{c}{\scriptsize{50}} & \mc{1}{c}{\scriptsize{0}} & \mc{1}{c}{\scriptsize{0}} & \mc{1}{c}{\scriptsize{67}} & \mc{1}{c}{\scriptsize{33}} & \mc{1}{c}{\scriptsize{33}} & \mc{1}{c}{\scriptsize{3}} \\  

     & \mc{1}{c}{\scriptsize{\textbf{(0.000)}}} & \mc{1}{c}{\scriptsize{(0.145)}} & \mc{1}{c}{\scriptsize{(0.316)}} & \mc{1}{c}{\scriptsize{(0.224)}} & \mc{1}{c}{\scriptsize{(0.447)}} & \mc{1}{c}{\scriptsize{\textbf{(0.026)}}} & \mc{1}{c}{\scriptsize{\textbf{(0.092)}}} & \mc{1}{c}{\scriptsize{(0.145)}} &  \\  

    \mc{1}{l}{\scriptsize{Childhood Health}} & \mc{1}{c}{\scriptsize{29}} & \mc{1}{c}{\scriptsize{36}} & \mc{1}{c}{\scriptsize{0}} & \mc{1}{c}{\scriptsize{0}} & \mc{1}{c}{\scriptsize{0}} & \mc{1}{c}{\scriptsize{29}} & \mc{1}{c}{\scriptsize{29}} & \mc{1}{c}{\scriptsize{43}} & \mc{1}{c}{\scriptsize{14}} \\  

     & \mc{1}{c}{\scriptsize{\textbf{(0.079)}}} & \mc{1}{c}{\scriptsize{\textbf{(0.013)}}} & \mc{1}{c}{\scriptsize{(0.605)}} & \mc{1}{c}{\scriptsize{(1.000)}} & \mc{1}{c}{\scriptsize{(0.671)}} & \mc{1}{c}{\scriptsize{(0.118)}} & \mc{1}{c}{\scriptsize{\textbf{(0.079)}}} & \mc{1}{c}{\scriptsize{\textbf{(0.000)}}} &  \\  

    \mc{1}{l}{\scriptsize{Adult Health}} & \mc{1}{c}{\scriptsize{14}} & \mc{1}{c}{\scriptsize{7}} & \mc{1}{c}{\scriptsize{19}} & \mc{1}{c}{\scriptsize{11}} & \mc{1}{c}{\scriptsize{19}} & \mc{1}{c}{\scriptsize{11}} & \mc{1}{c}{\scriptsize{10}} & \mc{1}{c}{\scriptsize{13}} & \mc{1}{c}{\scriptsize{84}} \\  

     & \mc{1}{c}{\scriptsize{(0.145)}} & \mc{1}{c}{\scriptsize{(0.829)}} & \mc{1}{c}{\scriptsize{(0.105)}} & \mc{1}{c}{\scriptsize{(0.303)}} & \mc{1}{c}{\scriptsize{(0.105)}} & \mc{1}{c}{\scriptsize{(0.342)}} & \mc{1}{c}{\scriptsize{(0.553)}} & \mc{1}{c}{\scriptsize{(0.237)}} &  \\  

    \mc{1}{l}{\scriptsize{Mental Health}} & \mc{1}{c}{\scriptsize{45}} & \mc{1}{c}{\scriptsize{52}} & \mc{1}{c}{\scriptsize{39}} & \mc{1}{c}{\scriptsize{38}} & \mc{1}{c}{\scriptsize{44}} & \mc{1}{c}{\scriptsize{36}} & \mc{1}{c}{\scriptsize{41}} & \mc{1}{c}{\scriptsize{45}} & \mc{1}{c}{\scriptsize{56}} \\  

     & \mc{1}{c}{\scriptsize{\textbf{(0.000)}}} & \mc{1}{c}{\scriptsize{\textbf{(0.000)}}} & \mc{1}{c}{\scriptsize{\textbf{(0.000)}}} & \mc{1}{c}{\scriptsize{\textbf{(0.000)}}} & \mc{1}{c}{\scriptsize{\textbf{(0.000)}}} & \mc{1}{c}{\scriptsize{\textbf{(0.039)}}} & \mc{1}{c}{\scriptsize{\textbf{(0.013)}}} & \mc{1}{c}{\scriptsize{\textbf{(0.026)}}} &  \\  

  \bottomrule
  \end{tabular}
	\end{sidewaystable}   