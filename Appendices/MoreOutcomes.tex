\section{{Expanding the Analyzed Outcomes}} \label{appendix:moreoutcomes}

\noindent \textbf{[JLG: As discussed yesterday, maybe this appendix should remain for internal circulation. The outcomes beyond the ``95'' are beyond what a treatment could clearly have an effect on and could potentially be measured with error. See the table below that lists the additional outcomes.]} \\

\noindent In this appendix, we expand the analysis in Section~\ref{section:results} by considering an additional set of outcomes. Instead of considering \noutcomes, we consider \noutcomesexpp\ when pooling the sample and \noutcomesexpm\ and \noutcomesexpf\ when for males and females, respectively. Table \ref{tab:suppl-outcomes} lists supplemental outcomes that we do not consider in our primary analysis.\footnote{We fail to estimate some treatment effects because in some cases the treatment indicator and the controls entirely determine the outcome. For the \noutcomes\ in Section~\ref{section:results}, we do not have this problem.} The difference between this set of outcomes and the set we present in Section~\ref{section:results} is that (i) the outcomes in Section~\ref{section:results} contain outcomes we monetize when computing the cost-benefit analysis; and (ii) it is not entirely clear if treatment should have an effect on the outcomes we add in this appendix. The \noutcomesexpp\ outcomes in this appendix include the \noutcomes\ outcomes in Section~\ref{section:results}.\\

\singlespacing


\begin{center}
\begin{ThreePartTable}

\begin{TableNotes}[para,flushleft]
Note: This table lists the supplementary outcomes that we test treatment effects for. We reverse the outcomes for which we consider a negative treatment effect socially positive.
\end{TableNotes}


\begin{longtable}{L{4cm} L{5cm} C{1cm} C{1cm} C{1.2cm} C{1.5cm}}

\caption{Supplemental Outcome Variables}\label{tab:suppl-outcomes}  \\

\toprule
Category	&	Variable	&	Age	&	ABC	&	CARE	&	Reversed	\\ \midrule
\endfirsthead

\toprule
Category	&	Variable	&	Age	&	ABC	&	CARE	&	Reversed	\\ \midrule
\endhead

\midrule
\endfoot

%\bottomrule
\endlastfoot

Infant Behavior Record (IBR)	&	Activity Level	&	0.5	&	\checkmark	&	\checkmark	&		\\
&	&	1	&	\checkmark	&	\checkmark	&		\\
&	&	1.5	&	\checkmark	&	\checkmark	&		\\
&	&	2	&	\checkmark	&	\checkmark	&		\\
						&	Cooperativeness	&	0.5	&		&	\checkmark	&		\\
&	&	1	&	&	\checkmark	&		\\
&	&	1.5	&	&	\checkmark	&		\\
&	&	2	&	&	\checkmark	&		\\
						&	Sociability		&	0.5	&	\checkmark	&	\checkmark	&		\\
&	&	1	&	\checkmark	&	\checkmark	&		\\
&	&	1.5	&	\checkmark	&	\checkmark	&		\\
&	&	2	&	\checkmark	&		&		\\
						&	Task Orientation &	0.5	&	\checkmark	&	\checkmark	&		\\
&	&	1	&	\checkmark	&	\checkmark	&		\\
&	&	1.5	&	\checkmark	&	\checkmark	&		\\
&	&	2	&	\checkmark	&	\checkmark	&		\\
Kohn \& Rosman 			& 	Attentive/Cooperative & 2 & 
\checkmark	&	\checkmark	&		\\
						& & 12 & \checkmark	&	\checkmark	&		\\
Classroom Behavior Inventory (CBI) & Considerateness & 6 &
\checkmark	&	\checkmark	&		\\
									& 				& 7 &
\checkmark	&	\checkmark	&		\\
									& 				& 8 &
\checkmark	&	\checkmark	&		\\
									& 				& 12 &
\checkmark	&	\checkmark	&		\\
								  & Creativity & 6 &
\checkmark	&	\checkmark	&		\\
									& 				& 7 &
\checkmark	&	\checkmark	&		\\
									& 				& 8 &
\checkmark	&	\checkmark	&		\\
									& 				& 12 &
\checkmark	&	\checkmark	&		\\
								& Extraversion & 6 &
\checkmark	&	\checkmark	&		\\
									& 				& 7 &
\checkmark	&	\checkmark	&		\\
									& 				& 8 &
\checkmark	&	\checkmark	&		\\
									& 				& 12 &
\checkmark	&	\checkmark	&		\\
								& Independence & 6 &
\checkmark	&	\checkmark	&		\\
									& 				& 7 &
\checkmark	&	\checkmark	&		\\
									& 				& 8 &
\checkmark	&	\checkmark	&		\\
									& 				& 12 &
\checkmark	&	\checkmark	&		\\
								& Task Orientation & 6 &
\checkmark	&	\checkmark	&		\\
									& 				& 7 &
\checkmark	&	\checkmark	&		\\
									& 				& 8 &
\checkmark	&	\checkmark	&		\\
									& 				& 12 &
\checkmark	&	\checkmark	&		\\
								& Verbal Intelligence & 6 &
\checkmark	&	\checkmark	&		\\
									& 				& 7 &
\checkmark	&	\checkmark	&		\\
									& 				& 8 &
\checkmark	&	\checkmark	&		\\
									& 				& 12 &
\checkmark	&	\checkmark	&		\\
								& Dependence & 6 &
\checkmark	&	\checkmark	&	\checkmark	\\
									& 				& 7 &
\checkmark	&	\checkmark	&	\checkmark	\\
									& 				& 8 &
\checkmark	&	\checkmark	&	\checkmark	\\
									& 				& 12 &
\checkmark	&	\checkmark	&	\checkmark	\\
								& Distractibility & 6 &
\checkmark	&	\checkmark	&	\checkmark	\\
									& 				& 7 &
\checkmark	&	\checkmark	&	\checkmark	\\
									& 				& 8 &
\checkmark	&	\checkmark	&	\checkmark	\\
									& 				& 12 &
\checkmark	&	\checkmark	&	\checkmark	\\
								& Hostility & 6 &
\checkmark	&	\checkmark	&	\checkmark	\\
									& 				& 7 &
\checkmark	&	\checkmark	&	\checkmark	\\
									& 				& 8 &
\checkmark	&	\checkmark	&	\checkmark	\\
									& 				& 12 &
\checkmark	&	\checkmark	&	\checkmark	\\
								& Introversion & 6 &
\checkmark	&	\checkmark	&	\checkmark	\\
									& 				& 7 &
\checkmark	&	\checkmark	&	\checkmark	\\
									& 				& 8 &
\checkmark	&	\checkmark	&	\checkmark	\\
									& 				& 12 &
\checkmark	&	\checkmark	&	\checkmark	\\

Emotional, Activity, Sociability, Impulsivity Survey & Activity - Tempo & 8 & \checkmark & \checkmark & \checkmark \\
								& Activity - Vigor & 8 & \checkmark & \checkmark &  \\ 
								& Emotionality - Anger & 8 & \checkmark & \checkmark & \checkmark \\ 	
								& Emotionality - Fear & 8 & \checkmark & \checkmark & \checkmark \\ 															& Emotionality - General & 8 & \checkmark & \checkmark & \checkmark \\ 															& Impulsitivity - Control & 8 & \checkmark & \checkmark & \checkmark \\ 				
								& Impulsitivity - Decisive & 8 & \checkmark & \checkmark & \checkmark \\ 	
								& Impulsitivity - Persevere & 8 & \checkmark & \checkmark & \checkmark \\ 	
								& Impulsitivity - Sensation & 8 & \checkmark & \checkmark & \checkmark \\ 	
								& Sociability & 8 & \checkmark & \checkmark & \\ 	
								
Harter Importance 			& Behavioral Conduct & 12 & \checkmark & \checkmark  & \\	
							& Physical Appearance & 12 & \checkmark & \checkmark  & \\	
							& Social Acceptance & 12 & \checkmark & \checkmark  & \\					

Achenbach Behavior T Score (Reported by Mother) & Activities & 8 & \checkmark & & \\
						&									& 12
& \checkmark & \checkmark & \\														
& Social & 8 & \checkmark & & \\
						&									& 12
& \checkmark & \checkmark & \\	

Achenbach Behavior T Score (Reported by Teacher) & 	Behave Appropriately & 12 & \checkmark & \checkmark & \\
& 	Happiness & 12 & \checkmark & \checkmark & \\
& 	Learning & 12 & \checkmark & \checkmark & \\
& 	Work Hard & 12 & \checkmark & \checkmark & \\															
Achenbach Symptom T Score (Reported by Mother) & Aggressive & 8 & \checkmark & \checkmark & \checkmark \\
& & 12 & \checkmark & \checkmark & \checkmark \\
										& Delinquent & 8 & \checkmark & \checkmark & \checkmark \\
& & 12 & \checkmark & \checkmark & \checkmark \\	
										& Depressed & 8 & \checkmark & \checkmark & \checkmark \\
										& Externalizing & 8 & \checkmark & \checkmark & \checkmark \\
& & 12 & \checkmark & \checkmark & \checkmark \\	
										& Hyperactive & 8 & \checkmark & \checkmark & \checkmark \\
										& Internalizing & 8 & \checkmark & \checkmark & \checkmark \\
& & 12 & \checkmark & \checkmark & \checkmark \\
										& Schizoid & 12 & \checkmark & \checkmark & \checkmark \\
										& Somatic Complaints & 8 & \checkmark & \checkmark & \checkmark \\
& & 12 & \checkmark & \checkmark & \checkmark \\
												
Achenbach Symptom T Score (Reported by Teacher) & Aggressive & 12 & \checkmark & \checkmark & \checkmark \\	
 										& Anxious & 12 & \checkmark & \checkmark & \checkmark \\						
 										& Externalizing & 12 & \checkmark & \checkmark & \checkmark \\		
 										& Immature & 12 & \checkmark & \checkmark & \checkmark \\		
 										& Inattentive & 12 & \checkmark & \checkmark & \checkmark \\		
 										& Internalizing & 12 & \checkmark & \checkmark & \checkmark \\		
 										& Self-Destructive & 12 & \checkmark & \checkmark & \checkmark \\		
 										& Socially Withdrawn & 12 & \checkmark & \checkmark & \checkmark \\		
 										& Unpopular & 12 & \checkmark & \checkmark & \checkmark \\		

Relation with Spouse & No Trouble with Spouse Family & 30 & \checkmark & \checkmark & \\
					& Get Along Well with Spouse & 30 & \checkmark & \checkmark & \\
					& No Disagreement on Living Arrangement
 & 30 & \checkmark & \checkmark & \\
 
Spouse Characteristics & Spouse Annual Income & 30 & \checkmark & \checkmark & \\
						& Spouse Employment Status & 30 & \checkmark & \checkmark & \\
						
Subject Home and Property & Room Density (Rooms/People) & 30 & \checkmark & \checkmark & \\					 
							& Own Computers & 30 & \checkmark & \checkmark & \\
							& Own Cars & 30 & \checkmark & \checkmark & \\
& Own Residences & 30 & \checkmark & \checkmark & \\

Job Attitude & Satisfied with Working Situation & 30 & \checkmark & \checkmark & \\
			& Do Work Well & 30 & \checkmark & \checkmark & \\
			& Not Worry Too Much About Work & 30 & \checkmark & \checkmark & \\
			& Work Well with Others & 30 & \checkmark & \checkmark & \\
			& Don't Do Things That Cause To Lose Job & 30 & \checkmark & \checkmark & \\
			& No Trouble Finishing Work & 30 & \checkmark & \checkmark & \\
			& Job Not Too Stressful & 30 & \checkmark & \checkmark & \\
			& Don't Stay Away from Job & 30 & \checkmark & \checkmark & \\
			& No Trouble with Boss & 30 & \checkmark & \checkmark & \\
			
Job Satisfaction Score & Recognition for Good Work & 30 & \checkmark & \checkmark & \\
						& Operating Policies and Procedures & 30 & \checkmark & \checkmark & \\			
						& Immediate Supervisor & 30 & \checkmark & \checkmark & \\
						& Pay and Renumeration & 30 & \checkmark & \checkmark & \\
						& Coworkers & 30 & \checkmark & \checkmark & \\
						& Job Tasks & 30 & \checkmark & \checkmark & \\
						& Fringe Benefits & 30 & \checkmark & \checkmark & \\
						& Communication with Organization & 30 & \checkmark & \checkmark & \\
						& Promotion Opportunities & 30 & \checkmark & \checkmark & \\
						& Total & 30 & \checkmark & \checkmark & \\
						
						
Childhood and Adolescence Physical Health & Body Mass Index (BMI) & 0 & \checkmark & & \checkmark \\
							& & 0.25 & \checkmark & & \checkmark \\
							& & 0.5 & \checkmark & \checkmark & \checkmark \\
							& & 0.75 & \checkmark & & \checkmark \\
							& & 1 & \checkmark & \checkmark & \checkmark \\
							& & 1.5 & \checkmark & \checkmark & \checkmark \\
							& & 2 & \checkmark & \checkmark & \checkmark \\
							& & 2.5 & \checkmark & & \checkmark \\
						& & 3 & \checkmark & & \checkmark \\
						& & 4 & \checkmark & \checkmark & \checkmark \\
						& & 5 & \checkmark & \checkmark & \checkmark \\
						& & 8 & \checkmark & \checkmark & \checkmark \\
						& Has Health Problems & 12 & \checkmark & \checkmark & \checkmark \\
						& Ever Hospitalized Over 1 Week & 12 & \checkmark & \checkmark & \checkmark \\
						
Drug Behavior and Substance Use & Problems Due to Alcohol or Drugs & 12 & \checkmark & \checkmark & \checkmark \\
								& Used Alcohol and/or Drugs & 12 & \checkmark & \checkmark & \checkmark \\
							& Cocaine: Smokes Regularly & 30 & \checkmark & \checkmark & \checkmark \\
								& Cocaine: Times Used
& 30 & \checkmark & \checkmark & \checkmark \\
								& Cocaine: Number of Times Used Crack Cocaine & 30 & \checkmark & \checkmark & \checkmark \\

							& Marijuana: Smokes Regularly & 30 & \checkmark & \checkmark & \checkmark \\
							& & Mid-30s & \checkmark & \checkmark & \checkmark \\
							& Marijuana: Times Used & 30 & \checkmark & \checkmark & \checkmark \\
							& Marijuana: Times Used in Past 30 Days
 & 21 & \checkmark & \checkmark & \checkmark \\
 & &  30 & \checkmark & \checkmark & \checkmark \\
 							& Times Used Other Illegal Drugs in Past 30 Days & 21 & \checkmark & \checkmark & \checkmark \\
 							& ASR Substance Use Scale: Alcohol  & 30 & \checkmark & \checkmark & \checkmark \\
 							& ASR Substance Use Scale: Mean Substance Abuse & 30 & \checkmark & \checkmark & \checkmark \\
 							& ASR Substance Use Scale: Tobacco & 30 & \checkmark & \checkmark & \checkmark \\
 							& ASR Substance Use Scale: Drugs & 30 & \checkmark & \checkmark & \checkmark \\
 							
Health Insurance & Has Health Insurance & 21  & \checkmark & \checkmark & \\							
				& & 31 & \checkmark & \checkmark & \\

Laboratory Test  - Metabolic Panel	&	Albumin/Globulin Ratio	&	Mid-30s	&	\checkmark	&	\checkmark	&		\\
	&	ALT	&	Mid-30s	&	\checkmark	&	\checkmark	&		\\
	&	Albumin	&	Mid-30s	&	\checkmark	&	\checkmark	&		\\
	&	Sodium	&	Mid-30s	&	\checkmark	&	\checkmark	&		\\
	&	Carbon Dioxide	&	Mid-30s	&	\checkmark	&	\checkmark	&		\\
	&	AST	&	Mid-30s	&	\checkmark	&	\checkmark	&		\\
	&	Urea Nitrogen	&	Mid-30s	&	\checkmark	&	\checkmark	&		\\
	&	Globulin	&	Mid-30s	&	\checkmark	&	\checkmark	&		\\
	&	Chloride	&	Mid-30s	&	\checkmark	&	\checkmark	&		\\
	&	Glucose	&	Mid-30s	&	\checkmark	&	\checkmark	&		\\
	&	Potassium	&	Mid-30s	&	\checkmark	&	\checkmark	&		\\
	&	Creatinine	&	Mid-30s	&	\checkmark	&	\checkmark	&		\\
	&	Calcium	&	Mid-30s	&	\checkmark	&	\checkmark	&		\\
	&	Bilirubin	&	Mid-30s	&	\checkmark	&	\checkmark	&		\\
	&	Alkaline Phosp	&	Mid-30s	&	\checkmark	&	\checkmark	&		\\
	&	Protein	&	Mid-30s	&	\checkmark	&	\checkmark	&		\\
Laboratory Test - Complete Blood Count	&	Mean Cell Volum	&	Mid-30s	&	\checkmark	&	\checkmark	&		\\
	&	Platelets	&	Mid-30s	&	\checkmark	&	\checkmark	&		\\
	&	Eosinophils	&	Mid-30s	&	\checkmark	&	\checkmark	&		\\
	&	Hemoglobin	&	Mid-30s	&	\checkmark	&	\checkmark	&		\\
	&	Red Cells	&	Mid-30s	&	\checkmark	&	\checkmark	&		\\
	&	Lymphocytes	&	Mid-30s	&	\checkmark	&	\checkmark	&		\\
	&	Monocytes	&	Mid-30s	&	\checkmark	&	\checkmark	&		\\
	&	Neutrophils	&	Mid-30s	&	\checkmark	&	\checkmark	&		\\
	&	Basophils	&	Mid-30s	&	\checkmark	&	\checkmark	&		\\
	&	Mean Hemoglobin	&	Mid-30s	&	\checkmark	&	\checkmark	&		\\
	&	White Cells	&	Mid-30s	&	\checkmark	&	\checkmark	&		\\
	&	Hematocrit	&	Mid-30s	&	\checkmark	&	\checkmark	&		\\
	&	Red Cell Width	&	Mid-30s	&	\checkmark	&	\checkmark	&		\\
	&	Mean Hb Concentration	&	Mid-30s	&	\checkmark	&	\checkmark	&		\\
Other Health-Related Information	&	Number of Days Very Healthy in Past 30 Days	&	Mid-30s	&	\checkmark	&	\checkmark	&		\\
	&	How Subject Thinks of Own Weight	&	30	&	\checkmark	&	\checkmark	&	\checkmark	\\
	&	Number of Days in Pain in Past 30 Days	&	Mid-30s	&	\checkmark	&	\checkmark	&	\checkmark	\\
	&	Physical/Nervous Condition Prevents Work	&	30	&	\checkmark	&	\checkmark	&	\checkmark	\\
Past Medical History (Self-Reported)	&	Ever Told Had: Arthritis/Gout/Lupus/Fibromyalgia	&	Mid-30s	&	\checkmark	&	\checkmark	&	\checkmark	\\
	&	Ever Told Had: Prediabetes	&	Mid-30s	&	\checkmark	&	\checkmark	&	\checkmark	\\
	&	Past Surgery: Cholecystectomy	&	Mid-30s	&	\checkmark	&	\checkmark	&	\checkmark	\\
	&	Past Surgery: Orthopedic Surgery	&	Mid-30s	&	\checkmark	&	\checkmark	&	\checkmark	\\
	&	Past Surgery: Appendectomy	&	Mid-30s	&	\checkmark	&	\checkmark	&	\checkmark	\\
	&	Past Surgery: Ectopic Pregnancy	&	Mid-30s	&	\checkmark	&	\checkmark	&	\checkmark	\\
	&	Past Surgery: Hysterectomy	&	Mid-30s	&	\checkmark	&	\checkmark	&	\checkmark	\\
Physical Activity	&	Level of Activity at Work	&	Mid-30s	&	\checkmark	&	\checkmark	&		\\
Pysical Exam	&	Ear: Auditory Canal	&	Mid-30s	&	\checkmark	&	\checkmark	&	\checkmark	\\
	&	Eye: Eyeball	&	Mid-30s	&	\checkmark	&	\checkmark	&	\checkmark	\\
	&	Eye: Fundi	&	Mid-30s	&	\checkmark	&	\checkmark	&	\checkmark	\\
	&	Respirations	&	Mid-30s	&	\checkmark	&	\checkmark	&	\checkmark	\\
	&	Temp (F)	&	Mid-30s	&	\checkmark	&	\checkmark	&	\checkmark	\\
	&	Pulse	&	Mid-30s	&	\checkmark	&	\checkmark	&	\checkmark	\\
	&	Nutrition	&	Mid-30s	&	\checkmark	&	\checkmark	&	\checkmark	\\
	&	Posture	&	Mid-30s	&	\checkmark	&	\checkmark	&	\checkmark	\\
	&	Chest and Lung General	&	Mid-30s	&	\checkmark	&	\checkmark	&	\checkmark	\\
	&	Cardiovascular General	&	Mid-30s	&	\checkmark	&	\checkmark	&	\checkmark	\\
	&	Skin General	&	Mid-30s	&	\checkmark	&	\checkmark	&	\checkmark	\\
	&	Musculoskeletal General	&	Mid-30s	&	\checkmark	&	\checkmark	&	\checkmark	\\
	&	Head General	&	Mid-30s	&	\checkmark	&	\checkmark	&	\checkmark	\\
	&	Mouth and Throat: Upper Teeth	&	Mid-30s	&	\checkmark	&	\checkmark	&	\checkmark	\\
	&	Muscle Strength: Reflexes	&	Mid-30s	&	\checkmark	&	\checkmark	&	\checkmark	\\
	&	Mouth and Throat: Lower Teeth	&	Mid-30s	&	\checkmark	&	\checkmark	&	\checkmark	\\
Adult Self Report DSM Scale $t$-Score	&	Somatic Problems	&	30	&	\checkmark	&	\checkmark	&	\checkmark	\\
	&	AD/H Problems	&	30	&	\checkmark	&	\checkmark	&	\checkmark	\\
	&	Depressive Problems	&	30	&	\checkmark	&	\checkmark	&	\checkmark	\\
	&	Avoidant Personality Problems	&	30	&	\checkmark	&	\checkmark	&	\checkmark	\\
	&	Anxiety Problems	&	30	&	\checkmark	&	\checkmark	&	\checkmark	\\
	&	Antisocial Personality Problems	&	30	&	\checkmark	&	\checkmark	&	\checkmark	\\
Adult Self Report Syndrome Scale $t$-Score	&	Somatic Complaints	&	30	&	\checkmark	&	\checkmark	&	\checkmark	\\
	&	Aggressive Behavior	&	30	&	\checkmark	&	\checkmark	&	\checkmark	\\
	&	Intrusive	&	30	&	\checkmark	&	\checkmark	&	\checkmark	\\
	&	Anxious/Depressed	&	30	&	\checkmark	&	\checkmark	&	\checkmark	\\
	&	Externalizing	&	30	&	\checkmark	&	\checkmark	&	\checkmark	\\
	&	Thought Problems	&	30	&	\checkmark	&	\checkmark	&	\checkmark	\\
	&	Total Problems	&	30	&	\checkmark	&	\checkmark	&	\checkmark	\\
	&	Withdrawn	&	30	&	\checkmark	&	\checkmark	&	\checkmark	\\
	&	Internalizing	&	30	&	\checkmark	&	\checkmark	&	\checkmark	\\
	&	Critical Items	&	30	&	\checkmark	&	\checkmark	&	\checkmark	\\
	&	Rule Breaking	&	30	&	\checkmark	&	\checkmark	&	\checkmark	\\
	&	Attention Problems	&	30	&	\checkmark	&	\checkmark	&	\checkmark	\\
Mid-30s Mental Health Conditions	&	Current Condition: Any Psychiatric Concern	&	Mid-30s	&	\checkmark	&	\checkmark	&	\checkmark	\\
	&	Current Condition: Sad/Depressed in Past 30 Days	&	Mid-30s	&	\checkmark	&	\checkmark	&	\checkmark	\\
	&	Current Condition: Mental Problems	&	Mid-30s	&	\checkmark	&	\checkmark	&	\checkmark	\\
	&	Current Condition: Worried/Anxious in Past 30 Days	&	Mid-30s	&	\checkmark	&	\checkmark	&	\checkmark	\\
	&	Current Condition: Anxiety	&	Mid-30s	&	\checkmark	&	\checkmark	&	\checkmark	\\
	&	Current Condition: Suicidal Ideation	&	Mid-30s	&	\checkmark	&	\checkmark	&	\checkmark	\\
	&	Current Condition: Insomnia	&	Mid-30s	&	\checkmark	&	\checkmark	&	\checkmark	\\
	&	Current Condition: Depression	&	Mid-30s	&	\checkmark	&	\checkmark	&	\checkmark	\\



							

						
								
\bottomrule
	
\insertTableNotes
\end{longtable}
\end{ThreePartTable}
\end{center}




\doublespacing

\noindent To illustrate the difference between the outcomes in this appendix and the outcomes in Section~\ref{section:results}, consider two outcomes related to employment. In Section~\ref{section:results}, we include employment at age 30. In this appendix, we include both employment at age 30 and job satisfaction at age 30. Job satisfaction is a subjective variable, and it is not as clear what the effect of treatment on outcome is.\\

\noindent Table~\ref{table:abccare_rslt_pooled_counts_all} presents the results for the pooled sample, and Table~\ref{table:abccare_rslt_male_counts_all} and \ref{table:abccare_rslt_female_counts_all} present the results for males and females. It is still the case both that (i) we find a majority of positive treatment effects; and (ii) more than 10\% of these positive treatment effects are significant at the 10\%. It is true that the results weaken if compared to those we consider in Section~\ref{section:results}, but this is expected given how we construct the lists of outcomes. Appendix~\ref{appendix:morebycat} present the results by categories. We present the detailed point estimates in an additional appendix, for brevity.

	\begin{table}[H]
     \caption{Combining Functions, Pooled Sample} 
     \label{table:abccare_rslt_pooled_counts_all}
	
  \begin{tabular}{ccccccccc}
  \toprule
     & \scriptsize{(1)} & \scriptsize{(2)} & \scriptsize{(3)} & \scriptsize{(4)} & \scriptsize{(5)} & \scriptsize{(6)} & \scriptsize{(7)} & \scriptsize{(8)} \\ 
    \midrule

    \mc{1}{l}{\scriptsize{\% Pos. TE}} & \mc{1}{c}{\scriptsize{60}} & \mc{1}{c}{\scriptsize{62}} & \mc{1}{c}{\scriptsize{60}} & \mc{1}{c}{\scriptsize{65}} & \mc{1}{c}{\scriptsize{63}} & \mc{1}{c}{\scriptsize{59}} & \mc{1}{c}{\scriptsize{59}} & \mc{1}{c}{\scriptsize{61}} \\  

     & \mc{1}{c}{\scriptsize{\textbf{(0.013)}}} & \mc{1}{c}{\scriptsize{\textbf{(0.000)}}} & \mc{1}{c}{\scriptsize{\textbf{(0.013)}}} & \mc{1}{c}{\scriptsize{\textbf{(0.000)}}} & \mc{1}{c}{\scriptsize{\textbf{(0.000)}}} & \mc{1}{c}{\scriptsize{\textbf{(0.013)}}} & \mc{1}{c}{\scriptsize{\textbf{(0.013)}}} & \mc{1}{c}{\scriptsize{\textbf{(0.013)}}} \\  

    \mc{1}{l}{\scriptsize{\% Pos. TE $|$ 10\% Significance}} & \mc{1}{c}{\scriptsize{28}} & \mc{1}{c}{\scriptsize{23}} & \mc{1}{c}{\scriptsize{19}} & \mc{1}{c}{\scriptsize{16}} & \mc{1}{c}{\scriptsize{21}} & \mc{1}{c}{\scriptsize{24}} & \mc{1}{c}{\scriptsize{21}} & \mc{1}{c}{\scriptsize{27}} \\  

     & \mc{1}{c}{\scriptsize{\textbf{(0.000)}}} & \mc{1}{c}{\scriptsize{\textbf{(0.000)}}} & \mc{1}{c}{\scriptsize{\textbf{(0.026)}}} & \mc{1}{c}{\scriptsize{\textbf{(0.092)}}} & \mc{1}{c}{\scriptsize{\textbf{(0.026)}}} & \mc{1}{c}{\scriptsize{\textbf{(0.000)}}} & \mc{1}{c}{\scriptsize{\textbf{(0.000)}}} & \mc{1}{c}{\scriptsize{\textbf{(0.000)}}} \\  

  \bottomrule
  \end{tabular}

	\end{table}  

	\begin{table}[H]
     \caption{Combining Functions, Male Sample} 
     \label{table:abccare_rslt_male_counts_all}
	
  \begin{tabular}{ccccccccc}
  \toprule
     & \scriptsize{(1)} & \scriptsize{(2)} & \scriptsize{(3)} & \scriptsize{(4)} & \scriptsize{(5)} & \scriptsize{(6)} & \scriptsize{(7)} & \scriptsize{(8)} \\ 
    \midrule

    \mc{1}{l}{\scriptsize{\% Pos. TE}} & \mc{1}{c}{\scriptsize{52}} & \mc{1}{c}{\scriptsize{54}} & \mc{1}{c}{\scriptsize{39}} & \mc{1}{c}{\scriptsize{46}} & \mc{1}{c}{\scriptsize{40}} & \mc{1}{c}{\scriptsize{55}} & \mc{1}{c}{\scriptsize{57}} & \mc{1}{c}{\scriptsize{56}} \\  

     & \mc{1}{c}{\scriptsize{(0.289)}} & \mc{1}{c}{\scriptsize{(0.184)}} & \mc{1}{c}{\scriptsize{(0.987)}} & \mc{1}{c}{\scriptsize{(0.671)}} & \mc{1}{c}{\scriptsize{(0.961)}} & \mc{1}{c}{\scriptsize{(0.158)}} & \mc{1}{c}{\scriptsize{\textbf{(0.079)}}} & \mc{1}{c}{\scriptsize{(0.171)}} \\  

    \mc{1}{l}{\scriptsize{\% Pos. TE $|$ 10\% Significance}} & \mc{1}{c}{\scriptsize{15}} & \mc{1}{c}{\scriptsize{16}} & \mc{1}{c}{\scriptsize{9}} & \mc{1}{c}{\scriptsize{8}} & \mc{1}{c}{\scriptsize{8}} & \mc{1}{c}{\scriptsize{17}} & \mc{1}{c}{\scriptsize{14}} & \mc{1}{c}{\scriptsize{16}} \\  

     & \mc{1}{c}{\scriptsize{(0.132)}} & \mc{1}{c}{\scriptsize{\textbf{(0.053)}}} & \mc{1}{c}{\scriptsize{(0.539)}} & \mc{1}{c}{\scriptsize{(0.618)}} & \mc{1}{c}{\scriptsize{(0.724)}} & \mc{1}{c}{\scriptsize{\textbf{(0.026)}}} & \mc{1}{c}{\scriptsize{(0.118)}} & \mc{1}{c}{\scriptsize{\textbf{(0.053)}}} \\  

  \bottomrule
  \end{tabular}
	\end{table}  

	\begin{table}[H]
     \caption{Combining Functions, Female Sample} 
     \label{table:abccare_rslt_female_counts_all}
	
  \begin{tabular}{ccccccccc}
  \toprule
     & \scriptsize{(1)} & \scriptsize{(2)} & \scriptsize{(3)} & \scriptsize{(4)} & \scriptsize{(5)} & \scriptsize{(6)} & \scriptsize{(7)} & \scriptsize{(8)} \\ 
    \midrule

     & \scriptsize{(1)} & \scriptsize{(2)} & \scriptsize{(3)} & \scriptsize{(4)} & \scriptsize{(5)} & \scriptsize{(6)} & \scriptsize{(7)} & \scriptsize{(8)} \\ 
    \hline  

    \mc{1}{l}{\scriptsize{\% Pos. TE}} & \mc{1}{c}{\scriptsize{67}} & \mc{1}{c}{\scriptsize{68}} & \mc{1}{c}{\scriptsize{70}} & \mc{1}{c}{\scriptsize{72}} & \mc{1}{c}{\scriptsize{71}} & \mc{1}{c}{\scriptsize{62}} & \mc{1}{c}{\scriptsize{61}} & \mc{1}{c}{\scriptsize{63}} \\  

     & \mc{1}{c}{\scriptsize{\textbf{(0.000)}}} & \mc{1}{c}{\scriptsize{\textbf{(0.000)}}} & \mc{1}{c}{\scriptsize{\textbf{(0.000)}}} & \mc{1}{c}{\scriptsize{\textbf{(0.000)}}} & \mc{1}{c}{\scriptsize{\textbf{(0.000)}}} & \mc{1}{c}{\scriptsize{\textbf{(0.013)}}} & \mc{1}{c}{\scriptsize{\textbf{(0.000)}}} & \mc{1}{c}{\scriptsize{\textbf{(0.000)}}} \\  

    \mc{1}{l}{\scriptsize{\% Pos. TE $|$ 10\% Significance}} & \mc{1}{c}{\scriptsize{31}} & \mc{1}{c}{\scriptsize{25}} & \mc{1}{c}{\scriptsize{31}} & \mc{1}{c}{\scriptsize{25}} & \mc{1}{c}{\scriptsize{33}} & \mc{1}{c}{\scriptsize{23}} & \mc{1}{c}{\scriptsize{18}} & \mc{1}{c}{\scriptsize{26}} \\  

     & \mc{1}{c}{\scriptsize{\textbf{(0.000)}}} & \mc{1}{c}{\scriptsize{\textbf{(0.000)}}} & \mc{1}{c}{\scriptsize{\textbf{(0.000)}}} & \mc{1}{c}{\scriptsize{\textbf{(0.000)}}} & \mc{1}{c}{\scriptsize{\textbf{(0.000)}}} & \mc{1}{c}{\scriptsize{\textbf{(0.000)}}} & \mc{1}{c}{\scriptsize{\textbf{(0.053)}}} & \mc{1}{c}{\scriptsize{\textbf{(0.000)}}} \\  

  \bottomrule
  \end{tabular}
	\end{table}  

\subsection{{Outcomes by Category}} \label{appendix:morebycat}


	\begin{sidewaystable}[H]
     \caption{Combining Functions by Category, Pooled Sample} 
     \label{table:abccare_rslt_pooled_counts_n50a100_all}
	  \begin{tabular}{cccccccccc}
  \toprule

    \scriptsize{Category} & \scriptsize{(1)} & \scriptsize{(2)} & \scriptsize{(3)} & \scriptsize{(4)} & \scriptsize{(5)} & \scriptsize{(6)} & \scriptsize{(7)} & \scriptsize{(8)} &  \\ 
    \midrule  

    \mc{1}{l}{\scriptsize{Cognitive Skills}} & \mc{1}{c}{\scriptsize{100}} & \mc{1}{c}{\scriptsize{96}} & \mc{1}{c}{\scriptsize{100}} & \mc{1}{c}{\scriptsize{96}} & \mc{1}{c}{\scriptsize{100}} & \mc{1}{c}{\scriptsize{96}} & \mc{1}{c}{\scriptsize{96}} & \mc{1}{c}{\scriptsize{96}} & \mc{1}{c}{\scriptsize{25}} \\  

     & \mc{1}{c}{\scriptsize{\textbf{(0.000)}}} & \mc{1}{c}{\scriptsize{\textbf{(0.000)}}} & \mc{1}{c}{\scriptsize{\textbf{(0.000)}}} & \mc{1}{c}{\scriptsize{\textbf{(0.000)}}} & \mc{1}{c}{\scriptsize{\textbf{(0.000)}}} & \mc{1}{c}{\scriptsize{\textbf{(0.000)}}} & \mc{1}{c}{\scriptsize{\textbf{(0.000)}}} & \mc{1}{c}{\scriptsize{\textbf{(0.000)}}} &  \\  

    \mc{1}{l}{\scriptsize{Noncognitive Skills}} & \mc{1}{c}{\scriptsize{49}} & \mc{1}{c}{\scriptsize{54}} & \mc{1}{c}{\scriptsize{49}} & \mc{1}{c}{\scriptsize{58}} & \mc{1}{c}{\scriptsize{60}} & \mc{1}{c}{\scriptsize{50}} & \mc{1}{c}{\scriptsize{52}} & \mc{1}{c}{\scriptsize{54}} & \mc{1}{c}{\scriptsize{117}} \\  

     & \mc{1}{c}{\scriptsize{(0.539)}} & \mc{1}{c}{\scriptsize{(0.316)}} & \mc{1}{c}{\scriptsize{(0.513)}} & \mc{1}{c}{\scriptsize{(0.184)}} & \mc{1}{c}{\scriptsize{(0.145)}} & \mc{1}{c}{\scriptsize{(0.461)}} & \mc{1}{c}{\scriptsize{(0.395)}} & \mc{1}{c}{\scriptsize{(0.329)}} &  \\  

    \mc{1}{l}{\scriptsize{Mother's Employment, Education, and Income}} & \mc{1}{c}{\scriptsize{50}} & \mc{1}{c}{\scriptsize{50}} & \mc{1}{c}{\scriptsize{25}} & \mc{1}{c}{\scriptsize{75}} & \mc{1}{c}{\scriptsize{50}} & \mc{1}{c}{\scriptsize{50}} & \mc{1}{c}{\scriptsize{50}} & \mc{1}{c}{\scriptsize{25}} & \mc{1}{c}{\scriptsize{4}} \\  

     & \mc{1}{c}{\scriptsize{(0.671)}} & \mc{1}{c}{\scriptsize{(0.237)}} & \mc{1}{c}{\scriptsize{(0.987)}} & \mc{1}{c}{\scriptsize{\textbf{(0.039)}}} & \mc{1}{c}{\scriptsize{(0.671)}} & \mc{1}{c}{\scriptsize{(0.566)}} & \mc{1}{c}{\scriptsize{(0.263)}} & \mc{1}{c}{\scriptsize{(0.987)}} &  \\  

    \mc{1}{l}{\scriptsize{Childhood Household Environment}} & \mc{1}{c}{\scriptsize{67}} & \mc{1}{c}{\scriptsize{80}} & \mc{1}{c}{\scriptsize{53}} & \mc{1}{c}{\scriptsize{53}} & \mc{1}{c}{\scriptsize{60}} & \mc{1}{c}{\scriptsize{80}} & \mc{1}{c}{\scriptsize{60}} & \mc{1}{c}{\scriptsize{93}} & \mc{1}{c}{\scriptsize{15}} \\  

     & \mc{1}{c}{\scriptsize{(0.105)}} & \mc{1}{c}{\scriptsize{\textbf{(0.013)}}} & \mc{1}{c}{\scriptsize{(0.132)}} & \mc{1}{c}{\scriptsize{(0.474)}} & \mc{1}{c}{\scriptsize{\textbf{(0.079)}}} & \mc{1}{c}{\scriptsize{\textbf{(0.026)}}} & \mc{1}{c}{\scriptsize{(0.224)}} & \mc{1}{c}{\scriptsize{\textbf{(0.000)}}} &  \\  

    \mc{1}{l}{\scriptsize{Adult Household Environment}} & \mc{1}{c}{\scriptsize{78}} & \mc{1}{c}{\scriptsize{67}} & \mc{1}{c}{\scriptsize{56}} & \mc{1}{c}{\scriptsize{56}} & \mc{1}{c}{\scriptsize{44}} & \mc{1}{c}{\scriptsize{78}} & \mc{1}{c}{\scriptsize{67}} & \mc{1}{c}{\scriptsize{67}} & \mc{1}{c}{\scriptsize{9}} \\  

     & \mc{1}{c}{\scriptsize{\textbf{(0.000)}}} & \mc{1}{c}{\scriptsize{\textbf{(0.066)}}} & \mc{1}{c}{\scriptsize{(0.342)}} & \mc{1}{c}{\scriptsize{(0.368)}} & \mc{1}{c}{\scriptsize{(0.684)}} & \mc{1}{c}{\scriptsize{\textbf{(0.000)}}} & \mc{1}{c}{\scriptsize{\textbf{(0.026)}}} & \mc{1}{c}{\scriptsize{\textbf{(0.026)}}} &  \\  

    \mc{1}{l}{\scriptsize{Education, Employment, Income}} & \mc{1}{c}{\scriptsize{54}} & \mc{1}{c}{\scriptsize{54}} & \mc{1}{c}{\scriptsize{59}} & \mc{1}{c}{\scriptsize{71}} & \mc{1}{c}{\scriptsize{57}} & \mc{1}{c}{\scriptsize{54}} & \mc{1}{c}{\scriptsize{54}} & \mc{1}{c}{\scriptsize{61}} & \mc{1}{c}{\scriptsize{28}} \\  

     & \mc{1}{c}{\scriptsize{(0.395)}} & \mc{1}{c}{\scriptsize{(0.382)}} & \mc{1}{c}{\scriptsize{(0.263)}} & \mc{1}{c}{\scriptsize{\textbf{(0.066)}}} & \mc{1}{c}{\scriptsize{(0.368)}} & \mc{1}{c}{\scriptsize{(0.329)}} & \mc{1}{c}{\scriptsize{(0.342)}} & \mc{1}{c}{\scriptsize{(0.171)}} &  \\  

    \mc{1}{l}{\scriptsize{Crime}} & \mc{1}{c}{\scriptsize{33}} & \mc{1}{c}{\scriptsize{33}} & \mc{1}{c}{\scriptsize{67}} & \mc{1}{c}{\scriptsize{33}} & \mc{1}{c}{\scriptsize{33}} & \mc{1}{c}{\scriptsize{33}} & \mc{1}{c}{\scriptsize{33}} & \mc{1}{c}{\scriptsize{33}} & \mc{1}{c}{\scriptsize{3}} \\  

     & \mc{1}{c}{\scriptsize{(0.513)}} & \mc{1}{c}{\scriptsize{(0.868)}} & \mc{1}{c}{\scriptsize{(0.513)}} & \mc{1}{c}{\scriptsize{(0.882)}} & \mc{1}{c}{\scriptsize{(0.908)}} & \mc{1}{c}{\scriptsize{(0.461)}} & \mc{1}{c}{\scriptsize{(0.816)}} & \mc{1}{c}{\scriptsize{(0.842)}} &  \\  

    \mc{1}{l}{\scriptsize{Childhood Health}} & \mc{1}{c}{\scriptsize{71}} & \mc{1}{c}{\scriptsize{71}} & \mc{1}{c}{\scriptsize{57}} & \mc{1}{c}{\scriptsize{57}} & \mc{1}{c}{\scriptsize{57}} & \mc{1}{c}{\scriptsize{71}} & \mc{1}{c}{\scriptsize{64}} & \mc{1}{c}{\scriptsize{79}} & \mc{1}{c}{\scriptsize{14}} \\  

     & \mc{1}{c}{\scriptsize{\textbf{(0.013)}}} & \mc{1}{c}{\scriptsize{\textbf{(0.053)}}} & \mc{1}{c}{\scriptsize{(0.237)}} & \mc{1}{c}{\scriptsize{(0.237)}} & \mc{1}{c}{\scriptsize{(0.224)}} & \mc{1}{c}{\scriptsize{\textbf{(0.013)}}} & \mc{1}{c}{\scriptsize{\textbf{(0.053)}}} & \mc{1}{c}{\scriptsize{\textbf{(0.000)}}} &  \\  

    \mc{1}{l}{\scriptsize{Adult Health}} & \mc{1}{c}{\scriptsize{63}} & \mc{1}{c}{\scriptsize{60}} & \mc{1}{c}{\scriptsize{66}} & \mc{1}{c}{\scriptsize{61}} & \mc{1}{c}{\scriptsize{62}} & \mc{1}{c}{\scriptsize{54}} & \mc{1}{c}{\scriptsize{51}} & \mc{1}{c}{\scriptsize{51}} & \mc{1}{c}{\scriptsize{86}} \\  

     & \mc{1}{c}{\scriptsize{\textbf{(0.013)}}} & \mc{1}{c}{\scriptsize{\textbf{(0.013)}}} & \mc{1}{c}{\scriptsize{\textbf{(0.000)}}} & \mc{1}{c}{\scriptsize{\textbf{(0.013)}}} & \mc{1}{c}{\scriptsize{\textbf{(0.026)}}} & \mc{1}{c}{\scriptsize{(0.237)}} & \mc{1}{c}{\scriptsize{(0.447)}} & \mc{1}{c}{\scriptsize{(0.395)}} &  \\  

    \mc{1}{l}{\scriptsize{Mental Health}} & \mc{1}{c}{\scriptsize{64}} & \mc{1}{c}{\scriptsize{68}} & \mc{1}{c}{\scriptsize{62}} & \mc{1}{c}{\scriptsize{73}} & \mc{1}{c}{\scriptsize{64}} & \mc{1}{c}{\scriptsize{64}} & \mc{1}{c}{\scriptsize{75}} & \mc{1}{c}{\scriptsize{68}} & \mc{1}{c}{\scriptsize{56}} \\  

     & \mc{1}{c}{\scriptsize{(0.158)}} & \mc{1}{c}{\scriptsize{\textbf{(0.026)}}} & \mc{1}{c}{\scriptsize{(0.303)}} & \mc{1}{c}{\scriptsize{\textbf{(0.013)}}} & \mc{1}{c}{\scriptsize{(0.237)}} & \mc{1}{c}{\scriptsize{(0.105)}} & \mc{1}{c}{\scriptsize{\textbf{(0.000)}}} & \mc{1}{c}{\scriptsize{\textbf{(0.026)}}} &  \\  

  \bottomrule
  \end{tabular}

	\end{sidewaystable}   

	\begin{sidewaystable}[H]
     \caption{Combining Functions by Category $|$ 10\% Significance, Pooled Sample} 
     \label{table:abccare_rslt_pooled_counts_n10a10_all}
	  \begin{tabular}{cccccccccc}
  \toprule

    \scriptsize{Category} & \scriptsize{(1)} & \scriptsize{(2)} & \scriptsize{(3)} & \scriptsize{(4)} & \scriptsize{(5)} & \scriptsize{(6)} & \scriptsize{(7)} & \scriptsize{(8)} &  \\ 
    \midrule  

    \mc{1}{l}{\scriptsize{Cognitive Skills}} & \mc{1}{c}{\scriptsize{92}} & \mc{1}{c}{\scriptsize{88}} & \mc{1}{c}{\scriptsize{48}} & \mc{1}{c}{\scriptsize{48}} & \mc{1}{c}{\scriptsize{48}} & \mc{1}{c}{\scriptsize{88}} & \mc{1}{c}{\scriptsize{68}} & \mc{1}{c}{\scriptsize{88}} & \mc{1}{c}{\scriptsize{25}} \\  

     & \mc{1}{c}{\scriptsize{\textbf{(0.000)}}} & \mc{1}{c}{\scriptsize{\textbf{(0.000)}}} & \mc{1}{c}{\scriptsize{\textbf{(0.013)}}} & \mc{1}{c}{\scriptsize{\textbf{(0.000)}}} & \mc{1}{c}{\scriptsize{\textbf{(0.013)}}} & \mc{1}{c}{\scriptsize{\textbf{(0.000)}}} & \mc{1}{c}{\scriptsize{\textbf{(0.000)}}} & \mc{1}{c}{\scriptsize{\textbf{(0.000)}}} &  \\  

    \mc{1}{l}{\scriptsize{Noncognitive Skills}} & \mc{1}{c}{\scriptsize{20}} & \mc{1}{c}{\scriptsize{18}} & \mc{1}{c}{\scriptsize{11}} & \mc{1}{c}{\scriptsize{8}} & \mc{1}{c}{\scriptsize{11}} & \mc{1}{c}{\scriptsize{16}} & \mc{1}{c}{\scriptsize{15}} & \mc{1}{c}{\scriptsize{21}} & \mc{1}{c}{\scriptsize{117}} \\  

     & \mc{1}{c}{\scriptsize{\textbf{(0.053)}}} & \mc{1}{c}{\scriptsize{\textbf{(0.079)}}} & \mc{1}{c}{\scriptsize{(0.303)}} & \mc{1}{c}{\scriptsize{(0.618)}} & \mc{1}{c}{\scriptsize{(0.303)}} & \mc{1}{c}{\scriptsize{(0.105)}} & \mc{1}{c}{\scriptsize{(0.118)}} & \mc{1}{c}{\scriptsize{\textbf{(0.026)}}} &  \\  

    \mc{1}{l}{\scriptsize{Mother's Employment, Education, and Income}} & \mc{1}{c}{\scriptsize{0}} & \mc{1}{c}{\scriptsize{25}} & \mc{1}{c}{\scriptsize{25}} & \mc{1}{c}{\scriptsize{0}} & \mc{1}{c}{\scriptsize{25}} & \mc{1}{c}{\scriptsize{0}} & \mc{1}{c}{\scriptsize{25}} & \mc{1}{c}{\scriptsize{0}} & \mc{1}{c}{\scriptsize{4}} \\  

     & \mc{1}{c}{\scriptsize{(1.000)}} & \mc{1}{c}{\scriptsize{(0.368)}} & \mc{1}{c}{\scriptsize{\textbf{(0.079)}}} & \mc{1}{c}{\scriptsize{(0.395)}} & \mc{1}{c}{\scriptsize{\textbf{(0.053)}}} & \mc{1}{c}{\scriptsize{(1.000)}} & \mc{1}{c}{\scriptsize{\textbf{(0.039)}}} & \mc{1}{c}{\scriptsize{(0.461)}} &  \\  

    \mc{1}{l}{\scriptsize{Childhood Household Environment}} & \mc{1}{c}{\scriptsize{40}} & \mc{1}{c}{\scriptsize{7}} & \mc{1}{c}{\scriptsize{40}} & \mc{1}{c}{\scriptsize{27}} & \mc{1}{c}{\scriptsize{47}} & \mc{1}{c}{\scriptsize{27}} & \mc{1}{c}{\scriptsize{27}} & \mc{1}{c}{\scriptsize{40}} & \mc{1}{c}{\scriptsize{15}} \\  

     & \mc{1}{c}{\scriptsize{\textbf{(0.000)}}} & \mc{1}{c}{\scriptsize{(0.579)}} & \mc{1}{c}{\scriptsize{\textbf{(0.000)}}} & \mc{1}{c}{\scriptsize{\textbf{(0.039)}}} & \mc{1}{c}{\scriptsize{\textbf{(0.000)}}} & \mc{1}{c}{\scriptsize{\textbf{(0.092)}}} & \mc{1}{c}{\scriptsize{\textbf{(0.079)}}} & \mc{1}{c}{\scriptsize{\textbf{(0.053)}}} &  \\  

    \mc{1}{l}{\scriptsize{Adult Household Environment}} & \mc{1}{c}{\scriptsize{11}} & \mc{1}{c}{\scriptsize{22}} & \mc{1}{c}{\scriptsize{22}} & \mc{1}{c}{\scriptsize{11}} & \mc{1}{c}{\scriptsize{22}} & \mc{1}{c}{\scriptsize{11}} & \mc{1}{c}{\scriptsize{0}} & \mc{1}{c}{\scriptsize{0}} & \mc{1}{c}{\scriptsize{9}} \\  

     & \mc{1}{c}{\scriptsize{(0.434)}} & \mc{1}{c}{\scriptsize{\textbf{(0.039)}}} & \mc{1}{c}{\scriptsize{(0.211)}} & \mc{1}{c}{\scriptsize{(0.224)}} & \mc{1}{c}{\scriptsize{(0.197)}} & \mc{1}{c}{\scriptsize{(0.355)}} & \mc{1}{c}{\scriptsize{(1.000)}} & \mc{1}{c}{\scriptsize{(0.645)}} &  \\  

    \mc{1}{l}{\scriptsize{Education, Employment, Income}} & \mc{1}{c}{\scriptsize{21}} & \mc{1}{c}{\scriptsize{18}} & \mc{1}{c}{\scriptsize{22}} & \mc{1}{c}{\scriptsize{14}} & \mc{1}{c}{\scriptsize{29}} & \mc{1}{c}{\scriptsize{21}} & \mc{1}{c}{\scriptsize{14}} & \mc{1}{c}{\scriptsize{21}} & \mc{1}{c}{\scriptsize{28}} \\  

     & \mc{1}{c}{\scriptsize{\textbf{(0.026)}}} & \mc{1}{c}{\scriptsize{(0.145)}} & \mc{1}{c}{\scriptsize{\textbf{(0.079)}}} & \mc{1}{c}{\scriptsize{(0.132)}} & \mc{1}{c}{\scriptsize{\textbf{(0.079)}}} & \mc{1}{c}{\scriptsize{\textbf{(0.039)}}} & \mc{1}{c}{\scriptsize{(0.250)}} & \mc{1}{c}{\scriptsize{(0.132)}} &  \\  

    \mc{1}{l}{\scriptsize{Crime}} & \mc{1}{c}{\scriptsize{33}} & \mc{1}{c}{\scriptsize{0}} & \mc{1}{c}{\scriptsize{33}} & \mc{1}{c}{\scriptsize{0}} & \mc{1}{c}{\scriptsize{0}} & \mc{1}{c}{\scriptsize{0}} & \mc{1}{c}{\scriptsize{0}} & \mc{1}{c}{\scriptsize{0}} & \mc{1}{c}{\scriptsize{3}} \\  

     & \mc{1}{c}{\scriptsize{\textbf{(0.039)}}} & \mc{1}{c}{\scriptsize{(0.303)}} & \mc{1}{c}{\scriptsize{(0.250)}} & \mc{1}{c}{\scriptsize{(1.000)}} & \mc{1}{c}{\scriptsize{(1.000)}} & \mc{1}{c}{\scriptsize{(0.316)}} & \mc{1}{c}{\scriptsize{(1.000)}} & \mc{1}{c}{\scriptsize{(0.303)}} &  \\  

    \mc{1}{l}{\scriptsize{Childhood Health}} & \mc{1}{c}{\scriptsize{43}} & \mc{1}{c}{\scriptsize{36}} & \mc{1}{c}{\scriptsize{36}} & \mc{1}{c}{\scriptsize{21}} & \mc{1}{c}{\scriptsize{43}} & \mc{1}{c}{\scriptsize{36}} & \mc{1}{c}{\scriptsize{36}} & \mc{1}{c}{\scriptsize{36}} & \mc{1}{c}{\scriptsize{14}} \\  

     & \mc{1}{c}{\scriptsize{\textbf{(0.000)}}} & \mc{1}{c}{\scriptsize{\textbf{(0.013)}}} & \mc{1}{c}{\scriptsize{\textbf{(0.066)}}} & \mc{1}{c}{\scriptsize{(0.132)}} & \mc{1}{c}{\scriptsize{\textbf{(0.000)}}} & \mc{1}{c}{\scriptsize{\textbf{(0.000)}}} & \mc{1}{c}{\scriptsize{\textbf{(0.013)}}} & \mc{1}{c}{\scriptsize{\textbf{(0.000)}}} &  \\  

    \mc{1}{l}{\scriptsize{Adult Health}} & \mc{1}{c}{\scriptsize{23}} & \mc{1}{c}{\scriptsize{14}} & \mc{1}{c}{\scriptsize{15}} & \mc{1}{c}{\scriptsize{15}} & \mc{1}{c}{\scriptsize{18}} & \mc{1}{c}{\scriptsize{17}} & \mc{1}{c}{\scriptsize{13}} & \mc{1}{c}{\scriptsize{16}} & \mc{1}{c}{\scriptsize{86}} \\  

     & \mc{1}{c}{\scriptsize{\textbf{(0.000)}}} & \mc{1}{c}{\scriptsize{(0.211)}} & \mc{1}{c}{\scriptsize{(0.171)}} & \mc{1}{c}{\scriptsize{(0.118)}} & \mc{1}{c}{\scriptsize{\textbf{(0.066)}}} & \mc{1}{c}{\scriptsize{\textbf{(0.066)}}} & \mc{1}{c}{\scriptsize{(0.250)}} & \mc{1}{c}{\scriptsize{\textbf{(0.079)}}} &  \\  

    \mc{1}{l}{\scriptsize{Mental Health}} & \mc{1}{c}{\scriptsize{27}} & \mc{1}{c}{\scriptsize{27}} & \mc{1}{c}{\scriptsize{14}} & \mc{1}{c}{\scriptsize{20}} & \mc{1}{c}{\scriptsize{18}} & \mc{1}{c}{\scriptsize{25}} & \mc{1}{c}{\scriptsize{29}} & \mc{1}{c}{\scriptsize{34}} & \mc{1}{c}{\scriptsize{56}} \\  

     & \mc{1}{c}{\scriptsize{\textbf{(0.079)}}} & \mc{1}{c}{\scriptsize{(0.132)}} & \mc{1}{c}{\scriptsize{(0.289)}} & \mc{1}{c}{\scriptsize{(0.211)}} & \mc{1}{c}{\scriptsize{(0.224)}} & \mc{1}{c}{\scriptsize{(0.132)}} & \mc{1}{c}{\scriptsize{\textbf{(0.092)}}} & \mc{1}{c}{\scriptsize{\textbf{(0.000)}}} &  \\  

  \bottomrule
  \end{tabular}
	\end{sidewaystable}   

	\begin{sidewaystable}[H]
     \caption{Combining Functions by Category, Male Sample} 
     \label{table:abccare_rslt_male_counts_n50a100_all}
	  \begin{tabular}{cccccccccc}
  \toprule

    \scriptsize{Category} & \scriptsize{(1)} & \scriptsize{(2)} & \scriptsize{(3)} & \scriptsize{(4)} & \scriptsize{(5)} & \scriptsize{(6)} & \scriptsize{(7)} & \scriptsize{(8)} &  \\ 
    \midrule  

    \mc{1}{l}{\scriptsize{Cognitive Skills}} & \mc{1}{c}{\scriptsize{96}} & \mc{1}{c}{\scriptsize{84}} & \mc{1}{c}{\scriptsize{64}} & \mc{1}{c}{\scriptsize{60}} & \mc{1}{c}{\scriptsize{52}} & \mc{1}{c}{\scriptsize{96}} & \mc{1}{c}{\scriptsize{84}} & \mc{1}{c}{\scriptsize{84}} & \mc{1}{c}{\scriptsize{25}} \\  

     & \mc{1}{c}{\scriptsize{\textbf{(0.000)}}} & \mc{1}{c}{\scriptsize{\textbf{(0.000)}}} & \mc{1}{c}{\scriptsize{(0.303)}} & \mc{1}{c}{\scriptsize{(0.382)}} & \mc{1}{c}{\scriptsize{(0.474)}} & \mc{1}{c}{\scriptsize{\textbf{(0.000)}}} & \mc{1}{c}{\scriptsize{\textbf{(0.000)}}} & \mc{1}{c}{\scriptsize{\textbf{(0.000)}}} &  \\  

    \mc{1}{l}{\scriptsize{Noncognitive Skills}} & \mc{1}{c}{\scriptsize{42}} & \mc{1}{c}{\scriptsize{45}} & \mc{1}{c}{\scriptsize{28}} & \mc{1}{c}{\scriptsize{35}} & \mc{1}{c}{\scriptsize{34}} & \mc{1}{c}{\scriptsize{48}} & \mc{1}{c}{\scriptsize{50}} & \mc{1}{c}{\scriptsize{50}} & \mc{1}{c}{\scriptsize{117}} \\  

     & \mc{1}{c}{\scriptsize{(0.908)}} & \mc{1}{c}{\scriptsize{(0.645)}} & \mc{1}{c}{\scriptsize{(1.000)}} & \mc{1}{c}{\scriptsize{(0.934)}} & \mc{1}{c}{\scriptsize{(0.987)}} & \mc{1}{c}{\scriptsize{(0.579)}} & \mc{1}{c}{\scriptsize{(0.461)}} & \mc{1}{c}{\scriptsize{(0.539)}} &  \\  

    \mc{1}{l}{\scriptsize{Mother's Employment, Education, and Income}} & \mc{1}{c}{\scriptsize{50}} & \mc{1}{c}{\scriptsize{50}} & \mc{1}{c}{\scriptsize{50}} & \mc{1}{c}{\scriptsize{100}} & \mc{1}{c}{\scriptsize{50}} & \mc{1}{c}{\scriptsize{25}} & \mc{1}{c}{\scriptsize{50}} & \mc{1}{c}{\scriptsize{50}} & \mc{1}{c}{\scriptsize{4}} \\  

     & \mc{1}{c}{\scriptsize{(0.513)}} & \mc{1}{c}{\scriptsize{(0.658)}} & \mc{1}{c}{\scriptsize{(0.368)}} & \mc{1}{c}{\scriptsize{\textbf{(0.000)}}} & \mc{1}{c}{\scriptsize{(0.368)}} & \mc{1}{c}{\scriptsize{(0.961)}} & \mc{1}{c}{\scriptsize{(0.605)}} & \mc{1}{c}{\scriptsize{(0.632)}} &  \\  

    \mc{1}{l}{\scriptsize{Childhood Household Environment}} & \mc{1}{c}{\scriptsize{60}} & \mc{1}{c}{\scriptsize{87}} & \mc{1}{c}{\scriptsize{47}} & \mc{1}{c}{\scriptsize{53}} & \mc{1}{c}{\scriptsize{47}} & \mc{1}{c}{\scriptsize{71}} & \mc{1}{c}{\scriptsize{87}} & \mc{1}{c}{\scriptsize{80}} & \mc{1}{c}{\scriptsize{15}} \\  

     & \mc{1}{c}{\scriptsize{(0.329)}} & \mc{1}{c}{\scriptsize{\textbf{(0.000)}}} & \mc{1}{c}{\scriptsize{(0.605)}} & \mc{1}{c}{\scriptsize{(0.421)}} & \mc{1}{c}{\scriptsize{(0.579)}} & \mc{1}{c}{\scriptsize{(0.184)}} & \mc{1}{c}{\scriptsize{\textbf{(0.000)}}} & \mc{1}{c}{\scriptsize{\textbf{(0.039)}}} &  \\  

    \mc{1}{l}{\scriptsize{Adult Household Environment}} & \mc{1}{c}{\scriptsize{56}} & \mc{1}{c}{\scriptsize{44}} & \mc{1}{c}{\scriptsize{38}} & \mc{1}{c}{\scriptsize{33}} & \mc{1}{c}{\scriptsize{22}} & \mc{1}{c}{\scriptsize{56}} & \mc{1}{c}{\scriptsize{56}} & \mc{1}{c}{\scriptsize{44}} & \mc{1}{c}{\scriptsize{9}} \\  

     & \mc{1}{c}{\scriptsize{(0.395)}} & \mc{1}{c}{\scriptsize{(0.724)}} & \mc{1}{c}{\scriptsize{(0.618)}} & \mc{1}{c}{\scriptsize{(0.803)}} & \mc{1}{c}{\scriptsize{(1.000)}} & \mc{1}{c}{\scriptsize{(0.276)}} & \mc{1}{c}{\scriptsize{(0.237)}} & \mc{1}{c}{\scriptsize{(0.592)}} &  \\  

    \mc{1}{l}{\scriptsize{Education, Employment, Income}} & \mc{1}{c}{\scriptsize{32}} & \mc{1}{c}{\scriptsize{36}} & \mc{1}{c}{\scriptsize{32}} & \mc{1}{c}{\scriptsize{36}} & \mc{1}{c}{\scriptsize{36}} & \mc{1}{c}{\scriptsize{43}} & \mc{1}{c}{\scriptsize{32}} & \mc{1}{c}{\scriptsize{36}} & \mc{1}{c}{\scriptsize{28}} \\  

     & \mc{1}{c}{\scriptsize{(0.987)}} & \mc{1}{c}{\scriptsize{(0.934)}} & \mc{1}{c}{\scriptsize{(0.974)}} & \mc{1}{c}{\scriptsize{(0.934)}} & \mc{1}{c}{\scriptsize{(0.947)}} & \mc{1}{c}{\scriptsize{(0.750)}} & \mc{1}{c}{\scriptsize{(0.974)}} & \mc{1}{c}{\scriptsize{(0.947)}} &  \\  

    \mc{1}{l}{\scriptsize{Crime}} & \mc{1}{c}{\scriptsize{33}} & \mc{1}{c}{\scriptsize{33}} & \mc{1}{c}{\scriptsize{33}} & \mc{1}{c}{\scriptsize{0}} & \mc{1}{c}{\scriptsize{33}} & \mc{1}{c}{\scriptsize{33}} & \mc{1}{c}{\scriptsize{33}} & \mc{1}{c}{\scriptsize{33}} & \mc{1}{c}{\scriptsize{3}} \\  

     & \mc{1}{c}{\scriptsize{(0.882)}} & \mc{1}{c}{\scriptsize{(0.724)}} & \mc{1}{c}{\scriptsize{(0.513)}} & \mc{1}{c}{\scriptsize{(0.987)}} & \mc{1}{c}{\scriptsize{(0.987)}} & \mc{1}{c}{\scriptsize{(0.500)}} & \mc{1}{c}{\scriptsize{(0.789)}} & \mc{1}{c}{\scriptsize{(0.921)}} &  \\  

    \mc{1}{l}{\scriptsize{Childhood Health}} & \mc{1}{c}{\scriptsize{79}} & \mc{1}{c}{\scriptsize{79}} & \mc{1}{c}{\scriptsize{77}} & \mc{1}{c}{\scriptsize{69}} & \mc{1}{c}{\scriptsize{77}} & \mc{1}{c}{\scriptsize{86}} & \mc{1}{c}{\scriptsize{79}} & \mc{1}{c}{\scriptsize{86}} & \mc{1}{c}{\scriptsize{14}} \\  

     & \mc{1}{c}{\scriptsize{\textbf{(0.000)}}} & \mc{1}{c}{\scriptsize{\textbf{(0.039)}}} & \mc{1}{c}{\scriptsize{\textbf{(0.000)}}} & \mc{1}{c}{\scriptsize{(0.132)}} & \mc{1}{c}{\scriptsize{\textbf{(0.000)}}} & \mc{1}{c}{\scriptsize{\textbf{(0.000)}}} & \mc{1}{c}{\scriptsize{\textbf{(0.026)}}} & \mc{1}{c}{\scriptsize{\textbf{(0.000)}}} &  \\  

    \mc{1}{l}{\scriptsize{Adult Health}} & \mc{1}{c}{\scriptsize{68}} & \mc{1}{c}{\scriptsize{62}} & \mc{1}{c}{\scriptsize{58}} & \mc{1}{c}{\scriptsize{62}} & \mc{1}{c}{\scriptsize{58}} & \mc{1}{c}{\scriptsize{59}} & \mc{1}{c}{\scriptsize{62}} & \mc{1}{c}{\scriptsize{59}} & \mc{1}{c}{\scriptsize{74}} \\  

     & \mc{1}{c}{\scriptsize{\textbf{(0.000)}}} & \mc{1}{c}{\scriptsize{\textbf{(0.013)}}} & \mc{1}{c}{\scriptsize{(0.237)}} & \mc{1}{c}{\scriptsize{\textbf{(0.066)}}} & \mc{1}{c}{\scriptsize{(0.237)}} & \mc{1}{c}{\scriptsize{\textbf{(0.039)}}} & \mc{1}{c}{\scriptsize{\textbf{(0.000)}}} & \mc{1}{c}{\scriptsize{\textbf{(0.039)}}} &  \\  

    \mc{1}{l}{\scriptsize{Mental Health}} & \mc{1}{c}{\scriptsize{41}} & \mc{1}{c}{\scriptsize{52}} & \mc{1}{c}{\scriptsize{17}} & \mc{1}{c}{\scriptsize{37}} & \mc{1}{c}{\scriptsize{22}} & \mc{1}{c}{\scriptsize{46}} & \mc{1}{c}{\scriptsize{56}} & \mc{1}{c}{\scriptsize{52}} & \mc{1}{c}{\scriptsize{54}} \\  

     & \mc{1}{c}{\scriptsize{(0.750)}} & \mc{1}{c}{\scriptsize{(0.487)}} & \mc{1}{c}{\scriptsize{(1.000)}} & \mc{1}{c}{\scriptsize{(0.789)}} & \mc{1}{c}{\scriptsize{(1.000)}} & \mc{1}{c}{\scriptsize{(0.645)}} & \mc{1}{c}{\scriptsize{(0.408)}} & \mc{1}{c}{\scriptsize{(0.553)}} &  \\  

  \bottomrule
  \end{tabular}
	\end{sidewaystable}   

	\begin{sidewaystable}[H]
     \caption{Combining Functions by Category $|$ 10\% Significance, Male Sample} 
     \label{table:abccare_rslt_male_counts_n10a10_all}
	  \begin{tabular}{cccccccccc}
  \toprule

    \scriptsize{Category} & \scriptsize{(1)} & \scriptsize{(2)} & \scriptsize{(3)} & \scriptsize{(4)} & \scriptsize{(5)} & \scriptsize{(6)} & \scriptsize{(7)} & \scriptsize{(8)} &  \\ 
    \midrule  

    \mc{1}{l}{\scriptsize{Cognitive Skills}} & \mc{1}{c}{\scriptsize{48}} & \mc{1}{c}{\scriptsize{48}} & \mc{1}{c}{\scriptsize{24}} & \mc{1}{c}{\scriptsize{24}} & \mc{1}{c}{\scriptsize{24}} & \mc{1}{c}{\scriptsize{60}} & \mc{1}{c}{\scriptsize{44}} & \mc{1}{c}{\scriptsize{56}} & \mc{1}{c}{\scriptsize{25}} \\  

     & \mc{1}{c}{\scriptsize{\textbf{(0.039)}}} & \mc{1}{c}{\scriptsize{\textbf{(0.013)}}} & \mc{1}{c}{\scriptsize{(0.237)}} & \mc{1}{c}{\scriptsize{(0.158)}} & \mc{1}{c}{\scriptsize{(0.197)}} & \mc{1}{c}{\scriptsize{\textbf{(0.000)}}} & \mc{1}{c}{\scriptsize{\textbf{(0.000)}}} & \mc{1}{c}{\scriptsize{\textbf{(0.000)}}} &  \\  

    \mc{1}{l}{\scriptsize{Noncognitive Skills}} & \mc{1}{c}{\scriptsize{9}} & \mc{1}{c}{\scriptsize{8}} & \mc{1}{c}{\scriptsize{5}} & \mc{1}{c}{\scriptsize{4}} & \mc{1}{c}{\scriptsize{5}} & \mc{1}{c}{\scriptsize{10}} & \mc{1}{c}{\scriptsize{9}} & \mc{1}{c}{\scriptsize{12}} & \mc{1}{c}{\scriptsize{117}} \\  

     & \mc{1}{c}{\scriptsize{(0.487)}} & \mc{1}{c}{\scriptsize{(0.684)}} & \mc{1}{c}{\scriptsize{(0.908)}} & \mc{1}{c}{\scriptsize{(0.987)}} & \mc{1}{c}{\scriptsize{(0.895)}} & \mc{1}{c}{\scriptsize{(0.342)}} & \mc{1}{c}{\scriptsize{(0.447)}} & \mc{1}{c}{\scriptsize{(0.316)}} &  \\  

    \mc{1}{l}{\scriptsize{Mother's Employment, Education, and Income}} & \mc{1}{c}{\scriptsize{0}} & \mc{1}{c}{\scriptsize{0}} & \mc{1}{c}{\scriptsize{25}} & \mc{1}{c}{\scriptsize{25}} & \mc{1}{c}{\scriptsize{25}} & \mc{1}{c}{\scriptsize{0}} & \mc{1}{c}{\scriptsize{0}} & \mc{1}{c}{\scriptsize{0}} & \mc{1}{c}{\scriptsize{4}} \\  

     & \mc{1}{c}{\scriptsize{(0.987)}} & \mc{1}{c}{\scriptsize{(0.987)}} & \mc{1}{c}{\scriptsize{(0.250)}} & \mc{1}{c}{\scriptsize{(0.237)}} & \mc{1}{c}{\scriptsize{(0.237)}} & \mc{1}{c}{\scriptsize{(0.974)}} & \mc{1}{c}{\scriptsize{(0.974)}} & \mc{1}{c}{\scriptsize{(0.974)}} &  \\  

    \mc{1}{l}{\scriptsize{Childhood Household Environment}} & \mc{1}{c}{\scriptsize{7}} & \mc{1}{c}{\scriptsize{13}} & \mc{1}{c}{\scriptsize{7}} & \mc{1}{c}{\scriptsize{13}} & \mc{1}{c}{\scriptsize{7}} & \mc{1}{c}{\scriptsize{14}} & \mc{1}{c}{\scriptsize{20}} & \mc{1}{c}{\scriptsize{27}} & \mc{1}{c}{\scriptsize{15}} \\  

     & \mc{1}{c}{\scriptsize{(0.658)}} & \mc{1}{c}{\scriptsize{(0.434)}} & \mc{1}{c}{\scriptsize{(0.697)}} & \mc{1}{c}{\scriptsize{(0.316)}} & \mc{1}{c}{\scriptsize{(0.618)}} & \mc{1}{c}{\scriptsize{(0.224)}} & \mc{1}{c}{\scriptsize{(0.250)}} & \mc{1}{c}{\scriptsize{(0.197)}} &  \\  

    \mc{1}{l}{\scriptsize{Adult Household Environment}} & \mc{1}{c}{\scriptsize{0}} & \mc{1}{c}{\scriptsize{0}} & \mc{1}{c}{\scriptsize{12}} & \mc{1}{c}{\scriptsize{0}} & \mc{1}{c}{\scriptsize{11}} & \mc{1}{c}{\scriptsize{0}} & \mc{1}{c}{\scriptsize{0}} & \mc{1}{c}{\scriptsize{0}} & \mc{1}{c}{\scriptsize{9}} \\  

     & \mc{1}{c}{\scriptsize{(1.000)}} & \mc{1}{c}{\scriptsize{(1.000)}} & \mc{1}{c}{\scriptsize{(0.197)}} & \mc{1}{c}{\scriptsize{(0.697)}} & \mc{1}{c}{\scriptsize{(0.408)}} & \mc{1}{c}{\scriptsize{(0.618)}} & \mc{1}{c}{\scriptsize{(1.000)}} & \mc{1}{c}{\scriptsize{(1.000)}} &  \\  

    \mc{1}{l}{\scriptsize{Education, Employment, Income}} & \mc{1}{c}{\scriptsize{11}} & \mc{1}{c}{\scriptsize{7}} & \mc{1}{c}{\scriptsize{0}} & \mc{1}{c}{\scriptsize{4}} & \mc{1}{c}{\scriptsize{0}} & \mc{1}{c}{\scriptsize{11}} & \mc{1}{c}{\scriptsize{4}} & \mc{1}{c}{\scriptsize{7}} & \mc{1}{c}{\scriptsize{28}} \\  

     & \mc{1}{c}{\scriptsize{(0.408)}} & \mc{1}{c}{\scriptsize{(0.632)}} & \mc{1}{c}{\scriptsize{(0.882)}} & \mc{1}{c}{\scriptsize{(1.000)}} & \mc{1}{c}{\scriptsize{(1.000)}} & \mc{1}{c}{\scriptsize{(0.421)}} & \mc{1}{c}{\scriptsize{(0.908)}} & \mc{1}{c}{\scriptsize{(0.592)}} &  \\  

    \mc{1}{l}{\scriptsize{Crime}} & \mc{1}{c}{\scriptsize{0}} & \mc{1}{c}{\scriptsize{0}} & \mc{1}{c}{\scriptsize{0}} & \mc{1}{c}{\scriptsize{0}} & \mc{1}{c}{\scriptsize{0}} & \mc{1}{c}{\scriptsize{0}} & \mc{1}{c}{\scriptsize{0}} & \mc{1}{c}{\scriptsize{0}} & \mc{1}{c}{\scriptsize{3}} \\  

     & \mc{1}{c}{\scriptsize{(1.000)}} & \mc{1}{c}{\scriptsize{(1.000)}} & \mc{1}{c}{\scriptsize{(0.987)}} & \mc{1}{c}{\scriptsize{(0.987)}} & \mc{1}{c}{\scriptsize{(0.987)}} & \mc{1}{c}{\scriptsize{(0.276)}} & \mc{1}{c}{\scriptsize{(1.000)}} & \mc{1}{c}{\scriptsize{(0.316)}} &  \\  

    \mc{1}{l}{\scriptsize{Childhood Health}} & \mc{1}{c}{\scriptsize{29}} & \mc{1}{c}{\scriptsize{29}} & \mc{1}{c}{\scriptsize{46}} & \mc{1}{c}{\scriptsize{31}} & \mc{1}{c}{\scriptsize{31}} & \mc{1}{c}{\scriptsize{21}} & \mc{1}{c}{\scriptsize{14}} & \mc{1}{c}{\scriptsize{21}} & \mc{1}{c}{\scriptsize{14}} \\  

     & \mc{1}{c}{\scriptsize{(0.184)}} & \mc{1}{c}{\scriptsize{(0.145)}} & \mc{1}{c}{\scriptsize{\textbf{(0.000)}}} & \mc{1}{c}{\scriptsize{\textbf{(0.066)}}} & \mc{1}{c}{\scriptsize{\textbf{(0.079)}}} & \mc{1}{c}{\scriptsize{(0.276)}} & \mc{1}{c}{\scriptsize{(0.276)}} & \mc{1}{c}{\scriptsize{(0.263)}} &  \\  

    \mc{1}{l}{\scriptsize{Adult Health}} & \mc{1}{c}{\scriptsize{27}} & \mc{1}{c}{\scriptsize{31}} & \mc{1}{c}{\scriptsize{12}} & \mc{1}{c}{\scriptsize{14}} & \mc{1}{c}{\scriptsize{10}} & \mc{1}{c}{\scriptsize{31}} & \mc{1}{c}{\scriptsize{27}} & \mc{1}{c}{\scriptsize{26}} & \mc{1}{c}{\scriptsize{74}} \\  

     & \mc{1}{c}{\scriptsize{\textbf{(0.000)}}} & \mc{1}{c}{\scriptsize{\textbf{(0.000)}}} & \mc{1}{c}{\scriptsize{(0.368)}} & \mc{1}{c}{\scriptsize{(0.224)}} & \mc{1}{c}{\scriptsize{(0.461)}} & \mc{1}{c}{\scriptsize{\textbf{(0.000)}}} & \mc{1}{c}{\scriptsize{\textbf{(0.000)}}} & \mc{1}{c}{\scriptsize{\textbf{(0.000)}}} &  \\  

    \mc{1}{l}{\scriptsize{Mental Health}} & \mc{1}{c}{\scriptsize{0}} & \mc{1}{c}{\scriptsize{7}} & \mc{1}{c}{\scriptsize{0}} & \mc{1}{c}{\scriptsize{0}} & \mc{1}{c}{\scriptsize{0}} & \mc{1}{c}{\scriptsize{2}} & \mc{1}{c}{\scriptsize{2}} & \mc{1}{c}{\scriptsize{2}} & \mc{1}{c}{\scriptsize{54}} \\  

     & \mc{1}{c}{\scriptsize{(1.000)}} & \mc{1}{c}{\scriptsize{(0.474)}} & \mc{1}{c}{\scriptsize{(1.000)}} & \mc{1}{c}{\scriptsize{(1.000)}} & \mc{1}{c}{\scriptsize{(1.000)}} & \mc{1}{c}{\scriptsize{(1.000)}} & \mc{1}{c}{\scriptsize{(1.000)}} & \mc{1}{c}{\scriptsize{(1.000)}} &  \\  

  \bottomrule
  \end{tabular}
	\end{sidewaystable}   

	\begin{sidewaystable}[H]
     \caption{Combining Functions by Category, Female Sample} 
     \label{table:abccare_rslt_female_counts_n50a100_all}
	  \begin{tabular}{cccccccccc}
  \toprule

    \scriptsize{Category} & \scriptsize{(1)} & \scriptsize{(2)} & \scriptsize{(3)} & \scriptsize{(4)} & \scriptsize{(5)} & \scriptsize{(6)} & \scriptsize{(7)} & \scriptsize{(8)} &  \\ 
    \midrule  

    \mc{1}{l}{\scriptsize{Cognitive Skills}} & \mc{1}{c}{\scriptsize{100}} & \mc{1}{c}{\scriptsize{96}} & \mc{1}{c}{\scriptsize{100}} & \mc{1}{c}{\scriptsize{100}} & \mc{1}{c}{\scriptsize{100}} & \mc{1}{c}{\scriptsize{100}} & \mc{1}{c}{\scriptsize{92}} & \mc{1}{c}{\scriptsize{100}} & \mc{1}{c}{\scriptsize{25}} \\  

     & \mc{1}{c}{\scriptsize{\textbf{(0.000)}}} & \mc{1}{c}{\scriptsize{\textbf{(0.000)}}} & \mc{1}{c}{\scriptsize{\textbf{(0.000)}}} & \mc{1}{c}{\scriptsize{\textbf{(0.000)}}} & \mc{1}{c}{\scriptsize{\textbf{(0.000)}}} & \mc{1}{c}{\scriptsize{\textbf{(0.000)}}} & \mc{1}{c}{\scriptsize{\textbf{(0.000)}}} & \mc{1}{c}{\scriptsize{\textbf{(0.000)}}} &  \\  

    \mc{1}{l}{\scriptsize{Noncognitive Skills}} & \mc{1}{c}{\scriptsize{70}} & \mc{1}{c}{\scriptsize{70}} & \mc{1}{c}{\scriptsize{74}} & \mc{1}{c}{\scriptsize{74}} & \mc{1}{c}{\scriptsize{77}} & \mc{1}{c}{\scriptsize{57}} & \mc{1}{c}{\scriptsize{61}} & \mc{1}{c}{\scriptsize{66}} & \mc{1}{c}{\scriptsize{117}} \\  

     & \mc{1}{c}{\scriptsize{\textbf{(0.000)}}} & \mc{1}{c}{\scriptsize{\textbf{(0.000)}}} & \mc{1}{c}{\scriptsize{\textbf{(0.000)}}} & \mc{1}{c}{\scriptsize{\textbf{(0.000)}}} & \mc{1}{c}{\scriptsize{\textbf{(0.000)}}} & \mc{1}{c}{\scriptsize{(0.237)}} & \mc{1}{c}{\scriptsize{(0.105)}} & \mc{1}{c}{\scriptsize{\textbf{(0.013)}}} &  \\  

    \mc{1}{l}{\scriptsize{Mother's Employment, Education, and Income}} & \mc{1}{c}{\scriptsize{50}} & \mc{1}{c}{\scriptsize{25}} & \mc{1}{c}{\scriptsize{50}} & \mc{1}{c}{\scriptsize{25}} & \mc{1}{c}{\scriptsize{50}} & \mc{1}{c}{\scriptsize{50}} & \mc{1}{c}{\scriptsize{25}} & \mc{1}{c}{\scriptsize{50}} & \mc{1}{c}{\scriptsize{4}} \\  

     & \mc{1}{c}{\scriptsize{(0.842)}} & \mc{1}{c}{\scriptsize{(0.974)}} & \mc{1}{c}{\scriptsize{(0.724)}} & \mc{1}{c}{\scriptsize{(0.750)}} & \mc{1}{c}{\scriptsize{(0.645)}} & \mc{1}{c}{\scriptsize{(0.224)}} & \mc{1}{c}{\scriptsize{(0.803)}} & \mc{1}{c}{\scriptsize{(0.250)}} &  \\  

    \mc{1}{l}{\scriptsize{Childhood Household Environment}} & \mc{1}{c}{\scriptsize{73}} & \mc{1}{c}{\scriptsize{80}} & \mc{1}{c}{\scriptsize{60}} & \mc{1}{c}{\scriptsize{67}} & \mc{1}{c}{\scriptsize{60}} & \mc{1}{c}{\scriptsize{64}} & \mc{1}{c}{\scriptsize{64}} & \mc{1}{c}{\scriptsize{79}} & \mc{1}{c}{\scriptsize{15}} \\  

     & \mc{1}{c}{\scriptsize{\textbf{(0.039)}}} & \mc{1}{c}{\scriptsize{\textbf{(0.026)}}} & \mc{1}{c}{\scriptsize{\textbf{(0.066)}}} & \mc{1}{c}{\scriptsize{\textbf{(0.053)}}} & \mc{1}{c}{\scriptsize{(0.132)}} & \mc{1}{c}{\scriptsize{(0.263)}} & \mc{1}{c}{\scriptsize{(0.145)}} & \mc{1}{c}{\scriptsize{\textbf{(0.053)}}} &  \\  

    \mc{1}{l}{\scriptsize{Adult Household Environment}} & \mc{1}{c}{\scriptsize{78}} & \mc{1}{c}{\scriptsize{67}} & \mc{1}{c}{\scriptsize{89}} & \mc{1}{c}{\scriptsize{78}} & \mc{1}{c}{\scriptsize{89}} & \mc{1}{c}{\scriptsize{89}} & \mc{1}{c}{\scriptsize{56}} & \mc{1}{c}{\scriptsize{56}} & \mc{1}{c}{\scriptsize{9}} \\  

     & \mc{1}{c}{\scriptsize{\textbf{(0.000)}}} & \mc{1}{c}{\scriptsize{\textbf{(0.092)}}} & \mc{1}{c}{\scriptsize{\textbf{(0.000)}}} & \mc{1}{c}{\scriptsize{\textbf{(0.053)}}} & \mc{1}{c}{\scriptsize{\textbf{(0.000)}}} & \mc{1}{c}{\scriptsize{\textbf{(0.000)}}} & \mc{1}{c}{\scriptsize{(0.303)}} & \mc{1}{c}{\scriptsize{(0.289)}} &  \\  

    \mc{1}{l}{\scriptsize{Education, Employment, Income}} & \mc{1}{c}{\scriptsize{71}} & \mc{1}{c}{\scriptsize{89}} & \mc{1}{c}{\scriptsize{74}} & \mc{1}{c}{\scriptsize{78}} & \mc{1}{c}{\scriptsize{74}} & \mc{1}{c}{\scriptsize{71}} & \mc{1}{c}{\scriptsize{75}} & \mc{1}{c}{\scriptsize{71}} & \mc{1}{c}{\scriptsize{28}} \\  

     & \mc{1}{c}{\scriptsize{\textbf{(0.013)}}} & \mc{1}{c}{\scriptsize{\textbf{(0.000)}}} & \mc{1}{c}{\scriptsize{\textbf{(0.000)}}} & \mc{1}{c}{\scriptsize{\textbf{(0.000)}}} & \mc{1}{c}{\scriptsize{\textbf{(0.000)}}} & \mc{1}{c}{\scriptsize{\textbf{(0.026)}}} & \mc{1}{c}{\scriptsize{\textbf{(0.000)}}} & \mc{1}{c}{\scriptsize{\textbf{(0.000)}}} &  \\  

    \mc{1}{l}{\scriptsize{Crime}} & \mc{1}{c}{\scriptsize{100}} & \mc{1}{c}{\scriptsize{100}} & \mc{1}{c}{\scriptsize{100}} & \mc{1}{c}{\scriptsize{100}} & \mc{1}{c}{\scriptsize{100}} & \mc{1}{c}{\scriptsize{100}} & \mc{1}{c}{\scriptsize{100}} & \mc{1}{c}{\scriptsize{67}} & \mc{1}{c}{\scriptsize{3}} \\  

     & \mc{1}{c}{\scriptsize{\textbf{(0.000)}}} & \mc{1}{c}{\scriptsize{\textbf{(0.000)}}} & \mc{1}{c}{\scriptsize{\textbf{(0.000)}}} & \mc{1}{c}{\scriptsize{\textbf{(0.000)}}} & \mc{1}{c}{\scriptsize{\textbf{(0.000)}}} & \mc{1}{c}{\scriptsize{\textbf{(0.000)}}} & \mc{1}{c}{\scriptsize{\textbf{(0.000)}}} & \mc{1}{c}{\scriptsize{(0.421)}} &  \\  

    \mc{1}{l}{\scriptsize{Childhood Health}} & \mc{1}{c}{\scriptsize{64}} & \mc{1}{c}{\scriptsize{50}} & \mc{1}{c}{\scriptsize{57}} & \mc{1}{c}{\scriptsize{50}} & \mc{1}{c}{\scriptsize{50}} & \mc{1}{c}{\scriptsize{64}} & \mc{1}{c}{\scriptsize{43}} & \mc{1}{c}{\scriptsize{50}} & \mc{1}{c}{\scriptsize{14}} \\  

     & \mc{1}{c}{\scriptsize{(0.145)}} & \mc{1}{c}{\scriptsize{(0.447)}} & \mc{1}{c}{\scriptsize{(0.382)}} & \mc{1}{c}{\scriptsize{(0.553)}} & \mc{1}{c}{\scriptsize{(0.566)}} & \mc{1}{c}{\scriptsize{(0.145)}} & \mc{1}{c}{\scriptsize{(0.737)}} & \mc{1}{c}{\scriptsize{(0.474)}} &  \\  

    \mc{1}{l}{\scriptsize{Adult Health}} & \mc{1}{c}{\scriptsize{45}} & \mc{1}{c}{\scriptsize{44}} & \mc{1}{c}{\scriptsize{50}} & \mc{1}{c}{\scriptsize{52}} & \mc{1}{c}{\scriptsize{51}} & \mc{1}{c}{\scriptsize{38}} & \mc{1}{c}{\scriptsize{34}} & \mc{1}{c}{\scriptsize{33}} & \mc{1}{c}{\scriptsize{84}} \\  

     & \mc{1}{c}{\scriptsize{(0.750)}} & \mc{1}{c}{\scriptsize{(0.882)}} & \mc{1}{c}{\scriptsize{(0.500)}} & \mc{1}{c}{\scriptsize{(0.329)}} & \mc{1}{c}{\scriptsize{(0.474)}} & \mc{1}{c}{\scriptsize{(0.987)}} & \mc{1}{c}{\scriptsize{(1.000)}} & \mc{1}{c}{\scriptsize{(1.000)}} &  \\  

    \mc{1}{l}{\scriptsize{Mental Health}} & \mc{1}{c}{\scriptsize{73}} & \mc{1}{c}{\scriptsize{82}} & \mc{1}{c}{\scriptsize{73}} & \mc{1}{c}{\scriptsize{87}} & \mc{1}{c}{\scriptsize{76}} & \mc{1}{c}{\scriptsize{77}} & \mc{1}{c}{\scriptsize{84}} & \mc{1}{c}{\scriptsize{79}} & \mc{1}{c}{\scriptsize{56}} \\  

     & \mc{1}{c}{\scriptsize{\textbf{(0.013)}}} & \mc{1}{c}{\scriptsize{\textbf{(0.000)}}} & \mc{1}{c}{\scriptsize{\textbf{(0.079)}}} & \mc{1}{c}{\scriptsize{\textbf{(0.000)}}} & \mc{1}{c}{\scriptsize{\textbf{(0.000)}}} & \mc{1}{c}{\scriptsize{\textbf{(0.000)}}} & \mc{1}{c}{\scriptsize{\textbf{(0.000)}}} & \mc{1}{c}{\scriptsize{\textbf{(0.000)}}} &  \\  

  \bottomrule
  \end{tabular}
	\end{sidewaystable}   

	\begin{sidewaystable}[H]
     \caption{Combining Functions by Category $|$ 10\% Significance, Female Sample} 
     \label{table:abccare_rslt_female_counts_n10a10_all}
	  \begin{tabular}{cccccccccc}
  \toprule

    \scriptsize{Category} & \scriptsize{(1)} & \scriptsize{(2)} & \scriptsize{(3)} & \scriptsize{(4)} & \scriptsize{(5)} & \scriptsize{(6)} & \scriptsize{(7)} & \scriptsize{(8)} &  \\ 
    \midrule  

    \mc{1}{l}{\scriptsize{Cognitive Skills}} & \mc{1}{c}{\scriptsize{92}} & \mc{1}{c}{\scriptsize{64}} & \mc{1}{c}{\scriptsize{92}} & \mc{1}{c}{\scriptsize{92}} & \mc{1}{c}{\scriptsize{88}} & \mc{1}{c}{\scriptsize{88}} & \mc{1}{c}{\scriptsize{28}} & \mc{1}{c}{\scriptsize{80}} & \mc{1}{c}{\scriptsize{25}} \\  

     & \mc{1}{c}{\scriptsize{\textbf{(0.000)}}} & \mc{1}{c}{\scriptsize{\textbf{(0.000)}}} & \mc{1}{c}{\scriptsize{\textbf{(0.000)}}} & \mc{1}{c}{\scriptsize{\textbf{(0.000)}}} & \mc{1}{c}{\scriptsize{\textbf{(0.000)}}} & \mc{1}{c}{\scriptsize{\textbf{(0.000)}}} & \mc{1}{c}{\scriptsize{(0.184)}} & \mc{1}{c}{\scriptsize{\textbf{(0.000)}}} &  \\  

    \mc{1}{l}{\scriptsize{Noncognitive Skills}} & \mc{1}{c}{\scriptsize{24}} & \mc{1}{c}{\scriptsize{17}} & \mc{1}{c}{\scriptsize{23}} & \mc{1}{c}{\scriptsize{15}} & \mc{1}{c}{\scriptsize{27}} & \mc{1}{c}{\scriptsize{15}} & \mc{1}{c}{\scriptsize{11}} & \mc{1}{c}{\scriptsize{16}} & \mc{1}{c}{\scriptsize{117}} \\  

     & \mc{1}{c}{\scriptsize{\textbf{(0.039)}}} & \mc{1}{c}{\scriptsize{\textbf{(0.092)}}} & \mc{1}{c}{\scriptsize{\textbf{(0.066)}}} & \mc{1}{c}{\scriptsize{(0.211)}} & \mc{1}{c}{\scriptsize{\textbf{(0.039)}}} & \mc{1}{c}{\scriptsize{(0.211)}} & \mc{1}{c}{\scriptsize{(0.395)}} & \mc{1}{c}{\scriptsize{(0.158)}} &  \\  

    \mc{1}{l}{\scriptsize{Mother's Employment, Education, and Income}} & \mc{1}{c}{\scriptsize{25}} & \mc{1}{c}{\scriptsize{25}} & \mc{1}{c}{\scriptsize{25}} & \mc{1}{c}{\scriptsize{25}} & \mc{1}{c}{\scriptsize{0}} & \mc{1}{c}{\scriptsize{25}} & \mc{1}{c}{\scriptsize{25}} & \mc{1}{c}{\scriptsize{25}} & \mc{1}{c}{\scriptsize{4}} \\  

     & \mc{1}{c}{\scriptsize{(0.158)}} & \mc{1}{c}{\scriptsize{\textbf{(0.039)}}} & \mc{1}{c}{\scriptsize{(0.211)}} & \mc{1}{c}{\scriptsize{(0.132)}} & \mc{1}{c}{\scriptsize{(0.632)}} & \mc{1}{c}{\scriptsize{(0.197)}} & \mc{1}{c}{\scriptsize{(0.276)}} & \mc{1}{c}{\scriptsize{(0.250)}} &  \\  

    \mc{1}{l}{\scriptsize{Childhood Household Environment}} & \mc{1}{c}{\scriptsize{27}} & \mc{1}{c}{\scriptsize{7}} & \mc{1}{c}{\scriptsize{47}} & \mc{1}{c}{\scriptsize{47}} & \mc{1}{c}{\scriptsize{47}} & \mc{1}{c}{\scriptsize{0}} & \mc{1}{c}{\scriptsize{7}} & \mc{1}{c}{\scriptsize{7}} & \mc{1}{c}{\scriptsize{15}} \\  

     & \mc{1}{c}{\scriptsize{\textbf{(0.079)}}} & \mc{1}{c}{\scriptsize{(0.724)}} & \mc{1}{c}{\scriptsize{\textbf{(0.000)}}} & \mc{1}{c}{\scriptsize{\textbf{(0.000)}}} & \mc{1}{c}{\scriptsize{\textbf{(0.000)}}} & \mc{1}{c}{\scriptsize{(0.908)}} & \mc{1}{c}{\scriptsize{(0.408)}} & \mc{1}{c}{\scriptsize{(0.395)}} &  \\  

    \mc{1}{l}{\scriptsize{Adult Household Environment}} & \mc{1}{c}{\scriptsize{22}} & \mc{1}{c}{\scriptsize{22}} & \mc{1}{c}{\scriptsize{11}} & \mc{1}{c}{\scriptsize{22}} & \mc{1}{c}{\scriptsize{22}} & \mc{1}{c}{\scriptsize{0}} & \mc{1}{c}{\scriptsize{0}} & \mc{1}{c}{\scriptsize{11}} & \mc{1}{c}{\scriptsize{9}} \\  

     & \mc{1}{c}{\scriptsize{(0.118)}} & \mc{1}{c}{\scriptsize{(0.118)}} & \mc{1}{c}{\scriptsize{(0.395)}} & \mc{1}{c}{\scriptsize{(0.211)}} & \mc{1}{c}{\scriptsize{(0.224)}} & \mc{1}{c}{\scriptsize{(0.816)}} & \mc{1}{c}{\scriptsize{(1.000)}} & \mc{1}{c}{\scriptsize{(0.421)}} &  \\  

    \mc{1}{l}{\scriptsize{Education, Employment, Income}} & \mc{1}{c}{\scriptsize{36}} & \mc{1}{c}{\scriptsize{29}} & \mc{1}{c}{\scriptsize{44}} & \mc{1}{c}{\scriptsize{26}} & \mc{1}{c}{\scriptsize{56}} & \mc{1}{c}{\scriptsize{29}} & \mc{1}{c}{\scriptsize{21}} & \mc{1}{c}{\scriptsize{32}} & \mc{1}{c}{\scriptsize{28}} \\  

     & \mc{1}{c}{\scriptsize{\textbf{(0.013)}}} & \mc{1}{c}{\scriptsize{\textbf{(0.039)}}} & \mc{1}{c}{\scriptsize{\textbf{(0.013)}}} & \mc{1}{c}{\scriptsize{(0.105)}} & \mc{1}{c}{\scriptsize{\textbf{(0.000)}}} & \mc{1}{c}{\scriptsize{\textbf{(0.013)}}} & \mc{1}{c}{\scriptsize{\textbf{(0.092)}}} & \mc{1}{c}{\scriptsize{\textbf{(0.053)}}} &  \\  

    \mc{1}{l}{\scriptsize{Crime}} & \mc{1}{c}{\scriptsize{100}} & \mc{1}{c}{\scriptsize{33}} & \mc{1}{c}{\scriptsize{50}} & \mc{1}{c}{\scriptsize{0}} & \mc{1}{c}{\scriptsize{0}} & \mc{1}{c}{\scriptsize{67}} & \mc{1}{c}{\scriptsize{33}} & \mc{1}{c}{\scriptsize{33}} & \mc{1}{c}{\scriptsize{3}} \\  

     & \mc{1}{c}{\scriptsize{\textbf{(0.000)}}} & \mc{1}{c}{\scriptsize{(0.145)}} & \mc{1}{c}{\scriptsize{(0.316)}} & \mc{1}{c}{\scriptsize{(0.224)}} & \mc{1}{c}{\scriptsize{(0.447)}} & \mc{1}{c}{\scriptsize{\textbf{(0.026)}}} & \mc{1}{c}{\scriptsize{\textbf{(0.092)}}} & \mc{1}{c}{\scriptsize{(0.145)}} &  \\  

    \mc{1}{l}{\scriptsize{Childhood Health}} & \mc{1}{c}{\scriptsize{29}} & \mc{1}{c}{\scriptsize{36}} & \mc{1}{c}{\scriptsize{0}} & \mc{1}{c}{\scriptsize{0}} & \mc{1}{c}{\scriptsize{0}} & \mc{1}{c}{\scriptsize{29}} & \mc{1}{c}{\scriptsize{29}} & \mc{1}{c}{\scriptsize{43}} & \mc{1}{c}{\scriptsize{14}} \\  

     & \mc{1}{c}{\scriptsize{\textbf{(0.079)}}} & \mc{1}{c}{\scriptsize{\textbf{(0.013)}}} & \mc{1}{c}{\scriptsize{(0.605)}} & \mc{1}{c}{\scriptsize{(1.000)}} & \mc{1}{c}{\scriptsize{(0.671)}} & \mc{1}{c}{\scriptsize{(0.118)}} & \mc{1}{c}{\scriptsize{\textbf{(0.079)}}} & \mc{1}{c}{\scriptsize{\textbf{(0.000)}}} &  \\  

    \mc{1}{l}{\scriptsize{Adult Health}} & \mc{1}{c}{\scriptsize{14}} & \mc{1}{c}{\scriptsize{7}} & \mc{1}{c}{\scriptsize{19}} & \mc{1}{c}{\scriptsize{11}} & \mc{1}{c}{\scriptsize{19}} & \mc{1}{c}{\scriptsize{11}} & \mc{1}{c}{\scriptsize{10}} & \mc{1}{c}{\scriptsize{13}} & \mc{1}{c}{\scriptsize{84}} \\  

     & \mc{1}{c}{\scriptsize{(0.145)}} & \mc{1}{c}{\scriptsize{(0.829)}} & \mc{1}{c}{\scriptsize{(0.105)}} & \mc{1}{c}{\scriptsize{(0.303)}} & \mc{1}{c}{\scriptsize{(0.105)}} & \mc{1}{c}{\scriptsize{(0.342)}} & \mc{1}{c}{\scriptsize{(0.553)}} & \mc{1}{c}{\scriptsize{(0.237)}} &  \\  

    \mc{1}{l}{\scriptsize{Mental Health}} & \mc{1}{c}{\scriptsize{45}} & \mc{1}{c}{\scriptsize{52}} & \mc{1}{c}{\scriptsize{39}} & \mc{1}{c}{\scriptsize{38}} & \mc{1}{c}{\scriptsize{44}} & \mc{1}{c}{\scriptsize{36}} & \mc{1}{c}{\scriptsize{41}} & \mc{1}{c}{\scriptsize{45}} & \mc{1}{c}{\scriptsize{56}} \\  

     & \mc{1}{c}{\scriptsize{\textbf{(0.000)}}} & \mc{1}{c}{\scriptsize{\textbf{(0.000)}}} & \mc{1}{c}{\scriptsize{\textbf{(0.000)}}} & \mc{1}{c}{\scriptsize{\textbf{(0.000)}}} & \mc{1}{c}{\scriptsize{\textbf{(0.000)}}} & \mc{1}{c}{\scriptsize{\textbf{(0.039)}}} & \mc{1}{c}{\scriptsize{\textbf{(0.013)}}} & \mc{1}{c}{\scriptsize{\textbf{(0.026)}}} &  \\  

  \bottomrule
  \end{tabular}
	\end{sidewaystable}   