\section{Background Variables and Tested Outcomes} \label{appendix:controls}
\subsection{Controlling for Background Variables}

\noindent We select the background variables that we control for as follows. First, we select the three variables, from 17 potential variables, that best predict enrollment into alternative preschool for subjects who were randomly assigned not to receive center-based childcare---the control group in ABC and the control and family education treatment groups in CARE. To choose the most predictive control set, we estimate the $17 \choose 3$ linear models, using the variables in Panels (1) and (2) of Table~\ref{tab:pselectvars}. The most predictive model is the one generating the lowest Bayesian Information Criterion (BIC).\\

\noindent We proceed in a similar way to choose the second set of background variables. Instead of predicting enrollment into alternative preschool, we predict the relevant outcome of interest. For this exercise we use the variables in Panel (2), so we estimate $14 \choose 3$ models.\\

\noindent Our set of control variables consists of these two sets of background variables.

\singlespacing
\begin{table}[H]
\centering
\begin{threeparttable}
\caption{Background Variables}
\label{tab:pselectvars}
\begin{tabular}{C{5cm} C{5cm} C{5cm}}
\toprule
\textbf{Maternal IQ}			& \textbf{Maternal education}		& Mother's age at birth \\
High Risk Index		& Parent income			& Premature birth \\
1 minute Apgar score	& 5 minute Apgar score	& Mother married \\
Teen pregnancy		& \textbf{Father at home}			& Number of siblings \\
Cohort 				& Mother is employed		& \\
\bottomrule
\end{tabular}
\begin{tablenotes}
\footnotesize
\item Note: This table lists the variables we permute over when selecting the background variables we control for in our estimations. We bold the variables we choose based on the procedure explained in this section.
\end{tablenotes}
\end{threeparttable}
\end{table}
\doublespacing
\subsection{Tested Outcomes}

\noindent The following table lists the 95 main outcomes that are used in the cost/benefit analysis. Appendix~\ref{appendix:moreoutcomes} presents treatment effects for a larger set of outcomes. We reverse the outcomes for which we consider a negative treatment effect socially positive.

\singlespacing
\begin{center}
\begin{ThreePartTable}

\begin{TableNotes}[para,flushleft]
Note: This table lists the outcomes that we test treatment effects for. We reverse the outcomes for which we consider a negative treatment effect socially positive.
\end{TableNotes}


\begin{longtable}{L{4cm} L{5cm} C{1cm} C{1cm} C{1.2cm} C{1.5cm}}

\caption{Outcome Variables} \\

\toprule
Category	&	Variable	&	Age	&	ABC	&	CARE	&	Reversed	\\ \midrule
\endfirsthead

\toprule
Category	&	Variable	&	Age	&	ABC	&	CARE	&	Reversed	\\ \midrule
\endhead

\midrule
\endfoot

%\bottomrule
\endlastfoot

IQ Scores	&	Std. IQ Test	&	2	&	\checkmark	&	\checkmark	&		\\
	&		&	2.5	&		&	\checkmark	&		\\
	&		&	3	&	\checkmark	&	\checkmark	&		\\
	&		&	3.5	&	\checkmark	&	\checkmark	&		\\
	&		&	4	&	\checkmark	&	\checkmark	&		\\
	&		&	4.5	&	\checkmark	&	\checkmark	&		\\
	&		&	5	&	\checkmark	&	\checkmark	&		\\
	&		&	6.6	&	\checkmark	&	\checkmark	&		\\
	&		&	7	&	\checkmark	&	\checkmark	&		\\
	&		&	8	&	\checkmark	&	\checkmark	&		\\
	&		&	12	&	\checkmark	&	\checkmark	&		\\
	&		&	15	&	\checkmark	&		&		\\
	&		&	21	&	\checkmark	&		&		\\
	&	IQ Factor	&	2 to 5	&	\checkmark	&	\checkmark	&		\\
	&		&	6 to 12	&	\checkmark	&	\checkmark	&		\\
	&		&	15 to 21	&	\checkmark	&		&		\\     
\\[0.1cm]
Achievement Scores	&	Std. Achv.  Test	&	5.5	&	\checkmark	&	\checkmark	&		\\
	&		&	6	&	\checkmark	&	\checkmark	&		\\
	&		&	6.5	&	\checkmark	&		&		\\
	&		&	7	&	\checkmark	&		&		\\
	&		&	7.5	&	\checkmark	&	\checkmark	&		\\
	&		&	8	&	\checkmark	&	\checkmark	&		\\
	&		&	8.5	&	\checkmark	&	\checkmark	&		\\
	&		&	12	&		&	\checkmark	&		\\
	&		&	15	&	\checkmark	&		&		\\
	&		&	21	&	\checkmark	&		&		\\
	&	PIAT Math Std. Score	&	7	&	\checkmark	&	\checkmark	&		\\
	&	Achievement Factor	&	5.5 to 12	&	\checkmark	&	\checkmark	&		\\
	&		&	15 to 21	&	\checkmark	&		&		\\
\\[0.1cm]
HOME Scores	&	HOME Score	&	0.5	&	\checkmark	&	\checkmark	&		\\
	&		&	1.5	&	\checkmark	&	\checkmark	&		\\
	&		&	2.5	&	\checkmark	&	\checkmark	&		\\
	&		&	3.5	&	\checkmark	&	\checkmark	&		\\
	&		&	4.5	&	\checkmark	&	\checkmark	&		\\
	&		&	8	&	\checkmark	&	\checkmark	&		\\
	&	HOME Factor	&	0.5 to 8	&	\checkmark	&	\checkmark	&		\\
\\[0.1cm]
Parent Income	&	Parental income	&	1.5	&	\checkmark	&	\checkmark	&		\\
	&		&	2.5	&	\checkmark	&	\checkmark	&		\\
	&		&	3.5	&	\checkmark	&	\checkmark	&		\\
	&		&	4.5	&	\checkmark	&	\checkmark	&		\\
	&		&	8	&	\checkmark	&		&		\\
	&		&	12	&	\checkmark	&		&		\\
	&		&	15	&	\checkmark	&		&		\\
	&	Parental Income Factor	&	1.5 to 15	&	\checkmark	&	\checkmark	&		\\
\\[0.1cm]
Mother's Employment	&	Mother Works	&	2	&	\checkmark	&	\checkmark	&		\\
	&		&	3	&	\checkmark	&	\checkmark	&		\\
	&		&	4	&	\checkmark	&	\checkmark	&		\\
	&		&	5	&	\checkmark	&	\checkmark	&		\\
	&		&	21	&	\checkmark	&		&		\\
	&	Mother Works Factor	&	2 to 21	&	\checkmark	&	\checkmark	&		\\
\\[0.1cm]
Mother's Education	&	Mother's Years of Edu.	&	2	&	\checkmark	&		&		\\
	&		&	3	&	\checkmark	&		&		\\
	&		&	4	&	\checkmark	&		&		\\
	&		&	5	&	\checkmark	&		&		\\
	&		&	9	&	\checkmark	&		&		\\
	&	Mother's Edu. Factor	&	2 to 9	&	\checkmark	&		&		\\
\\[0.1cm]
Father at Home	&	Father at Home	&	2	&	\checkmark	&	\checkmark	&		\\
	&		&	3	&	\checkmark	&	\checkmark	&		\\
	&		&	4	&	\checkmark	&	\checkmark	&		\\
	&		&	5	&	\checkmark	&	\checkmark	&		\\
	&		&	8	&	\checkmark	&	\checkmark	&		\\
	&	Father at Home Factor	&	2 to 8	&	\checkmark	&	\checkmark	&		\\
\\[0.1cm]
Adoption	&	Ever Adopted	&	       	&	\checkmark	&		&		\\
\\[0.1cm]
Education	&	Graduated High School	&	30	&	\checkmark	&	\checkmark	&		\\
	&	Attended Voc./Tech./Com. College	&	30	&	\checkmark	&	\checkmark	&		\\
	&	Graduated 4-year College	&	30	&	\checkmark	&	\checkmark	&		\\
	&	Years of Edu.	&	30	&	\checkmark	&	\checkmark	&		\\
	&	Education Factor	&	30	&	\checkmark	&	\checkmark	&		\\
\\[0.1cm]
Employment and Income	&	Employed	&	30	&	\checkmark	&	\checkmark	&		\\
	&	Labor Income	&	21	&	\checkmark	&	\checkmark	&		\\
	&		&	30	&	\checkmark	&	\checkmark	&		\\
	&	Public-Transfer Income	&	21	&	\checkmark	&	\checkmark	&	\checkmark	\\
	&		&	30	&	\checkmark	&	\checkmark	&	\checkmark	\\
	&	Employment Factor	&	21 to 30	&	\checkmark	&	\checkmark	&		\\
\\[0.1cm]
Crime	&	Total Felony Arrests	&	Mid-30s	&	\checkmark	&	\checkmark	&	\checkmark	\\
	&	Total Misdemeanor Arrests	&	Mid-30s	&	\checkmark	&	\checkmark	&	\checkmark	\\
	&	Total Years Incarcerated	&	30	&	\checkmark	&	\checkmark	&	\checkmark	\\
	&	Crime Factor	&	30 to Mid-30s	&	\checkmark	&	\checkmark	&	\checkmark	\\
\\[0.1cm]
Tobacco, Drugs, Alcohol	&	Cig. Smoked per day last month	&	30	&	\checkmark	&	\checkmark	&	\checkmark	\\
	&	Days drank alcohol last month	&	30	&	\checkmark	&	\checkmark	&	\checkmark	\\
	&	Days binge drank alcohol last month	&	30	&	\checkmark	&	\checkmark	&	\checkmark	\\
	&	Self-reported drug user	&	Mid-30s	&	\checkmark	&	\checkmark	&	\checkmark	\\
	&	Substance Use Factor	&	30 to Mid-30s	&	\checkmark	&	\checkmark	&	\checkmark	\\
\\[0.1cm]
Self-Reported Health	&	Self-reported Health	&	30	&	\checkmark	&	\checkmark	&	\checkmark	\\
	&		&	Mid-30s	&	\checkmark	&	\checkmark	&	\checkmark	\\
	&	Self-reported Health Factor	&	30 to Mid-30s	&	\checkmark	&	\checkmark	&	\checkmark	\\
\\[0.1cm]
Hypertension	&	Systolic Blood Pressure (mm Hg)	&	Mid-30s	&	\checkmark	&	\checkmark	&	\checkmark	\\
	&	Diastolic Blood Pressure (mm Hg)	&	Mid-30s	&	\checkmark	&	\checkmark	&	\checkmark	\\
	&	Prehypertension	&	Mid-30s	&	\checkmark	&	\checkmark	&	\checkmark	\\
	&	Hypertension	&	Mid-30s	&	\checkmark	&	\checkmark	&	\checkmark	\\
	&	Hypertension Factor	&	Mid-30s	&	\checkmark	&	\checkmark	&	\checkmark	\\
\\[0.1cm]
Cholesterol	&	High-Density Lipoprotein Chol. (mg/dL)	&	Mid-30s	&	\checkmark	&	\checkmark	&		\\
	&	Dyslipidemia	&	Mid-30s	&	\checkmark	&	\checkmark	&	\checkmark	\\
	&	Cholesterol Factor	&	Mid-30s	&	\checkmark	&	\checkmark	&	\checkmark	\\
\\[0.1cm]
Diabetes	&	Hemoglobin Level (\%)	&	Mid-30s	&	\checkmark	&	\checkmark	&	\checkmark	\\
	&	Prediabetes	&	Mid-30s	&	\checkmark	&	\checkmark	&	\checkmark	\\
	&	Diabetes	&	Mid-30s	&	\checkmark	&	\checkmark	&	\checkmark	\\
	&	Diabetes Factor	&	Mid-30s	&	\checkmark	&	\checkmark	&	\checkmark	\\
\\[0.1cm]
Vitamin D Deficiency	&	Vitamin D Deficiency	&	Mid-30s	&	\checkmark	&	\checkmark	&	\checkmark	\\
\\[0.1cm]
Obesity	&	Measured BMI	&	Mid-30s	&	\checkmark	&	\checkmark	&	\checkmark	\\
	&	Obesity	&	Mid-30s	&	\checkmark	&	\checkmark	&	\checkmark	\\
	&	Severe Obesity	&	Mid-30s	&	\checkmark	&	\checkmark	&	\checkmark	\\
	&	Waist-hip Ratio	&	Mid-30s	&	\checkmark	&	\checkmark	&	\checkmark	\\
	&	Abdominal Obesity	&	Mid-30s	&	\checkmark	&	\checkmark	&	\checkmark	\\
	&	Framingham Risk Score	&	Mid-30s	&	\checkmark	&	\checkmark	&	\checkmark	\\
	&	Obesity Factor	&	Mid-30s	&	\checkmark	&	\checkmark	&	\checkmark	\\
\\[0.1cm]
Mental Health (BSI)	&	Somatization	&	21	&	\checkmark	&	\checkmark	&	\checkmark	\\
	&		&	34	&	\checkmark	&	\checkmark	&	\checkmark	\\
	&	Depression	&	21	&	\checkmark	&	\checkmark	&	\checkmark	\\
	&		&	34	&	\checkmark	&	\checkmark	&	\checkmark	\\
	&	Anxiety	&	21	&	\checkmark	&	\checkmark	&	\checkmark	\\
	&		&	34	&	\checkmark	&	\checkmark	&	\checkmark	\\
	&	Hostility	&	21	&	\checkmark	&	\checkmark	&	\checkmark	\\
	&		&	34	&	\checkmark	&	\checkmark	&	\checkmark	\\
	&	Global Severity Index	&	21	&	\checkmark	&	\checkmark	&	\checkmark	\\
	&		&	34	&	\checkmark	&	\checkmark	&	\checkmark	\\
	&	Mental Health Factor	&	21 and 34	&	\checkmark	&	\checkmark	&	\checkmark	\\
\\[0.1cm]
Child Behavior (CAS)	&	Participates in Activity	&	12	&	\checkmark	&		&		\\
	&	Time Spent Reading	&	12	&	\checkmark	&		&		\\
	&	Good Description of Self	&	12	&	\checkmark	&		&		\\
	&	Views Self as Dumb	&	12	&	\checkmark	&		&	\checkmark	\\
	&	Views Self as Clumsy	&	12	&	\checkmark	&		&	\checkmark	\\
	&	Views Self as Not Liked	&	12	&	\checkmark	&		&	\checkmark	\\
	&	Proud About Self	&	12	&	\checkmark	&		&		\\
	&	Family Proud of You	&	12	&	\checkmark	&		&		\\
	&	Feels Inadequate, Inferior	&	12	&	\checkmark	&		&	\checkmark	\\
	&	Withdraws Excessively	&	12	&	\checkmark	&		&	\checkmark	\\
	&	Ignores Situation	&	12	&	\checkmark	&		&	\checkmark	\\
	&	Not Cope with Prob.	&	12	&	\checkmark	&		&	\checkmark	\\
	&	Often Mad of Angry	&	12	&	\checkmark	&		&	\checkmark	\\
	&	Impulsivity	&	12	&	\checkmark	&		&	\checkmark	\\
	&	Significant Fears	&	12	&	\checkmark	&		&	\checkmark	\\
	&	Denies Any Worries	&	12	&	\checkmark	&		&	\checkmark	\\

\bottomrule
	
\insertTableNotes
\end{longtable}
\end{ThreePartTable}
\end{center}




\doublespacing
