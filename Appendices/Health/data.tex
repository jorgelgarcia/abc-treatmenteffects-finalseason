\subsection{Data Sources} \label{section:data}
\noindent FAM uses data from ABC and CARE follow-up surveys to build the initial state of the cohort. 
The transition model parameters are estimated from the 1997 to 2013 waves of the Panel Study of Income Dynamics (PSID). 
We supplement the PSID with data from the Health and Retirement Study (HRS). 
To estimate medical care costs associated with health conditions, we use the Medical Expenditures Panel Survey (MEPS) and the Medicare Current Beneficiaries Survey (MCBS). \\
% \todo cite all sources

\subsubsection{PSID}
\label{section:data_psid}
%The Panel Survey of Income Dynamics (PSID) is a longitudinal household survey containing between 5,000 and 8,500 families in each wave, which began yearly in 1968 and is fielded biennially since 1996. When appropriately weighted, the PSID is designed to be representative of U.S. households. The PSID provides extensive information concerning demographics, economic outcomes, health care access, health outcomes, and health behaviors (such as smoking history, alcohol consumption, and exercise habits). Health outcome variables include diagnosis of diabetes, heart disease, hypertension, lung disease, cancer, etc. 

\noindent The Panel Study  of Income Dynamics (PSID) provides extensive information concerning demographics, economic outcomes, health care access, health outcomes, and health behaviors (such as smoking history, alcohol consumption, and exercise habits). Health outcome variables include diagnosis of diabetes, heart disease, hypertension, lung disease, and cancer, among others. See Section \ref{app:subject_income_psid} for additional details on the PSID. \\

\noindent We estimate the transition models using waves from 1997 to 2013. We create a dataset of respondents who have formed their own households, either
as single heads of households, cohabiting partners, or married partners.  These heads, wives, and husbands respond to the richest
set of PSID questions, including the health questions that are critical for our purposes. We use all respondents aged 25 and older.\footnote{While we use the full sample, we explored using a few different subsamples to better adapt to the demographics of the ABC and CARE subjects.}  
The length of the PSID is a significant advantage, because we can include past health behaviors as explanatory variables for current health outcomes. This dataset provides adequate sample sizes to explore health outcomes of specific groups. 
PSID does not follow individuals who are institutionalized in nursing homes or other long-term care facilities. To overcome this weakness, we pool the PSID sample with the Health and Retirement Study (HRS) sample when
estimating mortality models. \\

\subsubsection{HRS}

\noindent The Health and Retirement Study (HRS) is a longitudinal panel that surveys a nationally representative sample of individuals over the age of 50 and their spouses every two years.  When appropriately weighted, the HRS in 2010 is representative of U.S. households 
where at least one member is at least 51 years old.
This study collects in-depth information about income, work, health, and medical expenditures. In our model, waves from 1998 to 2012 are pooled with the PSID for estimation of mortality and 
widowhood models. The HRS data
are harmonized to the PSID for all relevant variables. We use the dataset created by RAND (RAND HRS, version N) as our basis 
for the analysis. We use all cohorts in the analysis. \\

\subsubsection{MCBS}
\noindent The Medicare Current Beneficiary Survey (MCBS) is a nationally representative sample of aged, disabled, 
and institutionalized Medicare beneficiaries.  The MCBS attempts to interview each respondent twelve 
times over three years, regardless of whether he or she resides in the community, a facility, or 
transitions between community and facility settings. The disabled (under 65 years of age) and 
very elderly (85 years of age or older) are over-sampled. The first round of interviewing was conducted 
in 1991. Originally, the survey was a longitudinal sample with periodic supplements and indefinite 
periods of participation. In 1994, the MCBS switched to a rotating panel design with limited periods 
of participation. Each fall, a new panel is introduced, with a target sample size of 12,000 respondents. Each summer, a panel is retired. Institutionalized respondents are interviewed by proxy.  The MCBS 
contains comprehensive self-reported information on the health status, health care use and 
expenditures, health insurance coverage, and socioeconomic and demographic characteristics of the 
entire spectrum of Medicare beneficiaries.  Medicare claims data for beneficiaries enrolled in 
fee-for-service plans are also used to provide more accurate information on health care use and 
expenditures.  MCBS data from 2007 to 2010 are used for estimating medical costs and enrollment models. \\

\subsubsection{MEPS}
\noindent The Medical Expenditure Panel Survey (MEPS), which began in 1996, is a set of large-scale surveys of families and individuals, their medical providers, and employers across the U.S. The Household Component (HC) of the MEPS provides data from 
individual households and their members, which is supplemented by data from their medical providers. 
The HC collects data from a representative subsample of households drawn from the 
previous year's National Health Interview Survey (NHIS). Since NHIS does not include the 
institutionalized population, neither does MEPS; this implies that we can only use the MEPS to 
estimate medical costs for the non-elderly (ages 25--64) population. Information collected during household 
interviews include: demographic characteristics, health conditions, health status, use of medical 
services, sources of medical payments, and body weight and height. Each year the household survey 
includes approximately 12,000 households, or 34,000 individuals. Sample size for those aged 25-64 is 
about 15,800 in each year.  MEPS has comparable measures of socioeconomic status as those in PSID, 
including age, race and ethnicity, educational attainment, census region, and marital status.  We estimate medical expenditure 
and utilization using data from 2007 to 2010. We use waves from 2001 to 2003 to estimate models of quality-adjusted life years (QALYs), due to availability of EQ-5D instrument in these waves.\footnote{Section \ref{section:FAM_models} explains the estimation of the QALY model.} \\


\subsubsection{ABC and CARE}
\noindent FAM uses ABC and CARE data to initialize the state of each ABC or CARE subject when they enter into the simulation.  
These data are taken from the the age-21 parent interview; age-30 subject interview; age-30 tobacco, alcohol, and drugs interview; and age-34 health interview.  
The goal is to have each subject's initial state match their status at the age-30 subject interview. Because several key FAM inputs are not available at the age-30 interview, we use surveys corresponding to other ages to impute missing elements. These imputations are discussed in Section \ref{section:FAM_ABC_impute}. \\

% \todo we could add a table of all the FAM input variables (rows) and three columns, as-is, rule imputed, model imputed, with checkmarks to indicate how the data were derived
