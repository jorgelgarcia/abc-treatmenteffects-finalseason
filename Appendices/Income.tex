\section{Interpolation of Parental Income}
\label{app:parent_income}


\noindent We observe labor income of the mother and, if applicable, her partner at
ages 0, 2, 3, 4, 5, 9, 12, and 15. To obtain our measure of parental income, we simply
sum the labor income of the respondent and her partner. \\

% Income of the parents of ABC subjects is observed at ages 2, 3, 4, 5, 9, 12, and 15.
% For ages 2 to 9, we observe only the annual income of parents, which we use in our
% estimates. The data do not distinguish the source of the annual income. Thus,
% it is unclear whether those variables are pre- or post-tax, and if they include
% transfers, income from additional contributors, or are restricted to labor income.
% At age 9, data on parent income are collected twice, so we use the average of the
% reported incomes.

% At age 12, in addition to observing annual income of parents,
% we also observe income from transfers, which includes child support, disability
% benefits, food stamps, rent subsidies, social security, and other sources of government
% transfers. As it is unclear whether or not the data we observe on annual parent income
% includes government transfers, we construct our variable for parent income at
% age 12 by taking the midpoint between annual parent income and annual parent income summed
% with income from transfers.

% At age 15, we observe annual earned income of parents, in addition to income from food
% stamps, rent subsidies, social security, AFDC, child support, and other government transfers.
% To construct parent income at age 15, we sum all the components above.

% At all ages, income from other family members are excluded, and all values are inflated
% to 2014 dollars.

\setcounter{figure}{0}  \renewcommand{\thefigure}{K.\arabic{figure}}
\setcounter{table}{0}   \renewcommand{\thetable}{Ks.\arabic{table}}
\section{Extrapolation of Individual Income}
\label{app:subject_income}

\subsection{Data Sources}

\noindent Income of the ABC/CARE subjects is observed in detail at ages 21 and 30. For both ages,
we separate labor income from public-transfer income. At age 21, public-transfer income includes Aid to
Families with Dependent Children (AFDC) subsidies, food stamps, survivor benefits, disability
benefits, social security, rent subsidies, and fuel subsidies. At age 30, public-transfer income includes food stamps, welfare, housing assistance, workman's
compensation, disability, social security, supplemental security income, unemployment benefits,
worker's compensation insurance, fuel subsidies, educational and aid grants, and other forms of welfare.
At both ages 21 and 30, labor income is observed. Also at both ages, income from other family members are excluded, and all values are in
to 2014 USD. We utilize the
NLSY, CNLY, and PSID datasets. We provide a brief overview of them below. \\

\subsubsection{NLSY}
\label{app:subject_income_nlsy}

\noindent The National Longitudinal Survey of Youth (NLSY) is a longitudinal survey beginning in 1979
that follows individuals born between 1957 and 1964. The initial interview included
12,686 respondents aged 14 to 22. The survey was designed to include 6,111
individuals representing the non-institutionalized civilian population, a supplemental
sample of 5,295 civilian Hispanics, Latinos, blacks, non-blacks/non-Hispanics, and economically
disadvantaged youth, and a sample of 1,280 who served in the military as of September 30,
1978. When appropriately weighted, the NLSY is nationally representative of the youth
living in the U.S. on January 1, 1979. We include individuals from all three subsamples
in our analysis. \\

\noindent The NLSY collected data on labor market participation, education, family background,
family life, health issues, assets and income, government program participation, and
measures of cognitive skills. We use these data to estimate a prediction model for
subject income for ages 31 through 67. \\

\noindent We restrict the NLSY sample to individuals with labor income less than
\$300,000 (2014 USD) at any given year. With the mean labor income (2014 USD) in the ABC
sample being \$12,232 at age 21 and \$32,782 at age 30, and the maximum reported
being \$189,938, the cut-off we impose on the auxiliary data is high enough
so that everyone in the ABC/CARE sample is represented, yet low enough to
exclude high-earning individuals in the auxiliary sample that do not reflect the ABC/CARE sample well. \\

\noindent We do not impose a restriction on birth year on the NLSY as all respondents are aged between 47 and 55
at the time of the last interview (conducted in 2012). This age range is within the 31--67 range for which
we extrapolate the income of the ABC/CARE subjects. \\

\noindent Given the biennial nature of the NLSY, we only observe each subject at either odd or even
ages. Not only does this reduce the size of the sample on which we fit our prediction model
at each age, but it can introduce biases associated with the odd-aged and even-aged cohorts.
To address this issue, we perform a linear interpolation on the variables in the NLSY data
that enter into our prediction model. This allows us to estimate our prediction model on
all subjects of the NLSY satisfying our restrictions at every age. \\

\subsubsection{CNLSY}
\label{app:subject_income_cnlsy}

\noindent The Children of the National Longitudinal Survey of Youth (CNLSY) is a survey of the children of the mothers from the NLSY, beginning in 1986. At the time of
the initial interview, the ages of the children surveyed ranged from 0 to 23. As of 2010,
the CNLSY sample includes 11,504 children born to NLSY mothers. With appropriate weights,
the CNLSY may be considered nationally representative of children born to women
who were aged 14 to 22 during 1979. Interviews were conducted annually between 1986 and 1994,
and biennially thereafter. \\

\noindent Similar to the NLSY, the CNLSY collected data on cognitive ability, motor and social development,
home environment, health information, education, attitudes, employment, income, family decisions,
and more. We use these data to estimate a prediction model for subject income for ages
22 through 29. \\

\noindent As we did with the NLSY, we restrict the CNLSY sample to individuals with labor income less than
\$300,000 (2014 USD) at any given year. In addition to this, we limit the sample to subjects born
between 1978 and 1983. Because the CNLSY data extend to 2012, this implies that we use the most
recent data from the CNLSY in which individuals are aged 29 to 34. Finally, given the biennial
nature of the survey, we perform a linear interpolation on the variables that enter into our prediction
model. This allows us to use as much of the CNLSY data as possible at every age when
interpolating subject income. \\

\subsubsection{PSID}
\label{app:subject_income_psid}

\noindent The Panel Survey of Income Dynamics (PSID) is a longitudinal household survey containing between 5,000
and 8,500 families in each wave. It began as yearly survey in 1968 and has been fielded biennially since 1996.
When appropriately weighted, the PSID is designed to be representative of U.S. households. The PSID
provides extensive information concerning demographics, economic outcomes, health outcomes, marriage
and fertility, and more. In addition to using the PSID to forecast future earnings of ABC and CARE subjects, we
use the PSID to forecast health outcomes. See Appendix \ref{section:data_psid} for details on how the
PSID relates to health outcomes. \\

\noindent Similar to the CNLSY, we restrict the PSID to individuals born between 1945 and 1981. Because the data
extend to 2013, this means we use the most recent subsample of individuals aged 30 to 67.
We also exclude all individuals with labor income exceeding \$300,000 (2014 USD) at any given year.
Finally, given the biennial nature of the survey, we perform a linear interpolation on the variables
that enter into our prediction model. This allows us to use as much of the PSID data as possible at
every age to interpolate subject income. \\

