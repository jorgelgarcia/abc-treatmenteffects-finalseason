\section{Procedures to Calculate Alternative Cost/Benefit Analyses}
\label{app:alt-cba}

In Table~\ref{table:comparing}, we present five estimates of benefit/cost ratios that compare our baseline estimates to the calculations done in other recent cost/benefit analyses. \\

\noindent \textbf{Estimate 1: 0.58 (s.e. 0.28)} \\

\begin{enumerate}
\item Obtain the result of on standardized gain in cognition from \citet{Chetty_Friedman_etal_2011_QJoE}, in terms of labor income at age 27 (this is how \citet{Chetty_Friedman_etal_2011_QJoE} report their return to cognition).
\item Obtain the standardized gain in cognition in ABC/CARE.
\item Assign a labor income gain to each individual based on the return of \citet{Chetty_Friedman_etal_2011_QJoE}.
\item Compute the average gain (netting out the control-group level).
\item Divide the average gain by the average cost of the program.
\item Standard errors come from the bootstrapping our sample, although the return remains constant in all the calculation.
\item This is the calculation that \citet{Kline_Walters_2016_QJE} use.
\end{enumerate} 

\noindent \textbf{Estimate 2: 0.09 (s.e. 0.04)} \\

We follow the same steps as used to predict Estimate 1, but instead of using the return to cognition from \citet{Chetty_Friedman_etal_2011_QJoE}, we use our own return to cognition in terms of labor income at age 30. In this case, the standard errors do account for variation in the return, as we calculate the return in ever bootstrap re-sample. In that sense, this is a ``better'' estimate if compared to \citet{Kline_Walters_2016_QJE}. The return is smaller because our sample is much more disadvantaged than that of \citet{Chetty_Friedman_etal_2011_QJoE}, whose individuals come from a more mixed background (Project STAR). \\

\noindent \textbf{Estimate 3: 0.15 (s.e. 0.05)} \\

This estimate is calculated using our method, but the only benefit is labor income up to the mid 30s, and the only cost is the cost of the program. \\

\noindent \textbf{Estimate 4: 3.20 (s.e. 1.04)} \\

This estimate is calculated using our method, but considering all of the benefits and costs up to the mid 30s (not just labor income). \\

\noindent \textbf{Estimate 5: 1.55 (s.e. 0.76)} \\ 

This estimate is calculated using our method, but only considering the benefit of labor income. This estimate is calculated throughout the life cycle.