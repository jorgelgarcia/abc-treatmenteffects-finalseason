\section{Gender Differences}
\label{appendix:genderdifferences}

\subsection{Survey of Gender Differences Literature}
\label{appendix:gdiff-survey}

We summarize (Table~\ref{tab:litreview-table}) work that examines early-life differences between boys and girls. It is generally found that boys are more fragile than girls early in life. While some of these papers consider the family environment, there is a dearth of work studying (1) the effect of low-quality preschool on children\footnote{Although \citet{Kottelenberg_Lehrer_2014_Gender-Effects} study gender gaps, they only consider intact families.} and (2) the interaction of this with family environments. We find that while low-quality programs can deteriorate the parent-child interaction, especially for boys, high-quality programs can enhance it.

\begin{sidewaystable}[H]
\centering
\caption{Literature Review on Early Gender Differences}
\label{tab:litreview-table}
\begin{adjustbox}{width=1.05\textwidth}
\begin{threeparttable}
\begin{tabular}{cccccc} \toprule											
\textbf{Paper}	&	\textbf{Program(s)}	&	\textbf{Main Gender-Difference Finding}	&	\textbf{Outcomes}	&	\textbf{Quality of Childcare Setting?} 	&	\textbf{Quality of Home Setting?} 	\\ \midrule
\citet{lundberg2005sons}	&	Literature survey	&	-females: divorce is likely if all children are girls	&	-fertility and divorce	&	No	&	No	\\ 
	&		&	less likely to live with fathers (US), spends more 	&		&		&		\\ 
	&		&	time with mothers	&		&		&		\\ 
	&		&	-males: increase marital stability, increase 	&		&		&		\\ 
	&		&	likelihood of subsequent child	&		&		&		\\ \\ \midrule
\citet{Anderson_2008_JASA}	&	ABC	&	-modest results for males	&	-child, adolescent, adult (up to age 21 for ABC)	&	No	&	No	\\ 
	&	Perry Preschool Program	&	-females especially affected in academic outcomes	&	-social, educational, employment	&		&		\\ 
	&	Early Training Program (ETP)	&	-accounting for multiple hypotheses	&	-test scores	&		&		\\ 
	&		&		&	reduces effects substantially	&		&		\\ 
	&		&		&	especially for males	&		&		\\ \\ \midrule
\citet{Ou_Reynolds_2010_Mechanisms_CYSR}	&	Chicago Child-Parent Center	&	Differences in treatment effects consequence	&	-educational attainment 	&	No	&	Yes	\\ 
	&	- 1334 youths (682 females, 652 males)	&	of difference in mediators	&	-HS or GED (jointly coded)	&		&		\\ 
	&	- center-based, served 3/4 year olds	&	-male mediators: preschool participation	&		&		&		\\ 
	&	- RCT 	&	-female mediators: family support, abuse/neglect	&		&		&		\\ \\ \midrule
\citet{Bertrand_Pan_2013_AEJAE}	&	ECLS-K (ATUS as complementary)	&	Stark gender differences 	&	-socio-emotional measures	&	No	&	Yes	\\ 
	&	-observational study up to 5th grade	&	- females: better on all socio-emotional measures	&	-grade suspension 	&		&		\\ 
	&		&	(gaps widen when children get older)	&	- tests scores (math and reading)	&		&		\\ 
	&		&	- males: worst at reading but better than math at 	&		&		&		\\ 
	&		&	1st grade	&		&		&		\\ \\ \midrule
\citet{cornwell2013noncognitive}	&	ECLS-K 	&	Gender differences in tests and grades	&	-reading, science, math tests scores 	&	No	&	No	\\ 
	&	-observational study up to 5th grade	&	-males: better in science and math; worst grades	&	-grades	&		&		\\ 
	&		&	overall 	&	-socio-emotional measures	&		&		\\ 
	&		&	females: better reading tests (gap wider than 	&		&		&		\\ 
	&		&	gap with respect to science and math)	&		&		&		\\ 
	&		&	- some but bot all of the gaps disappears when	&		&		&		\\ 
	&		&	accounting for socio-emotional measures 	&		&		&		\\ \\ \midrule
\citet{golsteyn2014gender}	&	Observational study in the Netherlands	&	Gender differences across skills and tests	&	-cognition	&	N/A 	&	No	\\ 
	&	- elementary school children, age 11/12 	&	- males: higher assertiveness and math	&	-socio-emotional measures	&		&		\\ 
	&		&	-females: higher social skills and language	&	-math and language tests	&		&		\\ \\ \midrule
\citet{kottelenberg_Gender}	&	NLSCY	&	-females: better parent-child relationship and 	&	-cognition	&	No	&	Yes	\\ 
	&		&	interactions across diverse measures	&	- socio-emotional outcomes	&		&		\\ 
	&		&	- no precise difference in cognition	&	-parental child relationship and quality of interactions	&		&		\\ 
	&		&		&	-maternal labor supply	&		&		\\ \\ \midrule
\citet{Baker-Milligan_2013_Boy-Girl-Differences}	&	Observational studies in three countries	&	Gender differences in parental investment 	&	-parental investment across different ages	&	No	&	Yes	\\ 
	&	-Canada: NLSCY (ages 1 to 5)	&	-no difference in mother's time at home 	&		&		&		\\ 
	&	-UK: Millennium Cohort Study (ages 1 to 7)	&	-females: more investment in teaching activities	&		&		&		\\ 
	&	-US: ECLS-B (ages 1 to 4)	&	-males: more father's investment at older ages	&		&		&		\\ \\ \midrule
\citet{Magnuson_Kelchen_Duncan_etal_2016_ECRQ}	&	23 programs (meta-analysis)	&	No gender differences, in general	&	-all programs: cognition	&	No	&	No	\\ 
	&	- at least 10 controls	&	- males/females: cognitive benefits	&	-some programs: achievement, behavior,	&		&		\\ 
	&	- from 1960 to 2013	&	- no effect on behavior or mental health	&	adult outcomes	&		&		\\ 
	&	- $<$ 50\%attrition	&		&		&		&		\\ 
	&	- RCTs	&		&		&		&		\\ \\ \midrule
\citet{Schore_2017_IMHJ}	&	Literature survey	&	Sex differences in brain maturation 	&	- brain maturation (right brain development)	&	N/A	&	Yes	\\ 
	&		&	-males: less time spent with mothers, more	&	- daycare behavior	&		&		\\ 
	&		&	sensitive to early infections and endotoxins; 	&	- maternal interaction 	&		&		\\ 
	&		&	respond poorly to daycare settings; amplify stress	&		&		&		\\ 
	&		&	more sensitive to single mother environment	&		&		&		\\ 
	&		&	-females: more rapid brain maturation	&		&		&		\\ \bottomrule
\end{tabular}											
\begin{tablenotes}
\Large
\item Note: This table presents a summary of papers studying early-life gender differences. (1) lists the paper; (2) lists the main program or sample of analysis; (3) lists the main finding with respect to gender differences; (4) the outcomes analyzed; (5) reports if the paper assesses or discusses the quality of the childcare setting; (6) reports if the paper assesses or discusses measures of home quality.
\end{tablenotes}
\end{threeparttable}
\end{adjustbox}
\end{sidewaystable} 
