\section{Alternative Definitions of Control Substitution and Two-Sided Statistical Tests} \label{appendix:vsensitivity}

\subsection{Alternative Definitions of Control Substitution}

\noindent In the main paper, we let $V$ be a dummy variable indicating whether or not the child attended alternative childcare arrangements. As we discuss in Section~\ref{section:background}, this dummy variable is a summary of a more complex reality in which children attend alternatives different months between ages 0 to 5. In this appendix, we explore three different alternative definitions of $V$: we let $V$ indicate if children attend alternatives (i) $2/5$ of the time between ages 0 to 5; (ii) $3/5$ of the time between ages 0 to 5; and (iii) $4/5$ of the time between ages 0 to 5. For each of these cases, we present a summary table of treatment effects analogous to Table~\ref{table:tescombined} in the main paper.\\

\noindent The results are robust to different choices for modeling $V$. What matters is the extensive margin decision to enroll children into alternative childcare arrangements, and not the intensive margin decision of the number of months they attend between ages 0 to 5.

\newgeometry{top=.6in, bottom=.8in, left=.8in, right=.8in}
\begin{table}[!htbp]
\centering
\begin{threeparttable}
\caption{Treatment Effects on Selected Outcomes, Control Substitution if Attended Treatment Alternatives $2/5$ of Time between Ages 0 to 5}\label{table:tescombinedv3}
\begin{scriptsize}
  \begin{tabular}{ccccccccccc}
  \toprule
   Category & Variable & Age & (1) & (2) & (3) & (4) & (5) & (6)\\

    \midrule
     \multicolumn{9}{c}{\textbf{\emph{Females}}} \\ \\
    \mc{1}{l}{\scriptsize{Parental Income}} &    \mc{1}{l}{\scriptsize{Parental Labor Income}} & \mc{1}{c}{\scriptsize{3.5}} & \mc{1}{c}{\scriptsize{2,756}} & \mc{1}{c}{\scriptsize{3,277}} & \mc{1}{c}{\scriptsize{5,464}} & \mc{1}{c}{\scriptsize{7,165}} & \mc{1}{c}{\scriptsize{-6,467}} & \mc{1}{c}{\scriptsize{-1,698}} \\  

  &   &  & \mc{1}{c}{\scriptsize{(0.198)}} & \mc{1}{c}{\scriptsize{(0.238)}} & \mc{1}{c}{\scriptsize{\textbf{(0.050)}}} & \mc{1}{c}{\scriptsize{\textbf{(0.040)}}} & \mc{1}{c}{\scriptsize{(0.683)}} & \mc{1}{c}{\scriptsize{(0.574)}} \\  

  &   & \mc{1}{c}{\scriptsize{12}} & \mc{1}{c}{\scriptsize{13,633}} & \mc{1}{c}{\scriptsize{19,386}} & \mc{1}{c}{\scriptsize{25,070}} & \mc{1}{c}{\scriptsize{25,917}} & \mc{1}{c}{\scriptsize{2,221}} & \mc{1}{c}{\scriptsize{6,214}} \\  

  &   &  & \mc{1}{c}{\scriptsize{\textbf{(0.040)}}} & \mc{1}{c}{\scriptsize{\textbf{(0.030)}}} & \mc{1}{c}{\scriptsize{\textbf{(0.010)}}} & \mc{1}{c}{\scriptsize{\textbf{(0.000)}}} & \mc{1}{c}{\scriptsize{(0.386)}} & \mc{1}{c}{\scriptsize{(0.238)}} \\  

  &   & \mc{1}{c}{\scriptsize{15}} & \mc{1}{c}{\scriptsize{8,565}} & \mc{1}{c}{\scriptsize{9,322}} & \mc{1}{c}{\scriptsize{9,108}} & \mc{1}{c}{\scriptsize{8,866}} & \mc{1}{c}{\scriptsize{3,588}} & \mc{1}{c}{\scriptsize{14,109}} \\  

  &   &  & \mc{1}{c}{\scriptsize{\textbf{(0.069)}}} & \mc{1}{c}{\scriptsize{\textbf{(0.089)}}} & \mc{1}{c}{\scriptsize{\textbf{(0.099)}}} & \mc{1}{c}{\scriptsize{\textbf{(0.059)}}} & \mc{1}{c}{\scriptsize{(0.297)}} & \mc{1}{c}{\scriptsize{\textbf{(0.010)}}} \\  

  &   & \mc{1}{c}{\scriptsize{21}} & \mc{1}{c}{\scriptsize{5,708}} & \mc{1}{c}{\scriptsize{6,944}} & \mc{1}{c}{\scriptsize{10,481}} & \mc{1}{c}{\scriptsize{8,526}} & \mc{1}{c}{\scriptsize{3,874}} & \mc{1}{c}{\scriptsize{1,224}} \\  

   &  &  & \mc{1}{c}{\scriptsize{(0.129)}} & \mc{1}{c}{\scriptsize{(0.158)}} & \mc{1}{c}{\scriptsize{(0.119)}} & \mc{1}{c}{\scriptsize{(0.109)}} & \mc{1}{c}{\scriptsize{(0.248)}} & \mc{1}{c}{\scriptsize{(0.406)}} \\  

    \mc{1}{l}{\scriptsize{Education}} &    \mc{1}{l}{\scriptsize{Graduated High School}} & \mc{1}{c}{\scriptsize{30}} & \mc{1}{c}{\scriptsize{0.253}} & \mc{1}{c}{\scriptsize{0.110}} & \mc{1}{c}{\scriptsize{0.144}} & \mc{1}{c}{\scriptsize{0.220}} & \mc{1}{c}{\scriptsize{0.038}} & \mc{1}{c}{\scriptsize{0.095}} \\  

 &    &  & \mc{1}{c}{\scriptsize{\textbf{(0.020)}}} & \mc{1}{c}{\scriptsize{(0.218)}} & \mc{1}{c}{\scriptsize{(0.168)}} & \mc{1}{c}{\scriptsize{\textbf{(0.030)}}} & \mc{1}{c}{\scriptsize{(0.416)}} & \mc{1}{c}{\scriptsize{(0.297)}} \\  

  &  \mc{1}{l}{\scriptsize{Graduated 4-year College}} & \mc{1}{c}{\scriptsize{30}} & \mc{1}{c}{\scriptsize{0.134}} & \mc{1}{c}{\scriptsize{0.119}} & \mc{1}{c}{\scriptsize{0.151}} & \mc{1}{c}{\scriptsize{0.165}} & \mc{1}{c}{\scriptsize{-0.005}} & \mc{1}{c}{\scriptsize{0.003}} \\  

  &   &  & \mc{1}{c}{\scriptsize{(0.109)}} & \mc{1}{c}{\scriptsize{(0.168)}} & \mc{1}{c}{\scriptsize{\textbf{(0.089)}}} & \mc{1}{c}{\scriptsize{\textbf{(0.030)}}} & \mc{1}{c}{\scriptsize{(0.465)}} & \mc{1}{c}{\scriptsize{(0.525)}} \\  

  &  \mc{1}{l}{\scriptsize{Years of Education}} & \mc{1}{c}{\scriptsize{30}} & \mc{1}{c}{\scriptsize{2.143}} & \mc{1}{c}{\scriptsize{1.715}} & \mc{1}{c}{\scriptsize{2.016}} & \mc{1}{c}{\scriptsize{2.373}} & \mc{1}{c}{\scriptsize{0.957}} & \mc{1}{c}{\scriptsize{0.802}} \\  

   &  &  & \mc{1}{c}{\scriptsize{\textbf{(0.000)}}} & \mc{1}{c}{\scriptsize{\textbf{(0.000)}}} & \mc{1}{c}{\scriptsize{\textbf{(0.000)}}} & \mc{1}{c}{\scriptsize{\textbf{(0.000)}}} & \mc{1}{c}{\scriptsize{(0.228)}} & \mc{1}{c}{\scriptsize{(0.238)}} \\  

     \mc{1}{l}{\scriptsize{Labor Income}} &   \mc{1}{l}{\scriptsize{Employed}} & \mc{1}{c}{\scriptsize{30}} & \mc{1}{c}{\scriptsize{0.131}} & \mc{1}{c}{\scriptsize{0.079}} & \mc{1}{c}{\scriptsize{0.064}} & \mc{1}{c}{\scriptsize{0.109}} & \mc{1}{c}{\scriptsize{0.146}} & \mc{1}{c}{\scriptsize{0.215}} \\  

  &   &  & \mc{1}{c}{\scriptsize{(0.129)}} & \mc{1}{c}{\scriptsize{(0.218)}} & \mc{1}{c}{\scriptsize{(0.218)}} & \mc{1}{c}{\scriptsize{(0.178)}} & \mc{1}{c}{\scriptsize{(0.277)}} & \mc{1}{c}{\scriptsize{\textbf{(0.089)}}} \\  

  &  \mc{1}{l}{\scriptsize{Labor Income}} & \mc{1}{c}{\scriptsize{30}} & \mc{1}{c}{\scriptsize{2,548}} & \mc{1}{c}{\scriptsize{2,412}} & \mc{1}{c}{\scriptsize{3,322}} & \mc{1}{c}{\scriptsize{3,955}} & \mc{1}{c}{\scriptsize{4,670}} & \mc{1}{c}{\scriptsize{1,179}} \\  

   &  &  & \mc{1}{c}{\scriptsize{(0.327)}} & \mc{1}{c}{\scriptsize{(0.307)}} & \mc{1}{c}{\scriptsize{(0.356)}} & \mc{1}{c}{\scriptsize{(0.287)}} & \mc{1}{c}{\scriptsize{(0.356)}} & \mc{1}{c}{\scriptsize{(0.475)}} \\  

     \mc{1}{l}{\scriptsize{Crime}} &   \mc{1}{l}{\scriptsize{Total Felony Arrests}} & \mc{1}{c}{\scriptsize{Mid-30s}} & \mc{1}{c}{\scriptsize{-0.328}} & \mc{1}{c}{\scriptsize{-0.394}} & \mc{1}{c}{\scriptsize{-0.533}} & \mc{1}{c}{\scriptsize{-0.415}} & \mc{1}{c}{\scriptsize{0.048}} & \mc{1}{c}{\scriptsize{0.124}} \\  

  &   &  & \mc{1}{c}{\scriptsize{\textbf{(0.099)}}} & \mc{1}{c}{\scriptsize{\textbf{(0.079)}}} & \mc{1}{c}{\scriptsize{\textbf{(0.099)}}} & \mc{1}{c}{\scriptsize{\textbf{(0.089)}}} & \mc{1}{c}{\scriptsize{(0.634)}} & \mc{1}{c}{\scriptsize{(0.871)}} \\  

  &  \mc{1}{l}{\scriptsize{Total Misdemeanor Arrests}} & \mc{1}{c}{\scriptsize{Mid-30s}} & \mc{1}{c}{\scriptsize{-0.973}} & \mc{1}{c}{\scriptsize{-1.212}} & \mc{1}{c}{\scriptsize{-1.419}} & \mc{1}{c}{\scriptsize{-1.097}} & \mc{1}{c}{\scriptsize{-0.220}} & \mc{1}{c}{\scriptsize{-0.138}} \\  

 &    &  & \mc{1}{c}{\scriptsize{\textbf{(0.030)}}} & \mc{1}{c}{\scriptsize{(0.119)}} & \mc{1}{c}{\scriptsize{(0.109)}} & \mc{1}{c}{\scriptsize{(0.139)}} & \mc{1}{c}{\scriptsize{(0.228)}} & \mc{1}{c}{\scriptsize{(0.317)}} \\  

     \mc{1}{l}{\scriptsize{Health}} &   \mc{1}{l}{\scriptsize{Self-reported drug user}} & \mc{1}{c}{\scriptsize{Mid-30s}} & \mc{1}{c}{\scriptsize{-0.033}} & \mc{1}{c}{\scriptsize{-0.039}} & \mc{1}{c}{\scriptsize{-0.047}} & \mc{1}{c}{\scriptsize{-0.027}} & \mc{1}{c}{\scriptsize{0.010}} & \mc{1}{c}{\scriptsize{0.083}} \\  

 &    &  & \mc{1}{c}{\scriptsize{(0.376)}} & \mc{1}{c}{\scriptsize{(0.337)}} & \mc{1}{c}{\scriptsize{(0.317)}} & \mc{1}{c}{\scriptsize{(0.406)}} & \mc{1}{c}{\scriptsize{(0.436)}} & \mc{1}{c}{\scriptsize{(0.614)}} \\  

 &   \mc{1}{l}{\scriptsize{Systolic Blood Pressure (mm Hg)}} & \mc{1}{c}{\scriptsize{Mid-30s}} & \mc{1}{c}{\scriptsize{-2.899}} & \mc{1}{c}{\scriptsize{-3.034}} &  & \mc{1}{c}{\scriptsize{-7.800}} & \mc{1}{c}{\scriptsize{6.792}} & \mc{1}{c}{\scriptsize{2.494}} \\  

  &   &  & \mc{1}{c}{\scriptsize{(0.307)}} & \mc{1}{c}{\scriptsize{(0.317)}} &  & \mc{1}{c}{\scriptsize{\textbf{(0.099)}}} & \mc{1}{c}{\scriptsize{(0.594)}} & \mc{1}{c}{\scriptsize{(0.634)}} \\  

   & \mc{1}{l}{\scriptsize{Diastolic Blood Pressure (mm Hg)}} & \mc{1}{c}{\scriptsize{Mid-30s}} & \mc{1}{c}{\scriptsize{-0.002}} & \mc{1}{c}{\scriptsize{2.341}} & \mc{1}{c}{\scriptsize{0.634}} & \mc{1}{c}{\scriptsize{-0.376}} & \mc{1}{c}{\scriptsize{3.056}} & \mc{1}{c}{\scriptsize{-0.375}} \\  

    & &  & \mc{1}{c}{\scriptsize{(0.455)}} & \mc{1}{c}{\scriptsize{(0.614)}} & \mc{1}{c}{\scriptsize{(0.525)}} & \mc{1}{c}{\scriptsize{(0.455)}} & \mc{1}{c}{\scriptsize{(0.515)}} & \mc{1}{c}{\scriptsize{(0.455)}} \\  

  &  \mc{1}{l}{\scriptsize{Hypertension}} & \mc{1}{c}{\scriptsize{Mid-30s}} & \mc{1}{c}{\scriptsize{0.172}} & \mc{1}{c}{\scriptsize{0.192}} & \mc{1}{c}{\scriptsize{0.123}} & \mc{1}{c}{\scriptsize{0.078}} & \mc{1}{c}{\scriptsize{0.267}} & \mc{1}{c}{\scriptsize{0.182}} \\  

   &  &  & \mc{1}{c}{\scriptsize{(0.891)}} & \mc{1}{c}{\scriptsize{(0.822)}} & \mc{1}{c}{\scriptsize{(0.752)}} & \mc{1}{c}{\scriptsize{(0.703)}} & \mc{1}{c}{\scriptsize{(0.723)}} & \mc{1}{c}{\scriptsize{(0.792)}} \\  

 
 % MALE
 
 
 
 
\midrule 
    \multicolumn{9}{c}{\textbf{\emph{Males}}} \\ \\

     \mc{1}{l}{\scriptsize{Parental Income}} &   \mc{1}{l}{\scriptsize{Parental Labor Income}} & \mc{1}{c}{\scriptsize{3.5}} & \mc{1}{c}{\scriptsize{1,036}} & \mc{1}{c}{\scriptsize{-1,185}} & \mc{1}{c}{\scriptsize{142}} & \mc{1}{c}{\scriptsize{2,393}} & \mc{1}{c}{\scriptsize{-17,476}} & \mc{1}{c}{\scriptsize{-14,914}} \\  

 &     &  & \mc{1}{c}{\scriptsize{(0.366)}} & \mc{1}{c}{\scriptsize{(0.673)}} & \mc{1}{c}{\scriptsize{(0.475)}} & \mc{1}{c}{\scriptsize{(0.198)}} & \mc{1}{c}{\scriptsize{(0.921)}} & \mc{1}{c}{\scriptsize{(0.881)}} \\  

  &   & \mc{1}{c}{\scriptsize{12}} & \mc{1}{c}{\scriptsize{7,085}} & \mc{1}{c}{\scriptsize{10,384}} & \mc{1}{c}{\scriptsize{12,334}} & \mc{1}{c}{\scriptsize{9,751}} & \mc{1}{c}{\scriptsize{-29,130}} & \mc{1}{c}{\scriptsize{-29,347}} \\  

  &   &  & \mc{1}{c}{\scriptsize{\textbf{(0.059)}}} & \mc{1}{c}{\scriptsize{\textbf{(0.040)}}} & \mc{1}{c}{\scriptsize{\textbf{(0.010)}}} & \mc{1}{c}{\scriptsize{\textbf{(0.020)}}} & \mc{1}{c}{\scriptsize{(0.881)}} & \mc{1}{c}{\scriptsize{(0.822)}} \\  

  &   & \mc{1}{c}{\scriptsize{15}} & \mc{1}{c}{\scriptsize{8,488}} & \mc{1}{c}{\scriptsize{7,185}} & \mc{1}{c}{\scriptsize{7,062}} & \mc{1}{c}{\scriptsize{5,829}} & \mc{1}{c}{\scriptsize{-12,275}} & \mc{1}{c}{\scriptsize{-15,574}} \\  

  &   &  & \mc{1}{c}{\scriptsize{\textbf{(0.059)}}} & \mc{1}{c}{\scriptsize{(0.139)}} & \mc{1}{c}{\scriptsize{(0.149)}} & \mc{1}{c}{\scriptsize{(0.218)}} & \mc{1}{c}{\scriptsize{(0.446)}} & \mc{1}{c}{\scriptsize{(0.663)}} \\  

  &   & \mc{1}{c}{\scriptsize{21}} & \mc{1}{c}{\scriptsize{12,732}} & \mc{1}{c}{\scriptsize{12,650}} & \mc{1}{c}{\scriptsize{12,960}} & \mc{1}{c}{\scriptsize{8,526}} & \mc{1}{c}{\scriptsize{-2,048}} & \mc{1}{c}{\scriptsize{-5,980}} \\  

  &   &  & \mc{1}{c}{\scriptsize{\textbf{(0.020)}}} & \mc{1}{c}{\scriptsize{\textbf{(0.069)}}} & \mc{1}{c}{\scriptsize{\textbf{(0.079)}}} & \mc{1}{c}{\scriptsize{\textbf{(0.069)}}} & \mc{1}{c}{\scriptsize{(0.228)}} & \mc{1}{c}{\scriptsize{(0.594)}} \\  

    \mc{1}{l}{\scriptsize{Education}} &  \mc{1}{l}{\scriptsize{Graduated High School}} & \mc{1}{c}{\scriptsize{30}} & \mc{1}{c}{\scriptsize{0.073}} & \mc{1}{c}{\scriptsize{0.130}} & \mc{1}{c}{\scriptsize{0.156}} & \mc{1}{c}{\scriptsize{0.094}} & \mc{1}{c}{\scriptsize{-0.002}} & \mc{1}{c}{\scriptsize{-0.093}} \\  

  &    &  & \mc{1}{c}{\scriptsize{(0.228)}} & \mc{1}{c}{\scriptsize{(0.139)}} & \mc{1}{c}{\scriptsize{(0.109)}} & \mc{1}{c}{\scriptsize{(0.238)}} & \mc{1}{c}{\scriptsize{(0.554)}} & \mc{1}{c}{\scriptsize{(0.584)}} \\  

  &  \mc{1}{l}{\scriptsize{Graduated 4-year College}} & \mc{1}{c}{\scriptsize{30}} & \mc{1}{c}{\scriptsize{0.170}} & \mc{1}{c}{\scriptsize{0.178}} & \mc{1}{c}{\scriptsize{0.156}} & \mc{1}{c}{\scriptsize{0.112}} & \mc{1}{c}{\scriptsize{0.513}} & \mc{1}{c}{\scriptsize{0.260}} \\  

  &   &  & \mc{1}{c}{\scriptsize{\textbf{(0.079)}}} & \mc{1}{c}{\scriptsize{(0.119)}} & \mc{1}{c}{\scriptsize{(0.149)}} & \mc{1}{c}{\scriptsize{(0.168)}} & \mc{1}{c}{\scriptsize{(0.119)}} & \mc{1}{c}{\scriptsize{\textbf{(0.000)}}} \\  

  &  \mc{1}{l}{\scriptsize{Years of Education}} & \mc{1}{c}{\scriptsize{30}} & \mc{1}{c}{\scriptsize{0.525}} & \mc{1}{c}{\scriptsize{0.785}} & \mc{1}{c}{\scriptsize{0.710}} & \mc{1}{c}{\scriptsize{0.425}} & \mc{1}{c}{\scriptsize{1.749}} & \mc{1}{c}{\scriptsize{0.595}} \\  

  &   &  & \mc{1}{c}{\scriptsize{(0.188)}} & \mc{1}{c}{\scriptsize{\textbf{(0.079)}}} & \mc{1}{c}{\scriptsize{(0.129)}} & \mc{1}{c}{\scriptsize{(0.198)}} & \mc{1}{c}{\scriptsize{\textbf{(0.059)}}} & \mc{1}{c}{\scriptsize{(0.178)}} \\  

   \mc{1}{l}{\scriptsize{Labor Income}} &   \mc{1}{l}{\scriptsize{Employed}} & \mc{1}{c}{\scriptsize{30}} & \mc{1}{c}{\scriptsize{0.119}} & \mc{1}{c}{\scriptsize{0.182}} & \mc{1}{c}{\scriptsize{0.197}} & \mc{1}{c}{\scriptsize{0.217}} & \mc{1}{c}{\scriptsize{0.174}} & \mc{1}{c}{\scriptsize{0.148}} \\  

 &    &  & \mc{1}{c}{\scriptsize{(0.129)}} & \mc{1}{c}{\scriptsize{\textbf{(0.020)}}} & \mc{1}{c}{\scriptsize{\textbf{(0.030)}}} & \mc{1}{c}{\scriptsize{\textbf{(0.000)}}} & \mc{1}{c}{\scriptsize{(0.307)}} & \mc{1}{c}{\scriptsize{(0.188)}} \\  

  &  \mc{1}{l}{\scriptsize{Labor Income}} & \mc{1}{c}{\scriptsize{30}} & \mc{1}{c}{\scriptsize{19,810}} & \mc{1}{c}{\scriptsize{27,373}} & \mc{1}{c}{\scriptsize{26,959}} & \mc{1}{c}{\scriptsize{20,998}} & \mc{1}{c}{\scriptsize{69,187}} & \mc{1}{c}{\scriptsize{27,682}} \\  

   &  &  & \mc{1}{c}{\scriptsize{(0.109)}} & \mc{1}{c}{\scriptsize{(0.208)}} & \mc{1}{c}{\scriptsize{(0.218)}} & \mc{1}{c}{\scriptsize{\textbf{(0.099)}}} & \mc{1}{c}{\scriptsize{(0.139)}} & \mc{1}{c}{\scriptsize{\textbf{(0.099)}}} \\  

    \mc{1}{l}{\scriptsize{Crime}} &  \mc{1}{l}{\scriptsize{Total Felony Arrests}} & \mc{1}{c}{\scriptsize{Mid-30s}} & \mc{1}{c}{\scriptsize{0.196}} & \mc{1}{c}{\scriptsize{0.392}} & \mc{1}{c}{\scriptsize{0.505}} & \mc{1}{c}{\scriptsize{0.689}} & \mc{1}{c}{\scriptsize{-0.034}} & \mc{1}{c}{\scriptsize{-0.629}} \\  

 &    &  & \mc{1}{c}{\scriptsize{(0.644)}} & \mc{1}{c}{\scriptsize{(0.644)}} & \mc{1}{c}{\scriptsize{(0.683)}} & \mc{1}{c}{\scriptsize{(0.822)}} & \mc{1}{c}{\scriptsize{(0.614)}} & \mc{1}{c}{\scriptsize{(0.347)}} \\  

 &   \mc{1}{l}{\scriptsize{Total Misdemeanor Arrests}} & \mc{1}{c}{\scriptsize{Mid-30s}} & \mc{1}{c}{\scriptsize{-0.501}} & \mc{1}{c}{\scriptsize{-0.243}} & \mc{1}{c}{\scriptsize{-0.317}} & \mc{1}{c}{\scriptsize{-0.356}} & \mc{1}{c}{\scriptsize{0.357}} & \mc{1}{c}{\scriptsize{-0.434}} \\  

  &   &  & \mc{1}{c}{\scriptsize{(0.119)}} & \mc{1}{c}{\scriptsize{(0.277)}} & \mc{1}{c}{\scriptsize{(0.238)}} & \mc{1}{c}{\scriptsize{(0.277)}} & \mc{1}{c}{\scriptsize{(0.614)}} & \mc{1}{c}{\scriptsize{(0.228)}} \\  

   \mc{1}{l}{\scriptsize{Health}} &   \mc{1}{l}{\scriptsize{Self-reported drug user}} & \mc{1}{c}{\scriptsize{Mid-30s}} & \mc{1}{c}{\scriptsize{-0.333}} & \mc{1}{c}{\scriptsize{-0.398}} & \mc{1}{c}{\scriptsize{-0.418}} & \mc{1}{c}{\scriptsize{-0.414}} & \mc{1}{c}{\scriptsize{0.149}} & \mc{1}{c}{\scriptsize{0.149}} \\  

  &   &  & \mc{1}{c}{\scriptsize{\textbf{(0.030)}}} & \mc{1}{c}{\scriptsize{\textbf{(0.020)}}} & \mc{1}{c}{\scriptsize{\textbf{(0.010)}}} & \mc{1}{c}{\scriptsize{\textbf{(0.010)}}} & \mc{1}{c}{\scriptsize{(0.406)}} & \mc{1}{c}{\scriptsize{(0.554)}} \\  

  &  \mc{1}{l}{\scriptsize{Systolic Blood Pressure (mm Hg)}} & \mc{1}{c}{\scriptsize{Mid-30s}} & \mc{1}{c}{\scriptsize{-9.791}} & \mc{1}{c}{\scriptsize{-19.475}} & \mc{1}{c}{\scriptsize{-19.868}} & \mc{1}{c}{\scriptsize{-21.234}} & \mc{1}{c}{\scriptsize{-12.168}} & \mc{1}{c}{\scriptsize{-18.841}} \\  

  &   &  & \mc{1}{c}{\scriptsize{(0.129)}} & \mc{1}{c}{\scriptsize{\textbf{(0.000)}}} & \mc{1}{c}{\scriptsize{\textbf{(0.000)}}} & \mc{1}{c}{\scriptsize{\textbf{(0.050)}}} & \mc{1}{c}{\scriptsize{\textbf{(0.099)}}} & \mc{1}{c}{\scriptsize{\textbf{(0.000)}}} \\  

  &  \mc{1}{l}{\scriptsize{Diastolic Blood Pressure (mm Hg)}} & \mc{1}{c}{\scriptsize{Mid-30s}} & \mc{1}{c}{\scriptsize{-10.854}} & \mc{1}{c}{\scriptsize{-19.401}} & \mc{1}{c}{\scriptsize{-20.255}} & \mc{1}{c}{\scriptsize{-19.838}} &  & \mc{1}{c}{\scriptsize{-6.102}} \\  

  &   &  & \mc{1}{c}{\scriptsize{\textbf{(0.040)}}} & \mc{1}{c}{\scriptsize{\textbf{(0.000)}}} & \mc{1}{c}{\scriptsize{\textbf{(0.000)}}} & \mc{1}{c}{\scriptsize{\textbf{(0.000)}}} &  & \mc{1}{c}{\scriptsize{\textbf{(0.000)}}} \\  

  &  \mc{1}{l}{\scriptsize{Hypertension}} & \mc{1}{c}{\scriptsize{Mid-30s}} & \mc{1}{c}{\scriptsize{-0.291}} & \mc{1}{c}{\scriptsize{-0.384}} & \mc{1}{c}{\scriptsize{-0.392}} & \mc{1}{c}{\scriptsize{-0.398}} & \mc{1}{c}{\scriptsize{-0.693}} & \mc{1}{c}{\scriptsize{-0.768}} \\  

   &  &  & \mc{1}{c}{\scriptsize{\textbf{(0.069)}}} & \mc{1}{c}{\scriptsize{\textbf{(0.010)}}} & \mc{1}{c}{\scriptsize{\textbf{(0.030)}}} & \mc{1}{c}{\scriptsize{\textbf{(0.000)}}} & \mc{1}{c}{\scriptsize{\textbf{(0.010)}}} & \mc{1}{c}{\scriptsize{\textbf{(0.000)}}} \\  


  
  
  
\bottomrule
    \end{tabular} 
\end{scriptsize}
\begin{tablenotes}
\tiny
Note: This table shows the treatment effects for categories outcomes that are important for our benefit/cost analysis. Systolic and diastolic blood pressure are measured in terms of mm Hg. Each column present estimates for the following parameters: (1) $\mathbb{E} \left [ \bm{Y}^1 -  \bm{Y}^0 | \bm{B} \in \mathcal{B}_{0} \right]$ (no controls); (2) $\mathbb{E} \left [ \bm{Y}^1 -  \bm{Y}^0 | \bm{B} \in \mathcal{B}_{0} \right] $(controls); (3) $\mathbb{E} \left [ \bm{Y}^1 | R = 1 \right] -  \mathbb{E} \left [ \bm{Y}^0 | R = 0,V = 0  \right]$ (no controls); (4) $\mathbb{E} \left [ \bm{Y}^1 -  \bm{Y}_H^0 | \bm{B} \in \mathcal{B}_{0} \right]$ (controls);  (5) $\mathbb{E} \left [ \bm{Y}^1 | R = 1 \right] -  \mathbb{E} \left [ \bm{Y}^0 | R = 0,V = 1 \right]$ (no controls); (6) $\mathbb{E} \left [ \bm{Y}^1 -  \bm{Y}_C^0 | \bm{B} \in \mathcal{B}_{0} \right]$ (controls). We account for the following background variables ($\bm{B}$): Apgar scores at minutes 1 and 5 and the high-risk index. We define the high-risk index in Appendix~\ref{appendix:background} and explain how we choose the control variables in Appendix~\ref{appendix:bvariables}. Columns (2), (4), and (6) correct for item non-response and attrition using inverse probability weighting as we explain in Appendix~\ref{app:method_partialobs}. Inference is based on non-parametric, one-sided $p$-values from the empirical bootstrap distribution. We highlight point estimates significant at the $10\%$ level. See Appendix~\ref{appendix:vsensitivity} for two-sided $p$-values.
\end{tablenotes}
\end{threeparttable}
\end{table}

\begin{table}[!htbp]
\centering
\begin{threeparttable}
\caption{Treatment Effects on Selected Outcomes, Control Substitution if Attended Treatment Alternatives $3/5$ of Time between Ages 0 to 5}\label{table:tescombinedv2}
\begin{scriptsize}
  \begin{tabular}{ccccccccccc}
  \toprule
   Category & Variable & Age & (1) & (2) & (3) & (4) & (5) & (6)\\

    \midrule
     \multicolumn{9}{c}{\textbf{\emph{Females}}} \\ \\
     \mc{1}{l}{\scriptsize{Parental Income}} &    \mc{1}{l}{\scriptsize{Parental Labor Income}} & \mc{1}{c}{\scriptsize{3.5}} & \mc{1}{c}{\scriptsize{2,756}} & \mc{1}{c}{\scriptsize{3,277}} & \mc{1}{c}{\scriptsize{6,384}} & \mc{1}{c}{\scriptsize{8,257}} & \mc{1}{c}{\scriptsize{-2,419}} & \mc{1}{c}{\scriptsize{334}} \\  

 &    &  & \mc{1}{c}{\scriptsize{(0.248)}} & \mc{1}{c}{\scriptsize{(0.208)}} & \mc{1}{c}{\scriptsize{(0.119)}} & \mc{1}{c}{\scriptsize{\textbf{(0.020)}}} & \mc{1}{c}{\scriptsize{(0.614)}} & \mc{1}{c}{\scriptsize{(0.426)}} \\  

  &   & \mc{1}{c}{\scriptsize{12}} & \mc{1}{c}{\scriptsize{13,633}} & \mc{1}{c}{\scriptsize{19,386}} & \mc{1}{c}{\scriptsize{21,331}} & \mc{1}{c}{\scriptsize{21,912}} & \mc{1}{c}{\scriptsize{15,568}} & \mc{1}{c}{\scriptsize{18,687}} \\  

  &   &  & \mc{1}{c}{\scriptsize{\textbf{(0.059)}}} & \mc{1}{c}{\scriptsize{\textbf{(0.030)}}} & \mc{1}{c}{\scriptsize{\textbf{(0.050)}}} & \mc{1}{c}{\scriptsize{\textbf{(0.010)}}} & \mc{1}{c}{\scriptsize{\textbf{(0.069)}}} & \mc{1}{c}{\scriptsize{\textbf{(0.040)}}} \\  

   &  & \mc{1}{c}{\scriptsize{15}} & \mc{1}{c}{\scriptsize{8,565}} & \mc{1}{c}{\scriptsize{9,322}} & \mc{1}{c}{\scriptsize{6,759}} & \mc{1}{c}{\scriptsize{7,803}} & \mc{1}{c}{\scriptsize{5,699}} & \mc{1}{c}{\scriptsize{14,228}} \\  

    & &  & \mc{1}{c}{\scriptsize{\textbf{(0.069)}}} & \mc{1}{c}{\scriptsize{\textbf{(0.079)}}} & \mc{1}{c}{\scriptsize{(0.168)}} & \mc{1}{c}{\scriptsize{(0.129)}} & \mc{1}{c}{\scriptsize{(0.228)}} & \mc{1}{c}{\scriptsize{\textbf{(0.030)}}} \\  

  &   & \mc{1}{c}{\scriptsize{21}} & \mc{1}{c}{\scriptsize{5,708}} & \mc{1}{c}{\scriptsize{6,944}} & \mc{1}{c}{\scriptsize{12,907}} & \mc{1}{c}{\scriptsize{8,065}} & \mc{1}{c}{\scriptsize{5,047}} & \mc{1}{c}{\scriptsize{4,429}} \\  

  &   &  & \mc{1}{c}{\scriptsize{(0.168)}} & \mc{1}{c}{\scriptsize{(0.198)}} & \mc{1}{c}{\scriptsize{\textbf{(0.099)}}} & \mc{1}{c}{\scriptsize{(0.149)}} & \mc{1}{c}{\scriptsize{(0.277)}} & \mc{1}{c}{\scriptsize{(0.248)}} \\  

      \mc{1}{l}{\scriptsize{Education}} &   \mc{1}{l}{\scriptsize{Graduated High School}} & \mc{1}{c}{\scriptsize{30}} & \mc{1}{c}{\scriptsize{0.253}} & \mc{1}{c}{\scriptsize{0.110}} & \mc{1}{c}{\scriptsize{0.160}} & \mc{1}{c}{\scriptsize{0.243}} & \mc{1}{c}{\scriptsize{0.028}} & \mc{1}{c}{\scriptsize{0.105}} \\  

 &    &  & \mc{1}{c}{\scriptsize{\textbf{(0.000)}}} & \mc{1}{c}{\scriptsize{(0.198)}} & \mc{1}{c}{\scriptsize{(0.228)}} & \mc{1}{c}{\scriptsize{\textbf{(0.069)}}} & \mc{1}{c}{\scriptsize{(0.396)}} & \mc{1}{c}{\scriptsize{(0.198)}} \\  

  &  \mc{1}{l}{\scriptsize{Graduated 4-year College}} & \mc{1}{c}{\scriptsize{30}} & \mc{1}{c}{\scriptsize{0.134}} & \mc{1}{c}{\scriptsize{0.119}} & \mc{1}{c}{\scriptsize{0.145}} & \mc{1}{c}{\scriptsize{0.154}} & \mc{1}{c}{\scriptsize{0.041}} & \mc{1}{c}{\scriptsize{0.076}} \\  

   &  &  & \mc{1}{c}{\scriptsize{\textbf{(0.050)}}} & \mc{1}{c}{\scriptsize{\textbf{(0.099)}}} & \mc{1}{c}{\scriptsize{\textbf{(0.059)}}} & \mc{1}{c}{\scriptsize{\textbf{(0.069)}}} & \mc{1}{c}{\scriptsize{(0.396)}} & \mc{1}{c}{\scriptsize{(0.277)}} \\  

  &  \mc{1}{l}{\scriptsize{Years of Education}} & \mc{1}{c}{\scriptsize{30}} & \mc{1}{c}{\scriptsize{2.143}} & \mc{1}{c}{\scriptsize{1.715}} & \mc{1}{c}{\scriptsize{2.089}} & \mc{1}{c}{\scriptsize{2.461}} & \mc{1}{c}{\scriptsize{1.142}} & \mc{1}{c}{\scriptsize{1.264}} \\  

   &  &  & \mc{1}{c}{\scriptsize{\textbf{(0.000)}}} & \mc{1}{c}{\scriptsize{\textbf{(0.000)}}} & \mc{1}{c}{\scriptsize{\textbf{(0.000)}}} & \mc{1}{c}{\scriptsize{\textbf{(0.000)}}} & \mc{1}{c}{\scriptsize{\textbf{(0.099)}}} & \mc{1}{c}{\scriptsize{\textbf{(0.069)}}} \\  

      \mc{1}{l}{\scriptsize{Labor Income}} &   \mc{1}{l}{\scriptsize{Employed}} & \mc{1}{c}{\scriptsize{30}} & \mc{1}{c}{\scriptsize{0.131}} & \mc{1}{c}{\scriptsize{0.079}} & \mc{1}{c}{\scriptsize{0.072}} & \mc{1}{c}{\scriptsize{0.125}} & \mc{1}{c}{\scriptsize{0.092}} & \mc{1}{c}{\scriptsize{0.148}} \\  

  &   &  & \mc{1}{c}{\scriptsize{\textbf{(0.059)}}} & \mc{1}{c}{\scriptsize{(0.238)}} & \mc{1}{c}{\scriptsize{(0.267)}} & \mc{1}{c}{\scriptsize{\textbf{(0.099)}}} & \mc{1}{c}{\scriptsize{(0.297)}} & \mc{1}{c}{\scriptsize{(0.129)}} \\  

 &   \mc{1}{l}{\scriptsize{Labor Income}} & \mc{1}{c}{\scriptsize{30}} & \mc{1}{c}{\scriptsize{2,548}} & \mc{1}{c}{\scriptsize{2,412}} & \mc{1}{c}{\scriptsize{3,176}} & \mc{1}{c}{\scriptsize{3,710}} & \mc{1}{c}{\scriptsize{5,076}} & \mc{1}{c}{\scriptsize{2,706}} \\  

 &    &  & \mc{1}{c}{\scriptsize{(0.356)}} & \mc{1}{c}{\scriptsize{(0.396)}} & \mc{1}{c}{\scriptsize{(0.386)}} & \mc{1}{c}{\scriptsize{(0.327)}} & \mc{1}{c}{\scriptsize{(0.257)}} & \mc{1}{c}{\scriptsize{(0.337)}} \\  

      \mc{1}{l}{\scriptsize{Crime}} &   \mc{1}{l}{\scriptsize{Total Felony Arrests}} & \mc{1}{c}{\scriptsize{Mid-30s}} & \mc{1}{c}{\scriptsize{-0.328}} & \mc{1}{c}{\scriptsize{-0.394}} & \mc{1}{c}{\scriptsize{-0.449}} & \mc{1}{c}{\scriptsize{-0.391}} & \mc{1}{c}{\scriptsize{-0.192}} & \mc{1}{c}{\scriptsize{-0.093}} \\  

  &   &  & \mc{1}{c}{\scriptsize{\textbf{(0.079)}}} & \mc{1}{c}{\scriptsize{\textbf{(0.089)}}} & \mc{1}{c}{\scriptsize{(0.139)}} & \mc{1}{c}{\scriptsize{(0.109)}} & \mc{1}{c}{\scriptsize{(0.158)}} & \mc{1}{c}{\scriptsize{(0.297)}} \\  

  &  \mc{1}{l}{\scriptsize{Total Misdemeanor Arrests}} & \mc{1}{c}{\scriptsize{Mid-30s}} & \mc{1}{c}{\scriptsize{-0.973}} & \mc{1}{c}{\scriptsize{-1.212}} & \mc{1}{c}{\scriptsize{-0.949}} & \mc{1}{c}{\scriptsize{-0.960}} & \mc{1}{c}{\scriptsize{-1.038}} & \mc{1}{c}{\scriptsize{-0.669}} \\  

  &   &  & \mc{1}{c}{\scriptsize{\textbf{(0.040)}}} & \mc{1}{c}{\scriptsize{\textbf{(0.079)}}} & \mc{1}{c}{\scriptsize{(0.119)}} & \mc{1}{c}{\scriptsize{(0.208)}} & \mc{1}{c}{\scriptsize{\textbf{(0.099)}}} & \mc{1}{c}{\scriptsize{(0.168)}} \\  

      \mc{1}{l}{\scriptsize{Health}} &   \mc{1}{l}{\scriptsize{Self-reported drug user}} & \mc{1}{c}{\scriptsize{Mid-30s}} & \mc{1}{c}{\scriptsize{-0.033}} & \mc{1}{c}{\scriptsize{-0.039}} & \mc{1}{c}{\scriptsize{-0.003}} & \mc{1}{c}{\scriptsize{-0.003}} & \mc{1}{c}{\scriptsize{-0.062}} & \mc{1}{c}{\scriptsize{0.009}} \\  

  &   &  & \mc{1}{c}{\scriptsize{(0.356)}} & \mc{1}{c}{\scriptsize{(0.406)}} & \mc{1}{c}{\scriptsize{(0.495)}} & \mc{1}{c}{\scriptsize{(0.465)}} & \mc{1}{c}{\scriptsize{(0.317)}} & \mc{1}{c}{\scriptsize{(0.515)}} \\  

  &  \mc{1}{l}{\scriptsize{Systolic Blood Pressure (mm Hg)}} & \mc{1}{c}{\scriptsize{Mid-30s}} & \mc{1}{c}{\scriptsize{-2.899}} & \mc{1}{c}{\scriptsize{-3.034}} &  & \mc{1}{c}{\scriptsize{-3.566}} & \mc{1}{c}{\scriptsize{-5.636}} & \mc{1}{c}{\scriptsize{-7.102}} \\  

   &  &  & \mc{1}{c}{\scriptsize{(0.307)}} & \mc{1}{c}{\scriptsize{(0.287)}} &  & \mc{1}{c}{\scriptsize{(0.307)}} & \mc{1}{c}{\scriptsize{(0.277)}} & \mc{1}{c}{\scriptsize{(0.208)}} \\  

  &  \mc{1}{l}{\scriptsize{Diastolic Blood Pressure (mm Hg)}} & \mc{1}{c}{\scriptsize{Mid-30s}} & \mc{1}{c}{\scriptsize{-0.002}} & \mc{1}{c}{\scriptsize{2.341}} & \mc{1}{c}{\scriptsize{4.069}} & \mc{1}{c}{\scriptsize{1.881}} & \mc{1}{c}{\scriptsize{-1.058}} & \mc{1}{c}{\scriptsize{-3.358}} \\  

  &   &  & \mc{1}{c}{\scriptsize{(0.564)}} & \mc{1}{c}{\scriptsize{(0.584)}} & \mc{1}{c}{\scriptsize{(0.723)}} & \mc{1}{c}{\scriptsize{(0.644)}} & \mc{1}{c}{\scriptsize{(0.406)}} & \mc{1}{c}{\scriptsize{(0.297)}} \\  

   &  \mc{1}{l}{\scriptsize{Hypertension}} & \mc{1}{c}{\scriptsize{Mid-30s}} & \mc{1}{c}{\scriptsize{0.172}} & \mc{1}{c}{\scriptsize{0.192}} & \mc{1}{c}{\scriptsize{0.160}} & \mc{1}{c}{\scriptsize{0.118}} & \mc{1}{c}{\scriptsize{0.145}} & \mc{1}{c}{\scriptsize{0.089}} \\  

   &  &  & \mc{1}{c}{\scriptsize{(0.891)}} & \mc{1}{c}{\scriptsize{(0.832)}} & \mc{1}{c}{\scriptsize{(0.713)}} & \mc{1}{c}{\scriptsize{(0.723)}} & \mc{1}{c}{\scriptsize{(0.673)}} & \mc{1}{c}{\scriptsize{(0.663)}} \\  



     
     
     
 
 % MALE
 
 
 
\midrule 
     \mc{1}{l}{\scriptsize{Parental Income}} &   \mc{1}{l}{\scriptsize{Parental Labor Income}} & \mc{1}{c}{\scriptsize{3.5}} & \mc{1}{c}{\scriptsize{1,036}} & \mc{1}{c}{\scriptsize{-1,185}} & \mc{1}{c}{\scriptsize{2,057}} & \mc{1}{c}{\scriptsize{4,603}} & \mc{1}{c}{\scriptsize{-12,759}} & \mc{1}{c}{\scriptsize{-9,236}} \\  

 &     &  & \mc{1}{c}{\scriptsize{(0.376)}} & \mc{1}{c}{\scriptsize{(0.624)}} & \mc{1}{c}{\scriptsize{(0.297)}} & \mc{1}{c}{\scriptsize{\textbf{(0.079)}}} & \mc{1}{c}{\scriptsize{(0.960)}} & \mc{1}{c}{\scriptsize{(0.950)}} \\  

  &   & \mc{1}{c}{\scriptsize{12}} & \mc{1}{c}{\scriptsize{7,085}} & \mc{1}{c}{\scriptsize{10,384}} & \mc{1}{c}{\scriptsize{11,337}} & \mc{1}{c}{\scriptsize{8,591}} & \mc{1}{c}{\scriptsize{9,203}} & \mc{1}{c}{\scriptsize{3,888}} \\  

   &  &  & \mc{1}{c}{\scriptsize{\textbf{(0.069)}}} & \mc{1}{c}{\scriptsize{\textbf{(0.020)}}} & \mc{1}{c}{\scriptsize{\textbf{(0.010)}}} & \mc{1}{c}{\scriptsize{\textbf{(0.059)}}} & \mc{1}{c}{\scriptsize{(0.228)}} & \mc{1}{c}{\scriptsize{(0.337)}} \\  

  &   & \mc{1}{c}{\scriptsize{15}} & \mc{1}{c}{\scriptsize{8,488}} & \mc{1}{c}{\scriptsize{7,185}} & \mc{1}{c}{\scriptsize{4,945}} & \mc{1}{c}{\scriptsize{3,753}} & \mc{1}{c}{\scriptsize{5,537}} & \mc{1}{c}{\scriptsize{6,939}} \\  

  &   &  & \mc{1}{c}{\scriptsize{\textbf{(0.040)}}} & \mc{1}{c}{\scriptsize{\textbf{(0.079)}}} & \mc{1}{c}{\scriptsize{(0.168)}} & \mc{1}{c}{\scriptsize{(0.287)}} & \mc{1}{c}{\scriptsize{(0.386)}} & \mc{1}{c}{\scriptsize{(0.366)}} \\  

   &  & \mc{1}{c}{\scriptsize{21}} & \mc{1}{c}{\scriptsize{12,732}} & \mc{1}{c}{\scriptsize{12,650}} & \mc{1}{c}{\scriptsize{12,604}} & \mc{1}{c}{\scriptsize{7,786}} & \mc{1}{c}{\scriptsize{14,954}} & \mc{1}{c}{\scriptsize{8,330}} \\  

    & &  & \mc{1}{c}{\scriptsize{\textbf{(0.000)}}} & \mc{1}{c}{\scriptsize{\textbf{(0.069)}}} & \mc{1}{c}{\scriptsize{\textbf{(0.059)}}} & \mc{1}{c}{\scriptsize{(0.119)}} & \mc{1}{c}{\scriptsize{(0.327)}} & \mc{1}{c}{\scriptsize{(0.158)}} \\  

   \mc{1}{l}{\scriptsize{Education}} &    \mc{1}{l}{\scriptsize{Graduated High School}} & \mc{1}{c}{\scriptsize{30}} & \mc{1}{c}{\scriptsize{0.073}} & \mc{1}{c}{\scriptsize{0.130}} & \mc{1}{c}{\scriptsize{0.144}} & \mc{1}{c}{\scriptsize{0.104}} & \mc{1}{c}{\scriptsize{0.083}} & \mc{1}{c}{\scriptsize{-0.011}} \\  

  &   &  & \mc{1}{c}{\scriptsize{(0.297)}} & \mc{1}{c}{\scriptsize{(0.149)}} & \mc{1}{c}{\scriptsize{(0.158)}} & \mc{1}{c}{\scriptsize{(0.218)}} & \mc{1}{c}{\scriptsize{(0.376)}} & \mc{1}{c}{\scriptsize{(0.554)}} \\  

   & \mc{1}{l}{\scriptsize{Graduated 4-year College}} & \mc{1}{c}{\scriptsize{30}} & \mc{1}{c}{\scriptsize{0.170}} & \mc{1}{c}{\scriptsize{0.178}} & \mc{1}{c}{\scriptsize{0.126}} & \mc{1}{c}{\scriptsize{0.077}} & \mc{1}{c}{\scriptsize{0.472}} & \mc{1}{c}{\scriptsize{0.260}} \\  

  &   &  & \mc{1}{c}{\scriptsize{\textbf{(0.079)}}} & \mc{1}{c}{\scriptsize{\textbf{(0.099)}}} & \mc{1}{c}{\scriptsize{(0.178)}} & \mc{1}{c}{\scriptsize{(0.287)}} & \mc{1}{c}{\scriptsize{\textbf{(0.030)}}} & \mc{1}{c}{\scriptsize{\textbf{(0.000)}}} \\  

   & \mc{1}{l}{\scriptsize{Years of Education}} & \mc{1}{c}{\scriptsize{30}} & \mc{1}{c}{\scriptsize{0.525}} & \mc{1}{c}{\scriptsize{0.785}} & \mc{1}{c}{\scriptsize{0.574}} & \mc{1}{c}{\scriptsize{0.344}} & \mc{1}{c}{\scriptsize{1.771}} & \mc{1}{c}{\scriptsize{0.679}} \\  

  &   &  & \mc{1}{c}{\scriptsize{(0.168)}} & \mc{1}{c}{\scriptsize{\textbf{(0.079)}}} & \mc{1}{c}{\scriptsize{(0.188)}} & \mc{1}{c}{\scriptsize{(0.277)}} & \mc{1}{c}{\scriptsize{\textbf{(0.010)}}} & \mc{1}{c}{\scriptsize{\textbf{(0.089)}}} \\  

    \mc{1}{l}{\scriptsize{Labor Income}} &   \mc{1}{l}{\scriptsize{Employed}} & \mc{1}{c}{\scriptsize{30}} & \mc{1}{c}{\scriptsize{0.119}} & \mc{1}{c}{\scriptsize{0.182}} & \mc{1}{c}{\scriptsize{0.144}} & \mc{1}{c}{\scriptsize{0.147}} & \mc{1}{c}{\scriptsize{0.354}} & \mc{1}{c}{\scriptsize{0.343}} \\  

  &   &  & \mc{1}{c}{\scriptsize{(0.109)}} & \mc{1}{c}{\scriptsize{\textbf{(0.030)}}} & \mc{1}{c}{\scriptsize{(0.119)}} & \mc{1}{c}{\scriptsize{(0.119)}} & \mc{1}{c}{\scriptsize{\textbf{(0.030)}}} & \mc{1}{c}{\scriptsize{\textbf{(0.030)}}} \\  

   & \mc{1}{l}{\scriptsize{Labor Income}} & \mc{1}{c}{\scriptsize{30}} & \mc{1}{c}{\scriptsize{19,810}} & \mc{1}{c}{\scriptsize{27,373}} & \mc{1}{c}{\scriptsize{23,796}} & \mc{1}{c}{\scriptsize{17,442}} & \mc{1}{c}{\scriptsize{63,404}} & \mc{1}{c}{\scriptsize{32,179}} \\  

   &  &  & \mc{1}{c}{\scriptsize{\textbf{(0.079)}}} & \mc{1}{c}{\scriptsize{(0.178)}} & \mc{1}{c}{\scriptsize{(0.188)}} & \mc{1}{c}{\scriptsize{(0.178)}} & \mc{1}{c}{\scriptsize{\textbf{(0.099)}}} & \mc{1}{c}{\scriptsize{\textbf{(0.059)}}} \\  

    \mc{1}{l}{\scriptsize{Crime}} &   \mc{1}{l}{\scriptsize{Total Felony Arrests}} & \mc{1}{c}{\scriptsize{Mid-30s}} & \mc{1}{c}{\scriptsize{0.196}} & \mc{1}{c}{\scriptsize{0.392}} & \mc{1}{c}{\scriptsize{0.660}} & \mc{1}{c}{\scriptsize{0.886}} & \mc{1}{c}{\scriptsize{-0.470}} & \mc{1}{c}{\scriptsize{-0.408}} \\  

  &   &  & \mc{1}{c}{\scriptsize{(0.653)}} & \mc{1}{c}{\scriptsize{(0.693)}} & \mc{1}{c}{\scriptsize{(0.703)}} & \mc{1}{c}{\scriptsize{(0.832)}} & \mc{1}{c}{\scriptsize{(0.366)}} & \mc{1}{c}{\scriptsize{(0.366)}} \\  

  &  \mc{1}{l}{\scriptsize{Total Misdemeanor Arrests}} & \mc{1}{c}{\scriptsize{Mid-30s}} & \mc{1}{c}{\scriptsize{-0.501}} & \mc{1}{c}{\scriptsize{-0.243}} & \mc{1}{c}{\scriptsize{-0.105}} & \mc{1}{c}{\scriptsize{-0.077}} & \mc{1}{c}{\scriptsize{-0.445}} & \mc{1}{c}{\scriptsize{-1.128}} \\  

  &   &  & \mc{1}{c}{\scriptsize{(0.208)}} & \mc{1}{c}{\scriptsize{(0.337)}} & \mc{1}{c}{\scriptsize{(0.386)}} & \mc{1}{c}{\scriptsize{(0.416)}} & \mc{1}{c}{\scriptsize{(0.277)}} & \mc{1}{c}{\scriptsize{\textbf{(0.040)}}} \\  

    \mc{1}{l}{\scriptsize{Health}} &   \mc{1}{l}{\scriptsize{Self-reported drug user}} & \mc{1}{c}{\scriptsize{Mid-30s}} & \mc{1}{c}{\scriptsize{-0.333}} & \mc{1}{c}{\scriptsize{-0.398}} & \mc{1}{c}{\scriptsize{-0.392}} & \mc{1}{c}{\scriptsize{-0.424}} & \mc{1}{c}{\scriptsize{-0.471}} & \mc{1}{c}{\scriptsize{-0.189}} \\  

  &   &  & \mc{1}{c}{\scriptsize{\textbf{(0.020)}}} & \mc{1}{c}{\scriptsize{\textbf{(0.010)}}} & \mc{1}{c}{\scriptsize{\textbf{(0.030)}}} & \mc{1}{c}{\scriptsize{\textbf{(0.030)}}} & \mc{1}{c}{\scriptsize{\textbf{(0.079)}}} & \mc{1}{c}{\scriptsize{(0.228)}} \\  

  &  \mc{1}{l}{\scriptsize{Systolic Blood Pressure (mm Hg)}} & \mc{1}{c}{\scriptsize{Mid-30s}} & \mc{1}{c}{\scriptsize{-9.791}} & \mc{1}{c}{\scriptsize{-19.475}} & \mc{1}{c}{\scriptsize{-20.403}} & \mc{1}{c}{\scriptsize{-23.619}} & \mc{1}{c}{\scriptsize{-9.749}} & \mc{1}{c}{\scriptsize{-10.654}} \\  

   &  &  & \mc{1}{c}{\scriptsize{(0.109)}} & \mc{1}{c}{\scriptsize{\textbf{(0.010)}}} & \mc{1}{c}{\scriptsize{\textbf{(0.020)}}} & \mc{1}{c}{\scriptsize{\textbf{(0.059)}}} & \mc{1}{c}{\scriptsize{(0.139)}} & \mc{1}{c}{\scriptsize{\textbf{(0.030)}}} \\  

&    \mc{1}{l}{\scriptsize{Diastolic Blood Pressure (mm Hg)}} & \mc{1}{c}{\scriptsize{Mid-30s}} & \mc{1}{c}{\scriptsize{-10.854}} & \mc{1}{c}{\scriptsize{-19.401}} & \mc{1}{c}{\scriptsize{-21.691}} & \mc{1}{c}{\scriptsize{-22.863}} & \mc{1}{c}{\scriptsize{-7.755}} & \mc{1}{c}{\scriptsize{-3.081}} \\  

 &    &  & \mc{1}{c}{\scriptsize{\textbf{(0.040)}}} & \mc{1}{c}{\scriptsize{\textbf{(0.000)}}} & \mc{1}{c}{\scriptsize{\textbf{(0.000)}}} & \mc{1}{c}{\scriptsize{\textbf{(0.000)}}} & \mc{1}{c}{\scriptsize{\textbf{(0.089)}}} & \mc{1}{c}{\scriptsize{(0.178)}} \\  

  &  \mc{1}{l}{\scriptsize{Hypertension}} & \mc{1}{c}{\scriptsize{Mid-30s}} & \mc{1}{c}{\scriptsize{-0.291}} & \mc{1}{c}{\scriptsize{-0.384}} & \mc{1}{c}{\scriptsize{-0.480}} & \mc{1}{c}{\scriptsize{-0.503}} & \mc{1}{c}{\scriptsize{-0.085}} & \mc{1}{c}{\scriptsize{-0.087}} \\  

   &  &  & \mc{1}{c}{\scriptsize{\textbf{(0.030)}}} & \mc{1}{c}{\scriptsize{\textbf{(0.020)}}} & \mc{1}{c}{\scriptsize{\textbf{(0.000)}}} & \mc{1}{c}{\scriptsize{\textbf{(0.000)}}} & \mc{1}{c}{\scriptsize{(0.327)}} & \mc{1}{c}{\scriptsize{(0.337)}} \\  

  
\bottomrule
    \end{tabular} 
\end{scriptsize}
\begin{tablenotes}
\tiny
Note: This table shows the treatment effects for categories outcomes that are important for our benefit/cost analysis. Systolic and diastolic blood pressure are measured in terms of mm Hg. Each column present estimates for the following parameters: (1) $\mathbb{E} \left [ \bm{Y}^1 -  \bm{Y}^0 | \bm{B} \in \mathcal{B}_{0} \right]$ (no controls); (2) $\mathbb{E} \left [ \bm{Y}^1 -  \bm{Y}^0 | \bm{B} \in \mathcal{B}_{0} \right] $(controls); (3) $\mathbb{E} \left [ \bm{Y}^1 | R = 1 \right] -  \mathbb{E} \left [ \bm{Y}^0 | R = 0,V = 0  \right]$ (no controls); (4) $\mathbb{E} \left [ \bm{Y}^1 -  \bm{Y}_H^0 | \bm{B} \in \mathcal{B}_{0} \right]$ (controls);  (5) $\mathbb{E} \left [ \bm{Y}^1 | R = 1 \right] -  \mathbb{E} \left [ \bm{Y}^0 | R = 0,V = 1 \right]$ (no controls); (6) $\mathbb{E} \left [ \bm{Y}^1 -  \bm{Y}_C^0 | \bm{B} \in \mathcal{B}_{0} \right]$ (controls). We account for the following background variables ($\bm{B}$): Apgar scores at minutes 1 and 5 and the high-risk index. We define the high-risk index in Appendix~\ref{appendix:background} and explain how we choose the control variables in Appendix~\ref{appendix:bvariables}. Columns (2), (4), and (6) correct for item non-response and attrition using inverse probability weighting as we explain in Appendix~\ref{app:method_partialobs}. Inference is based on non-parametric, one-sided $p$-values from the empirical bootstrap distribution. We highlight point estimates significant at the $10\%$ level. See Appendix~\ref{appendix:vsensitivity} for two-sided $p$-values.
\end{tablenotes}
\end{threeparttable}
\end{table}

\begin{table}[!htbp]
\centering
\begin{threeparttable}
\caption{Treatment Effects on Selected Outcomes, Control Substitution if Attended Treatment Alternatives $4/5$ of Time between Ages 0 to 5}\label{table:tescombinedv1}
\begin{scriptsize}
  \begin{tabular}{ccccccccccc}
  \toprule
   Category & Variable & Age & (1) & (2) & (3) & (4) & (5) & (6)\\

    \midrule
     \multicolumn{9}{c}{\textbf{\emph{Females}}} \\ \\
    %cat 3
    \mc{1}{l}{\scriptsize{Parental Income}} &     \mc{1}{l}{\scriptsize{Parental Income}} & \mc{1}{c}{\scriptsize{3.5}} & \mc{1}{c}{\scriptsize{2,756}} & \mc{1}{c}{\scriptsize{3,277}} & \mc{1}{c}{\scriptsize{8,849}} & \mc{1}{c}{\scriptsize{9,658}} & \mc{1}{c}{\scriptsize{-1,086}} & \mc{1}{c}{\scriptsize{1,440}} \\  

  &   &  & \mc{1}{c}{\scriptsize{(0.188)}} & \mc{1}{c}{\scriptsize{(0.188)}} & \mc{1}{c}{\scriptsize{\textbf{(0.089)}}} & \mc{1}{c}{\scriptsize{\textbf{(0.000)}}} & \mc{1}{c}{\scriptsize{(0.545)}} & \mc{1}{c}{\scriptsize{(0.337)}} \\  

  &   & \mc{1}{c}{\scriptsize{12}} & \mc{1}{c}{\scriptsize{13,633}} & \mc{1}{c}{\scriptsize{19,386}} & \mc{1}{c}{\scriptsize{32,972}} & \mc{1}{c}{\scriptsize{28,194}} & \mc{1}{c}{\scriptsize{10,992}} & \mc{1}{c}{\scriptsize{17,690}} \\  

 &    &  & \mc{1}{c}{\scriptsize{\textbf{(0.099)}}} & \mc{1}{c}{\scriptsize{\textbf{(0.020)}}} & \mc{1}{c}{\scriptsize{\textbf{(0.059)}}} & \mc{1}{c}{\scriptsize{\textbf{(0.010)}}} & \mc{1}{c}{\scriptsize{(0.178)}} & \mc{1}{c}{\scriptsize{\textbf{(0.030)}}} \\  

  &   & \mc{1}{c}{\scriptsize{15}} & \mc{1}{c}{\scriptsize{8,565}} & \mc{1}{c}{\scriptsize{9,322}} & \mc{1}{c}{\scriptsize{3,316}} & \mc{1}{c}{\scriptsize{6,383}} & \mc{1}{c}{\scriptsize{10,675}} & \mc{1}{c}{\scriptsize{12,104}} \\  

  &   &  & \mc{1}{c}{\scriptsize{\textbf{(0.059)}}} & \mc{1}{c}{\scriptsize{(0.119)}} & \mc{1}{c}{\scriptsize{(0.396)}} & \mc{1}{c}{\scriptsize{(0.277)}} & \mc{1}{c}{\scriptsize{\textbf{(0.099)}}} & \mc{1}{c}{\scriptsize{\textbf{(0.059)}}} \\  

   &  & \mc{1}{c}{\scriptsize{21}} & \mc{1}{c}{\scriptsize{5,708}} & \mc{1}{c}{\scriptsize{6,944}} & \mc{1}{c}{\scriptsize{26,722}} & \mc{1}{c}{\scriptsize{13,060}} & \mc{1}{c}{\scriptsize{3,844}} & \mc{1}{c}{\scriptsize{4,186}} \\  

  &   &  & \mc{1}{c}{\scriptsize{(0.109)}} & \mc{1}{c}{\scriptsize{(0.149)}} & \mc{1}{c}{\scriptsize{\textbf{(0.050)}}} & \mc{1}{c}{\scriptsize{\textbf{(0.079)}}} & \mc{1}{c}{\scriptsize{(0.257)}} & \mc{1}{c}{\scriptsize{(0.307)}} \\  

      \mc{1}{l}{\scriptsize{Education}} &   \mc{1}{l}{\scriptsize{Graduated High School}} & \mc{1}{c}{\scriptsize{30}} & \mc{1}{c}{\scriptsize{0.253}} & \mc{1}{c}{\scriptsize{0.110}} & \mc{1}{c}{\scriptsize{0.308}} & \mc{1}{c}{\scriptsize{0.335}} & \mc{1}{c}{\scriptsize{0.027}} & \mc{1}{c}{\scriptsize{0.101}} \\  

  &   &  & \mc{1}{c}{\scriptsize{\textbf{(0.020)}}} & \mc{1}{c}{\scriptsize{(0.208)}} & \mc{1}{c}{\scriptsize{(0.119)}} & \mc{1}{c}{\scriptsize{\textbf{(0.010)}}} & \mc{1}{c}{\scriptsize{(0.396)}} & \mc{1}{c}{\scriptsize{(0.238)}} \\  

 &   \mc{1}{l}{\scriptsize{Graduated 4-year College}} & \mc{1}{c}{\scriptsize{30}} & \mc{1}{c}{\scriptsize{0.134}} & \mc{1}{c}{\scriptsize{0.119}} & \mc{1}{c}{\scriptsize{0.166}} & \mc{1}{c}{\scriptsize{0.219}} & \mc{1}{c}{\scriptsize{0.065}} & \mc{1}{c}{\scriptsize{0.060}} \\  

  &   &  & \mc{1}{c}{\scriptsize{(0.109)}} & \mc{1}{c}{\scriptsize{(0.188)}} & \mc{1}{c}{\scriptsize{\textbf{(0.079)}}} & \mc{1}{c}{\scriptsize{\textbf{(0.020)}}} & \mc{1}{c}{\scriptsize{(0.406)}} & \mc{1}{c}{\scriptsize{(0.366)}} \\  

&    \mc{1}{l}{\scriptsize{Years of Education}} & \mc{1}{c}{\scriptsize{30}} & \mc{1}{c}{\scriptsize{2.143}} & \mc{1}{c}{\scriptsize{1.715}} & \mc{1}{c}{\scriptsize{2.733}} & \mc{1}{c}{\scriptsize{3.103}} & \mc{1}{c}{\scriptsize{1.362}} & \mc{1}{c}{\scriptsize{1.322}} \\  

  &   &  & \mc{1}{c}{\scriptsize{\textbf{(0.000)}}} & \mc{1}{c}{\scriptsize{\textbf{(0.020)}}} & \mc{1}{c}{\scriptsize{\textbf{(0.020)}}} & \mc{1}{c}{\scriptsize{\textbf{(0.000)}}} & \mc{1}{c}{\scriptsize{\textbf{(0.059)}}} & \mc{1}{c}{\scriptsize{\textbf{(0.040)}}} \\  

     \mc{1}{l}{\scriptsize{Labor Income}} &    \mc{1}{l}{\scriptsize{Employed}} & \mc{1}{c}{\scriptsize{30}} & \mc{1}{c}{\scriptsize{0.131}} & \mc{1}{c}{\scriptsize{0.079}} & \mc{1}{c}{\scriptsize{0.221}} & \mc{1}{c}{\scriptsize{0.224}} & \mc{1}{c}{\scriptsize{-0.004}} & \mc{1}{c}{\scriptsize{0.078}} \\  

  &   &  & \mc{1}{c}{\scriptsize{\textbf{(0.079)}}} & \mc{1}{c}{\scriptsize{(0.248)}} & \mc{1}{c}{\scriptsize{(0.168)}} & \mc{1}{c}{\scriptsize{\textbf{(0.099)}}} & \mc{1}{c}{\scriptsize{(0.525)}} & \mc{1}{c}{\scriptsize{(0.238)}} \\  

  &  \mc{1}{l}{\scriptsize{Labor Income}} & \mc{1}{c}{\scriptsize{30}} & \mc{1}{c}{\scriptsize{2,548}} & \mc{1}{c}{\scriptsize{2,412}} & \mc{1}{c}{\scriptsize{9,737}} & \mc{1}{c}{\scriptsize{10,827}} & \mc{1}{c}{\scriptsize{-1,336}} & \mc{1}{c}{\scriptsize{-1,311}} \\  

  &   &  & \mc{1}{c}{\scriptsize{(0.347)}} & \mc{1}{c}{\scriptsize{(0.396)}} & \mc{1}{c}{\scriptsize{(0.218)}} & \mc{1}{c}{\scriptsize{\textbf{(0.059)}}} & \mc{1}{c}{\scriptsize{(0.604)}} & \mc{1}{c}{\scriptsize{(0.564)}} \\  

     \mc{1}{l}{\scriptsize{Crime}} &    \mc{1}{l}{\scriptsize{Total Felony Arrests}} & \mc{1}{c}{\scriptsize{Mid-30s}} & \mc{1}{c}{\scriptsize{-0.328}} & \mc{1}{c}{\scriptsize{-0.394}} & \mc{1}{c}{\scriptsize{-0.802}} & \mc{1}{c}{\scriptsize{-0.649}} & \mc{1}{c}{\scriptsize{-0.109}} & \mc{1}{c}{\scriptsize{-0.019}} \\  

 &    &  & \mc{1}{c}{\scriptsize{\textbf{(0.079)}}} & \mc{1}{c}{\scriptsize{\textbf{(0.059)}}} & \mc{1}{c}{\scriptsize{\textbf{(0.079)}}} & \mc{1}{c}{\scriptsize{(0.139)}} & \mc{1}{c}{\scriptsize{(0.208)}} & \mc{1}{c}{\scriptsize{(0.446)}} \\  

  &  \mc{1}{l}{\scriptsize{Total Misdemeanor Arrests}} & \mc{1}{c}{\scriptsize{Mid-30s}} & \mc{1}{c}{\scriptsize{-0.973}} & \mc{1}{c}{\scriptsize{-1.212}} & \mc{1}{c}{\scriptsize{-1.562}} & \mc{1}{c}{\scriptsize{-1.629}} & \mc{1}{c}{\scriptsize{-0.692}} & \mc{1}{c}{\scriptsize{-0.314}} \\  

  &   &  & \mc{1}{c}{\scriptsize{\textbf{(0.059)}}} & \mc{1}{c}{\scriptsize{\textbf{(0.079)}}} & \mc{1}{c}{\scriptsize{(0.129)}} & \mc{1}{c}{\scriptsize{(0.158)}} & \mc{1}{c}{\scriptsize{(0.139)}} & \mc{1}{c}{\scriptsize{(0.248)}} \\  

     \mc{1}{l}{\scriptsize{Health}} &    \mc{1}{l}{\scriptsize{Self-reported drug user}} & \mc{1}{c}{\scriptsize{Mid-30s}} & \mc{1}{c}{\scriptsize{-0.033}} & \mc{1}{c}{\scriptsize{-0.039}} & \mc{1}{c}{\scriptsize{-0.111}} & \mc{1}{c}{\scriptsize{-0.088}} & \mc{1}{c}{\scriptsize{0.040}} & \mc{1}{c}{\scriptsize{0.052}} \\  

  &   &  & \mc{1}{c}{\scriptsize{(0.337)}} & \mc{1}{c}{\scriptsize{(0.366)}} & \mc{1}{c}{\scriptsize{(0.228)}} & \mc{1}{c}{\scriptsize{(0.267)}} & \mc{1}{c}{\scriptsize{(0.545)}} & \mc{1}{c}{\scriptsize{(0.693)}} \\  

 &   \mc{1}{l}{\scriptsize{Systolic Blood Pressure (mm Hg)}} & \mc{1}{c}{\scriptsize{Mid-30s}} & \mc{1}{c}{\scriptsize{-2.899}} & \mc{1}{c}{\scriptsize{-3.034}} & \mc{1}{c}{\scriptsize{-1.191}} & \mc{1}{c}{\scriptsize{-0.869}} & \mc{1}{c}{\scriptsize{-3.818}} & \mc{1}{c}{\scriptsize{-7.447}} \\  

  &   &  & \mc{1}{c}{\scriptsize{(0.307)}} & \mc{1}{c}{\scriptsize{(0.297)}} & \mc{1}{c}{\scriptsize{(0.495)}} & \mc{1}{c}{\scriptsize{(0.465)}} & \mc{1}{c}{\scriptsize{(0.277)}} & \mc{1}{c}{\scriptsize{(0.168)}} \\  

 &   \mc{1}{l}{\scriptsize{Diastolic Blood Pressure (mm Hg)}} & \mc{1}{c}{\scriptsize{Mid-30s}} & \mc{1}{c}{\scriptsize{-0.002}} & \mc{1}{c}{\scriptsize{2.341}} & \mc{1}{c}{\scriptsize{5.457}} & \mc{1}{c}{\scriptsize{4.000}} & \mc{1}{c}{\scriptsize{0.524}} & \mc{1}{c}{\scriptsize{-2.820}} \\  

  &   &  & \mc{1}{c}{\scriptsize{(0.515)}} & \mc{1}{c}{\scriptsize{(0.653)}} & \mc{1}{c}{\scriptsize{(0.752)}} & \mc{1}{c}{\scriptsize{(0.762)}} & \mc{1}{c}{\scriptsize{(0.495)}} & \mc{1}{c}{\scriptsize{(0.317)}} \\  

  &  \mc{1}{l}{\scriptsize{Hypertension}} & \mc{1}{c}{\scriptsize{Mid-30s}} & \mc{1}{c}{\scriptsize{0.172}} & \mc{1}{c}{\scriptsize{0.192}} & \mc{1}{c}{\scriptsize{0.035}} & \mc{1}{c}{\scriptsize{0.092}} & \mc{1}{c}{\scriptsize{0.243}} & \mc{1}{c}{\scriptsize{0.113}} \\  

  &   &  & \mc{1}{c}{\scriptsize{(0.901)}} & \mc{1}{c}{\scriptsize{(0.822)}} & \mc{1}{c}{\scriptsize{(0.545)}} & \mc{1}{c}{\scriptsize{(0.673)}} & \mc{1}{c}{\scriptsize{(0.822)}} & \mc{1}{c}{\scriptsize{(0.772)}} \\  

     
     
     
     
     
     
     
     
     
     
     
     
\midrule
    \multicolumn{9}{c}{\textbf{\emph{Males}}} \\ \\
    % cat 3
     \mc{1}{l}{\scriptsize{Parental Income}} &     \mc{1}{l}{\scriptsize{Parental Income}} & \mc{1}{c}{\scriptsize{3.5}} & \mc{1}{c}{\scriptsize{1,036}} & \mc{1}{c}{\scriptsize{-1,185}} & \mc{1}{c}{\scriptsize{-1,154}} & \mc{1}{c}{\scriptsize{3,199}} & \mc{1}{c}{\scriptsize{-1,503}} & \mc{1}{c}{\scriptsize{352}} \\  

  &   &  & \mc{1}{c}{\scriptsize{(0.366)}} & \mc{1}{c}{\scriptsize{(0.644)}} & \mc{1}{c}{\scriptsize{(0.545)}} & \mc{1}{c}{\scriptsize{(0.238)}} & \mc{1}{c}{\scriptsize{(0.693)}} & \mc{1}{c}{\scriptsize{(0.475)}} \\  

  &   & \mc{1}{c}{\scriptsize{12}} & \mc{1}{c}{\scriptsize{7,085}} & \mc{1}{c}{\scriptsize{10,384}} & \mc{1}{c}{\scriptsize{23,037}} & \mc{1}{c}{\scriptsize{15,288}} & \mc{1}{c}{\scriptsize{4,785}} & \mc{1}{c}{\scriptsize{3,905}} \\  

  &   &  & \mc{1}{c}{\scriptsize{\textbf{(0.099)}}} & \mc{1}{c}{\scriptsize{\textbf{(0.050)}}} & \mc{1}{c}{\scriptsize{\textbf{(0.010)}}} & \mc{1}{c}{\scriptsize{\textbf{(0.020)}}} & \mc{1}{c}{\scriptsize{(0.178)}} & \mc{1}{c}{\scriptsize{(0.228)}} \\  

  &   & \mc{1}{c}{\scriptsize{15}} & \mc{1}{c}{\scriptsize{8,488}} & \mc{1}{c}{\scriptsize{7,185}} & \mc{1}{c}{\scriptsize{17,045}} & \mc{1}{c}{\scriptsize{10,825}} & \mc{1}{c}{\scriptsize{939}} & \mc{1}{c}{\scriptsize{1,799}} \\  

  &   &  & \mc{1}{c}{\scriptsize{\textbf{(0.089)}}} & \mc{1}{c}{\scriptsize{(0.178)}} & \mc{1}{c}{\scriptsize{\textbf{(0.050)}}} & \mc{1}{c}{\scriptsize{\textbf{(0.089)}}} & \mc{1}{c}{\scriptsize{(0.416)}} & \mc{1}{c}{\scriptsize{(0.416)}} \\  

  &   & \mc{1}{c}{\scriptsize{21}} & \mc{1}{c}{\scriptsize{12,732}} & \mc{1}{c}{\scriptsize{12,650}} & \mc{1}{c}{\scriptsize{-2,880}} & \mc{1}{c}{\scriptsize{-1,000}} & \mc{1}{c}{\scriptsize{17,027}} & \mc{1}{c}{\scriptsize{10,323}} \\  

  &   &  & \mc{1}{c}{\scriptsize{\textbf{(0.010)}}} & \mc{1}{c}{\scriptsize{\textbf{(0.059)}}} & \mc{1}{c}{\scriptsize{(0.495)}} & \mc{1}{c}{\scriptsize{(0.495)}} & \mc{1}{c}{\scriptsize{\textbf{(0.030)}}} & \mc{1}{c}{\scriptsize{\textbf{(0.059)}}} \\  

      \mc{1}{l}{\scriptsize{Education}} &   \mc{1}{l}{\scriptsize{Graduated High School}} & \mc{1}{c}{\scriptsize{30}} & \mc{1}{c}{\scriptsize{0.073}} & \mc{1}{c}{\scriptsize{0.130}} & \mc{1}{c}{\scriptsize{0.158}} & \mc{1}{c}{\scriptsize{0.099}} & \mc{1}{c}{\scriptsize{0.137}} & \mc{1}{c}{\scriptsize{0.054}} \\  

  &   &  & \mc{1}{c}{\scriptsize{(0.287)}} & \mc{1}{c}{\scriptsize{(0.178)}} & \mc{1}{c}{\scriptsize{(0.257)}} & \mc{1}{c}{\scriptsize{(0.337)}} & \mc{1}{c}{\scriptsize{(0.208)}} & \mc{1}{c}{\scriptsize{(0.386)}} \\  

 &   \mc{1}{l}{\scriptsize{Graduated 4-year College}} & \mc{1}{c}{\scriptsize{30}} & \mc{1}{c}{\scriptsize{0.170}} & \mc{1}{c}{\scriptsize{0.178}} & \mc{1}{c}{\scriptsize{0.299}} & \mc{1}{c}{\scriptsize{0.136}} & \mc{1}{c}{\scriptsize{0.172}} & \mc{1}{c}{\scriptsize{0.128}} \\  

  &   &  & \mc{1}{c}{\scriptsize{\textbf{(0.069)}}} & \mc{1}{c}{\scriptsize{\textbf{(0.059)}}} & \mc{1}{c}{\scriptsize{\textbf{(0.050)}}} & \mc{1}{c}{\scriptsize{(0.198)}} & \mc{1}{c}{\scriptsize{(0.139)}} & \mc{1}{c}{\scriptsize{(0.158)}} \\  

  &  \mc{1}{l}{\scriptsize{Years of Education}} & \mc{1}{c}{\scriptsize{30}} & \mc{1}{c}{\scriptsize{0.525}} & \mc{1}{c}{\scriptsize{0.785}} & \mc{1}{c}{\scriptsize{1.386}} & \mc{1}{c}{\scriptsize{0.906}} & \mc{1}{c}{\scriptsize{0.690}} & \mc{1}{c}{\scriptsize{0.243}} \\  

 &    &  & \mc{1}{c}{\scriptsize{(0.149)}} & \mc{1}{c}{\scriptsize{\textbf{(0.089)}}} & \mc{1}{c}{\scriptsize{\textbf{(0.040)}}} & \mc{1}{c}{\scriptsize{\textbf{(0.040)}}} & \mc{1}{c}{\scriptsize{(0.168)}} & \mc{1}{c}{\scriptsize{(0.347)}} \\  

     \mc{1}{l}{\scriptsize{Labor Income}} &    \mc{1}{l}{\scriptsize{Employed}} & \mc{1}{c}{\scriptsize{30}} & \mc{1}{c}{\scriptsize{0.119}} & \mc{1}{c}{\scriptsize{0.182}} & \mc{1}{c}{\scriptsize{-0.006}} & \mc{1}{c}{\scriptsize{0.008}} & \mc{1}{c}{\scriptsize{0.277}} & \mc{1}{c}{\scriptsize{0.298}} \\  

 &    &  & \mc{1}{c}{\scriptsize{(0.129)}} & \mc{1}{c}{\scriptsize{\textbf{(0.040)}}} & \mc{1}{c}{\scriptsize{(0.495)}} & \mc{1}{c}{\scriptsize{(0.396)}} & \mc{1}{c}{\scriptsize{\textbf{(0.010)}}} & \mc{1}{c}{\scriptsize{\textbf{(0.010)}}} \\  

  &  \mc{1}{l}{\scriptsize{Labor Income}} & \mc{1}{c}{\scriptsize{30}} & \mc{1}{c}{\scriptsize{19,810}} & \mc{1}{c}{\scriptsize{27,373}} & \mc{1}{c}{\scriptsize{36,136}} & \mc{1}{c}{\scriptsize{24,479}} & \mc{1}{c}{\scriptsize{29,622}} & \mc{1}{c}{\scriptsize{20,514}} \\  

 &    &  & \mc{1}{c}{\scriptsize{(0.119)}} & \mc{1}{c}{\scriptsize{\textbf{(0.069)}}} & \mc{1}{c}{\scriptsize{(0.149)}} & \mc{1}{c}{\scriptsize{\textbf{(0.099)}}} & \mc{1}{c}{\scriptsize{(0.129)}} & \mc{1}{c}{\scriptsize{(0.158)}} \\  

     \mc{1}{l}{\scriptsize{Crime}} &    \mc{1}{l}{\scriptsize{Total Felony Arrests}} & \mc{1}{c}{\scriptsize{Mid-30s}} & \mc{1}{c}{\scriptsize{0.196}} & \mc{1}{c}{\scriptsize{0.392}} & \mc{1}{c}{\scriptsize{1.656}} & \mc{1}{c}{\scriptsize{1.387}} & \mc{1}{c}{\scriptsize{0.004}} & \mc{1}{c}{\scriptsize{0.110}} \\  

  &   &  & \mc{1}{c}{\scriptsize{(0.683)}} & \mc{1}{c}{\scriptsize{(0.653)}} & \mc{1}{c}{\scriptsize{(0.861)}} & \mc{1}{c}{\scriptsize{(1.000)}} & \mc{1}{c}{\scriptsize{(0.446)}} & \mc{1}{c}{\scriptsize{(0.554)}} \\  

 &   \mc{1}{l}{\scriptsize{Total Misdemeanor Arrests}} & \mc{1}{c}{\scriptsize{Mid-30s}} & \mc{1}{c}{\scriptsize{-0.501}} & \mc{1}{c}{\scriptsize{-0.243}} & \mc{1}{c}{\scriptsize{0.053}} & \mc{1}{c}{\scriptsize{0.058}} & \mc{1}{c}{\scriptsize{-0.371}} & \mc{1}{c}{\scriptsize{-0.574}} \\  

  &   &  & \mc{1}{c}{\scriptsize{(0.178)}} & \mc{1}{c}{\scriptsize{(0.277)}} & \mc{1}{c}{\scriptsize{(0.485)}} & \mc{1}{c}{\scriptsize{(0.485)}} & \mc{1}{c}{\scriptsize{(0.297)}} & \mc{1}{c}{\scriptsize{(0.139)}} \\  

     \mc{1}{l}{\scriptsize{Health}} &    \mc{1}{l}{\scriptsize{Self-reported drug user}} & \mc{1}{c}{\scriptsize{Mid-30s}} & \mc{1}{c}{\scriptsize{-0.333}} & \mc{1}{c}{\scriptsize{-0.398}} & \mc{1}{c}{\scriptsize{-0.693}} & \mc{1}{c}{\scriptsize{-0.557}} & \mc{1}{c}{\scriptsize{-0.309}} & \mc{1}{c}{\scriptsize{-0.330}} \\  

 &    &  & \mc{1}{c}{\scriptsize{\textbf{(0.010)}}} & \mc{1}{c}{\scriptsize{\textbf{(0.000)}}} & \mc{1}{c}{\scriptsize{\textbf{(0.000)}}} & \mc{1}{c}{\scriptsize{\textbf{(0.000)}}} & \mc{1}{c}{\scriptsize{\textbf{(0.050)}}} & \mc{1}{c}{\scriptsize{\textbf{(0.050)}}} \\  

  &  \mc{1}{l}{\scriptsize{Systolic Blood Pressure (mm Hg)}} & \mc{1}{c}{\scriptsize{Mid-30s}} & \mc{1}{c}{\scriptsize{-9.791}} & \mc{1}{c}{\scriptsize{-19.475}} & \mc{1}{c}{\scriptsize{17.366}} & \mc{1}{c}{\scriptsize{14.259}} & \mc{1}{c}{\scriptsize{-29.384}} & \mc{1}{c}{\scriptsize{-30.633}} \\  

  &   &  & \mc{1}{c}{\scriptsize{\textbf{(0.079)}}} & \mc{1}{c}{\scriptsize{\textbf{(0.020)}}} & \mc{1}{c}{\scriptsize{(0.723)}} & \mc{1}{c}{\scriptsize{(0.931)}} & \mc{1}{c}{\scriptsize{\textbf{(0.000)}}} & \mc{1}{c}{\scriptsize{\textbf{(0.000)}}} \\  

  &  \mc{1}{l}{\scriptsize{Diastolic Blood Pressure (mm Hg)}} & \mc{1}{c}{\scriptsize{Mid-30s}} & \mc{1}{c}{\scriptsize{-10.854}} & \mc{1}{c}{\scriptsize{-19.401}} & \mc{1}{c}{\scriptsize{-10.746}} & \mc{1}{c}{\scriptsize{-8.117}} & \mc{1}{c}{\scriptsize{-22.079}} & \mc{1}{c}{\scriptsize{-21.893}} \\  

 &    &  & \mc{1}{c}{\scriptsize{\textbf{(0.020)}}} & \mc{1}{c}{\scriptsize{\textbf{(0.000)}}} & \mc{1}{c}{\scriptsize{\textbf{(0.079)}}} & \mc{1}{c}{\scriptsize{\textbf{(0.079)}}} & \mc{1}{c}{\scriptsize{\textbf{(0.000)}}} & \mc{1}{c}{\scriptsize{\textbf{(0.000)}}} \\  

  &  \mc{1}{l}{\scriptsize{Hypertension}} & \mc{1}{c}{\scriptsize{Mid-30s}} & \mc{1}{c}{\scriptsize{-0.291}} & \mc{1}{c}{\scriptsize{-0.384}} & \mc{1}{c}{\scriptsize{0.003}} & \mc{1}{c}{\scriptsize{-0.063}} & \mc{1}{c}{\scriptsize{-0.488}} & \mc{1}{c}{\scriptsize{-0.518}} \\  

  &   &  & \mc{1}{c}{\scriptsize{\textbf{(0.040)}}} & \mc{1}{c}{\scriptsize{\textbf{(0.020)}}} & \mc{1}{c}{\scriptsize{(0.317)}} & \mc{1}{c}{\scriptsize{(0.356)}} & \mc{1}{c}{\scriptsize{\textbf{(0.010)}}} & \mc{1}{c}{\scriptsize{\textbf{(0.000)}}} \\  

\bottomrule
    \end{tabular} 
\end{scriptsize}
\begin{tablenotes}
\tiny
Note: This table shows the treatment effects for categories outcomes that are important for our benefit/cost analysis. Systolic and diastolic blood pressure are measured in terms of mm Hg. Each column present estimates for the following parameters: \textbf{(1)} $\mathbb{E} \big[ \bm{Y}^1 - \bm{Y}^0 | W = 1]$; {\textbf{(2)} $\mathbb{E} \big[ \bm{Y}^1 - \bm{Y}^0 | \bm{B}, W=1 \big]$}; {\textbf{(3)} $\mathbb{E} \big[ \bm{Y}^1 | \bm{B}, D=1 \big] - \mathbb{E} \big[ \bm{Y}^0 | \bm{B}, V=0, D=0 \big]$}; {\textbf{(4)} $\mathbb{E} \big[ \bm{Y}^1 - \bm{Y}^0 | \bm{B}, V=0, W = 1 \big] $}; {\textbf{(5)} $\mathbb{E} \big[ \bm{Y}^1 | \bm{B}, D=1 \big] - \mathbb{E} \big[ \bm{Y}^0 | \bm{B}, V=1, D = 0 \big]$}; {\textbf{(6)} $\mathbb{E} \big[ \bm{Y}^1 - \bm{Y}^0 | \bm{B}, V=1 , W = 1\big]$}. We account for the following background variables ($\bm{B}$): Apgar scores at minutes 1 and 5 and the high-risk index. We define the high-risk index in Appendix~\ref{appendix:background} and explain how we choose the control variables in Appendix~\ref{appendix:bvariables}. Inference is based on non-parametric, one-sided $p$-values from the empirical bootstrap distribution. We highlight point estimates significant at the $10\%$ level.
\end{tablenotes}
\end{threeparttable}
\end{table}
\doublespacing

\subsection{Two-Sided Statistical Tests}

\noindent In the main paper, we classify the outcomes of interest as ``beneficial'' (see Appendix~\ref{appendix:results} our classification) and perform one-sided tests. The next table presents two-sided inferences for the main table of treatment-effect estimates in the main paper, Table~\ref{table:tescombined}. The main treatment effects survive two-sided testing. A full replication of the results throughout the main text using two-sided statistical tests is available under request. As is evident from the standard errors, our combining functions and cost-benefit analysis results generally survive two-sided testing.

\newgeometry{top=.6in, bottom=.8in, left=.8in, right=.8in}
\begin{table}[!htbp]
\centering
\begin{threeparttable}
\caption{Treatment Effects on Selected Outcomes, Two-Sided Inference}\label{table:tescombinedtwo}
\begin{scriptsize}
  \begin{tabular}{ccccccccccc}
  \toprule
   Category & Variable & Age & (1) & (2) & (3) & (4) & (5) & (6)\\

    \midrule
     \multicolumn{9}{c}{\textbf{\emph{Females}}} \\ \\
    %cat 3
 
    \mc{1}{l}{\scriptsize{Parental Income}} &  \mc{1}{l}{\scriptsize{Parental Labor Income}} & \mc{1}{c}{\scriptsize{3.5}} & \mc{1}{c}{\scriptsize{2,756}} & \mc{1}{c}{\scriptsize{3,277}} & \mc{1}{c}{\scriptsize{10,509}} & \mc{1}{c}{\scriptsize{8,601}} &  & \mc{1}{c}{\scriptsize{3,762}} \\  

   &  &  & \mc{1}{c}{\scriptsize{(0.419)}} & \mc{1}{c}{\scriptsize{(0.368)}} & \mc{1}{c}{\scriptsize{(0.111)}} & \mc{1}{c}{\scriptsize{(0.120)}} &  & \mc{1}{c}{\scriptsize{(0.363)}} \\  

   &  & \mc{1}{c}{\scriptsize{12}} & \mc{1}{c}{\scriptsize{13,633}} & \mc{1}{c}{\scriptsize{19,386}} & \mc{1}{c}{\scriptsize{33,624}} & \mc{1}{c}{\scriptsize{26,474}} & \mc{1}{c}{\scriptsize{11,176}} & \mc{1}{c}{\scriptsize{18,629}} \\  

   &  &  & \mc{1}{c}{\scriptsize{(0.124)}} & \mc{1}{c}{\scriptsize{\textbf{(0.071)}}} & \mc{1}{c}{\scriptsize{(0.255)}} & \mc{1}{c}{\scriptsize{\textbf{(0.025)}}} & \mc{1}{c}{\scriptsize{(0.277)}} & \mc{1}{c}{\scriptsize{\textbf{(0.023)}}} \\  

   &  & \mc{1}{c}{\scriptsize{15}} & \mc{1}{c}{\scriptsize{8,565}} & \mc{1}{c}{\scriptsize{9,322}} & \mc{1}{c}{\scriptsize{5,533}} & \mc{1}{c}{\scriptsize{8,435}} & \mc{1}{c}{\scriptsize{8,817}} & \mc{1}{c}{\scriptsize{10,480}} \\  

   &  &  & \mc{1}{c}{\scriptsize{(0.116)}} & \mc{1}{c}{\scriptsize{(0.188)}} & \mc{1}{c}{\scriptsize{(0.690)}} & \mc{1}{c}{\scriptsize{(0.618)}} & \mc{1}{c}{\scriptsize{(0.215)}} & \mc{1}{c}{\scriptsize{\textbf{(0.099)}}} \\  

    & & \mc{1}{c}{\scriptsize{21}} & \mc{1}{c}{\scriptsize{5,708}} & \mc{1}{c}{\scriptsize{6,944}} & \mc{1}{c}{\scriptsize{41,245}} & \mc{1}{c}{\scriptsize{25,135}} & \mc{1}{c}{\scriptsize{4,608}} & \mc{1}{c}{\scriptsize{3,926}} \\  

    & &  & \mc{1}{c}{\scriptsize{(0.261)}} & \mc{1}{c}{\scriptsize{(0.444)}} & \mc{1}{c}{\scriptsize{\textbf{(0.052)}}} & \mc{1}{c}{\scriptsize{\textbf{(0.001)}}} & \mc{1}{c}{\scriptsize{(0.532)}} & \mc{1}{c}{\scriptsize{(0.540)}} \\  

    \mc{1}{l}{\scriptsize{Education}} &  \mc{1}{l}{\scriptsize{Graduated High School}} & \mc{1}{c}{\scriptsize{30}} & \mc{1}{c}{\scriptsize{0.253}} & \mc{1}{c}{\scriptsize{0.110}} & \mc{1}{c}{\scriptsize{0.561}} & \mc{1}{c}{\scriptsize{0.596}} & \mc{1}{c}{\scriptsize{-0.027}} & \mc{1}{c}{\scriptsize{0.066}} \\  

   &  &  & \mc{1}{c}{\scriptsize{\textbf{(0.025)}}} & \mc{1}{c}{\scriptsize{(0.419)}} & \mc{1}{c}{\scriptsize{(1.000)}} & \mc{1}{c}{\scriptsize{\textbf{(0.001)}}} & \mc{1}{c}{\scriptsize{(0.846)}} & \mc{1}{c}{\scriptsize{(0.611)}} \\  

   & \mc{1}{l}{\scriptsize{Graduated 4-year College}} & \mc{1}{c}{\scriptsize{30}} & \mc{1}{c}{\scriptsize{0.134}} &  & \mc{1}{c}{\scriptsize{0.112}} & \mc{1}{c}{\scriptsize{0.219}} & \mc{1}{c}{\scriptsize{0.095}} & \mc{1}{c}{\scriptsize{0.094}} \\  

   &  &  & \mc{1}{c}{\scriptsize{(0.155)}} &  & \mc{1}{c}{\scriptsize{(0.267)}} & \mc{1}{c}{\scriptsize{\textbf{(0.012)}}} & \mc{1}{c}{\scriptsize{(0.468)}} & \mc{1}{c}{\scriptsize{(0.421)}} \\  

  &  \mc{1}{l}{\scriptsize{Years of Education}} & \mc{1}{c}{\scriptsize{30}} & \mc{1}{c}{\scriptsize{2.143}} & \mc{1}{c}{\scriptsize{1.715}} & \mc{1}{c}{\scriptsize{3.370}} & \mc{1}{c}{\scriptsize{3.925}} & \mc{1}{c}{\scriptsize{1.238}} & \mc{1}{c}{\scriptsize{1.412}} \\  

  &   &  & \mc{1}{c}{\scriptsize{\textbf{(0.000)}}} & \mc{1}{c}{\scriptsize{\textbf{(0.021)}}} & \mc{1}{c}{\scriptsize{(1.000)}} & \mc{1}{c}{\scriptsize{\textbf{(0.000)}}} & \mc{1}{c}{\scriptsize{(0.122)}} & \mc{1}{c}{\scriptsize{\textbf{(0.036)}}} \\  

    \mc{1}{l}{\scriptsize{Labor Income}} &  \mc{1}{l}{\scriptsize{Employed}} & \mc{1}{c}{\scriptsize{30}} & \mc{1}{c}{\scriptsize{0.131}} & \mc{1}{c}{\scriptsize{0.079}} & \mc{1}{c}{\scriptsize{0.395}} & \mc{1}{c}{\scriptsize{0.340}} & \mc{1}{c}{\scriptsize{-0.004}} & \mc{1}{c}{\scriptsize{0.070}} \\  

  &   &  & \mc{1}{c}{\scriptsize{(0.177)}} & \mc{1}{c}{\scriptsize{(0.462)}} & \mc{1}{c}{\scriptsize{(1.000)}} & \mc{1}{c}{\scriptsize{\textbf{(0.085)}}} & \mc{1}{c}{\scriptsize{(0.970)}} & \mc{1}{c}{\scriptsize{(0.507)}} \\  

  &  \mc{1}{l}{\scriptsize{Labor Income}} & \mc{1}{c}{\scriptsize{30}} & \mc{1}{c}{\scriptsize{2,548}} & \mc{1}{c}{\scriptsize{2,412}} & \mc{1}{c}{\scriptsize{10,256}} & \mc{1}{c}{\scriptsize{14,862}} & \mc{1}{c}{\scriptsize{-1,078}} & \mc{1}{c}{\scriptsize{-822}} \\  

  &   &  & \mc{1}{c}{\scriptsize{(0.694)}} & \mc{1}{c}{\scriptsize{(0.703)}} & \mc{1}{c}{\scriptsize{(0.313)}} & \mc{1}{c}{\scriptsize{\textbf{(0.042)}}} & \mc{1}{c}{\scriptsize{(0.865)}} & \mc{1}{c}{\scriptsize{(0.882)}} \\  

    \mc{1}{l}{\scriptsize{Crime}} &  \mc{1}{l}{\scriptsize{Total Felony Arrests}} & \mc{1}{c}{\scriptsize{Mid-30s}} & \mc{1}{c}{\scriptsize{-0.328}} & \mc{1}{c}{\scriptsize{-0.394}} & \mc{1}{c}{\scriptsize{-1.006}} & \mc{1}{c}{\scriptsize{-0.965}} & \mc{1}{c}{\scriptsize{-0.083}} & \mc{1}{c}{\scriptsize{0.005}} \\  

  &   &  & \mc{1}{c}{\scriptsize{(0.143)}} & \mc{1}{c}{\scriptsize{(0.116)}} & \mc{1}{c}{\scriptsize{(0.195)}} & \mc{1}{c}{\scriptsize{(0.169)}} & \mc{1}{c}{\scriptsize{(0.500)}} & \mc{1}{c}{\scriptsize{(0.956)}} \\  

  &  \mc{1}{l}{\scriptsize{Total Misdemeanor Arrests}} & \mc{1}{c}{\scriptsize{Mid-30s}} & \mc{1}{c}{\scriptsize{-0.973}} & \mc{1}{c}{\scriptsize{-1.212}} & \mc{1}{c}{\scriptsize{-2.303}} & \mc{1}{c}{\scriptsize{-2.448}} & \mc{1}{c}{\scriptsize{-0.466}} & \mc{1}{c}{\scriptsize{-0.201}} \\  

  &   &  & \mc{1}{c}{\scriptsize{\textbf{(0.082)}}} & \mc{1}{c}{\scriptsize{(0.163)}} & \mc{1}{c}{\scriptsize{(0.316)}} & \mc{1}{c}{\scriptsize{(0.274)}} & \mc{1}{c}{\scriptsize{(0.394)}} & \mc{1}{c}{\scriptsize{(0.630)}} \\  

    \mc{1}{l}{\scriptsize{Health}} &  \mc{1}{l}{\scriptsize{Self-reported drug user}} & \mc{1}{c}{\scriptsize{Mid-30s}} & \mc{1}{c}{\scriptsize{-0.033}} & \mc{1}{c}{\scriptsize{-0.039}} & \mc{1}{c}{\scriptsize{-0.221}} & \mc{1}{c}{\scriptsize{-0.101}} & \mc{1}{c}{\scriptsize{0.031}} & \mc{1}{c}{\scriptsize{0.033}} \\  

 &     &  & \mc{1}{c}{\scriptsize{(0.790)}} & \mc{1}{c}{\scriptsize{(0.769)}} & \mc{1}{c}{\scriptsize{(0.330)}} & \mc{1}{c}{\scriptsize{(0.643)}} & \mc{1}{c}{\scriptsize{(0.824)}} & \mc{1}{c}{\scriptsize{(0.789)}} \\  

  &  \mc{1}{l}{\scriptsize{Systolic Blood Pressure (mm Hg)}} & \mc{1}{c}{\scriptsize{Mid-30s}} & \mc{1}{c}{\scriptsize{-2.899}} & \mc{1}{c}{\scriptsize{-4.316}} & \mc{1}{c}{\scriptsize{-2.825}} & \mc{1}{c}{\scriptsize{-0.827}} & \mc{1}{c}{\scriptsize{-3.915}} & \mc{1}{c}{\scriptsize{-6.805}} \\  

  &   &  & \mc{1}{c}{\scriptsize{(0.617)}} & \mc{1}{c}{\scriptsize{(0.553)}} & \mc{1}{c}{\scriptsize{(0.787)}} & \mc{1}{c}{\scriptsize{(0.905)}} & \mc{1}{c}{\scriptsize{(0.651)}} & \mc{1}{c}{\scriptsize{(0.373)}} \\  

  &  \mc{1}{l}{\scriptsize{Diastolic Blood Pressure (mm Hg)}} & \mc{1}{c}{\scriptsize{Mid-30s}} & \mc{1}{c}{\scriptsize{-0.002}} & \mc{1}{c}{\scriptsize{1.323}} & \mc{1}{c}{\scriptsize{5.667}} & \mc{1}{c}{\scriptsize{4.120}} & \mc{1}{c}{\scriptsize{0.834}} & \mc{1}{c}{\scriptsize{-2.186}} \\  

  &   &  & \mc{1}{c}{\scriptsize{(1.000)}} & \mc{1}{c}{\scriptsize{(0.814)}} & \mc{1}{c}{\scriptsize{(0.580)}} & \mc{1}{c}{\scriptsize{(0.466)}} & \mc{1}{c}{\scriptsize{(0.865)}} & \mc{1}{c}{\scriptsize{(0.707)}} \\  

   & \mc{1}{l}{\scriptsize{Hypertension}} & \mc{1}{c}{\scriptsize{Mid-30s}} & \mc{1}{c}{\scriptsize{0.172}} & \mc{1}{c}{\scriptsize{0.151}} & \mc{1}{c}{\scriptsize{0.112}} & \mc{1}{c}{\scriptsize{0.162}} & \mc{1}{c}{\scriptsize{0.177}} & \mc{1}{c}{\scriptsize{0.107}} \\  

 &    &  & \mc{1}{c}{\scriptsize{(0.225)}} & \mc{1}{c}{\scriptsize{(0.383)}} & \mc{1}{c}{\scriptsize{(0.661)}} & \mc{1}{c}{\scriptsize{(0.518)}} & \mc{1}{c}{\scriptsize{(0.370)}} & \mc{1}{c}{\scriptsize{(0.519)}} \\  



     
     
     
     
     
     
     
     
     
     
     
     
     
     
     
\midrule
    \multicolumn{9}{c}{\textbf{\emph{Males}}} \\ \\
    % cat 3
          \mc{1}{l}{\scriptsize{Parental Income}} &  \mc{1}{l}{\scriptsize{Parental Labor Income}} & \mc{1}{c}{\scriptsize{3.5}} & \mc{1}{c}{\scriptsize{1,036}} & \mc{1}{c}{\scriptsize{-1,185}} & \mc{1}{c}{\scriptsize{-2,321}} & \mc{1}{c}{\scriptsize{1,452}} & \mc{1}{c}{\scriptsize{-1,171}} & \mc{1}{c}{\scriptsize{703}} \\  

   &  &  & \mc{1}{c}{\scriptsize{(0.754)}} & \mc{1}{c}{\scriptsize{(0.705)}} & \mc{1}{c}{\scriptsize{(0.723)}} & \mc{1}{c}{\scriptsize{(0.773)}} & \mc{1}{c}{\scriptsize{(0.738)}} & \mc{1}{c}{\scriptsize{(0.845)}} \\  

   &  & \mc{1}{c}{\scriptsize{12}} & \mc{1}{c}{\scriptsize{7,085}} & \mc{1}{c}{\scriptsize{10,384}} & \mc{1}{c}{\scriptsize{20,007}} & \mc{1}{c}{\scriptsize{12,682}} & \mc{1}{c}{\scriptsize{7,791}} & \mc{1}{c}{\scriptsize{5,411}} \\  

   &  &  & \mc{1}{c}{\scriptsize{(0.183)}} & \mc{1}{c}{\scriptsize{\textbf{(0.059)}}} & \mc{1}{c}{\scriptsize{\textbf{(0.084)}}} & \mc{1}{c}{\scriptsize{(0.171)}} & \mc{1}{c}{\scriptsize{(0.193)}} & \mc{1}{c}{\scriptsize{(0.277)}} \\  

   &  & \mc{1}{c}{\scriptsize{15}} & \mc{1}{c}{\scriptsize{8,488}} & \mc{1}{c}{\scriptsize{7,185}} & \mc{1}{c}{\scriptsize{10,024}} & \mc{1}{c}{\scriptsize{4,915}} & \mc{1}{c}{\scriptsize{5,020}} & \mc{1}{c}{\scriptsize{4,379}} \\  

   &  &  & \mc{1}{c}{\scriptsize{(0.179)}} & \mc{1}{c}{\scriptsize{(0.267)}} & \mc{1}{c}{\scriptsize{(0.298)}} & \mc{1}{c}{\scriptsize{(0.578)}} & \mc{1}{c}{\scriptsize{(0.490)}} & \mc{1}{c}{\scriptsize{(0.582)}} \\  

   &  & \mc{1}{c}{\scriptsize{21}} & \mc{1}{c}{\scriptsize{12,732}} & \mc{1}{c}{\scriptsize{12,650}} & \mc{1}{c}{\scriptsize{-2,880}} & \mc{1}{c}{\scriptsize{-1,000}} & \mc{1}{c}{\scriptsize{17,027}} & \mc{1}{c}{\scriptsize{10,323}} \\  

   &  &  & \mc{1}{c}{\scriptsize{\textbf{(0.026)}}} & \mc{1}{c}{\scriptsize{(0.114)}} & \mc{1}{c}{\scriptsize{(0.786)}} & \mc{1}{c}{\scriptsize{(0.847)}} & \mc{1}{c}{\scriptsize{\textbf{(0.036)}}} & \mc{1}{c}{\scriptsize{\textbf{(0.082)}}} \\  

   \mc{1}{l}{\scriptsize{Education}} &  \mc{1}{l}{\scriptsize{Graduated High School}} & \mc{1}{c}{\scriptsize{30}} & \mc{1}{c}{\scriptsize{0.073}} & \mc{1}{c}{\scriptsize{0.130}} & \mc{1}{c}{\scriptsize{0.186}} & \mc{1}{c}{\scriptsize{0.084}} & \mc{1}{c}{\scriptsize{0.136}} & \mc{1}{c}{\scriptsize{0.063}} \\  

   &  &  & \mc{1}{c}{\scriptsize{(0.530)}} & \mc{1}{c}{\scriptsize{(0.320)}} & \mc{1}{c}{\scriptsize{(1.000)}} & \mc{1}{c}{\scriptsize{(0.687)}} & \mc{1}{c}{\scriptsize{(0.356)}} & \mc{1}{c}{\scriptsize{(0.641)}} \\  

  &  \mc{1}{l}{\scriptsize{Graduated 4-year College}} & \mc{1}{c}{\scriptsize{30}} & \mc{1}{c}{\scriptsize{0.170}} & \mc{1}{c}{\scriptsize{0.178}} & \mc{1}{c}{\scriptsize{0.347}} & \mc{1}{c}{\scriptsize{0.100}} & \mc{1}{c}{\scriptsize{0.167}} & \mc{1}{c}{\scriptsize{0.142}} \\  

  &   &  & \mc{1}{c}{\scriptsize{(0.104)}} & \mc{1}{c}{\scriptsize{(0.202)}} & \mc{1}{c}{\scriptsize{(0.132)}} & \mc{1}{c}{\scriptsize{(0.611)}} & \mc{1}{c}{\scriptsize{(0.265)}} & \mc{1}{c}{\scriptsize{(0.224)}} \\  

  &  \mc{1}{l}{\scriptsize{Years of Education}} & \mc{1}{c}{\scriptsize{30}} & \mc{1}{c}{\scriptsize{0.525}} & \mc{1}{c}{\scriptsize{0.785}} & \mc{1}{c}{\scriptsize{1.619}} & \mc{1}{c}{\scriptsize{0.782}} & \mc{1}{c}{\scriptsize{0.649}} & \mc{1}{c}{\scriptsize{0.343}} \\  

  &   &  & \mc{1}{c}{\scriptsize{(0.295)}} & \mc{1}{c}{\scriptsize{(0.158)}} & \mc{1}{c}{\scriptsize{(1.000)}} & \mc{1}{c}{\scriptsize{(0.307)}} & \mc{1}{c}{\scriptsize{(0.275)}} & \mc{1}{c}{\scriptsize{(0.514)}} \\  

   \mc{1}{l}{\scriptsize{Labor Income}} &  \mc{1}{l}{\scriptsize{Employed}} & \mc{1}{c}{\scriptsize{30}} & \mc{1}{c}{\scriptsize{0.119}} & \mc{1}{c}{\scriptsize{0.182}} & \mc{1}{c}{\scriptsize{0.048}} & \mc{1}{c}{\scriptsize{0.039}} & \mc{1}{c}{\scriptsize{0.231}} & \mc{1}{c}{\scriptsize{0.261}} \\  

  &   &  & \mc{1}{c}{\scriptsize{(0.253)}} & \mc{1}{c}{\scriptsize{\textbf{(0.070)}}} & \mc{1}{c}{\scriptsize{(1.000)}} & \mc{1}{c}{\scriptsize{(0.822)}} & \mc{1}{c}{\scriptsize{\textbf{(0.044)}}} & \mc{1}{c}{\scriptsize{\textbf{(0.024)}}} \\  

 &   \mc{1}{l}{\scriptsize{Labor Income}} & \mc{1}{c}{\scriptsize{30}} & \mc{1}{c}{\scriptsize{19,810}} & \mc{1}{c}{\scriptsize{27,373}} & \mc{1}{c}{\scriptsize{42,616}} & \mc{1}{c}{\scriptsize{23,950}} & \mc{1}{c}{\scriptsize{26,715}} & \mc{1}{c}{\scriptsize{21,068}} \\  

 &    &  & \mc{1}{c}{\scriptsize{(0.159)}} & \mc{1}{c}{\scriptsize{(0.306)}} & \mc{1}{c}{\scriptsize{(0.313)}} & \mc{1}{c}{\scriptsize{(0.214)}} & \mc{1}{c}{\scriptsize{(0.323)}} & \mc{1}{c}{\scriptsize{(0.270)}} \\  

   \mc{1}{l}{\scriptsize{Crime}} &  \mc{1}{l}{\scriptsize{Total Felony Arrests}} & \mc{1}{c}{\scriptsize{Mid-30s}} & \mc{1}{c}{\scriptsize{0.196}} & \mc{1}{c}{\scriptsize{0.392}} & \mc{1}{c}{\scriptsize{1.481}} & \mc{1}{c}{\scriptsize{1.338}} & \mc{1}{c}{\scriptsize{0.096}} & \mc{1}{c}{\scriptsize{0.184}} \\  

 &    &  & \mc{1}{c}{\scriptsize{(0.760)}} & \mc{1}{c}{\scriptsize{(0.641)}} & \mc{1}{c}{\scriptsize{(0.242)}} & \mc{1}{c}{\scriptsize{\textbf{(0.047)}}} & \mc{1}{c}{\scriptsize{(0.883)}} & \mc{1}{c}{\scriptsize{(0.827)}} \\  

 &   \mc{1}{l}{\scriptsize{Total Misdemeanor Arrests}} & \mc{1}{c}{\scriptsize{Mid-30s}} & \mc{1}{c}{\scriptsize{-0.501}} & \mc{1}{c}{\scriptsize{-0.243}} & \mc{1}{c}{\scriptsize{-0.193}} & \mc{1}{c}{\scriptsize{-0.033}} & \mc{1}{c}{\scriptsize{-0.276}} & \mc{1}{c}{\scriptsize{-0.508}} \\  

  &   &  & \mc{1}{c}{\scriptsize{(0.345)}} & \mc{1}{c}{\scriptsize{(0.656)}} & \mc{1}{c}{\scriptsize{(0.795)}} & \mc{1}{c}{\scriptsize{(0.971)}} & \mc{1}{c}{\scriptsize{(0.655)}} & \mc{1}{c}{\scriptsize{(0.357)}} \\  

   \mc{1}{l}{\scriptsize{Health}} &  \mc{1}{l}{\scriptsize{Self-reported drug user}} & \mc{1}{c}{\scriptsize{Mid-30s}} & \mc{1}{c}{\scriptsize{-0.333}} & \mc{1}{c}{\scriptsize{-0.398}} & \mc{1}{c}{\scriptsize{-0.693}} & \mc{1}{c}{\scriptsize{-0.557}} & \mc{1}{c}{\scriptsize{-0.309}} & \mc{1}{c}{\scriptsize{-0.330}} \\  

  &   &  & \mc{1}{c}{\scriptsize{\textbf{(0.045)}}} & \mc{1}{c}{\scriptsize{\textbf{(0.016)}}} & \mc{1}{c}{\scriptsize{\textbf{(0.031)}}} & \mc{1}{c}{\scriptsize{\textbf{(0.091)}}} & \mc{1}{c}{\scriptsize{\textbf{(0.072)}}} & \mc{1}{c}{\scriptsize{\textbf{(0.075)}}} \\  

   & \mc{1}{l}{\scriptsize{Systolic Blood Pressure (mm Hg)}} & \mc{1}{c}{\scriptsize{Mid-30s}} & \mc{1}{c}{\scriptsize{-9.791}} & \mc{1}{c}{\scriptsize{-13.511}} & \mc{1}{c}{\scriptsize{19.304}} & \mc{1}{c}{\scriptsize{14.979}} & \mc{1}{c}{\scriptsize{-23.674}} & \mc{1}{c}{\scriptsize{-18.537}} \\  

   &  &  & \mc{1}{c}{\scriptsize{(0.211)}} & \mc{1}{c}{\scriptsize{(0.160)}} & \mc{1}{c}{\scriptsize{\textbf{(0.053)}}} & \mc{1}{c}{\scriptsize{\textbf{(0.000)}}} & \mc{1}{c}{\scriptsize{\textbf{(0.005)}}} & \mc{1}{c}{\scriptsize{\textbf{(0.019)}}} \\  

   & \mc{1}{l}{\scriptsize{Diastolic Blood Pressure (mm Hg)}} & \mc{1}{c}{\scriptsize{Mid-30s}} & \mc{1}{c}{\scriptsize{-10.854}} & \mc{1}{c}{\scriptsize{-16.689}} & \mc{1}{c}{\scriptsize{-11.320}} & \mc{1}{c}{\scriptsize{-8.741}} & \mc{1}{c}{\scriptsize{-19.311}} & \mc{1}{c}{\scriptsize{-13.988}} \\  

   &  &  & \mc{1}{c}{\scriptsize{\textbf{(0.074)}}} & \mc{1}{c}{\scriptsize{\textbf{(0.003)}}} & \mc{1}{c}{\scriptsize{(0.159)}} & \mc{1}{c}{\scriptsize{\textbf{(0.058)}}} & \mc{1}{c}{\scriptsize{\textbf{(0.000)}}} & \mc{1}{c}{\scriptsize{\textbf{(0.021)}}} \\  

  &  \mc{1}{l}{\scriptsize{Hypertension}} & \mc{1}{c}{\scriptsize{Mid-30s}} & \mc{1}{c}{\scriptsize{-0.291}} & \mc{1}{c}{\scriptsize{-0.352}} & \mc{1}{c}{\scriptsize{0.020}} & \mc{1}{c}{\scriptsize{-0.075}} & \mc{1}{c}{\scriptsize{-0.470}} & \mc{1}{c}{\scriptsize{-0.435}} \\  

  &   &  & \mc{1}{c}{\scriptsize{\textbf{(0.086)}}} & \mc{1}{c}{\scriptsize{\textbf{(0.100)}}} & \mc{1}{c}{\scriptsize{(0.767)}} & \mc{1}{c}{\scriptsize{(0.824)}} & \mc{1}{c}{\scriptsize{\textbf{(0.033)}}} & \mc{1}{c}{\scriptsize{\textbf{(0.020)}}} \\  


     \bottomrule
    \end{tabular} 
\end{scriptsize}
\begin{tablenotes}
\tiny
Note: This table shows the treatment effects for categories outcomes that are important for our benefit/cost analysis. Systolic and diastolic blood pressure are measured in terms of mm Hg. Each column present estimates for the following parameters: (1) $\mathbb{E} \left [ \bm{Y}^1 -  \bm{Y}^0 | \bm{B} \in \mathcal{B}_{0} \right]$ (no controls); (2) $\mathbb{E} \left [ \bm{Y}^1 -  \bm{Y}^0 | \bm{B} \in \mathcal{B}_{0} \right] $(controls); (3) $\mathbb{E} \left [ \bm{Y}^1 | R = 1 \right] -  \mathbb{E} \left [ \bm{Y}^0 | R = 0,V = 0  \right]$ (no controls); (4) $\mathbb{E} \left [ \bm{Y}^1 -  \bm{Y}_H^0 | \bm{B} \in \mathcal{B}_{0} \right]$ (controls);  (5) $\mathbb{E} \left [ \bm{Y}^1 | R = 1 \right] -  \mathbb{E} \left [ \bm{Y}^0 | R = 0,V = 1 \right]$ (no controls); (6) $\mathbb{E} \left [ \bm{Y}^1 -  \bm{Y}_C^0 | \bm{B} \in \mathcal{B}_{0} \right]$ (controls). We account for the following background variables ($\bm{B}$): Apgar scores at minutes 1 and 5 and the high-risk index. We define the high-risk index in Appendix~\ref{appendix:background} and explain how we choose the control variables in Appendix~\ref{appendix:bvariables}. Columns (2), (4), and (6) correct for item non-response and attrition using inverse probability weighting as we explain in Appendix~\ref{app:method_partialobs}. Inference is based on non-parametric, one-sided $p$-values from the empirical bootstrap distribution. We highlight point estimates significant at the $10\%$ level. See Appendix~\ref{appendix:vsensitivity} for two-sided $p$-values.
\end{tablenotes}
\end{threeparttable}
\end{table}
\restoregeometry
\doublespacing














\restoregeometry
\doublespacing 