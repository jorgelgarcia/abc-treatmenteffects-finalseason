%Input preamble
\documentclass[11pt]{article}

% colors
\usepackage[table]{xcolor}
\definecolor{maroon}{RGB}{153,0,18}
\definecolor{lime}{RGB}{190,213,88}
\definecolor{sand}{RGB}{217,202,179}
\definecolor{fire}{RGB}{144,50,61}
\definecolor{brick}{RGB}{94,11,21}
\definecolor{olive}{RGB}{117,109,84}
\definecolor{lavpink}{RGB}{172,123,132}
\definecolor{darkpurp}{RGB}{49,10,49}
\definecolor{salmon}{RGB}{204,90,113}
\definecolor{mauve}{RGB}{94,73,85}
\definecolor{greyblue}{RGB}{125,132,145}
\definecolor{greypurp}{RGB}{68,56,80}
\definecolor{brightpurp}{RGB}{96,20,255}

% packages (please add in alphabetical order)
\usepackage{adjustbox}
\usepackage{amsfonts}
\usepackage{amsmath}
\usepackage{amssymb}
\usepackage{array}
\usepackage{bm}
\usepackage{booktabs}
\usepackage{caption}
\usepackage{epstopdf}
\usepackage{float}
\usepackage[margin=1in]{geometry}
\usepackage{graphicx}
\usepackage[colorlinks=true, linkcolor=brightpurp, citecolor=brightpurp, urlcolor=salmon]{hyperref}
\usepackage{lipsum}
\usepackage{longtable}
\usepackage{mathtools}
\usepackage{multirow}
\usepackage{natbib}
\usepackage{rotating}
\usepackage{setspace}
\usepackage{subcaption}
%\usepackage{threeparttable}
\usepackage{threeparttablex}
\usepackage{xr}
\usepackage[printwatermark]{xwatermark}


\newcolumntype{L}[1]{>{\raggedright\let\newline\\\arraybackslash\hspace{0pt}}m{#1}}
\newcolumntype{C}[1]{>{\centering\let\newline\\\arraybackslash\hspace{0pt}}m{#1}}
\newcolumntype{R}[1]{>{\raggedleft\let\newline\\\arraybackslash\hspace{0pt}}m{#1}}

% commands
\newcommand{\mr}{\multirow}
\newcommand{\mc}{\multicolumn}


\begin{document}

\doublespacing

\noindent Back-of-the-envelope CBA Calculations \\

\noindent \citet{Kline-Walters_2016_QJE} perform the following exercise:\\

\noindent (i)  Calculate the treatment effect on WPPSI at age 5 for the Head Start Impact Study (HSIS)---HSIS is a one-year-long randomized version of Head Start.\\

\noindent (ii) Monetize this gain using \citet{Chetty_Friedman_etal_2010_HowDoesYour} return to WPPSI at age 5 in terms of net present value of earnings at age-27. \citet{Chetty_Friedman_etal_2010_HowDoesYour} calculations indicate that a 1 standard deviation gain in WPPSI at age 5 implies a $13.1\%$ increase in the net present value of earnings at age 27.\\

\noindent (iii) Calculate the benefit-to-cost ratio by dividing the gain in ``(ii)" by the total costs of the program. This calculation is based on giving a value to net-present value of earnings at age 27 of $\$385,907.17$ to the control-group participants, which is provided by \citet{Chetty_Friedman_etal_2010_HowDoesYour}. The treatment-group participants are given $13.1\%$ increase on this amount. \citet{Kline-Walters_2016_QJE} calculation indicates that the benefit-to-cost ratio of HSIS ranges between $1.50$ and $1.84$.\\ 

\noindent We perform three exercises related to this.\\

\noindent First, we perform the same calculation described before, but use the gain in WPPSI at age 5 and the program costs of ABC/CARE. This gain amounts to .48 of a standard deviation. Using the same return to WPPSI in terms of earnings and the same net present value of income as in\citet{Chetty_Friedman_etal_2010_HowDoesYour} we obtain a benefit-to-cost of $.45$.\\

\noindent Second, we calculate the return of WPPSI in terms of net present value of earnings. As opposed to  \citet{Chetty_Friedman_etal_2010_HowDoesYour}, we consider the net present value of life-cycle earnings and not the net-present value of earnings at age 27. We provide details on how to obtain this life-cycle value in the main body of the paper. A 1 standard deviation gain in WPPSI at age 5 implies a $84.6\%$ increase in life-cycle net present value of earnings. The net present value of life-cycle earnings for the control group is $\$298,611.90$ (in 2014 USD). Applying the ``return to WPPSI'' in terms of life-cycle earnings to this quantity to the participants of the treatment group, we are able to produce a benefit-to-cost ratio. The calculation indicates that this number is: $1.48$.\\

\noindent Finally, we calculate the return of WPPSI in terms of life-cycle net present value of all the components we monetize in the paper. A 1 standard deviation gain in WPPSI at age 5 implies an increase in the life-cycle net present value of all components of $119.03\%$. The life-cycle net present value of all components for the control group is $\$753,692.98$. Applying the ``return to WPPSI'' in terms of the life-cycle earnings to this quantity to the participants of the treatment group yields a benefit-to-cost ratio of $5.15$.\footnote{The difference between this number and the number we report in the main paper due to leaving out CARE from this exercise for simplicity.}\\

%References
\singlespace
\bibliographystyle{chicago}
\bibliography{heckman}
\end{document}