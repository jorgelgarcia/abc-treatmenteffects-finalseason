%Input preamble
\documentclass[11pt]{article}

% colors
\usepackage[table]{xcolor}
\definecolor{maroon}{RGB}{153,0,18}
\definecolor{lime}{RGB}{190,213,88}
\definecolor{sand}{RGB}{217,202,179}
\definecolor{fire}{RGB}{144,50,61}
\definecolor{brick}{RGB}{94,11,21}
\definecolor{olive}{RGB}{117,109,84}
\definecolor{lavpink}{RGB}{172,123,132}
\definecolor{darkpurp}{RGB}{49,10,49}
\definecolor{salmon}{RGB}{204,90,113}
\definecolor{mauve}{RGB}{94,73,85}
\definecolor{greyblue}{RGB}{125,132,145}
\definecolor{greypurp}{RGB}{68,56,80}
\definecolor{brightpurp}{RGB}{96,20,255}

% packages (please add in alphabetical order)
\usepackage{adjustbox}
\usepackage{amsfonts}
\usepackage{amsmath}
\usepackage{amssymb}
\usepackage{array}
\usepackage{bm}
\usepackage{booktabs}
\usepackage{caption}
\usepackage{epstopdf}
\usepackage{float}
\usepackage[margin=1in]{geometry}
\usepackage{graphicx}
\usepackage[colorlinks=true, linkcolor=brightpurp, citecolor=brightpurp, urlcolor=salmon]{hyperref}
\usepackage{lipsum}
\usepackage{longtable}
\usepackage{mathtools}
\usepackage{multirow}
\usepackage{natbib}
\usepackage{rotating}
\usepackage{setspace}
\usepackage{subcaption}
%\usepackage{threeparttable}
\usepackage{threeparttablex}
\usepackage{xr}
\usepackage[printwatermark]{xwatermark}


\newcolumntype{L}[1]{>{\raggedright\let\newline\\\arraybackslash\hspace{0pt}}m{#1}}
\newcolumntype{C}[1]{>{\centering\let\newline\\\arraybackslash\hspace{0pt}}m{#1}}
\newcolumntype{R}[1]{>{\raggedleft\let\newline\\\arraybackslash\hspace{0pt}}m{#1}}

% commands
\newcommand{\mr}{\multirow}
\newcommand{\mc}{\multicolumn}

%Other parameters
\newcommand{\noutcomes}{95}
\newcommand{\treatsubsabc}{$75\%$}
\newcommand{\treatsubscarec}{$74\%$}
\newcommand{\treatsubscaref}{$63\%$}

%Counts
%Males
\newcommand{\positivem}{$79\%$}
\newcommand{\positivesm}{$37\%$}

%Females
\newcommand{\positivef}{$73\%$}
\newcommand{\positivesf}{$35\%$}

%Counts, control substitution
%Males
\newcommand{\positivecsnm}{$58\%$}
\newcommand{\positivescsnm}{$25\%$}

\newcommand{\positivecsam}{$74\%$}
\newcommand{\positivescsam}{$38\%$}

%Females
%% no alternative
\newcommand{\positivecsnf}{$83\%$}
\newcommand{\positivescsnf}{$46\%$}

%% alternative
\newcommand{\positivecsaf}{$73\%$}
\newcommand{\positivescsaf}{$23\%$}

%Pooled

%Effects
%Males

%Females
\newcommand{\hsgradf}{$7$}
\newcommand{\yearsedf}{$1.2$}



%Pooled

%CBA
%IRR
%Males
\newcommand{\irrm}{$15\%$}
\newcommand{\irrsem}{$13\%$}

%Females
\newcommand{\irrf}{$10\%$}
\newcommand{\irrsef}{$12\%$}

%Pooled
\newcommand{\irrp}{$13\%$}
\newcommand{\irrsep}{$11\%$}

%BC
%Males
\newcommand{\bcm}{$7.88$}
\newcommand{\bcsem}{$8.06$}

%Females
\newcommand{\bcf}{$2.30$}
\newcommand{\bcsef}{$1.56$}

%Pooled
\newcommand{\bcp}{$4.35$}
\newcommand{\bcsep}{$2.57$}

%NPV streams
%Pooled
\newcommand{\parincomenpvp}{$\$115,026$}

\usepackage[stable]{footmisc}

\newcommand*\leftright[2]{%
  \leavevmode
  \rlap{#1}%
  \hspace{0.5\linewidth}%
  #2}

\newcommand{\orth}{\ensuremath{\perp\!\!\!\perp}}%
\newcommand{\indep}{\orth}%
\newcommand{\notorth}{\ensuremath{\perp\!\!\!\!\!\!\diagup\!\!\!\!\!\!\perp}}%
\newcommand{\notindep}{\notorth}

\externaldocument{abc_comprehensivecba}
\pagenumbering{roman}

\begin{document}


\doublespacing
\section*{Data Combination Estimator in GMM Framework} \label{section:datacomb}

\noindent Recall that our analysis is based on a causal model for treatment ($d=1$) and control ($d=0$) outcomes for measure $j$ at age $a$ in sample $k \in \{e,n\}$ where $e$ denotes membership in the experimental sample and $n$ denotes membership in the auxiliary sample:\\

\begin{equation}\label{eq:outcome}
Y^d_{k,j,a} = \phi^d_{k,j,a} (\bm{X}^d_{k,a}, \bm{B}_k) + \varepsilon^d_{k,j,a}, \quad j \in \mathcal{J}_a.
\end{equation}

\noindent $\phi^d_{k,j,a}\left( \cdot, \cdot \right)$ is an invariant structural production relationship mapping inputs $\bm{X}^d_{k,a}, \bm{B}_k$ into output $Y^d_{k,j,a}$ holding error term $\varepsilon^d_{k,j,a}$ fixed.\\

\noindent We show in the main text that $\phi^d_{n,j,a} (\cdot, \cdot)$ is a consistent predictor of $Y_{e,j,a}^d$. This holds at any age, although we are specifically interested in $a > a^*$, given that we do not observe $Y_{e,j,a}^d$ after $a^*$.\\

\noindent We want to take a step back and frame the estimation of $\phi^d_{n,j,a} (\cdot, \cdot)$ in the generalized method of moments (GMM) framework. This helps us illustrate how our prediction theory in the main text provides theoretical conditions for the combination of multiple datasets.\\

\noindent First, note that Assumption ~\ref{ass:exog} (Exogeneity) implies the following holds in the \textit{experimental sample}: 

\begin{eqnarray}
\mathbb{E} \left[ Y_{e,j,a}^d | \bm{X}_{e,a}^d, \bm{B}_{e} \right] =  \phi^d_{e,j,a}\left( \bm{X}_{e,a}^d, \bm{B}_{e} \right). \label{eq:main}
\end{eqnarray}

\noindent By Assumption~\ref{ass:contain} (Support Conditions), the support of the auxiliary sample contains the support of the experimental sample. Thus, Equation~\eqref{eq:main} is valid for any $\bm{X}^d_{k,a}$, $\bm{B}_{k}$, and $\bm{Y}_{k,a}$, with $k \in \{ e, n\}$. We then can write: 

\begin{eqnarray}
\mathbb{E} \left[ Y_{n,j,a}^d | \bm{X}_{n,a}^d, \bm{B}_{n} \right] =  \phi^d_{e,j,a}\left( \bm{X}_{n,a}^d, \bm{B}_{n} \right).
\end{eqnarray}

\noindent This relates the moment $\mathbb{E} \left[ Y_{n,j,a}^d | \bm{X}_{n,a}^d, \bm{B}_{n} \right]$ in the auxiliary sample to the function  $\phi^d_{e,j,a}\left( \cdot, \cdot \right)$ in the experimental sample.\\

\noindent By Assumption~\ref{ass:summary} (Invariance), we can write this as: 

\begin{eqnarray}\label{eq:momentlink}
\mathbb{E} \left[ Y_{n,j,a}^d | \bm{X}_{n,a}^d, \bm{B}_{n} \right] &=&  \phi^d_{e,j,a}\left( \bm{X}_{n,a}^d, \bm{B}_{n} \right) \nonumber \\ 
 &=&  \phi^d_{n,j,a}\left( \bm{X}_{n,a}^d, \bm{B}_{n} \right) \nonumber \\
 &=&  \phi_{j,a}\left( \bm{X}_{n,a}^d, \bm{B}_{n} \right). \label{eq:mainpred}
\end{eqnarray}

\noindent Equation~\eqref{eq:momentlink} directly states the relationship between the moment observed in the auxiliary sample, the left-hand side, and the function that we use to construct the predictions. Our objective is to estimate the parameters characterizing this function, $\bm{\theta}_{j,a}$.\\ 

\noindent The key moment condition implied by Assumption~\ref{ass:exog} (Exogeneity) is 

\begin{equation}
\mathbb{E} \left[ \bm{m} \left( \bm{\theta}_{j,a} \right) \right] = 0, \label{eq:moment}
\end{equation}

\noindent where $\bm{m} \left( \bm{\theta}_{j,a} \right) := {\bm{X_{n,a}^d}}^{'} \left( Y_{n,j,a}^d -   \phi^d_{j,a} \left( \bm{X}_{n,a}^d, \bm{B}_{n} \right) \right)$. This condition enables us to discuss three issues: 

\begin{enumerate}
\item \textbf{Initializing the dynamic prediction.} 
\noindent For the time being, assume that we do not place specific weights on the different moments implied by \eqref{eq:moment}. We come back to this issue below. Then, a valid estimation approach is to minimize the following quadratic loss function: 

\begin{equation}
q :=  {\bm{m} \left( \bm{\theta}_{j,a} \right)}^{'}{\bm{m} \left( \bm{\theta}_{j,a} \right)}, \label{eq:loss}
\end{equation}

\noindent which provides us with a standard non-linear least square estimator of $\bm{\theta}_{j,a}$.\\

\noindent $\bm{X}_{n,a}^d$ contains a one-period lagged value of $Y_{n,j,a}^d$. To construct our predictions, we do not require an estimate of $\phi_{j,a}$ at age $a^*$. We observe $Y_{e,j,a}$ at $a^*$. We need an estimate of $\phi_{j,a}$ for $a > a^*$.\\ 

\noindent To illustrate how we use the estimates to form the predictions dynamically, suppose there are two periods after $a^*$. We can stack the moment conditions for each of these ages and write: 

\begin{equation}
\mathbb{E} \left[  \left[ \bm{m} \left( \bm{\theta}_{j,a^* + 1}  \right)   , \bm{m} \left( \bm{\theta}_{j,a^* + 2} \right)  \right]'  \right] = \bm{0}, 
\end{equation}

\noindent and write a corresponding loss function. Once the estimates of $\bm{\theta}_{j,a^* + 1}, \bm{\theta}_{j,a^* + 2}$ are available, we form the predictions as follows: 

\begin{enumerate}
\item Form the prediction of $Y_{e,j,a}$ at $a^* + 1$. This is feasible because we can evaluate $ \hat{\phi}_{j,a}\left( \bm{X}_{n,a^*+1}^d, \bm{B}_{n} \right)$ in the experimental sample given that we observe $Y_{e,j,a^*}$.

\item Form the prediction of $Y_{e,j,a}$ at $a^* + 2$. This is feasible because we can evaluate $ \hat{\phi}_{j,a}\left( \bm{X}_{n,a^*+2}^d, \bm{B}_{n} \right)$ in the experimental sample given that we have a prediction of $Y_{e,j,a^*+1}$ from the previous step (this is the value at which we evaluate). The rest of the variables in $\bm{X}_{n,a}^d$ are post-treatment variables that we observe in the experimental sample at any age $a \leq a^*$.
\end{enumerate}

\item \textbf{Pooling information from different datasets.} Minimizing the loss function in \eqref{eq:loss} provides a consistent estimate of $\bm{\theta}_{j,a}$, but we can do better. We can minimize a weighted loss of the form 

\begin{equation}
q :=  {\bm{m} \left( \bm{\theta}_{j,a} \right)}^{'} W_{j,a}^{-1} {\bm{m} \left( \bm{\theta}_{j,a} \right)}, \label{eq:wloss},
\end{equation}

\noindent where $W_{j,a}^{-1}$ is the inverse of the variance of the set of moments in $\bm{m} \left( \bm{\theta}_{j,a} \right)$. More generally, we can stack the moments as we do before and think of the problem as 

\begin{equation}
q :=  {\bm{m} \left( \bm{\theta}_{j} \right)}^{'} W ^{-1} {\bm{m} \left( \bm{\theta}_{j}\right)}, \label{eq:wloss}
\end{equation}

\noindent where $\bm{m} \left( \bm{\theta}_{j} \right)$ and  $\bm{\theta}_{j}$  stack the moment conditions and parameters of at all ages $a > a^*$.\\

\noindent It is of specific interest for us to weight the variance of the moments in \eqref{eq:wloss} because, in order to satisfy Assumption~\ref{ass:contain} (Support Conditions), we pool different auxiliary data sources (CNLSY, PSID and NLSY79), as we explain in the main text. $W ^{-1} $ is not restricted to be diagonal so that these moments are allowed to correlate. Iterated, feasible procedures to obtain an estimate of $W ^{-1}$ jointly with the parameters of interest guarantee efficiency and are straightforward to implement \citep{Hansen_1982_Econometrica,Amemiya_1985_advanced}.

\item \textbf{Serial correlation.} Assumption~\ref{ass:exog} (Exogeneity) does not rule out serial correlation of $\varepsilon^d_{n,j,a}$. To account for this, it is possible to bootstrap the two steps described before to obtain the standard errors of the estimates of $ \bm{\theta}_{j,a}$. To do that, we resample the auxiliary sample with replacement at the individual level. This gives us a new (re-sampled) panel dataset in each bootstrapped sample. Information on the entire history of each individual is obtained in each re-sample. Thus, criterion in \eqref{eq:wloss} is minimized with observations that are all allowed to serially correlate at the individual level.\\

\noindent The bootstrap procedure in the experimental sample is analogous.

\end{enumerate}









%References
\pagebreak
\singlespace
\bibliographystyle{chicago}
\bibliography{heckman}

\end{document}


