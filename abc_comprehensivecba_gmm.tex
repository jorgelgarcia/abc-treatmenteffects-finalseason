%Input preamble
\documentclass[11pt]{article}

% colors
\usepackage[table]{xcolor}
\definecolor{maroon}{RGB}{153,0,18}
\definecolor{lime}{RGB}{190,213,88}
\definecolor{sand}{RGB}{217,202,179}
\definecolor{fire}{RGB}{144,50,61}
\definecolor{brick}{RGB}{94,11,21}
\definecolor{olive}{RGB}{117,109,84}
\definecolor{lavpink}{RGB}{172,123,132}
\definecolor{darkpurp}{RGB}{49,10,49}
\definecolor{salmon}{RGB}{204,90,113}
\definecolor{mauve}{RGB}{94,73,85}
\definecolor{greyblue}{RGB}{125,132,145}
\definecolor{greypurp}{RGB}{68,56,80}
\definecolor{brightpurp}{RGB}{96,20,255}

% packages (please add in alphabetical order)
\usepackage{adjustbox}
\usepackage{amsfonts}
\usepackage{amsmath}
\usepackage{amssymb}
\usepackage{array}
\usepackage{bm}
\usepackage{booktabs}
\usepackage{caption}
\usepackage{epstopdf}
\usepackage{float}
\usepackage[margin=1in]{geometry}
\usepackage{graphicx}
\usepackage[colorlinks=true, linkcolor=brightpurp, citecolor=brightpurp, urlcolor=salmon]{hyperref}
\usepackage{lipsum}
\usepackage{longtable}
\usepackage{mathtools}
\usepackage{multirow}
\usepackage{natbib}
\usepackage{rotating}
\usepackage{setspace}
\usepackage{subcaption}
%\usepackage{threeparttable}
\usepackage{threeparttablex}
\usepackage{xr}
\usepackage[printwatermark]{xwatermark}


\newcolumntype{L}[1]{>{\raggedright\let\newline\\\arraybackslash\hspace{0pt}}m{#1}}
\newcolumntype{C}[1]{>{\centering\let\newline\\\arraybackslash\hspace{0pt}}m{#1}}
\newcolumntype{R}[1]{>{\raggedleft\let\newline\\\arraybackslash\hspace{0pt}}m{#1}}

% commands
\newcommand{\mr}{\multirow}
\newcommand{\mc}{\multicolumn}

%Other parameters
\newcommand{\noutcomes}{95}
\newcommand{\treatsubsabc}{$75\%$}
\newcommand{\treatsubscarec}{$74\%$}
\newcommand{\treatsubscaref}{$63\%$}

%Counts
%Males
\newcommand{\positivem}{$79\%$}
\newcommand{\positivesm}{$37\%$}

%Females
\newcommand{\positivef}{$73\%$}
\newcommand{\positivesf}{$35\%$}

%Counts, control substitution
%Males
\newcommand{\positivecsnm}{$58\%$}
\newcommand{\positivescsnm}{$25\%$}

\newcommand{\positivecsam}{$74\%$}
\newcommand{\positivescsam}{$38\%$}

%Females
%% no alternative
\newcommand{\positivecsnf}{$83\%$}
\newcommand{\positivescsnf}{$46\%$}

%% alternative
\newcommand{\positivecsaf}{$73\%$}
\newcommand{\positivescsaf}{$23\%$}

%Pooled

%Effects
%Males

%Females
\newcommand{\hsgradf}{$7$}
\newcommand{\yearsedf}{$1.2$}



%Pooled

%CBA
%IRR
%Males
\newcommand{\irrm}{$15\%$}
\newcommand{\irrsem}{$13\%$}

%Females
\newcommand{\irrf}{$10\%$}
\newcommand{\irrsef}{$12\%$}

%Pooled
\newcommand{\irrp}{$13\%$}
\newcommand{\irrsep}{$11\%$}

%BC
%Males
\newcommand{\bcm}{$7.88$}
\newcommand{\bcsem}{$8.06$}

%Females
\newcommand{\bcf}{$2.30$}
\newcommand{\bcsef}{$1.56$}

%Pooled
\newcommand{\bcp}{$4.35$}
\newcommand{\bcsep}{$2.57$}

%NPV streams
%Pooled
\newcommand{\parincomenpvp}{$\$115,026$}

\usepackage[stable]{footmisc}

\newcommand*\leftright[2]{%
  \leavevmode
  \rlap{#1}%
  \hspace{0.5\linewidth}%
  #2}

\newcommand{\orth}{\ensuremath{\perp\!\!\!\perp}}%
\newcommand{\indep}{\orth}%
\newcommand{\notorth}{\ensuremath{\perp\!\!\!\!\!\!\diagup\!\!\!\!\!\!\perp}}%
\newcommand{\notindep}{\notorth}

\externaldocument{abc_comprehensivecba}
\externaldocument{abc_comprehensivecba_appendix-pub}
\pagenumbering{roman}

\begin{document}


\doublespacing

\noindent \textbf{[JJH: Conceptually just use $Y_t = \alpha + \beta X_t + \phi + \varepsilon_t \quad \varepsilon_t = \rho \varepsilon_{t-1} + \eta_t$. We then form moments. Please do this.]}

\noindent \textbf{[JLG: This is the updated estimation procedure and framing in GMM. It's based on our discussion of this afternoon.]} \\
\noindent \textbf{[JLG: Next steps are (i) estimate the model using 21 and 30, to have a comparison; (ii) matching. I'm hoping to get this before the week ends.]}\\
\noindent \textbf{[JLG: This does not account yet for correlation. I'm waiting for our meeting so we can agree on how to deal with this and then I'll update it right the way.]}

\subsection{Estimation Procedure and Data Combination Estimator in the GMM Framework} \label{appendix:gmm}

\noindent Our analysis is based on a causal model for treatment ($d=1$) and control ($d=0$) outcomes for measure $j$ at age $a$ in sample $k \in \{e,n\}$ where $e$ denotes membership in the experimental sample and $n$ denotes membership in the auxiliary sample:\\

\begin{equation}\label{eq:outcome}
Y^d_{k,j,a} = \phi^d_{k,j,a} (\bm{X}^d_{k,a}, \bm{B}_k) + \varepsilon^d_{k,j,a}, \quad k \in \{n,e\}, \quad j \in \mathcal{J}_a, \quad d \in \{0, 1\}.
\end{equation}

\noindent $\phi^d_{k,j,a}\left( \cdot, \cdot \right)$ is an invariant structural production relationship mapping inputs $\bm{X}^d_{k,a}, \bm{B}_k$ into output $Y^d_{k,j,a}$ holding error term $\varepsilon^d_{k,j,a}$ fixed. We assume Assumption ~\ref{ass:exog} (Exogeneity) hold. That is, once we condition on $\bm{X}^d_{k,a}, \bm{B}_k$, $ \varepsilon^d_{k,j,a}$ is not serially correlated. \textbf{[JJH: No, no, no! No, we reject this -- we cannot assume it.]}\\

\noindent In this section, we explain: (i) the procedure that we follow to form out-of-sample predictions; and (ii) formulate the estimation procedure in the generalized method of moments (GMM) framework.\\

\noindent In the auxiliary sample, we observe the outcome $Y_{n,j,a}$ for $a \in [a^*, \ldots, A]$. In the experimental sample, we observe the outcome $Y_{e,j,a}$ at most at two ages, depending on the outcome. For the time being, suppose that we observe the outcome at one age. We come back to this below.\\

\noindent Before explaining the estimation procedure, note that Assumption~\ref{ass:summary} (Invariance) implies that $\phi_{k,j,a}^d \left (\cdot, \cdot \right) = \phi_{k,j,a}  \left (\cdot, \cdot \right) = \phi_{j,a}  \left (\cdot, \cdot \right)$. That is, invariance holds across the treatment and the control groups and invariance holds across the experimental and the auxiliary samples. It is important to note that invariance across the treatment and the control groups implies that the variables $\bm{X}_{k,a}^d$ summarize the effect of the treatment on the outcome. Given this and Assumption ~\ref{ass:exog} (Exogeneity), the distribution of $\varepsilon_{k,j,a}^d$ is the same across the treatment and the control groups. We then drop the superindex in $\varepsilon_{k,j,a}^d$.\\

\noindent We test invariance across the treatment and the control groups and invariance across the experimental and the auxiliary samples  in Appendix~\ref{app:invariance}.\\

\noindent In Appendix~\ref{app:containsupport}, we also document that the support of $Y_{n,j,a}^d, \bm{X}_{n,a}^d, \bm{B}_{n}$ covers the support of $Y_{e,j,a}^d, \bm{X}_{e,a}, \bm{B}_{e}$ for $d \in \{0, 1\}$. So we drop the $d$ superindex in $Y_{n,j,a}^d, \bm{X}_{n,a}^d$ given that we estimate an invariant model. We write:

\begin{equation}\label{eq:routcome}
Y_{k,j,a} = \phi_{j,a} (\bm{X}_{k,a}, \bm{B}_k) + \varepsilon_{k,j,a}, \quad k \in \{n,e\}, \quad j \in \mathcal{J}_a.
\end{equation}

\noindent First, we explain our estimation and prediction procedure. It is as follows:

\begin{enumerate}
\item Use the auxiliary sample ($n$) to estimate the the coefficients characterizing $\phi_{j,a} \left( \cdot , \cdot \right)$.\footnote{In practice, we use a weighted version of the auxiliary samples. The weights give relatively high importance to the individuals in the auxiliary sample whose characteristics $\bm{B}_k$ are close to the those of the individuals in the experimental sample. See Appendix~\ref{appendix:match}.}

We denote these coefficients by $\bm{\theta}_{j,a}$ and the estimate of this function as $\hat{\phi}_{j,a} \left( \cdot , \cdot \right)$. At each age, we are able to compute the residuals from this estimation procedure as follows:

\begin{equation}
Y_{n,j,a} -  \hat{\phi}_{j,a} (\bm{X}_{k,a}, \bm{B}_k) : = \hat{\varepsilon}_{n,j,a}.
\end{equation}

For outcome $j$, we form the vector of residuals $\hat{\bm{\varepsilon}}_{n,j} : = \left[ \hat{\varepsilon_{n,j,a^*+1}}, \ldots, \hat{\varepsilon_{n,j,A}} \right]$.

\item At age $a^*+1$, we construct the predicted outcome for the experimental sample (e) for each individual as follows:

\begin{equation}
\hat{Y}_{e,j,a^*+1} = \hat{\phi}_{j,a^*+1} \left( \bm{X}_{e,a^*+1}, \bm{B}_e \right).
\end{equation}

\noindent We are able to evaluate $\hat{\phi}_{j,a^*+1}$ at $ \bm{X}_{e,a^*+1}, \bm{B}_e $ even when $\bm{X}_{e,a^*+1}$ contains a one-period lag of $Y_{e,j,a^*+1}$ because we observe $Y_{e,j,a^*}$. This prediction does not account for estimation error. We discuss estimation error below.

\item At age $a^*+2$, we construct the predicted outcome in the experimental sample (e) as follows:

\begin{equation}
\hat{Y}_{e,j,a^*+2} = \hat{\phi}_{j,a^*+1} \left( \bm{X}_{e,a^*+1}, \bm{B}_e \right).
\end{equation}

\noindent We are able to evaluate $\hat{\phi}_{j,a^*+2}$ at $ \bm{X}_{e,a^*+2}, \bm{B}_e $ even when $\bm{X}_{e,a^*+2}$ contains a one-period lag of $Y_{e,j,a^*+2}$ because we observe have a prediction of $Y_{e,j,a^*+1}$ from the step before (we evaluate at the predicted value).

\item We iterate this procedure up to age $A$. For outcome $j$, we form the vector of predictions $\hat{\bm{Y}}_{e,j} : = \left[ \hat{Y}_{e,j,a^*+1}, \ldots,  \hat{Y}_{e,j,A} \right]$.

\item Under Assumption~\ref{ass:summary} (Invariance), $\hat{\bm{\varepsilon}}_{n,j}$ is a consistent estimate of $\hat{\bm{\varepsilon}}_{e,j}$. Thus, we form a prediction that accounts for prediction error as follows:

\begin{equation}
\tilde{\bm{Y}}_{e,j} = \hat{\bm{Y}}_{e,j} + \hat{\bm{\varepsilon}}_{n,j}.
\end{equation}

\noindent In practice, we randomly sample a vector of residuals from an individual $j$ in the auxiliary sample ($n$) and pair it with the vector $\hat{\bm{Y}}_{e,j}$ of individual $i$ in the experimental sample ($e$) to form the prediction $\tilde{\bm{Y}}_{e,j}$ for individual $i$ in the experimental sample. Random sampling is valid under invariance and exogeneity, i.e. under this assumption the vector of residuals from any individual $j$ in the auxiliary sample is a valid estimate for the vector of residuals of any individual $i$ in the experimental sample.
\end{enumerate}

\noindent Second, we formulate our estimation in the GMM framework. To this end, note that Assumption ~\ref{ass:exog} (Exogeneity) and Assumption~\ref{ass:summary} (Invariance) imply the following moment condition:

\begin{equation}
\mathbb{E} \left[ \bm{m}_{j,a} \left( \bm{X}_{n,a}^d, \bm{B}_{n}; \bm{\theta}_{j,a} \right) \right] = 0,  \quad k \in \{n,e\}, \quad j \in \mathcal{J}_a \label{eq:moment}
\end{equation}

\noindent where $\bm{m}_{j,a} \left( \bm{X}_{n,a}, \bm{B}_{n} ; \bm{\theta}_{j,a} \right) := {\bm{X_{n,a}}}^{'} \left( Y_{n,j,a}^d -   \phi_{j,a} \left( \bm{X}_{n,a}^d, \bm{B}_{n} \right) \right)$ for $a \in [0, \ldots A]$.\\

\noindent We use the auxiliary sample ($n$) to estimate the vector of coefficients. Let $\bm{m} \left ( \cdot, \bm{\theta} \right)$, stack the function $\bm{m}_{j,a} \left( \bm{X}_{n,a}, \bm{B}_{n} ; \bm{\theta}_{j,a} \right)$  for all $j \in \mathcal{J}_a$, all $a \in [0, \ldots A]$, and $k = n$.\\

\noindent Observing the outcomes at age $a^*$ provides us with additional moment conditions. To see this, note that, in our analysis,  $\bm{X}_{k,a^*+1}$ contains a lagged variable of the outcome to predict and define the moment: $h_{j,a^*+1}  \left( \bm{X}_{e,a^*+1}, \bm{B}_{n} ; \bm{\theta}_{j,a^*+1} \right) =:  {\bm{X}_{e,a^*+1}}^{'} \left( \hat{Y}_{e,j,a^*+1} - \phi_{j,a^*+1} \left ( \bm{X}_{e,a^*+1}, \bm{B}_{e} \right) \right)$, where $\hat{Y}_{e,j,a^*+1}$ is defined as before. Although this moment uses information in the auxilliary sample (through the construction of $\hat{Y}_{e,j,a^*+1}$), it provides additional information (not in \eqref{eq:moment}) through $\bm{X}_{e,a^*+1}$.\\

\noindent For some outcomes, we observe intermittent information in the experimental sample. For example, we observe labor and transfer income at ages 21 and 30. In this case, we have two additional moments, not only one. Stack these set of additional moments and let  $\bm{h} \left ( \cdot, \bm{\theta} \right)$ denote them. These additional set of moments overidentify the parameter vector of interest, $\bm{\theta}$. Standard procedures allow us to use these set of additional moments to improve efficiency.\\

\noindent Let $\bm{W}$ be a positive definite matrix. A consistent estimation approach for $\bm{\theta}$ is to minimize

\begin{equation}
Q :=  {\begin{bmatrix} {\bm{\bar{m}} \left( \cdot ; \bm{\theta} \right) }  \\ {\bm{\bar{h}} \left( \cdot ; \bm{\theta} \right) }  \end{bmatrix}}^{'}
\bm{W} ^{-1}{\begin{bmatrix} {\bm{\bar{m}} \left( \cdot ; \bm{\theta} \right) }  \\ {\bm{\bar{h}} \left( \cdot ; \bm{\theta} \right) }  \end{bmatrix}}, \label{eq:wloss}
\end{equation}\\

\noindent where $\bar{u}$ denotes the empirical counterpart of $u$.\\

\noindent $\bm{W}$ is not restricted to be diagonal so that these moments are allowed to correlate. Iterated, feasible procedures to obtain an estimate of $\bm{W}$ jointly with the parameters of interest guarantee efficiency and are straightforward to implement \citep{Hansen_1982_Econometrica,Amemiya_1985_advanced}.\footnote{\citet{Altonji_Segal_1996_JoBaES} show that GMM presents downwards bias in absolute value in small-sample size setting, which could be a concern in our setting.}\\

\noindent We explain the samples used to construct each empirical counterpart and the procedure to obtain standard errors on the predictions in Appendix~\ref{app:method_noobs} and Appendix~\ref{appendix:bootstrap}, respectively.

\textbf{[JJH: We have to account for serial correlation.]}

%References
\singlespace
\bibliographystyle{chicago}
\bibliography{heckman}

\end{document}


