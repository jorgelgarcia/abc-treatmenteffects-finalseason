%Input preamble
\documentclass[11pt]{article}

% colors
\usepackage[table]{xcolor}
\definecolor{maroon}{RGB}{153,0,18}
\definecolor{lime}{RGB}{190,213,88}
\definecolor{sand}{RGB}{217,202,179}
\definecolor{fire}{RGB}{144,50,61}
\definecolor{brick}{RGB}{94,11,21}
\definecolor{olive}{RGB}{117,109,84}
\definecolor{lavpink}{RGB}{172,123,132}
\definecolor{darkpurp}{RGB}{49,10,49}
\definecolor{salmon}{RGB}{204,90,113}
\definecolor{mauve}{RGB}{94,73,85}
\definecolor{greyblue}{RGB}{125,132,145}
\definecolor{greypurp}{RGB}{68,56,80}
\definecolor{brightpurp}{RGB}{96,20,255}

% packages (please add in alphabetical order)
\usepackage{adjustbox}
\usepackage{amsfonts}
\usepackage{amsmath}
\usepackage{amssymb}
\usepackage{array}
\usepackage{bm}
\usepackage{booktabs}
\usepackage{caption}
\usepackage{epstopdf}
\usepackage{float}
\usepackage[margin=1in]{geometry}
\usepackage{graphicx}
\usepackage[colorlinks=true, linkcolor=brightpurp, citecolor=brightpurp, urlcolor=salmon]{hyperref}
\usepackage{lipsum}
\usepackage{longtable}
\usepackage{mathtools}
\usepackage{multirow}
\usepackage{natbib}
\usepackage{rotating}
\usepackage{setspace}
\usepackage{subcaption}
%\usepackage{threeparttable}
\usepackage{threeparttablex}
\usepackage{xr}
\usepackage[printwatermark]{xwatermark}


\newcolumntype{L}[1]{>{\raggedright\let\newline\\\arraybackslash\hspace{0pt}}m{#1}}
\newcolumntype{C}[1]{>{\centering\let\newline\\\arraybackslash\hspace{0pt}}m{#1}}
\newcolumntype{R}[1]{>{\raggedleft\let\newline\\\arraybackslash\hspace{0pt}}m{#1}}

% commands
\newcommand{\mr}{\multirow}
\newcommand{\mc}{\multicolumn}

%Other parameters
\newcommand{\noutcomes}{95}
\newcommand{\treatsubsabc}{$75\%$}
\newcommand{\treatsubscarec}{$74\%$}
\newcommand{\treatsubscaref}{$63\%$}

%Counts
%Males
\newcommand{\positivem}{$79\%$}
\newcommand{\positivesm}{$37\%$}

%Females
\newcommand{\positivef}{$73\%$}
\newcommand{\positivesf}{$35\%$}

%Counts, control substitution
%Males
\newcommand{\positivecsnm}{$58\%$}
\newcommand{\positivescsnm}{$25\%$}

\newcommand{\positivecsam}{$74\%$}
\newcommand{\positivescsam}{$38\%$}

%Females
%% no alternative
\newcommand{\positivecsnf}{$83\%$}
\newcommand{\positivescsnf}{$46\%$}

%% alternative
\newcommand{\positivecsaf}{$73\%$}
\newcommand{\positivescsaf}{$23\%$}

%Pooled

%Effects
%Males

%Females
\newcommand{\hsgradf}{$7$}
\newcommand{\yearsedf}{$1.2$}



%Pooled

%CBA
%IRR
%Males
\newcommand{\irrm}{$15\%$}
\newcommand{\irrsem}{$13\%$}

%Females
\newcommand{\irrf}{$10\%$}
\newcommand{\irrsef}{$12\%$}

%Pooled
\newcommand{\irrp}{$13\%$}
\newcommand{\irrsep}{$11\%$}

%BC
%Males
\newcommand{\bcm}{$7.88$}
\newcommand{\bcsem}{$8.06$}

%Females
\newcommand{\bcf}{$2.30$}
\newcommand{\bcsef}{$1.56$}

%Pooled
\newcommand{\bcp}{$4.35$}
\newcommand{\bcsep}{$2.57$}

%NPV streams
%Pooled
\newcommand{\parincomenpvp}{$\$115,026$}

\usepackage[stable]{footmisc}

\newcommand*\leftright[2]{%
  \leavevmode
  \rlap{#1}%
  \hspace{0.5\linewidth}%
  #2}

\newcommand{\orth}{\ensuremath{\perp\!\!\!\perp}}%
\newcommand{\indep}{\orth}%
\newcommand{\notorth}{\ensuremath{\perp\!\!\!\!\!\!\diagup\!\!\!\!\!\!\perp}}%
\newcommand{\notindep}{\notorth}

\externaldocument{abc_comprehensivecba}
\pagenumbering{roman}

\begin{document}


\doublespacing
\section*{Data Combination Estimator in GMM Framework} \label{section:datacomb}

\noindent Recall that our analysis is based on a causal model for treatment ($d=1$) and control ($d=0$) outcomes for measure $j$ at age $a$ in sample $k \in \{e,n\}$ where $e$ denotes membership in the experimental sample and $n$ denotes membership in the auxiliary sample:\\

\begin{equation}\label{eq:outcome}
Y^d_{k,j,a} = \phi^d_{k,j,a} (\bm{X}^d_{k,a}, \bm{B}_k) + \varepsilon^d_{k,j,a}, \quad j \in \mathcal{J}_a.
\end{equation}

\noindent $\phi^d_{k,j,a}\left( \cdot, \cdot \right)$ is an invariant structural production relationship mapping inputs $\bm{X}^d_{k,a}, \bm{B}_k$ into output $Y^d_{k,j,a}$ holding error term $\varepsilon^d_{k,j,a}$ fixed.\\

\noindent We show in the main text that $\phi^d_{n,j,a} (\cdot, \cdot)$ is a consistent predictor of $Y_{e,j,a}^d$. This holds at any age, although we are specifically interested in $a > a^*$, given that we do not observe $Y_{e,j,a}^d$ after $a^*$.\\

\noindent We want to take a step back and frame the estimation of $\phi^d_{n,j,a} (\cdot, \cdot)$ in the generalized method of moments (GMM) framework. This helps us illustrate how our prediction theory in the main text provides theoretical conditions for the combination of multiple datasets.\\

\noindent First, note that Assumption~\ref{ass:exog} (Exogeneity) implies

\begin{eqnarray}
\mathbb{E} \left[ Y_{e,j,a}^d | \bm{X}_{e}^d, \bm{B}_{e} \right] =  \phi^d_{e,j,a}\left( \bm{X}_{e}^d, \bm{B}_{e} \right). \label{eq:main}
\end{eqnarray}

\noindent By Assumption~\ref{ass:contain} (Support Conditions), the support of the auxiliary sample contains the support of the experimental sample. Thus, Equation~\eqref{eq:main} is valid for any $\bm{X}^d_{k,a}$, $\bm{B}$, and $\bm{Y}_{k,a}$, with $k \in \{ e, n\}$. We then can write: 

\begin{eqnarray}
\mathbb{E} \left[ Y_{n,j,a}^d | \bm{X}_{n}^d, \bm{B}_{n} \right] =  \phi^d_{e,j,a}\left( \bm{X}_{n}^d, \bm{B}_{n} \right).
\end{eqnarray}

\noindent By Assumption~\ref{ass:summary} (Invariance), we can write this as: 

\begin{eqnarray}
\mathbb{E} \left[ Y_{n,j,a}^d | \bm{X}_{n}^d, \bm{B}_{n} \right] &=&  \phi^d_{e,j,a}\left( \bm{X}_{n}^d, \bm{B}_{n} \right) \nonumber \\ 
 &=&  \phi^d_{n,j,a}\left( \bm{X}_{n}^d, \bm{B}_{n} \right) \nonumber \\
 &=&  \phi_{j,a}\left( \bm{X}_{n}^d, \bm{B}_{n} \right). \label{eq:mainpred}
\end{eqnarray}

\noindent Equation~\eqref{eq:main} is key in framing the data combination approach we use in the GMM framework. Its states that the prediction function $\phi_{j,a} \left( \cdot, \cdot \right) $ is invariance across (i) the auxiliary and experimental datasets; and (ii) the control and treatment groups. Thus, it enables to predict (identify and estimate) the $Y_{e,j,a}^d$ for $d \in \{0, 1\}$ based on the data moment $\mathbb{E} \left[ Y_{n,j,a}^d | \bm{X}_{n}^d, \bm{B}_{n} \right] $.\\

\noindent Let $\bm{\theta}_{j,a}$ be the set of parameters characterizing $\phi_{j,a}\left( \cdot, \cdot \right)$. It is useful to consider a standard estimation approach: non-linear least squares (NLS). That is, 

\begin{equation}
 \bm{\theta}_{j,a}^{\text{NLS}} := \argmin_{\bm{\theta}_{j,a}} \sum \limits _{i \in \mathcal{I}} {\left( Y^d_{n,j,a}  - \phi_{j,a} (\cdot , \cdot); \bm{\theta}_{j,a} \right)}^2, 
\end{equation}

\noindent where $\mathcal{I}$ indexes the individuals in the sample. The conditions provided in the main text grant identification of  $\bm{\theta}_{j,a}^{\text{NLS}}$ (see Theorem~\ref{theorem:main}). The first order condition for the NLS estimator is: 

\begin{eqnarray} 
\sum \limits _{i \in \mathcal{I}}  \psi \left( \cdot , \cdot); \bm{\theta}_{j,a} \right) \left( Y^d_{n,j,a}  - \phi_{j,a} (\cdot , \cdot; \bm{\theta}_{j,a} \right) = 0, \label{eq:nlsfoc}
\end{eqnarray}

\noindent where $\psi \left( \cdot , \cdot; \bm{\theta}_{j,a} \right) := \frac{ \phi_{j,a} \left( \cdot , \cdot; \bm{\theta}_{j,a} \right)}{\partial \bm{\theta}_{j,a}}$.\\

\noindent Equation~\eqref{eq:main} provides a ``natural'' moment condition for framing the problem in the GMM framework. That is, the moment condition allowing us to identify (and estimate) the post-$a^*$ prediction for the experimental sample as a function of observed data in the auxiliary sample: 

\begin{equation}
\mathbb{E} \left[\psi \left( \cdot , \cdot; \bm{\theta}_{j,a} \right) \left( Y^d_{n,j,a}  - \phi_{j,a} (\cdot , \cdot; \bm{\theta}_{j,a} \right) \right]  = 0.
\end{equation}



%References
\pagebreak
\singlespace
\bibliographystyle{chicago}
\bibliography{heckman}

\end{document}


