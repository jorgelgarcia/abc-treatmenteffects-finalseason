%Input preamble
\documentclass[11pt]{article}

% colors
\usepackage[table]{xcolor}
\definecolor{maroon}{RGB}{153,0,18}
\definecolor{lime}{RGB}{190,213,88}
\definecolor{sand}{RGB}{217,202,179}
\definecolor{fire}{RGB}{144,50,61}
\definecolor{brick}{RGB}{94,11,21}
\definecolor{olive}{RGB}{117,109,84}
\definecolor{lavpink}{RGB}{172,123,132}
\definecolor{darkpurp}{RGB}{49,10,49}
\definecolor{salmon}{RGB}{204,90,113}
\definecolor{mauve}{RGB}{94,73,85}
\definecolor{greyblue}{RGB}{125,132,145}
\definecolor{greypurp}{RGB}{68,56,80}
\definecolor{brightpurp}{RGB}{96,20,255}

% packages (please add in alphabetical order)
\usepackage{adjustbox}
\usepackage{amsfonts}
\usepackage{amsmath}
\usepackage{amssymb}
\usepackage{array}
\usepackage{bm}
\usepackage{booktabs}
\usepackage{caption}
\usepackage{epstopdf}
\usepackage{float}
\usepackage[margin=1in]{geometry}
\usepackage{graphicx}
\usepackage[colorlinks=true, linkcolor=brightpurp, citecolor=brightpurp, urlcolor=salmon]{hyperref}
\usepackage{lipsum}
\usepackage{longtable}
\usepackage{mathtools}
\usepackage{multirow}
\usepackage{natbib}
\usepackage{rotating}
\usepackage{setspace}
\usepackage{subcaption}
%\usepackage{threeparttable}
\usepackage{threeparttablex}
\usepackage{xr}
\usepackage[printwatermark]{xwatermark}


\newcolumntype{L}[1]{>{\raggedright\let\newline\\\arraybackslash\hspace{0pt}}m{#1}}
\newcolumntype{C}[1]{>{\centering\let\newline\\\arraybackslash\hspace{0pt}}m{#1}}
\newcolumntype{R}[1]{>{\raggedleft\let\newline\\\arraybackslash\hspace{0pt}}m{#1}}

% commands
\newcommand{\mr}{\multirow}
\newcommand{\mc}{\multicolumn}


\begin{document}

\subsection{Using our Estimates to Understand Recent Cost-benefit Analyses}

\noindent An example of the approach recent studies take on cost-benefit analyses is \citet{Kline-Walters_2016_QJE}. They use data of the Head Start Impact Study (HSIS) to evaluate Head Start and find that its benefit/cost between $1.50$ and $1.84$.\footnote{HSIS is a one-year-long randomized version of Head Start.} They proceed in three steps: (i) calculate the treatment effect on Wechsler Preschool and Primary Scale of Intelligence (WPPSI) at age 5; (ii) monetize this gain using the return to WPPSI at age 5 in terms of net present value of earnings at age 27 \citep{Chetty_Friedman_etal_2011_QJoE}. Calculations from \citet{Chetty_Friedman_etal_2011_QJoE} indicate that a 1 standard deviation gain in WPPSI at age 5 implies a $13.1\%$ increase in the net present value of earnings at age 27;\footnote{This is based on combining information from Project Star and administrative data at age 27.} and (iii) calculate the benefit-to-cost ratio based on this gain and their own calculations of the program's cost. This calculation is based on assigning the net present value of earnings at age 27 of $\$385,907.17$ to the control-group participants, which is provided by  \citet{Chetty_Friedman_etal_2011_QJoE}.\footnote{All money values that we provide in this section are in 2014 USD.}\\


\begin{table}[H] 
\begin{threeparttable}
\caption{Alternative Cost-benefit Analyses Calculations}
\label{table:comparing}
\centering 
\footnotesize

\begin{tabular}{cllcc}
\toprule
Age & \mc{1}{c}{NPV Source} & Component & \citet{Kline-Walters_2016_QJE} & Authors' Method \\
& & & Method & \\
\midrule
\multirow{2}{*}{27} & \cite{Chetty_Friedman_etal_2010_HowDoesYour} & Earnings & 1.04 (s.e. 0.36) &  \\
& ABC/CARE-calculated & Earnings & 1.36 (s.e. 0.04) &  0.14 (s.e. 0.05)\\
\midrule
\multirow{2}{*}{34} & ABC/CARE-calculated & Earnings & 0.45 (s.e. 0.04) & 0.45 (s.e. 0.17) \\
& ABC/CARE-calculated & All & 0.88 (s.e. 0.04) &  0.88 (s.e. 0.34) \\
\midrule
\multirow{2}{*}{Life-cycle} &  ABC/CARE-calculated & Earnings & 1.58 (s.e. 0.07) & 1.58 (s.e. 0.60) \\
& ABC/CARE-calculated & All & 4.58 (s.e. 0.25) & 5.63 (s.e. 2.15) \\
\bottomrule
\end{tabular}

\begin{tablenotes}
\footnotesize
\item Note: This table displays displays benefit-cost ratios based on the methodology \citet{Kline-Walters_2016_QJE} and based on our own methodology. Age: age at which we stop calculating the net-present value. NPV Source: source where we obtain the net present value. Component: item used to compute net present value (all refers to the net present value of all the components). \citet{Kline-Walters_2016_QJE}: estimate based on these authors methodology. Author's method: estimates based on our methodology.
\end{tablenotes}
\end{threeparttable}
\end{table}

\noindent To analyze how our estimates compares to the method in \citet{Kline-Walters_2016_QJE}, we present a series of exercises in the third column of Table~\ref{table:comparing}. For comparison, the fourth column of Table~\ref{table:comparing} displays the analogous exercise using our own method.\\

\noindent In the first exercise, we use both the ``return to WPPSI'' and the net-present value of earnings at age 27 in  \citet{Chetty_Friedman_etal_2011_QJoE}. In the second exercise, we perform a similar exercise but we use our own estimate of the net-present value of earnings at age 27.\footnote{This allows us to compute our own ``return to WPPSI'' and impute it to the treatment-group individuals.} The reminder of exercises are similar, but (i) vary the age up to where we consider the net-present value of earnings; (ii) consider the value of all the components we analyze throughout the paper; or (iii) both.\\

\noindent Our methodology has three gains: (i) provides a more precise estimate of the net-present value (and the return to WPPSI) of the components, as our interpolation and extrapolation is based on matching our experimental sample to various non-experimental samples. We do not impute a net present value or a return of an \textit{ad hoc} experiment; (ii) we better quantify the effects of the analyzed experiment by considering the whole life-cycle; (iii) we better approximate the statistical uncertainty of our estimates by considering both the sampling error in the experimental and non-experimental samples and the forecasting error due to the interpolation and extrapolation.

%References
\singlespace
\bibliographystyle{chicago}
\bibliography{heckman}
\end{document}