%Input preamble
\documentclass[11pt]{article}

% colors
\usepackage[table]{xcolor}
\definecolor{maroon}{RGB}{153,0,18}
\definecolor{lime}{RGB}{190,213,88}
\definecolor{sand}{RGB}{217,202,179}
\definecolor{fire}{RGB}{144,50,61}
\definecolor{brick}{RGB}{94,11,21}
\definecolor{olive}{RGB}{117,109,84}
\definecolor{lavpink}{RGB}{172,123,132}
\definecolor{darkpurp}{RGB}{49,10,49}
\definecolor{salmon}{RGB}{204,90,113}
\definecolor{mauve}{RGB}{94,73,85}
\definecolor{greyblue}{RGB}{125,132,145}
\definecolor{greypurp}{RGB}{68,56,80}
\definecolor{brightpurp}{RGB}{96,20,255}

% packages (please add in alphabetical order)
\usepackage{adjustbox}
\usepackage{amsfonts}
\usepackage{amsmath}
\usepackage{amssymb}
\usepackage{array}
\usepackage{bm}
\usepackage{booktabs}
\usepackage{caption}
\usepackage{epstopdf}
\usepackage{float}
\usepackage[margin=1in]{geometry}
\usepackage{graphicx}
\usepackage[colorlinks=true, linkcolor=brightpurp, citecolor=brightpurp, urlcolor=salmon]{hyperref}
\usepackage{lipsum}
\usepackage{longtable}
\usepackage{mathtools}
\usepackage{multirow}
\usepackage{natbib}
\usepackage{rotating}
\usepackage{setspace}
\usepackage{subcaption}
%\usepackage{threeparttable}
\usepackage{threeparttablex}
\usepackage{xr}
\usepackage[printwatermark]{xwatermark}


\newcolumntype{L}[1]{>{\raggedright\let\newline\\\arraybackslash\hspace{0pt}}m{#1}}
\newcolumntype{C}[1]{>{\centering\let\newline\\\arraybackslash\hspace{0pt}}m{#1}}
\newcolumntype{R}[1]{>{\raggedleft\let\newline\\\arraybackslash\hspace{0pt}}m{#1}}

% commands
\newcommand{\mr}{\multirow}
\newcommand{\mc}{\multicolumn}

%Other parameters
\newcommand{\noutcomes}{95}
\newcommand{\treatsubsabc}{$75\%$}
\newcommand{\treatsubscarec}{$74\%$}
\newcommand{\treatsubscaref}{$63\%$}

%Counts
%Males
\newcommand{\positivem}{$79\%$}
\newcommand{\positivesm}{$37\%$}

%Females
\newcommand{\positivef}{$73\%$}
\newcommand{\positivesf}{$35\%$}

%Counts, control substitution
%Males
\newcommand{\positivecsnm}{$58\%$}
\newcommand{\positivescsnm}{$25\%$}

\newcommand{\positivecsam}{$74\%$}
\newcommand{\positivescsam}{$38\%$}

%Females
%% no alternative
\newcommand{\positivecsnf}{$83\%$}
\newcommand{\positivescsnf}{$46\%$}

%% alternative
\newcommand{\positivecsaf}{$73\%$}
\newcommand{\positivescsaf}{$23\%$}

%Pooled

%Effects
%Males

%Females
\newcommand{\hsgradf}{$7$}
\newcommand{\yearsedf}{$1.2$}



%Pooled

%CBA
%IRR
%Males
\newcommand{\irrm}{$15\%$}
\newcommand{\irrsem}{$13\%$}

%Females
\newcommand{\irrf}{$10\%$}
\newcommand{\irrsef}{$12\%$}

%Pooled
\newcommand{\irrp}{$13\%$}
\newcommand{\irrsep}{$11\%$}

%BC
%Males
\newcommand{\bcm}{$7.88$}
\newcommand{\bcsem}{$8.06$}

%Females
\newcommand{\bcf}{$2.30$}
\newcommand{\bcsef}{$1.56$}

%Pooled
\newcommand{\bcp}{$4.35$}
\newcommand{\bcsep}{$2.57$}

%NPV streams
%Pooled
\newcommand{\parincomenpvp}{$\$115,026$}

\newcommand*\leftright[2]{%
  \leavevmode
  \rlap{#1}%
  \hspace{0.5\linewidth}%
  #2}

\newcommand{\orth}{\ensuremath{\perp\!\!\!\perp}}%
\newcommand{\indep}{\orth}%
\newcommand{\notorth}{\ensuremath{\perp\!\!\!\!\!\!\diagup\!\!\!\!\!\!\perp}}%
\newcommand{\notindep}{\notorth}

\externaldocument{abc_treatmenteffects_appendix}
\pagenumbering{roman}

\begin{document}

\noindent Jorge - story line unclear still! (We will soon be booted out) \textbf{[JLG: That would be sad, but we simply seem not to be understanding each other, or maybe I'm just incompetent. But I keep explaining the same thing and you keep asking the same questions. Maybe these samples are much more limited than expected, even when we have done so much to them!]}\\

\noindent I am still looking for a clean story. At baseline, who is more disadvantaged boys or girls? In terms of:\\

\noindent {\bf [JLG: These are the numbers comparing all boys to all girls]}

\begin{itemize}
\item Education of mother
	\begin{itemize}
		\item Boys: 10.44 (s.e. .19)
		\item Girls: 10.38 (s.e. .21)
		\item Difference: 0.05 (s.e. .20)
\noindent \textbf{[JJH: The same.]} 
	\end{itemize}
\item Education of father
	\begin{itemize}
		\item Boys: 10.94 (s.e. 0.24)
		\item Girls: 10.92 (s.e. 0.16)
		\item Difference: 0.5 (s.e. .29)
\noindent \textbf{[JJH: The same.]} 
	\end{itemize}
\item Intact families 
	\begin{itemize}
		\item Boys: .28 (s.e. 0.05)
		\item Girls: .24 (s.e. 0.04)
		\item Difference: 0.04 (s.e. 0.07)
\noindent \textbf{[JJH: Slightly better.]} 
	\end{itemize}
\item Income + earnings of mother
	\begin{itemize}
		\item  Boys: 7,544 (s.e. 2,312) 
		\item  Girls: 5,158 (s.e. 1,531)
		\item Difference: 2,385 (s.e. 3,133) 
	\end{itemize}
\item Income + earnings of father
\noindent {\bf [JLG: not observed at baseline, observed later at some ages.] [JJH: Ok - so it's not baseline.] [JLG: Correct.]}
\end{itemize}

\noindent  {\bf [JLG: So, as we have discussed multiple times, boys seem to have advantage relative to girls if we take the differences in the levels of these variables seriously, but the differences are not statistically significant. That is why we use the index in Table 5. You keep saying in the text that the index in Table 5 is not at baseline, but as the note of the table clearly states, all of the measures are at baseline. You also say that the measure is arbitrary. That measure is a factor estimated from the following variables: Mother's age, mother's education, mother's IQ, mother's marital status, mother's employment, and number of siblings.] [JJH: But conceptually we want earnings at home including father at baseline.] [JLG: When we use the parental income aggregate in the paper (e.g., Tables 1 and 4) it is composed of what we have at that age. It includes mother, father, or parental income depending on the child's family situation. But recall that father income by itself is only observed at ages 12 and 21 and has very few observations.]}\\

\noindent {\bf [JLG: We estimate a factor using these variables to form the index, so the weight given to each of them is not arbitrary. Actually, the weights given in the HRI index are arbitrary given that it was set by the researchers at UNC.] [JJH: What do you mean by arbitrary?] [JLG: They constructed the index weighting the different components linearly according to some weights they literally made up.]}\\

\noindent {\bf [JLG: If you look at row [1], you will see that we reject the null of this index being the same across genders using the Rosenbaum test. The average level of this index is greater for boys than it is for girls, so we conclude that boys are more advantaged than girls.] [JJH: Do you form the factor using the same weights for boys and girls] [JLG: Yes, but we decided to do so because we failed to reject the null of equality.]}\\

\noindent {\bf [JLG: You also comment that ``everyone is disadvantaged in this sample!'' Of course, but we are obviously using the term disadvantaged as relatively disadvantaged within the experiment, not relative to the population of the US.]\\}

\noindent In the experiment, same question. What is the experimentally induced impact on family environment?

\begin{itemize} 
\item More fathers working? 
\item More mothers working? 
\item Family income? 
\end{itemize}

\noindent {\bf [JLG: An item-by-item breakout is going to leave to similar results as above and it could be arbitrary. But take a look at Table 4 and there are two rows relevant to your question.] [JJH: You misuse the word ``arbitrary.'' It's a legitimate question - more mothers work at baseline? Or after treatment. I thought the latter.] [JLG: Yes. It is the latter.] [JJH: Does it differ for girls.] [JLG: No. But we iterated this by email, and I think that we agreed on this. Also, in Section 5, I am answering a few questions related to this. Please also see that.] 

\begin{itemize} 
\item Parental Income: this is the set of variables containing mother's income, father's income (when we observe it), mother's and father's labor supply. As you see, the impact is REALLY large in terms of combining functions and the Rosenbaum tests indicate rejection. 
\item Parenting: this is basically the time series of the HOME scores from age 1 to age 8. Similarly, the effect of the experiment is large.
\end{itemize}

\noindent The sharp gender differences are there in the sense that in each of the outcome categories, girls benefit more from treatment than boys. Now, these are of course the treatment effects. But revisit the female-male gaps, pre- and post-treatment in Table 1 for each of the categories in Table 1. The differences are sharp. Most are significant!]}\\

\noindent I really do not see sharp differences in after or before effects. Before we were emphasizing greater vulnerability of boys:\\

\begin{itemize}
\item How do we know that this is not the story?
\item How to explain to referees?
\item Why now both? 
\item What is the strength of the evidence on either? 
\end{itemize}

\noindent {\bf [JLG: Yes, we were previously emphasizing greater vulnerability of boys. But we did not distill differences in baseline characteristics by gender to explain the differences in the treatment effects by gender to explain why the treatment effects differed depending on the counterfactual. Referee 3, the referee whose comments you praised, asked us to clarify that. That's why we explored the baseline differences in the control-group children who stayed at home and the control-group children who attended alternatives.\\ 

\noindent In Table 5, and please note that Table 5 is at baseline, we see sharp differences by gender by take-up of alternatives. \\

\noindent Within the control-group boys, the relatively more disadvantaged are sent more to alternatives. The relatively less disadvantaged are at home. Therefore, the treatment effect is greater for boys who went to alternatives - their counterparts who stayed at home were in better environments. Instead, these children were going to preschools of lower quality and were beginning the preschools at age 3 instead of at 8 weeks like ABC/CARE). \\

\noindent Now, for girls the opposite pattern appears, but not as strong. As a consequence, girls benefit more from treatment if compared to staying at home but the difference is not the same as in the case for boys.\\ 

\noindent Before, we explained this using quality. A careful inspection of the data implies that baseline characteristics could be an explanation as well. \\ 

\noindent Now, we discussed that it could very well be that both stories are there. But there's one big difference. Our information on quality is not good, while we have measures of disadvantage at baseline with which we can perform tests! So they both could be playing a role. As we discussed in person, it's going to be impossible to separate the stories in our sample or tell what's the strength for either.]}

\pagebreak
\doublespacing

\begin{center}
\noindent {\centering \textbf{Memo: June 13, 2018}}
\end{center}

\noindent \textbf{[JJH: JORGE and Anna: I am becoming deeply frustrated by this process as I am sure both of you are I cannot get a coherent story.]} \textbf{[JLG: We are providing answers to your questions day and night based on the data. If we were so frustrated, we wouldn't put so many hours into this. You seem insulted from the work that we do. You said two nights ago: ``SO WHY is an expanding and clarifying story an endless loop.'' Please give us some credit. We are grateful for your comments, but let's try to bounce ideas. Didn't you teach me that careful work takes time and pain? Our scientific responsibility is to tell you what the data looks like, not to make things easy for us and make up a story that you would like. Some would do so and we never will.]}\\

\noindent \textbf{[JJH: You say the treatment effect for family income is larger for BOYS.] [JLG: No. Baseline resources are greater but not the treatment effects.]}\\ 

\noindent \textbf{[JJH: Does this ``treatment effect'' include the higher baseline family income for boys?] [JLG: No. That would convolute pre- and post-treatment and none of the results in Table 1 or 4 do so. Why would they?]}\\

\noindent \textbf{[JJH: That's not a treatment effect.] [JLG: Thanks for the clarification, but it \textit{does not} include baseline income.} \\

\noindent \textbf{[JJH: As far as I can tell Boys start with higher income than girls (baseline).] [JLG: Correct.]}\\

\noindent \textbf{[JJH: Then the  mother is induced to work by program The INCREASE IN INCOME is greater for Girls than Boys.] [JLG: Yes.]}\\ 

\noindent \textbf{[JJH: That's the only logical story Boys start better Girls gain ground.] [JLG: Yes. That's why the gaps decrease.]}\\

\noindent \textbf{[JJH: The GROWTH in income is for girls.] [JLG: Yes.]}\\

\noindent \textbf{[JJH: The level of income may still be lower for treatment girls but they catch up.] [JLG: Yes.]}\\ 

\noindent \textbf{[JJH: The home scores go up for girls because of greater resources.] [JLG: Yes.]}\\

\noindent \textbf{[JJH: Fathers  are more likely to be present  for boys but the program does not induce in more fathers with boys or girls.] [JLG: Yes.]}\\

\noindent \textbf{[JJH: Fathers add income but fathers income does not increase.] [JLG: Yes.]}\\

\noindent \textbf{[JJH: I just keep getting part of the story from you.] [JLG: Is this so true? Or we are trying to understand the data and this is a difficult problem? Also, given the questions that you are bringing up, you seem to know the story.]}\\

\noindent \textbf{[JJH: I wont sign off unless we have a coherent story.] [JLG: We haven't asked you to sign off. We want to have a good study not a rotten publication. Francesconi may be in a rush. That's his deal, not ours.]}\\ 

\noindent \textbf{[JJH: As I said earlier the old draft from last month section 5 was frighteningly bad and has scared the hell out of me in regards to this paper.]} \textbf{[JLG: Is it constructive to bring this up now? As I said before, I am not sure why our work becomes so insulting to you. We worked and gave you a draft. You didn't like Section 5 and, as far as I am concerned, there's nothing bad about that. But we are coauthors and we can build a better Section 5 together. If the point is that we should take responsibility for writing a bad Section 5, then I am very happy to take it and move on. I thought writing a better Section 5 was part of the ``expanding and clarifying story,'' not that a section written months ago was going to disable us from moving on with the paper.]}\\

\noindent \textbf{[JJH: I feel I am fighting brick by brick house by house I still do not know if th treatment girlsget more incomefrm motherworking orfatherarriving do you know.]} \textbf{[JLG/AZ: ABC/CARE does not affect fathers presence. It does induce mothers to work. If the mothers are working and the father is present, then we do not know if the effect of that interaction given that we cannot distinguish father's income from mother's income. We do know that father's presence and family income are positively correlated. Please note that the treatment effects on parental income are actually larger for boys than for girls.]}\\

\noindent \textbf{[JJH: Table 4 is very inconclusive The description does not match the reality of the table  There is no Dominance of girls.]} \textbf{[JLG/AZ:  In Table 4, 7 out of 10 of the average effect sizes are larger for girls than for boys. In 9 out of 10 of the proportion of positive treatment effects, girls at least weakly dominate boys. In 8 out of 10 of the proportion of significant, positive treatment effects, girls at least weakly dominate boys. In addition, Table 1 shows that ABC/CARE reverses male-female gaps indicating that the treatment effects for girls dominate those for boys. This seems to indicate dominance of girls.]}\\

\noindent \textbf{[JJH: Figure 3 IS NOT at all clear and few differnces precisely determined I dropped 3.] [JLG: You requested that we brought that figure from the old draft the weekend before last because it ``told a story.'' I agree that this figure is not clear.]} \\

\noindent \textbf{[JJH: Its really unfortunate that we do not have POST TRatment Family enviornments in terms of income.]} \\\textbf{[JLG: Yes it is.]}\\

\noindent \textbf{[JJH: More mothers work  For gorls more motherswor but home score goes down  on net worse off.]}  \textbf{[JLG/AZ: Yes, treatment induces mothers to work. For girls, the home scores actually go up. This is seen in Table 4 under "Parenting," in which girls dominate boys. For example, the average effect size is 0.274 (and statistically significant) for girls, but 0.060 for boys. Recall that HOME scores include a subscale that negatively rates punishment, and the boys preform poorly in this specific subscale.]}\\

\noindent \textbf{[JJH: So its impossible to see if more fathers came home once child I nteatemt?]} \textbf{[JLG/AZ:It is possible to see if fathers are present. It is impossible to see their individual income. As we show in Tables D.57 and D.41 in the Appendix, treatment does not induce fathers to come back. For both boys and girls, the effect on father's presence after baseline is negative comparing to the control-group boys who stay at home. This is not consistent across ages though, indicating that fathers come and go in all subgroups. We verified in the data that indeed this is what is happening. Although we cannot formally disentangle the mediators of treatment, it seems that father's presence is not one of them.]}

























\end{document}