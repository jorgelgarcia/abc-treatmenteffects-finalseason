\begin{table}[H]
\small
\begin{threeparttable}
\caption{Internal Rate of Return, Benefit-Cost Ratio Comparisons}
\label{tab:irrbcr_compare}
\begin{tabular}{cccc|ccc}
\hline\hline
& \multicolumn{3}{c|}{Not Controlling for Imbalance} & \multicolumn{3}{c}{Controlling for Imbalance} \\
%\cmidrule(lr){2-4} \cmidrule(lr){5-7} 
Group & Estimate & Standard Error & Confidence Interval & Estimate & Standard Error & Confidence Interval \\
\hline
\multicolumn{4}{l|}{\emph{Internal Rate of Return}}\\
Male & 0.1310 & 0.0548 & (0.0417, 0.1882) & 0.1387 & 0.0564 & (0.0511, 0.1970) \\
Female & 0.0426 & 0.0906 & (-0.0230, 0.1009) & 0.0462 & 0.1041 & (-0.0302, 0.1139) \\
Pooled & 0.1057 & 0.0546 & (0.0512, 0.1535) &  0.1129 & 0.0564 & (0.0544, 0.1626) \\
& & & & & & \\
\multicolumn{4}{l|}{\emph{Benefit-Cost Ratio}}\\
Male & 5.1999 & 3.3813 & (1.0632, 9.5777) & 5.3787 & 3.3959 & (1.2778, 9.8837) \\
Female & 1.3600 & 0.9213 & (0.2086, 2.5500) & 1.3599 & 1.0493 & (0.0672, 2.7033) \\
Pooled & 3.1944 & 1.5526 & (1.3009, 5.2425) & 3.2740 & 1.5707 & (1.3467, 5.2856) \\
\hline\hline
\end{tabular}
\begin{tablenotes}
\footnotesize
\item Note: this table describes how the estimates of the internal rate of return and
benefit-cost ratios change when we account for the imbalances in baseline characteristics 
between treatment and control groups. Point estimates are means of the bootstrap distribution. 
Standard errors are the standard deviation of the bootstrap distribution. The 80\%
confidence interval are constructed by taking the 10\textsuperscript{th} and 
90\textsuperscript{th} percentiles of the bootstrap distributions. 
\end{tablenotes}
\end{threeparttable}
\end{table}
