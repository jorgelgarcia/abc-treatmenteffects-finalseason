\documentclass[11pt]{article}

% colors
\usepackage[table]{xcolor}
\definecolor{maroon}{RGB}{153,0,18}
\definecolor{lime}{RGB}{190,213,88}
\definecolor{sand}{RGB}{217,202,179}
\definecolor{fire}{RGB}{144,50,61}
\definecolor{brick}{RGB}{94,11,21}
\definecolor{olive}{RGB}{117,109,84}
\definecolor{lavpink}{RGB}{172,123,132}
\definecolor{darkpurp}{RGB}{49,10,49}
\definecolor{salmon}{RGB}{204,90,113}
\definecolor{mauve}{RGB}{94,73,85}
\definecolor{greyblue}{RGB}{125,132,145}
\definecolor{greypurp}{RGB}{68,56,80}
\definecolor{brightpurp}{RGB}{96,20,255}

% packages (please add in alphabetical order)
\usepackage{adjustbox}
\usepackage{amsfonts}
\usepackage{amsmath}
\usepackage{amssymb}
\usepackage{array}
\usepackage{bm}
\usepackage{booktabs}
\usepackage{caption}
\usepackage{epstopdf}
\usepackage{float}
\usepackage[margin=1in]{geometry}
\usepackage{graphicx}
\usepackage[colorlinks=true, linkcolor=brightpurp, citecolor=brightpurp, urlcolor=salmon]{hyperref}
\usepackage{lipsum}
\usepackage{longtable}
\usepackage{mathtools}
\usepackage{multirow}
\usepackage{natbib}
\usepackage{rotating}
\usepackage{setspace}
\usepackage{subcaption}
%\usepackage{threeparttable}
\usepackage{threeparttablex}
\usepackage{xr}
\usepackage[printwatermark]{xwatermark}


\newcolumntype{L}[1]{>{\raggedright\let\newline\\\arraybackslash\hspace{0pt}}m{#1}}
\newcolumntype{C}[1]{>{\centering\let\newline\\\arraybackslash\hspace{0pt}}m{#1}}
\newcolumntype{R}[1]{>{\raggedleft\let\newline\\\arraybackslash\hspace{0pt}}m{#1}}

% commands
\newcommand{\mr}{\multirow}
\newcommand{\mc}{\multicolumn}


\begin{document}

\doublespacing


\noindent (1) Structural approach:\\

\noindent (A) Find people in the ballpark.\\
\noindent Comment: correct, using Algorithm 1.\\

\noindent (B) Write out model/assumption.\\
\noindent  Comment: we assume a causal model for treatment $(d = 1)$ and control $(d = 0)$ outcomes for measure $j$ at age $a$ in sample $k \in \{ e, n \}$ where $e$ denotes membership in the experimental sample and $n$ denotes membership 

\begin{equation}
Y_{k,j,a}^d = \phi_{k,j,a}^d \left( \bm{X}_{k,a}^d, \bm{B}_{k} \right) + \varepsilon_{k,j,a}^d \ \ \ j \in \mathcal{J}_{a}.
\end{equation}Comment: 

\noindent Then impose (sufficient) assumptions leading to Theorem 1 (consistent predictions).\\

\noindent (C) Structural invariance role.\\
\noindent Comment: An invariant function across the experimental and auxiliary samples. This enables us to use the function $\phi_{k,j,a}^d \left( \cdot, \cdot \right)$ estimated in the auxiliary sample to predict in the experimental sample.\\

\noindent (D How do we use it to make out-of-sample comparisons for treatment and controls.\\
\noindent Comment: By assuming that $\phi_{e,j,a}^d \left( \cdot, \cdot \right) = \phi_{n,j,a}^d \left( \cdot, \cdot \right)$ so the predictions based on the functions estimates in the auxiliary sample are valid in the experimental sample.\\

\noindent (E) Don't need exogeneity - we can correct for it.\\ 
\noindent Comment: correct, although we can only do it at $a^*$ (as a test) due to lack of data. We only have the adequate measures in CNLSY, and not in NLSY79 and PSID. We do have the measures in the experimental group. We cannot reject that exogeneity holds.\\

\noindent (G) Also, how do we forecast future $ \bm{X}_{k,a}^d$ if it lags? And how do we correct the standard errors.\\
\noindent Comment: the only case where we include predicted components in $ \bm{X}_{k,a}^d$ is for lagged outcomes. This happens for labor, transfer, and parental income. We predict them in the step before estimating the function $\phi_{k,j,a}^d \left( \cdot, \cdot \right)$. These steps are taken into account when creating the bootstrap distribution, so sampling variation when estimating this. Put differently, the prediction procedure is iterated given that we need to predict the lagged outcomes for each function $\phi_{k,j,a}^d \left( \cdot, \cdot \right)$ that we estimate. We start this iteration by estimating $\phi_{k,j,a}^d \left( \cdot, \cdot \right)$ at age 22, given that we observe the outcomes at age 21.\\ 



\noindent (2) Matching:\\ 
\noindent (A) Same step as 1.\\
\noindent Comment: correct.\\

\noindent (B) Assume exogeneity andComment:  find counterparts; we don't need exogeneity in levels just in differences.\\
\noindent Comment: correct.\\

\noindent (C) Do we need structural invariance? (Not necessarily). But invariance as to levels.\\
\noindent Comment: I agree wit this; not in the parameterization.\\

\noindent (D) These are counterparts to future $\bm{X}$ question - how to match?\\ 
\noindent Comment: Possible to extend Algorithm 1 to match on all future $\bm{X}$. 

\noindent (3) Issues:\\

\noindent (A) Exogeneity? Only in differences. 
\noindent A = the exogeneity tests are not as satisfactory, as you well mentioned they could be based on the power of the null.\\

\end{document}