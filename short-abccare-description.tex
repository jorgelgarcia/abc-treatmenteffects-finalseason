\section[Background and Data Sources]{Background and Data Sources} \label{section:background}

ABC/CARE targeted disadvantaged, predominately African-American children in Chapel Hill/Durham, North Carolina.\footnote{Both ABC and CARE were designed and implemented by researchers at the Frank Porter Graham Center of the University of North Carolina in Chapel Hill.} Table~\ref{tab:programcomparison} compares the two virtually identical programs. Appendix~\ref{appendix:background} describes these programs in detail. Here, we summarize their main features.

The goal of these programs was to enhance the early-life skills of disadvantaged children. Both programs supported language, motor, and cognitive development as well as socio-emotional competencies considered crucial for school success including task-orientation, ability to communicate, independence, and pro-social behavior.\footnote{\citet{Ramey_Collier_etal_1976_CarolinaAbecedarianProject, Ramey_etal_1985_Project-CARE_TiECSE, Sparling_1974_Synth_Edu_Infant_SPEECH, Wasik_Ramey_etal_1990_CD, Ramey-etal_2012-ABC}.}

ABC recruited four cohorts of children born between 1972 and 1976. CARE recruited two cohorts of children, born between 1978 and 1980. For both programs, families of potential participants were referred to researchers by local social service agencies and hospitals at the beginning of the mother's last trimester of pregnancy. Eligibility was determined by a score on a childhood risk index.\footnote{See Appendix~\ref{appendix:background} for details on the construction for the index used. The index weighs the following variables (listed from the most to the least important according to the index): maternal and paternal education, family income, father's presence at home, lack of maternal relatives in the area, siblings behind appropriate grade in school, family in welfare, father in unstable job, maternal IQ, siblings' IQ, social agency indicates that the family is disadvantaged, one or more family members has sought a form of professional help in the last three years, and any other special circumstance detected by program's staff.}

The design and implementation of ABC and CARE were very similar. Both had two phases, the first of which lasted from birth until age 5. In this phase, children were randomly assigned to treatment. The second phase of the study consisted of child academic support through home visits from ages 5 through 8. The first phase of CARE from birth until age 5, had an additional treatment arm of home visits designed to improve home environments.\footnote{\citet{Wasik_Ramey_etal_1990_CD}.} Our analysis is based on the first phase and pools the CARE treatment group with the ABC treatment group. We do not use the data on the CARE group that only received home visits in the early years. 

For both programs, from birth until age 8, data were collected annually on cognitive and socio-emotional skills, home environments, family structure, and family economic characteristics. After age 8, data on cognitive and socio-emotional skills, education, and family economic characteristics were collected at ages 12, 15, 21, and 30.\footnote{At age 30, measures of cognitive skills are unavailable for both ABC and CARE.} In addition, we have access to administrative criminal records and a physician-administered medical survey at the mid 30's.\footnote{See Appendix~\ref{appendix:data} for a more comprehensive description of the data. There, we document the balance in observed baseline characteristics across the treatment and control groups, once we drop the individuals for whom we have no crime or health information, for which there is substantial attrition. Further, the methodology we propose addresses missing data in either of these two outcome categories.}

Randomization for ABC/CARE was conducted on child pairs matched on family background. Siblings and twins were jointly randomized into either treatment or control groups.\footnote{For siblings, this occurred when two siblings were close enough in age such that both of them were eligible for the program.} Randomization pairing was based on a risk index, maternal education, maternal age, and gender of the subject.\footnote{We do not know the original pairs.} Dropouts are evenly balanced and are primarily related to the health of the child and mobility of families and not to dissatisfaction with the program.\footnote{The 22 dropouts include four children who died, four children who left the study because their parents moved, and two children who were diagnosed as developmentally delayed. Details are in Table~\ref{table:abccompromises}. Everyone offered the program was randomized to either treatment or control. All eligible families agreed to participate. Dropping out occurs \emph{after} randomization.}

\treatsubsabc\ of the ABC control group and \treatsubscarec\ of the CARE control (but no children from families offered treatment) attended alternative (to home) childcare or preschool centers.\footnote{See \cite{Heckman_Hohmann_etal_2000_QJE} on the issue of substitution bias in social experiments.} Those who enroll generally stay enrolled. As control children age, they are more likely to enter childcare (see Appendix~\ref{app:control-subbb}). Children in the control group who are enrolled in alternative early childcare programs are less economically disadvantaged at baseline compared to children who stay at home. On average, they are children of mothers who are more likely to be working at baseline.\footnote{Statistically significant at 10\%.} Parents of girls are much more likely to use alternative childcare if assigned to the control group.\footnote{Most of the alternative childcare centers received federal subsidies and were subject to the federal regulations of the era (Appendix~\ref{appendix:tetanus}). See \citet{Department-of-Health_1968_DayCareRequirements,NCGA_1971_House-Bill-100,Ramey-et-al_1977_Intro-to-ABC,Ramey_Campbell_1979_SR,Ramey_McGinness_etal_1982_Abecedarianapproach, Burchinal_Campbell_etal_1997_CD}. They had relatively low quality compared to ABC/CARE.}

\begin{table}[!htbp]
\centering
\caption{ABC and CARE, Program Comparison} \label{tab:programcomparison}
\begin{adjustbox}{max width=\textwidth}
\begin{threeparttable}
	\small
	\begin{table}[H]
\begin{center}
\begin{threeparttable}
\caption{ABC and CARE, Programs Comparison} \label{tab:programcomparison}
\scriptsize
\scalebox{.9}{\begin{tabular}{L{4cm} L{7cm} L{5cm}}
\hline \hline
& \multicolumn{1}{c}{ABC}& \multicolumn{1}{c}{CARE}\\
\hline 
Program Overview &&\\
\hspace{.5cm} Years Implemented &1972--1982&1978--1985\\
\hspace{.5cm} Age of Entry/Exit & birth to 5 years old &\checkmark\\
\hspace{.5cm} Initial Sample &122&64\\
\hspace{.5cm} \# of Cohorts &4&2\\
\midrule
Eligibility & socio-economic disadvantage according to a multi-factor index (see Section \ref{section:eligibility})&\checkmark\\
 \midrule
Control &&\\
\hspace{.5cm} N &54&23\\
\hspace{.5cm} Compensation & Diapers from birth to age 3, unlimited formula from birth to 15 months & \checkmark \\
\hspace{.5cm} Treatment Substitution & 70\% & $\sim$ 70\%\\
\midrule
Treatment & Center-based childcare & Center-based childcare and family education\\
\hspace{.5cm} \textbf{Center-base} &&\\
\hspace{.5cm} \textbf{Childcare} &&\\
\hspace{.5cm} N &57&17\\
\hspace{.5cm} Intensity &6.5--9.75 hours a day for 50 weeks per year&\checkmark\\
\hspace{.5cm} Components & Instruction, medical care, nutrition, social services &\checkmark\\
\hspace{.5cm} Staff-to-child Ratio &1:3 during ages 0--1 &\checkmark\\
&1:4--5 during age 1--4 &\checkmark\\&1:5--6 during ages 4--5 &\checkmark\\
\hspace{.5cm} Staff Qualifications &Mixed diplomas; experienced&\checkmark\\
\hspace{.5cm} \textbf{Family Education} & Not part of the program &24\\
\hspace{.5cm} Intensity && One hour-long home visits. 2--3 per month during ages 0--3. 1--2 per month during ages 4--5\\
\hspace{.5cm} Curriculum & & Social and mental stimulation; parent-child interaction\\
\hspace{.5cm} Staff-to-child Ratio &&1:1\\
\hspace{.5cm} Staff Qualifications &&Home visitor training\\
\midrule
 School-age Treatment \\
 \hspace{.5cm} N&46&39\\
\hspace{.5cm} Intensity &Every other week& \checkmark\\
\hspace{.5cm} Components &Parent-teacher meetings& \checkmark\\
\hspace{.5cm} Curriculum & Reading and math &\checkmark\\
\hspace{.5cm} Staff-to-child Ratio &1:1&\checkmark\\
\hspace{.5cm} Staff Qualifications &Graduate degree and training in special education & \checkmark\\
\midrule
Data Availability \\
Questionnaires & Ages 0--5, 8, 12, 15, 21, 30--34 & Ages 0--5, 8, 12, 21, 30--34 \\
Parent Interview & Ages 0--5, 8, 12, 15, 21& Ages 0--5, 9, 12 \\  
Health Follow-up & Ages 30--34&\checkmark\\
\hline \hline
\end{tabular}}
\footnotesize
\begin{tablenotes}
\item Note: This table compares the main elements of ABC and CARE, summarized within this section.
\end{tablenotes}
\end{threeparttable}
\end{center}
\end{table}
\begin{tablenotes}
\small
\item Note: This table compares the main elements of ABC and CARE, summarized in this section. A \checkmark\ indicates that ABC and CARE had the same feature. A blank space indicates that the indicated component was not part of the program.\\
    $^*$ As documented in Appendix~\ref{app:eligibility-pop}, there were losses in the initial samples due to death, parental moving, and diagnoses of mental pathologies for the children.
\end{tablenotes}
\end{threeparttable}
\end{adjustbox}
\end{table}
