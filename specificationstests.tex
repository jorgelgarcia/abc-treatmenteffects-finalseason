
\documentclass{article}
\begin{document}

%\subsubsubsection{Specification Tests} \label{section:firstorder}

\noindent Our adaptation of the FAM model is based on a first-order Markov process assumption. To make this clear, take the example of heart disease. As we explain before and state in Table~\ref{}, we predict ``heart disease'' at age $a+1$ based on hypertension and diabetes at age $a$, as well as other risk factors and health behaviors (smoking, BMI, and physical activity).\footnote{The diseases that help predicting each other are based on research and advice of clinicians, as explained and justified in \citet{}.} Importantly, in this model, heart disease is a an absorbing state by assumption. That is, if an individual suffers heart disease at age $a$, her probability of having heart disease at age $a+1$ is identical to one.\\

\noindent In our empirical analysis, we estimate the transition probabilities for each disease using a Probit model, and the variables indicated in Table~\ref{} as well as background characteristics (no affected by treatment). We test the first-order Markov process assumption as follows using a likelihood test ratio comparing: (i) a model based on first-order lags (first-order Markov process) to predict the disease of interest (null) and a model based on second-order lags (second-order Markov process); (ii) an analogous test where to compare second- and third-order Markov processes; (iii) third- and fourth- order Markov processes; and (iv) fourth and fifth Markov processes. Table~\ref{table:lrtests} show the results from this test.\\

\begin{table}[H] 
\begin{threeparttable}
\caption{Likelihood Ratios for Tests Comparing Different Markov Orders for Disease Transition Specifications} \label{table:lrtests}
\centering 
\begin{tabular}{lcccc}
Test	                   & 1st- vs. 2nd-order Markov & 2nd- vs. 3rd-order Markov & 3rd- vs. 4th-order Markov & 4th- vs. 5th-order Markov & Degrees of Freedom \\ \midrule
Heart Disease    & 2.18  & 0.49 & 9.54   & 5.8    & 4.01 & 2 \\  
Hypertention      & 0.05  & 0.06 & 0.41    & 1.96 & 2.46 & 1 \\  
Stroke                & 3.94 & \textbf{12.74} & \textbf{16.86} & 6.27 & .66 & 4 \\ 
\end{tabular}
\begin{tablenotes}
\footnotesize
\item Note: This table presents likelihood ratios comparing different orders of Markov processes to predict diseases at age $a+1$, based on diseases and other health and demographic conditions at age $a$. We highlight the likelihood ratios for tests that are significant at the 10\%.
\end{tablenotes}
\end{threeparttable}
\end{table}

\noindent The interpretation of the results in Table~\ref{table:lrtests} is the following. The heart disease of Column ``1st- vs. 2nd-order Markov'' tests the null that 1st order lags suffice to construct the transition of heart disease from age $a$ to age $a+1$. We test the first-order Markov assumption with respect to other disease. Limited support on smoking, BMI, and physical activity does not allow us to test the first-order Markov assumption with respect to this variables.\footnote{We lack observations in the auxiliary sample for which information is available for this conditions and the diseases of interest for multiple lags.}We cannot reject that a 1st-order Markov model is enough to predict heart disease at age $a+1$, if compared to a second-order Markov model. The number of degrees of freedom for this test is $2$ because in the alternative model we add second-order lags for hypertension and diabetes (see Table~\ref{}). As we go across the columns, we find that higher order Markov processes do are not better to predict heart disease at age $a+1$, relative to a first-order Markov model. The results are similar for hypertension. For stroke, we find mixed evidence: 1st order Markov is a better model than second-order Markov, although third-order Markov and fourth-order Markov are better models than second- and third-order Markov models, respectively.\\ 

\noindent Unfortunately, limited support on the rest of the transitions models that we estimate does not allow us to make credible tests, as we would need to drop thousands of observations in the auxiliary samples and perform the tests in very selected samples.\\

\noindent An alternative (less parametric) test for the first-order Markov process is the following: (i) use a linear probability model and the variables in Table~\ref{} to predict the disease of interest (consider multiple versions with different numbers of lags); (ii) calculate the correlation of the residual of these predictions with higher order lags than what the prediction of interest considers. Table~\ref{table:1storderresids} shows these correlations. Although all of the correlations are significant at the 10\%, their magnitude is very low. The significance is due to the relatively large sample size in which these correlations are estimated, but no to an economically relevant magnitude of these statistics. 




\end{document}

