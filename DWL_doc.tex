\documentclass[11pt]{article}

% colors
\usepackage[table]{xcolor}
\definecolor{maroon}{RGB}{153,0,18}
\definecolor{lime}{RGB}{190,213,88}
\definecolor{sand}{RGB}{217,202,179}
\definecolor{fire}{RGB}{144,50,61}
\definecolor{brick}{RGB}{94,11,21}
\definecolor{olive}{RGB}{117,109,84}
\definecolor{lavpink}{RGB}{172,123,132}
\definecolor{darkpurp}{RGB}{49,10,49}
\definecolor{salmon}{RGB}{204,90,113}
\definecolor{mauve}{RGB}{94,73,85}
\definecolor{greyblue}{RGB}{125,132,145}
\definecolor{greypurp}{RGB}{68,56,80}
\definecolor{brightpurp}{RGB}{96,20,255}

% packages (please add in alphabetical order)
\usepackage{adjustbox}
\usepackage{amsfonts}
\usepackage{amsmath}
\usepackage{amssymb}
\usepackage{array}
\usepackage{bm}
\usepackage{booktabs}
\usepackage{caption}
\usepackage{epstopdf}
\usepackage{float}
\usepackage[margin=1in]{geometry}
\usepackage{graphicx}
\usepackage[colorlinks=true, linkcolor=brightpurp, citecolor=brightpurp, urlcolor=salmon]{hyperref}
\usepackage{lipsum}
\usepackage{longtable}
\usepackage{mathtools}
\usepackage{multirow}
\usepackage{natbib}
\usepackage{rotating}
\usepackage{setspace}
\usepackage{subcaption}
%\usepackage{threeparttable}
\usepackage{threeparttablex}
\usepackage{xr}
\usepackage[printwatermark]{xwatermark}


\newcolumntype{L}[1]{>{\raggedright\let\newline\\\arraybackslash\hspace{0pt}}m{#1}}
\newcolumntype{C}[1]{>{\centering\let\newline\\\arraybackslash\hspace{0pt}}m{#1}}
\newcolumntype{R}[1]{>{\raggedleft\let\newline\\\arraybackslash\hspace{0pt}}m{#1}}

% commands
\newcommand{\mr}{\multirow}
\newcommand{\mc}{\multicolumn}


\begin{document}

\doublespacing

This brief document explains the public costs detailed in \citet{Garcia_etal_2016_Comp_CBA_Unpublished}. Table~\ref{tab:dwl-componets} lists the components of the benefit/cost analysis that include public expenditure.

\begin{table}[htbp]
\centering
\begin{threeparttable}
\caption{Public Costs}\label{tab:dwl-componets}
\begin{tabular}{llc}
\toprule
Category & Detail & Transfer \\
\midrule
Preschool & Public subsidies for alternative preschool & \\
		& Payment for treatment at Frank Porter Graham & \\
Income & Transfer income & \checkmark \\
Education & Public education costs  &\\
& Public education costs of the mother &\\
Crime & Public costs of crime &\\
Health & Public costs of health &\\
&Disability insurance (DI) & \checkmark\\
&Social security (SS) & \checkmark\\
&Supplemental security income (SSI) & \checkmark\\
\bottomrule
\end{tabular}
\begin{tablenotes}
\raggedright
Note: The last column indicates whether the public cost is a transfer (\checkmark). A cost that is a transfer is multiplied by the marginal tax rate to consider the cost of DWL. A cost that is not a transfer is multiplied by the marginal tax rate plus one to consider the cost of DWL and the governmental cost of the expenditure.
\end{tablenotes}
\end{threeparttable}
\end{table}

The baseline estimates in \citet{Garcia_etal_2016_Comp_CBA_Unpublished} are calculated assuming that every dollar of public spending generates 50 cents of DWL. In this analysis we show the amount of government investment without DWL. We present the non-discounted value.

Tables~\ref{tab:dwl-npv-rslts2} through~\ref{tab:dwl-npv-rslts8} present the values of the different public components, per person of the government investment under a marginal tax rate of 0. 

\begin{table}[htbp]
\centering
\begin{adjustbox}{width=1.2\textwidth,center}
\begin{threeparttable}
\caption{Government Investment per Individual, Treatment vs. Next Best}\label{tab:dwl-npv-rslts2}
\begin{tabular}{lccccccccccccc}
\toprule
Sample	&	Treatment	&	\mc{1}{c}{Alternative }	&	Crime	&	\mc{1}{c}{Subject's}&	\mc{1}{c}{Mother's}	&	Health	&\mc{1}{c}{Transfer} & DI Claim & SSI Claim & SS Claim & Subtotal & DWL &	Total	\\
		& 		&	\mc{1}{c}{Preschool}			&			&	\mc{1}{c}{Education}		&	\mc{1}{c}{Education}	&		&  \mc{1}{c}{Income} &		&	&		&		&		& 	\\
\midrule
Female	&		-67,805		&		14,717		&		32,288		&		12,669		&		1,663		&		-40,310		&		10,441		&		-423		&		-1,195		&		-156		&	-38,110	&	-19,055	&	-57,165	\\
	&	$(	3,377	)$	&	$(	1,799	)$	&	$(	19,432	)$	&	$(	9,310	)$	&	$(	1,772	)$	&	$(	106,815	)$	&	$(	56,979	)$	&	$(	824	)$	&	$(	23,645	)$	&	$(	9,963	)$	&		&		&		\\
Male	&		-67,805		&		13,059		&		19,285		&		8,126		&		33		&		-68,555		&		-21,803		&		559		&		-28,376		&		6,955		&	-138,522	&	-69,261	&	-207,783	\\
	&	$(	3,377	)$	&	$(	1,589	)$	&	$(	36,224	)$	&	$(	6,585	)$	&	$(	2,033	)$	&	$(	183,345	)$	&	$(	21,887	)$	&	$(	1,901	)$	&	$(	45,085	)$	&	$(	14,852	)$	&		&		&		\\
Pooled	&		-67,805		&		13,581		&		26,955		&		6,312		&		1,155		&		-68,359		&		-9,041		&		26		&		-19,903		&		5,466		&	-111,614	&	-55,807	&	-167,421	\\
	&	$(	3,377	)$	&	$(	1,213	)$	&	$(	21,858	)$	&	$(	6,479	)$	&	$(	1,190	)$	&	$(	108,850	)$	&	$(	32,703	)$	&	$(	996	)$	&	$(	25,851	)$	&	$(	10,343	)$	&		&		&		\\
\bottomrule
\end{tabular}
\begin{tablenotes}
\raggedright
Note: This table shows the non-discounted value of government investment by component. The Crime and Health components only include public costs. The Subject's Education and Mother's Education components only include education costs up to age 18, after which costs are considered private. The DWL column presents just the DWL of public expenditures (including transfers), assuming a marginal tax rate of 0.5. The Total is the sum of all the previous columns. These estimates compare those in treatment with those in the next best option.
\end{tablenotes}
\end{threeparttable}
\end{adjustbox}
\end{table}


\begin{table}[htbp]
\centering
\footnotesize
\begin{adjustbox}{width=1.2\textwidth,center}
\begin{threeparttable}
\caption{Government Investment per Individual, Treatment vs. Stay at Home}\label{tab:dwl-npv-rslts5}
\begin{tabular}{lccccccccccccc}
\toprule
Sample	&	Treatment	&	\mc{1}{c}{Alternative }	&	Crime	&	\mc{1}{c}{Subject's}&	\mc{1}{c}{Mother's}	&	Health	&\mc{1}{c}{Transfer} & DI Claim & SSI Claim & SS Claim & Subtotal & DWL &	Total	\\
		& 		&	\mc{1}{c}{Preschool}			&			&	\mc{1}{c}{Education}		&	\mc{1}{c}{Education}	&		&  \mc{1}{c}{Income} &		&	&		&		&		& 	\\
\midrule
Female	&		-72,436		&		286		&		78,696		&		-20,173		&		3,914		&		13,192		&		4,038		&		867		&		-19,703		&		17,043		&	5,725	&	2,862	&	8,587	\\
	&	$(	4,060	)$	&	$(	122	)$	&	$(	62,421	)$	&	$(	6,339	)$	&	$(	2,489	)$	&	$(	156,345	)$	&	$(	68,235	)$	&	$(	1,445	)$	&	$(	34,889	)$	&	$(	18,863	)$	&		&		&		\\
Male	&		-72,436		&				&		1,899		&		1,239		&		-1,257		&		-152,816		&		-34,963		&		-168		&		-51,759		&		3,308		&	-306,954	&	-153,477	&	-460,431	\\
	&	$(	4,060	)$	&				&	$(	46,069	)$	&	$(	11,222	)$	&	$(	2,847	)$	&	$(	240,772	)$	&	$(	25,359	)$	&	$(	2,500	)$	&	$(	57,368	)$	&	$(	18,202	)$	&		&		&		\\
Pooled	&		-72,436		&		149		&		38,273		&		-9,495		&		1,744		&		-51,179		&		-26,458		&		593		&		-38,808		&		16,337		&	-141,278	&	-70,639	&	-211,917	\\
	&	$(	4,060	)$	&	$(	68	)$	&	$(	36,950	)$	&	$(	7,286	)$	&	$(	1,891	)$	&	$(	167,140	)$	&	$(	50,614	)$	&	$(	1,807	)$	&	$(	36,634	)$	&	$(	21,184	)$	&		&		&		\\
\bottomrule
\end{tabular}
\begin{tablenotes}
\raggedright
Note: This table shows the non-discounted value of government investment by component. The Crime and Health components only include public costs. The Subject's Education and Mother's Education components only include education costs up to age 18, after which costs are considered private. The DWL column presents just the DWL of public expenditures (including transfers), assuming a marginal tax rate of 0.5. The Total is the sum of all the previous columns. These estimates compare those in treatment with those who stayed at home.
\end{tablenotes}
\end{threeparttable}
\end{adjustbox}
\end{table}

\begin{table}[htbp]
\centering
\footnotesize
\begin{adjustbox}{width=1.2\textwidth,center}
\begin{threeparttable}
\caption{Government Investment per Individual, Treatment vs. Alternative Preschool}\label{tab:dwl-npv-rslts8}
\begin{tabular}{lccccccccccccc}
\toprule
Sample	&	Treatment	&	\mc{1}{c}{Alternative }	&	Crime	&	\mc{1}{c}{Subject's}&	\mc{1}{c}{Mother's}	&	Health	&\mc{1}{c}{Transfer} & DI Claim & SSI Claim & SS Claim & Subtotal & DWL &	Total	\\
		& 		&	\mc{1}{c}{Preschool}			&			&	\mc{1}{c}{Education}		&	\mc{1}{c}{Education}	&		&  \mc{1}{c}{Income} &		&	&		&		&		& 	\\
\midrule
Female	&		-36,716		&		13,568		&		3,757		&		7,640		&		1,088		&		-40,453		&		10,462		&		-193		&		-3,969		&		-251		&	-45,066	&	-22,533	&	-67,599	\\
	&	$(	1,938	)$	&	$(	1,358	)$	&	$(	7,131	)$	&	$(	4,377	)$	&	$(	887	)$	&	$(	53,608	)$	&	$(	24,884	)$	&	$(	404	)$	&	$(	12,269	)$	&	$(	5,445	)$	&		&		&		\\
Male	&		-36,716		&		13,320		&		12,038		&		2,603		&		-259		&		-34		&		-7,553		&		124		&		-7,857		&		1,675		&	-22,660	&	-11,330	&	-33,990	\\
	&	$(	1,938	)$	&	$(	1,418	)$	&	$(	17,508	)$	&	$(	3,511	)$	&	$(	1,181	)$	&	$(	84,613	)$	&	$(	10,017	)$	&	$(	849	)$	&	$(	21,221	)$	&	$(	6,199	)$	&		&		&		\\
Pooled	&		-36,716		&		13,593		&		3,897		&		3,924		&		418		&		-28,001		&		-1,444		&		-49		&		-5,825		&		2,114		&	-48,090	&	-24,045	&	-72,134	\\
	&	$(	1,938	)$	&	$(	1,020	)$	&	$(	10,401	)$	&	$(	3,087	)$	&	$(	803	)$	&	$(	54,992	)$	&	$(	16,572	)$	&	$(	487	)$	&	$(	13,881	)$	&	$(	4,920	)$	&		&		&		\\
\bottomrule
\end{tabular}
\begin{tablenotes}
\raggedright
Note: This table shows the non-discounted value of government investment by component. The Crime and Health components only include public costs. The Subject's Education and Mother's Education components only include education costs up to age 18, after which costs are considered private. The DWL column presents just the DWL of public expenditures (including transfers), assuming a marginal tax rate of 0.5. The Total is the sum of all the previous columns. These estimates compare those in treatment with those who attended alternative preschool.
\end{tablenotes}
\end{threeparttable}
\end{adjustbox}
\end{table}


\pagebreak
\singlespace
\bibliography{heckman}
\bibliographystyle{chicago}

\end{document}