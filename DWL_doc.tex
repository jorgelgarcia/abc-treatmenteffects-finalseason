\documentclass[11pt]{article}

% colors
\usepackage[table]{xcolor}
\definecolor{maroon}{RGB}{153,0,18}
\definecolor{lime}{RGB}{190,213,88}
\definecolor{sand}{RGB}{217,202,179}
\definecolor{fire}{RGB}{144,50,61}
\definecolor{brick}{RGB}{94,11,21}
\definecolor{olive}{RGB}{117,109,84}
\definecolor{lavpink}{RGB}{172,123,132}
\definecolor{darkpurp}{RGB}{49,10,49}
\definecolor{salmon}{RGB}{204,90,113}
\definecolor{mauve}{RGB}{94,73,85}
\definecolor{greyblue}{RGB}{125,132,145}
\definecolor{greypurp}{RGB}{68,56,80}
\definecolor{brightpurp}{RGB}{96,20,255}

% packages (please add in alphabetical order)
\usepackage{adjustbox}
\usepackage{amsfonts}
\usepackage{amsmath}
\usepackage{amssymb}
\usepackage{array}
\usepackage{bm}
\usepackage{booktabs}
\usepackage{caption}
\usepackage{epstopdf}
\usepackage{float}
\usepackage[margin=1in]{geometry}
\usepackage{graphicx}
\usepackage[colorlinks=true, linkcolor=brightpurp, citecolor=brightpurp, urlcolor=salmon]{hyperref}
\usepackage{lipsum}
\usepackage{longtable}
\usepackage{mathtools}
\usepackage{multirow}
\usepackage{natbib}
\usepackage{rotating}
\usepackage{setspace}
\usepackage{subcaption}
%\usepackage{threeparttable}
\usepackage{threeparttablex}
\usepackage{xr}
\usepackage[printwatermark]{xwatermark}


\newcolumntype{L}[1]{>{\raggedright\let\newline\\\arraybackslash\hspace{0pt}}m{#1}}
\newcolumntype{C}[1]{>{\centering\let\newline\\\arraybackslash\hspace{0pt}}m{#1}}
\newcolumntype{R}[1]{>{\raggedleft\let\newline\\\arraybackslash\hspace{0pt}}m{#1}}

% commands
\newcommand{\mr}{\multirow}
\newcommand{\mc}{\multicolumn}


\begin{document}

\doublespacing

This brief document explains how \citet{Garcia_etal_2016_Comp_CBA_Unpublished} calculate the net present value (NPV) of the deadweight loss (DWL) generated by ABC/CARE. Table~\ref{tab:dwl-componets} lists the components of the benefit/cost analysis that include public expenditure.

\begin{table}[htbp]
\centering
\caption{Public Costs}\label{tab:dwl-componets}
\begin{tabular}{ll}
\toprule
Category & Detail \\
\midrule
Alternative Preschool & Public subsidies \\
Income & Transfer income \\
Education & Public education costs  \\
& Public education costs of the mother \\
Crime & Public costs of crime \\
Health & Public costs of health \\
&Disability insurance payout \\
&Social security payout \\
&Social security income \\
\bottomrule
\end{tabular}
\end{table}

The baseline estimate in \citet{Garcia_etal_2016_Comp_CBA_Unpublished} is calculated assuming that every dollar of public spending generates 50 cents of DWL. The difference between the NPV under a marginal tax rate of 0.5 and the NPV under a marginal tax rate of 0 indirectly gives the cost of public expenditure under a marginal tax rate of 0.5. 

Table~\ref{tab:dwl-npv-rslts} presents the NPV per person of the DWL under a marginal tax rate of 0.5. This amount for the male subsample is much smaller than for the female subsample. This difference is due to the public costs of the domains for which the treatment effects are more substantial for men, such as crime. 

\begin{table}[htbp]
\centering
\caption{NPV per Person of the DWL from ABC/CARE}\label{tab:dwl-npv-rslts}
\begin{tabular}{l c }
\toprule
Sample & NPV of DWL \\
\midrule
Females 	& 	63,721		\\
		&  	(40,793)		\\ \\
Males 	&	488,858		\\
		& 	(261,197)		\\ \\
Pooled 	&	214,167  		\\
		&	(108,366) 		\\
\bottomrule
\end{tabular}
\end{table}

\pagebreak
\singlespace
\bibliography{heckman}
\bibliographystyle{chicago}

\end{document}