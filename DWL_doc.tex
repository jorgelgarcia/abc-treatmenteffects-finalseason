\documentclass[11pt]{article}

% colors
\usepackage[table]{xcolor}
\definecolor{maroon}{RGB}{153,0,18}
\definecolor{lime}{RGB}{190,213,88}
\definecolor{sand}{RGB}{217,202,179}
\definecolor{fire}{RGB}{144,50,61}
\definecolor{brick}{RGB}{94,11,21}
\definecolor{olive}{RGB}{117,109,84}
\definecolor{lavpink}{RGB}{172,123,132}
\definecolor{darkpurp}{RGB}{49,10,49}
\definecolor{salmon}{RGB}{204,90,113}
\definecolor{mauve}{RGB}{94,73,85}
\definecolor{greyblue}{RGB}{125,132,145}
\definecolor{greypurp}{RGB}{68,56,80}
\definecolor{brightpurp}{RGB}{96,20,255}

% packages (please add in alphabetical order)
\usepackage{adjustbox}
\usepackage{amsfonts}
\usepackage{amsmath}
\usepackage{amssymb}
\usepackage{array}
\usepackage{bm}
\usepackage{booktabs}
\usepackage{caption}
\usepackage{epstopdf}
\usepackage{float}
\usepackage[margin=1in]{geometry}
\usepackage{graphicx}
\usepackage[colorlinks=true, linkcolor=brightpurp, citecolor=brightpurp, urlcolor=salmon]{hyperref}
\usepackage{lipsum}
\usepackage{longtable}
\usepackage{mathtools}
\usepackage{multirow}
\usepackage{natbib}
\usepackage{rotating}
\usepackage{setspace}
\usepackage{subcaption}
%\usepackage{threeparttable}
\usepackage{threeparttablex}
\usepackage{xr}
\usepackage[printwatermark]{xwatermark}


\newcolumntype{L}[1]{>{\raggedright\let\newline\\\arraybackslash\hspace{0pt}}m{#1}}
\newcolumntype{C}[1]{>{\centering\let\newline\\\arraybackslash\hspace{0pt}}m{#1}}
\newcolumntype{R}[1]{>{\raggedleft\let\newline\\\arraybackslash\hspace{0pt}}m{#1}}

% commands
\newcommand{\mr}{\multirow}
\newcommand{\mc}{\multicolumn}


\begin{document}

\doublespacing

This brief document explains the public costs detailed in \citet{Garcia_etal_2016_Comp_CBA_Unpublished}. Table~\ref{tab:dwl-componets} lists the components of the benefit/cost analysis that include public expenditure.

\begin{table}[htbp]
\centering
\begin{threeparttable}
\caption{Public Costs}\label{tab:dwl-componets}
\begin{tabular}{llc}
\toprule
Category & Detail & Transfer \\
\midrule
Preschool & Public subsidies for alternative preschool & \\
		& Payment for treatment at Frank Porter Graham & \\
Income & Transfer income & \checkmark \\
Education & Public education costs  &\\
& Public education costs of the mother &\\
Crime & Public costs of crime &\\
Health & Public costs of health &\\
&Disability insurance payout & \checkmark\\
&Social security payout & \checkmark\\
&Social security income & \checkmark\\
\bottomrule
\end{tabular}
\begin{tablenotes}
\raggedright
Note: The last column indicates whether the public cost is a transfer (\checkmark). A cost that is a transfer is multiplied by the marginal tax rate to consider the cost of DWL. A cost that is not a transfer is multiplied by the marginal tax rate plus one to consider the cost of DWL and the governmental cost of the expenditure.
\end{tablenotes}
\end{threeparttable}
\end{table}

The baseline estimates in \citet{Garcia_etal_2016_Comp_CBA_Unpublished} are calculated assuming that every dollar of public spending generates 50 cents of DWL. In this analysis we show the amount of government investment without DWL. We present the non-discounted value.

Tables~\ref{tab:dwl-npv-rslts2} through~\ref{tab:dwl-npv-rslts5} present the values of the different public components, per person of the government investment under a marginal tax rate of 0. 

\begin{table}[htbp]
\centering
\footnotesize
\begin{threeparttable}
\caption{Government Investment per Individual, Treatment vs. Next Best}\label{tab:dwl-npv-rslts2}
\begin{tabular}{lccccccc}
\toprule
Sample	&	\mc{1}{c}{Alternative }	&	Treatment	&	Crime	&	\mc{1}{c}{Subject's}	&	\mc{1}{c}{Mother's}	&	Health	&	Total	Value\\
		& 	\mc{1}{c}{Preschool}		&			&			&	\mc{1}{c}{Education}		&	\mc{1}{c}{Education}	&			&	of ABC/CARE	\\
\midrule
Female	&	14,717	&	-67,805	&	14,508	&	9,064	&	1,033	&	-40,310	&	434,208	\\
		&	(1,799)	&	(3,377)	&	(7,636)	&	(6,562)	&	(1,101)	&	(106,815)	&	(342,657)	\\
Male		&	13,059	&	-67,805	&	10,025	&	6,219	&	-3		&	-68,555	&	3,076,272	\\
		&	(1,589)	&	(3,377)	&	(16,796)	&	(4,729)	&	(1,260)	&	(183,345)	&	(1,514,641)	\\
Pooled	&	13,581	&	-67,805	&	14,010	&	4,833	&	704		&	-68,359	&	1,352,969	\\
		&	(1,213)	&	(3,377)	&	(8,937)	&	(4,529)	&	(734)		&	(108,850)	&	(645,803)	\\
\bottomrule
\end{tabular}
\begin{tablenotes}
\raggedright
Note: This table shows the non-discounted value of government investment by component. The Crime and Health components only include public costs. The Subject's Education and Mother's Education components only include education costs up to age 18, after which costs are considered private. The Total Value of ABC/CARE is the total non-discounted value of ABC/CARE assuming a marginal tax rate of 0. This total includes private costs and transfers. These estimates compare those in treatment with those in the next best option.
\end{tablenotes}
\end{threeparttable}
\end{table}


\begin{table}[htbp]
\centering
\footnotesize
\begin{threeparttable}
\caption{Government Investment per Individual, Treatment vs. Next Best, no IPW}\label{tab:dwl-npv-rslts9}
\begin{tabular}{lccccccc}
\toprule
Sample	&	\mc{1}{c}{Alternative }	&	Treatment	&	Crime	&	\mc{1}{c}{Subject's}&	\mc{1}{c}{Mother's}	&	Health	&	Total	\\
		& 	\mc{1}{c}{Preschool}		&			&			&	\mc{1}{c}{Education}		&	\mc{1}{c}{Education}	&			&		\\
\midrule
Female	&	14,717	&	-67,805	&	14,508	&	9,064	&	1,033	&	-35,061	&	432,811	\\
		&	(1,799)	&	(3,377)	&	(7,636)	&	(6,562)	&	(1,101)	&	(112,446)	&	(328,459)	\\
Male		&	13,059	&	-67,805	&	13,341	&	6,219	&	-3		&	-67,802	&	2,759,132	\\
		&	(1,589)	&	(3,377)	&	(15,536)	&	(4,729)	&	(1,260)	&	(186,042)	&	(1,401,760)	\\
Pooled	&	13,581	&	-67,805	&	14,047	&	4,833	&	704		&	-63,985	&	1,359,551	\\
		&	(1,213)	&	(3,377)	&	(9,052)	&	(4,529)	&	(734)		&	(109,833)	&	(630,924)	\\
\bottomrule
\end{tabular}
\begin{tablenotes}
\raggedright
Note: This table shows the non-discounted value of government investment by component. The Crime and Health components only include public costs. The Subject's Education and Mother's Education components only include education costs up to age 18, after which costs are considered private. The Total Value of ABC/CARE is the total non-discounted value of ABC/CARE assuming a marginal tax rate of 0. This total includes private costs and transfers. These estimates compare those in treatment with those in the next best option, without IPW.
\end{tablenotes}
\end{threeparttable}
\end{table}

\begin{table}[htbp]
\centering
\footnotesize
\begin{threeparttable}
\caption{Government Investment per Individual, Treatment vs. Stay at Home}\label{tab:dwl-npv-rslts5}
\begin{tabular}{lccccccc}
\toprule
Sample	&	\mc{1}{c}{Alternative }	&	Treatment	&	Crime	&	\mc{1}{c}{Subject's}	&	\mc{1}{c}{Mother's}	&	Health	&	Total	\\
		& 	\mc{1}{c}{Preschool}		&			&			&	\mc{1}{c}{Education}		&	\mc{1}{c}{Education}	&			&		\\
\midrule
Female	&	286	&	-72,436	&	36,581	&	-13,844	&	2,368	&	13,192	&	1,297,493	\\
		&	(122)	&	(4,060)	&	(26,838)	&	(4,504)	&	(1,545)	&	(156,345)	&	(533,732)	\\
Male		&	.	&	-72,436	&	4,337	&	792		&	-874		&	-152,816	&	1,865,859	\\
		&	.	&	(4,060)	&	(21,877)	&	(8,687)	&	(1,740)	&	(240,772)	&	(1,296,416)	\\
Pooled	&	149	&	-72,436	&	19,663	&	-6,566	&	1,000	&	-51,179	&	1,490,507	\\
		&	(68)	&	(4,060)	&	(16,782)	&	(5,293)	&	(1,167)	&	(167,140)	&	(631,785)	\\
\bottomrule
\end{tabular}
\begin{tablenotes}
\raggedright
Note: This table shows the non-discounted value of government investment by component. The Crime and Health components only include public costs. The Subject's Education and Mother's Education components only include education costs up to age 18, after which costs are considered private. The Total Value of ABC/CARE is the total non-discounted value of ABC/CARE assuming a marginal tax rate of 0. This total includes private costs and transfers. These estimates compare those in treatment with those who stay at home.
\end{tablenotes}
\end{threeparttable}
\end{table}

\pagebreak
\singlespace
\bibliography{heckman}
\bibliographystyle{chicago}

\end{document}