%Input preamble
\documentclass[11pt]{article}

% colors
\usepackage[table]{xcolor}
\definecolor{maroon}{RGB}{153,0,18}
\definecolor{lime}{RGB}{190,213,88}
\definecolor{sand}{RGB}{217,202,179}
\definecolor{fire}{RGB}{144,50,61}
\definecolor{brick}{RGB}{94,11,21}
\definecolor{olive}{RGB}{117,109,84}
\definecolor{lavpink}{RGB}{172,123,132}
\definecolor{darkpurp}{RGB}{49,10,49}
\definecolor{salmon}{RGB}{204,90,113}
\definecolor{mauve}{RGB}{94,73,85}
\definecolor{greyblue}{RGB}{125,132,145}
\definecolor{greypurp}{RGB}{68,56,80}
\definecolor{brightpurp}{RGB}{96,20,255}

% packages (please add in alphabetical order)
\usepackage{adjustbox}
\usepackage{amsfonts}
\usepackage{amsmath}
\usepackage{amssymb}
\usepackage{array}
\usepackage{bm}
\usepackage{booktabs}
\usepackage{caption}
\usepackage{epstopdf}
\usepackage{float}
\usepackage[margin=1in]{geometry}
\usepackage{graphicx}
\usepackage[colorlinks=true, linkcolor=brightpurp, citecolor=brightpurp, urlcolor=salmon]{hyperref}
\usepackage{lipsum}
\usepackage{longtable}
\usepackage{mathtools}
\usepackage{multirow}
\usepackage{natbib}
\usepackage{rotating}
\usepackage{setspace}
\usepackage{subcaption}
%\usepackage{threeparttable}
\usepackage{threeparttablex}
\usepackage{xr}
\usepackage[printwatermark]{xwatermark}


\newcolumntype{L}[1]{>{\raggedright\let\newline\\\arraybackslash\hspace{0pt}}m{#1}}
\newcolumntype{C}[1]{>{\centering\let\newline\\\arraybackslash\hspace{0pt}}m{#1}}
\newcolumntype{R}[1]{>{\raggedleft\let\newline\\\arraybackslash\hspace{0pt}}m{#1}}

% commands
\newcommand{\mr}{\multirow}
\newcommand{\mc}{\multicolumn}


\externaldocument{abccaretreatmenteffects_report_main}

\begin{document}
\title{\Large \textbf{Appendix: \\ Analyzing the Short and Long-term Effects of Early Childhood Education on Multiple Dimensions of Human Development}}

\author{
Jorge Luis Garc\'{i}a\\
The University of Chicago \and
James J. Heckman \\
American Bar Foundation \\
The University of Chicago \and
Andr\'{e}s Hojman\\
The University of Chicago \and
Yu Kyung Koh \\ 
The University of Chicago \and
Joshua Shea \\
The University of Chicago \and
Anna Ziff \\ 
The University of Chicago}
\date{First Draft: April 16 5, 2016\\ This Draft: \today}
\maketitle

\singlespacing
\pagebreak
\tableofcontents
\listoffigures
\listoftables
\pagebreak

\section{Background}

\subsection{ABC}

\subsubsection{Overview}

\noindent The Carolina Abecedarian Project was a high-quality early childhood education program with randomized controlled design. It was implemented at the Frank Porter Graham of the University of North Carolina in Chapel Hill and served four cohorts of children between the years 1972 and 1977. In the main paper we motivate analyzing the program and provide the fundamentals of its description. In this section of the appendix, expand on some important details.

\subsubsection{Eligibility Criteria and Populations Served}

\noindent The mothers of the ABC participant children were recruited during the last trimester  of pregnancy, typically. Potential families were referred by local social service agencies and local hospitals. Eligibility was determined by a score of 11 or more on a weighted 13-factor High-Risk Index.\\ 

\noindent The HRI was based on 13 weighted variables, which are listed here with weights in parentheses: (i) maternal education level measured by years of education - ≤6 (8), 7 (7), 8 (6), 9 (3), 10 (2), 11 (1), ≥12 (0); (ii) paternal education level with weights identical to those for maternal education; (iii) family income measured in current dollars - ≤1,000 (8), 1,001-2000 (7), 2,001-3,000 (6), 3,001-4,000 (5), 4,001-5,000 (4), ≥ 5,001 (0); (iv) father’s absence from the household for reasons other than health or death (3); (v) absence of maternal relatives in the area (3); (vi) siblings of school age one or more grades behind age-appropriate level, or with equivalently low scores on school-administered achievement tests (3); (vii) received payments from welfare agencies within past 3 years (3); (viii) record of father's work indicates instability or unskilled and semi-skilled labor (3); (ix) record of maternal or paternal IQ score of 90 or below (3); (x) record of sibling IQ score of 90 or below (3); (xi) relevant social agencies in the community indicate the family is in need of assistance (3); (xii) one or more family members has sought counseling or professional help in the past 3 years (1); and (xiii) special circumstances not included in any of the above that are likely contributors to cultural or social disadvantage (1).\footnote{\citet{Ramey_Smith_1977_AJMD, Ramey_Campbell_1984_AJMD,Ramey_Campbell_1991_childreninpoverty,Ramey_Campbell_etal_2000_ADS}.} The weighting scale aimed to establish the relevant importance of each item in the index.\footnote{\citet{Ramey_Smith_1977_AJMD}.} Race was not  considered for eligibility; however, 98\% of the families who agreed to participate were African-American in addition to being low SES.\footnote{\citet{Ramey_Smith_1977_AJMD,Ramey_Campbell_1979_SR}.} \\

\noindent Figure~\ref{figure:hridistabc} displays the distribution of the HRI among all participants.  Mean maternal age when the target child was born was 19.9 years, and approximately half of the mothers of both treatment and control groups were aged 19 years or younger. Maternal age at target child's birth ranges from 13 to 43 years of age. One third of mothers were 17 or younger. Mean maternal IQ for both groups was approximately 85, one standard deviation below the national mean. Only 25\% of the ABC children lived with both biological parents, and more than 50\% lived with extended families in multi-generational households (61\% of treated children and 56\% of control children).\footnote{\citet{Ramey_Campbell_1991_childreninpoverty,Campbell_Ramey_1994_CD}.}

\subsubsection{Randomization Protocol and Compromises}

\subsubsection{Treatment Substitution}

\section{Results}

\subsection{CARE}

\noindent \textbf{[JLG: Anna and Yu Kyun working in this subsection of CARE, as part of the objective of their April 2016 trip]}.

%References
\renewcommand{\refname}{Appendix References}
\clearpage
\singlespace
\bibliographystyle{chicago}
\bibliography{heckman}

\end{document} 