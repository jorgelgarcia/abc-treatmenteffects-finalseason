%Input preamble
\documentclass[11pt]{article}

% colors
\usepackage[table]{xcolor}
\definecolor{maroon}{RGB}{153,0,18}
\definecolor{lime}{RGB}{190,213,88}
\definecolor{sand}{RGB}{217,202,179}
\definecolor{fire}{RGB}{144,50,61}
\definecolor{brick}{RGB}{94,11,21}
\definecolor{olive}{RGB}{117,109,84}
\definecolor{lavpink}{RGB}{172,123,132}
\definecolor{darkpurp}{RGB}{49,10,49}
\definecolor{salmon}{RGB}{204,90,113}
\definecolor{mauve}{RGB}{94,73,85}
\definecolor{greyblue}{RGB}{125,132,145}
\definecolor{greypurp}{RGB}{68,56,80}
\definecolor{brightpurp}{RGB}{96,20,255}

% packages (please add in alphabetical order)
\usepackage{adjustbox}
\usepackage{amsfonts}
\usepackage{amsmath}
\usepackage{amssymb}
\usepackage{array}
\usepackage{bm}
\usepackage{booktabs}
\usepackage{caption}
\usepackage{epstopdf}
\usepackage{float}
\usepackage[margin=1in]{geometry}
\usepackage{graphicx}
\usepackage[colorlinks=true, linkcolor=brightpurp, citecolor=brightpurp, urlcolor=salmon]{hyperref}
\usepackage{lipsum}
\usepackage{longtable}
\usepackage{mathtools}
\usepackage{multirow}
\usepackage{natbib}
\usepackage{rotating}
\usepackage{setspace}
\usepackage{subcaption}
%\usepackage{threeparttable}
\usepackage{threeparttablex}
\usepackage{xr}
\usepackage[printwatermark]{xwatermark}


\newcolumntype{L}[1]{>{\raggedright\let\newline\\\arraybackslash\hspace{0pt}}m{#1}}
\newcolumntype{C}[1]{>{\centering\let\newline\\\arraybackslash\hspace{0pt}}m{#1}}
\newcolumntype{R}[1]{>{\raggedleft\let\newline\\\arraybackslash\hspace{0pt}}m{#1}}

% commands
\newcommand{\mr}{\multirow}
\newcommand{\mc}{\multicolumn}


\pagenumbering{roman}

\begin{document}
\title{\Large \textbf{Results Appendix: \\ Analyzing the Short- and Long-term Effects of Early Childhood Education on Multiple Dimensions of Human Development}}

\author{
Jorge Luis Garc\'{i}a\\
The University of Chicago \and
James J. Heckman \\
American Bar Foundation \\
The University of Chicago \and
Andr\'{e}s Hojman\\
The University of Chicago \and
Duncan Ermini Leaf \\ 
University of Southern California \and
Mar\'{i}a Jos\'{e} Prados \\
University of Southern California \and
Joshua Shea \\
The University of Chicago \and 
Jake C. Torcasso \\
The University of Chicago}
\date{First Draft: January 5, 2016\\ This Draft: \today}
\maketitle
\thispagestyle{empty}

%\pagebreak
\tableofcontents
%\listoffigures
\listoftables
\pagebreak
\doublespacing
\pagenumbering{arabic}

\noindent In this appendix we present the treatment effects estimates of the center-based treatment in ABC and CARE. For each set of estimates, we first present a summary of the effect of the program(s) using a combining function counting the number of socially positive treatment effects. We then present tables of treatment effect estimates for each outcome. Finally, we present the treatment effect estimates using the step-down procedure to test multiple hypotheses. \\

\noindent We analyze 95 outcomes when combining the ABC and CARE samples. Table~\ref{table:abccare_rslt_pooled_counts} shows that across all methods of estimation, pooling males and females, over 70\% of the treatment effect estimates are socially positive. When conditioning on 10\% statistical significance, almost 40\% of all estimates are socially positive. These statistics allow us to reject the hypothesis that the treatment and control groups have the same marginal distributions in outcomes. \\

\noindent For both males and females, we find positive effects in IQ test scores, achievement test scores, as well as educational attainment. Males also enjoy additional benefits in the areas of employment, labor earnings, and hypertension. \\

\noindent In each of the tables for combining functions and treatment effect estimates, we present 8 different estimates. Column (1) corresponds to the mean difference between the groups randomly assigned to receive center-based childcare and the groups randomly assigned not to. Column (2) adjusts the estimates in (1) for attrition and controls for a set of covariates. Column (3) corresponds to the mean difference between the groups randomly assigned to receive center-based childcare and the groups randomly assigned not to, restricting the latter to subjects who did not receive preschool alternatives. Column (4) adjusts the estimates in (3) for attrition and controls for a set of covariates. Column (5) corresponds to the mean difference between the groups randomly assigned to receive center-based childcare and the groups randomly assigned not to, placing a relatively high weight on the subjects who are likely not to be enrolled in alternative preschools. Column (6) corresponds to the mean difference between the groups randomly assigned to receive center-based childcare and the groups randomly assigned not to, restricting the latter to subjects who received preschool alternatives. Column (7) adjusts the estimates in (6) for attrition and controls for a set of covariates. Column (8) corresponds to the mean difference between the groups randomly assigned to receive center-based childcare and the groups randomly assigned not to, placing a relatively high weight on the children who are likely to be enrolled in alternative preschools. The results in bold are significant at the 10\% level in a single-sided, non-parametric, bootstrapped test. \\

\def\arraystretch{0.6}

\setlength\tabcolsep{0.3em}

\section{Combining Functions, Aggregated}


	\begin{table}[H]
     \caption{Combining Functions, Pooled Sample} 
     \label{table:abccare_rslt_pooled_counts}
	
  \begin{tabular}{ccccccccc}
  \toprule
     & \scriptsize{(1)} & \scriptsize{(2)} & \scriptsize{(3)} & \scriptsize{(4)} & \scriptsize{(5)} & \scriptsize{(6)} & \scriptsize{(7)} & \scriptsize{(8)} \\ 
    \midrule

    \mc{1}{l}{\scriptsize{\% Pos. TE}} & \mc{1}{c}{\scriptsize{60}} & \mc{1}{c}{\scriptsize{62}} & \mc{1}{c}{\scriptsize{60}} & \mc{1}{c}{\scriptsize{65}} & \mc{1}{c}{\scriptsize{63}} & \mc{1}{c}{\scriptsize{59}} & \mc{1}{c}{\scriptsize{59}} & \mc{1}{c}{\scriptsize{61}} \\  

     & \mc{1}{c}{\scriptsize{\textbf{(0.013)}}} & \mc{1}{c}{\scriptsize{\textbf{(0.000)}}} & \mc{1}{c}{\scriptsize{\textbf{(0.013)}}} & \mc{1}{c}{\scriptsize{\textbf{(0.000)}}} & \mc{1}{c}{\scriptsize{\textbf{(0.000)}}} & \mc{1}{c}{\scriptsize{\textbf{(0.013)}}} & \mc{1}{c}{\scriptsize{\textbf{(0.013)}}} & \mc{1}{c}{\scriptsize{\textbf{(0.013)}}} \\  

    \mc{1}{l}{\scriptsize{\% Pos. TE $|$ 10\% Significance}} & \mc{1}{c}{\scriptsize{28}} & \mc{1}{c}{\scriptsize{23}} & \mc{1}{c}{\scriptsize{19}} & \mc{1}{c}{\scriptsize{16}} & \mc{1}{c}{\scriptsize{21}} & \mc{1}{c}{\scriptsize{24}} & \mc{1}{c}{\scriptsize{21}} & \mc{1}{c}{\scriptsize{27}} \\  

     & \mc{1}{c}{\scriptsize{\textbf{(0.000)}}} & \mc{1}{c}{\scriptsize{\textbf{(0.000)}}} & \mc{1}{c}{\scriptsize{\textbf{(0.026)}}} & \mc{1}{c}{\scriptsize{\textbf{(0.092)}}} & \mc{1}{c}{\scriptsize{\textbf{(0.026)}}} & \mc{1}{c}{\scriptsize{\textbf{(0.000)}}} & \mc{1}{c}{\scriptsize{\textbf{(0.000)}}} & \mc{1}{c}{\scriptsize{\textbf{(0.000)}}} \\  

  \bottomrule
  \end{tabular}

	\end{table}  

	\begin{table}[H]
     \caption{Combining Functions, Male Sample} 
     \label{table:abccare_rslt_male_counts}
	
  \begin{tabular}{ccccccccc}
  \toprule
     & \scriptsize{(1)} & \scriptsize{(2)} & \scriptsize{(3)} & \scriptsize{(4)} & \scriptsize{(5)} & \scriptsize{(6)} & \scriptsize{(7)} & \scriptsize{(8)} \\ 
    \midrule

    \mc{1}{l}{\scriptsize{\% Pos. TE}} & \mc{1}{c}{\scriptsize{52}} & \mc{1}{c}{\scriptsize{54}} & \mc{1}{c}{\scriptsize{39}} & \mc{1}{c}{\scriptsize{46}} & \mc{1}{c}{\scriptsize{40}} & \mc{1}{c}{\scriptsize{55}} & \mc{1}{c}{\scriptsize{57}} & \mc{1}{c}{\scriptsize{56}} \\  

     & \mc{1}{c}{\scriptsize{(0.289)}} & \mc{1}{c}{\scriptsize{(0.184)}} & \mc{1}{c}{\scriptsize{(0.987)}} & \mc{1}{c}{\scriptsize{(0.671)}} & \mc{1}{c}{\scriptsize{(0.961)}} & \mc{1}{c}{\scriptsize{(0.158)}} & \mc{1}{c}{\scriptsize{\textbf{(0.079)}}} & \mc{1}{c}{\scriptsize{(0.171)}} \\  

    \mc{1}{l}{\scriptsize{\% Pos. TE $|$ 10\% Significance}} & \mc{1}{c}{\scriptsize{15}} & \mc{1}{c}{\scriptsize{16}} & \mc{1}{c}{\scriptsize{9}} & \mc{1}{c}{\scriptsize{8}} & \mc{1}{c}{\scriptsize{8}} & \mc{1}{c}{\scriptsize{17}} & \mc{1}{c}{\scriptsize{14}} & \mc{1}{c}{\scriptsize{16}} \\  

     & \mc{1}{c}{\scriptsize{(0.132)}} & \mc{1}{c}{\scriptsize{\textbf{(0.053)}}} & \mc{1}{c}{\scriptsize{(0.539)}} & \mc{1}{c}{\scriptsize{(0.618)}} & \mc{1}{c}{\scriptsize{(0.724)}} & \mc{1}{c}{\scriptsize{\textbf{(0.026)}}} & \mc{1}{c}{\scriptsize{(0.118)}} & \mc{1}{c}{\scriptsize{\textbf{(0.053)}}} \\  

  \bottomrule
  \end{tabular}
	\end{table}  

	\begin{table}[H]
     \caption{Combining Functions, Female Sample} 
     \label{table:abccare_rslt_female_counts}
	
  \begin{tabular}{ccccccccc}
  \toprule
     & \scriptsize{(1)} & \scriptsize{(2)} & \scriptsize{(3)} & \scriptsize{(4)} & \scriptsize{(5)} & \scriptsize{(6)} & \scriptsize{(7)} & \scriptsize{(8)} \\ 
    \midrule

     & \scriptsize{(1)} & \scriptsize{(2)} & \scriptsize{(3)} & \scriptsize{(4)} & \scriptsize{(5)} & \scriptsize{(6)} & \scriptsize{(7)} & \scriptsize{(8)} \\ 
    \hline  

    \mc{1}{l}{\scriptsize{\% Pos. TE}} & \mc{1}{c}{\scriptsize{67}} & \mc{1}{c}{\scriptsize{68}} & \mc{1}{c}{\scriptsize{70}} & \mc{1}{c}{\scriptsize{72}} & \mc{1}{c}{\scriptsize{71}} & \mc{1}{c}{\scriptsize{62}} & \mc{1}{c}{\scriptsize{61}} & \mc{1}{c}{\scriptsize{63}} \\  

     & \mc{1}{c}{\scriptsize{\textbf{(0.000)}}} & \mc{1}{c}{\scriptsize{\textbf{(0.000)}}} & \mc{1}{c}{\scriptsize{\textbf{(0.000)}}} & \mc{1}{c}{\scriptsize{\textbf{(0.000)}}} & \mc{1}{c}{\scriptsize{\textbf{(0.000)}}} & \mc{1}{c}{\scriptsize{\textbf{(0.013)}}} & \mc{1}{c}{\scriptsize{\textbf{(0.000)}}} & \mc{1}{c}{\scriptsize{\textbf{(0.000)}}} \\  

    \mc{1}{l}{\scriptsize{\% Pos. TE $|$ 10\% Significance}} & \mc{1}{c}{\scriptsize{31}} & \mc{1}{c}{\scriptsize{25}} & \mc{1}{c}{\scriptsize{31}} & \mc{1}{c}{\scriptsize{25}} & \mc{1}{c}{\scriptsize{33}} & \mc{1}{c}{\scriptsize{23}} & \mc{1}{c}{\scriptsize{18}} & \mc{1}{c}{\scriptsize{26}} \\  

     & \mc{1}{c}{\scriptsize{\textbf{(0.000)}}} & \mc{1}{c}{\scriptsize{\textbf{(0.000)}}} & \mc{1}{c}{\scriptsize{\textbf{(0.000)}}} & \mc{1}{c}{\scriptsize{\textbf{(0.000)}}} & \mc{1}{c}{\scriptsize{\textbf{(0.000)}}} & \mc{1}{c}{\scriptsize{\textbf{(0.000)}}} & \mc{1}{c}{\scriptsize{\textbf{(0.053)}}} & \mc{1}{c}{\scriptsize{\textbf{(0.000)}}} \\  

  \bottomrule
  \end{tabular}
	\end{table}  
\clearpage

\section{Combining Functions, by Category}


	\begin{table}[H]
     \caption{Combining Functions by Category, Pooled Sample} 
     \label{table:abccare_rslt_pooled_counts_n50a100}
	\input{AppResOutput/abccare/rslt_pooled_counts_n50a100_all}
	\end{table}   

	\begin{table}[H]
     \caption{Combining Functions by Category $|$ 10\% Significance, Pooled Sample} 
     \label{table:abccare_rslt_pooled_counts_n10a10}
	\input{AppResOutput/abccare/rslt_pooled_counts_n10a10_all}
	\end{table}   

	\begin{table}[H]
     \caption{Combining Functions by Category, Male Sample} 
     \label{table:abccare_rslt_male_counts_n50a100}
	  \begin{tabular}{cccccccccc}
  \toprule

    \scriptsize{Category} & \scriptsize{(1)} & \scriptsize{(2)} & \scriptsize{(3)} & \scriptsize{(4)} & \scriptsize{(5)} & \scriptsize{(6)} & \scriptsize{(7)} & \scriptsize{(8)} &  \\ 
    \midrule  

    \mc{1}{l}{\scriptsize{Cognitive Skills}} & \mc{1}{c}{\scriptsize{96}} & \mc{1}{c}{\scriptsize{84}} & \mc{1}{c}{\scriptsize{64}} & \mc{1}{c}{\scriptsize{60}} & \mc{1}{c}{\scriptsize{52}} & \mc{1}{c}{\scriptsize{96}} & \mc{1}{c}{\scriptsize{84}} & \mc{1}{c}{\scriptsize{84}} & \mc{1}{c}{\scriptsize{25}} \\  

     & \mc{1}{c}{\scriptsize{\textbf{(0.000)}}} & \mc{1}{c}{\scriptsize{\textbf{(0.000)}}} & \mc{1}{c}{\scriptsize{(0.303)}} & \mc{1}{c}{\scriptsize{(0.382)}} & \mc{1}{c}{\scriptsize{(0.474)}} & \mc{1}{c}{\scriptsize{\textbf{(0.000)}}} & \mc{1}{c}{\scriptsize{\textbf{(0.000)}}} & \mc{1}{c}{\scriptsize{\textbf{(0.000)}}} &  \\  

    \mc{1}{l}{\scriptsize{Noncognitive Skills}} & \mc{1}{c}{\scriptsize{42}} & \mc{1}{c}{\scriptsize{45}} & \mc{1}{c}{\scriptsize{28}} & \mc{1}{c}{\scriptsize{35}} & \mc{1}{c}{\scriptsize{34}} & \mc{1}{c}{\scriptsize{48}} & \mc{1}{c}{\scriptsize{50}} & \mc{1}{c}{\scriptsize{50}} & \mc{1}{c}{\scriptsize{117}} \\  

     & \mc{1}{c}{\scriptsize{(0.908)}} & \mc{1}{c}{\scriptsize{(0.645)}} & \mc{1}{c}{\scriptsize{(1.000)}} & \mc{1}{c}{\scriptsize{(0.934)}} & \mc{1}{c}{\scriptsize{(0.987)}} & \mc{1}{c}{\scriptsize{(0.579)}} & \mc{1}{c}{\scriptsize{(0.461)}} & \mc{1}{c}{\scriptsize{(0.539)}} &  \\  

    \mc{1}{l}{\scriptsize{Mother's Employment, Education, and Income}} & \mc{1}{c}{\scriptsize{50}} & \mc{1}{c}{\scriptsize{50}} & \mc{1}{c}{\scriptsize{50}} & \mc{1}{c}{\scriptsize{100}} & \mc{1}{c}{\scriptsize{50}} & \mc{1}{c}{\scriptsize{25}} & \mc{1}{c}{\scriptsize{50}} & \mc{1}{c}{\scriptsize{50}} & \mc{1}{c}{\scriptsize{4}} \\  

     & \mc{1}{c}{\scriptsize{(0.513)}} & \mc{1}{c}{\scriptsize{(0.658)}} & \mc{1}{c}{\scriptsize{(0.368)}} & \mc{1}{c}{\scriptsize{\textbf{(0.000)}}} & \mc{1}{c}{\scriptsize{(0.368)}} & \mc{1}{c}{\scriptsize{(0.961)}} & \mc{1}{c}{\scriptsize{(0.605)}} & \mc{1}{c}{\scriptsize{(0.632)}} &  \\  

    \mc{1}{l}{\scriptsize{Childhood Household Environment}} & \mc{1}{c}{\scriptsize{60}} & \mc{1}{c}{\scriptsize{87}} & \mc{1}{c}{\scriptsize{47}} & \mc{1}{c}{\scriptsize{53}} & \mc{1}{c}{\scriptsize{47}} & \mc{1}{c}{\scriptsize{71}} & \mc{1}{c}{\scriptsize{87}} & \mc{1}{c}{\scriptsize{80}} & \mc{1}{c}{\scriptsize{15}} \\  

     & \mc{1}{c}{\scriptsize{(0.329)}} & \mc{1}{c}{\scriptsize{\textbf{(0.000)}}} & \mc{1}{c}{\scriptsize{(0.605)}} & \mc{1}{c}{\scriptsize{(0.421)}} & \mc{1}{c}{\scriptsize{(0.579)}} & \mc{1}{c}{\scriptsize{(0.184)}} & \mc{1}{c}{\scriptsize{\textbf{(0.000)}}} & \mc{1}{c}{\scriptsize{\textbf{(0.039)}}} &  \\  

    \mc{1}{l}{\scriptsize{Adult Household Environment}} & \mc{1}{c}{\scriptsize{56}} & \mc{1}{c}{\scriptsize{44}} & \mc{1}{c}{\scriptsize{38}} & \mc{1}{c}{\scriptsize{33}} & \mc{1}{c}{\scriptsize{22}} & \mc{1}{c}{\scriptsize{56}} & \mc{1}{c}{\scriptsize{56}} & \mc{1}{c}{\scriptsize{44}} & \mc{1}{c}{\scriptsize{9}} \\  

     & \mc{1}{c}{\scriptsize{(0.395)}} & \mc{1}{c}{\scriptsize{(0.724)}} & \mc{1}{c}{\scriptsize{(0.618)}} & \mc{1}{c}{\scriptsize{(0.803)}} & \mc{1}{c}{\scriptsize{(1.000)}} & \mc{1}{c}{\scriptsize{(0.276)}} & \mc{1}{c}{\scriptsize{(0.237)}} & \mc{1}{c}{\scriptsize{(0.592)}} &  \\  

    \mc{1}{l}{\scriptsize{Education, Employment, Income}} & \mc{1}{c}{\scriptsize{32}} & \mc{1}{c}{\scriptsize{36}} & \mc{1}{c}{\scriptsize{32}} & \mc{1}{c}{\scriptsize{36}} & \mc{1}{c}{\scriptsize{36}} & \mc{1}{c}{\scriptsize{43}} & \mc{1}{c}{\scriptsize{32}} & \mc{1}{c}{\scriptsize{36}} & \mc{1}{c}{\scriptsize{28}} \\  

     & \mc{1}{c}{\scriptsize{(0.987)}} & \mc{1}{c}{\scriptsize{(0.934)}} & \mc{1}{c}{\scriptsize{(0.974)}} & \mc{1}{c}{\scriptsize{(0.934)}} & \mc{1}{c}{\scriptsize{(0.947)}} & \mc{1}{c}{\scriptsize{(0.750)}} & \mc{1}{c}{\scriptsize{(0.974)}} & \mc{1}{c}{\scriptsize{(0.947)}} &  \\  

    \mc{1}{l}{\scriptsize{Crime}} & \mc{1}{c}{\scriptsize{33}} & \mc{1}{c}{\scriptsize{33}} & \mc{1}{c}{\scriptsize{33}} & \mc{1}{c}{\scriptsize{0}} & \mc{1}{c}{\scriptsize{33}} & \mc{1}{c}{\scriptsize{33}} & \mc{1}{c}{\scriptsize{33}} & \mc{1}{c}{\scriptsize{33}} & \mc{1}{c}{\scriptsize{3}} \\  

     & \mc{1}{c}{\scriptsize{(0.882)}} & \mc{1}{c}{\scriptsize{(0.724)}} & \mc{1}{c}{\scriptsize{(0.513)}} & \mc{1}{c}{\scriptsize{(0.987)}} & \mc{1}{c}{\scriptsize{(0.987)}} & \mc{1}{c}{\scriptsize{(0.500)}} & \mc{1}{c}{\scriptsize{(0.789)}} & \mc{1}{c}{\scriptsize{(0.921)}} &  \\  

    \mc{1}{l}{\scriptsize{Childhood Health}} & \mc{1}{c}{\scriptsize{79}} & \mc{1}{c}{\scriptsize{79}} & \mc{1}{c}{\scriptsize{77}} & \mc{1}{c}{\scriptsize{69}} & \mc{1}{c}{\scriptsize{77}} & \mc{1}{c}{\scriptsize{86}} & \mc{1}{c}{\scriptsize{79}} & \mc{1}{c}{\scriptsize{86}} & \mc{1}{c}{\scriptsize{14}} \\  

     & \mc{1}{c}{\scriptsize{\textbf{(0.000)}}} & \mc{1}{c}{\scriptsize{\textbf{(0.039)}}} & \mc{1}{c}{\scriptsize{\textbf{(0.000)}}} & \mc{1}{c}{\scriptsize{(0.132)}} & \mc{1}{c}{\scriptsize{\textbf{(0.000)}}} & \mc{1}{c}{\scriptsize{\textbf{(0.000)}}} & \mc{1}{c}{\scriptsize{\textbf{(0.026)}}} & \mc{1}{c}{\scriptsize{\textbf{(0.000)}}} &  \\  

    \mc{1}{l}{\scriptsize{Adult Health}} & \mc{1}{c}{\scriptsize{68}} & \mc{1}{c}{\scriptsize{62}} & \mc{1}{c}{\scriptsize{58}} & \mc{1}{c}{\scriptsize{62}} & \mc{1}{c}{\scriptsize{58}} & \mc{1}{c}{\scriptsize{59}} & \mc{1}{c}{\scriptsize{62}} & \mc{1}{c}{\scriptsize{59}} & \mc{1}{c}{\scriptsize{74}} \\  

     & \mc{1}{c}{\scriptsize{\textbf{(0.000)}}} & \mc{1}{c}{\scriptsize{\textbf{(0.013)}}} & \mc{1}{c}{\scriptsize{(0.237)}} & \mc{1}{c}{\scriptsize{\textbf{(0.066)}}} & \mc{1}{c}{\scriptsize{(0.237)}} & \mc{1}{c}{\scriptsize{\textbf{(0.039)}}} & \mc{1}{c}{\scriptsize{\textbf{(0.000)}}} & \mc{1}{c}{\scriptsize{\textbf{(0.039)}}} &  \\  

    \mc{1}{l}{\scriptsize{Mental Health}} & \mc{1}{c}{\scriptsize{41}} & \mc{1}{c}{\scriptsize{52}} & \mc{1}{c}{\scriptsize{17}} & \mc{1}{c}{\scriptsize{37}} & \mc{1}{c}{\scriptsize{22}} & \mc{1}{c}{\scriptsize{46}} & \mc{1}{c}{\scriptsize{56}} & \mc{1}{c}{\scriptsize{52}} & \mc{1}{c}{\scriptsize{54}} \\  

     & \mc{1}{c}{\scriptsize{(0.750)}} & \mc{1}{c}{\scriptsize{(0.487)}} & \mc{1}{c}{\scriptsize{(1.000)}} & \mc{1}{c}{\scriptsize{(0.789)}} & \mc{1}{c}{\scriptsize{(1.000)}} & \mc{1}{c}{\scriptsize{(0.645)}} & \mc{1}{c}{\scriptsize{(0.408)}} & \mc{1}{c}{\scriptsize{(0.553)}} &  \\  

  \bottomrule
  \end{tabular}
	\end{table}   

	\begin{table}[H]
     \caption{Combining Functions by Category $|$ 10\% Significance, Male Sample} 
     \label{table:abccare_rslt_male_counts_n10a10}
	  \begin{tabular}{cccccccccc}
  \toprule

    \scriptsize{Category} & \scriptsize{(1)} & \scriptsize{(2)} & \scriptsize{(3)} & \scriptsize{(4)} & \scriptsize{(5)} & \scriptsize{(6)} & \scriptsize{(7)} & \scriptsize{(8)} &  \\ 
    \midrule  

    \mc{1}{l}{\scriptsize{Cognitive Skills}} & \mc{1}{c}{\scriptsize{48}} & \mc{1}{c}{\scriptsize{48}} & \mc{1}{c}{\scriptsize{24}} & \mc{1}{c}{\scriptsize{24}} & \mc{1}{c}{\scriptsize{24}} & \mc{1}{c}{\scriptsize{60}} & \mc{1}{c}{\scriptsize{44}} & \mc{1}{c}{\scriptsize{56}} & \mc{1}{c}{\scriptsize{25}} \\  

     & \mc{1}{c}{\scriptsize{\textbf{(0.039)}}} & \mc{1}{c}{\scriptsize{\textbf{(0.013)}}} & \mc{1}{c}{\scriptsize{(0.237)}} & \mc{1}{c}{\scriptsize{(0.158)}} & \mc{1}{c}{\scriptsize{(0.197)}} & \mc{1}{c}{\scriptsize{\textbf{(0.000)}}} & \mc{1}{c}{\scriptsize{\textbf{(0.000)}}} & \mc{1}{c}{\scriptsize{\textbf{(0.000)}}} &  \\  

    \mc{1}{l}{\scriptsize{Noncognitive Skills}} & \mc{1}{c}{\scriptsize{9}} & \mc{1}{c}{\scriptsize{8}} & \mc{1}{c}{\scriptsize{5}} & \mc{1}{c}{\scriptsize{4}} & \mc{1}{c}{\scriptsize{5}} & \mc{1}{c}{\scriptsize{10}} & \mc{1}{c}{\scriptsize{9}} & \mc{1}{c}{\scriptsize{12}} & \mc{1}{c}{\scriptsize{117}} \\  

     & \mc{1}{c}{\scriptsize{(0.487)}} & \mc{1}{c}{\scriptsize{(0.684)}} & \mc{1}{c}{\scriptsize{(0.908)}} & \mc{1}{c}{\scriptsize{(0.987)}} & \mc{1}{c}{\scriptsize{(0.895)}} & \mc{1}{c}{\scriptsize{(0.342)}} & \mc{1}{c}{\scriptsize{(0.447)}} & \mc{1}{c}{\scriptsize{(0.316)}} &  \\  

    \mc{1}{l}{\scriptsize{Mother's Employment, Education, and Income}} & \mc{1}{c}{\scriptsize{0}} & \mc{1}{c}{\scriptsize{0}} & \mc{1}{c}{\scriptsize{25}} & \mc{1}{c}{\scriptsize{25}} & \mc{1}{c}{\scriptsize{25}} & \mc{1}{c}{\scriptsize{0}} & \mc{1}{c}{\scriptsize{0}} & \mc{1}{c}{\scriptsize{0}} & \mc{1}{c}{\scriptsize{4}} \\  

     & \mc{1}{c}{\scriptsize{(0.987)}} & \mc{1}{c}{\scriptsize{(0.987)}} & \mc{1}{c}{\scriptsize{(0.250)}} & \mc{1}{c}{\scriptsize{(0.237)}} & \mc{1}{c}{\scriptsize{(0.237)}} & \mc{1}{c}{\scriptsize{(0.974)}} & \mc{1}{c}{\scriptsize{(0.974)}} & \mc{1}{c}{\scriptsize{(0.974)}} &  \\  

    \mc{1}{l}{\scriptsize{Childhood Household Environment}} & \mc{1}{c}{\scriptsize{7}} & \mc{1}{c}{\scriptsize{13}} & \mc{1}{c}{\scriptsize{7}} & \mc{1}{c}{\scriptsize{13}} & \mc{1}{c}{\scriptsize{7}} & \mc{1}{c}{\scriptsize{14}} & \mc{1}{c}{\scriptsize{20}} & \mc{1}{c}{\scriptsize{27}} & \mc{1}{c}{\scriptsize{15}} \\  

     & \mc{1}{c}{\scriptsize{(0.658)}} & \mc{1}{c}{\scriptsize{(0.434)}} & \mc{1}{c}{\scriptsize{(0.697)}} & \mc{1}{c}{\scriptsize{(0.316)}} & \mc{1}{c}{\scriptsize{(0.618)}} & \mc{1}{c}{\scriptsize{(0.224)}} & \mc{1}{c}{\scriptsize{(0.250)}} & \mc{1}{c}{\scriptsize{(0.197)}} &  \\  

    \mc{1}{l}{\scriptsize{Adult Household Environment}} & \mc{1}{c}{\scriptsize{0}} & \mc{1}{c}{\scriptsize{0}} & \mc{1}{c}{\scriptsize{12}} & \mc{1}{c}{\scriptsize{0}} & \mc{1}{c}{\scriptsize{11}} & \mc{1}{c}{\scriptsize{0}} & \mc{1}{c}{\scriptsize{0}} & \mc{1}{c}{\scriptsize{0}} & \mc{1}{c}{\scriptsize{9}} \\  

     & \mc{1}{c}{\scriptsize{(1.000)}} & \mc{1}{c}{\scriptsize{(1.000)}} & \mc{1}{c}{\scriptsize{(0.197)}} & \mc{1}{c}{\scriptsize{(0.697)}} & \mc{1}{c}{\scriptsize{(0.408)}} & \mc{1}{c}{\scriptsize{(0.618)}} & \mc{1}{c}{\scriptsize{(1.000)}} & \mc{1}{c}{\scriptsize{(1.000)}} &  \\  

    \mc{1}{l}{\scriptsize{Education, Employment, Income}} & \mc{1}{c}{\scriptsize{11}} & \mc{1}{c}{\scriptsize{7}} & \mc{1}{c}{\scriptsize{0}} & \mc{1}{c}{\scriptsize{4}} & \mc{1}{c}{\scriptsize{0}} & \mc{1}{c}{\scriptsize{11}} & \mc{1}{c}{\scriptsize{4}} & \mc{1}{c}{\scriptsize{7}} & \mc{1}{c}{\scriptsize{28}} \\  

     & \mc{1}{c}{\scriptsize{(0.408)}} & \mc{1}{c}{\scriptsize{(0.632)}} & \mc{1}{c}{\scriptsize{(0.882)}} & \mc{1}{c}{\scriptsize{(1.000)}} & \mc{1}{c}{\scriptsize{(1.000)}} & \mc{1}{c}{\scriptsize{(0.421)}} & \mc{1}{c}{\scriptsize{(0.908)}} & \mc{1}{c}{\scriptsize{(0.592)}} &  \\  

    \mc{1}{l}{\scriptsize{Crime}} & \mc{1}{c}{\scriptsize{0}} & \mc{1}{c}{\scriptsize{0}} & \mc{1}{c}{\scriptsize{0}} & \mc{1}{c}{\scriptsize{0}} & \mc{1}{c}{\scriptsize{0}} & \mc{1}{c}{\scriptsize{0}} & \mc{1}{c}{\scriptsize{0}} & \mc{1}{c}{\scriptsize{0}} & \mc{1}{c}{\scriptsize{3}} \\  

     & \mc{1}{c}{\scriptsize{(1.000)}} & \mc{1}{c}{\scriptsize{(1.000)}} & \mc{1}{c}{\scriptsize{(0.987)}} & \mc{1}{c}{\scriptsize{(0.987)}} & \mc{1}{c}{\scriptsize{(0.987)}} & \mc{1}{c}{\scriptsize{(0.276)}} & \mc{1}{c}{\scriptsize{(1.000)}} & \mc{1}{c}{\scriptsize{(0.316)}} &  \\  

    \mc{1}{l}{\scriptsize{Childhood Health}} & \mc{1}{c}{\scriptsize{29}} & \mc{1}{c}{\scriptsize{29}} & \mc{1}{c}{\scriptsize{46}} & \mc{1}{c}{\scriptsize{31}} & \mc{1}{c}{\scriptsize{31}} & \mc{1}{c}{\scriptsize{21}} & \mc{1}{c}{\scriptsize{14}} & \mc{1}{c}{\scriptsize{21}} & \mc{1}{c}{\scriptsize{14}} \\  

     & \mc{1}{c}{\scriptsize{(0.184)}} & \mc{1}{c}{\scriptsize{(0.145)}} & \mc{1}{c}{\scriptsize{\textbf{(0.000)}}} & \mc{1}{c}{\scriptsize{\textbf{(0.066)}}} & \mc{1}{c}{\scriptsize{\textbf{(0.079)}}} & \mc{1}{c}{\scriptsize{(0.276)}} & \mc{1}{c}{\scriptsize{(0.276)}} & \mc{1}{c}{\scriptsize{(0.263)}} &  \\  

    \mc{1}{l}{\scriptsize{Adult Health}} & \mc{1}{c}{\scriptsize{27}} & \mc{1}{c}{\scriptsize{31}} & \mc{1}{c}{\scriptsize{12}} & \mc{1}{c}{\scriptsize{14}} & \mc{1}{c}{\scriptsize{10}} & \mc{1}{c}{\scriptsize{31}} & \mc{1}{c}{\scriptsize{27}} & \mc{1}{c}{\scriptsize{26}} & \mc{1}{c}{\scriptsize{74}} \\  

     & \mc{1}{c}{\scriptsize{\textbf{(0.000)}}} & \mc{1}{c}{\scriptsize{\textbf{(0.000)}}} & \mc{1}{c}{\scriptsize{(0.368)}} & \mc{1}{c}{\scriptsize{(0.224)}} & \mc{1}{c}{\scriptsize{(0.461)}} & \mc{1}{c}{\scriptsize{\textbf{(0.000)}}} & \mc{1}{c}{\scriptsize{\textbf{(0.000)}}} & \mc{1}{c}{\scriptsize{\textbf{(0.000)}}} &  \\  

    \mc{1}{l}{\scriptsize{Mental Health}} & \mc{1}{c}{\scriptsize{0}} & \mc{1}{c}{\scriptsize{7}} & \mc{1}{c}{\scriptsize{0}} & \mc{1}{c}{\scriptsize{0}} & \mc{1}{c}{\scriptsize{0}} & \mc{1}{c}{\scriptsize{2}} & \mc{1}{c}{\scriptsize{2}} & \mc{1}{c}{\scriptsize{2}} & \mc{1}{c}{\scriptsize{54}} \\  

     & \mc{1}{c}{\scriptsize{(1.000)}} & \mc{1}{c}{\scriptsize{(0.474)}} & \mc{1}{c}{\scriptsize{(1.000)}} & \mc{1}{c}{\scriptsize{(1.000)}} & \mc{1}{c}{\scriptsize{(1.000)}} & \mc{1}{c}{\scriptsize{(1.000)}} & \mc{1}{c}{\scriptsize{(1.000)}} & \mc{1}{c}{\scriptsize{(1.000)}} &  \\  

  \bottomrule
  \end{tabular}
	\end{table}   

	\begin{table}[H]
     \caption{Combining Functions by Category, Female Sample} 
     \label{table:abccare_rslt_female_counts_n50a100}
	  \begin{tabular}{cccccccccc}
  \toprule

    \scriptsize{Category} & \scriptsize{(1)} & \scriptsize{(2)} & \scriptsize{(3)} & \scriptsize{(4)} & \scriptsize{(5)} & \scriptsize{(6)} & \scriptsize{(7)} & \scriptsize{(8)} &  \\ 
    \midrule  

    \mc{1}{l}{\scriptsize{Cognitive Skills}} & \mc{1}{c}{\scriptsize{100}} & \mc{1}{c}{\scriptsize{96}} & \mc{1}{c}{\scriptsize{100}} & \mc{1}{c}{\scriptsize{100}} & \mc{1}{c}{\scriptsize{100}} & \mc{1}{c}{\scriptsize{100}} & \mc{1}{c}{\scriptsize{92}} & \mc{1}{c}{\scriptsize{100}} & \mc{1}{c}{\scriptsize{25}} \\  

     & \mc{1}{c}{\scriptsize{\textbf{(0.000)}}} & \mc{1}{c}{\scriptsize{\textbf{(0.000)}}} & \mc{1}{c}{\scriptsize{\textbf{(0.000)}}} & \mc{1}{c}{\scriptsize{\textbf{(0.000)}}} & \mc{1}{c}{\scriptsize{\textbf{(0.000)}}} & \mc{1}{c}{\scriptsize{\textbf{(0.000)}}} & \mc{1}{c}{\scriptsize{\textbf{(0.000)}}} & \mc{1}{c}{\scriptsize{\textbf{(0.000)}}} &  \\  

    \mc{1}{l}{\scriptsize{Noncognitive Skills}} & \mc{1}{c}{\scriptsize{70}} & \mc{1}{c}{\scriptsize{70}} & \mc{1}{c}{\scriptsize{74}} & \mc{1}{c}{\scriptsize{74}} & \mc{1}{c}{\scriptsize{77}} & \mc{1}{c}{\scriptsize{57}} & \mc{1}{c}{\scriptsize{61}} & \mc{1}{c}{\scriptsize{66}} & \mc{1}{c}{\scriptsize{117}} \\  

     & \mc{1}{c}{\scriptsize{\textbf{(0.000)}}} & \mc{1}{c}{\scriptsize{\textbf{(0.000)}}} & \mc{1}{c}{\scriptsize{\textbf{(0.000)}}} & \mc{1}{c}{\scriptsize{\textbf{(0.000)}}} & \mc{1}{c}{\scriptsize{\textbf{(0.000)}}} & \mc{1}{c}{\scriptsize{(0.237)}} & \mc{1}{c}{\scriptsize{(0.105)}} & \mc{1}{c}{\scriptsize{\textbf{(0.013)}}} &  \\  

    \mc{1}{l}{\scriptsize{Mother's Employment, Education, and Income}} & \mc{1}{c}{\scriptsize{50}} & \mc{1}{c}{\scriptsize{25}} & \mc{1}{c}{\scriptsize{50}} & \mc{1}{c}{\scriptsize{25}} & \mc{1}{c}{\scriptsize{50}} & \mc{1}{c}{\scriptsize{50}} & \mc{1}{c}{\scriptsize{25}} & \mc{1}{c}{\scriptsize{50}} & \mc{1}{c}{\scriptsize{4}} \\  

     & \mc{1}{c}{\scriptsize{(0.842)}} & \mc{1}{c}{\scriptsize{(0.974)}} & \mc{1}{c}{\scriptsize{(0.724)}} & \mc{1}{c}{\scriptsize{(0.750)}} & \mc{1}{c}{\scriptsize{(0.645)}} & \mc{1}{c}{\scriptsize{(0.224)}} & \mc{1}{c}{\scriptsize{(0.803)}} & \mc{1}{c}{\scriptsize{(0.250)}} &  \\  

    \mc{1}{l}{\scriptsize{Childhood Household Environment}} & \mc{1}{c}{\scriptsize{73}} & \mc{1}{c}{\scriptsize{80}} & \mc{1}{c}{\scriptsize{60}} & \mc{1}{c}{\scriptsize{67}} & \mc{1}{c}{\scriptsize{60}} & \mc{1}{c}{\scriptsize{64}} & \mc{1}{c}{\scriptsize{64}} & \mc{1}{c}{\scriptsize{79}} & \mc{1}{c}{\scriptsize{15}} \\  

     & \mc{1}{c}{\scriptsize{\textbf{(0.039)}}} & \mc{1}{c}{\scriptsize{\textbf{(0.026)}}} & \mc{1}{c}{\scriptsize{\textbf{(0.066)}}} & \mc{1}{c}{\scriptsize{\textbf{(0.053)}}} & \mc{1}{c}{\scriptsize{(0.132)}} & \mc{1}{c}{\scriptsize{(0.263)}} & \mc{1}{c}{\scriptsize{(0.145)}} & \mc{1}{c}{\scriptsize{\textbf{(0.053)}}} &  \\  

    \mc{1}{l}{\scriptsize{Adult Household Environment}} & \mc{1}{c}{\scriptsize{78}} & \mc{1}{c}{\scriptsize{67}} & \mc{1}{c}{\scriptsize{89}} & \mc{1}{c}{\scriptsize{78}} & \mc{1}{c}{\scriptsize{89}} & \mc{1}{c}{\scriptsize{89}} & \mc{1}{c}{\scriptsize{56}} & \mc{1}{c}{\scriptsize{56}} & \mc{1}{c}{\scriptsize{9}} \\  

     & \mc{1}{c}{\scriptsize{\textbf{(0.000)}}} & \mc{1}{c}{\scriptsize{\textbf{(0.092)}}} & \mc{1}{c}{\scriptsize{\textbf{(0.000)}}} & \mc{1}{c}{\scriptsize{\textbf{(0.053)}}} & \mc{1}{c}{\scriptsize{\textbf{(0.000)}}} & \mc{1}{c}{\scriptsize{\textbf{(0.000)}}} & \mc{1}{c}{\scriptsize{(0.303)}} & \mc{1}{c}{\scriptsize{(0.289)}} &  \\  

    \mc{1}{l}{\scriptsize{Education, Employment, Income}} & \mc{1}{c}{\scriptsize{71}} & \mc{1}{c}{\scriptsize{89}} & \mc{1}{c}{\scriptsize{74}} & \mc{1}{c}{\scriptsize{78}} & \mc{1}{c}{\scriptsize{74}} & \mc{1}{c}{\scriptsize{71}} & \mc{1}{c}{\scriptsize{75}} & \mc{1}{c}{\scriptsize{71}} & \mc{1}{c}{\scriptsize{28}} \\  

     & \mc{1}{c}{\scriptsize{\textbf{(0.013)}}} & \mc{1}{c}{\scriptsize{\textbf{(0.000)}}} & \mc{1}{c}{\scriptsize{\textbf{(0.000)}}} & \mc{1}{c}{\scriptsize{\textbf{(0.000)}}} & \mc{1}{c}{\scriptsize{\textbf{(0.000)}}} & \mc{1}{c}{\scriptsize{\textbf{(0.026)}}} & \mc{1}{c}{\scriptsize{\textbf{(0.000)}}} & \mc{1}{c}{\scriptsize{\textbf{(0.000)}}} &  \\  

    \mc{1}{l}{\scriptsize{Crime}} & \mc{1}{c}{\scriptsize{100}} & \mc{1}{c}{\scriptsize{100}} & \mc{1}{c}{\scriptsize{100}} & \mc{1}{c}{\scriptsize{100}} & \mc{1}{c}{\scriptsize{100}} & \mc{1}{c}{\scriptsize{100}} & \mc{1}{c}{\scriptsize{100}} & \mc{1}{c}{\scriptsize{67}} & \mc{1}{c}{\scriptsize{3}} \\  

     & \mc{1}{c}{\scriptsize{\textbf{(0.000)}}} & \mc{1}{c}{\scriptsize{\textbf{(0.000)}}} & \mc{1}{c}{\scriptsize{\textbf{(0.000)}}} & \mc{1}{c}{\scriptsize{\textbf{(0.000)}}} & \mc{1}{c}{\scriptsize{\textbf{(0.000)}}} & \mc{1}{c}{\scriptsize{\textbf{(0.000)}}} & \mc{1}{c}{\scriptsize{\textbf{(0.000)}}} & \mc{1}{c}{\scriptsize{(0.421)}} &  \\  

    \mc{1}{l}{\scriptsize{Childhood Health}} & \mc{1}{c}{\scriptsize{64}} & \mc{1}{c}{\scriptsize{50}} & \mc{1}{c}{\scriptsize{57}} & \mc{1}{c}{\scriptsize{50}} & \mc{1}{c}{\scriptsize{50}} & \mc{1}{c}{\scriptsize{64}} & \mc{1}{c}{\scriptsize{43}} & \mc{1}{c}{\scriptsize{50}} & \mc{1}{c}{\scriptsize{14}} \\  

     & \mc{1}{c}{\scriptsize{(0.145)}} & \mc{1}{c}{\scriptsize{(0.447)}} & \mc{1}{c}{\scriptsize{(0.382)}} & \mc{1}{c}{\scriptsize{(0.553)}} & \mc{1}{c}{\scriptsize{(0.566)}} & \mc{1}{c}{\scriptsize{(0.145)}} & \mc{1}{c}{\scriptsize{(0.737)}} & \mc{1}{c}{\scriptsize{(0.474)}} &  \\  

    \mc{1}{l}{\scriptsize{Adult Health}} & \mc{1}{c}{\scriptsize{45}} & \mc{1}{c}{\scriptsize{44}} & \mc{1}{c}{\scriptsize{50}} & \mc{1}{c}{\scriptsize{52}} & \mc{1}{c}{\scriptsize{51}} & \mc{1}{c}{\scriptsize{38}} & \mc{1}{c}{\scriptsize{34}} & \mc{1}{c}{\scriptsize{33}} & \mc{1}{c}{\scriptsize{84}} \\  

     & \mc{1}{c}{\scriptsize{(0.750)}} & \mc{1}{c}{\scriptsize{(0.882)}} & \mc{1}{c}{\scriptsize{(0.500)}} & \mc{1}{c}{\scriptsize{(0.329)}} & \mc{1}{c}{\scriptsize{(0.474)}} & \mc{1}{c}{\scriptsize{(0.987)}} & \mc{1}{c}{\scriptsize{(1.000)}} & \mc{1}{c}{\scriptsize{(1.000)}} &  \\  

    \mc{1}{l}{\scriptsize{Mental Health}} & \mc{1}{c}{\scriptsize{73}} & \mc{1}{c}{\scriptsize{82}} & \mc{1}{c}{\scriptsize{73}} & \mc{1}{c}{\scriptsize{87}} & \mc{1}{c}{\scriptsize{76}} & \mc{1}{c}{\scriptsize{77}} & \mc{1}{c}{\scriptsize{84}} & \mc{1}{c}{\scriptsize{79}} & \mc{1}{c}{\scriptsize{56}} \\  

     & \mc{1}{c}{\scriptsize{\textbf{(0.013)}}} & \mc{1}{c}{\scriptsize{\textbf{(0.000)}}} & \mc{1}{c}{\scriptsize{\textbf{(0.079)}}} & \mc{1}{c}{\scriptsize{\textbf{(0.000)}}} & \mc{1}{c}{\scriptsize{\textbf{(0.000)}}} & \mc{1}{c}{\scriptsize{\textbf{(0.000)}}} & \mc{1}{c}{\scriptsize{\textbf{(0.000)}}} & \mc{1}{c}{\scriptsize{\textbf{(0.000)}}} &  \\  

  \bottomrule
  \end{tabular}
	\end{table}   

	\begin{table}[H]
     \caption{Combining Functions by Category $|$ 10\% Significance, Female Sample} 
     \label{table:abccare_rslt_female_counts_n10a10}
	\input{AppResOutput/abccare/rslt_female_counts_n10a10_all}
	\end{table}   
\clearpage

\section{Treatment Effects for Pooled Sample}


	\begin{table}[H]
     \caption{Treatment Effects on IQ Scores, Pooled Sample}
     \label{table:abccare_rslt_pooled_cat0}
	  \begin{tabular}{cccccccccc}
  \toprule

    \scriptsize{Variable} & \scriptsize{Age} & \scriptsize{(1)} & \scriptsize{(2)} & \scriptsize{(3)} & \scriptsize{(4)} & \scriptsize{(5)} & \scriptsize{(6)} & \scriptsize{(7)} & \scriptsize{(8)} \\ 
    \midrule  

    \mc{1}{l}{\scriptsize{Std. IQ Test}} & \mc{1}{c}{\scriptsize{2}} & \mc{1}{c}{\scriptsize{10.116}} & \mc{1}{c}{\scriptsize{10.461}} & \mc{1}{c}{\scriptsize{10.609}} & \mc{1}{c}{\scriptsize{12.312}} & \mc{1}{c}{\scriptsize{11.140}} & \mc{1}{c}{\scriptsize{9.863}} & \mc{1}{c}{\scriptsize{10.116}} & \mc{1}{c}{\scriptsize{10.214}} \\  

     &  & \mc{1}{c}{\scriptsize{\textbf{(0.000)}}} & \mc{1}{c}{\scriptsize{\textbf{(0.000)}}} & \mc{1}{c}{\scriptsize{\textbf{(0.000)}}} & \mc{1}{c}{\scriptsize{\textbf{(0.000)}}} & \mc{1}{c}{\scriptsize{\textbf{(0.000)}}} & \mc{1}{c}{\scriptsize{\textbf{(0.000)}}} & \mc{1}{c}{\scriptsize{\textbf{(0.000)}}} & \mc{1}{c}{\scriptsize{\textbf{(0.000)}}} \\  

     & \mc{1}{c}{\scriptsize{3}} & \mc{1}{c}{\scriptsize{13.450}} & \mc{1}{c}{\scriptsize{14.071}} & \mc{1}{c}{\scriptsize{19.242}} & \mc{1}{c}{\scriptsize{21.795}} & \mc{1}{c}{\scriptsize{20.287}} & \mc{1}{c}{\scriptsize{11.314}} & \mc{1}{c}{\scriptsize{12.053}} & \mc{1}{c}{\scriptsize{11.777}} \\  

     &  & \mc{1}{c}{\scriptsize{\textbf{(0.000)}}} & \mc{1}{c}{\scriptsize{\textbf{(0.000)}}} & \mc{1}{c}{\scriptsize{\textbf{(0.000)}}} & \mc{1}{c}{\scriptsize{\textbf{(0.000)}}} & \mc{1}{c}{\scriptsize{\textbf{(0.000)}}} & \mc{1}{c}{\scriptsize{\textbf{(0.000)}}} & \mc{1}{c}{\scriptsize{\textbf{(0.000)}}} & \mc{1}{c}{\scriptsize{\textbf{(0.000)}}} \\  

     & \mc{1}{c}{\scriptsize{3.5}} & \mc{1}{c}{\scriptsize{8.387}} & \mc{1}{c}{\scriptsize{7.886}} & \mc{1}{c}{\scriptsize{11.255}} & \mc{1}{c}{\scriptsize{12.220}} & \mc{1}{c}{\scriptsize{11.438}} & \mc{1}{c}{\scriptsize{7.276}} & \mc{1}{c}{\scriptsize{6.711}} & \mc{1}{c}{\scriptsize{7.008}} \\  

     &  & \mc{1}{c}{\scriptsize{\textbf{(0.000)}}} & \mc{1}{c}{\scriptsize{\textbf{(0.000)}}} & \mc{1}{c}{\scriptsize{\textbf{(0.000)}}} & \mc{1}{c}{\scriptsize{\textbf{(0.000)}}} & \mc{1}{c}{\scriptsize{\textbf{(0.000)}}} & \mc{1}{c}{\scriptsize{\textbf{(0.000)}}} & \mc{1}{c}{\scriptsize{\textbf{(0.013)}}} & \mc{1}{c}{\scriptsize{\textbf{(0.000)}}} \\  

     & \mc{1}{c}{\scriptsize{4}} & \mc{1}{c}{\scriptsize{9.166}} & \mc{1}{c}{\scriptsize{8.768}} & \mc{1}{c}{\scriptsize{11.985}} & \mc{1}{c}{\scriptsize{12.570}} & \mc{1}{c}{\scriptsize{12.562}} & \mc{1}{c}{\scriptsize{8.149}} & \mc{1}{c}{\scriptsize{7.853}} & \mc{1}{c}{\scriptsize{8.525}} \\  

     &  & \mc{1}{c}{\scriptsize{\textbf{(0.000)}}} & \mc{1}{c}{\scriptsize{\textbf{(0.000)}}} & \mc{1}{c}{\scriptsize{\textbf{(0.000)}}} & \mc{1}{c}{\scriptsize{\textbf{(0.000)}}} & \mc{1}{c}{\scriptsize{\textbf{(0.000)}}} & \mc{1}{c}{\scriptsize{\textbf{(0.000)}}} & \mc{1}{c}{\scriptsize{\textbf{(0.000)}}} & \mc{1}{c}{\scriptsize{\textbf{(0.000)}}} \\  

     & \mc{1}{c}{\scriptsize{4.5}} & \mc{1}{c}{\scriptsize{8.380}} & \mc{1}{c}{\scriptsize{8.004}} & \mc{1}{c}{\scriptsize{13.287}} & \mc{1}{c}{\scriptsize{13.937}} & \mc{1}{c}{\scriptsize{13.523}} & \mc{1}{c}{\scriptsize{6.717}} & \mc{1}{c}{\scriptsize{6.246}} & \mc{1}{c}{\scriptsize{6.827}} \\  

     &  & \mc{1}{c}{\scriptsize{\textbf{(0.000)}}} & \mc{1}{c}{\scriptsize{\textbf{(0.000)}}} & \mc{1}{c}{\scriptsize{\textbf{(0.000)}}} & \mc{1}{c}{\scriptsize{\textbf{(0.000)}}} & \mc{1}{c}{\scriptsize{\textbf{(0.000)}}} & \mc{1}{c}{\scriptsize{\textbf{(0.000)}}} & \mc{1}{c}{\scriptsize{\textbf{(0.013)}}} & \mc{1}{c}{\scriptsize{\textbf{(0.013)}}} \\  

     & \mc{1}{c}{\scriptsize{5}} & \mc{1}{c}{\scriptsize{6.362}} & \mc{1}{c}{\scriptsize{5.621}} & \mc{1}{c}{\scriptsize{8.310}} & \mc{1}{c}{\scriptsize{9.003}} & \mc{1}{c}{\scriptsize{8.749}} & \mc{1}{c}{\scriptsize{5.760}} & \mc{1}{c}{\scriptsize{4.776}} & \mc{1}{c}{\scriptsize{5.595}} \\  

     &  & \mc{1}{c}{\scriptsize{\textbf{(0.013)}}} & \mc{1}{c}{\scriptsize{\textbf{(0.013)}}} & \mc{1}{c}{\scriptsize{\textbf{(0.000)}}} & \mc{1}{c}{\scriptsize{\textbf{(0.000)}}} & \mc{1}{c}{\scriptsize{\textbf{(0.000)}}} & \mc{1}{c}{\scriptsize{\textbf{(0.013)}}} & \mc{1}{c}{\scriptsize{\textbf{(0.039)}}} & \mc{1}{c}{\scriptsize{\textbf{(0.013)}}} \\  

     & \mc{1}{c}{\scriptsize{6.6}} & \mc{1}{c}{\scriptsize{5.956}} & \mc{1}{c}{\scriptsize{5.497}} & \mc{1}{c}{\scriptsize{4.088}} & \mc{1}{c}{\scriptsize{4.646}} & \mc{1}{c}{\scriptsize{4.845}} & \mc{1}{c}{\scriptsize{5.850}} & \mc{1}{c}{\scriptsize{5.369}} & \mc{1}{c}{\scriptsize{6.138}} \\  

     &  & \mc{1}{c}{\scriptsize{\textbf{(0.013)}}} & \mc{1}{c}{\scriptsize{\textbf{(0.013)}}} & \mc{1}{c}{\scriptsize{(0.132)}} & \mc{1}{c}{\scriptsize{\textbf{(0.092)}}} & \mc{1}{c}{\scriptsize{\textbf{(0.079)}}} & \mc{1}{c}{\scriptsize{\textbf{(0.026)}}} & \mc{1}{c}{\scriptsize{\textbf{(0.039)}}} & \mc{1}{c}{\scriptsize{\textbf{(0.013)}}} \\  

     & \mc{1}{c}{\scriptsize{7}} & \mc{1}{c}{\scriptsize{5.373}} & \mc{1}{c}{\scriptsize{4.912}} & \mc{1}{c}{\scriptsize{6.575}} & \mc{1}{c}{\scriptsize{7.626}} & \mc{1}{c}{\scriptsize{6.822}} & \mc{1}{c}{\scriptsize{5.066}} & \mc{1}{c}{\scriptsize{3.970}} & \mc{1}{c}{\scriptsize{4.945}} \\  

     &  & \mc{1}{c}{\scriptsize{\textbf{(0.013)}}} & \mc{1}{c}{\scriptsize{\textbf{(0.026)}}} & \mc{1}{c}{\scriptsize{\textbf{(0.026)}}} & \mc{1}{c}{\scriptsize{\textbf{(0.013)}}} & \mc{1}{c}{\scriptsize{\textbf{(0.026)}}} & \mc{1}{c}{\scriptsize{\textbf{(0.026)}}} & \mc{1}{c}{\scriptsize{\textbf{(0.079)}}} & \mc{1}{c}{\scriptsize{\textbf{(0.039)}}} \\  

     & \mc{1}{c}{\scriptsize{8}} & \mc{1}{c}{\scriptsize{4.932}} & \mc{1}{c}{\scriptsize{3.848}} & \mc{1}{c}{\scriptsize{2.570}} & \mc{1}{c}{\scriptsize{2.766}} & \mc{1}{c}{\scriptsize{2.709}} & \mc{1}{c}{\scriptsize{4.948}} & \mc{1}{c}{\scriptsize{3.952}} & \mc{1}{c}{\scriptsize{4.799}} \\  

     &  & \mc{1}{c}{\scriptsize{\textbf{(0.026)}}} & \mc{1}{c}{\scriptsize{\textbf{(0.066)}}} & \mc{1}{c}{\scriptsize{(0.276)}} & \mc{1}{c}{\scriptsize{(0.289)}} & \mc{1}{c}{\scriptsize{(0.303)}} & \mc{1}{c}{\scriptsize{\textbf{(0.026)}}} & \mc{1}{c}{\scriptsize{\textbf{(0.079)}}} & \mc{1}{c}{\scriptsize{\textbf{(0.026)}}} \\  

     & \mc{1}{c}{\scriptsize{12}} & \mc{1}{c}{\scriptsize{4.524}} & \mc{1}{c}{\scriptsize{3.612}} & \mc{1}{c}{\scriptsize{3.251}} & \mc{1}{c}{\scriptsize{2.881}} & \mc{1}{c}{\scriptsize{2.731}} & \mc{1}{c}{\scriptsize{4.766}} & \mc{1}{c}{\scriptsize{3.628}} & \mc{1}{c}{\scriptsize{3.579}} \\  

     &  & \mc{1}{c}{\scriptsize{\textbf{(0.000)}}} & \mc{1}{c}{\scriptsize{\textbf{(0.066)}}} & \mc{1}{c}{\scriptsize{(0.224)}} & \mc{1}{c}{\scriptsize{(0.276)}} & \mc{1}{c}{\scriptsize{(0.250)}} & \mc{1}{c}{\scriptsize{\textbf{(0.013)}}} & \mc{1}{c}{\scriptsize{\textbf{(0.053)}}} & \mc{1}{c}{\scriptsize{\textbf{(0.039)}}} \\  

    \mc{1}{l}{\scriptsize{IQ Factor}} & \mc{1}{c}{\scriptsize{2 to 5}} & \mc{1}{c}{\scriptsize{0.744}} & \mc{1}{c}{\scriptsize{0.723}} & \mc{1}{c}{\scriptsize{1.017}} & \mc{1}{c}{\scriptsize{1.113}} & \mc{1}{c}{\scriptsize{1.055}} & \mc{1}{c}{\scriptsize{0.662}} & \mc{1}{c}{\scriptsize{0.625}} & \mc{1}{c}{\scriptsize{0.670}} \\  

     &  & \mc{1}{c}{\scriptsize{\textbf{(0.000)}}} & \mc{1}{c}{\scriptsize{\textbf{(0.000)}}} & \mc{1}{c}{\scriptsize{\textbf{(0.000)}}} & \mc{1}{c}{\scriptsize{\textbf{(0.000)}}} & \mc{1}{c}{\scriptsize{\textbf{(0.000)}}} & \mc{1}{c}{\scriptsize{\textbf{(0.000)}}} & \mc{1}{c}{\scriptsize{\textbf{(0.000)}}} & \mc{1}{c}{\scriptsize{\textbf{(0.000)}}} \\  

     & \mc{1}{c}{\scriptsize{6 to 12}} & \mc{1}{c}{\scriptsize{0.403}} & \mc{1}{c}{\scriptsize{0.377}} & \mc{1}{c}{\scriptsize{0.373}} & \mc{1}{c}{\scriptsize{0.434}} & \mc{1}{c}{\scriptsize{0.400}} & \mc{1}{c}{\scriptsize{0.411}} & \mc{1}{c}{\scriptsize{0.355}} & \mc{1}{c}{\scriptsize{0.411}} \\  

     &  & \mc{1}{c}{\scriptsize{\textbf{(0.026)}}} & \mc{1}{c}{\scriptsize{\textbf{(0.039)}}} & \mc{1}{c}{\scriptsize{(0.105)}} & \mc{1}{c}{\scriptsize{\textbf{(0.079)}}} & \mc{1}{c}{\scriptsize{(0.105)}} & \mc{1}{c}{\scriptsize{\textbf{(0.026)}}} & \mc{1}{c}{\scriptsize{\textbf{(0.066)}}} & \mc{1}{c}{\scriptsize{\textbf{(0.039)}}} \\ 
    \midrule  

    \mc{2}{l}{\scriptsize{\% of Pos. TE ($H_0$: $\le$ 50\%)}} & \mc{1}{c}{\scriptsize{100}} & \mc{1}{c}{\scriptsize{100}} & \mc{1}{c}{\scriptsize{100}} & \mc{1}{c}{\scriptsize{100}} & \mc{1}{c}{\scriptsize{100}} & \mc{1}{c}{\scriptsize{100}} & \mc{1}{c}{\scriptsize{100}} & \mc{1}{c}{\scriptsize{100}} \\  

     &  & \mc{1}{c}{\scriptsize{\textbf{(0.000)}}} & \mc{1}{c}{\scriptsize{\textbf{(0.000)}}} & \mc{1}{c}{\scriptsize{\textbf{(0.000)}}} & \mc{1}{c}{\scriptsize{\textbf{(0.000)}}} & \mc{1}{c}{\scriptsize{\textbf{(0.000)}}} & \mc{1}{c}{\scriptsize{\textbf{(0.000)}}} & \mc{1}{c}{\scriptsize{\textbf{(0.000)}}} & \mc{1}{c}{\scriptsize{\textbf{(0.000)}}} \\  

    \mc{2}{l}{\scriptsize{\% of Pos. TE ($H_0$: $\le$ 10\% $|$ 10\% Significance)}} & \mc{1}{c}{\scriptsize{100}} & \mc{1}{c}{\scriptsize{100}} & \mc{1}{c}{\scriptsize{75}} & \mc{1}{c}{\scriptsize{75}} & \mc{1}{c}{\scriptsize{75}} & \mc{1}{c}{\scriptsize{100}} & \mc{1}{c}{\scriptsize{100}} & \mc{1}{c}{\scriptsize{100}} \\  

     &  & \mc{1}{c}{\scriptsize{\textbf{(0.000)}}} & \mc{1}{c}{\scriptsize{\textbf{(0.000)}}} & \mc{1}{c}{\scriptsize{\textbf{(0.000)}}} & \mc{1}{c}{\scriptsize{\textbf{(0.000)}}} & \mc{1}{c}{\scriptsize{\textbf{(0.000)}}} & \mc{1}{c}{\scriptsize{\textbf{(0.000)}}} & \mc{1}{c}{\scriptsize{\textbf{(0.000)}}} & \mc{1}{c}{\scriptsize{\textbf{(0.000)}}} \\  

  \bottomrule
  \end{tabular}
	\end{table} 

	\begin{table}[H]
     \caption{Treatment Effects on Achievement Scores, Pooled Sample}
     \label{table:abccare_rslt_pooled_cat1}
	  \begin{tabular}{cccccccccc}
  \toprule

    \scriptsize{Variable} & \scriptsize{Age} & \scriptsize{(1)} & \scriptsize{(2)} & \scriptsize{(3)} & \scriptsize{(4)} & \scriptsize{(5)} & \scriptsize{(6)} & \scriptsize{(7)} & \scriptsize{(8)} \\ 
    \midrule  

    \mc{1}{l}{\scriptsize{Heart Attack}} & \mc{1}{c}{\scriptsize{Mid-30s}} &  &  &  &  &  &  &  &  \\  

     &  &  &  &  &  &  &  &  &  \\  

    \mc{1}{l}{\scriptsize{Sickle Cell Anemia}} & \mc{1}{c}{\scriptsize{Mid-30s}} &  &  &  &  &  &  &  &  \\  

     &  &  &  &  &  &  &  &  &  \\  

    \mc{1}{l}{\scriptsize{Asthma}} & \mc{1}{c}{\scriptsize{Mid-30s}} & \mc{1}{c}{\scriptsize{0.016}} & \mc{1}{c}{\scriptsize{-0.002}} & \mc{1}{c}{\scriptsize{0.040}} & \mc{1}{c}{\scriptsize{0.021}} & \mc{1}{c}{\scriptsize{0.025}} & \mc{1}{c}{\scriptsize{0.009}} & \mc{1}{c}{\scriptsize{-0.016}} & \mc{1}{c}{\scriptsize{-0.006}} \\  

     &  & \mc{1}{c}{\scriptsize{(0.605)}} & \mc{1}{c}{\scriptsize{(0.355)}} & \mc{1}{c}{\scriptsize{(0.789)}} & \mc{1}{c}{\scriptsize{(0.382)}} & \mc{1}{c}{\scriptsize{(0.461)}} & \mc{1}{c}{\scriptsize{(0.447)}} & \mc{1}{c}{\scriptsize{(0.250)}} & \mc{1}{c}{\scriptsize{(0.355)}} \\  

    \mc{1}{l}{\scriptsize{Stroke}} & \mc{1}{c}{\scriptsize{Mid-30s}} &  &  &  &  &  &  &  &  \\  

     &  &  &  &  &  &  &  &  &  \\  

    \mc{1}{l}{\scriptsize{High Blood Pressure (Hypertension)}} & \mc{1}{c}{\scriptsize{Mid-30s}} & \mc{1}{c}{\scriptsize{-0.003}} & \mc{1}{c}{\scriptsize{-0.010}} & \mc{1}{c}{\scriptsize{0.021}} & \mc{1}{c}{\scriptsize{0.008}} & \mc{1}{c}{\scriptsize{0.026}} & \mc{1}{c}{\scriptsize{-0.010}} & \mc{1}{c}{\scriptsize{-0.017}} & \mc{1}{c}{\scriptsize{-0.007}} \\  

     &  & \mc{1}{c}{\scriptsize{(0.408)}} & \mc{1}{c}{\scriptsize{(0.329)}} & \mc{1}{c}{\scriptsize{(0.539)}} & \mc{1}{c}{\scriptsize{(0.487)}} & \mc{1}{c}{\scriptsize{(0.553)}} & \mc{1}{c}{\scriptsize{(0.395)}} & \mc{1}{c}{\scriptsize{(0.342)}} & \mc{1}{c}{\scriptsize{(0.395)}} \\  

    \mc{1}{l}{\scriptsize{Arthritis or Generative Disease}} & \mc{1}{c}{\scriptsize{Mid-30s}} & \mc{1}{c}{\scriptsize{0.021}} & \mc{1}{c}{\scriptsize{0.024}} & \mc{1}{c}{\scriptsize{0.021}} & \mc{1}{c}{\scriptsize{0.016}} & \mc{1}{c}{\scriptsize{0.022}} & \mc{1}{c}{\scriptsize{0.021}} & \mc{1}{c}{\scriptsize{0.028}} & \mc{1}{c}{\scriptsize{0.022}} \\  

     &  & \mc{1}{c}{\scriptsize{(0.487)}} & \mc{1}{c}{\scriptsize{(0.474)}} & \mc{1}{c}{\scriptsize{(0.487)}} & \mc{1}{c}{\scriptsize{(0.382)}} & \mc{1}{c}{\scriptsize{(0.487)}} & \mc{1}{c}{\scriptsize{(0.487)}} & \mc{1}{c}{\scriptsize{(0.474)}} & \mc{1}{c}{\scriptsize{(0.487)}} \\  

    \mc{1}{l}{\scriptsize{Diabetes}} & \mc{1}{c}{\scriptsize{Mid-30s}} & \mc{1}{c}{\scriptsize{0.021}} & \mc{1}{c}{\scriptsize{0.025}} & \mc{1}{c}{\scriptsize{0.021}} & \mc{1}{c}{\scriptsize{0.002}} & \mc{1}{c}{\scriptsize{0.026}} & \mc{1}{c}{\scriptsize{0.021}} & \mc{1}{c}{\scriptsize{0.030}} & \mc{1}{c}{\scriptsize{0.026}} \\  

     &  & \mc{1}{c}{\scriptsize{(0.579)}} & \mc{1}{c}{\scriptsize{(0.579)}} & \mc{1}{c}{\scriptsize{(0.579)}} & \mc{1}{c}{\scriptsize{(0.276)}} & \mc{1}{c}{\scriptsize{(0.566)}} & \mc{1}{c}{\scriptsize{(0.579)}} & \mc{1}{c}{\scriptsize{(0.579)}} & \mc{1}{c}{\scriptsize{(0.566)}} \\  

    \mc{1}{l}{\scriptsize{Cancer}} & \mc{1}{c}{\scriptsize{Mid-30s}} &  &  &  &  &  &  &  &  \\  

     &  &  &  &  &  &  &  &  &  \\  

    \mc{1}{l}{\scriptsize{Heart Attack or Coronary Disease}} & \mc{1}{c}{\scriptsize{Mid-30s}} &  &  &  &  &  &  &  &  \\  

     &  &  &  &  &  &  &  &  &  \\  

    \mc{1}{l}{\scriptsize{High Cholesterol}} & \mc{1}{c}{\scriptsize{Mid-30s}} &  &  &  &  &  &  &  &  \\  

     &  &  &  &  &  &  &  &  &  \\  

    \mc{1}{l}{\scriptsize{Dementia}} & \mc{1}{c}{\scriptsize{Mid-30s}} &  &  &  &  &  &  &  &  \\  

     &  &  &  &  &  &  &  &  &  \\  

  \bottomrule
  \end{tabular}
	\end{table} 

	\begin{table}[H]
     \caption{Treatment Effects on Infant Behavior Record, Pooled Sample}
     \label{table:abccare_rslt_pooled_cat2}
	  \begin{tabular}{cccccccccc}
  \toprule

    \scriptsize{Variable} & \scriptsize{Age} & \scriptsize{(1)} & \scriptsize{(2)} & \scriptsize{(3)} & \scriptsize{(4)} & \scriptsize{(5)} & \scriptsize{(6)} & \scriptsize{(7)} & \scriptsize{(8)} \\ 
    \midrule  

    \mc{1}{l}{\scriptsize{Temperament cluster - activity level}} & \mc{1}{c}{\scriptsize{0.5}} & \mc{1}{c}{\scriptsize{0.609}} & \mc{1}{c}{\scriptsize{0.349}} & \mc{1}{c}{\scriptsize{0.981}} & \mc{1}{c}{\scriptsize{0.608}} & \mc{1}{c}{\scriptsize{0.876}} & \mc{1}{c}{\scriptsize{0.555}} & \mc{1}{c}{\scriptsize{0.272}} & \mc{1}{c}{\scriptsize{0.322}} \\  

     &  & \mc{1}{c}{\scriptsize{(0.132)}} & \mc{1}{c}{\scriptsize{(0.263)}} & \mc{1}{c}{\scriptsize{(0.132)}} & \mc{1}{c}{\scriptsize{(0.197)}} & \mc{1}{c}{\scriptsize{(0.145)}} & \mc{1}{c}{\scriptsize{(0.145)}} & \mc{1}{c}{\scriptsize{(0.303)}} & \mc{1}{c}{\scriptsize{(0.276)}} \\  

     & \mc{1}{c}{\scriptsize{1}} & \mc{1}{c}{\scriptsize{1.076}} & \mc{1}{c}{\scriptsize{0.743}} & \mc{1}{c}{\scriptsize{1.152}} & \mc{1}{c}{\scriptsize{0.831}} & \mc{1}{c}{\scriptsize{1.084}} & \mc{1}{c}{\scriptsize{0.930}} & \mc{1}{c}{\scriptsize{0.714}} & \mc{1}{c}{\scriptsize{0.815}} \\  

     &  & \mc{1}{c}{\scriptsize{\textbf{(0.013)}}} & \mc{1}{c}{\scriptsize{(0.105)}} & \mc{1}{c}{\scriptsize{(0.118)}} & \mc{1}{c}{\scriptsize{(0.211)}} & \mc{1}{c}{\scriptsize{(0.171)}} & \mc{1}{c}{\scriptsize{\textbf{(0.053)}}} & \mc{1}{c}{\scriptsize{(0.132)}} & \mc{1}{c}{\scriptsize{(0.132)}} \\  

     & \mc{1}{c}{\scriptsize{1.5}} & \mc{1}{c}{\scriptsize{-0.496}} & \mc{1}{c}{\scriptsize{-0.506}} & \mc{1}{c}{\scriptsize{0.240}} & \mc{1}{c}{\scriptsize{0.444}} & \mc{1}{c}{\scriptsize{0.637}} & \mc{1}{c}{\scriptsize{-0.893}} & \mc{1}{c}{\scriptsize{-0.749}} & \mc{1}{c}{\scriptsize{-0.567}} \\  

     &  & \mc{1}{c}{\scriptsize{(0.816)}} & \mc{1}{c}{\scriptsize{(0.776)}} & \mc{1}{c}{\scriptsize{(0.355)}} & \mc{1}{c}{\scriptsize{(0.342)}} & \mc{1}{c}{\scriptsize{(0.250)}} & \mc{1}{c}{\scriptsize{(0.921)}} & \mc{1}{c}{\scriptsize{(0.882)}} & \mc{1}{c}{\scriptsize{(0.855)}} \\  

     & \mc{1}{c}{\scriptsize{2}} & \mc{1}{c}{\scriptsize{-0.839}} & \mc{1}{c}{\scriptsize{-1.997}} & \mc{1}{c}{\scriptsize{-1.149}} & \mc{1}{c}{\scriptsize{-1.954}} & \mc{1}{c}{\scriptsize{-2.104}} & \mc{1}{c}{\scriptsize{-0.755}} & \mc{1}{c}{\scriptsize{-1.935}} & \mc{1}{c}{\scriptsize{-1.343}} \\  

     &  & \mc{1}{c}{\scriptsize{(0.855)}} & \mc{1}{c}{\scriptsize{(1.000)}} & \mc{1}{c}{\scriptsize{(0.829)}} & \mc{1}{c}{\scriptsize{(0.961)}} & \mc{1}{c}{\scriptsize{(0.974)}} & \mc{1}{c}{\scriptsize{(0.803)}} & \mc{1}{c}{\scriptsize{(1.000)}} & \mc{1}{c}{\scriptsize{(0.947)}} \\  

    \mc{1}{l}{\scriptsize{Temperament cluster - cooperativeness}} & \mc{1}{c}{\scriptsize{0.5}} & \mc{1}{c}{\scriptsize{0.422}} & \mc{1}{c}{\scriptsize{-0.948}} & \mc{1}{c}{\scriptsize{-1.476}} & \mc{1}{c}{\scriptsize{-3.544}} & \mc{1}{c}{\scriptsize{-1.940}} & \mc{1}{c}{\scriptsize{1.420}} & \mc{1}{c}{\scriptsize{1.183}} & \mc{1}{c}{\scriptsize{1.266}} \\  

     &  & \mc{1}{c}{\scriptsize{(0.329)}} & \mc{1}{c}{\scriptsize{(0.671)}} & \mc{1}{c}{\scriptsize{(0.829)}} & \mc{1}{c}{\scriptsize{(0.566)}} & \mc{1}{c}{\scriptsize{(0.855)}} & \mc{1}{c}{\scriptsize{(0.171)}} & \mc{1}{c}{\scriptsize{(0.355)}} & \mc{1}{c}{\scriptsize{(0.237)}} \\  

     & \mc{1}{c}{\scriptsize{1}} & \mc{1}{c}{\scriptsize{0.100}} & \mc{1}{c}{\scriptsize{2.358}} & \mc{1}{c}{\scriptsize{-0.733}} & \mc{1}{c}{\scriptsize{2.184}} & \mc{1}{c}{\scriptsize{0.778}} & \mc{1}{c}{\scriptsize{0.412}} & \mc{1}{c}{\scriptsize{2.462}} & \mc{1}{c}{\scriptsize{2.226}} \\  

     &  & \mc{1}{c}{\scriptsize{(0.461)}} & \mc{1}{c}{\scriptsize{(0.105)}} & \mc{1}{c}{\scriptsize{(0.618)}} & \mc{1}{c}{\scriptsize{\textbf{(0.066)}}} & \mc{1}{c}{\scriptsize{(0.395)}} & \mc{1}{c}{\scriptsize{(0.408)}} & \mc{1}{c}{\scriptsize{(0.118)}} & \mc{1}{c}{\scriptsize{(0.171)}} \\  

     & \mc{1}{c}{\scriptsize{1.5}} & \mc{1}{c}{\scriptsize{0.694}} & \mc{1}{c}{\scriptsize{2.704}} & \mc{1}{c}{\scriptsize{-0.700}} & \mc{1}{c}{\scriptsize{0.985}} & \mc{1}{c}{\scriptsize{1.191}} & \mc{1}{c}{\scriptsize{1.217}} & \mc{1}{c}{\scriptsize{3.787}} & \mc{1}{c}{\scriptsize{3.621}} \\  

     &  & \mc{1}{c}{\scriptsize{(0.329)}} & \mc{1}{c}{\scriptsize{\textbf{(0.053)}}} & \mc{1}{c}{\scriptsize{(0.605)}} & \mc{1}{c}{\scriptsize{(0.118)}} & \mc{1}{c}{\scriptsize{(0.342)}} & \mc{1}{c}{\scriptsize{(0.250)}} & \mc{1}{c}{\scriptsize{\textbf{(0.013)}}} & \mc{1}{c}{\scriptsize{\textbf{(0.026)}}} \\  

     & \mc{1}{c}{\scriptsize{2}} & \mc{1}{c}{\scriptsize{2.439}} & \mc{1}{c}{\scriptsize{2.385}} & \mc{1}{c}{\scriptsize{2.500}} & \mc{1}{c}{\scriptsize{3.612}} & \mc{1}{c}{\scriptsize{2.746}} & \mc{1}{c}{\scriptsize{2.417}} & \mc{1}{c}{\scriptsize{2.289}} & \mc{1}{c}{\scriptsize{2.500}} \\  

     &  & \mc{1}{c}{\scriptsize{(0.118)}} & \mc{1}{c}{\scriptsize{(0.145)}} & \mc{1}{c}{\scriptsize{(0.132)}} & \mc{1}{c}{\scriptsize{\textbf{(0.026)}}} & \mc{1}{c}{\scriptsize{(0.145)}} & \mc{1}{c}{\scriptsize{(0.118)}} & \mc{1}{c}{\scriptsize{(0.145)}} & \mc{1}{c}{\scriptsize{(0.158)}} \\  

    \mc{1}{l}{\scriptsize{Temperament cluster - sociability}} & \mc{1}{c}{\scriptsize{0.5}} & \mc{1}{c}{\scriptsize{-0.007}} & \mc{1}{c}{\scriptsize{-0.053}} & \mc{1}{c}{\scriptsize{0.383}} & \mc{1}{c}{\scriptsize{0.424}} & \mc{1}{c}{\scriptsize{0.389}} & \mc{1}{c}{\scriptsize{-0.147}} & \mc{1}{c}{\scriptsize{-0.160}} & \mc{1}{c}{\scriptsize{-0.185}} \\  

     &  & \mc{1}{c}{\scriptsize{(0.553)}} & \mc{1}{c}{\scriptsize{(0.605)}} & \mc{1}{c}{\scriptsize{(0.105)}} & \mc{1}{c}{\scriptsize{\textbf{(0.092)}}} & \mc{1}{c}{\scriptsize{(0.118)}} & \mc{1}{c}{\scriptsize{(0.737)}} & \mc{1}{c}{\scriptsize{(0.750)}} & \mc{1}{c}{\scriptsize{(0.737)}} \\  

     & \mc{1}{c}{\scriptsize{1}} & \mc{1}{c}{\scriptsize{0.353}} & \mc{1}{c}{\scriptsize{0.395}} & \mc{1}{c}{\scriptsize{-0.159}} & \mc{1}{c}{\scriptsize{-0.083}} & \mc{1}{c}{\scriptsize{-0.068}} & \mc{1}{c}{\scriptsize{0.537}} & \mc{1}{c}{\scriptsize{0.630}} & \mc{1}{c}{\scriptsize{0.670}} \\  

     &  & \mc{1}{c}{\scriptsize{\textbf{(0.053)}}} & \mc{1}{c}{\scriptsize{\textbf{(0.026)}}} & \mc{1}{c}{\scriptsize{(0.671)}} & \mc{1}{c}{\scriptsize{(0.605)}} & \mc{1}{c}{\scriptsize{(0.592)}} & \mc{1}{c}{\scriptsize{\textbf{(0.026)}}} & \mc{1}{c}{\scriptsize{\textbf{(0.000)}}} & \mc{1}{c}{\scriptsize{\textbf{(0.000)}}} \\  

     & \mc{1}{c}{\scriptsize{1.5}} & \mc{1}{c}{\scriptsize{0.156}} & \mc{1}{c}{\scriptsize{0.463}} & \mc{1}{c}{\scriptsize{0.044}} & \mc{1}{c}{\scriptsize{0.458}} & \mc{1}{c}{\scriptsize{0.364}} & \mc{1}{c}{\scriptsize{0.224}} & \mc{1}{c}{\scriptsize{0.519}} & \mc{1}{c}{\scriptsize{0.554}} \\  

     &  & \mc{1}{c}{\scriptsize{(0.250)}} & \mc{1}{c}{\scriptsize{\textbf{(0.039)}}} & \mc{1}{c}{\scriptsize{(0.395)}} & \mc{1}{c}{\scriptsize{(0.171)}} & \mc{1}{c}{\scriptsize{(0.171)}} & \mc{1}{c}{\scriptsize{(0.211)}} & \mc{1}{c}{\scriptsize{\textbf{(0.013)}}} & \mc{1}{c}{\scriptsize{\textbf{(0.039)}}} \\  

     & \mc{1}{c}{\scriptsize{2}} & \mc{1}{c}{\scriptsize{-0.047}} & \mc{1}{c}{\scriptsize{-0.191}} & \mc{1}{c}{\scriptsize{-0.557}} & \mc{1}{c}{\scriptsize{-0.161}} & \mc{1}{c}{\scriptsize{-0.760}} & \mc{1}{c}{\scriptsize{0.058}} & \mc{1}{c}{\scriptsize{-0.142}} & \mc{1}{c}{\scriptsize{-0.099}} \\  

     &  & \mc{1}{c}{\scriptsize{(0.566)}} & \mc{1}{c}{\scriptsize{(0.684)}} & \mc{1}{c}{\scriptsize{(0.868)}} & \mc{1}{c}{\scriptsize{(0.539)}} & \mc{1}{c}{\scriptsize{(0.947)}} & \mc{1}{c}{\scriptsize{(0.434)}} & \mc{1}{c}{\scriptsize{(0.592)}} & \mc{1}{c}{\scriptsize{(0.579)}} \\  

    \mc{1}{l}{\scriptsize{Temperament cluster - task orientation}} & \mc{1}{c}{\scriptsize{0.5}} & \mc{1}{c}{\scriptsize{-0.071}} & \mc{1}{c}{\scriptsize{0.429}} & \mc{1}{c}{\scriptsize{0.767}} & \mc{1}{c}{\scriptsize{1.466}} & \mc{1}{c}{\scriptsize{1.171}} & \mc{1}{c}{\scriptsize{-0.309}} & \mc{1}{c}{\scriptsize{-0.109}} & \mc{1}{c}{\scriptsize{-0.133}} \\  

     &  & \mc{1}{c}{\scriptsize{(0.553)}} & \mc{1}{c}{\scriptsize{(0.276)}} & \mc{1}{c}{\scriptsize{(0.289)}} & \mc{1}{c}{\scriptsize{(0.105)}} & \mc{1}{c}{\scriptsize{(0.197)}} & \mc{1}{c}{\scriptsize{(0.592)}} & \mc{1}{c}{\scriptsize{(0.539)}} & \mc{1}{c}{\scriptsize{(0.539)}} \\  

     & \mc{1}{c}{\scriptsize{1}} & \mc{1}{c}{\scriptsize{0.946}} & \mc{1}{c}{\scriptsize{1.327}} & \mc{1}{c}{\scriptsize{0.737}} & \mc{1}{c}{\scriptsize{1.472}} & \mc{1}{c}{\scriptsize{0.949}} & \mc{1}{c}{\scriptsize{1.055}} & \mc{1}{c}{\scriptsize{1.429}} & \mc{1}{c}{\scriptsize{1.284}} \\  

     &  & \mc{1}{c}{\scriptsize{\textbf{(0.013)}}} & \mc{1}{c}{\scriptsize{\textbf{(0.026)}}} & \mc{1}{c}{\scriptsize{(0.118)}} & \mc{1}{c}{\scriptsize{\textbf{(0.039)}}} & \mc{1}{c}{\scriptsize{(0.105)}} & \mc{1}{c}{\scriptsize{\textbf{(0.026)}}} & \mc{1}{c}{\scriptsize{\textbf{(0.039)}}} & \mc{1}{c}{\scriptsize{\textbf{(0.013)}}} \\  

     & \mc{1}{c}{\scriptsize{1.5}} & \mc{1}{c}{\scriptsize{2.363}} & \mc{1}{c}{\scriptsize{3.176}} & \mc{1}{c}{\scriptsize{1.762}} & \mc{1}{c}{\scriptsize{2.714}} & \mc{1}{c}{\scriptsize{2.219}} & \mc{1}{c}{\scriptsize{2.483}} & \mc{1}{c}{\scriptsize{3.162}} & \mc{1}{c}{\scriptsize{2.910}} \\  

     &  & \mc{1}{c}{\scriptsize{\textbf{(0.000)}}} & \mc{1}{c}{\scriptsize{\textbf{(0.000)}}} & \mc{1}{c}{\scriptsize{\textbf{(0.039)}}} & \mc{1}{c}{\scriptsize{\textbf{(0.000)}}} & \mc{1}{c}{\scriptsize{\textbf{(0.000)}}} & \mc{1}{c}{\scriptsize{\textbf{(0.000)}}} & \mc{1}{c}{\scriptsize{\textbf{(0.000)}}} & \mc{1}{c}{\scriptsize{\textbf{(0.000)}}} \\  

     & \mc{1}{c}{\scriptsize{2}} & \mc{1}{c}{\scriptsize{1.929}} & \mc{1}{c}{\scriptsize{2.982}} & \mc{1}{c}{\scriptsize{1.893}} & \mc{1}{c}{\scriptsize{3.522}} & \mc{1}{c}{\scriptsize{3.408}} & \mc{1}{c}{\scriptsize{1.938}} & \mc{1}{c}{\scriptsize{2.798}} & \mc{1}{c}{\scriptsize{2.781}} \\  

     &  & \mc{1}{c}{\scriptsize{\textbf{(0.000)}}} & \mc{1}{c}{\scriptsize{\textbf{(0.000)}}} & \mc{1}{c}{\scriptsize{\textbf{(0.026)}}} & \mc{1}{c}{\scriptsize{\textbf{(0.000)}}} & \mc{1}{c}{\scriptsize{\textbf{(0.000)}}} & \mc{1}{c}{\scriptsize{\textbf{(0.000)}}} & \mc{1}{c}{\scriptsize{\textbf{(0.000)}}} & \mc{1}{c}{\scriptsize{\textbf{(0.000)}}} \\  

  \bottomrule
  \end{tabular}
	\end{table} 

	\begin{table}[H]
     \caption{Treatment Effects on Kohn and Rosman: Attentive/Cooperative, Pooled Sample}
     \label{table:abccare_rslt_pooled_cat3}
	  \begin{tabular}{cccccccccc}
  \toprule

    \scriptsize{Variable} & \scriptsize{Age} & \scriptsize{(1)} & \scriptsize{(2)} & \scriptsize{(3)} & \scriptsize{(4)} & \scriptsize{(5)} & \scriptsize{(6)} & \scriptsize{(7)} & \scriptsize{(8)} \\ 
    \midrule  

    \mc{1}{l}{\scriptsize{Parental Income}} & \mc{1}{c}{\scriptsize{1.5}} & \mc{1}{c}{\scriptsize{2,248}} & \mc{1}{c}{\scriptsize{3,277}} & \mc{1}{c}{\scriptsize{2,553}} & \mc{1}{c}{\scriptsize{4,721}} & \mc{1}{c}{\scriptsize{5,028}} & \mc{1}{c}{\scriptsize{1,870}} & \mc{1}{c}{\scriptsize{2,901}} & \mc{1}{c}{\scriptsize{3,718}} \\  

     &  & \mc{1}{c}{\scriptsize{(0.132)}} & \mc{1}{c}{\scriptsize{\textbf{(0.066)}}} & \mc{1}{c}{\scriptsize{(0.211)}} & \mc{1}{c}{\scriptsize{(0.118)}} & \mc{1}{c}{\scriptsize{\textbf{(0.066)}}} & \mc{1}{c}{\scriptsize{(0.184)}} & \mc{1}{c}{\scriptsize{(0.105)}} & \mc{1}{c}{\scriptsize{\textbf{(0.026)}}} \\  

     & \mc{1}{c}{\scriptsize{2.5}} & \mc{1}{c}{\scriptsize{516}} & \mc{1}{c}{\scriptsize{366}} & \mc{1}{c}{\scriptsize{-2,455}} & \mc{1}{c}{\scriptsize{-851}} & \mc{1}{c}{\scriptsize{93.152}} & \mc{1}{c}{\scriptsize{988}} & \mc{1}{c}{\scriptsize{469}} & \mc{1}{c}{\scriptsize{1,555}} \\  

     &  & \mc{1}{c}{\scriptsize{(0.355)}} & \mc{1}{c}{\scriptsize{(0.500)}} & \mc{1}{c}{\scriptsize{(0.763)}} & \mc{1}{c}{\scriptsize{(0.500)}} & \mc{1}{c}{\scriptsize{(0.461)}} & \mc{1}{c}{\scriptsize{(0.329)}} & \mc{1}{c}{\scriptsize{(0.447)}} & \mc{1}{c}{\scriptsize{(0.237)}} \\  

     & \mc{1}{c}{\scriptsize{3.5}} & \mc{1}{c}{\scriptsize{1,821}} & \mc{1}{c}{\scriptsize{1,901}} & \mc{1}{c}{\scriptsize{3,984}} & \mc{1}{c}{\scriptsize{4,990}} & \mc{1}{c}{\scriptsize{5,265}} & \mc{1}{c}{\scriptsize{961}} & \mc{1}{c}{\scriptsize{961}} & \mc{1}{c}{\scriptsize{2,108}} \\  

     &  & \mc{1}{c}{\scriptsize{(0.184)}} & \mc{1}{c}{\scriptsize{(0.237)}} & \mc{1}{c}{\scriptsize{\textbf{(0.066)}}} & \mc{1}{c}{\scriptsize{\textbf{(0.092)}}} & \mc{1}{c}{\scriptsize{\textbf{(0.053)}}} & \mc{1}{c}{\scriptsize{(0.289)}} & \mc{1}{c}{\scriptsize{(0.329)}} & \mc{1}{c}{\scriptsize{(0.197)}} \\  

     & \mc{1}{c}{\scriptsize{4.5}} & \mc{1}{c}{\scriptsize{2,336}} & \mc{1}{c}{\scriptsize{3,565}} & \mc{1}{c}{\scriptsize{4,159}} & \mc{1}{c}{\scriptsize{4,738}} & \mc{1}{c}{\scriptsize{2,703}} & \mc{1}{c}{\scriptsize{1,434}} & \mc{1}{c}{\scriptsize{2,926}} & \mc{1}{c}{\scriptsize{2,697}} \\  

     &  & \mc{1}{c}{\scriptsize{(0.105)}} & \mc{1}{c}{\scriptsize{\textbf{(0.092)}}} & \mc{1}{c}{\scriptsize{\textbf{(0.066)}}} & \mc{1}{c}{\scriptsize{\textbf{(0.079)}}} & \mc{1}{c}{\scriptsize{(0.197)}} & \mc{1}{c}{\scriptsize{(0.263)}} & \mc{1}{c}{\scriptsize{(0.132)}} & \mc{1}{c}{\scriptsize{(0.158)}} \\  

    \mc{1}{l}{\scriptsize{Parental Income Factor}} & \mc{1}{c}{\scriptsize{1.5 to 15}} & \mc{1}{c}{\scriptsize{0.156}} & \mc{1}{c}{\scriptsize{0.120}} & \mc{1}{c}{\scriptsize{0.158}} & \mc{1}{c}{\scriptsize{0.303}} & \mc{1}{c}{\scriptsize{0.207}} & \mc{1}{c}{\scriptsize{0.127}} & \mc{1}{c}{\scriptsize{0.063}} & \mc{1}{c}{\scriptsize{0.183}} \\  

     &  & \mc{1}{c}{\scriptsize{(0.184)}} & \mc{1}{c}{\scriptsize{(0.329)}} & \mc{1}{c}{\scriptsize{(0.355)}} & \mc{1}{c}{\scriptsize{(0.184)}} & \mc{1}{c}{\scriptsize{(0.224)}} & \mc{1}{c}{\scriptsize{(0.224)}} & \mc{1}{c}{\scriptsize{(0.434)}} & \mc{1}{c}{\scriptsize{(0.211)}} \\  

  \bottomrule
  \end{tabular}
	\end{table} 

	\begin{table}[H]
     \caption{Treatment Effects on Classroom Behavior Inventory (Part I), Pooled Sample}
     \label{table:abccare_rslt_pooled_cat4}
	  \begin{tabular}{cccccccccc}
  \toprule

    \scriptsize{Variable} & \scriptsize{Age} & \scriptsize{(1)} & \scriptsize{(2)} & \scriptsize{(3)} & \scriptsize{(4)} & \scriptsize{(5)} & \scriptsize{(6)} & \scriptsize{(7)} & \scriptsize{(8)} \\ 
    \midrule  

    \mc{1}{l}{\scriptsize{Has Health Insurance}} & \mc{1}{c}{\scriptsize{30}} & \mc{1}{c}{\scriptsize{0.074}} & \mc{1}{c}{\scriptsize{0.079}} & \mc{1}{c}{\scriptsize{0.077}} & \mc{1}{c}{\scriptsize{0.041}} & \mc{1}{c}{\scriptsize{0.087}} & \mc{1}{c}{\scriptsize{0.077}} & \mc{1}{c}{\scriptsize{0.057}} & \mc{1}{c}{\scriptsize{0.081}} \\  

     &  & \mc{1}{c}{\scriptsize{(0.171)}} & \mc{1}{c}{\scriptsize{(0.184)}} & \mc{1}{c}{\scriptsize{(0.237)}} & \mc{1}{c}{\scriptsize{(0.355)}} & \mc{1}{c}{\scriptsize{(0.263)}} & \mc{1}{c}{\scriptsize{(0.211)}} & \mc{1}{c}{\scriptsize{(0.263)}} & \mc{1}{c}{\scriptsize{(0.250)}} \\  

     & \mc{1}{c}{\scriptsize{21}} & \mc{1}{c}{\scriptsize{0.090}} & \mc{1}{c}{\scriptsize{0.070}} & \mc{1}{c}{\scriptsize{0.166}} & \mc{1}{c}{\scriptsize{0.181}} & \mc{1}{c}{\scriptsize{0.167}} & \mc{1}{c}{\scriptsize{0.022}} & \mc{1}{c}{\scriptsize{0.026}} & \mc{1}{c}{\scriptsize{0.051}} \\  

     &  & \mc{1}{c}{\scriptsize{\textbf{(0.079)}}} & \mc{1}{c}{\scriptsize{(0.224)}} & \mc{1}{c}{\scriptsize{\textbf{(0.053)}}} & \mc{1}{c}{\scriptsize{\textbf{(0.066)}}} & \mc{1}{c}{\scriptsize{\textbf{(0.079)}}} & \mc{1}{c}{\scriptsize{(0.382)}} & \mc{1}{c}{\scriptsize{(0.355)}} & \mc{1}{c}{\scriptsize{(0.263)}} \\  

  \bottomrule
  \end{tabular}
	\end{table} 

	\begin{table}[H]
     \caption{Treatment Effects on Classroom Behavior Inventory (Part II), Pooled Sample}
     \label{table:abccare_rslt_pooled_cat5}
	  \begin{tabular}{cccccccccc}
  \toprule

    \scriptsize{Variable} & \scriptsize{Age} & \scriptsize{(1)} & \scriptsize{(2)} & \scriptsize{(3)} & \scriptsize{(4)} & \scriptsize{(5)} & \scriptsize{(6)} & \scriptsize{(7)} & \scriptsize{(8)} \\ 
    \midrule  

    \mc{1}{l}{\scriptsize{Prehypertension}} & \mc{1}{c}{\scriptsize{Mid-30s}} & \mc{1}{c}{\scriptsize{-0.176}} & \mc{1}{c}{\scriptsize{-0.181}} & \mc{1}{c}{\scriptsize{-0.049}} & \mc{1}{c}{\scriptsize{-0.055}} & \mc{1}{c}{\scriptsize{-0.060}} & \mc{1}{c}{\scriptsize{-0.240}} & \mc{1}{c}{\scriptsize{-0.268}} & \mc{1}{c}{\scriptsize{-0.280}} \\  

     &  & \mc{1}{c}{\scriptsize{\textbf{(0.000)}}} & \mc{1}{c}{\scriptsize{\textbf{(0.053)}}} & \mc{1}{c}{\scriptsize{(0.382)}} & \mc{1}{c}{\scriptsize{(0.355)}} & \mc{1}{c}{\scriptsize{(0.342)}} & \mc{1}{c}{\scriptsize{\textbf{(0.000)}}} & \mc{1}{c}{\scriptsize{\textbf{(0.000)}}} & \mc{1}{c}{\scriptsize{\textbf{(0.000)}}} \\  

    \mc{1}{l}{\scriptsize{Systolic Blood Pressure (mm Hg)}} & \mc{1}{c}{\scriptsize{Mid-30s}} & \mc{1}{c}{\scriptsize{-5.625}} & \mc{1}{c}{\scriptsize{-10.849}} & \mc{1}{c}{\scriptsize{5.375}} & \mc{1}{c}{\scriptsize{6.809}} & \mc{1}{c}{\scriptsize{4.383}} & \mc{1}{c}{\scriptsize{-9.438}} & \mc{1}{c}{\scriptsize{-16.448}} & \mc{1}{c}{\scriptsize{-15.882}} \\  

     &  & \mc{1}{c}{\scriptsize{\textbf{(0.092)}}} & \mc{1}{c}{\scriptsize{\textbf{(0.026)}}} & \mc{1}{c}{\scriptsize{(0.882)}} & \mc{1}{c}{\scriptsize{(0.803)}} & \mc{1}{c}{\scriptsize{(0.829)}} & \mc{1}{c}{\scriptsize{\textbf{(0.066)}}} & \mc{1}{c}{\scriptsize{\textbf{(0.000)}}} & \mc{1}{c}{\scriptsize{\textbf{(0.013)}}} \\  

    \mc{1}{l}{\scriptsize{Diastolic Blood Pressure (mm Hg)}} & \mc{1}{c}{\scriptsize{Mid-30s}} & \mc{1}{c}{\scriptsize{-5.312}} & \mc{1}{c}{\scriptsize{-7.745}} & \mc{1}{c}{\scriptsize{-1.424}} & \mc{1}{c}{\scriptsize{0.519}} & \mc{1}{c}{\scriptsize{-1.637}} & \mc{1}{c}{\scriptsize{-7.219}} & \mc{1}{c}{\scriptsize{-9.589}} & \mc{1}{c}{\scriptsize{-10.696}} \\  

     &  & \mc{1}{c}{\scriptsize{\textbf{(0.039)}}} & \mc{1}{c}{\scriptsize{\textbf{(0.026)}}} & \mc{1}{c}{\scriptsize{(0.316)}} & \mc{1}{c}{\scriptsize{(0.579)}} & \mc{1}{c}{\scriptsize{(0.303)}} & \mc{1}{c}{\scriptsize{\textbf{(0.026)}}} & \mc{1}{c}{\scriptsize{\textbf{(0.053)}}} & \mc{1}{c}{\scriptsize{\textbf{(0.000)}}} \\  

    \mc{1}{l}{\scriptsize{Hypertension}} & \mc{1}{c}{\scriptsize{Mid-30s}} & \mc{1}{c}{\scriptsize{-0.036}} & \mc{1}{c}{\scriptsize{-0.097}} & \mc{1}{c}{\scriptsize{0.083}} & \mc{1}{c}{\scriptsize{0.143}} & \mc{1}{c}{\scriptsize{0.045}} & \mc{1}{c}{\scriptsize{-0.083}} & \mc{1}{c}{\scriptsize{-0.151}} & \mc{1}{c}{\scriptsize{-0.187}} \\  

     &  & \mc{1}{c}{\scriptsize{(0.316)}} & \mc{1}{c}{\scriptsize{(0.158)}} & \mc{1}{c}{\scriptsize{(0.724)}} & \mc{1}{c}{\scriptsize{(0.763)}} & \mc{1}{c}{\scriptsize{(0.658)}} & \mc{1}{c}{\scriptsize{(0.171)}} & \mc{1}{c}{\scriptsize{(0.118)}} & \mc{1}{c}{\scriptsize{\textbf{(0.066)}}} \\  

  \bottomrule
  \end{tabular}
	\end{table} 

	\begin{table}[H]
     \caption{Treatment Effects on Emotional, Activity, Sociability, Impulsivity Survey, Pooled Sample}
     \label{table:abccare_rslt_pooled_cat6}
	  \begin{tabular}{cccccccccc}
  \toprule

    \scriptsize{Variable} & \scriptsize{Age} & \scriptsize{(1)} & \scriptsize{(2)} & \scriptsize{(3)} & \scriptsize{(4)} & \scriptsize{(5)} & \scriptsize{(6)} & \scriptsize{(7)} & \scriptsize{(8)} \\ 
    \midrule  

    \mc{1}{l}{\scriptsize{Albumin/Globulin Ratio}} & \mc{1}{c}{\scriptsize{Mid-30s}} & \mc{1}{c}{\scriptsize{0.052}} & \mc{1}{c}{\scriptsize{0.012}} & \mc{1}{c}{\scriptsize{0.059}} & \mc{1}{c}{\scriptsize{0.076}} & \mc{1}{c}{\scriptsize{0.044}} & \mc{1}{c}{\scriptsize{0.060}} & \mc{1}{c}{\scriptsize{0.012}} & \mc{1}{c}{\scriptsize{0.028}} \\  

     &  & \mc{1}{c}{\scriptsize{(0.132)}} & \mc{1}{c}{\scriptsize{(0.408)}} & \mc{1}{c}{\scriptsize{(0.263)}} & \mc{1}{c}{\scriptsize{(0.105)}} & \mc{1}{c}{\scriptsize{(0.303)}} & \mc{1}{c}{\scriptsize{(0.145)}} & \mc{1}{c}{\scriptsize{(0.382)}} & \mc{1}{c}{\scriptsize{(0.316)}} \\  

    \mc{1}{l}{\scriptsize{ALT}} & \mc{1}{c}{\scriptsize{Mid-30s}} & \mc{1}{c}{\scriptsize{1.840}} & \mc{1}{c}{\scriptsize{-0.268}} & \mc{1}{c}{\scriptsize{1.174}} & \mc{1}{c}{\scriptsize{-3.828}} & \mc{1}{c}{\scriptsize{-0.366}} & \mc{1}{c}{\scriptsize{1.929}} & \mc{1}{c}{\scriptsize{0.774}} & \mc{1}{c}{\scriptsize{0.420}} \\  

     &  & \mc{1}{c}{\scriptsize{(0.316)}} & \mc{1}{c}{\scriptsize{(0.500)}} & \mc{1}{c}{\scriptsize{(0.434)}} & \mc{1}{c}{\scriptsize{(0.763)}} & \mc{1}{c}{\scriptsize{(0.566)}} & \mc{1}{c}{\scriptsize{(0.303)}} & \mc{1}{c}{\scriptsize{(0.355)}} & \mc{1}{c}{\scriptsize{(0.421)}} \\  

    \mc{1}{l}{\scriptsize{Albumin}} & \mc{1}{c}{\scriptsize{Mid-30s}} & \mc{1}{c}{\scriptsize{0.113}} & \mc{1}{c}{\scriptsize{0.044}} & \mc{1}{c}{\scriptsize{0.142}} & \mc{1}{c}{\scriptsize{0.105}} & \mc{1}{c}{\scriptsize{0.113}} & \mc{1}{c}{\scriptsize{0.113}} & \mc{1}{c}{\scriptsize{0.040}} & \mc{1}{c}{\scriptsize{0.059}} \\  

     &  & \mc{1}{c}{\scriptsize{\textbf{(0.079)}}} & \mc{1}{c}{\scriptsize{(0.224)}} & \mc{1}{c}{\scriptsize{(0.197)}} & \mc{1}{c}{\scriptsize{(0.118)}} & \mc{1}{c}{\scriptsize{(0.211)}} & \mc{1}{c}{\scriptsize{\textbf{(0.066)}}} & \mc{1}{c}{\scriptsize{(0.263)}} & \mc{1}{c}{\scriptsize{(0.158)}} \\  

    \mc{1}{l}{\scriptsize{Sodium}} & \mc{1}{c}{\scriptsize{Mid-30s}} & \mc{1}{c}{\scriptsize{0.145}} & \mc{1}{c}{\scriptsize{0.120}} & \mc{1}{c}{\scriptsize{0.701}} & \mc{1}{c}{\scriptsize{1.413}} & \mc{1}{c}{\scriptsize{0.699}} & \mc{1}{c}{\scriptsize{0.053}} & \mc{1}{c}{\scriptsize{-0.071}} & \mc{1}{c}{\scriptsize{-0.020}} \\  

     &  & \mc{1}{c}{\scriptsize{(0.329)}} & \mc{1}{c}{\scriptsize{(0.461)}} & \mc{1}{c}{\scriptsize{\textbf{(0.092)}}} & \mc{1}{c}{\scriptsize{\textbf{(0.013)}}} & \mc{1}{c}{\scriptsize{\textbf{(0.079)}}} & \mc{1}{c}{\scriptsize{(0.461)}} & \mc{1}{c}{\scriptsize{(0.566)}} & \mc{1}{c}{\scriptsize{(0.500)}} \\  

    \mc{1}{l}{\scriptsize{Carbon Dioxide}} & \mc{1}{c}{\scriptsize{Mid-30s}} & \mc{1}{c}{\scriptsize{0.810}} & \mc{1}{c}{\scriptsize{0.525}} & \mc{1}{c}{\scriptsize{0.832}} & \mc{1}{c}{\scriptsize{0.265}} & \mc{1}{c}{\scriptsize{0.405}} & \mc{1}{c}{\scriptsize{0.874}} & \mc{1}{c}{\scriptsize{0.737}} & \mc{1}{c}{\scriptsize{0.832}} \\  

     &  & \mc{1}{c}{\scriptsize{\textbf{(0.000)}}} & \mc{1}{c}{\scriptsize{(0.118)}} & \mc{1}{c}{\scriptsize{(0.171)}} & \mc{1}{c}{\scriptsize{(0.408)}} & \mc{1}{c}{\scriptsize{(0.355)}} & \mc{1}{c}{\scriptsize{\textbf{(0.013)}}} & \mc{1}{c}{\scriptsize{\textbf{(0.039)}}} & \mc{1}{c}{\scriptsize{\textbf{(0.039)}}} \\  

    \mc{1}{l}{\scriptsize{AST}} & \mc{1}{c}{\scriptsize{Mid-30s}} & \mc{1}{c}{\scriptsize{-0.064}} & \mc{1}{c}{\scriptsize{-0.697}} & \mc{1}{c}{\scriptsize{1.736}} & \mc{1}{c}{\scriptsize{-1.746}} & \mc{1}{c}{\scriptsize{0.347}} & \mc{1}{c}{\scriptsize{-0.585}} & \mc{1}{c}{\scriptsize{-0.132}} & \mc{1}{c}{\scriptsize{-0.713}} \\  

     &  & \mc{1}{c}{\scriptsize{(0.513)}} & \mc{1}{c}{\scriptsize{(0.632)}} & \mc{1}{c}{\scriptsize{(0.224)}} & \mc{1}{c}{\scriptsize{(0.737)}} & \mc{1}{c}{\scriptsize{(0.434)}} & \mc{1}{c}{\scriptsize{(0.632)}} & \mc{1}{c}{\scriptsize{(0.566)}} & \mc{1}{c}{\scriptsize{(0.566)}} \\  

    \mc{1}{l}{\scriptsize{Urea Nitrogen}} & \mc{1}{c}{\scriptsize{Mid-30s}} & \mc{1}{c}{\scriptsize{0.579}} & \mc{1}{c}{\scriptsize{0.497}} & \mc{1}{c}{\scriptsize{0.868}} & \mc{1}{c}{\scriptsize{0.705}} & \mc{1}{c}{\scriptsize{1.162}} & \mc{1}{c}{\scriptsize{0.351}} & \mc{1}{c}{\scriptsize{0.070}} & \mc{1}{c}{\scriptsize{0.154}} \\  

     &  & \mc{1}{c}{\scriptsize{(0.145)}} & \mc{1}{c}{\scriptsize{(0.224)}} & \mc{1}{c}{\scriptsize{(0.224)}} & \mc{1}{c}{\scriptsize{(0.355)}} & \mc{1}{c}{\scriptsize{(0.184)}} & \mc{1}{c}{\scriptsize{(0.263)}} & \mc{1}{c}{\scriptsize{(0.461)}} & \mc{1}{c}{\scriptsize{(0.355)}} \\  

    \mc{1}{l}{\scriptsize{Globulin}} & \mc{1}{c}{\scriptsize{Mid-30s}} & \mc{1}{c}{\scriptsize{-0.003}} & \mc{1}{c}{\scriptsize{0.039}} & \mc{1}{c}{\scriptsize{0.004}} & \mc{1}{c}{\scriptsize{-0.029}} & \mc{1}{c}{\scriptsize{0.019}} & \mc{1}{c}{\scriptsize{-0.016}} & \mc{1}{c}{\scriptsize{0.039}} & \mc{1}{c}{\scriptsize{0.015}} \\  

     &  & \mc{1}{c}{\scriptsize{(0.500)}} & \mc{1}{c}{\scriptsize{(0.303)}} & \mc{1}{c}{\scriptsize{(0.539)}} & \mc{1}{c}{\scriptsize{(0.553)}} & \mc{1}{c}{\scriptsize{(0.500)}} & \mc{1}{c}{\scriptsize{(0.539)}} & \mc{1}{c}{\scriptsize{(0.303)}} & \mc{1}{c}{\scriptsize{(0.395)}} \\  

    \mc{1}{l}{\scriptsize{Chloride}} & \mc{1}{c}{\scriptsize{Mid-30s}} & \mc{1}{c}{\scriptsize{-0.880}} & \mc{1}{c}{\scriptsize{-0.630}} & \mc{1}{c}{\scriptsize{-1.903}} & \mc{1}{c}{\scriptsize{-1.001}} & \mc{1}{c}{\scriptsize{-1.834}} & \mc{1}{c}{\scriptsize{-0.547}} & \mc{1}{c}{\scriptsize{-0.400}} & \mc{1}{c}{\scriptsize{-0.574}} \\  

     &  & \mc{1}{c}{\scriptsize{(0.961)}} & \mc{1}{c}{\scriptsize{(0.868)}} & \mc{1}{c}{\scriptsize{(1.000)}} & \mc{1}{c}{\scriptsize{(0.895)}} & \mc{1}{c}{\scriptsize{(1.000)}} & \mc{1}{c}{\scriptsize{(0.855)}} & \mc{1}{c}{\scriptsize{(0.763)}} & \mc{1}{c}{\scriptsize{(0.882)}} \\  

    \mc{1}{l}{\scriptsize{Glucose}} & \mc{1}{c}{\scriptsize{Mid-30s}} & \mc{1}{c}{\scriptsize{1.083}} & \mc{1}{c}{\scriptsize{4.971}} & \mc{1}{c}{\scriptsize{8.349}} & \mc{1}{c}{\scriptsize{1.869}} & \mc{1}{c}{\scriptsize{9.773}} & \mc{1}{c}{\scriptsize{-1.029}} & \mc{1}{c}{\scriptsize{5.969}} & \mc{1}{c}{\scriptsize{-0.456}} \\  

     &  & \mc{1}{c}{\scriptsize{(0.487)}} & \mc{1}{c}{\scriptsize{(0.276)}} & \mc{1}{c}{\scriptsize{(0.184)}} & \mc{1}{c}{\scriptsize{(0.526)}} & \mc{1}{c}{\scriptsize{\textbf{(0.092)}}} & \mc{1}{c}{\scriptsize{(0.618)}} & \mc{1}{c}{\scriptsize{(0.224)}} & \mc{1}{c}{\scriptsize{(0.566)}} \\  

    \mc{1}{l}{\scriptsize{Potassium}} & \mc{1}{c}{\scriptsize{Mid-30s}} & \mc{1}{c}{\scriptsize{0.099}} & \mc{1}{c}{\scriptsize{0.135}} & \mc{1}{c}{\scriptsize{0.102}} & \mc{1}{c}{\scriptsize{0.050}} & \mc{1}{c}{\scriptsize{0.130}} & \mc{1}{c}{\scriptsize{0.106}} & \mc{1}{c}{\scriptsize{0.152}} & \mc{1}{c}{\scriptsize{0.146}} \\  

     &  & \mc{1}{c}{\scriptsize{\textbf{(0.079)}}} & \mc{1}{c}{\scriptsize{\textbf{(0.013)}}} & \mc{1}{c}{\scriptsize{(0.171)}} & \mc{1}{c}{\scriptsize{(0.342)}} & \mc{1}{c}{\scriptsize{(0.118)}} & \mc{1}{c}{\scriptsize{\textbf{(0.053)}}} & \mc{1}{c}{\scriptsize{\textbf{(0.013)}}} & \mc{1}{c}{\scriptsize{\textbf{(0.013)}}} \\  

    \mc{1}{l}{\scriptsize{Creatinine}} & \mc{1}{c}{\scriptsize{Mid-30s}} & \mc{1}{c}{\scriptsize{0.073}} & \mc{1}{c}{\scriptsize{0.002}} & \mc{1}{c}{\scriptsize{0.174}} & \mc{1}{c}{\scriptsize{0.021}} & \mc{1}{c}{\scriptsize{0.151}} & \mc{1}{c}{\scriptsize{0.049}} & \mc{1}{c}{\scriptsize{-0.019}} & \mc{1}{c}{\scriptsize{0.007}} \\  

     &  & \mc{1}{c}{\scriptsize{\textbf{(0.079)}}} & \mc{1}{c}{\scriptsize{(0.421)}} & \mc{1}{c}{\scriptsize{\textbf{(0.000)}}} & \mc{1}{c}{\scriptsize{(0.368)}} & \mc{1}{c}{\scriptsize{\textbf{(0.000)}}} & \mc{1}{c}{\scriptsize{(0.118)}} & \mc{1}{c}{\scriptsize{(0.618)}} & \mc{1}{c}{\scriptsize{(0.447)}} \\  

    \mc{1}{l}{\scriptsize{Calcium}} & \mc{1}{c}{\scriptsize{Mid-30s}} & \mc{1}{c}{\scriptsize{0.038}} & \mc{1}{c}{\scriptsize{-0.032}} & \mc{1}{c}{\scriptsize{0.178}} & \mc{1}{c}{\scriptsize{0.135}} & \mc{1}{c}{\scriptsize{0.177}} & \mc{1}{c}{\scriptsize{0.006}} & \mc{1}{c}{\scriptsize{-0.067}} & \mc{1}{c}{\scriptsize{-0.026}} \\  

     &  & \mc{1}{c}{\scriptsize{(0.303)}} & \mc{1}{c}{\scriptsize{(0.711)}} & \mc{1}{c}{\scriptsize{(0.158)}} & \mc{1}{c}{\scriptsize{\textbf{(0.079)}}} & \mc{1}{c}{\scriptsize{(0.158)}} & \mc{1}{c}{\scriptsize{(0.447)}} & \mc{1}{c}{\scriptsize{(0.895)}} & \mc{1}{c}{\scriptsize{(0.671)}} \\  

    \mc{1}{l}{\scriptsize{Bilirubin}} & \mc{1}{c}{\scriptsize{Mid-30s}} & \mc{1}{c}{\scriptsize{0.111}} & \mc{1}{c}{\scriptsize{0.101}} & \mc{1}{c}{\scriptsize{0.158}} & \mc{1}{c}{\scriptsize{0.118}} & \mc{1}{c}{\scriptsize{0.157}} & \mc{1}{c}{\scriptsize{0.101}} & \mc{1}{c}{\scriptsize{0.093}} & \mc{1}{c}{\scriptsize{0.108}} \\  

     &  & \mc{1}{c}{\scriptsize{\textbf{(0.000)}}} & \mc{1}{c}{\scriptsize{\textbf{(0.013)}}} & \mc{1}{c}{\scriptsize{\textbf{(0.000)}}} & \mc{1}{c}{\scriptsize{\textbf{(0.026)}}} & \mc{1}{c}{\scriptsize{\textbf{(0.000)}}} & \mc{1}{c}{\scriptsize{\textbf{(0.000)}}} & \mc{1}{c}{\scriptsize{\textbf{(0.026)}}} & \mc{1}{c}{\scriptsize{\textbf{(0.013)}}} \\  

    \mc{1}{l}{\scriptsize{Alkaline Phosp}} & \mc{1}{c}{\scriptsize{Mid-30s}} & \mc{1}{c}{\scriptsize{-0.301}} & \mc{1}{c}{\scriptsize{1.522}} & \mc{1}{c}{\scriptsize{-4.368}} & \mc{1}{c}{\scriptsize{-11.066}} & \mc{1}{c}{\scriptsize{-3.745}} & \mc{1}{c}{\scriptsize{0.292}} & \mc{1}{c}{\scriptsize{4.263}} & \mc{1}{c}{\scriptsize{1.354}} \\  

     &  & \mc{1}{c}{\scriptsize{(0.553)}} & \mc{1}{c}{\scriptsize{(0.368)}} & \mc{1}{c}{\scriptsize{(0.737)}} & \mc{1}{c}{\scriptsize{(0.855)}} & \mc{1}{c}{\scriptsize{(0.711)}} & \mc{1}{c}{\scriptsize{(0.526)}} & \mc{1}{c}{\scriptsize{(0.250)}} & \mc{1}{c}{\scriptsize{(0.382)}} \\  

    \mc{1}{l}{\scriptsize{Protein}} & \mc{1}{c}{\scriptsize{Mid-30s}} & \mc{1}{c}{\scriptsize{0.110}} & \mc{1}{c}{\scriptsize{0.084}} & \mc{1}{c}{\scriptsize{0.146}} & \mc{1}{c}{\scriptsize{0.076}} & \mc{1}{c}{\scriptsize{0.132}} & \mc{1}{c}{\scriptsize{0.097}} & \mc{1}{c}{\scriptsize{0.080}} & \mc{1}{c}{\scriptsize{0.074}} \\  

     &  & \mc{1}{c}{\scriptsize{\textbf{(0.053)}}} & \mc{1}{c}{\scriptsize{(0.184)}} & \mc{1}{c}{\scriptsize{(0.263)}} & \mc{1}{c}{\scriptsize{(0.303)}} & \mc{1}{c}{\scriptsize{(0.250)}} & \mc{1}{c}{\scriptsize{(0.105)}} & \mc{1}{c}{\scriptsize{(0.250)}} & \mc{1}{c}{\scriptsize{(0.197)}} \\  

  \bottomrule
  \end{tabular}
	\end{table} 

	\begin{table}[H]
     \caption{Treatment Effects on Harter Importance, Pooled Sample}
     \label{table:abccare_rslt_pooled_cat7}
	  \begin{tabular}{cccccccccc}
  \toprule

    \scriptsize{Variable} & \scriptsize{Age} & \scriptsize{(1)} & \scriptsize{(2)} & \scriptsize{(3)} & \scriptsize{(4)} & \scriptsize{(5)} & \scriptsize{(6)} & \scriptsize{(7)} & \scriptsize{(8)} \\ 
    \midrule  

    \mc{1}{l}{\scriptsize{Employed}} & \mc{1}{c}{\scriptsize{30}} & \mc{1}{c}{\scriptsize{0.125}} & \mc{1}{c}{\scriptsize{0.132}} & \mc{1}{c}{\scriptsize{0.164}} & \mc{1}{c}{\scriptsize{0.128}} & \mc{1}{c}{\scriptsize{0.205}} & \mc{1}{c}{\scriptsize{0.111}} & \mc{1}{c}{\scriptsize{0.148}} & \mc{1}{c}{\scriptsize{0.163}} \\  

     &  & \mc{1}{c}{\scriptsize{\textbf{(0.039)}}} & \mc{1}{c}{\scriptsize{\textbf{(0.026)}}} & \mc{1}{c}{\scriptsize{(0.105)}} & \mc{1}{c}{\scriptsize{(0.158)}} & \mc{1}{c}{\scriptsize{\textbf{(0.053)}}} & \mc{1}{c}{\scriptsize{\textbf{(0.066)}}} & \mc{1}{c}{\scriptsize{\textbf{(0.039)}}} & \mc{1}{c}{\scriptsize{\textbf{(0.000)}}} \\  

    \mc{1}{l}{\scriptsize{Labor Income}} & \mc{1}{c}{\scriptsize{21}} & \mc{1}{c}{\scriptsize{167}} & \mc{1}{c}{\scriptsize{-579}} & \mc{1}{c}{\scriptsize{1,577}} & \mc{1}{c}{\scriptsize{1,469}} & \mc{1}{c}{\scriptsize{1,749}} & \mc{1}{c}{\scriptsize{-429}} & \mc{1}{c}{\scriptsize{-1,211}} & \mc{1}{c}{\scriptsize{-1,527}} \\  

     &  & \mc{1}{c}{\scriptsize{(0.382)}} & \mc{1}{c}{\scriptsize{(0.566)}} & \mc{1}{c}{\scriptsize{(0.289)}} & \mc{1}{c}{\scriptsize{(0.395)}} & \mc{1}{c}{\scriptsize{(0.368)}} & \mc{1}{c}{\scriptsize{(0.566)}} & \mc{1}{c}{\scriptsize{(0.697)}} & \mc{1}{c}{\scriptsize{(0.776)}} \\  

     & \mc{1}{c}{\scriptsize{30}} & \mc{1}{c}{\scriptsize{12,377}} & \mc{1}{c}{\scriptsize{10,752}} & \mc{1}{c}{\scriptsize{17,677}} & \mc{1}{c}{\scriptsize{13,888}} & \mc{1}{c}{\scriptsize{21,198}} & \mc{1}{c}{\scriptsize{10,847}} & \mc{1}{c}{\scriptsize{9,790}} & \mc{1}{c}{\scriptsize{11,232}} \\  

     &  & \mc{1}{c}{\scriptsize{\textbf{(0.026)}}} & \mc{1}{c}{\scriptsize{(0.158)}} & \mc{1}{c}{\scriptsize{\textbf{(0.013)}}} & \mc{1}{c}{\scriptsize{(0.132)}} & \mc{1}{c}{\scriptsize{\textbf{(0.026)}}} & \mc{1}{c}{\scriptsize{\textbf{(0.092)}}} & \mc{1}{c}{\scriptsize{(0.132)}} & \mc{1}{c}{\scriptsize{(0.145)}} \\  

    \mc{1}{l}{\scriptsize{Public-Transfer Income}} & \mc{1}{c}{\scriptsize{21}} & \mc{1}{c}{\scriptsize{-728}} & \mc{1}{c}{\scriptsize{-923}} & \mc{1}{c}{\scriptsize{-247}} & \mc{1}{c}{\scriptsize{-1,161}} & \mc{1}{c}{\scriptsize{-1,625}} & \mc{1}{c}{\scriptsize{-1,054}} & \mc{1}{c}{\scriptsize{-824}} & \mc{1}{c}{\scriptsize{-813}} \\  

     &  & \mc{1}{c}{\scriptsize{(0.211)}} & \mc{1}{c}{\scriptsize{(0.158)}} & \mc{1}{c}{\scriptsize{(0.382)}} & \mc{1}{c}{\scriptsize{(0.171)}} & \mc{1}{c}{\scriptsize{(0.171)}} & \mc{1}{c}{\scriptsize{(0.118)}} & \mc{1}{c}{\scriptsize{(0.184)}} & \mc{1}{c}{\scriptsize{(0.197)}} \\  

     & \mc{1}{c}{\scriptsize{30}} & \mc{1}{c}{\scriptsize{-1,832}} & \mc{1}{c}{\scriptsize{-1,005}} & \mc{1}{c}{\scriptsize{-1,613}} & \mc{1}{c}{\scriptsize{-1,480}} & \mc{1}{c}{\scriptsize{-1,600}} & \mc{1}{c}{\scriptsize{-1,483}} & \mc{1}{c}{\scriptsize{-992}} & \mc{1}{c}{\scriptsize{-1,680}} \\  

     &  & \mc{1}{c}{\scriptsize{\textbf{(0.026)}}} & \mc{1}{c}{\scriptsize{(0.171)}} & \mc{1}{c}{\scriptsize{\textbf{(0.079)}}} & \mc{1}{c}{\scriptsize{(0.105)}} & \mc{1}{c}{\scriptsize{(0.118)}} & \mc{1}{c}{\scriptsize{\textbf{(0.079)}}} & \mc{1}{c}{\scriptsize{(0.145)}} & \mc{1}{c}{\scriptsize{\textbf{(0.079)}}} \\  

    \mc{1}{l}{\scriptsize{Employment Factor}} & \mc{1}{c}{\scriptsize{21 to 30}} & \mc{1}{c}{\scriptsize{0.344}} & \mc{1}{c}{\scriptsize{0.267}} & \mc{1}{c}{\scriptsize{0.393}} & \mc{1}{c}{\scriptsize{0.315}} & \mc{1}{c}{\scriptsize{0.423}} & \mc{1}{c}{\scriptsize{0.307}} & \mc{1}{c}{\scriptsize{0.291}} & \mc{1}{c}{\scriptsize{0.313}} \\  

     &  & \mc{1}{c}{\scriptsize{\textbf{(0.000)}}} & \mc{1}{c}{\scriptsize{\textbf{(0.013)}}} & \mc{1}{c}{\scriptsize{\textbf{(0.053)}}} & \mc{1}{c}{\scriptsize{(0.132)}} & \mc{1}{c}{\scriptsize{\textbf{(0.026)}}} & \mc{1}{c}{\scriptsize{\textbf{(0.039)}}} & \mc{1}{c}{\scriptsize{\textbf{(0.026)}}} & \mc{1}{c}{\scriptsize{\textbf{(0.026)}}} \\ 
    \midrule  

    \mc{2}{l}{\scriptsize{\% of Pos. TE ($H_0$: $\le$ 50\%)}} & \mc{1}{c}{\scriptsize{100}} & \mc{1}{c}{\scriptsize{83}} & \mc{1}{c}{\scriptsize{100}} & \mc{1}{c}{\scriptsize{100}} & \mc{1}{c}{\scriptsize{100}} & \mc{1}{c}{\scriptsize{83}} & \mc{1}{c}{\scriptsize{83}} & \mc{1}{c}{\scriptsize{83}} \\  

     &  & \mc{1}{c}{\scriptsize{\textbf{(0.000)}}} & \mc{1}{c}{\scriptsize{\textbf{(0.000)}}} & \mc{1}{c}{\scriptsize{\textbf{(0.000)}}} & \mc{1}{c}{\scriptsize{\textbf{(0.000)}}} & \mc{1}{c}{\scriptsize{\textbf{(0.000)}}} & \mc{1}{c}{\scriptsize{\textbf{(0.000)}}} & \mc{1}{c}{\scriptsize{\textbf{(0.000)}}} & \mc{1}{c}{\scriptsize{\textbf{(0.000)}}} \\  

    \mc{2}{l}{\scriptsize{\% of Pos. TE ($H_0$: $\le$ 10\% $|$ 10\% Significance)}} & \mc{1}{c}{\scriptsize{50}} & \mc{1}{c}{\scriptsize{50}} & \mc{1}{c}{\scriptsize{67}} & \mc{1}{c}{\scriptsize{0}} & \mc{1}{c}{\scriptsize{83}} & \mc{1}{c}{\scriptsize{50}} & \mc{1}{c}{\scriptsize{33}} & \mc{1}{c}{\scriptsize{50}} \\  

     &  & \mc{1}{c}{\scriptsize{\textbf{(0.066)}}} & \mc{1}{c}{\scriptsize{\textbf{(0.013)}}} & \mc{1}{c}{\scriptsize{\textbf{(0.013)}}} & \mc{1}{c}{\scriptsize{(0.579)}} & \mc{1}{c}{\scriptsize{\textbf{(0.000)}}} & \mc{1}{c}{\scriptsize{\textbf{(0.026)}}} & \mc{1}{c}{\scriptsize{\textbf{(0.026)}}} & \mc{1}{c}{\scriptsize{\textbf{(0.013)}}} \\  

  \bottomrule
  \end{tabular}
	\end{table} 

	\begin{table}[H]
     \caption{Treatment Effects on Achenbach Behavior, Pooled Sample}
     \label{table:abccare_rslt_pooled_cat8}
	  \begin{tabular}{cccccccccc}
  \toprule

    \scriptsize{Variable} & \scriptsize{Age} & \scriptsize{(1)} & \scriptsize{(2)} & \scriptsize{(3)} & \scriptsize{(4)} & \scriptsize{(5)} & \scriptsize{(6)} & \scriptsize{(7)} & \scriptsize{(8)} \\ 
    \midrule  

    \mc{1}{l}{\scriptsize{Total Felony Arrests}} & \mc{1}{c}{\scriptsize{Mid-30s}} & \mc{1}{c}{\scriptsize{0.045}} & \mc{1}{c}{\scriptsize{0.237}} & \mc{1}{c}{\scriptsize{-0.132}} & \mc{1}{c}{\scriptsize{0.227}} & \mc{1}{c}{\scriptsize{0.205}} & \mc{1}{c}{\scriptsize{0.112}} & \mc{1}{c}{\scriptsize{0.224}} & \mc{1}{c}{\scriptsize{0.185}} \\  

     &  & \mc{1}{c}{\scriptsize{(0.566)}} & \mc{1}{c}{\scriptsize{(0.737)}} & \mc{1}{c}{\scriptsize{(0.342)}} & \mc{1}{c}{\scriptsize{(0.566)}} & \mc{1}{c}{\scriptsize{(0.658)}} & \mc{1}{c}{\scriptsize{(0.645)}} & \mc{1}{c}{\scriptsize{(0.671)}} & \mc{1}{c}{\scriptsize{(0.671)}} \\  

    \mc{1}{l}{\scriptsize{Total Misdemeanor Arrests}} & \mc{1}{c}{\scriptsize{Mid-30s}} & \mc{1}{c}{\scriptsize{-0.689}} & \mc{1}{c}{\scriptsize{-0.481}} & \mc{1}{c}{\scriptsize{-1.445}} & \mc{1}{c}{\scriptsize{-1.198}} & \mc{1}{c}{\scriptsize{-1.277}} & \mc{1}{c}{\scriptsize{-0.546}} & \mc{1}{c}{\scriptsize{-0.293}} & \mc{1}{c}{\scriptsize{-0.310}} \\  

     &  & \mc{1}{c}{\scriptsize{\textbf{(0.039)}}} & \mc{1}{c}{\scriptsize{(0.197)}} & \mc{1}{c}{\scriptsize{\textbf{(0.079)}}} & \mc{1}{c}{\scriptsize{(0.171)}} & \mc{1}{c}{\scriptsize{(0.158)}} & \mc{1}{c}{\scriptsize{(0.105)}} & \mc{1}{c}{\scriptsize{(0.276)}} & \mc{1}{c}{\scriptsize{(0.211)}} \\  

    \mc{1}{l}{\scriptsize{Total Years Incarcerated}} & \mc{1}{c}{\scriptsize{30}} & \mc{1}{c}{\scriptsize{0.168}} & \mc{1}{c}{\scriptsize{0.187}} & \mc{1}{c}{\scriptsize{0.280}} & \mc{1}{c}{\scriptsize{0.325}} & \mc{1}{c}{\scriptsize{0.350}} & \mc{1}{c}{\scriptsize{0.153}} & \mc{1}{c}{\scriptsize{0.176}} & \mc{1}{c}{\scriptsize{0.215}} \\  

     &  & \mc{1}{c}{\scriptsize{(0.895)}} & \mc{1}{c}{\scriptsize{(0.895)}} & \mc{1}{c}{\scriptsize{(1.000)}} & \mc{1}{c}{\scriptsize{(1.000)}} & \mc{1}{c}{\scriptsize{(1.000)}} & \mc{1}{c}{\scriptsize{(0.868)}} & \mc{1}{c}{\scriptsize{(0.855)}} & \mc{1}{c}{\scriptsize{(0.908)}} \\  

    \mc{1}{l}{\scriptsize{Crime Factor}} & \mc{1}{c}{\scriptsize{30 to Mid-30s}} & \mc{1}{c}{\scriptsize{0.054}} & \mc{1}{c}{\scriptsize{0.072}} & \mc{1}{c}{\scriptsize{0.012}} & \mc{1}{c}{\scriptsize{0.045}} & \mc{1}{c}{\scriptsize{0.051}} & \mc{1}{c}{\scriptsize{0.076}} & \mc{1}{c}{\scriptsize{0.091}} & \mc{1}{c}{\scriptsize{0.131}} \\  

     &  & \mc{1}{c}{\scriptsize{(0.632)}} & \mc{1}{c}{\scriptsize{(0.684)}} & \mc{1}{c}{\scriptsize{(0.408)}} & \mc{1}{c}{\scriptsize{(0.474)}} & \mc{1}{c}{\scriptsize{(0.500)}} & \mc{1}{c}{\scriptsize{(0.645)}} & \mc{1}{c}{\scriptsize{(0.711)}} & \mc{1}{c}{\scriptsize{(0.750)}} \\  

  \bottomrule
  \end{tabular}
	\end{table} 

	\begin{table}[H]
     \caption{Treatment Effects on Achenbach Symptom T Score (Reported by Mother), Pooled Sample}
     \label{table:abccare_rslt_pooled_cat9}
	  \begin{tabular}{cccccccccc}
  \toprule

    \scriptsize{Variable} & \scriptsize{Age} & \scriptsize{(1)} & \scriptsize{(2)} & \scriptsize{(3)} & \scriptsize{(4)} & \scriptsize{(5)} & \scriptsize{(6)} & \scriptsize{(7)} & \scriptsize{(8)} \\ 
    \midrule  

    \mc{1}{l}{\scriptsize{Cig. Smoked per day last month}} & \mc{1}{c}{\scriptsize{30}} & \mc{1}{c}{\scriptsize{0.033}} & \mc{1}{c}{\scriptsize{0.321}} & \mc{1}{c}{\scriptsize{-0.826}} & \mc{1}{c}{\scriptsize{-0.150}} & \mc{1}{c}{\scriptsize{-0.802}} & \mc{1}{c}{\scriptsize{0.434}} & \mc{1}{c}{\scriptsize{0.576}} & \mc{1}{c}{\scriptsize{0.434}} \\  

     &  & \mc{1}{c}{\scriptsize{(0.461)}} & \mc{1}{c}{\scriptsize{(0.618)}} & \mc{1}{c}{\scriptsize{(0.250)}} & \mc{1}{c}{\scriptsize{(0.434)}} & \mc{1}{c}{\scriptsize{(0.237)}} & \mc{1}{c}{\scriptsize{(0.645)}} & \mc{1}{c}{\scriptsize{(0.684)}} & \mc{1}{c}{\scriptsize{(0.658)}} \\  

    \mc{1}{l}{\scriptsize{Days drank alcohol last month}} & \mc{1}{c}{\scriptsize{30}} & \mc{1}{c}{\scriptsize{0.244}} & \mc{1}{c}{\scriptsize{0.419}} & \mc{1}{c}{\scriptsize{-0.156}} & \mc{1}{c}{\scriptsize{-0.700}} & \mc{1}{c}{\scriptsize{0.113}} & \mc{1}{c}{\scriptsize{0.207}} & \mc{1}{c}{\scriptsize{0.618}} & \mc{1}{c}{\scriptsize{0.630}} \\  

     &  & \mc{1}{c}{\scriptsize{(0.645)}} & \mc{1}{c}{\scriptsize{(0.618)}} & \mc{1}{c}{\scriptsize{(0.447)}} & \mc{1}{c}{\scriptsize{(0.342)}} & \mc{1}{c}{\scriptsize{(0.526)}} & \mc{1}{c}{\scriptsize{(0.566)}} & \mc{1}{c}{\scriptsize{(0.645)}} & \mc{1}{c}{\scriptsize{(0.658)}} \\  

    \mc{1}{l}{\scriptsize{Days binge drank alcohol last month}} & \mc{1}{c}{\scriptsize{30}} & \mc{1}{c}{\scriptsize{0.085}} & \mc{1}{c}{\scriptsize{0.412}} & \mc{1}{c}{\scriptsize{-0.267}} & \mc{1}{c}{\scriptsize{-0.218}} & \mc{1}{c}{\scriptsize{-0.126}} & \mc{1}{c}{\scriptsize{0.151}} & \mc{1}{c}{\scriptsize{0.674}} & \mc{1}{c}{\scriptsize{0.393}} \\  

     &  & \mc{1}{c}{\scriptsize{(0.632)}} & \mc{1}{c}{\scriptsize{(0.803)}} & \mc{1}{c}{\scriptsize{(0.395)}} & \mc{1}{c}{\scriptsize{(0.408)}} & \mc{1}{c}{\scriptsize{(0.434)}} & \mc{1}{c}{\scriptsize{(0.632)}} & \mc{1}{c}{\scriptsize{(0.882)}} & \mc{1}{c}{\scriptsize{(0.776)}} \\  

    \mc{1}{l}{\scriptsize{Self-reported drug user}} & \mc{1}{c}{\scriptsize{Mid-30s}} & \mc{1}{c}{\scriptsize{-0.142}} & \mc{1}{c}{\scriptsize{-0.183}} & \mc{1}{c}{\scriptsize{-0.253}} & \mc{1}{c}{\scriptsize{-0.303}} & \mc{1}{c}{\scriptsize{-0.274}} & \mc{1}{c}{\scriptsize{-0.090}} & \mc{1}{c}{\scriptsize{-0.122}} & \mc{1}{c}{\scriptsize{-0.114}} \\  

     &  & \mc{1}{c}{\scriptsize{\textbf{(0.026)}}} & \mc{1}{c}{\scriptsize{\textbf{(0.053)}}} & \mc{1}{c}{\scriptsize{\textbf{(0.013)}}} & \mc{1}{c}{\scriptsize{\textbf{(0.039)}}} & \mc{1}{c}{\scriptsize{\textbf{(0.026)}}} & \mc{1}{c}{\scriptsize{(0.197)}} & \mc{1}{c}{\scriptsize{(0.158)}} & \mc{1}{c}{\scriptsize{(0.118)}} \\  

    \mc{1}{l}{\scriptsize{Substance Use Factor}} & \mc{1}{c}{\scriptsize{30 to Mid-30s}} & \mc{1}{c}{\scriptsize{0.187}} & \mc{1}{c}{\scriptsize{0.213}} & \mc{1}{c}{\scriptsize{0.335}} & \mc{1}{c}{\scriptsize{0.330}} & \mc{1}{c}{\scriptsize{0.339}} & \mc{1}{c}{\scriptsize{0.149}} & \mc{1}{c}{\scriptsize{0.201}} & \mc{1}{c}{\scriptsize{0.202}} \\  

     &  & \mc{1}{c}{\scriptsize{(0.842)}} & \mc{1}{c}{\scriptsize{(0.921)}} & \mc{1}{c}{\scriptsize{(0.987)}} & \mc{1}{c}{\scriptsize{(0.908)}} & \mc{1}{c}{\scriptsize{(0.987)}} & \mc{1}{c}{\scriptsize{(0.724)}} & \mc{1}{c}{\scriptsize{(0.855)}} & \mc{1}{c}{\scriptsize{(0.855)}} \\ 
    \midrule  

    \mc{2}{l}{\scriptsize{\% of Pos. TE ($H_0$: $\le$ 50\%)}} & \mc{1}{c}{\scriptsize{20}} & \mc{1}{c}{\scriptsize{20}} & \mc{1}{c}{\scriptsize{80}} & \mc{1}{c}{\scriptsize{80}} & \mc{1}{c}{\scriptsize{60}} & \mc{1}{c}{\scriptsize{20}} & \mc{1}{c}{\scriptsize{20}} & \mc{1}{c}{\scriptsize{20}} \\  

     &  & \mc{1}{c}{\scriptsize{(0.961)}} & \mc{1}{c}{\scriptsize{(0.987)}} & \mc{1}{c}{\scriptsize{\textbf{(0.026)}}} & \mc{1}{c}{\scriptsize{\textbf{(0.053)}}} & \mc{1}{c}{\scriptsize{(0.250)}} & \mc{1}{c}{\scriptsize{(0.868)}} & \mc{1}{c}{\scriptsize{(0.908)}} & \mc{1}{c}{\scriptsize{(0.908)}} \\  

    \mc{2}{l}{\scriptsize{\% of Pos. TE ($H_0$: $\le$ 10\% $|$ 10\% Significance)}} & \mc{1}{c}{\scriptsize{20}} & \mc{1}{c}{\scriptsize{20}} & \mc{1}{c}{\scriptsize{0}} & \mc{1}{c}{\scriptsize{20}} & \mc{1}{c}{\scriptsize{20}} & \mc{1}{c}{\scriptsize{0}} & \mc{1}{c}{\scriptsize{0}} & \mc{1}{c}{\scriptsize{0}} \\  

     &  & \mc{1}{c}{\scriptsize{(0.447)}} & \mc{1}{c}{\scriptsize{\textbf{(0.053)}}} & \mc{1}{c}{\scriptsize{(1.000)}} & \mc{1}{c}{\scriptsize{\textbf{(0.039)}}} & \mc{1}{c}{\scriptsize{(0.421)}} & \mc{1}{c}{\scriptsize{(1.000)}} & \mc{1}{c}{\scriptsize{(1.000)}} & \mc{1}{c}{\scriptsize{(1.000)}} \\  

  \bottomrule
  \end{tabular}
	\end{table} 

	\begin{table}[H]
     \caption{Treatment Effects on Achenbach Symptom T Score (Reported by Teacher), Pooled Sample}
     \label{table:abccare_rslt_pooled_cat10}
	  \begin{tabular}{cccccccccc}
  \toprule

    \scriptsize{Variable} & \scriptsize{Age} & \scriptsize{(1)} & \scriptsize{(2)} & \scriptsize{(3)} & \scriptsize{(4)} & \scriptsize{(5)} & \scriptsize{(6)} & \scriptsize{(7)} & \scriptsize{(8)} \\ 
    \midrule  

    \mc{1}{l}{\scriptsize{Self-reported Health}} & \mc{1}{c}{\scriptsize{30}} & \mc{1}{c}{\scriptsize{-0.066}} & \mc{1}{c}{\scriptsize{0.007}} & \mc{1}{c}{\scriptsize{-0.369}} & \mc{1}{c}{\scriptsize{-0.264}} & \mc{1}{c}{\scriptsize{-0.330}} & \mc{1}{c}{\scriptsize{-0.009}} & \mc{1}{c}{\scriptsize{0.004}} & \mc{1}{c}{\scriptsize{0.036}} \\  

     &  & \mc{1}{c}{\scriptsize{(0.316)}} & \mc{1}{c}{\scriptsize{(0.487)}} & \mc{1}{c}{\scriptsize{(0.105)}} & \mc{1}{c}{\scriptsize{(0.211)}} & \mc{1}{c}{\scriptsize{(0.184)}} & \mc{1}{c}{\scriptsize{(0.434)}} & \mc{1}{c}{\scriptsize{(0.553)}} & \mc{1}{c}{\scriptsize{(0.553)}} \\  

     & \mc{1}{c}{\scriptsize{Mid-30s}} & \mc{1}{c}{\scriptsize{0.079}} & \mc{1}{c}{\scriptsize{-0.019}} & \mc{1}{c}{\scriptsize{-0.429}} & \mc{1}{c}{\scriptsize{-0.442}} & \mc{1}{c}{\scriptsize{-0.498}} & \mc{1}{c}{\scriptsize{0.179}} & \mc{1}{c}{\scriptsize{0.113}} & \mc{1}{c}{\scriptsize{0.066}} \\  

     &  & \mc{1}{c}{\scriptsize{(0.684)}} & \mc{1}{c}{\scriptsize{(0.447)}} & \mc{1}{c}{\scriptsize{\textbf{(0.039)}}} & \mc{1}{c}{\scriptsize{\textbf{(0.066)}}} & \mc{1}{c}{\scriptsize{\textbf{(0.039)}}} & \mc{1}{c}{\scriptsize{(0.789)}} & \mc{1}{c}{\scriptsize{(0.658)}} & \mc{1}{c}{\scriptsize{(0.697)}} \\  

    \mc{1}{l}{\scriptsize{Self-reported Health Factor}} & \mc{1}{c}{\scriptsize{30 to Mid-30s}} & \mc{1}{c}{\scriptsize{-0.012}} & \mc{1}{c}{\scriptsize{-0.052}} & \mc{1}{c}{\scriptsize{-0.083}} & \mc{1}{c}{\scriptsize{-0.093}} & \mc{1}{c}{\scriptsize{-0.129}} & \mc{1}{c}{\scriptsize{-0.012}} & \mc{1}{c}{\scriptsize{-0.021}} & \mc{1}{c}{\scriptsize{-0.051}} \\  

     &  & \mc{1}{c}{\scriptsize{(0.434)}} & \mc{1}{c}{\scriptsize{(0.342)}} & \mc{1}{c}{\scriptsize{(0.303)}} & \mc{1}{c}{\scriptsize{(0.329)}} & \mc{1}{c}{\scriptsize{(0.224)}} & \mc{1}{c}{\scriptsize{(0.447)}} & \mc{1}{c}{\scriptsize{(0.474)}} & \mc{1}{c}{\scriptsize{(0.329)}} \\ 
    \midrule  

    \mc{2}{l}{\scriptsize{\% of Pos. TE ($H_0$: $\le$ 50\%)}} & \mc{1}{c}{\scriptsize{67}} & \mc{1}{c}{\scriptsize{67}} & \mc{1}{c}{\scriptsize{100}} & \mc{1}{c}{\scriptsize{100}} & \mc{1}{c}{\scriptsize{100}} & \mc{1}{c}{\scriptsize{67}} & \mc{1}{c}{\scriptsize{33}} & \mc{1}{c}{\scriptsize{33}} \\  

     &  & \mc{1}{c}{\scriptsize{(0.171)}} & \mc{1}{c}{\scriptsize{(0.145)}} & \mc{1}{c}{\scriptsize{\textbf{(0.000)}}} & \mc{1}{c}{\scriptsize{\textbf{(0.000)}}} & \mc{1}{c}{\scriptsize{\textbf{(0.000)}}} & \mc{1}{c}{\scriptsize{(0.329)}} & \mc{1}{c}{\scriptsize{(0.882)}} & \mc{1}{c}{\scriptsize{(0.921)}} \\  

    \mc{2}{l}{\scriptsize{\% of Pos. TE ($H_0$: $\le$ 10\% $|$ 10\% Significance)}} & \mc{1}{c}{\scriptsize{0}} & \mc{1}{c}{\scriptsize{0}} & \mc{1}{c}{\scriptsize{33}} & \mc{1}{c}{\scriptsize{33}} & \mc{1}{c}{\scriptsize{33}} & \mc{1}{c}{\scriptsize{0}} & \mc{1}{c}{\scriptsize{0}} & \mc{1}{c}{\scriptsize{0}} \\  

     &  & \mc{1}{c}{\scriptsize{(1.000)}} & \mc{1}{c}{\scriptsize{(1.000)}} & \mc{1}{c}{\scriptsize{(0.105)}} & \mc{1}{c}{\scriptsize{\textbf{(0.039)}}} & \mc{1}{c}{\scriptsize{(0.105)}} & \mc{1}{c}{\scriptsize{(1.000)}} & \mc{1}{c}{\scriptsize{(1.000)}} & \mc{1}{c}{\scriptsize{(1.000)}} \\  

  \bottomrule
  \end{tabular}
	\end{table} 

	\begin{table}[H]
     \caption{Treatment Effects on Child Assessment Schedule (CAS), Pooled Sample}
     \label{table:abccare_rslt_pooled_cat11}
	  \begin{tabular}{cccccccccc}
  \toprule

    \scriptsize{Variable} & \scriptsize{Age} & \scriptsize{(1)} & \scriptsize{(2)} & \scriptsize{(3)} & \scriptsize{(4)} & \scriptsize{(5)} & \scriptsize{(6)} & \scriptsize{(7)} & \scriptsize{(8)} \\ 
    \midrule  

    \mc{1}{l}{\scriptsize{Systolic Blood Pressure (mm Hg)}} & \mc{1}{c}{\scriptsize{Mid-30s}} & \mc{1}{c}{\scriptsize{-5.625}} & \mc{1}{c}{\scriptsize{-10.849}} & \mc{1}{c}{\scriptsize{5.375}} & \mc{1}{c}{\scriptsize{6.809}} & \mc{1}{c}{\scriptsize{4.383}} & \mc{1}{c}{\scriptsize{-9.438}} & \mc{1}{c}{\scriptsize{-16.448}} & \mc{1}{c}{\scriptsize{-15.882}} \\  

     &  & \mc{1}{c}{\scriptsize{\textbf{(0.092)}}} & \mc{1}{c}{\scriptsize{\textbf{(0.026)}}} & \mc{1}{c}{\scriptsize{(0.882)}} & \mc{1}{c}{\scriptsize{(0.803)}} & \mc{1}{c}{\scriptsize{(0.829)}} & \mc{1}{c}{\scriptsize{\textbf{(0.066)}}} & \mc{1}{c}{\scriptsize{\textbf{(0.000)}}} & \mc{1}{c}{\scriptsize{\textbf{(0.013)}}} \\  

    \mc{1}{l}{\scriptsize{Diastolic Blood Pressure (mm Hg)}} & \mc{1}{c}{\scriptsize{Mid-30s}} & \mc{1}{c}{\scriptsize{-5.312}} & \mc{1}{c}{\scriptsize{-7.745}} & \mc{1}{c}{\scriptsize{-1.424}} & \mc{1}{c}{\scriptsize{0.519}} & \mc{1}{c}{\scriptsize{-1.637}} & \mc{1}{c}{\scriptsize{-7.219}} & \mc{1}{c}{\scriptsize{-9.589}} & \mc{1}{c}{\scriptsize{-10.696}} \\  

     &  & \mc{1}{c}{\scriptsize{\textbf{(0.039)}}} & \mc{1}{c}{\scriptsize{\textbf{(0.026)}}} & \mc{1}{c}{\scriptsize{(0.316)}} & \mc{1}{c}{\scriptsize{(0.579)}} & \mc{1}{c}{\scriptsize{(0.303)}} & \mc{1}{c}{\scriptsize{\textbf{(0.026)}}} & \mc{1}{c}{\scriptsize{\textbf{(0.053)}}} & \mc{1}{c}{\scriptsize{\textbf{(0.000)}}} \\  

    \mc{1}{l}{\scriptsize{Prehypertension}} & \mc{1}{c}{\scriptsize{Mid-30s}} & \mc{1}{c}{\scriptsize{-0.176}} & \mc{1}{c}{\scriptsize{-0.181}} & \mc{1}{c}{\scriptsize{-0.049}} & \mc{1}{c}{\scriptsize{-0.055}} & \mc{1}{c}{\scriptsize{-0.060}} & \mc{1}{c}{\scriptsize{-0.240}} & \mc{1}{c}{\scriptsize{-0.268}} & \mc{1}{c}{\scriptsize{-0.280}} \\  

     &  & \mc{1}{c}{\scriptsize{\textbf{(0.000)}}} & \mc{1}{c}{\scriptsize{\textbf{(0.053)}}} & \mc{1}{c}{\scriptsize{(0.382)}} & \mc{1}{c}{\scriptsize{(0.355)}} & \mc{1}{c}{\scriptsize{(0.342)}} & \mc{1}{c}{\scriptsize{\textbf{(0.000)}}} & \mc{1}{c}{\scriptsize{\textbf{(0.000)}}} & \mc{1}{c}{\scriptsize{\textbf{(0.000)}}} \\  

    \mc{1}{l}{\scriptsize{Hypertension}} & \mc{1}{c}{\scriptsize{Mid-30s}} & \mc{1}{c}{\scriptsize{-0.036}} & \mc{1}{c}{\scriptsize{-0.097}} & \mc{1}{c}{\scriptsize{0.083}} & \mc{1}{c}{\scriptsize{0.143}} & \mc{1}{c}{\scriptsize{0.045}} & \mc{1}{c}{\scriptsize{-0.083}} & \mc{1}{c}{\scriptsize{-0.151}} & \mc{1}{c}{\scriptsize{-0.187}} \\  

     &  & \mc{1}{c}{\scriptsize{(0.316)}} & \mc{1}{c}{\scriptsize{(0.158)}} & \mc{1}{c}{\scriptsize{(0.724)}} & \mc{1}{c}{\scriptsize{(0.763)}} & \mc{1}{c}{\scriptsize{(0.658)}} & \mc{1}{c}{\scriptsize{(0.171)}} & \mc{1}{c}{\scriptsize{(0.118)}} & \mc{1}{c}{\scriptsize{\textbf{(0.066)}}} \\  

    \mc{1}{l}{\scriptsize{Hypertension Factor}} & \mc{1}{c}{\scriptsize{Mid-30s}} & \mc{1}{c}{\scriptsize{-0.293}} & \mc{1}{c}{\scriptsize{-0.475}} & \mc{1}{c}{\scriptsize{0.089}} & \mc{1}{c}{\scriptsize{0.185}} & \mc{1}{c}{\scriptsize{0.046}} & \mc{1}{c}{\scriptsize{-0.447}} & \mc{1}{c}{\scriptsize{-0.670}} & \mc{1}{c}{\scriptsize{-0.703}} \\  

     &  & \mc{1}{c}{\scriptsize{\textbf{(0.053)}}} & \mc{1}{c}{\scriptsize{\textbf{(0.039)}}} & \mc{1}{c}{\scriptsize{(0.632)}} & \mc{1}{c}{\scriptsize{(0.684)}} & \mc{1}{c}{\scriptsize{(0.539)}} & \mc{1}{c}{\scriptsize{\textbf{(0.026)}}} & \mc{1}{c}{\scriptsize{\textbf{(0.000)}}} & \mc{1}{c}{\scriptsize{\textbf{(0.000)}}} \\ 
    \midrule  

    \mc{2}{l}{\scriptsize{\% of Pos. TE ($H_0$: $\le$ 50\%)}} & \mc{1}{c}{\scriptsize{100}} & \mc{1}{c}{\scriptsize{100}} & \mc{1}{c}{\scriptsize{40}} & \mc{1}{c}{\scriptsize{20}} & \mc{1}{c}{\scriptsize{40}} & \mc{1}{c}{\scriptsize{100}} & \mc{1}{c}{\scriptsize{100}} & \mc{1}{c}{\scriptsize{100}} \\  

     &  & \mc{1}{c}{\scriptsize{\textbf{(0.000)}}} & \mc{1}{c}{\scriptsize{\textbf{(0.000)}}} & \mc{1}{c}{\scriptsize{(0.526)}} & \mc{1}{c}{\scriptsize{(0.737)}} & \mc{1}{c}{\scriptsize{(0.566)}} & \mc{1}{c}{\scriptsize{\textbf{(0.000)}}} & \mc{1}{c}{\scriptsize{\textbf{(0.000)}}} & \mc{1}{c}{\scriptsize{\textbf{(0.000)}}} \\  

    \mc{2}{l}{\scriptsize{\% of Pos. TE ($H_0$: $\le$ 10\% $|$ 10\% Significance)}} & \mc{1}{c}{\scriptsize{60}} & \mc{1}{c}{\scriptsize{80}} & \mc{1}{c}{\scriptsize{0}} & \mc{1}{c}{\scriptsize{0}} & \mc{1}{c}{\scriptsize{0}} & \mc{1}{c}{\scriptsize{80}} & \mc{1}{c}{\scriptsize{80}} & \mc{1}{c}{\scriptsize{100}} \\  

     &  & \mc{1}{c}{\scriptsize{\textbf{(0.039)}}} & \mc{1}{c}{\scriptsize{\textbf{(0.000)}}} & \mc{1}{c}{\scriptsize{(1.000)}} & \mc{1}{c}{\scriptsize{(1.000)}} & \mc{1}{c}{\scriptsize{(1.000)}} & \mc{1}{c}{\scriptsize{\textbf{(0.000)}}} & \mc{1}{c}{\scriptsize{\textbf{(0.000)}}} & \mc{1}{c}{\scriptsize{\textbf{(0.000)}}} \\  

  \bottomrule
  \end{tabular}
	\end{table} 

	\begin{table}[H]
     \caption{Treatment Effects on Mother's Income, Pooled Sample}
     \label{table:abccare_rslt_pooled_cat12}
	  \begin{tabular}{cccccccccc}
  \toprule

    \scriptsize{Variable} & \scriptsize{Age} & \scriptsize{(1)} & \scriptsize{(2)} & \scriptsize{(3)} & \scriptsize{(4)} & \scriptsize{(5)} & \scriptsize{(6)} & \scriptsize{(7)} & \scriptsize{(8)} \\ 
    \midrule  

    \mc{1}{l}{\scriptsize{High-Density Lipoprotein Chol. (mg/dL)}} & \mc{1}{c}{\scriptsize{Mid-30s}} & \mc{1}{c}{\scriptsize{3.872}} & \mc{1}{c}{\scriptsize{4.790}} & \mc{1}{c}{\scriptsize{5.806}} & \mc{1}{c}{\scriptsize{3.305}} & \mc{1}{c}{\scriptsize{5.518}} & \mc{1}{c}{\scriptsize{2.964}} & \mc{1}{c}{\scriptsize{5.364}} & \mc{1}{c}{\scriptsize{4.456}} \\  

     &  & \mc{1}{c}{\scriptsize{\textbf{(0.092)}}} & \mc{1}{c}{\scriptsize{\textbf{(0.053)}}} & \mc{1}{c}{\scriptsize{\textbf{(0.013)}}} & \mc{1}{c}{\scriptsize{(0.211)}} & \mc{1}{c}{\scriptsize{\textbf{(0.092)}}} & \mc{1}{c}{\scriptsize{(0.184)}} & \mc{1}{c}{\scriptsize{\textbf{(0.039)}}} & \mc{1}{c}{\scriptsize{\textbf{(0.053)}}} \\  

    \mc{1}{l}{\scriptsize{Dyslipidemia}} & \mc{1}{c}{\scriptsize{Mid-30s}} & \mc{1}{c}{\scriptsize{0.013}} & \mc{1}{c}{\scriptsize{-0.003}} & \mc{1}{c}{\scriptsize{0.035}} & \mc{1}{c}{\scriptsize{0.082}} & \mc{1}{c}{\scriptsize{0.031}} & \mc{1}{c}{\scriptsize{0.032}} & \mc{1}{c}{\scriptsize{-0.009}} & \mc{1}{c}{\scriptsize{0.010}} \\  

     &  & \mc{1}{c}{\scriptsize{(0.579)}} & \mc{1}{c}{\scriptsize{(0.500)}} & \mc{1}{c}{\scriptsize{(0.553)}} & \mc{1}{c}{\scriptsize{(0.671)}} & \mc{1}{c}{\scriptsize{(0.566)}} & \mc{1}{c}{\scriptsize{(0.645)}} & \mc{1}{c}{\scriptsize{(0.474)}} & \mc{1}{c}{\scriptsize{(0.553)}} \\  

    \mc{1}{l}{\scriptsize{Cholesterol Factor}} & \mc{1}{c}{\scriptsize{Mid-30s}} & \mc{1}{c}{\scriptsize{-0.097}} & \mc{1}{c}{\scriptsize{-0.141}} & \mc{1}{c}{\scriptsize{-0.127}} & \mc{1}{c}{\scriptsize{-0.000}} & \mc{1}{c}{\scriptsize{-0.124}} & \mc{1}{c}{\scriptsize{-0.049}} & \mc{1}{c}{\scriptsize{-0.165}} & \mc{1}{c}{\scriptsize{-0.117}} \\  

     &  & \mc{1}{c}{\scriptsize{(0.289)}} & \mc{1}{c}{\scriptsize{(0.158)}} & \mc{1}{c}{\scriptsize{(0.237)}} & \mc{1}{c}{\scriptsize{(0.513)}} & \mc{1}{c}{\scriptsize{(0.237)}} & \mc{1}{c}{\scriptsize{(0.395)}} & \mc{1}{c}{\scriptsize{(0.132)}} & \mc{1}{c}{\scriptsize{(0.276)}} \\ 
    \midrule  

    \mc{2}{l}{\scriptsize{\% of Pos. TE ($H_0$: $\le$ 50\%)}} & \mc{1}{c}{\scriptsize{67}} & \mc{1}{c}{\scriptsize{100}} & \mc{1}{c}{\scriptsize{67}} & \mc{1}{c}{\scriptsize{67}} & \mc{1}{c}{\scriptsize{67}} & \mc{1}{c}{\scriptsize{67}} & \mc{1}{c}{\scriptsize{100}} & \mc{1}{c}{\scriptsize{67}} \\  

     &  & \mc{1}{c}{\scriptsize{(0.487)}} & \mc{1}{c}{\scriptsize{\textbf{(0.000)}}} & \mc{1}{c}{\scriptsize{(0.329)}} & \mc{1}{c}{\scriptsize{(0.329)}} & \mc{1}{c}{\scriptsize{(0.316)}} & \mc{1}{c}{\scriptsize{(0.382)}} & \mc{1}{c}{\scriptsize{\textbf{(0.000)}}} & \mc{1}{c}{\scriptsize{(0.434)}} \\  

    \mc{2}{l}{\scriptsize{\% of Pos. TE ($H_0$: $\le$ 10\% $|$ 10\% Significance)}} & \mc{1}{c}{\scriptsize{33}} & \mc{1}{c}{\scriptsize{33}} & \mc{1}{c}{\scriptsize{33}} & \mc{1}{c}{\scriptsize{0}} & \mc{1}{c}{\scriptsize{33}} & \mc{1}{c}{\scriptsize{0}} & \mc{1}{c}{\scriptsize{33}} & \mc{1}{c}{\scriptsize{33}} \\  

     &  & \mc{1}{c}{\scriptsize{(0.197)}} & \mc{1}{c}{\scriptsize{(0.171)}} & \mc{1}{c}{\scriptsize{\textbf{(0.079)}}} & \mc{1}{c}{\scriptsize{(1.000)}} & \mc{1}{c}{\scriptsize{(0.118)}} & \mc{1}{c}{\scriptsize{(0.237)}} & \mc{1}{c}{\scriptsize{(0.224)}} & \mc{1}{c}{\scriptsize{(0.158)}} \\  

  \bottomrule
  \end{tabular}
	\end{table} 

	\begin{table}[H]
     \caption{Treatment Effects on Parental Labor Income, Pooled Sample}
     \label{table:abccare_rslt_pooled_cat13}
	  \begin{tabular}{cccccccccc}
  \toprule

    \scriptsize{Variable} & \scriptsize{Age} & \scriptsize{(1)} & \scriptsize{(2)} & \scriptsize{(3)} & \scriptsize{(4)} & \scriptsize{(5)} & \scriptsize{(6)} & \scriptsize{(7)} & \scriptsize{(8)} \\ 
    \midrule  

    \mc{1}{l}{\scriptsize{Ear: Pinna}} & \mc{1}{c}{\scriptsize{Mid-30s}} &  &  &  &  &  &  &  &  \\  

     &  &  &  &  &  &  &  &  &  \\  

    \mc{1}{l}{\scriptsize{Ear: Tympanic Membrane}} & \mc{1}{c}{\scriptsize{Mid-30s}} &  &  &  &  &  &  &  &  \\  

     &  &  &  &  &  &  &  &  &  \\  

    \mc{1}{l}{\scriptsize{Ear: Auditory Canal}} & \mc{1}{c}{\scriptsize{Mid-30s}} & \mc{1}{c}{\scriptsize{-0.005}} & \mc{1}{c}{\scriptsize{-0.028}} & \mc{1}{c}{\scriptsize{0.043}} & \mc{1}{c}{\scriptsize{0.041}} & \mc{1}{c}{\scriptsize{0.043}} & \mc{1}{c}{\scriptsize{-0.020}} & \mc{1}{c}{\scriptsize{-0.056}} & \mc{1}{c}{\scriptsize{-0.021}} \\  

     &  & \mc{1}{c}{\scriptsize{(0.447)}} & \mc{1}{c}{\scriptsize{(0.237)}} & \mc{1}{c}{\scriptsize{(0.750)}} & \mc{1}{c}{\scriptsize{(0.658)}} & \mc{1}{c}{\scriptsize{(0.750)}} & \mc{1}{c}{\scriptsize{(0.316)}} & \mc{1}{c}{\scriptsize{(0.145)}} & \mc{1}{c}{\scriptsize{(0.316)}} \\  

    \mc{1}{l}{\scriptsize{Eye: Pupil}} & \mc{1}{c}{\scriptsize{Mid-30s}} &  &  &  &  &  &  &  &  \\  

     &  &  &  &  &  &  &  &  &  \\  

    \mc{1}{l}{\scriptsize{Eye: Eyeball}} & \mc{1}{c}{\scriptsize{Mid-30s}} & \mc{1}{c}{\scriptsize{-0.024}} & \mc{1}{c}{\scriptsize{-0.014}} & \mc{1}{c}{\scriptsize{-0.111}} & \mc{1}{c}{\scriptsize{-0.111}} & \mc{1}{c}{\scriptsize{-0.106}} &  &  &  \\  

     &  & \mc{1}{c}{\scriptsize{\textbf{(0.092)}}} & \mc{1}{c}{\scriptsize{(0.132)}} & \mc{1}{c}{\scriptsize{\textbf{(0.079)}}} & \mc{1}{c}{\scriptsize{\textbf{(0.053)}}} & \mc{1}{c}{\scriptsize{\textbf{(0.079)}}} &  &  &  \\  

    \mc{1}{l}{\scriptsize{Eye: Fundi}} & \mc{1}{c}{\scriptsize{Mid-30s}} & \mc{1}{c}{\scriptsize{-0.050}} & \mc{1}{c}{\scriptsize{-0.088}} & \mc{1}{c}{\scriptsize{-0.090}} & \mc{1}{c}{\scriptsize{-0.118}} & \mc{1}{c}{\scriptsize{-0.083}} & \mc{1}{c}{\scriptsize{-0.041}} & \mc{1}{c}{\scriptsize{-0.089}} & \mc{1}{c}{\scriptsize{-0.095}} \\  

     &  & \mc{1}{c}{\scriptsize{\textbf{(0.066)}}} & \mc{1}{c}{\scriptsize{\textbf{(0.039)}}} & \mc{1}{c}{\scriptsize{(0.118)}} & \mc{1}{c}{\scriptsize{(0.105)}} & \mc{1}{c}{\scriptsize{(0.158)}} & \mc{1}{c}{\scriptsize{(0.171)}} & \mc{1}{c}{\scriptsize{\textbf{(0.092)}}} & \mc{1}{c}{\scriptsize{\textbf{(0.053)}}} \\  

    \mc{1}{l}{\scriptsize{Eye: Sclera/Conjunctiva}} & \mc{1}{c}{\scriptsize{Mid-30s}} &  &  &  &  &  &  &  &  \\  

     &  &  &  &  &  &  &  &  &  \\  

    \mc{1}{l}{\scriptsize{Ear: Mastoid}} & \mc{1}{c}{\scriptsize{Mid-30s}} &  &  &  &  &  &  &  &  \\  

     &  &  &  &  &  &  &  &  &  \\  

  \bottomrule
  \end{tabular}
	\end{table} 

	\begin{table}[H]
     \caption{Treatment Effects on Parental Public Transfer Income, Pooled Sample}
     \label{table:abccare_rslt_pooled_cat14}
	  \begin{tabular}{cccccccccc}
  \toprule

    \scriptsize{Variable} & \scriptsize{Age} & \scriptsize{(1)} & \scriptsize{(2)} & \scriptsize{(3)} & \scriptsize{(4)} & \scriptsize{(5)} & \scriptsize{(6)} & \scriptsize{(7)} & \scriptsize{(8)} \\ 
    \midrule  

    \mc{1}{l}{\scriptsize{Vitamin D Deficiency}} & \mc{1}{c}{\scriptsize{Mid-30s}} & \mc{1}{c}{\scriptsize{-0.153}} & \mc{1}{c}{\scriptsize{-0.069}} & \mc{1}{c}{\scriptsize{-0.264}} & \mc{1}{c}{\scriptsize{-0.085}} & \mc{1}{c}{\scriptsize{-0.233}} & \mc{1}{c}{\scriptsize{-0.118}} & \mc{1}{c}{\scriptsize{-0.031}} & \mc{1}{c}{\scriptsize{-0.078}} \\  

     &  & \mc{1}{c}{\scriptsize{\textbf{(0.039)}}} & \mc{1}{c}{\scriptsize{(0.237)}} & \mc{1}{c}{\scriptsize{\textbf{(0.000)}}} & \mc{1}{c}{\scriptsize{(0.342)}} & \mc{1}{c}{\scriptsize{\textbf{(0.000)}}} & \mc{1}{c}{\scriptsize{(0.132)}} & \mc{1}{c}{\scriptsize{(0.408)}} & \mc{1}{c}{\scriptsize{(0.263)}} \\  

  \bottomrule
  \end{tabular}
	\end{table} 

	\begin{table}[H]
     \caption{Treatment Effects on Adoption, Pooled Sample}
     \label{table:abccare_rslt_pooled_cat15}
	  \begin{tabular}{cccccccccc}
  \toprule

    \scriptsize{Variable} & \scriptsize{Age} & \scriptsize{(1)} & \scriptsize{(2)} & \scriptsize{(3)} & \scriptsize{(4)} & \scriptsize{(5)} & \scriptsize{(6)} & \scriptsize{(7)} & \scriptsize{(8)} \\ 
    \midrule  

    \mc{1}{l}{\scriptsize{Measured BMI}} & \mc{1}{c}{\scriptsize{Mid-30s}} & \mc{1}{c}{\scriptsize{0.999}} & \mc{1}{c}{\scriptsize{3.094}} & \mc{1}{c}{\scriptsize{-0.202}} & \mc{1}{c}{\scriptsize{1.254}} & \mc{1}{c}{\scriptsize{0.710}} & \mc{1}{c}{\scriptsize{1.072}} & \mc{1}{c}{\scriptsize{3.548}} & \mc{1}{c}{\scriptsize{1.827}} \\  

     &  & \mc{1}{c}{\scriptsize{(0.750)}} & \mc{1}{c}{\scriptsize{(0.921)}} & \mc{1}{c}{\scriptsize{(0.500)}} & \mc{1}{c}{\scriptsize{(0.671)}} & \mc{1}{c}{\scriptsize{(0.605)}} & \mc{1}{c}{\scriptsize{(0.711)}} & \mc{1}{c}{\scriptsize{(0.961)}} & \mc{1}{c}{\scriptsize{(0.816)}} \\  

    \mc{1}{l}{\scriptsize{Obesity}} & \mc{1}{c}{\scriptsize{Mid-30s}} & \mc{1}{c}{\scriptsize{-0.050}} & \mc{1}{c}{\scriptsize{0.063}} & \mc{1}{c}{\scriptsize{-0.256}} & \mc{1}{c}{\scriptsize{-0.144}} & \mc{1}{c}{\scriptsize{-0.144}} & \mc{1}{c}{\scriptsize{-0.013}} & \mc{1}{c}{\scriptsize{0.113}} & \mc{1}{c}{\scriptsize{0.011}} \\  

     &  & \mc{1}{c}{\scriptsize{(0.329)}} & \mc{1}{c}{\scriptsize{(0.724)}} & \mc{1}{c}{\scriptsize{\textbf{(0.000)}}} & \mc{1}{c}{\scriptsize{(0.118)}} & \mc{1}{c}{\scriptsize{(0.197)}} & \mc{1}{c}{\scriptsize{(0.461)}} & \mc{1}{c}{\scriptsize{(0.803)}} & \mc{1}{c}{\scriptsize{(0.513)}} \\  

    \mc{1}{l}{\scriptsize{Severe Obesity}} & \mc{1}{c}{\scriptsize{Mid-30s}} & \mc{1}{c}{\scriptsize{-0.126}} & \mc{1}{c}{\scriptsize{-0.018}} & \mc{1}{c}{\scriptsize{-0.093}} & \mc{1}{c}{\scriptsize{-0.069}} & \mc{1}{c}{\scriptsize{-0.065}} & \mc{1}{c}{\scriptsize{-0.147}} & \mc{1}{c}{\scriptsize{-0.011}} & \mc{1}{c}{\scriptsize{-0.107}} \\  

     &  & \mc{1}{c}{\scriptsize{\textbf{(0.092)}}} & \mc{1}{c}{\scriptsize{(0.382)}} & \mc{1}{c}{\scriptsize{(0.289)}} & \mc{1}{c}{\scriptsize{(0.355)}} & \mc{1}{c}{\scriptsize{(0.303)}} & \mc{1}{c}{\scriptsize{(0.105)}} & \mc{1}{c}{\scriptsize{(0.382)}} & \mc{1}{c}{\scriptsize{(0.171)}} \\  

    \mc{1}{l}{\scriptsize{Waist-hip Ratio}} & \mc{1}{c}{\scriptsize{Mid-30s}} & \mc{1}{c}{\scriptsize{-0.006}} & \mc{1}{c}{\scriptsize{0.000}} & \mc{1}{c}{\scriptsize{-0.037}} & \mc{1}{c}{\scriptsize{-0.047}} & \mc{1}{c}{\scriptsize{-0.031}} & \mc{1}{c}{\scriptsize{0.003}} & \mc{1}{c}{\scriptsize{0.012}} & \mc{1}{c}{\scriptsize{0.006}} \\  

     &  & \mc{1}{c}{\scriptsize{(0.382)}} & \mc{1}{c}{\scriptsize{(0.474)}} & \mc{1}{c}{\scriptsize{(0.171)}} & \mc{1}{c}{\scriptsize{(0.197)}} & \mc{1}{c}{\scriptsize{(0.224)}} & \mc{1}{c}{\scriptsize{(0.618)}} & \mc{1}{c}{\scriptsize{(0.776)}} & \mc{1}{c}{\scriptsize{(0.711)}} \\  

    \mc{1}{l}{\scriptsize{Abdominal Obesity}} & \mc{1}{c}{\scriptsize{Mid-30s}} & \mc{1}{c}{\scriptsize{-0.091}} & \mc{1}{c}{\scriptsize{-0.021}} & \mc{1}{c}{\scriptsize{-0.230}} & \mc{1}{c}{\scriptsize{-0.139}} & \mc{1}{c}{\scriptsize{-0.157}} & \mc{1}{c}{\scriptsize{-0.041}} & \mc{1}{c}{\scriptsize{0.031}} & \mc{1}{c}{\scriptsize{-0.030}} \\  

     &  & \mc{1}{c}{\scriptsize{(0.158)}} & \mc{1}{c}{\scriptsize{(0.487)}} & \mc{1}{c}{\scriptsize{\textbf{(0.013)}}} & \mc{1}{c}{\scriptsize{(0.105)}} & \mc{1}{c}{\scriptsize{(0.118)}} & \mc{1}{c}{\scriptsize{(0.342)}} & \mc{1}{c}{\scriptsize{(0.671)}} & \mc{1}{c}{\scriptsize{(0.474)}} \\  

    \mc{1}{l}{\scriptsize{Framingham Risk Score}} & \mc{1}{c}{\scriptsize{Mid-30s}} & \mc{1}{c}{\scriptsize{0.348}} & \mc{1}{c}{\scriptsize{-0.399}} & \mc{1}{c}{\scriptsize{0.948}} & \mc{1}{c}{\scriptsize{0.299}} & \mc{1}{c}{\scriptsize{0.902}} & \mc{1}{c}{\scriptsize{0.351}} & \mc{1}{c}{\scriptsize{-0.704}} & \mc{1}{c}{\scriptsize{0.088}} \\  

     &  & \mc{1}{c}{\scriptsize{(0.711)}} & \mc{1}{c}{\scriptsize{(0.289)}} & \mc{1}{c}{\scriptsize{(0.921)}} & \mc{1}{c}{\scriptsize{(0.605)}} & \mc{1}{c}{\scriptsize{(0.908)}} & \mc{1}{c}{\scriptsize{(0.724)}} & \mc{1}{c}{\scriptsize{(0.145)}} & \mc{1}{c}{\scriptsize{(0.500)}} \\  

    \mc{1}{l}{\scriptsize{Obesity Factor}} & \mc{1}{c}{\scriptsize{Mid-30s}} & \mc{1}{c}{\scriptsize{-0.022}} & \mc{1}{c}{\scriptsize{0.162}} & \mc{1}{c}{\scriptsize{-0.268}} & \mc{1}{c}{\scriptsize{-0.138}} & \mc{1}{c}{\scriptsize{-0.134}} & \mc{1}{c}{\scriptsize{0.033}} & \mc{1}{c}{\scriptsize{0.250}} & \mc{1}{c}{\scriptsize{0.077}} \\  

     &  & \mc{1}{c}{\scriptsize{(0.434)}} & \mc{1}{c}{\scriptsize{(0.789)}} & \mc{1}{c}{\scriptsize{(0.211)}} & \mc{1}{c}{\scriptsize{(0.303)}} & \mc{1}{c}{\scriptsize{(0.342)}} & \mc{1}{c}{\scriptsize{(0.579)}} & \mc{1}{c}{\scriptsize{(0.829)}} & \mc{1}{c}{\scriptsize{(0.618)}} \\ 
    \midrule  

    \mc{2}{l}{\scriptsize{\% of Pos. TE ($H_0$: $\le$ 50\%)}} & \mc{1}{c}{\scriptsize{71}} & \mc{1}{c}{\scriptsize{43}} & \mc{1}{c}{\scriptsize{86}} & \mc{1}{c}{\scriptsize{71}} & \mc{1}{c}{\scriptsize{71}} & \mc{1}{c}{\scriptsize{43}} & \mc{1}{c}{\scriptsize{29}} & \mc{1}{c}{\scriptsize{29}} \\  

     &  & \mc{1}{c}{\scriptsize{\textbf{(0.079)}}} & \mc{1}{c}{\scriptsize{(0.632)}} & \mc{1}{c}{\scriptsize{\textbf{(0.000)}}} & \mc{1}{c}{\scriptsize{\textbf{(0.092)}}} & \mc{1}{c}{\scriptsize{\textbf{(0.039)}}} & \mc{1}{c}{\scriptsize{(0.579)}} & \mc{1}{c}{\scriptsize{(0.750)}} & \mc{1}{c}{\scriptsize{(0.776)}} \\  

    \mc{2}{l}{\scriptsize{\% of Pos. TE ($H_0$: $\le$ 10\% $|$ 10\% Significance)}} & \mc{1}{c}{\scriptsize{14}} & \mc{1}{c}{\scriptsize{0}} & \mc{1}{c}{\scriptsize{29}} & \mc{1}{c}{\scriptsize{0}} & \mc{1}{c}{\scriptsize{0}} & \mc{1}{c}{\scriptsize{14}} & \mc{1}{c}{\scriptsize{14}} & \mc{1}{c}{\scriptsize{0}} \\  

     &  & \mc{1}{c}{\scriptsize{(0.316)}} & \mc{1}{c}{\scriptsize{(1.000)}} & \mc{1}{c}{\scriptsize{(0.237)}} & \mc{1}{c}{\scriptsize{(0.526)}} & \mc{1}{c}{\scriptsize{(0.579)}} & \mc{1}{c}{\scriptsize{(0.461)}} & \mc{1}{c}{\scriptsize{(0.421)}} & \mc{1}{c}{\scriptsize{(1.000)}} \\  

  \bottomrule
  \end{tabular}
	\end{table} 

	\begin{table}[H]
     \caption{Treatment Effects on Childhood Household Income, Pooled Sample}
     \label{table:abccare_rslt_pooled_cat16}
	  \begin{tabular}{cccccccccc}
  \toprule

    \scriptsize{Variable} & \scriptsize{Age} & \scriptsize{(1)} & \scriptsize{(2)} & \scriptsize{(3)} & \scriptsize{(4)} & \scriptsize{(5)} & \scriptsize{(6)} & \scriptsize{(7)} & \scriptsize{(8)} \\ 
    \midrule  

    \mc{1}{l}{\scriptsize{Somatization}} & \mc{1}{c}{\scriptsize{21}} & \mc{1}{c}{\scriptsize{-0.015}} & \mc{1}{c}{\scriptsize{-0.068}} & \mc{1}{c}{\scriptsize{-0.118}} & \mc{1}{c}{\scriptsize{-0.240}} & \mc{1}{c}{\scriptsize{-0.182}} & \mc{1}{c}{\scriptsize{0.012}} & \mc{1}{c}{\scriptsize{-0.046}} & \mc{1}{c}{\scriptsize{-0.071}} \\  

     &  & \mc{1}{c}{\scriptsize{(0.461)}} & \mc{1}{c}{\scriptsize{(0.184)}} & \mc{1}{c}{\scriptsize{(0.211)}} & \mc{1}{c}{\scriptsize{\textbf{(0.066)}}} & \mc{1}{c}{\scriptsize{(0.105)}} & \mc{1}{c}{\scriptsize{(0.539)}} & \mc{1}{c}{\scriptsize{(0.263)}} & \mc{1}{c}{\scriptsize{(0.171)}} \\  

     & \mc{1}{c}{\scriptsize{34}} & \mc{1}{c}{\scriptsize{-0.128}} & \mc{1}{c}{\scriptsize{-0.119}} & \mc{1}{c}{\scriptsize{-0.191}} & \mc{1}{c}{\scriptsize{-0.202}} & \mc{1}{c}{\scriptsize{-0.205}} & \mc{1}{c}{\scriptsize{-0.122}} & \mc{1}{c}{\scriptsize{-0.114}} & \mc{1}{c}{\scriptsize{-0.123}} \\  

     &  & \mc{1}{c}{\scriptsize{(0.171)}} & \mc{1}{c}{\scriptsize{(0.184)}} & \mc{1}{c}{\scriptsize{(0.250)}} & \mc{1}{c}{\scriptsize{(0.250)}} & \mc{1}{c}{\scriptsize{(0.237)}} & \mc{1}{c}{\scriptsize{(0.184)}} & \mc{1}{c}{\scriptsize{(0.197)}} & \mc{1}{c}{\scriptsize{(0.145)}} \\  

    \mc{1}{l}{\scriptsize{Depression}} & \mc{1}{c}{\scriptsize{21}} & \mc{1}{c}{\scriptsize{-0.256}} & \mc{1}{c}{\scriptsize{-0.207}} & \mc{1}{c}{\scriptsize{-0.329}} & \mc{1}{c}{\scriptsize{-0.373}} & \mc{1}{c}{\scriptsize{-0.365}} & \mc{1}{c}{\scriptsize{-0.184}} & \mc{1}{c}{\scriptsize{-0.164}} & \mc{1}{c}{\scriptsize{-0.202}} \\  

     &  & \mc{1}{c}{\scriptsize{\textbf{(0.000)}}} & \mc{1}{c}{\scriptsize{\textbf{(0.039)}}} & \mc{1}{c}{\scriptsize{(0.105)}} & \mc{1}{c}{\scriptsize{\textbf{(0.066)}}} & \mc{1}{c}{\scriptsize{\textbf{(0.066)}}} & \mc{1}{c}{\scriptsize{\textbf{(0.079)}}} & \mc{1}{c}{\scriptsize{(0.132)}} & \mc{1}{c}{\scriptsize{\textbf{(0.039)}}} \\  

     & \mc{1}{c}{\scriptsize{34}} & \mc{1}{c}{\scriptsize{-0.165}} & \mc{1}{c}{\scriptsize{-0.203}} & \mc{1}{c}{\scriptsize{-0.061}} & \mc{1}{c}{\scriptsize{0.001}} & \mc{1}{c}{\scriptsize{-0.117}} & \mc{1}{c}{\scriptsize{-0.211}} & \mc{1}{c}{\scriptsize{-0.292}} & \mc{1}{c}{\scriptsize{-0.236}} \\  

     &  & \mc{1}{c}{\scriptsize{(0.184)}} & \mc{1}{c}{\scriptsize{(0.132)}} & \mc{1}{c}{\scriptsize{(0.395)}} & \mc{1}{c}{\scriptsize{(0.421)}} & \mc{1}{c}{\scriptsize{(0.289)}} & \mc{1}{c}{\scriptsize{(0.158)}} & \mc{1}{c}{\scriptsize{\textbf{(0.039)}}} & \mc{1}{c}{\scriptsize{\textbf{(0.079)}}} \\  

    \mc{1}{l}{\scriptsize{Anxiety}} & \mc{1}{c}{\scriptsize{21}} & \mc{1}{c}{\scriptsize{-0.119}} & \mc{1}{c}{\scriptsize{-0.110}} & \mc{1}{c}{\scriptsize{-0.192}} & \mc{1}{c}{\scriptsize{-0.344}} & \mc{1}{c}{\scriptsize{-0.208}} & \mc{1}{c}{\scriptsize{-0.077}} & \mc{1}{c}{\scriptsize{-0.083}} & \mc{1}{c}{\scriptsize{-0.102}} \\  

     &  & \mc{1}{c}{\scriptsize{(0.132)}} & \mc{1}{c}{\scriptsize{(0.132)}} & \mc{1}{c}{\scriptsize{(0.158)}} & \mc{1}{c}{\scriptsize{\textbf{(0.053)}}} & \mc{1}{c}{\scriptsize{(0.158)}} & \mc{1}{c}{\scriptsize{(0.224)}} & \mc{1}{c}{\scriptsize{(0.237)}} & \mc{1}{c}{\scriptsize{(0.158)}} \\  

     & \mc{1}{c}{\scriptsize{34}} & \mc{1}{c}{\scriptsize{-0.256}} & \mc{1}{c}{\scriptsize{-0.259}} & \mc{1}{c}{\scriptsize{-0.224}} & \mc{1}{c}{\scriptsize{-0.217}} & \mc{1}{c}{\scriptsize{-0.262}} & \mc{1}{c}{\scriptsize{-0.280}} & \mc{1}{c}{\scriptsize{-0.286}} & \mc{1}{c}{\scriptsize{-0.300}} \\  

     &  & \mc{1}{c}{\scriptsize{\textbf{(0.039)}}} & \mc{1}{c}{\scriptsize{\textbf{(0.066)}}} & \mc{1}{c}{\scriptsize{(0.171)}} & \mc{1}{c}{\scriptsize{(0.237)}} & \mc{1}{c}{\scriptsize{(0.145)}} & \mc{1}{c}{\scriptsize{\textbf{(0.039)}}} & \mc{1}{c}{\scriptsize{\textbf{(0.026)}}} & \mc{1}{c}{\scriptsize{\textbf{(0.013)}}} \\  

    \mc{1}{l}{\scriptsize{Hostility}} & \mc{1}{c}{\scriptsize{21}} & \mc{1}{c}{\scriptsize{-0.296}} & \mc{1}{c}{\scriptsize{-0.279}} & \mc{1}{c}{\scriptsize{-0.504}} & \mc{1}{c}{\scriptsize{-0.666}} & \mc{1}{c}{\scriptsize{-0.564}} & \mc{1}{c}{\scriptsize{-0.191}} & \mc{1}{c}{\scriptsize{-0.199}} & \mc{1}{c}{\scriptsize{-0.229}} \\  

     &  & \mc{1}{c}{\scriptsize{\textbf{(0.013)}}} & \mc{1}{c}{\scriptsize{\textbf{(0.013)}}} & \mc{1}{c}{\scriptsize{\textbf{(0.013)}}} & \mc{1}{c}{\scriptsize{\textbf{(0.026)}}} & \mc{1}{c}{\scriptsize{\textbf{(0.013)}}} & \mc{1}{c}{\scriptsize{\textbf{(0.079)}}} & \mc{1}{c}{\scriptsize{\textbf{(0.092)}}} & \mc{1}{c}{\scriptsize{\textbf{(0.039)}}} \\  

     & \mc{1}{c}{\scriptsize{34}} & \mc{1}{c}{\scriptsize{-0.145}} & \mc{1}{c}{\scriptsize{-0.126}} & \mc{1}{c}{\scriptsize{-0.090}} & \mc{1}{c}{\scriptsize{-0.057}} & \mc{1}{c}{\scriptsize{-0.129}} & \mc{1}{c}{\scriptsize{-0.174}} & \mc{1}{c}{\scriptsize{-0.144}} & \mc{1}{c}{\scriptsize{-0.150}} \\  

     &  & \mc{1}{c}{\scriptsize{(0.158)}} & \mc{1}{c}{\scriptsize{(0.118)}} & \mc{1}{c}{\scriptsize{(0.342)}} & \mc{1}{c}{\scriptsize{(0.355)}} & \mc{1}{c}{\scriptsize{(0.184)}} & \mc{1}{c}{\scriptsize{(0.118)}} & \mc{1}{c}{\scriptsize{(0.118)}} & \mc{1}{c}{\scriptsize{(0.118)}} \\  

    \mc{1}{l}{\scriptsize{Global Severity Index}} & \mc{1}{c}{\scriptsize{21}} & \mc{1}{c}{\scriptsize{-0.147}} & \mc{1}{c}{\scriptsize{-0.140}} & \mc{1}{c}{\scriptsize{-0.221}} & \mc{1}{c}{\scriptsize{-0.325}} & \mc{1}{c}{\scriptsize{-0.258}} & \mc{1}{c}{\scriptsize{-0.107}} & \mc{1}{c}{\scriptsize{-0.107}} & \mc{1}{c}{\scriptsize{-0.144}} \\  

     &  & \mc{1}{c}{\scriptsize{\textbf{(0.066)}}} & \mc{1}{c}{\scriptsize{(0.118)}} & \mc{1}{c}{\scriptsize{\textbf{(0.053)}}} & \mc{1}{c}{\scriptsize{\textbf{(0.039)}}} & \mc{1}{c}{\scriptsize{\textbf{(0.039)}}} & \mc{1}{c}{\scriptsize{(0.132)}} & \mc{1}{c}{\scriptsize{(0.184)}} & \mc{1}{c}{\scriptsize{\textbf{(0.092)}}} \\  

     & \mc{1}{c}{\scriptsize{34}} & \mc{1}{c}{\scriptsize{-3.294}} & \mc{1}{c}{\scriptsize{-3.487}} & \mc{1}{c}{\scriptsize{-2.858}} & \mc{1}{c}{\scriptsize{-2.506}} & \mc{1}{c}{\scriptsize{-3.502}} & \mc{1}{c}{\scriptsize{-3.674}} & \mc{1}{c}{\scriptsize{-4.149}} & \mc{1}{c}{\scriptsize{-3.952}} \\  

     &  & \mc{1}{c}{\scriptsize{(0.118)}} & \mc{1}{c}{\scriptsize{(0.105)}} & \mc{1}{c}{\scriptsize{(0.276)}} & \mc{1}{c}{\scriptsize{(0.303)}} & \mc{1}{c}{\scriptsize{(0.197)}} & \mc{1}{c}{\scriptsize{\textbf{(0.039)}}} & \mc{1}{c}{\scriptsize{\textbf{(0.039)}}} & \mc{1}{c}{\scriptsize{\textbf{(0.066)}}} \\  

    \mc{1}{l}{\scriptsize{BSI Factor}} & \mc{1}{c}{\scriptsize{21 and 34}} & \mc{1}{c}{\scriptsize{-0.559}} & \mc{1}{c}{\scriptsize{-0.330}} & \mc{1}{c}{\scriptsize{-0.600}} & \mc{1}{c}{\scriptsize{-0.405}} & \mc{1}{c}{\scriptsize{-0.485}} & \mc{1}{c}{\scriptsize{-0.547}} & \mc{1}{c}{\scriptsize{-0.371}} & \mc{1}{c}{\scriptsize{-0.350}} \\  

     &  & \mc{1}{c}{\scriptsize{\textbf{(0.000)}}} & \mc{1}{c}{\scriptsize{\textbf{(0.066)}}} & \mc{1}{c}{\scriptsize{\textbf{(0.066)}}} & \mc{1}{c}{\scriptsize{(0.197)}} & \mc{1}{c}{\scriptsize{(0.105)}} & \mc{1}{c}{\scriptsize{\textbf{(0.000)}}} & \mc{1}{c}{\scriptsize{\textbf{(0.079)}}} & \mc{1}{c}{\scriptsize{\textbf{(0.039)}}} \\  

  \bottomrule
  \end{tabular}
	\end{table} 

	\begin{table}[H]
     \caption{Treatment Effects on Father at Home, Pooled Sample}
     \label{table:abccare_rslt_pooled_cat17}
	\begin{table}[H]
\captionsetup{singlelinecheck=false,justification=centering}
\caption{ABC/CARE Average Treatment Effects, Males and Females \\ Mental Health \label{tab:ate_pooled_apx17}}

  \begin{threeparttable}
  \begin{tabular}{cccccccccc}
  \hline\hline

     &  & \scriptsize{(1)} & \scriptsize{(2)} & \scriptsize{(3)} & \scriptsize{(4)} & \scriptsize{(5)} & \scriptsize{(6)} & \scriptsize{(7)} & \scriptsize{(8)} \\  

     &  &  &  & \mc{3}{c}{\scriptsize{$P=0$}} & \mc{3}{c}{\scriptsize{$P=1$}} \\ 
    \cmidrule(lr){5-7} \cmidrule(lr){8-10} 

    \scriptsize{Variable} & \scriptsize{Age} & \scriptsize{ITT} & \scriptsize{ITT$|X,W$} & \scriptsize{ITT} & \scriptsize{ITT$|X,W$} & \scriptsize{KE$|X,W$} & \scriptsize{ITT} & \scriptsize{ITT$|X,W$} & \scriptsize{KE$|X,W$} \\ 
    \hline  

    \mc{1}{l}{\scriptsize{Somatization}} & \mc{1}{c}{\scriptsize{21}} & \mc{1}{c}{\scriptsize{0.010}} & \mc{1}{c}{\scriptsize{-0.022}} & \mc{1}{c}{\scriptsize{-0.011}} & \mc{1}{c}{\scriptsize{-0.074}} & \mc{1}{c}{\scriptsize{-0.088}} & \mc{1}{c}{\scriptsize{0.019}} & \mc{1}{c}{\scriptsize{0.003}} & \mc{1}{c}{\scriptsize{-0.024}} \\  

     &  & \mc{1}{c}{\scriptsize{(0.569)}} & \mc{1}{c}{\scriptsize{(0.333)}} & \mc{1}{c}{\scriptsize{(0.373)}} & \mc{1}{c}{\scriptsize{(0.333)}} & \mc{1}{c}{\scriptsize{(0.235)}} & \mc{1}{c}{\scriptsize{(0.627)}} & \mc{1}{c}{\scriptsize{(0.510)}} & \mc{1}{c}{\scriptsize{(0.353)}} \\  

     & \mc{1}{c}{\scriptsize{34}} & \mc{1}{c}{\scriptsize{-0.122}} & \mc{1}{c}{\scriptsize{-0.154}} & \mc{1}{c}{\scriptsize{-0.117}} & \mc{1}{c}{\scriptsize{-0.256}} & \mc{1}{c}{\scriptsize{-0.095}} & \mc{1}{c}{\scriptsize{-0.124}} & \mc{1}{c}{\scriptsize{-0.101}} & \mc{1}{c}{\scriptsize{-0.161}} \\  

     &  & \mc{1}{c}{\scriptsize{(0.176)}} & \mc{1}{c}{\scriptsize{(0.196)}} & \mc{1}{c}{\scriptsize{(0.255)}} & \mc{1}{c}{\scriptsize{(0.196)}} & \mc{1}{c}{\scriptsize{(0.216)}} & \mc{1}{c}{\scriptsize{(0.176)}} & \mc{1}{c}{\scriptsize{(0.275)}} & \mc{1}{c}{\scriptsize{\textbf{(0.098)}}} \\  

    \mc{1}{l}{\scriptsize{Depression}} & \mc{1}{c}{\scriptsize{21}} & \mc{1}{c}{\scriptsize{-0.154}} & \mc{1}{c}{\scriptsize{-0.224}} & \mc{1}{c}{\scriptsize{-0.195}} & \mc{1}{c}{\scriptsize{-0.278}} & \mc{1}{c}{\scriptsize{-0.173}} & \mc{1}{c}{\scriptsize{-0.136}} & \mc{1}{c}{\scriptsize{-0.181}} & \mc{1}{c}{\scriptsize{-0.128}} \\  

     &  & \mc{1}{c}{\scriptsize{(0.118)}} & \mc{1}{c}{\scriptsize{\textbf{(0.078)}}} & \mc{1}{c}{\scriptsize{(0.137)}} & \mc{1}{c}{\scriptsize{(0.118)}} & \mc{1}{c}{\scriptsize{(0.176)}} & \mc{1}{c}{\scriptsize{\textbf{(0.098)}}} & \mc{1}{c}{\scriptsize{(0.118)}} & \mc{1}{c}{\scriptsize{(0.157)}} \\  

     & \mc{1}{c}{\scriptsize{34}} & \mc{1}{c}{\scriptsize{-0.164}} & \mc{1}{c}{\scriptsize{-0.134}} & \mc{1}{c}{\scriptsize{0.060}} & \mc{1}{c}{\scriptsize{-0.033}} & \mc{1}{c}{\scriptsize{0.106}} & \mc{1}{c}{\scriptsize{-0.261}} & \mc{1}{c}{\scriptsize{-0.174}} & \mc{1}{c}{\scriptsize{-0.275}} \\  

     &  & \mc{1}{c}{\scriptsize{(0.176)}} & \mc{1}{c}{\scriptsize{(0.235)}} & \mc{1}{c}{\scriptsize{(0.588)}} & \mc{1}{c}{\scriptsize{(0.373)}} & \mc{1}{c}{\scriptsize{(0.451)}} & \mc{1}{c}{\scriptsize{(0.137)}} & \mc{1}{c}{\scriptsize{(0.216)}} & \mc{1}{c}{\scriptsize{\textbf{(0.059)}}} \\  

    \mc{1}{l}{\scriptsize{Anxiety}} & \mc{1}{c}{\scriptsize{21}} & \mc{1}{c}{\scriptsize{-0.046}} & \mc{1}{c}{\scriptsize{-0.102}} & \mc{1}{c}{\scriptsize{-0.112}} & \mc{1}{c}{\scriptsize{-0.165}} & \mc{1}{c}{\scriptsize{-0.091}} & \mc{1}{c}{\scriptsize{-0.017}} & \mc{1}{c}{\scriptsize{-0.068}} & \mc{1}{c}{\scriptsize{-0.002}} \\  

     &  & \mc{1}{c}{\scriptsize{(0.275)}} & \mc{1}{c}{\scriptsize{(0.176)}} & \mc{1}{c}{\scriptsize{(0.275)}} & \mc{1}{c}{\scriptsize{(0.176)}} & \mc{1}{c}{\scriptsize{(0.294)}} & \mc{1}{c}{\scriptsize{(0.392)}} & \mc{1}{c}{\scriptsize{(0.275)}} & \mc{1}{c}{\scriptsize{(0.451)}} \\  

     & \mc{1}{c}{\scriptsize{34}} & \mc{1}{c}{\scriptsize{-0.211}} & \mc{1}{c}{\scriptsize{-0.228}} & \mc{1}{c}{\scriptsize{-0.086}} & \mc{1}{c}{\scriptsize{-0.164}} & \mc{1}{c}{\scriptsize{-0.083}} & \mc{1}{c}{\scriptsize{-0.264}} & \mc{1}{c}{\scriptsize{-0.230}} & \mc{1}{c}{\scriptsize{-0.306}} \\  

     &  & \mc{1}{c}{\scriptsize{(0.118)}} & \mc{1}{c}{\scriptsize{(0.176)}} & \mc{1}{c}{\scriptsize{(0.196)}} & \mc{1}{c}{\scriptsize{(0.235)}} & \mc{1}{c}{\scriptsize{(0.196)}} & \mc{1}{c}{\scriptsize{\textbf{(0.098)}}} & \mc{1}{c}{\scriptsize{(0.176)}} & \mc{1}{c}{\scriptsize{\textbf{(0.039)}}} \\  

    \mc{1}{l}{\scriptsize{Hostility}} & \mc{1}{c}{\scriptsize{21}} & \mc{1}{c}{\scriptsize{-0.218}} & \mc{1}{c}{\scriptsize{-0.302}} & \mc{1}{c}{\scriptsize{-0.335}} & \mc{1}{c}{\scriptsize{-0.441}} & \mc{1}{c}{\scriptsize{-0.313}} & \mc{1}{c}{\scriptsize{-0.166}} & \mc{1}{c}{\scriptsize{-0.236}} & \mc{1}{c}{\scriptsize{-0.178}} \\  

     &  & \mc{1}{c}{\scriptsize{\textbf{(0.059)}}} & \mc{1}{c}{\scriptsize{\textbf{(0.039)}}} & \mc{1}{c}{\scriptsize{\textbf{(0.059)}}} & \mc{1}{c}{\scriptsize{\textbf{(0.039)}}} & \mc{1}{c}{\scriptsize{\textbf{(0.039)}}} & \mc{1}{c}{\scriptsize{\textbf{(0.098)}}} & \mc{1}{c}{\scriptsize{\textbf{(0.059)}}} & \mc{1}{c}{\scriptsize{(0.137)}} \\  

     & \mc{1}{c}{\scriptsize{34}} & \mc{1}{c}{\scriptsize{-0.132}} & \mc{1}{c}{\scriptsize{-0.116}} & \mc{1}{c}{\scriptsize{-0.023}} & \mc{1}{c}{\scriptsize{-0.013}} & \mc{1}{c}{\scriptsize{-0.017}} & \mc{1}{c}{\scriptsize{-0.179}} & \mc{1}{c}{\scriptsize{-0.127}} & \mc{1}{c}{\scriptsize{-0.181}} \\  

     &  & \mc{1}{c}{\scriptsize{(0.216)}} & \mc{1}{c}{\scriptsize{(0.275)}} & \mc{1}{c}{\scriptsize{(0.412)}} & \mc{1}{c}{\scriptsize{(0.392)}} & \mc{1}{c}{\scriptsize{(0.294)}} & \mc{1}{c}{\scriptsize{(0.176)}} & \mc{1}{c}{\scriptsize{(0.294)}} & \mc{1}{c}{\scriptsize{(0.118)}} \\  

    \mc{1}{l}{\scriptsize{Global Severity Index}} & \mc{1}{c}{\scriptsize{21}} & \mc{1}{c}{\scriptsize{-0.078}} & \mc{1}{c}{\scriptsize{-0.126}} & \mc{1}{c}{\scriptsize{-0.091}} & \mc{1}{c}{\scriptsize{-0.138}} & \mc{1}{c}{\scriptsize{-0.086}} & \mc{1}{c}{\scriptsize{-0.072}} & \mc{1}{c}{\scriptsize{-0.108}} & \mc{1}{c}{\scriptsize{-0.076}} \\  

     &  & \mc{1}{c}{\scriptsize{(0.157)}} & \mc{1}{c}{\scriptsize{(0.118)}} & \mc{1}{c}{\scriptsize{(0.176)}} & \mc{1}{c}{\scriptsize{(0.196)}} & \mc{1}{c}{\scriptsize{(0.314)}} & \mc{1}{c}{\scriptsize{(0.137)}} & \mc{1}{c}{\scriptsize{(0.137)}} & \mc{1}{c}{\scriptsize{(0.255)}} \\  

     & \mc{1}{c}{\scriptsize{34}} & \mc{1}{c}{\scriptsize{-2.980}} & \mc{1}{c}{\scriptsize{-3.095}} & \mc{1}{c}{\scriptsize{-0.858}} & \mc{1}{c}{\scriptsize{-2.717}} & \mc{1}{c}{\scriptsize{-0.428}} & \mc{1}{c}{\scriptsize{-3.889}} & \mc{1}{c}{\scriptsize{-3.033}} & \mc{1}{c}{\scriptsize{-4.447}} \\  

     &  & \mc{1}{c}{\scriptsize{(0.157)}} & \mc{1}{c}{\scriptsize{(0.196)}} & \mc{1}{c}{\scriptsize{(0.314)}} & \mc{1}{c}{\scriptsize{(0.255)}} & \mc{1}{c}{\scriptsize{(0.275)}} & \mc{1}{c}{\scriptsize{(0.137)}} & \mc{1}{c}{\scriptsize{(0.176)}} & \mc{1}{c}{\scriptsize{\textbf{(0.059)}}} \\  

    \mc{1}{l}{\scriptsize{BSI Factor}} & \mc{1}{c}{\scriptsize{21 and 34}} & \mc{1}{c}{\scriptsize{-0.488}} & \mc{1}{c}{\scriptsize{-0.525}} & \mc{1}{c}{\scriptsize{-0.454}} & \mc{1}{c}{\scriptsize{-0.760}} & \mc{1}{c}{\scriptsize{-0.207}} & \mc{1}{c}{\scriptsize{-0.503}} & \mc{1}{c}{\scriptsize{-0.445}} & \mc{1}{c}{\scriptsize{-0.335}} \\  

     &  & \mc{1}{c}{\scriptsize{\textbf{(0.000)}}} & \mc{1}{c}{\scriptsize{\textbf{(0.020)}}} & \mc{1}{c}{\scriptsize{\textbf{(0.078)}}} & \mc{1}{c}{\scriptsize{\textbf{(0.020)}}} & \mc{1}{c}{\scriptsize{(0.157)}} & \mc{1}{c}{\scriptsize{\textbf{(0.000)}}} & \mc{1}{c}{\scriptsize{\textbf{(0.020)}}} & \mc{1}{c}{\scriptsize{\textbf{(0.020)}}} \\ 
    \hline  

    \\[0.1cm]
    \mc{2}{l}{\scriptsize{\% of Pos. TE ($H_0$: $\le$ 25\% $|$ 10\% Significance)}} & \mc{1}{c}{\scriptsize{18}} & \mc{1}{c}{\scriptsize{27}} & \mc{1}{c}{\scriptsize{18}} & \mc{1}{c}{\scriptsize{18}} & \mc{1}{c}{\scriptsize{9}} & \mc{1}{c}{\scriptsize{36}} & \mc{1}{c}{\scriptsize{18}} & \mc{1}{c}{\scriptsize{45}} \\  

     &  & \mc{1}{c}{\scriptsize{(0.647)}} & \mc{1}{c}{\scriptsize{(0.373)}} & \mc{1}{c}{\scriptsize{(0.667)}} & \mc{1}{c}{\scriptsize{(0.588)}} & \mc{1}{c}{\scriptsize{(0.647)}} & \mc{1}{c}{\scriptsize{(0.333)}} & \mc{1}{c}{\scriptsize{(0.529)}} & \mc{1}{c}{\scriptsize{(0.216)}} \\  

    \mc{2}{l}{\scriptsize{\% of Pos. TE ($H_0$: $\le$ 50\% $|$ 10\% Significance)}} & \mc{1}{c}{\scriptsize{18}} & \mc{1}{c}{\scriptsize{27}} & \mc{1}{c}{\scriptsize{18}} & \mc{1}{c}{\scriptsize{18}} & \mc{1}{c}{\scriptsize{9}} & \mc{1}{c}{\scriptsize{36}} & \mc{1}{c}{\scriptsize{18}} & \mc{1}{c}{\scriptsize{45}} \\  

     &  & \mc{1}{c}{\scriptsize{(0.882)}} & \mc{1}{c}{\scriptsize{(0.804)}} & \mc{1}{c}{\scriptsize{(0.941)}} & \mc{1}{c}{\scriptsize{(0.941)}} & \mc{1}{c}{\scriptsize{(1.000)}} & \mc{1}{c}{\scriptsize{(0.745)}} & \mc{1}{c}{\scriptsize{(0.843)}} & \mc{1}{c}{\scriptsize{(0.627)}} \\  

    \mc{2}{l}{\scriptsize{\% of Pos. TE ($H_0$: $\le$ 75\% $|$ 10\% Significance)}} & \mc{1}{c}{\scriptsize{18}} & \mc{1}{c}{\scriptsize{27}} & \mc{1}{c}{\scriptsize{18}} & \mc{1}{c}{\scriptsize{18}} & \mc{1}{c}{\scriptsize{9}} & \mc{1}{c}{\scriptsize{36}} & \mc{1}{c}{\scriptsize{18}} & \mc{1}{c}{\scriptsize{45}} \\  

     &  & \mc{1}{c}{\scriptsize{(1.000)}} & \mc{1}{c}{\scriptsize{(0.941)}} & \mc{1}{c}{\scriptsize{(1.000)}} & \mc{1}{c}{\scriptsize{(1.000)}} & \mc{1}{c}{\scriptsize{(1.000)}} & \mc{1}{c}{\scriptsize{(0.922)}} & \mc{1}{c}{\scriptsize{(1.000)}} & \mc{1}{c}{\scriptsize{(0.804)}} \\  

  \hline\hline
  \end{tabular}
    \begin{tablenotes}
    \scriptsize
    \item 
Note: This table displays various estimates of the treatment effect of ABC/CARE's center-based care.
Column (1) displays the ITT, without accounting for any controls.
Column (2) displays the ITT conditioning on vector of controls, $X$, consisting of APGAR scores 1 
minute after birth, an indicator for the subject being born prematurely, and an indicator for the 
father being home at baseline. We also apply IPW weights, $W$, to account for attrition.
Columns (3)--(4) are analogous to columns (1)--(2), but we restrict the control sample to subjects
who did not enroll in any alternative care.
Column (5) displys the matching estimate, where we use the Mahalanobis metric and Epanechnikov kernel
to match on controls $X$ listed above, and restrict the control sample to subjects who did not enroll
in any alternative care. Additionally, we apply IPW weights, $W$.
Columns (6)--(8) are analogous to Columns (3)--(5), except we restrict the control sample to subejcts
who did enroll in alternative care. 
The final three pairs of rows display the proportion of treatment effects in the table that are 
socially positive. The first row in each pair displays the percentage of treatment effects, and the
second row presents the inference.

Numbers in parentheses represent the $p$-value from a single hypothesis test, and are obtained from 
the empirical bootstrap distribution generated by 200 resamples of the original data. 
Bold $p$-values indicate significance at the 10\% level.
Blank point estimates indicate that we are unable to obtain estimates due to a lack of support in the data. 

    \end{tablenotes}
  \end{threeparttable}

\end{table}
	\end{table} 

	\begin{table}[H]
     \caption{Treatment Effects on HOME Scores, Pooled Sample}
     \label{table:abccare_rslt_pooled_cat18}
	  \begin{tabular}{cccccccccc}
  \toprule

    \scriptsize{Variable} & \scriptsize{Age} & \scriptsize{(1)} & \scriptsize{(2)} & \scriptsize{(3)} & \scriptsize{(4)} & \scriptsize{(5)} & \scriptsize{(6)} & \scriptsize{(7)} & \scriptsize{(8)} \\ 
    \midrule  

    \mc{1}{l}{\scriptsize{Somatization}} & \mc{1}{c}{\scriptsize{21}} & \mc{1}{c}{\scriptsize{-0.021}} & \mc{1}{c}{\scriptsize{-0.038}} & \mc{1}{c}{\scriptsize{-0.259}} & \mc{1}{c}{\scriptsize{-0.459}} & \mc{1}{c}{\scriptsize{-0.248}} & \mc{1}{c}{\scriptsize{0.028}} & \mc{1}{c}{\scriptsize{0.019}} & \mc{1}{c}{\scriptsize{0.026}} \\  

     &  & \mc{1}{c}{\scriptsize{(0.461)}} & \mc{1}{c}{\scriptsize{(0.408)}} & \mc{1}{c}{\scriptsize{(0.145)}} & \mc{1}{c}{\scriptsize{\textbf{(0.053)}}} & \mc{1}{c}{\scriptsize{(0.158)}} & \mc{1}{c}{\scriptsize{(0.605)}} & \mc{1}{c}{\scriptsize{(0.632)}} & \mc{1}{c}{\scriptsize{(0.592)}} \\  

     & \mc{1}{c}{\scriptsize{34}} & \mc{1}{c}{\scriptsize{-0.244}} & \mc{1}{c}{\scriptsize{-0.253}} & \mc{1}{c}{\scriptsize{-0.619}} & \mc{1}{c}{\scriptsize{-0.530}} & \mc{1}{c}{\scriptsize{-0.313}} & \mc{1}{c}{\scriptsize{-0.205}} & \mc{1}{c}{\scriptsize{-0.156}} & \mc{1}{c}{\scriptsize{-0.194}} \\  

     &  & \mc{1}{c}{\scriptsize{(0.145)}} & \mc{1}{c}{\scriptsize{(0.132)}} & \mc{1}{c}{\scriptsize{(0.250)}} & \mc{1}{c}{\scriptsize{(0.184)}} & \mc{1}{c}{\scriptsize{(0.237)}} & \mc{1}{c}{\scriptsize{(0.184)}} & \mc{1}{c}{\scriptsize{(0.197)}} & \mc{1}{c}{\scriptsize{(0.184)}} \\  

    \mc{1}{l}{\scriptsize{Depression}} & \mc{1}{c}{\scriptsize{21}} & \mc{1}{c}{\scriptsize{-0.317}} & \mc{1}{c}{\scriptsize{-0.216}} & \mc{1}{c}{\scriptsize{-0.427}} & \mc{1}{c}{\scriptsize{-0.449}} & \mc{1}{c}{\scriptsize{-0.285}} & \mc{1}{c}{\scriptsize{-0.229}} & \mc{1}{c}{\scriptsize{-0.158}} & \mc{1}{c}{\scriptsize{-0.168}} \\  

     &  & \mc{1}{c}{\scriptsize{\textbf{(0.026)}}} & \mc{1}{c}{\scriptsize{(0.132)}} & \mc{1}{c}{\scriptsize{\textbf{(0.066)}}} & \mc{1}{c}{\scriptsize{(0.105)}} & \mc{1}{c}{\scriptsize{(0.171)}} & \mc{1}{c}{\scriptsize{\textbf{(0.053)}}} & \mc{1}{c}{\scriptsize{(0.158)}} & \mc{1}{c}{\scriptsize{(0.145)}} \\  

     & \mc{1}{c}{\scriptsize{34}} & \mc{1}{c}{\scriptsize{-0.195}} & \mc{1}{c}{\scriptsize{-0.240}} & \mc{1}{c}{\scriptsize{-0.495}} & \mc{1}{c}{\scriptsize{-0.356}} & \mc{1}{c}{\scriptsize{-0.246}} & \mc{1}{c}{\scriptsize{-0.171}} & \mc{1}{c}{\scriptsize{-0.176}} & \mc{1}{c}{\scriptsize{-0.194}} \\  

     &  & \mc{1}{c}{\scriptsize{(0.237)}} & \mc{1}{c}{\scriptsize{(0.171)}} & \mc{1}{c}{\scriptsize{(0.263)}} & \mc{1}{c}{\scriptsize{(0.276)}} & \mc{1}{c}{\scriptsize{(0.237)}} & \mc{1}{c}{\scriptsize{(0.250)}} & \mc{1}{c}{\scriptsize{(0.224)}} & \mc{1}{c}{\scriptsize{(0.197)}} \\  

    \mc{1}{l}{\scriptsize{Anxiety}} & \mc{1}{c}{\scriptsize{21}} & \mc{1}{c}{\scriptsize{-0.142}} & \mc{1}{c}{\scriptsize{-0.032}} & \mc{1}{c}{\scriptsize{-0.164}} & \mc{1}{c}{\scriptsize{-0.291}} & \mc{1}{c}{\scriptsize{-0.063}} & \mc{1}{c}{\scriptsize{-0.111}} & \mc{1}{c}{\scriptsize{-0.011}} & \mc{1}{c}{\scriptsize{-0.041}} \\  

     &  & \mc{1}{c}{\scriptsize{(0.132)}} & \mc{1}{c}{\scriptsize{(0.421)}} & \mc{1}{c}{\scriptsize{(0.250)}} & \mc{1}{c}{\scriptsize{\textbf{(0.092)}}} & \mc{1}{c}{\scriptsize{(0.395)}} & \mc{1}{c}{\scriptsize{(0.171)}} & \mc{1}{c}{\scriptsize{(0.461)}} & \mc{1}{c}{\scriptsize{(0.329)}} \\  

     & \mc{1}{c}{\scriptsize{34}} & \mc{1}{c}{\scriptsize{-0.343}} & \mc{1}{c}{\scriptsize{-0.363}} & \mc{1}{c}{\scriptsize{-0.732}} & \mc{1}{c}{\scriptsize{-0.653}} & \mc{1}{c}{\scriptsize{-0.436}} & \mc{1}{c}{\scriptsize{-0.306}} & \mc{1}{c}{\scriptsize{-0.266}} & \mc{1}{c}{\scriptsize{-0.334}} \\  

     &  & \mc{1}{c}{\scriptsize{\textbf{(0.066)}}} & \mc{1}{c}{\scriptsize{\textbf{(0.079)}}} & \mc{1}{c}{\scriptsize{(0.171)}} & \mc{1}{c}{\scriptsize{(0.145)}} & \mc{1}{c}{\scriptsize{(0.184)}} & \mc{1}{c}{\scriptsize{(0.132)}} & \mc{1}{c}{\scriptsize{(0.105)}} & \mc{1}{c}{\scriptsize{(0.105)}} \\  

    \mc{1}{l}{\scriptsize{Hostility}} & \mc{1}{c}{\scriptsize{21}} & \mc{1}{c}{\scriptsize{-0.336}} & \mc{1}{c}{\scriptsize{-0.231}} & \mc{1}{c}{\scriptsize{-0.469}} & \mc{1}{c}{\scriptsize{-0.652}} & \mc{1}{c}{\scriptsize{-0.379}} & \mc{1}{c}{\scriptsize{-0.249}} & \mc{1}{c}{\scriptsize{-0.214}} & \mc{1}{c}{\scriptsize{-0.207}} \\  

     &  & \mc{1}{c}{\scriptsize{\textbf{(0.026)}}} & \mc{1}{c}{\scriptsize{(0.118)}} & \mc{1}{c}{\scriptsize{(0.118)}} & \mc{1}{c}{\scriptsize{\textbf{(0.092)}}} & \mc{1}{c}{\scriptsize{(0.171)}} & \mc{1}{c}{\scriptsize{\textbf{(0.053)}}} & \mc{1}{c}{\scriptsize{(0.105)}} & \mc{1}{c}{\scriptsize{(0.118)}} \\  

     & \mc{1}{c}{\scriptsize{34}} & \mc{1}{c}{\scriptsize{-0.255}} & \mc{1}{c}{\scriptsize{-0.195}} & \mc{1}{c}{\scriptsize{-0.439}} & \mc{1}{c}{\scriptsize{-0.429}} & \mc{1}{c}{\scriptsize{-0.287}} & \mc{1}{c}{\scriptsize{-0.246}} & \mc{1}{c}{\scriptsize{-0.145}} & \mc{1}{c}{\scriptsize{-0.189}} \\  

     &  & \mc{1}{c}{\scriptsize{(0.132)}} & \mc{1}{c}{\scriptsize{(0.158)}} & \mc{1}{c}{\scriptsize{(0.105)}} & \mc{1}{c}{\scriptsize{(0.132)}} & \mc{1}{c}{\scriptsize{(0.158)}} & \mc{1}{c}{\scriptsize{(0.171)}} & \mc{1}{c}{\scriptsize{(0.184)}} & \mc{1}{c}{\scriptsize{(0.211)}} \\  

    \mc{1}{l}{\scriptsize{Global Severity Index}} & \mc{1}{c}{\scriptsize{21}} & \mc{1}{c}{\scriptsize{-0.192}} & \mc{1}{c}{\scriptsize{-0.127}} & \mc{1}{c}{\scriptsize{-0.244}} & \mc{1}{c}{\scriptsize{-0.361}} & \mc{1}{c}{\scriptsize{-0.150}} & \mc{1}{c}{\scriptsize{-0.157}} & \mc{1}{c}{\scriptsize{-0.099}} & \mc{1}{c}{\scriptsize{-0.106}} \\  

     &  & \mc{1}{c}{\scriptsize{\textbf{(0.053)}}} & \mc{1}{c}{\scriptsize{(0.145)}} & \mc{1}{c}{\scriptsize{(0.145)}} & \mc{1}{c}{\scriptsize{\textbf{(0.092)}}} & \mc{1}{c}{\scriptsize{(0.250)}} & \mc{1}{c}{\scriptsize{\textbf{(0.092)}}} & \mc{1}{c}{\scriptsize{(0.224)}} & \mc{1}{c}{\scriptsize{(0.145)}} \\  

     & \mc{1}{c}{\scriptsize{34}} & \mc{1}{c}{\scriptsize{-4.691}} & \mc{1}{c}{\scriptsize{-5.134}} & \mc{1}{c}{\scriptsize{-11.070}} & \mc{1}{c}{\scriptsize{-9.238}} & \mc{1}{c}{\scriptsize{-5.971}} & \mc{1}{c}{\scriptsize{-4.090}} & \mc{1}{c}{\scriptsize{-3.590}} & \mc{1}{c}{\scriptsize{-4.332}} \\  

     &  & \mc{1}{c}{\scriptsize{(0.145)}} & \mc{1}{c}{\scriptsize{(0.105)}} & \mc{1}{c}{\scriptsize{(0.250)}} & \mc{1}{c}{\scriptsize{(0.184)}} & \mc{1}{c}{\scriptsize{(0.211)}} & \mc{1}{c}{\scriptsize{(0.197)}} & \mc{1}{c}{\scriptsize{(0.171)}} & \mc{1}{c}{\scriptsize{(0.184)}} \\  

    \mc{1}{l}{\scriptsize{BSI Factor}} & \mc{1}{c}{\scriptsize{21 and 34}} & \mc{1}{c}{\scriptsize{-0.549}} & \mc{1}{c}{\scriptsize{-0.302}} & \mc{1}{c}{\scriptsize{-1.130}} & \mc{1}{c}{\scriptsize{-1.165}} & \mc{1}{c}{\scriptsize{-0.587}} & \mc{1}{c}{\scriptsize{-0.456}} & \mc{1}{c}{\scriptsize{-0.097}} & \mc{1}{c}{\scriptsize{-0.207}} \\  

     &  & \mc{1}{c}{\scriptsize{\textbf{(0.000)}}} & \mc{1}{c}{\scriptsize{(0.132)}} & \mc{1}{c}{\scriptsize{\textbf{(0.079)}}} & \mc{1}{c}{\scriptsize{(0.105)}} & \mc{1}{c}{\scriptsize{(0.171)}} & \mc{1}{c}{\scriptsize{\textbf{(0.000)}}} & \mc{1}{c}{\scriptsize{(0.382)}} & \mc{1}{c}{\scriptsize{(0.250)}} \\ 
    \midrule  

    \mc{2}{l}{\scriptsize{\% of Pos. TE ($H_0$: $\le$ 50\%)}} & \mc{1}{c}{\scriptsize{100}} & \mc{1}{c}{\scriptsize{100}} & \mc{1}{c}{\scriptsize{100}} & \mc{1}{c}{\scriptsize{100}} & \mc{1}{c}{\scriptsize{100}} & \mc{1}{c}{\scriptsize{91}} & \mc{1}{c}{\scriptsize{91}} & \mc{1}{c}{\scriptsize{91}} \\  

     &  & \mc{1}{c}{\scriptsize{\textbf{(0.000)}}} & \mc{1}{c}{\scriptsize{\textbf{(0.000)}}} & \mc{1}{c}{\scriptsize{\textbf{(0.000)}}} & \mc{1}{c}{\scriptsize{\textbf{(0.000)}}} & \mc{1}{c}{\scriptsize{\textbf{(0.000)}}} & \mc{1}{c}{\scriptsize{\textbf{(0.000)}}} & \mc{1}{c}{\scriptsize{\textbf{(0.000)}}} & \mc{1}{c}{\scriptsize{\textbf{(0.000)}}} \\  

    \mc{2}{l}{\scriptsize{\% of Pos. TE ($H_0$: $\le$ 10\% $|$ 10\% Significance)}} & \mc{1}{c}{\scriptsize{64}} & \mc{1}{c}{\scriptsize{27}} & \mc{1}{c}{\scriptsize{9}} & \mc{1}{c}{\scriptsize{36}} & \mc{1}{c}{\scriptsize{0}} & \mc{1}{c}{\scriptsize{55}} & \mc{1}{c}{\scriptsize{9}} & \mc{1}{c}{\scriptsize{18}} \\  

     &  & \mc{1}{c}{\scriptsize{\textbf{(0.079)}}} & \mc{1}{c}{\scriptsize{(0.224)}} & \mc{1}{c}{\scriptsize{(0.513)}} & \mc{1}{c}{\scriptsize{(0.105)}} & \mc{1}{c}{\scriptsize{(0.539)}} & \mc{1}{c}{\scriptsize{\textbf{(0.092)}}} & \mc{1}{c}{\scriptsize{(0.250)}} & \mc{1}{c}{\scriptsize{(0.303)}} \\  

  \bottomrule
  \end{tabular}
	\end{table} 

	\begin{table}[H]
     \caption{Treatment Effects on Relation with Spouse, Pooled Sample}
     \label{table:abccare_rslt_pooled_cat19}
	  \begin{tabular}{cccccccccc}
  \toprule

    \scriptsize{Variable} & \scriptsize{Age} & \scriptsize{(1)} & \scriptsize{(2)} & \scriptsize{(3)} & \scriptsize{(4)} & \scriptsize{(5)} & \scriptsize{(6)} & \scriptsize{(7)} & \scriptsize{(8)} \\ 
    \midrule  

    \mc{1}{l}{\scriptsize{Participates in Activity}} & \mc{1}{c}{\scriptsize{12}} & \mc{1}{c}{\scriptsize{0.114}} & \mc{1}{c}{\scriptsize{0.089}} & \mc{1}{c}{\scriptsize{0.111}} & \mc{1}{c}{\scriptsize{0.087}} & \mc{1}{c}{\scriptsize{0.081}} & \mc{1}{c}{\scriptsize{0.126}} & \mc{1}{c}{\scriptsize{0.112}} & \mc{1}{c}{\scriptsize{0.114}} \\  

     &  & \mc{1}{c}{\scriptsize{(0.145)}} & \mc{1}{c}{\scriptsize{(0.211)}} & \mc{1}{c}{\scriptsize{(0.211)}} & \mc{1}{c}{\scriptsize{(0.329)}} & \mc{1}{c}{\scriptsize{(0.316)}} & \mc{1}{c}{\scriptsize{(0.145)}} & \mc{1}{c}{\scriptsize{(0.211)}} & \mc{1}{c}{\scriptsize{(0.184)}} \\  

    \mc{1}{l}{\scriptsize{Time spent reading}} & \mc{1}{c}{\scriptsize{12}} & \mc{1}{c}{\scriptsize{1.696}} & \mc{1}{c}{\scriptsize{1.438}} & \mc{1}{c}{\scriptsize{0.619}} & \mc{1}{c}{\scriptsize{-0.088}} & \mc{1}{c}{\scriptsize{1.106}} & \mc{1}{c}{\scriptsize{1.862}} & \mc{1}{c}{\scriptsize{1.709}} & \mc{1}{c}{\scriptsize{2.165}} \\  

     &  & \mc{1}{c}{\scriptsize{(0.132)}} & \mc{1}{c}{\scriptsize{(0.171)}} & \mc{1}{c}{\scriptsize{(0.382)}} & \mc{1}{c}{\scriptsize{(0.500)}} & \mc{1}{c}{\scriptsize{(0.342)}} & \mc{1}{c}{\scriptsize{\textbf{(0.092)}}} & \mc{1}{c}{\scriptsize{(0.145)}} & \mc{1}{c}{\scriptsize{\textbf{(0.066)}}} \\  

    \mc{1}{l}{\scriptsize{Good Description of Self}} & \mc{1}{c}{\scriptsize{12}} & \mc{1}{c}{\scriptsize{0.053}} & \mc{1}{c}{\scriptsize{-0.040}} & \mc{1}{c}{\scriptsize{-0.072}} & \mc{1}{c}{\scriptsize{-0.366}} & \mc{1}{c}{\scriptsize{-0.178}} & \mc{1}{c}{\scriptsize{0.102}} & \mc{1}{c}{\scriptsize{0.010}} & \mc{1}{c}{\scriptsize{-0.045}} \\  

     &  & \mc{1}{c}{\scriptsize{(0.355)}} & \mc{1}{c}{\scriptsize{(0.645)}} & \mc{1}{c}{\scriptsize{(0.618)}} & \mc{1}{c}{\scriptsize{(0.947)}} & \mc{1}{c}{\scriptsize{(0.855)}} & \mc{1}{c}{\scriptsize{(0.224)}} & \mc{1}{c}{\scriptsize{(0.447)}} & \mc{1}{c}{\scriptsize{(0.658)}} \\  

    \mc{1}{l}{\scriptsize{Views Self as Dumb}} & \mc{1}{c}{\scriptsize{12}} & \mc{1}{c}{\scriptsize{0.025}} & \mc{1}{c}{\scriptsize{0.030}} & \mc{1}{c}{\scriptsize{-0.105}} & \mc{1}{c}{\scriptsize{-0.286}} & \mc{1}{c}{\scriptsize{-0.127}} & \mc{1}{c}{\scriptsize{0.073}} & \mc{1}{c}{\scriptsize{0.070}} & \mc{1}{c}{\scriptsize{0.068}} \\  

     &  & \mc{1}{c}{\scriptsize{(0.618)}} & \mc{1}{c}{\scriptsize{(0.526)}} & \mc{1}{c}{\scriptsize{(0.303)}} & \mc{1}{c}{\scriptsize{(0.158)}} & \mc{1}{c}{\scriptsize{(0.289)}} & \mc{1}{c}{\scriptsize{(0.724)}} & \mc{1}{c}{\scriptsize{(0.618)}} & \mc{1}{c}{\scriptsize{(0.697)}} \\  

    \mc{1}{l}{\scriptsize{Views Self as Clumsy}} & \mc{1}{c}{\scriptsize{12}} & \mc{1}{c}{\scriptsize{-0.059}} & \mc{1}{c}{\scriptsize{-0.098}} & \mc{1}{c}{\scriptsize{0.085}} & \mc{1}{c}{\scriptsize{0.098}} & \mc{1}{c}{\scriptsize{0.089}} & \mc{1}{c}{\scriptsize{-0.101}} & \mc{1}{c}{\scriptsize{-0.140}} & \mc{1}{c}{\scriptsize{-0.141}} \\  

     &  & \mc{1}{c}{\scriptsize{(0.211)}} & \mc{1}{c}{\scriptsize{(0.211)}} & \mc{1}{c}{\scriptsize{(0.737)}} & \mc{1}{c}{\scriptsize{(0.763)}} & \mc{1}{c}{\scriptsize{(0.776)}} & \mc{1}{c}{\scriptsize{(0.145)}} & \mc{1}{c}{\scriptsize{(0.145)}} & \mc{1}{c}{\scriptsize{(0.132)}} \\  

    \mc{1}{l}{\scriptsize{Views Self as Not Liked}} & \mc{1}{c}{\scriptsize{12}} & \mc{1}{c}{\scriptsize{-0.116}} & \mc{1}{c}{\scriptsize{-0.055}} & \mc{1}{c}{\scriptsize{0.006}} & \mc{1}{c}{\scriptsize{-0.019}} & \mc{1}{c}{\scriptsize{-0.062}} & \mc{1}{c}{\scriptsize{-0.153}} & \mc{1}{c}{\scriptsize{-0.083}} & \mc{1}{c}{\scriptsize{-0.214}} \\  

     &  & \mc{1}{c}{\scriptsize{(0.118)}} & \mc{1}{c}{\scriptsize{(0.303)}} & \mc{1}{c}{\scriptsize{(0.526)}} & \mc{1}{c}{\scriptsize{(0.421)}} & \mc{1}{c}{\scriptsize{(0.342)}} & \mc{1}{c}{\scriptsize{\textbf{(0.092)}}} & \mc{1}{c}{\scriptsize{(0.211)}} & \mc{1}{c}{\scriptsize{\textbf{(0.053)}}} \\  

    \mc{1}{l}{\scriptsize{Proud about Self}} & \mc{1}{c}{\scriptsize{12}} & \mc{1}{c}{\scriptsize{-0.047}} & \mc{1}{c}{\scriptsize{-0.060}} & \mc{1}{c}{\scriptsize{0.013}} & \mc{1}{c}{\scriptsize{0.068}} & \mc{1}{c}{\scriptsize{0.023}} & \mc{1}{c}{\scriptsize{-0.062}} & \mc{1}{c}{\scriptsize{-0.069}} & \mc{1}{c}{\scriptsize{-0.058}} \\  

     &  & \mc{1}{c}{\scriptsize{(0.658)}} & \mc{1}{c}{\scriptsize{(0.697)}} & \mc{1}{c}{\scriptsize{(0.474)}} & \mc{1}{c}{\scriptsize{(0.368)}} & \mc{1}{c}{\scriptsize{(0.421)}} & \mc{1}{c}{\scriptsize{(0.724)}} & \mc{1}{c}{\scriptsize{(0.697)}} & \mc{1}{c}{\scriptsize{(0.711)}} \\  

    \mc{1}{l}{\scriptsize{Family Proud of You}} & \mc{1}{c}{\scriptsize{12}} & \mc{1}{c}{\scriptsize{0.021}} & \mc{1}{c}{\scriptsize{-0.007}} & \mc{1}{c}{\scriptsize{0.033}} & \mc{1}{c}{\scriptsize{-0.069}} & \mc{1}{c}{\scriptsize{0.023}} & \mc{1}{c}{\scriptsize{0.012}} & \mc{1}{c}{\scriptsize{0.008}} & \mc{1}{c}{\scriptsize{0.002}} \\  

     &  & \mc{1}{c}{\scriptsize{(0.408)}} & \mc{1}{c}{\scriptsize{(0.566)}} & \mc{1}{c}{\scriptsize{(0.447)}} & \mc{1}{c}{\scriptsize{(0.658)}} & \mc{1}{c}{\scriptsize{(0.447)}} & \mc{1}{c}{\scriptsize{(0.434)}} & \mc{1}{c}{\scriptsize{(0.500)}} & \mc{1}{c}{\scriptsize{(0.500)}} \\  

    \mc{1}{l}{\scriptsize{Feels Inadequate, Inferior}} & \mc{1}{c}{\scriptsize{12}} & \mc{1}{c}{\scriptsize{-0.025}} & \mc{1}{c}{\scriptsize{-0.017}} & \mc{1}{c}{\scriptsize{0.183}} & \mc{1}{c}{\scriptsize{0.178}} & \mc{1}{c}{\scriptsize{0.220}} & \mc{1}{c}{\scriptsize{-0.030}} & \mc{1}{c}{\scriptsize{-0.063}} & \mc{1}{c}{\scriptsize{-0.051}} \\  

     &  & \mc{1}{c}{\scriptsize{(0.461)}} & \mc{1}{c}{\scriptsize{(0.408)}} & \mc{1}{c}{\scriptsize{(0.908)}} & \mc{1}{c}{\scriptsize{(0.803)}} & \mc{1}{c}{\scriptsize{(0.947)}} & \mc{1}{c}{\scriptsize{(0.434)}} & \mc{1}{c}{\scriptsize{(0.355)}} & \mc{1}{c}{\scriptsize{(0.395)}} \\  

    \mc{1}{l}{\scriptsize{Withdraws Excessively}} & \mc{1}{c}{\scriptsize{12}} & \mc{1}{c}{\scriptsize{-0.040}} & \mc{1}{c}{\scriptsize{-0.029}} & \mc{1}{c}{\scriptsize{0.137}} & \mc{1}{c}{\scriptsize{0.018}} & \mc{1}{c}{\scriptsize{0.131}} & \mc{1}{c}{\scriptsize{-0.070}} & \mc{1}{c}{\scriptsize{-0.064}} & \mc{1}{c}{\scriptsize{-0.068}} \\  

     &  & \mc{1}{c}{\scriptsize{(0.368)}} & \mc{1}{c}{\scriptsize{(0.421)}} & \mc{1}{c}{\scriptsize{(0.684)}} & \mc{1}{c}{\scriptsize{(0.461)}} & \mc{1}{c}{\scriptsize{(0.724)}} & \mc{1}{c}{\scriptsize{(0.263)}} & \mc{1}{c}{\scriptsize{(0.329)}} & \mc{1}{c}{\scriptsize{(0.329)}} \\  

    \mc{1}{l}{\scriptsize{Ignores Situation}} & \mc{1}{c}{\scriptsize{12}} & \mc{1}{c}{\scriptsize{-0.192}} & \mc{1}{c}{\scriptsize{-0.275}} & \mc{1}{c}{\scriptsize{-0.268}} & \mc{1}{c}{\scriptsize{-0.511}} & \mc{1}{c}{\scriptsize{-0.312}} & \mc{1}{c}{\scriptsize{-0.166}} & \mc{1}{c}{\scriptsize{-0.218}} & \mc{1}{c}{\scriptsize{-0.131}} \\  

     &  & \mc{1}{c}{\scriptsize{\textbf{(0.039)}}} & \mc{1}{c}{\scriptsize{\textbf{(0.026)}}} & \mc{1}{c}{\scriptsize{(0.224)}} & \mc{1}{c}{\scriptsize{\textbf{(0.053)}}} & \mc{1}{c}{\scriptsize{(0.158)}} & \mc{1}{c}{\scriptsize{\textbf{(0.092)}}} & \mc{1}{c}{\scriptsize{\textbf{(0.053)}}} & \mc{1}{c}{\scriptsize{(0.211)}} \\  

    \mc{1}{l}{\scriptsize{Not Cope with Prob.}} & \mc{1}{c}{\scriptsize{12}} & \mc{1}{c}{\scriptsize{-0.084}} & \mc{1}{c}{\scriptsize{-0.106}} & \mc{1}{c}{\scriptsize{-0.065}} & \mc{1}{c}{\scriptsize{-0.177}} & \mc{1}{c}{\scriptsize{-0.021}} & \mc{1}{c}{\scriptsize{-0.077}} & \mc{1}{c}{\scriptsize{-0.107}} & \mc{1}{c}{\scriptsize{-0.033}} \\  

     &  & \mc{1}{c}{\scriptsize{(0.197)}} & \mc{1}{c}{\scriptsize{(0.171)}} & \mc{1}{c}{\scriptsize{(0.368)}} & \mc{1}{c}{\scriptsize{(0.145)}} & \mc{1}{c}{\scriptsize{(0.474)}} & \mc{1}{c}{\scriptsize{(0.237)}} & \mc{1}{c}{\scriptsize{(0.197)}} & \mc{1}{c}{\scriptsize{(0.408)}} \\  

    \mc{1}{l}{\scriptsize{Often Mad of Angry}} & \mc{1}{c}{\scriptsize{12}} & \mc{1}{c}{\scriptsize{-0.130}} & \mc{1}{c}{\scriptsize{-0.223}} & \mc{1}{c}{\scriptsize{0.104}} & \mc{1}{c}{\scriptsize{0.004}} & \mc{1}{c}{\scriptsize{0.047}} & \mc{1}{c}{\scriptsize{-0.193}} & \mc{1}{c}{\scriptsize{-0.263}} & \mc{1}{c}{\scriptsize{-0.272}} \\  

     &  & \mc{1}{c}{\scriptsize{(0.105)}} & \mc{1}{c}{\scriptsize{\textbf{(0.053)}}} & \mc{1}{c}{\scriptsize{(0.934)}} & \mc{1}{c}{\scriptsize{(0.421)}} & \mc{1}{c}{\scriptsize{(0.842)}} & \mc{1}{c}{\scriptsize{\textbf{(0.092)}}} & \mc{1}{c}{\scriptsize{\textbf{(0.066)}}} & \mc{1}{c}{\scriptsize{\textbf{(0.092)}}} \\  

    \mc{1}{l}{\scriptsize{Impulsivity}} & \mc{1}{c}{\scriptsize{12}} & \mc{1}{c}{\scriptsize{-0.039}} & \mc{1}{c}{\scriptsize{0.037}} & \mc{1}{c}{\scriptsize{-0.072}} & \mc{1}{c}{\scriptsize{-0.170}} & \mc{1}{c}{\scriptsize{-0.042}} & \mc{1}{c}{\scriptsize{-0.051}} & \mc{1}{c}{\scriptsize{0.048}} & \mc{1}{c}{\scriptsize{-0.009}} \\  

     &  & \mc{1}{c}{\scriptsize{(0.355)}} & \mc{1}{c}{\scriptsize{(0.645)}} & \mc{1}{c}{\scriptsize{(0.355)}} & \mc{1}{c}{\scriptsize{(0.158)}} & \mc{1}{c}{\scriptsize{(0.382)}} & \mc{1}{c}{\scriptsize{(0.316)}} & \mc{1}{c}{\scriptsize{(0.671)}} & \mc{1}{c}{\scriptsize{(0.474)}} \\  

    \mc{1}{l}{\scriptsize{Significant Fears}} & \mc{1}{c}{\scriptsize{12}} & \mc{1}{c}{\scriptsize{-0.160}} & \mc{1}{c}{\scriptsize{-0.192}} & \mc{1}{c}{\scriptsize{0.183}} & \mc{1}{c}{\scriptsize{0.187}} & \mc{1}{c}{\scriptsize{0.200}} & \mc{1}{c}{\scriptsize{-0.264}} & \mc{1}{c}{\scriptsize{-0.268}} & \mc{1}{c}{\scriptsize{-0.296}} \\  

     &  & \mc{1}{c}{\scriptsize{\textbf{(0.000)}}} & \mc{1}{c}{\scriptsize{\textbf{(0.013)}}} & \mc{1}{c}{\scriptsize{(0.842)}} & \mc{1}{c}{\scriptsize{(0.842)}} & \mc{1}{c}{\scriptsize{(0.855)}} & \mc{1}{c}{\scriptsize{\textbf{(0.000)}}} & \mc{1}{c}{\scriptsize{\textbf{(0.013)}}} & \mc{1}{c}{\scriptsize{\textbf{(0.000)}}} \\  

    \mc{1}{l}{\scriptsize{Denies Any Worries}} & \mc{1}{c}{\scriptsize{12}} & \mc{1}{c}{\scriptsize{-0.239}} & \mc{1}{c}{\scriptsize{-0.331}} & \mc{1}{c}{\scriptsize{-0.163}} & \mc{1}{c}{\scriptsize{-0.164}} & \mc{1}{c}{\scriptsize{-0.156}} & \mc{1}{c}{\scriptsize{-0.266}} & \mc{1}{c}{\scriptsize{-0.357}} & \mc{1}{c}{\scriptsize{-0.281}} \\  

     &  & \mc{1}{c}{\scriptsize{\textbf{(0.000)}}} & \mc{1}{c}{\scriptsize{\textbf{(0.000)}}} & \mc{1}{c}{\scriptsize{(0.224)}} & \mc{1}{c}{\scriptsize{(0.158)}} & \mc{1}{c}{\scriptsize{(0.224)}} & \mc{1}{c}{\scriptsize{\textbf{(0.000)}}} & \mc{1}{c}{\scriptsize{\textbf{(0.000)}}} & \mc{1}{c}{\scriptsize{\textbf{(0.000)}}} \\ 
    \midrule  

    \mc{2}{l}{\scriptsize{\% of Pos. TE ($H_0$: $\le$ 50\%)}} & \mc{1}{c}{\scriptsize{88}} & \mc{1}{c}{\scriptsize{69}} & \mc{1}{c}{\scriptsize{56}} & \mc{1}{c}{\scriptsize{50}} & \mc{1}{c}{\scriptsize{62}} & \mc{1}{c}{\scriptsize{88}} & \mc{1}{c}{\scriptsize{81}} & \mc{1}{c}{\scriptsize{81}} \\  

     &  & \mc{1}{c}{\scriptsize{\textbf{(0.000)}}} & \mc{1}{c}{\scriptsize{\textbf{(0.039)}}} & \mc{1}{c}{\scriptsize{(0.303)}} & \mc{1}{c}{\scriptsize{(0.513)}} & \mc{1}{c}{\scriptsize{(0.118)}} & \mc{1}{c}{\scriptsize{\textbf{(0.000)}}} & \mc{1}{c}{\scriptsize{\textbf{(0.000)}}} & \mc{1}{c}{\scriptsize{\textbf{(0.000)}}} \\  

    \mc{2}{l}{\scriptsize{\% of Pos. TE ($H_0$: $\le$ 10\% $|$ 10\% Significance)}} & \mc{1}{c}{\scriptsize{31}} & \mc{1}{c}{\scriptsize{25}} & \mc{1}{c}{\scriptsize{0}} & \mc{1}{c}{\scriptsize{6}} & \mc{1}{c}{\scriptsize{0}} & \mc{1}{c}{\scriptsize{38}} & \mc{1}{c}{\scriptsize{31}} & \mc{1}{c}{\scriptsize{31}} \\  

     &  & \mc{1}{c}{\scriptsize{\textbf{(0.039)}}} & \mc{1}{c}{\scriptsize{\textbf{(0.039)}}} & \mc{1}{c}{\scriptsize{(1.000)}} & \mc{1}{c}{\scriptsize{(0.461)}} & \mc{1}{c}{\scriptsize{(0.789)}} & \mc{1}{c}{\scriptsize{\textbf{(0.000)}}} & \mc{1}{c}{\scriptsize{\textbf{(0.000)}}} & \mc{1}{c}{\scriptsize{\textbf{(0.000)}}} \\  

  \bottomrule
  \end{tabular}
	\end{table} 

	\begin{table}[H]
     \caption{Treatment Effects on Spouse Characteristics, Pooled Sample}
     \label{table:abccare_rslt_pooled_cat20}
	  \begin{tabular}{cccccccccc}
  \toprule

    \scriptsize{Variable} & \scriptsize{Age} & \scriptsize{(1)} & \scriptsize{(2)} & \scriptsize{(3)} & \scriptsize{(4)} & \scriptsize{(5)} & \scriptsize{(6)} & \scriptsize{(7)} & \scriptsize{(8)} \\ 
    \midrule  

    \mc{1}{l}{\scriptsize{Spouse annual income}} & \mc{1}{c}{\scriptsize{30}} & \mc{1}{c}{\scriptsize{3,221}} & \mc{1}{c}{\scriptsize{4,380}} & \mc{1}{c}{\scriptsize{-3,085}} & \mc{1}{c}{\scriptsize{-8,302}} & \mc{1}{c}{\scriptsize{-6,349}} & \mc{1}{c}{\scriptsize{3,211}} & \mc{1}{c}{\scriptsize{4,359}} & \mc{1}{c}{\scriptsize{402}} \\  

     &  & \mc{1}{c}{\scriptsize{(0.237)}} & \mc{1}{c}{\scriptsize{(0.276)}} & \mc{1}{c}{\scriptsize{(0.645)}} & \mc{1}{c}{\scriptsize{(0.592)}} & \mc{1}{c}{\scriptsize{(0.737)}} & \mc{1}{c}{\scriptsize{(0.263)}} & \mc{1}{c}{\scriptsize{(0.342)}} & \mc{1}{c}{\scriptsize{(0.474)}} \\  

    \mc{1}{l}{\scriptsize{Spouse employment status}} & \mc{1}{c}{\scriptsize{30}} & \mc{1}{c}{\scriptsize{-0.127}} & \mc{1}{c}{\scriptsize{-0.274}} & \mc{1}{c}{\scriptsize{-0.231}} & \mc{1}{c}{\scriptsize{-0.463}} & \mc{1}{c}{\scriptsize{-0.433}} & \mc{1}{c}{\scriptsize{-0.147}} & \mc{1}{c}{\scriptsize{-0.282}} & \mc{1}{c}{\scriptsize{-0.323}} \\  

     &  & \mc{1}{c}{\scriptsize{(0.908)}} & \mc{1}{c}{\scriptsize{(0.974)}} & \mc{1}{c}{\scriptsize{(0.987)}} & \mc{1}{c}{\scriptsize{(0.921)}} & \mc{1}{c}{\scriptsize{(0.987)}} & \mc{1}{c}{\scriptsize{(0.961)}} & \mc{1}{c}{\scriptsize{(0.987)}} & \mc{1}{c}{\scriptsize{(1.000)}} \\  

  \bottomrule
  \end{tabular}
	\end{table} 

	\begin{table}[H]
     \caption{Treatment Effects on Subject Home and Property, Pooled Sample}
     \label{table:abccare_rslt_pooled_cat21}
	  \begin{tabular}{cccccccccc}
  \toprule

    \scriptsize{Variable} & \scriptsize{Age} & \scriptsize{(1)} & \scriptsize{(2)} & \scriptsize{(3)} & \scriptsize{(4)} & \scriptsize{(5)} & \scriptsize{(6)} & \scriptsize{(7)} & \scriptsize{(8)} \\ 
    \midrule  

    \mc{1}{l}{\scriptsize{Room density (room/people)}} & \mc{1}{c}{\scriptsize{30}} & \mc{1}{c}{\scriptsize{0.109}} & \mc{1}{c}{\scriptsize{0.070}} & \mc{1}{c}{\scriptsize{-0.053}} & \mc{1}{c}{\scriptsize{-0.318}} & \mc{1}{c}{\scriptsize{-0.078}} & \mc{1}{c}{\scriptsize{0.165}} & \mc{1}{c}{\scriptsize{0.132}} & \mc{1}{c}{\scriptsize{0.140}} \\  

     &  & \mc{1}{c}{\scriptsize{(0.289)}} & \mc{1}{c}{\scriptsize{(0.434)}} & \mc{1}{c}{\scriptsize{(0.539)}} & \mc{1}{c}{\scriptsize{(0.724)}} & \mc{1}{c}{\scriptsize{(0.553)}} & \mc{1}{c}{\scriptsize{(0.224)}} & \mc{1}{c}{\scriptsize{(0.289)}} & \mc{1}{c}{\scriptsize{(0.211)}} \\  

    \mc{1}{l}{\scriptsize{Own computers}} & \mc{1}{c}{\scriptsize{30}} & \mc{1}{c}{\scriptsize{0.047}} & \mc{1}{c}{\scriptsize{0.007}} & \mc{1}{c}{\scriptsize{0.056}} & \mc{1}{c}{\scriptsize{0.043}} & \mc{1}{c}{\scriptsize{0.037}} & \mc{1}{c}{\scriptsize{0.043}} & \mc{1}{c}{\scriptsize{0.009}} & \mc{1}{c}{\scriptsize{0.016}} \\  

     &  & \mc{1}{c}{\scriptsize{(0.250)}} & \mc{1}{c}{\scriptsize{(0.513)}} & \mc{1}{c}{\scriptsize{(0.316)}} & \mc{1}{c}{\scriptsize{(0.355)}} & \mc{1}{c}{\scriptsize{(0.355)}} & \mc{1}{c}{\scriptsize{(0.263)}} & \mc{1}{c}{\scriptsize{(0.461)}} & \mc{1}{c}{\scriptsize{(0.434)}} \\  

    \mc{1}{l}{\scriptsize{Own cars}} & \mc{1}{c}{\scriptsize{30}} & \mc{1}{c}{\scriptsize{0.095}} & \mc{1}{c}{\scriptsize{0.104}} & \mc{1}{c}{\scriptsize{0.215}} & \mc{1}{c}{\scriptsize{0.154}} & \mc{1}{c}{\scriptsize{0.252}} & \mc{1}{c}{\scriptsize{0.075}} & \mc{1}{c}{\scriptsize{0.085}} & \mc{1}{c}{\scriptsize{0.102}} \\  

     &  & \mc{1}{c}{\scriptsize{\textbf{(0.092)}}} & \mc{1}{c}{\scriptsize{(0.118)}} & \mc{1}{c}{\scriptsize{(0.132)}} & \mc{1}{c}{\scriptsize{(0.171)}} & \mc{1}{c}{\scriptsize{\textbf{(0.079)}}} & \mc{1}{c}{\scriptsize{(0.184)}} & \mc{1}{c}{\scriptsize{(0.171)}} & \mc{1}{c}{\scriptsize{\textbf{(0.092)}}} \\  

    \mc{1}{l}{\scriptsize{Own residences}} & \mc{1}{c}{\scriptsize{30}} & \mc{1}{c}{\scriptsize{0.055}} & \mc{1}{c}{\scriptsize{0.026}} & \mc{1}{c}{\scriptsize{0.046}} & \mc{1}{c}{\scriptsize{0.069}} & \mc{1}{c}{\scriptsize{0.042}} & \mc{1}{c}{\scriptsize{0.066}} & \mc{1}{c}{\scriptsize{0.034}} & \mc{1}{c}{\scriptsize{0.054}} \\  

     &  & \mc{1}{c}{\scriptsize{(0.197)}} & \mc{1}{c}{\scriptsize{(0.382)}} & \mc{1}{c}{\scriptsize{(0.329)}} & \mc{1}{c}{\scriptsize{(0.316)}} & \mc{1}{c}{\scriptsize{(0.329)}} & \mc{1}{c}{\scriptsize{(0.145)}} & \mc{1}{c}{\scriptsize{(0.355)}} & \mc{1}{c}{\scriptsize{(0.224)}} \\  

  \bottomrule
  \end{tabular}
	\end{table} 

	\begin{table}[H]
     \caption{Treatment Effects on Education, Pooled Sample}
     \label{table:abccare_rslt_pooled_cat22}
	  \begin{tabular}{cccccccccc}
  \toprule

    \scriptsize{Variable} & \scriptsize{Age} & \scriptsize{(1)} & \scriptsize{(2)} & \scriptsize{(3)} & \scriptsize{(4)} & \scriptsize{(5)} & \scriptsize{(6)} & \scriptsize{(7)} & \scriptsize{(8)} \\ 
    \midrule  

    \mc{1}{l}{\scriptsize{Household Earned Income}} & \mc{1}{c}{\scriptsize{9}} & \mc{1}{c}{\scriptsize{17,425}} & \mc{1}{c}{\scriptsize{724}} & \mc{1}{c}{\scriptsize{16,656}} & \mc{1}{c}{\scriptsize{31,008}} & \mc{1}{c}{\scriptsize{12,782}} & \mc{1}{c}{\scriptsize{20,504}} & \mc{1}{c}{\scriptsize{-65,101}} & \mc{1}{c}{\scriptsize{16,741}} \\  

     &  & \mc{1}{c}{\scriptsize{\textbf{(0.039)}}} & \mc{1}{c}{\scriptsize{(0.724)}} & \mc{1}{c}{\scriptsize{\textbf{(0.039)}}} & \mc{1}{c}{\scriptsize{(0.158)}} & \mc{1}{c}{\scriptsize{\textbf{(0.013)}}} & \mc{1}{c}{\scriptsize{\textbf{(0.000)}}} & \mc{1}{c}{\scriptsize{(0.329)}} & \mc{1}{c}{\scriptsize{\textbf{(0.000)}}} \\  

     & \mc{1}{c}{\scriptsize{0}} & \mc{1}{c}{\scriptsize{-1,594}} & \mc{1}{c}{\scriptsize{3,403}} & \mc{1}{c}{\scriptsize{-982}} & \mc{1}{c}{\scriptsize{-652}} & \mc{1}{c}{\scriptsize{2,188}} & \mc{1}{c}{\scriptsize{-2,593}} & \mc{1}{c}{\scriptsize{3,337}} & \mc{1}{c}{\scriptsize{128}} \\  

     &  & \mc{1}{c}{\scriptsize{(0.658)}} & \mc{1}{c}{\scriptsize{(0.197)}} & \mc{1}{c}{\scriptsize{(0.632)}} & \mc{1}{c}{\scriptsize{(0.263)}} & \mc{1}{c}{\scriptsize{(0.408)}} & \mc{1}{c}{\scriptsize{(0.829)}} & \mc{1}{c}{\scriptsize{(0.250)}} & \mc{1}{c}{\scriptsize{(0.461)}} \\  

     & \mc{1}{c}{\scriptsize{12}} & \mc{1}{c}{\scriptsize{8,249}} & \mc{1}{c}{\scriptsize{9,060}} & \mc{1}{c}{\scriptsize{16,420}} & \mc{1}{c}{\scriptsize{20,483}} & \mc{1}{c}{\scriptsize{15,047}} & \mc{1}{c}{\scriptsize{6,159}} & \mc{1}{c}{\scriptsize{6,399}} & \mc{1}{c}{\scriptsize{6,205}} \\  

     &  & \mc{1}{c}{\scriptsize{\textbf{(0.013)}}} & \mc{1}{c}{\scriptsize{\textbf{(0.039)}}} & \mc{1}{c}{\scriptsize{\textbf{(0.000)}}} & \mc{1}{c}{\scriptsize{\textbf{(0.013)}}} & \mc{1}{c}{\scriptsize{\textbf{(0.013)}}} & \mc{1}{c}{\scriptsize{\textbf{(0.079)}}} & \mc{1}{c}{\scriptsize{\textbf{(0.079)}}} & \mc{1}{c}{\scriptsize{\textbf{(0.079)}}} \\  

  \bottomrule
  \end{tabular}
	\end{table} 

	\begin{table}[H]
     \caption{Treatment Effects on Subject Employment and Income, Pooled Sample}
     \label{table:abccare_rslt_pooled_cat23}
	\input{AppResOutput/abccare/rslt_pooled_cat23}
	\end{table} 

	\begin{table}[H]
     \caption{Treatment Effects on Job Attitude, Pooled Sample}
     \label{table:abccare_rslt_pooled_cat24}
	  \begin{tabular}{cccccccccc}
  \toprule

    \scriptsize{Variable} & \scriptsize{Age} & \scriptsize{(1)} & \scriptsize{(2)} & \scriptsize{(3)} & \scriptsize{(4)} & \scriptsize{(5)} & \scriptsize{(6)} & \scriptsize{(7)} & \scriptsize{(8)} \\ 
    \midrule  

    \mc{1}{l}{\scriptsize{Satisfied with working situation}} & \mc{1}{c}{\scriptsize{30}} & \mc{1}{c}{\scriptsize{0.009}} & \mc{1}{c}{\scriptsize{0.084}} & \mc{1}{c}{\scriptsize{-0.031}} & \mc{1}{c}{\scriptsize{-0.146}} & \mc{1}{c}{\scriptsize{0.060}} & \mc{1}{c}{\scriptsize{0.037}} & \mc{1}{c}{\scriptsize{0.159}} & \mc{1}{c}{\scriptsize{0.150}} \\  

     &  & \mc{1}{c}{\scriptsize{(0.382)}} & \mc{1}{c}{\scriptsize{(0.211)}} & \mc{1}{c}{\scriptsize{(0.526)}} & \mc{1}{c}{\scriptsize{(0.803)}} & \mc{1}{c}{\scriptsize{(0.329)}} & \mc{1}{c}{\scriptsize{(0.316)}} & \mc{1}{c}{\scriptsize{\textbf{(0.092)}}} & \mc{1}{c}{\scriptsize{\textbf{(0.066)}}} \\  

    \mc{1}{l}{\scriptsize{Do work well}} & \mc{1}{c}{\scriptsize{30}} & \mc{1}{c}{\scriptsize{0.019}} & \mc{1}{c}{\scriptsize{0.015}} & \mc{1}{c}{\scriptsize{-0.061}} & \mc{1}{c}{\scriptsize{-0.047}} & \mc{1}{c}{\scriptsize{-0.063}} & \mc{1}{c}{\scriptsize{0.047}} & \mc{1}{c}{\scriptsize{0.031}} & \mc{1}{c}{\scriptsize{0.040}} \\  

     &  & \mc{1}{c}{\scriptsize{(0.329)}} & \mc{1}{c}{\scriptsize{(0.368)}} & \mc{1}{c}{\scriptsize{(0.882)}} & \mc{1}{c}{\scriptsize{(0.776)}} & \mc{1}{c}{\scriptsize{(0.895)}} & \mc{1}{c}{\scriptsize{(0.211)}} & \mc{1}{c}{\scriptsize{(0.250)}} & \mc{1}{c}{\scriptsize{(0.289)}} \\  

    \mc{1}{l}{\scriptsize{Not worry too much about work}} & \mc{1}{c}{\scriptsize{30}} & \mc{1}{c}{\scriptsize{0.028}} & \mc{1}{c}{\scriptsize{0.090}} & \mc{1}{c}{\scriptsize{0.114}} & \mc{1}{c}{\scriptsize{0.249}} & \mc{1}{c}{\scriptsize{0.151}} & \mc{1}{c}{\scriptsize{0.004}} & \mc{1}{c}{\scriptsize{0.080}} & \mc{1}{c}{\scriptsize{0.018}} \\  

     &  & \mc{1}{c}{\scriptsize{(0.355)}} & \mc{1}{c}{\scriptsize{(0.197)}} & \mc{1}{c}{\scriptsize{(0.211)}} & \mc{1}{c}{\scriptsize{(0.145)}} & \mc{1}{c}{\scriptsize{(0.184)}} & \mc{1}{c}{\scriptsize{(0.487)}} & \mc{1}{c}{\scriptsize{(0.224)}} & \mc{1}{c}{\scriptsize{(0.395)}} \\  

    \mc{1}{l}{\scriptsize{Work well with others}} & \mc{1}{c}{\scriptsize{30}} & \mc{1}{c}{\scriptsize{0.037}} & \mc{1}{c}{\scriptsize{-0.026}} & \mc{1}{c}{\scriptsize{0.057}} & \mc{1}{c}{\scriptsize{0.012}} & \mc{1}{c}{\scriptsize{-0.002}} & \mc{1}{c}{\scriptsize{0.019}} & \mc{1}{c}{\scriptsize{-0.003}} & \mc{1}{c}{\scriptsize{-0.035}} \\  

     &  & \mc{1}{c}{\scriptsize{(0.289)}} & \mc{1}{c}{\scriptsize{(0.592)}} & \mc{1}{c}{\scriptsize{(0.303)}} & \mc{1}{c}{\scriptsize{(0.368)}} & \mc{1}{c}{\scriptsize{(0.447)}} & \mc{1}{c}{\scriptsize{(0.368)}} & \mc{1}{c}{\scriptsize{(0.461)}} & \mc{1}{c}{\scriptsize{(0.684)}} \\  

    \mc{1}{l}{\scriptsize{Don't do things that cause to lose job}} & \mc{1}{c}{\scriptsize{30}} & \mc{1}{c}{\scriptsize{0.014}} & \mc{1}{c}{\scriptsize{0.070}} & \mc{1}{c}{\scriptsize{-0.143}} & \mc{1}{c}{\scriptsize{-0.108}} & \mc{1}{c}{\scriptsize{-0.140}} & \mc{1}{c}{\scriptsize{0.041}} & \mc{1}{c}{\scriptsize{0.085}} & \mc{1}{c}{\scriptsize{0.065}} \\  

     &  & \mc{1}{c}{\scriptsize{(0.382)}} & \mc{1}{c}{\scriptsize{(0.211)}} & \mc{1}{c}{\scriptsize{(0.987)}} & \mc{1}{c}{\scriptsize{(0.829)}} & \mc{1}{c}{\scriptsize{(0.987)}} & \mc{1}{c}{\scriptsize{(0.276)}} & \mc{1}{c}{\scriptsize{(0.171)}} & \mc{1}{c}{\scriptsize{(0.211)}} \\  

    \mc{1}{l}{\scriptsize{No trouble finishing work}} & \mc{1}{c}{\scriptsize{30}} & \mc{1}{c}{\scriptsize{-0.082}} & \mc{1}{c}{\scriptsize{-0.149}} & \mc{1}{c}{\scriptsize{-0.022}} & \mc{1}{c}{\scriptsize{-0.175}} & \mc{1}{c}{\scriptsize{-0.104}} & \mc{1}{c}{\scriptsize{-0.095}} & \mc{1}{c}{\scriptsize{-0.123}} & \mc{1}{c}{\scriptsize{-0.151}} \\  

     &  & \mc{1}{c}{\scriptsize{(0.868)}} & \mc{1}{c}{\scriptsize{(0.961)}} & \mc{1}{c}{\scriptsize{(0.513)}} & \mc{1}{c}{\scriptsize{(0.882)}} & \mc{1}{c}{\scriptsize{(0.855)}} & \mc{1}{c}{\scriptsize{(0.974)}} & \mc{1}{c}{\scriptsize{(0.895)}} & \mc{1}{c}{\scriptsize{(0.947)}} \\  

    \mc{1}{l}{\scriptsize{Job not too stressful}} & \mc{1}{c}{\scriptsize{30}} & \mc{1}{c}{\scriptsize{-0.029}} & \mc{1}{c}{\scriptsize{0.003}} & \mc{1}{c}{\scriptsize{-0.245}} & \mc{1}{c}{\scriptsize{-0.190}} & \mc{1}{c}{\scriptsize{-0.186}} & \mc{1}{c}{\scriptsize{0.018}} & \mc{1}{c}{\scriptsize{0.040}} & \mc{1}{c}{\scriptsize{0.068}} \\  

     &  & \mc{1}{c}{\scriptsize{(0.618)}} & \mc{1}{c}{\scriptsize{(0.461)}} & \mc{1}{c}{\scriptsize{(1.000)}} & \mc{1}{c}{\scriptsize{(0.934)}} & \mc{1}{c}{\scriptsize{(1.000)}} & \mc{1}{c}{\scriptsize{(0.395)}} & \mc{1}{c}{\scriptsize{(0.303)}} & \mc{1}{c}{\scriptsize{(0.263)}} \\  

    \mc{1}{l}{\scriptsize{Don't stay away from job}} & \mc{1}{c}{\scriptsize{30}} & \mc{1}{c}{\scriptsize{-0.049}} & \mc{1}{c}{\scriptsize{-0.022}} & \mc{1}{c}{\scriptsize{-0.029}} & \mc{1}{c}{\scriptsize{0.006}} & \mc{1}{c}{\scriptsize{-0.008}} & \mc{1}{c}{\scriptsize{-0.040}} & \mc{1}{c}{\scriptsize{-0.025}} & \mc{1}{c}{\scriptsize{-0.073}} \\  

     &  & \mc{1}{c}{\scriptsize{(0.737)}} & \mc{1}{c}{\scriptsize{(0.539)}} & \mc{1}{c}{\scriptsize{(0.605)}} & \mc{1}{c}{\scriptsize{(0.434)}} & \mc{1}{c}{\scriptsize{(0.539)}} & \mc{1}{c}{\scriptsize{(0.684)}} & \mc{1}{c}{\scriptsize{(0.566)}} & \mc{1}{c}{\scriptsize{(0.750)}} \\  

    \mc{1}{l}{\scriptsize{No trouble with boss}} & \mc{1}{c}{\scriptsize{30}} & \mc{1}{c}{\scriptsize{0.119}} & \mc{1}{c}{\scriptsize{0.132}} & \mc{1}{c}{\scriptsize{-0.087}} & \mc{1}{c}{\scriptsize{0.015}} & \mc{1}{c}{\scriptsize{-0.059}} & \mc{1}{c}{\scriptsize{0.174}} & \mc{1}{c}{\scriptsize{0.164}} & \mc{1}{c}{\scriptsize{0.168}} \\  

     &  & \mc{1}{c}{\scriptsize{\textbf{(0.066)}}} & \mc{1}{c}{\scriptsize{\textbf{(0.053)}}} & \mc{1}{c}{\scriptsize{(0.750)}} & \mc{1}{c}{\scriptsize{(0.500)}} & \mc{1}{c}{\scriptsize{(0.645)}} & \mc{1}{c}{\scriptsize{\textbf{(0.026)}}} & \mc{1}{c}{\scriptsize{\textbf{(0.039)}}} & \mc{1}{c}{\scriptsize{\textbf{(0.053)}}} \\  

  \bottomrule
  \end{tabular}
	\end{table} 

	\begin{table}[H]
     \caption{Treatment Effects on Job Satisfaction Score, Pooled Sample}
     \label{table:abccare_rslt_pooled_cat25}
	  \begin{tabular}{cccccccccc}
  \toprule

    \scriptsize{Variable} & \scriptsize{Age} & \scriptsize{(1)} & \scriptsize{(2)} & \scriptsize{(3)} & \scriptsize{(4)} & \scriptsize{(5)} & \scriptsize{(6)} & \scriptsize{(7)} & \scriptsize{(8)} \\ 
    \midrule  

    \mc{1}{l}{\scriptsize{Std. Achv.  Test}} & \mc{1}{c}{\scriptsize{8}} & \mc{1}{c}{\scriptsize{4.207}} & \mc{1}{c}{\scriptsize{4.715}} & \mc{1}{c}{\scriptsize{1.630}} & \mc{1}{c}{\scriptsize{5.607}} & \mc{1}{c}{\scriptsize{2.393}} & \mc{1}{c}{\scriptsize{4.958}} & \mc{1}{c}{\scriptsize{4.637}} & \mc{1}{c}{\scriptsize{5.508}} \\  

     &  & \mc{1}{c}{\scriptsize{\textbf{(0.000)}}} & \mc{1}{c}{\scriptsize{\textbf{(0.000)}}} & \mc{1}{c}{\scriptsize{(0.342)}} & \mc{1}{c}{\scriptsize{\textbf{(0.039)}}} & \mc{1}{c}{\scriptsize{(0.224)}} & \mc{1}{c}{\scriptsize{\textbf{(0.000)}}} & \mc{1}{c}{\scriptsize{\textbf{(0.026)}}} & \mc{1}{c}{\scriptsize{\textbf{(0.013)}}} \\  

     & \mc{1}{c}{\scriptsize{21}} & \mc{1}{c}{\scriptsize{5.217}} & \mc{1}{c}{\scriptsize{1.981}} & \mc{1}{c}{\scriptsize{4.504}} & \mc{1}{c}{\scriptsize{1.521}} & \mc{1}{c}{\scriptsize{2.825}} & \mc{1}{c}{\scriptsize{5.521}} & \mc{1}{c}{\scriptsize{1.565}} & \mc{1}{c}{\scriptsize{3.488}} \\  

     &  & \mc{1}{c}{\scriptsize{\textbf{(0.026)}}} & \mc{1}{c}{\scriptsize{(0.158)}} & \mc{1}{c}{\scriptsize{\textbf{(0.092)}}} & \mc{1}{c}{\scriptsize{(0.368)}} & \mc{1}{c}{\scriptsize{(0.224)}} & \mc{1}{c}{\scriptsize{\textbf{(0.026)}}} & \mc{1}{c}{\scriptsize{(0.276)}} & \mc{1}{c}{\scriptsize{\textbf{(0.079)}}} \\  

     & \mc{1}{c}{\scriptsize{15}} & \mc{1}{c}{\scriptsize{5.163}} & \mc{1}{c}{\scriptsize{3.390}} & \mc{1}{c}{\scriptsize{5.176}} & \mc{1}{c}{\scriptsize{3.943}} & \mc{1}{c}{\scriptsize{4.141}} & \mc{1}{c}{\scriptsize{5.424}} & \mc{1}{c}{\scriptsize{2.821}} & \mc{1}{c}{\scriptsize{4.146}} \\  

     &  & \mc{1}{c}{\scriptsize{\textbf{(0.000)}}} & \mc{1}{c}{\scriptsize{\textbf{(0.039)}}} & \mc{1}{c}{\scriptsize{\textbf{(0.066)}}} & \mc{1}{c}{\scriptsize{(0.224)}} & \mc{1}{c}{\scriptsize{(0.118)}} & \mc{1}{c}{\scriptsize{\textbf{(0.013)}}} & \mc{1}{c}{\scriptsize{\textbf{(0.092)}}} & \mc{1}{c}{\scriptsize{\textbf{(0.039)}}} \\  

     & \mc{1}{c}{\scriptsize{6.5}} & \mc{1}{c}{\scriptsize{2.767}} & \mc{1}{c}{\scriptsize{3.062}} & \mc{1}{c}{\scriptsize{2.049}} & \mc{1}{c}{\scriptsize{3.367}} & \mc{1}{c}{\scriptsize{2.127}} & \mc{1}{c}{\scriptsize{2.931}} & \mc{1}{c}{\scriptsize{3.051}} & \mc{1}{c}{\scriptsize{3.600}} \\  

     &  & \mc{1}{c}{\scriptsize{\textbf{(0.026)}}} & \mc{1}{c}{\scriptsize{\textbf{(0.026)}}} & \mc{1}{c}{\scriptsize{(0.237)}} & \mc{1}{c}{\scriptsize{(0.158)}} & \mc{1}{c}{\scriptsize{(0.211)}} & \mc{1}{c}{\scriptsize{\textbf{(0.026)}}} & \mc{1}{c}{\scriptsize{\textbf{(0.039)}}} & \mc{1}{c}{\scriptsize{\textbf{(0.026)}}} \\  

     & \mc{1}{c}{\scriptsize{12}} & \mc{1}{c}{\scriptsize{0.003}} & \mc{1}{c}{\scriptsize{-0.711}} & \mc{1}{c}{\scriptsize{5.057}} & \mc{1}{c}{\scriptsize{4.383}} & \mc{1}{c}{\scriptsize{7.750}} & \mc{1}{c}{\scriptsize{-1.893}} & \mc{1}{c}{\scriptsize{-1.833}} & \mc{1}{c}{\scriptsize{-0.137}} \\  

     &  & \mc{1}{c}{\scriptsize{(0.474)}} & \mc{1}{c}{\scriptsize{(0.632)}} & \mc{1}{c}{\scriptsize{(0.197)}} & \mc{1}{c}{\scriptsize{(0.895)}} & \mc{1}{c}{\scriptsize{\textbf{(0.066)}}} & \mc{1}{c}{\scriptsize{(0.724)}} & \mc{1}{c}{\scriptsize{(0.632)}} & \mc{1}{c}{\scriptsize{(0.474)}} \\  

     & \mc{1}{c}{\scriptsize{7.5}} & \mc{1}{c}{\scriptsize{1.937}} & \mc{1}{c}{\scriptsize{2.136}} & \mc{1}{c}{\scriptsize{0.667}} & \mc{1}{c}{\scriptsize{3.618}} & \mc{1}{c}{\scriptsize{0.558}} & \mc{1}{c}{\scriptsize{2.308}} & \mc{1}{c}{\scriptsize{1.762}} & \mc{1}{c}{\scriptsize{2.129}} \\  

     &  & \mc{1}{c}{\scriptsize{(0.118)}} & \mc{1}{c}{\scriptsize{\textbf{(0.066)}}} & \mc{1}{c}{\scriptsize{(0.447)}} & \mc{1}{c}{\scriptsize{(0.132)}} & \mc{1}{c}{\scriptsize{(0.447)}} & \mc{1}{c}{\scriptsize{(0.118)}} & \mc{1}{c}{\scriptsize{(0.132)}} & \mc{1}{c}{\scriptsize{(0.132)}} \\  

     & \mc{1}{c}{\scriptsize{7}} & \mc{1}{c}{\scriptsize{3.435}} & \mc{1}{c}{\scriptsize{3.163}} & \mc{1}{c}{\scriptsize{5.227}} & \mc{1}{c}{\scriptsize{5.897}} & \mc{1}{c}{\scriptsize{5.843}} & \mc{1}{c}{\scriptsize{3.025}} & \mc{1}{c}{\scriptsize{2.502}} & \mc{1}{c}{\scriptsize{3.596}} \\  

     &  & \mc{1}{c}{\scriptsize{\textbf{(0.026)}}} & \mc{1}{c}{\scriptsize{\textbf{(0.053)}}} & \mc{1}{c}{\scriptsize{\textbf{(0.053)}}} & \mc{1}{c}{\scriptsize{\textbf{(0.026)}}} & \mc{1}{c}{\scriptsize{\textbf{(0.039)}}} & \mc{1}{c}{\scriptsize{\textbf{(0.066)}}} & \mc{1}{c}{\scriptsize{(0.105)}} & \mc{1}{c}{\scriptsize{\textbf{(0.026)}}} \\  

    \mc{1}{l}{\scriptsize{PIAT Math Std. Score}} & \mc{1}{c}{\scriptsize{7}} & \mc{1}{c}{\scriptsize{2.600}} & \mc{1}{c}{\scriptsize{2.931}} & \mc{1}{c}{\scriptsize{2.178}} & \mc{1}{c}{\scriptsize{2.802}} & \mc{1}{c}{\scriptsize{2.572}} & \mc{1}{c}{\scriptsize{2.696}} & \mc{1}{c}{\scriptsize{2.966}} & \mc{1}{c}{\scriptsize{3.095}} \\  

     &  & \mc{1}{c}{\scriptsize{(0.118)}} & \mc{1}{c}{\scriptsize{(0.105)}} & \mc{1}{c}{\scriptsize{(0.289)}} & \mc{1}{c}{\scriptsize{(0.276)}} & \mc{1}{c}{\scriptsize{(0.289)}} & \mc{1}{c}{\scriptsize{\textbf{(0.092)}}} & \mc{1}{c}{\scriptsize{(0.118)}} & \mc{1}{c}{\scriptsize{\textbf{(0.092)}}} \\  

    \mc{1}{l}{\scriptsize{Std. Achv.  Test}} & \mc{1}{c}{\scriptsize{5.5}} & \mc{1}{c}{\scriptsize{8.029}} & \mc{1}{c}{\scriptsize{7.303}} & \mc{1}{c}{\scriptsize{14.284}} & \mc{1}{c}{\scriptsize{16.121}} & \mc{1}{c}{\scriptsize{14.161}} & \mc{1}{c}{\scriptsize{6.223}} & \mc{1}{c}{\scriptsize{4.444}} & \mc{1}{c}{\scriptsize{5.796}} \\  

     &  & \mc{1}{c}{\scriptsize{\textbf{(0.000)}}} & \mc{1}{c}{\scriptsize{\textbf{(0.000)}}} & \mc{1}{c}{\scriptsize{\textbf{(0.000)}}} & \mc{1}{c}{\scriptsize{\textbf{(0.000)}}} & \mc{1}{c}{\scriptsize{\textbf{(0.000)}}} & \mc{1}{c}{\scriptsize{\textbf{(0.013)}}} & \mc{1}{c}{\scriptsize{\textbf{(0.053)}}} & \mc{1}{c}{\scriptsize{\textbf{(0.013)}}} \\  

     & \mc{1}{c}{\scriptsize{8.5}} & \mc{1}{c}{\scriptsize{5.938}} & \mc{1}{c}{\scriptsize{6.119}} & \mc{1}{c}{\scriptsize{5.046}} & \mc{1}{c}{\scriptsize{7.926}} & \mc{1}{c}{\scriptsize{5.390}} & \mc{1}{c}{\scriptsize{5.507}} & \mc{1}{c}{\scriptsize{5.008}} & \mc{1}{c}{\scriptsize{5.828}} \\  

     &  & \mc{1}{c}{\scriptsize{\textbf{(0.000)}}} & \mc{1}{c}{\scriptsize{\textbf{(0.000)}}} & \mc{1}{c}{\scriptsize{(0.118)}} & \mc{1}{c}{\scriptsize{\textbf{(0.066)}}} & \mc{1}{c}{\scriptsize{(0.105)}} & \mc{1}{c}{\scriptsize{\textbf{(0.000)}}} & \mc{1}{c}{\scriptsize{\textbf{(0.000)}}} & \mc{1}{c}{\scriptsize{\textbf{(0.000)}}} \\  

     & \mc{1}{c}{\scriptsize{6}} & \mc{1}{c}{\scriptsize{4.543}} & \mc{1}{c}{\scriptsize{4.519}} & \mc{1}{c}{\scriptsize{6.178}} & \mc{1}{c}{\scriptsize{8.140}} & \mc{1}{c}{\scriptsize{6.804}} & \mc{1}{c}{\scriptsize{4.075}} & \mc{1}{c}{\scriptsize{3.306}} & \mc{1}{c}{\scriptsize{3.932}} \\  

     &  & \mc{1}{c}{\scriptsize{\textbf{(0.000)}}} & \mc{1}{c}{\scriptsize{\textbf{(0.013)}}} & \mc{1}{c}{\scriptsize{\textbf{(0.013)}}} & \mc{1}{c}{\scriptsize{\textbf{(0.092)}}} & \mc{1}{c}{\scriptsize{\textbf{(0.066)}}} & \mc{1}{c}{\scriptsize{\textbf{(0.000)}}} & \mc{1}{c}{\scriptsize{(0.105)}} & \mc{1}{c}{\scriptsize{\textbf{(0.026)}}} \\  

  \bottomrule
  \end{tabular}
	\end{table} 

	\begin{table}[H]
     \caption{Treatment Effects on Crime, Pooled Sample}
     \label{table:abccare_rslt_pooled_cat26}
	  \begin{tabular}{cccccccccc}
  \toprule

    \scriptsize{Variable} & \scriptsize{Age} & \scriptsize{(1)} & \scriptsize{(2)} & \scriptsize{(3)} & \scriptsize{(4)} & \scriptsize{(5)} & \scriptsize{(6)} & \scriptsize{(7)} & \scriptsize{(8)} \\ 
    \midrule  

    \mc{1}{l}{\scriptsize{Std. IQ Test}} & \mc{1}{c}{\scriptsize{6.6}} & \mc{1}{c}{\scriptsize{5.956}} & \mc{1}{c}{\scriptsize{5.497}} & \mc{1}{c}{\scriptsize{4.088}} & \mc{1}{c}{\scriptsize{4.646}} & \mc{1}{c}{\scriptsize{4.845}} & \mc{1}{c}{\scriptsize{5.850}} & \mc{1}{c}{\scriptsize{5.369}} & \mc{1}{c}{\scriptsize{6.138}} \\  

     &  & \mc{1}{c}{\scriptsize{\textbf{(0.013)}}} & \mc{1}{c}{\scriptsize{\textbf{(0.013)}}} & \mc{1}{c}{\scriptsize{(0.132)}} & \mc{1}{c}{\scriptsize{\textbf{(0.092)}}} & \mc{1}{c}{\scriptsize{\textbf{(0.079)}}} & \mc{1}{c}{\scriptsize{\textbf{(0.026)}}} & \mc{1}{c}{\scriptsize{\textbf{(0.039)}}} & \mc{1}{c}{\scriptsize{\textbf{(0.013)}}} \\  

     & \mc{1}{c}{\scriptsize{2.5}} & \mc{1}{c}{\scriptsize{8.033}} & \mc{1}{c}{\scriptsize{10.494}} & \mc{1}{c}{\scriptsize{13.533}} & \mc{1}{c}{\scriptsize{16.896}} & \mc{1}{c}{\scriptsize{15.713}} & \mc{1}{c}{\scriptsize{5.971}} & \mc{1}{c}{\scriptsize{10.431}} & \mc{1}{c}{\scriptsize{8.138}} \\  

     &  & \mc{1}{c}{\scriptsize{\textbf{(0.000)}}} & \mc{1}{c}{\scriptsize{\textbf{(0.026)}}} & \mc{1}{c}{\scriptsize{\textbf{(0.000)}}} & \mc{1}{c}{\scriptsize{(0.461)}} & \mc{1}{c}{\scriptsize{\textbf{(0.000)}}} & \mc{1}{c}{\scriptsize{\textbf{(0.066)}}} & \mc{1}{c}{\scriptsize{\textbf{(0.079)}}} & \mc{1}{c}{\scriptsize{\textbf{(0.039)}}} \\  

     & \mc{1}{c}{\scriptsize{4}} & \mc{1}{c}{\scriptsize{9.166}} & \mc{1}{c}{\scriptsize{8.768}} & \mc{1}{c}{\scriptsize{11.985}} & \mc{1}{c}{\scriptsize{12.570}} & \mc{1}{c}{\scriptsize{12.562}} & \mc{1}{c}{\scriptsize{8.149}} & \mc{1}{c}{\scriptsize{7.853}} & \mc{1}{c}{\scriptsize{8.525}} \\  

     &  & \mc{1}{c}{\scriptsize{\textbf{(0.000)}}} & \mc{1}{c}{\scriptsize{\textbf{(0.000)}}} & \mc{1}{c}{\scriptsize{\textbf{(0.000)}}} & \mc{1}{c}{\scriptsize{\textbf{(0.000)}}} & \mc{1}{c}{\scriptsize{\textbf{(0.000)}}} & \mc{1}{c}{\scriptsize{\textbf{(0.000)}}} & \mc{1}{c}{\scriptsize{\textbf{(0.000)}}} & \mc{1}{c}{\scriptsize{\textbf{(0.000)}}} \\  

     & \mc{1}{c}{\scriptsize{15}} & \mc{1}{c}{\scriptsize{5.771}} & \mc{1}{c}{\scriptsize{3.137}} & \mc{1}{c}{\scriptsize{1.497}} & \mc{1}{c}{\scriptsize{-1.596}} & \mc{1}{c}{\scriptsize{0.540}} & \mc{1}{c}{\scriptsize{6.522}} & \mc{1}{c}{\scriptsize{3.913}} & \mc{1}{c}{\scriptsize{5.120}} \\  

     &  & \mc{1}{c}{\scriptsize{\textbf{(0.000)}}} & \mc{1}{c}{\scriptsize{(0.132)}} & \mc{1}{c}{\scriptsize{(0.355)}} & \mc{1}{c}{\scriptsize{(0.645)}} & \mc{1}{c}{\scriptsize{(0.421)}} & \mc{1}{c}{\scriptsize{\textbf{(0.013)}}} & \mc{1}{c}{\scriptsize{(0.105)}} & \mc{1}{c}{\scriptsize{\textbf{(0.053)}}} \\  

     & \mc{1}{c}{\scriptsize{6}} & \mc{1}{c}{\scriptsize{9.137}} & \mc{1}{c}{\scriptsize{8.983}} & \mc{1}{c}{\scriptsize{6.272}} & \mc{1}{c}{\scriptsize{4.343}} & \mc{1}{c}{\scriptsize{5.522}} & \mc{1}{c}{\scriptsize{9.658}} & \mc{1}{c}{\scriptsize{10.779}} & \mc{1}{c}{\scriptsize{9.776}} \\  

     &  & \mc{1}{c}{\scriptsize{\textbf{(0.000)}}} & \mc{1}{c}{\scriptsize{\textbf{(0.000)}}} & \mc{1}{c}{\scriptsize{(0.184)}} & \mc{1}{c}{\scriptsize{(0.132)}} & \mc{1}{c}{\scriptsize{(0.224)}} & \mc{1}{c}{\scriptsize{\textbf{(0.000)}}} & \mc{1}{c}{\scriptsize{\textbf{(0.013)}}} & \mc{1}{c}{\scriptsize{\textbf{(0.000)}}} \\  

     & \mc{1}{c}{\scriptsize{3}} & \mc{1}{c}{\scriptsize{13.450}} & \mc{1}{c}{\scriptsize{14.071}} & \mc{1}{c}{\scriptsize{19.242}} & \mc{1}{c}{\scriptsize{21.795}} & \mc{1}{c}{\scriptsize{20.287}} & \mc{1}{c}{\scriptsize{11.314}} & \mc{1}{c}{\scriptsize{12.053}} & \mc{1}{c}{\scriptsize{11.777}} \\  

     &  & \mc{1}{c}{\scriptsize{\textbf{(0.000)}}} & \mc{1}{c}{\scriptsize{\textbf{(0.000)}}} & \mc{1}{c}{\scriptsize{\textbf{(0.000)}}} & \mc{1}{c}{\scriptsize{\textbf{(0.000)}}} & \mc{1}{c}{\scriptsize{\textbf{(0.000)}}} & \mc{1}{c}{\scriptsize{\textbf{(0.000)}}} & \mc{1}{c}{\scriptsize{\textbf{(0.000)}}} & \mc{1}{c}{\scriptsize{\textbf{(0.000)}}} \\  

     & \mc{1}{c}{\scriptsize{4.5}} & \mc{1}{c}{\scriptsize{8.380}} & \mc{1}{c}{\scriptsize{8.004}} & \mc{1}{c}{\scriptsize{13.287}} & \mc{1}{c}{\scriptsize{13.937}} & \mc{1}{c}{\scriptsize{13.523}} & \mc{1}{c}{\scriptsize{6.717}} & \mc{1}{c}{\scriptsize{6.246}} & \mc{1}{c}{\scriptsize{6.827}} \\  

     &  & \mc{1}{c}{\scriptsize{\textbf{(0.000)}}} & \mc{1}{c}{\scriptsize{\textbf{(0.000)}}} & \mc{1}{c}{\scriptsize{\textbf{(0.000)}}} & \mc{1}{c}{\scriptsize{\textbf{(0.000)}}} & \mc{1}{c}{\scriptsize{\textbf{(0.000)}}} & \mc{1}{c}{\scriptsize{\textbf{(0.000)}}} & \mc{1}{c}{\scriptsize{\textbf{(0.013)}}} & \mc{1}{c}{\scriptsize{\textbf{(0.013)}}} \\  

     & \mc{1}{c}{\scriptsize{8}} & \mc{1}{c}{\scriptsize{4.932}} & \mc{1}{c}{\scriptsize{3.848}} & \mc{1}{c}{\scriptsize{2.570}} & \mc{1}{c}{\scriptsize{2.766}} & \mc{1}{c}{\scriptsize{2.709}} & \mc{1}{c}{\scriptsize{4.948}} & \mc{1}{c}{\scriptsize{3.952}} & \mc{1}{c}{\scriptsize{4.799}} \\  

     &  & \mc{1}{c}{\scriptsize{\textbf{(0.026)}}} & \mc{1}{c}{\scriptsize{\textbf{(0.066)}}} & \mc{1}{c}{\scriptsize{(0.276)}} & \mc{1}{c}{\scriptsize{(0.289)}} & \mc{1}{c}{\scriptsize{(0.303)}} & \mc{1}{c}{\scriptsize{\textbf{(0.026)}}} & \mc{1}{c}{\scriptsize{\textbf{(0.079)}}} & \mc{1}{c}{\scriptsize{\textbf{(0.026)}}} \\  

     & \mc{1}{c}{\scriptsize{7}} & \mc{1}{c}{\scriptsize{5.373}} & \mc{1}{c}{\scriptsize{5.714}} & \mc{1}{c}{\scriptsize{6.575}} & \mc{1}{c}{\scriptsize{6.926}} & \mc{1}{c}{\scriptsize{5.225}} & \mc{1}{c}{\scriptsize{5.066}} & \mc{1}{c}{\scriptsize{5.202}} & \mc{1}{c}{\scriptsize{5.537}} \\  

     &  & \mc{1}{c}{\scriptsize{\textbf{(0.013)}}} & \mc{1}{c}{\scriptsize{\textbf{(0.000)}}} & \mc{1}{c}{\scriptsize{\textbf{(0.026)}}} & \mc{1}{c}{\scriptsize{\textbf{(0.013)}}} & \mc{1}{c}{\scriptsize{\textbf{(0.066)}}} & \mc{1}{c}{\scriptsize{\textbf{(0.026)}}} & \mc{1}{c}{\scriptsize{\textbf{(0.026)}}} & \mc{1}{c}{\scriptsize{\textbf{(0.013)}}} \\  

     & \mc{1}{c}{\scriptsize{5}} & \mc{1}{c}{\scriptsize{6.362}} & \mc{1}{c}{\scriptsize{5.621}} & \mc{1}{c}{\scriptsize{8.310}} & \mc{1}{c}{\scriptsize{9.003}} & \mc{1}{c}{\scriptsize{8.749}} & \mc{1}{c}{\scriptsize{5.760}} & \mc{1}{c}{\scriptsize{4.776}} & \mc{1}{c}{\scriptsize{5.595}} \\  

     &  & \mc{1}{c}{\scriptsize{\textbf{(0.013)}}} & \mc{1}{c}{\scriptsize{\textbf{(0.013)}}} & \mc{1}{c}{\scriptsize{\textbf{(0.000)}}} & \mc{1}{c}{\scriptsize{\textbf{(0.000)}}} & \mc{1}{c}{\scriptsize{\textbf{(0.000)}}} & \mc{1}{c}{\scriptsize{\textbf{(0.013)}}} & \mc{1}{c}{\scriptsize{\textbf{(0.039)}}} & \mc{1}{c}{\scriptsize{\textbf{(0.013)}}} \\  

     & \mc{1}{c}{\scriptsize{12}} & \mc{1}{c}{\scriptsize{4.524}} & \mc{1}{c}{\scriptsize{3.612}} & \mc{1}{c}{\scriptsize{3.251}} & \mc{1}{c}{\scriptsize{2.881}} & \mc{1}{c}{\scriptsize{2.731}} & \mc{1}{c}{\scriptsize{4.766}} & \mc{1}{c}{\scriptsize{3.628}} & \mc{1}{c}{\scriptsize{3.579}} \\  

     &  & \mc{1}{c}{\scriptsize{\textbf{(0.000)}}} & \mc{1}{c}{\scriptsize{\textbf{(0.066)}}} & \mc{1}{c}{\scriptsize{(0.224)}} & \mc{1}{c}{\scriptsize{(0.276)}} & \mc{1}{c}{\scriptsize{(0.250)}} & \mc{1}{c}{\scriptsize{\textbf{(0.013)}}} & \mc{1}{c}{\scriptsize{\textbf{(0.053)}}} & \mc{1}{c}{\scriptsize{\textbf{(0.039)}}} \\  

     & \mc{1}{c}{\scriptsize{3.5}} & \mc{1}{c}{\scriptsize{8.387}} & \mc{1}{c}{\scriptsize{7.886}} & \mc{1}{c}{\scriptsize{11.255}} & \mc{1}{c}{\scriptsize{12.220}} & \mc{1}{c}{\scriptsize{11.438}} & \mc{1}{c}{\scriptsize{7.276}} & \mc{1}{c}{\scriptsize{6.711}} & \mc{1}{c}{\scriptsize{7.008}} \\  

     &  & \mc{1}{c}{\scriptsize{\textbf{(0.000)}}} & \mc{1}{c}{\scriptsize{\textbf{(0.000)}}} & \mc{1}{c}{\scriptsize{\textbf{(0.000)}}} & \mc{1}{c}{\scriptsize{\textbf{(0.000)}}} & \mc{1}{c}{\scriptsize{\textbf{(0.000)}}} & \mc{1}{c}{\scriptsize{\textbf{(0.000)}}} & \mc{1}{c}{\scriptsize{\textbf{(0.013)}}} & \mc{1}{c}{\scriptsize{\textbf{(0.000)}}} \\  

     & \mc{1}{c}{\scriptsize{2}} & \mc{1}{c}{\scriptsize{10.116}} & \mc{1}{c}{\scriptsize{10.461}} & \mc{1}{c}{\scriptsize{10.609}} & \mc{1}{c}{\scriptsize{12.312}} & \mc{1}{c}{\scriptsize{11.140}} & \mc{1}{c}{\scriptsize{9.863}} & \mc{1}{c}{\scriptsize{10.116}} & \mc{1}{c}{\scriptsize{10.214}} \\  

     &  & \mc{1}{c}{\scriptsize{\textbf{(0.000)}}} & \mc{1}{c}{\scriptsize{\textbf{(0.000)}}} & \mc{1}{c}{\scriptsize{\textbf{(0.000)}}} & \mc{1}{c}{\scriptsize{\textbf{(0.000)}}} & \mc{1}{c}{\scriptsize{\textbf{(0.000)}}} & \mc{1}{c}{\scriptsize{\textbf{(0.000)}}} & \mc{1}{c}{\scriptsize{\textbf{(0.000)}}} & \mc{1}{c}{\scriptsize{\textbf{(0.000)}}} \\  

     & \mc{1}{c}{\scriptsize{21}} & \mc{1}{c}{\scriptsize{4.425}} & \mc{1}{c}{\scriptsize{1.595}} & \mc{1}{c}{\scriptsize{4.549}} & \mc{1}{c}{\scriptsize{1.641}} & \mc{1}{c}{\scriptsize{3.129}} & \mc{1}{c}{\scriptsize{4.353}} & \mc{1}{c}{\scriptsize{1.530}} & \mc{1}{c}{\scriptsize{2.346}} \\  

     &  & \mc{1}{c}{\scriptsize{\textbf{(0.013)}}} & \mc{1}{c}{\scriptsize{(0.197)}} & \mc{1}{c}{\scriptsize{\textbf{(0.000)}}} & \mc{1}{c}{\scriptsize{(0.289)}} & \mc{1}{c}{\scriptsize{\textbf{(0.026)}}} & \mc{1}{c}{\scriptsize{\textbf{(0.039)}}} & \mc{1}{c}{\scriptsize{(0.250)}} & \mc{1}{c}{\scriptsize{(0.145)}} \\  

  \bottomrule
  \end{tabular}
	\end{table} 

	\begin{table}[H]
     \caption{Treatment Effects on Childhood and Adolescence Physical Health, Pooled Sample}
     \label{table:abccare_rslt_pooled_cat27}
	  \begin{tabular}{cccccccccc}
  \toprule

    \scriptsize{Variable} & \scriptsize{Age} & \scriptsize{(1)} & \scriptsize{(2)} & \scriptsize{(3)} & \scriptsize{(4)} & \scriptsize{(5)} & \scriptsize{(6)} & \scriptsize{(7)} & \scriptsize{(8)} \\ 
    \midrule  

    \mc{1}{l}{\scriptsize{Body Mass Index (BMI)}} & \mc{1}{c}{\scriptsize{0}} & \mc{1}{c}{\scriptsize{-0.739}} & \mc{1}{c}{\scriptsize{-0.223}} & \mc{1}{c}{\scriptsize{-2.490}} & \mc{1}{c}{\scriptsize{-1.810}} & \mc{1}{c}{\scriptsize{-1.884}} & \mc{1}{c}{\scriptsize{-0.358}} & \mc{1}{c}{\scriptsize{0.129}} & \mc{1}{c}{\scriptsize{-0.592}} \\  

     &  & \mc{1}{c}{\scriptsize{(0.211)}} & \mc{1}{c}{\scriptsize{(0.368)}} & \mc{1}{c}{\scriptsize{(0.171)}} & \mc{1}{c}{\scriptsize{(0.250)}} & \mc{1}{c}{\scriptsize{(0.145)}} & \mc{1}{c}{\scriptsize{(0.289)}} & \mc{1}{c}{\scriptsize{(0.500)}} & \mc{1}{c}{\scriptsize{(0.355)}} \\  

     & \mc{1}{c}{\scriptsize{0.25}} & \mc{1}{c}{\scriptsize{-0.880}} & \mc{1}{c}{\scriptsize{-0.710}} & \mc{1}{c}{\scriptsize{-1.141}} & \mc{1}{c}{\scriptsize{-0.987}} & \mc{1}{c}{\scriptsize{-1.084}} & \mc{1}{c}{\scriptsize{-0.965}} & \mc{1}{c}{\scriptsize{-0.773}} & \mc{1}{c}{\scriptsize{-0.902}} \\  

     &  & \mc{1}{c}{\scriptsize{\textbf{(0.000)}}} & \mc{1}{c}{\scriptsize{\textbf{(0.026)}}} & \mc{1}{c}{\scriptsize{\textbf{(0.000)}}} & \mc{1}{c}{\scriptsize{\textbf{(0.053)}}} & \mc{1}{c}{\scriptsize{\textbf{(0.000)}}} & \mc{1}{c}{\scriptsize{\textbf{(0.000)}}} & \mc{1}{c}{\scriptsize{\textbf{(0.066)}}} & \mc{1}{c}{\scriptsize{\textbf{(0.000)}}} \\  

     & \mc{1}{c}{\scriptsize{0.5}} & \mc{1}{c}{\scriptsize{-0.670}} & \mc{1}{c}{\scriptsize{-0.661}} & \mc{1}{c}{\scriptsize{-0.610}} & \mc{1}{c}{\scriptsize{-0.418}} & \mc{1}{c}{\scriptsize{-0.730}} & \mc{1}{c}{\scriptsize{-0.636}} & \mc{1}{c}{\scriptsize{-0.656}} & \mc{1}{c}{\scriptsize{-0.690}} \\  

     &  & \mc{1}{c}{\scriptsize{\textbf{(0.013)}}} & \mc{1}{c}{\scriptsize{\textbf{(0.013)}}} & \mc{1}{c}{\scriptsize{(0.132)}} & \mc{1}{c}{\scriptsize{(0.237)}} & \mc{1}{c}{\scriptsize{\textbf{(0.066)}}} & \mc{1}{c}{\scriptsize{\textbf{(0.000)}}} & \mc{1}{c}{\scriptsize{\textbf{(0.013)}}} & \mc{1}{c}{\scriptsize{\textbf{(0.013)}}} \\  

     & \mc{1}{c}{\scriptsize{0.75}} & \mc{1}{c}{\scriptsize{-1.412}} & \mc{1}{c}{\scriptsize{-1.329}} & \mc{1}{c}{\scriptsize{-1.382}} & \mc{1}{c}{\scriptsize{-1.279}} & \mc{1}{c}{\scriptsize{-1.170}} & \mc{1}{c}{\scriptsize{-1.479}} & \mc{1}{c}{\scriptsize{-1.363}} & \mc{1}{c}{\scriptsize{-1.307}} \\  

     &  & \mc{1}{c}{\scriptsize{\textbf{(0.000)}}} & \mc{1}{c}{\scriptsize{\textbf{(0.000)}}} & \mc{1}{c}{\scriptsize{\textbf{(0.026)}}} & \mc{1}{c}{\scriptsize{(0.105)}} & \mc{1}{c}{\scriptsize{\textbf{(0.013)}}} & \mc{1}{c}{\scriptsize{\textbf{(0.000)}}} & \mc{1}{c}{\scriptsize{\textbf{(0.000)}}} & \mc{1}{c}{\scriptsize{\textbf{(0.000)}}} \\  

     & \mc{1}{c}{\scriptsize{1}} & \mc{1}{c}{\scriptsize{-0.698}} & \mc{1}{c}{\scriptsize{-0.760}} & \mc{1}{c}{\scriptsize{-0.592}} & \mc{1}{c}{\scriptsize{-0.455}} & \mc{1}{c}{\scriptsize{-0.826}} & \mc{1}{c}{\scriptsize{-0.672}} & \mc{1}{c}{\scriptsize{-0.824}} & \mc{1}{c}{\scriptsize{-0.903}} \\  

     &  & \mc{1}{c}{\scriptsize{\textbf{(0.013)}}} & \mc{1}{c}{\scriptsize{\textbf{(0.013)}}} & \mc{1}{c}{\scriptsize{(0.145)}} & \mc{1}{c}{\scriptsize{(0.197)}} & \mc{1}{c}{\scriptsize{\textbf{(0.079)}}} & \mc{1}{c}{\scriptsize{\textbf{(0.039)}}} & \mc{1}{c}{\scriptsize{\textbf{(0.026)}}} & \mc{1}{c}{\scriptsize{\textbf{(0.013)}}} \\  

     & \mc{1}{c}{\scriptsize{1.5}} & \mc{1}{c}{\scriptsize{-0.798}} & \mc{1}{c}{\scriptsize{-0.823}} & \mc{1}{c}{\scriptsize{-0.934}} & \mc{1}{c}{\scriptsize{-0.791}} & \mc{1}{c}{\scriptsize{-1.140}} & \mc{1}{c}{\scriptsize{-0.803}} & \mc{1}{c}{\scriptsize{-0.809}} & \mc{1}{c}{\scriptsize{-0.933}} \\  

     &  & \mc{1}{c}{\scriptsize{\textbf{(0.000)}}} & \mc{1}{c}{\scriptsize{\textbf{(0.000)}}} & \mc{1}{c}{\scriptsize{\textbf{(0.039)}}} & \mc{1}{c}{\scriptsize{(0.105)}} & \mc{1}{c}{\scriptsize{\textbf{(0.000)}}} & \mc{1}{c}{\scriptsize{\textbf{(0.000)}}} & \mc{1}{c}{\scriptsize{\textbf{(0.013)}}} & \mc{1}{c}{\scriptsize{\textbf{(0.000)}}} \\  

     & \mc{1}{c}{\scriptsize{2}} & \mc{1}{c}{\scriptsize{-0.393}} & \mc{1}{c}{\scriptsize{-0.272}} & \mc{1}{c}{\scriptsize{-0.520}} & \mc{1}{c}{\scriptsize{-0.356}} & \mc{1}{c}{\scriptsize{-0.588}} & \mc{1}{c}{\scriptsize{-0.351}} & \mc{1}{c}{\scriptsize{-0.131}} & \mc{1}{c}{\scriptsize{-0.230}} \\  

     &  & \mc{1}{c}{\scriptsize{\textbf{(0.026)}}} & \mc{1}{c}{\scriptsize{(0.224)}} & \mc{1}{c}{\scriptsize{\textbf{(0.079)}}} & \mc{1}{c}{\scriptsize{(0.237)}} & \mc{1}{c}{\scriptsize{\textbf{(0.026)}}} & \mc{1}{c}{\scriptsize{(0.105)}} & \mc{1}{c}{\scriptsize{(0.342)}} & \mc{1}{c}{\scriptsize{(0.289)}} \\  

     & \mc{1}{c}{\scriptsize{2.5}} & \mc{1}{c}{\scriptsize{-0.071}} & \mc{1}{c}{\scriptsize{0.264}} & \mc{1}{c}{\scriptsize{0.250}} & \mc{1}{c}{\scriptsize{0.113}} & \mc{1}{c}{\scriptsize{0.635}} & \mc{1}{c}{\scriptsize{-0.310}} & \mc{1}{c}{\scriptsize{0.649}} & \mc{1}{c}{\scriptsize{-0.097}} \\  

     &  & \mc{1}{c}{\scriptsize{(0.382)}} & \mc{1}{c}{\scriptsize{(0.671)}} & \mc{1}{c}{\scriptsize{(0.697)}} & \mc{1}{c}{\scriptsize{(0.645)}} & \mc{1}{c}{\scriptsize{(0.789)}} & \mc{1}{c}{\scriptsize{(0.211)}} & \mc{1}{c}{\scriptsize{(0.829)}} & \mc{1}{c}{\scriptsize{(0.382)}} \\  

     & \mc{1}{c}{\scriptsize{3}} & \mc{1}{c}{\scriptsize{-0.255}} & \mc{1}{c}{\scriptsize{-0.042}} & \mc{1}{c}{\scriptsize{-0.377}} & \mc{1}{c}{\scriptsize{-0.059}} & \mc{1}{c}{\scriptsize{-0.390}} & \mc{1}{c}{\scriptsize{-0.221}} & \mc{1}{c}{\scriptsize{-0.027}} & \mc{1}{c}{\scriptsize{-0.133}} \\  

     &  & \mc{1}{c}{\scriptsize{(0.211)}} & \mc{1}{c}{\scriptsize{(0.434)}} & \mc{1}{c}{\scriptsize{(0.250)}} & \mc{1}{c}{\scriptsize{(0.500)}} & \mc{1}{c}{\scriptsize{(0.237)}} & \mc{1}{c}{\scriptsize{(0.250)}} & \mc{1}{c}{\scriptsize{(0.461)}} & \mc{1}{c}{\scriptsize{(0.342)}} \\  

     & \mc{1}{c}{\scriptsize{4}} & \mc{1}{c}{\scriptsize{0.378}} & \mc{1}{c}{\scriptsize{0.394}} & \mc{1}{c}{\scriptsize{0.558}} & \mc{1}{c}{\scriptsize{0.818}} & \mc{1}{c}{\scriptsize{0.407}} & \mc{1}{c}{\scriptsize{0.323}} & \mc{1}{c}{\scriptsize{0.343}} & \mc{1}{c}{\scriptsize{0.211}} \\  

     &  & \mc{1}{c}{\scriptsize{(0.895)}} & \mc{1}{c}{\scriptsize{(0.934)}} & \mc{1}{c}{\scriptsize{(0.908)}} & \mc{1}{c}{\scriptsize{(0.961)}} & \mc{1}{c}{\scriptsize{(0.855)}} & \mc{1}{c}{\scriptsize{(0.855)}} & \mc{1}{c}{\scriptsize{(0.868)}} & \mc{1}{c}{\scriptsize{(0.776)}} \\  

     & \mc{1}{c}{\scriptsize{5}} & \mc{1}{c}{\scriptsize{0.157}} & \mc{1}{c}{\scriptsize{0.290}} & \mc{1}{c}{\scriptsize{0.472}} & \mc{1}{c}{\scriptsize{0.780}} & \mc{1}{c}{\scriptsize{0.459}} & \mc{1}{c}{\scriptsize{0.068}} & \mc{1}{c}{\scriptsize{0.211}} & \mc{1}{c}{\scriptsize{0.093}} \\  

     &  & \mc{1}{c}{\scriptsize{(0.763)}} & \mc{1}{c}{\scriptsize{(0.816)}} & \mc{1}{c}{\scriptsize{(0.868)}} & \mc{1}{c}{\scriptsize{(0.947)}} & \mc{1}{c}{\scriptsize{(0.882)}} & \mc{1}{c}{\scriptsize{(0.566)}} & \mc{1}{c}{\scriptsize{(0.697)}} & \mc{1}{c}{\scriptsize{(0.632)}} \\  

     & \mc{1}{c}{\scriptsize{8}} & \mc{1}{c}{\scriptsize{0.403}} & \mc{1}{c}{\scriptsize{0.748}} & \mc{1}{c}{\scriptsize{0.713}} & \mc{1}{c}{\scriptsize{1.393}} & \mc{1}{c}{\scriptsize{0.650}} & \mc{1}{c}{\scriptsize{0.261}} & \mc{1}{c}{\scriptsize{0.654}} & \mc{1}{c}{\scriptsize{0.252}} \\  

     &  & \mc{1}{c}{\scriptsize{(0.776)}} & \mc{1}{c}{\scriptsize{(0.882)}} & \mc{1}{c}{\scriptsize{(0.829)}} & \mc{1}{c}{\scriptsize{(0.882)}} & \mc{1}{c}{\scriptsize{(0.855)}} & \mc{1}{c}{\scriptsize{(0.658)}} & \mc{1}{c}{\scriptsize{(0.816)}} & \mc{1}{c}{\scriptsize{(0.671)}} \\  

  \bottomrule
  \end{tabular}
	\end{table} 

	\begin{table}[H]
     \caption{Treatment Effects on Childhood Health Problems, Pooled Sample}
     \label{table:abccare_rslt_pooled_cat28}
	  \begin{tabular}{cccccccccc}
  \toprule

    \scriptsize{Variable} & \scriptsize{Age} & \scriptsize{(1)} & \scriptsize{(2)} & \scriptsize{(3)} & \scriptsize{(4)} & \scriptsize{(5)} & \scriptsize{(6)} & \scriptsize{(7)} & \scriptsize{(8)} \\ 
    \midrule  

    \mc{1}{l}{\scriptsize{Has Health Problems}} & \mc{1}{c}{\scriptsize{12}} & \mc{1}{c}{\scriptsize{0.058}} & \mc{1}{c}{\scriptsize{-0.048}} & \mc{1}{c}{\scriptsize{0.075}} & \mc{1}{c}{\scriptsize{0.040}} & \mc{1}{c}{\scriptsize{0.058}} & \mc{1}{c}{\scriptsize{0.050}} & \mc{1}{c}{\scriptsize{-0.073}} & \mc{1}{c}{\scriptsize{-0.009}} \\  

     &  & \mc{1}{c}{\scriptsize{(0.697)}} & \mc{1}{c}{\scriptsize{(0.342)}} & \mc{1}{c}{\scriptsize{(0.697)}} & \mc{1}{c}{\scriptsize{(0.566)}} & \mc{1}{c}{\scriptsize{(0.697)}} & \mc{1}{c}{\scriptsize{(0.684)}} & \mc{1}{c}{\scriptsize{(0.303)}} & \mc{1}{c}{\scriptsize{(0.513)}} \\  

    \mc{1}{l}{\scriptsize{Ever Hospitalized for Over 1 Week}} & \mc{1}{c}{\scriptsize{12}} & \mc{1}{c}{\scriptsize{-0.014}} & \mc{1}{c}{\scriptsize{-0.034}} & \mc{1}{c}{\scriptsize{0.088}} & \mc{1}{c}{\scriptsize{0.058}} & \mc{1}{c}{\scriptsize{0.074}} & \mc{1}{c}{\scriptsize{-0.043}} & \mc{1}{c}{\scriptsize{-0.061}} & \mc{1}{c}{\scriptsize{-0.052}} \\  

     &  & \mc{1}{c}{\scriptsize{(0.461)}} & \mc{1}{c}{\scriptsize{(0.329)}} & \mc{1}{c}{\scriptsize{(1.000)}} & \mc{1}{c}{\scriptsize{(0.842)}} & \mc{1}{c}{\scriptsize{(0.961)}} & \mc{1}{c}{\scriptsize{(0.276)}} & \mc{1}{c}{\scriptsize{(0.197)}} & \mc{1}{c}{\scriptsize{(0.171)}} \\  

  \bottomrule
  \end{tabular}
	\end{table} 

	\begin{table}[H]
     \caption{Treatment Effects on Cholesterol, Pooled Sample}
     \label{table:abccare_rslt_pooled_cat29}
	  \begin{tabular}{cccccccccc}
  \toprule

    \scriptsize{Variable} & \scriptsize{Age} & \scriptsize{(1)} & \scriptsize{(2)} & \scriptsize{(3)} & \scriptsize{(4)} & \scriptsize{(5)} & \scriptsize{(6)} & \scriptsize{(7)} & \scriptsize{(8)} \\ 
    \midrule  

    \mc{1}{l}{\scriptsize{Days drank alcohol last month}} & \mc{1}{c}{\scriptsize{30}} & \mc{1}{c}{\scriptsize{0.244}} & \mc{1}{c}{\scriptsize{0.419}} & \mc{1}{c}{\scriptsize{-0.156}} & \mc{1}{c}{\scriptsize{-0.700}} & \mc{1}{c}{\scriptsize{0.113}} & \mc{1}{c}{\scriptsize{0.207}} & \mc{1}{c}{\scriptsize{0.618}} & \mc{1}{c}{\scriptsize{0.630}} \\  

     &  & \mc{1}{c}{\scriptsize{(0.645)}} & \mc{1}{c}{\scriptsize{(0.618)}} & \mc{1}{c}{\scriptsize{(0.447)}} & \mc{1}{c}{\scriptsize{(0.342)}} & \mc{1}{c}{\scriptsize{(0.526)}} & \mc{1}{c}{\scriptsize{(0.566)}} & \mc{1}{c}{\scriptsize{(0.645)}} & \mc{1}{c}{\scriptsize{(0.658)}} \\  

    \mc{1}{l}{\scriptsize{Days binge drank alcohol last month}} & \mc{1}{c}{\scriptsize{30}} & \mc{1}{c}{\scriptsize{0.085}} & \mc{1}{c}{\scriptsize{0.412}} & \mc{1}{c}{\scriptsize{-0.267}} & \mc{1}{c}{\scriptsize{-0.218}} & \mc{1}{c}{\scriptsize{-0.126}} & \mc{1}{c}{\scriptsize{0.151}} & \mc{1}{c}{\scriptsize{0.674}} & \mc{1}{c}{\scriptsize{0.393}} \\  

     &  & \mc{1}{c}{\scriptsize{(0.632)}} & \mc{1}{c}{\scriptsize{(0.803)}} & \mc{1}{c}{\scriptsize{(0.395)}} & \mc{1}{c}{\scriptsize{(0.408)}} & \mc{1}{c}{\scriptsize{(0.434)}} & \mc{1}{c}{\scriptsize{(0.632)}} & \mc{1}{c}{\scriptsize{(0.882)}} & \mc{1}{c}{\scriptsize{(0.776)}} \\  

  \bottomrule
  \end{tabular}
	\end{table} 

	\begin{table}[H]
     \caption{Treatment Effects on Current Health Condition (Self-Reported), Pooled Sample}
     \label{table:abccare_rslt_pooled_cat30}
	\input{AppResOutput/abccare/rslt_pooled_cat30}
	\end{table} 

	\begin{table}[H]
     \caption{Treatment Effects on Diabetes, Pooled Sample}
     \label{table:abccare_rslt_pooled_cat31}
	\input{AppResOutput/abccare/rslt_pooled_cat31}
	\end{table} 

	\begin{table}[H]
     \caption{Treatment Effects on Drug Behavior and ASR Substance Scale, Pooled Sample}
     \label{table:abccare_rslt_pooled_cat32}
	  \begin{tabular}{cccccccccc}
  \toprule

    \scriptsize{Variable} & \scriptsize{Age} & \scriptsize{(1)} & \scriptsize{(2)} & \scriptsize{(3)} & \scriptsize{(4)} & \scriptsize{(5)} & \scriptsize{(6)} & \scriptsize{(7)} & \scriptsize{(8)} \\ 
    \midrule  

    \mc{1}{l}{\scriptsize{Cocaine: Smokes Reguarly}} & \mc{1}{c}{\scriptsize{30}} & \mc{1}{c}{\scriptsize{-0.028}} & \mc{1}{c}{\scriptsize{-0.022}} & \mc{1}{c}{\scriptsize{-0.139}} & \mc{1}{c}{\scriptsize{-0.137}} & \mc{1}{c}{\scriptsize{-0.145}} & \mc{1}{c}{\scriptsize{0.021}} & \mc{1}{c}{\scriptsize{0.017}} & \mc{1}{c}{\scriptsize{0.013}} \\  

     &  & \mc{1}{c}{\scriptsize{(0.237)}} & \mc{1}{c}{\scriptsize{(0.368)}} & \mc{1}{c}{\scriptsize{\textbf{(0.066)}}} & \mc{1}{c}{\scriptsize{\textbf{(0.053)}}} & \mc{1}{c}{\scriptsize{\textbf{(0.092)}}} & \mc{1}{c}{\scriptsize{(0.750)}} & \mc{1}{c}{\scriptsize{(0.618)}} & \mc{1}{c}{\scriptsize{(0.592)}} \\  

    \mc{1}{l}{\scriptsize{Marijuana: Times Used}} & \mc{1}{c}{\scriptsize{30}} & \mc{1}{c}{\scriptsize{0.011}} & \mc{1}{c}{\scriptsize{0.056}} & \mc{1}{c}{\scriptsize{-0.728}} & \mc{1}{c}{\scriptsize{-0.593}} & \mc{1}{c}{\scriptsize{-0.698}} & \mc{1}{c}{\scriptsize{0.343}} & \mc{1}{c}{\scriptsize{0.251}} & \mc{1}{c}{\scriptsize{0.404}} \\  

     &  & \mc{1}{c}{\scriptsize{(0.487)}} & \mc{1}{c}{\scriptsize{(0.566)}} & \mc{1}{c}{\scriptsize{(0.118)}} & \mc{1}{c}{\scriptsize{(0.211)}} & \mc{1}{c}{\scriptsize{(0.171)}} & \mc{1}{c}{\scriptsize{(0.724)}} & \mc{1}{c}{\scriptsize{(0.684)}} & \mc{1}{c}{\scriptsize{(0.789)}} \\  

    \mc{1}{l}{\scriptsize{Marijuana: Smokes Regularly}} & \mc{1}{c}{\scriptsize{30}} & \mc{1}{c}{\scriptsize{-0.116}} & \mc{1}{c}{\scriptsize{-0.147}} & \mc{1}{c}{\scriptsize{-0.189}} & \mc{1}{c}{\scriptsize{-0.235}} & \mc{1}{c}{\scriptsize{-0.194}} & \mc{1}{c}{\scriptsize{-0.085}} & \mc{1}{c}{\scriptsize{-0.092}} & \mc{1}{c}{\scriptsize{-0.075}} \\  

     &  & \mc{1}{c}{\scriptsize{\textbf{(0.026)}}} & \mc{1}{c}{\scriptsize{\textbf{(0.026)}}} & \mc{1}{c}{\scriptsize{\textbf{(0.053)}}} & \mc{1}{c}{\scriptsize{\textbf{(0.013)}}} & \mc{1}{c}{\scriptsize{\textbf{(0.053)}}} & \mc{1}{c}{\scriptsize{\textbf{(0.066)}}} & \mc{1}{c}{\scriptsize{(0.118)}} & \mc{1}{c}{\scriptsize{(0.105)}} \\  

    \mc{1}{l}{\scriptsize{Cocaine: Times Used}} & \mc{1}{c}{\scriptsize{30}} & \mc{1}{c}{\scriptsize{-0.246}} & \mc{1}{c}{\scriptsize{-0.217}} & \mc{1}{c}{\scriptsize{-0.872}} & \mc{1}{c}{\scriptsize{-0.833}} & \mc{1}{c}{\scriptsize{-0.876}} & \mc{1}{c}{\scriptsize{0.017}} & \mc{1}{c}{\scriptsize{-0.013}} & \mc{1}{c}{\scriptsize{0.006}} \\  

     &  & \mc{1}{c}{\scriptsize{\textbf{(0.079)}}} & \mc{1}{c}{\scriptsize{(0.184)}} & \mc{1}{c}{\scriptsize{\textbf{(0.039)}}} & \mc{1}{c}{\scriptsize{\textbf{(0.039)}}} & \mc{1}{c}{\scriptsize{\textbf{(0.039)}}} & \mc{1}{c}{\scriptsize{(0.553)}} & \mc{1}{c}{\scriptsize{(0.474)}} & \mc{1}{c}{\scriptsize{(0.434)}} \\  

    \mc{1}{l}{\scriptsize{Marijuana: Times Used in Past 30 Days}} & \mc{1}{c}{\scriptsize{30}} & \mc{1}{c}{\scriptsize{-0.442}} & \mc{1}{c}{\scriptsize{-0.491}} & \mc{1}{c}{\scriptsize{-0.665}} & \mc{1}{c}{\scriptsize{-0.725}} & \mc{1}{c}{\scriptsize{-0.640}} & \mc{1}{c}{\scriptsize{-0.327}} & \mc{1}{c}{\scriptsize{-0.267}} & \mc{1}{c}{\scriptsize{-0.223}} \\  

     &  & \mc{1}{c}{\scriptsize{\textbf{(0.026)}}} & \mc{1}{c}{\scriptsize{\textbf{(0.053)}}} & \mc{1}{c}{\scriptsize{(0.118)}} & \mc{1}{c}{\scriptsize{(0.118)}} & \mc{1}{c}{\scriptsize{(0.145)}} & \mc{1}{c}{\scriptsize{(0.105)}} & \mc{1}{c}{\scriptsize{(0.224)}} & \mc{1}{c}{\scriptsize{(0.197)}} \\  

    \mc{1}{l}{\scriptsize{Times Used Other Illegal Drugs in Past 30 Days}} & \mc{1}{c}{\scriptsize{21}} & \mc{1}{c}{\scriptsize{-0.079}} & \mc{1}{c}{\scriptsize{-0.072}} & \mc{1}{c}{\scriptsize{-0.087}} & \mc{1}{c}{\scriptsize{-0.063}} & \mc{1}{c}{\scriptsize{-0.078}} & \mc{1}{c}{\scriptsize{-0.084}} & \mc{1}{c}{\scriptsize{-0.090}} & \mc{1}{c}{\scriptsize{-0.082}} \\  

     &  & \mc{1}{c}{\scriptsize{\textbf{(0.092)}}} & \mc{1}{c}{\scriptsize{(0.118)}} & \mc{1}{c}{\scriptsize{(0.132)}} & \mc{1}{c}{\scriptsize{(0.211)}} & \mc{1}{c}{\scriptsize{(0.145)}} & \mc{1}{c}{\scriptsize{(0.118)}} & \mc{1}{c}{\scriptsize{(0.118)}} & \mc{1}{c}{\scriptsize{(0.118)}} \\  

    \mc{1}{l}{\scriptsize{ASR Substance Use Scale: Alcohol}} & \mc{1}{c}{\scriptsize{30}} & \mc{1}{c}{\scriptsize{0.679}} & \mc{1}{c}{\scriptsize{1.040}} & \mc{1}{c}{\scriptsize{0.794}} & \mc{1}{c}{\scriptsize{1.214}} & \mc{1}{c}{\scriptsize{1.004}} & \mc{1}{c}{\scriptsize{0.725}} & \mc{1}{c}{\scriptsize{1.100}} & \mc{1}{c}{\scriptsize{1.254}} \\  

     &  & \mc{1}{c}{\scriptsize{(0.803)}} & \mc{1}{c}{\scriptsize{(0.803)}} & \mc{1}{c}{\scriptsize{(0.671)}} & \mc{1}{c}{\scriptsize{(0.711)}} & \mc{1}{c}{\scriptsize{(0.724)}} & \mc{1}{c}{\scriptsize{(0.789)}} & \mc{1}{c}{\scriptsize{(0.816)}} & \mc{1}{c}{\scriptsize{(0.882)}} \\  

    \mc{1}{l}{\scriptsize{Marijuana: Times Used in Past 30 Days}} & \mc{1}{c}{\scriptsize{21}} & \mc{1}{c}{\scriptsize{-0.621}} & \mc{1}{c}{\scriptsize{-0.550}} & \mc{1}{c}{\scriptsize{-0.385}} & \mc{1}{c}{\scriptsize{-0.215}} & \mc{1}{c}{\scriptsize{-0.410}} & \mc{1}{c}{\scriptsize{-0.644}} & \mc{1}{c}{\scriptsize{-0.559}} & \mc{1}{c}{\scriptsize{-0.527}} \\  

     &  & \mc{1}{c}{\scriptsize{\textbf{(0.000)}}} & \mc{1}{c}{\scriptsize{\textbf{(0.026)}}} & \mc{1}{c}{\scriptsize{(0.224)}} & \mc{1}{c}{\scriptsize{(0.329)}} & \mc{1}{c}{\scriptsize{(0.197)}} & \mc{1}{c}{\scriptsize{\textbf{(0.000)}}} & \mc{1}{c}{\scriptsize{\textbf{(0.039)}}} & \mc{1}{c}{\scriptsize{\textbf{(0.026)}}} \\  

    \mc{1}{l}{\scriptsize{ASR Substance Use Scale: Mean Substance Abuse}} & \mc{1}{c}{\scriptsize{30}} & \mc{1}{c}{\scriptsize{0.661}} & \mc{1}{c}{\scriptsize{1.306}} & \mc{1}{c}{\scriptsize{0.495}} & \mc{1}{c}{\scriptsize{1.976}} & \mc{1}{c}{\scriptsize{0.720}} & \mc{1}{c}{\scriptsize{0.817}} & \mc{1}{c}{\scriptsize{1.601}} & \mc{1}{c}{\scriptsize{1.336}} \\  

     &  & \mc{1}{c}{\scriptsize{(0.789)}} & \mc{1}{c}{\scriptsize{(0.868)}} & \mc{1}{c}{\scriptsize{(0.671)}} & \mc{1}{c}{\scriptsize{(0.882)}} & \mc{1}{c}{\scriptsize{(0.724)}} & \mc{1}{c}{\scriptsize{(0.855)}} & \mc{1}{c}{\scriptsize{(0.934)}} & \mc{1}{c}{\scriptsize{(0.921)}} \\  

    \mc{1}{l}{\scriptsize{Cocaine: Number of Times Used Crack Cocaine}} & \mc{1}{c}{\scriptsize{30}} & \mc{1}{c}{\scriptsize{-0.162}} & \mc{1}{c}{\scriptsize{-0.037}} & \mc{1}{c}{\scriptsize{-0.923}} & \mc{1}{c}{\scriptsize{-0.815}} & \mc{1}{c}{\scriptsize{-0.908}} & \mc{1}{c}{\scriptsize{0.057}} & \mc{1}{c}{\scriptsize{0.117}} & \mc{1}{c}{\scriptsize{0.072}} \\  

     &  & \mc{1}{c}{\scriptsize{\textbf{(0.092)}}} & \mc{1}{c}{\scriptsize{(0.395)}} & \mc{1}{c}{\scriptsize{\textbf{(0.039)}}} & \mc{1}{c}{\scriptsize{\textbf{(0.039)}}} & \mc{1}{c}{\scriptsize{\textbf{(0.039)}}} & \mc{1}{c}{\scriptsize{(0.816)}} & \mc{1}{c}{\scriptsize{(0.868)}} & \mc{1}{c}{\scriptsize{(0.855)}} \\  

    \mc{1}{l}{\scriptsize{ASR Substance Use Scale: Tobacco}} & \mc{1}{c}{\scriptsize{30}} & \mc{1}{c}{\scriptsize{0.257}} & \mc{1}{c}{\scriptsize{0.341}} & \mc{1}{c}{\scriptsize{-1.226}} & \mc{1}{c}{\scriptsize{-0.522}} & \mc{1}{c}{\scriptsize{-1.372}} & \mc{1}{c}{\scriptsize{0.746}} & \mc{1}{c}{\scriptsize{0.773}} & \mc{1}{c}{\scriptsize{0.714}} \\  

     &  & \mc{1}{c}{\scriptsize{(0.645)}} & \mc{1}{c}{\scriptsize{(0.697)}} & \mc{1}{c}{\scriptsize{(0.211)}} & \mc{1}{c}{\scriptsize{(0.329)}} & \mc{1}{c}{\scriptsize{(0.145)}} & \mc{1}{c}{\scriptsize{(0.908)}} & \mc{1}{c}{\scriptsize{(0.842)}} & \mc{1}{c}{\scriptsize{(0.842)}} \\  

    \mc{1}{l}{\scriptsize{Marijuana: Smokes Regularly}} & \mc{1}{c}{\scriptsize{Mid-30s}} & \mc{1}{c}{\scriptsize{-0.026}} & \mc{1}{c}{\scriptsize{-0.085}} & \mc{1}{c}{\scriptsize{-0.112}} & \mc{1}{c}{\scriptsize{-0.134}} & \mc{1}{c}{\scriptsize{-0.111}} & \mc{1}{c}{\scriptsize{0.026}} & \mc{1}{c}{\scriptsize{-0.019}} & \mc{1}{c}{\scriptsize{-0.007}} \\  

     &  & \mc{1}{c}{\scriptsize{(0.355)}} & \mc{1}{c}{\scriptsize{(0.197)}} & \mc{1}{c}{\scriptsize{(0.224)}} & \mc{1}{c}{\scriptsize{(0.184)}} & \mc{1}{c}{\scriptsize{(0.224)}} & \mc{1}{c}{\scriptsize{(0.605)}} & \mc{1}{c}{\scriptsize{(0.461)}} & \mc{1}{c}{\scriptsize{(0.474)}} \\  

    \mc{1}{l}{\scriptsize{ASR Substance Use Scale: Drugs}} & \mc{1}{c}{\scriptsize{30}} & \mc{1}{c}{\scriptsize{-0.539}} & \mc{1}{c}{\scriptsize{-0.134}} & \mc{1}{c}{\scriptsize{-3.599}} & \mc{1}{c}{\scriptsize{-1.465}} & \mc{1}{c}{\scriptsize{-3.307}} & \mc{1}{c}{\scriptsize{0.399}} & \mc{1}{c}{\scriptsize{1.123}} & \mc{1}{c}{\scriptsize{1.023}} \\  

     &  & \mc{1}{c}{\scriptsize{(0.342)}} & \mc{1}{c}{\scriptsize{(0.461)}} & \mc{1}{c}{\scriptsize{(0.197)}} & \mc{1}{c}{\scriptsize{(0.355)}} & \mc{1}{c}{\scriptsize{(0.197)}} & \mc{1}{c}{\scriptsize{(0.658)}} & \mc{1}{c}{\scriptsize{(0.789)}} & \mc{1}{c}{\scriptsize{(0.776)}} \\  

  \bottomrule
  \end{tabular}
	\end{table} 

	\begin{table}[H]
     \caption{Treatment Effects on Health Insurance, Pooled Sample}
     \label{table:abccare_rslt_pooled_cat33}
	\input{AppResOutput/abccare/rslt_pooled_cat33}
	\end{table} 

	\begin{table}[H]
     \caption{Treatment Effects on Hypertension, Pooled Sample}
     \label{table:abccare_rslt_pooled_cat34}
	  \begin{tabular}{cccccccccc}
  \toprule

    \scriptsize{Variable} & \scriptsize{Age} & \scriptsize{(1)} & \scriptsize{(2)} & \scriptsize{(3)} & \scriptsize{(4)} & \scriptsize{(5)} & \scriptsize{(6)} & \scriptsize{(7)} & \scriptsize{(8)} \\ 
    \midrule  

    \mc{1}{l}{\scriptsize{Paranoid Ideation}} & \mc{1}{c}{\scriptsize{21}} & \mc{1}{c}{\scriptsize{-1.121}} & \mc{1}{c}{\scriptsize{-0.183}} & \mc{1}{c}{\scriptsize{-4.286}} & \mc{1}{c}{\scriptsize{-3.908}} & \mc{1}{c}{\scriptsize{-3.655}} & \mc{1}{c}{\scriptsize{-0.174}} & \mc{1}{c}{\scriptsize{0.891}} & \mc{1}{c}{\scriptsize{0.535}} \\  

     &  & \mc{1}{c}{\scriptsize{(0.250)}} & \mc{1}{c}{\scriptsize{(0.421)}} & \mc{1}{c}{\scriptsize{\textbf{(0.026)}}} & \mc{1}{c}{\scriptsize{\textbf{(0.053)}}} & \mc{1}{c}{\scriptsize{\textbf{(0.039)}}} & \mc{1}{c}{\scriptsize{(0.461)}} & \mc{1}{c}{\scriptsize{(0.592)}} & \mc{1}{c}{\scriptsize{(0.579)}} \\  

    \mc{1}{l}{\scriptsize{Obsessive-Compulsive}} & \mc{1}{c}{\scriptsize{21}} & \mc{1}{c}{\scriptsize{-2.963}} & \mc{1}{c}{\scriptsize{-2.702}} & \mc{1}{c}{\scriptsize{-1.792}} & \mc{1}{c}{\scriptsize{-1.868}} & \mc{1}{c}{\scriptsize{-1.704}} & \mc{1}{c}{\scriptsize{-3.505}} & \mc{1}{c}{\scriptsize{-3.157}} & \mc{1}{c}{\scriptsize{-3.420}} \\  

     &  & \mc{1}{c}{\scriptsize{\textbf{(0.026)}}} & \mc{1}{c}{\scriptsize{\textbf{(0.092)}}} & \mc{1}{c}{\scriptsize{(0.184)}} & \mc{1}{c}{\scriptsize{(0.197)}} & \mc{1}{c}{\scriptsize{(0.158)}} & \mc{1}{c}{\scriptsize{\textbf{(0.000)}}} & \mc{1}{c}{\scriptsize{\textbf{(0.053)}}} & \mc{1}{c}{\scriptsize{\textbf{(0.013)}}} \\  

    \mc{1}{l}{\scriptsize{Interpersonal Sense}} & \mc{1}{c}{\scriptsize{21}} & \mc{1}{c}{\scriptsize{-2.109}} & \mc{1}{c}{\scriptsize{-2.553}} & \mc{1}{c}{\scriptsize{-4.127}} & \mc{1}{c}{\scriptsize{-5.768}} & \mc{1}{c}{\scriptsize{-4.144}} & \mc{1}{c}{\scriptsize{-1.533}} & \mc{1}{c}{\scriptsize{-1.939}} & \mc{1}{c}{\scriptsize{-1.447}} \\  

     &  & \mc{1}{c}{\scriptsize{\textbf{(0.079)}}} & \mc{1}{c}{\scriptsize{\textbf{(0.053)}}} & \mc{1}{c}{\scriptsize{\textbf{(0.053)}}} & \mc{1}{c}{\scriptsize{\textbf{(0.013)}}} & \mc{1}{c}{\scriptsize{\textbf{(0.053)}}} & \mc{1}{c}{\scriptsize{(0.158)}} & \mc{1}{c}{\scriptsize{(0.118)}} & \mc{1}{c}{\scriptsize{(0.237)}} \\  

    \mc{1}{l}{\scriptsize{Positive Symptom Distress Index (PSI)}} & \mc{1}{c}{\scriptsize{21}} & \mc{1}{c}{\scriptsize{-3.119}} & \mc{1}{c}{\scriptsize{-2.170}} & \mc{1}{c}{\scriptsize{-3.181}} & \mc{1}{c}{\scriptsize{-2.662}} & \mc{1}{c}{\scriptsize{-3.450}} & \mc{1}{c}{\scriptsize{-2.874}} & \mc{1}{c}{\scriptsize{-1.880}} & \mc{1}{c}{\scriptsize{-3.159}} \\  

     &  & \mc{1}{c}{\scriptsize{\textbf{(0.000)}}} & \mc{1}{c}{\scriptsize{(0.118)}} & \mc{1}{c}{\scriptsize{\textbf{(0.079)}}} & \mc{1}{c}{\scriptsize{(0.145)}} & \mc{1}{c}{\scriptsize{\textbf{(0.039)}}} & \mc{1}{c}{\scriptsize{\textbf{(0.039)}}} & \mc{1}{c}{\scriptsize{(0.145)}} & \mc{1}{c}{\scriptsize{\textbf{(0.066)}}} \\  

    \mc{1}{l}{\scriptsize{Psychoticism}} & \mc{1}{c}{\scriptsize{21}} & \mc{1}{c}{\scriptsize{-4.769}} & \mc{1}{c}{\scriptsize{-4.703}} & \mc{1}{c}{\scriptsize{-8.020}} & \mc{1}{c}{\scriptsize{-7.015}} & \mc{1}{c}{\scriptsize{-8.503}} & \mc{1}{c}{\scriptsize{-3.809}} & \mc{1}{c}{\scriptsize{-3.603}} & \mc{1}{c}{\scriptsize{-4.225}} \\  

     &  & \mc{1}{c}{\scriptsize{\textbf{(0.013)}}} & \mc{1}{c}{\scriptsize{\textbf{(0.000)}}} & \mc{1}{c}{\scriptsize{\textbf{(0.000)}}} & \mc{1}{c}{\scriptsize{\textbf{(0.000)}}} & \mc{1}{c}{\scriptsize{\textbf{(0.000)}}} & \mc{1}{c}{\scriptsize{\textbf{(0.013)}}} & \mc{1}{c}{\scriptsize{\textbf{(0.026)}}} & \mc{1}{c}{\scriptsize{\textbf{(0.026)}}} \\  

    \mc{1}{l}{\scriptsize{Phobic Anxiety}} & \mc{1}{c}{\scriptsize{21}} & \mc{1}{c}{\scriptsize{-2.980}} & \mc{1}{c}{\scriptsize{-3.808}} & \mc{1}{c}{\scriptsize{-4.298}} & \mc{1}{c}{\scriptsize{-7.409}} & \mc{1}{c}{\scriptsize{-4.557}} & \mc{1}{c}{\scriptsize{-2.547}} & \mc{1}{c}{\scriptsize{-2.779}} & \mc{1}{c}{\scriptsize{-2.963}} \\  

     &  & \mc{1}{c}{\scriptsize{\textbf{(0.039)}}} & \mc{1}{c}{\scriptsize{\textbf{(0.013)}}} & \mc{1}{c}{\scriptsize{\textbf{(0.000)}}} & \mc{1}{c}{\scriptsize{\textbf{(0.000)}}} & \mc{1}{c}{\scriptsize{\textbf{(0.000)}}} & \mc{1}{c}{\scriptsize{\textbf{(0.079)}}} & \mc{1}{c}{\scriptsize{\textbf{(0.053)}}} & \mc{1}{c}{\scriptsize{\textbf{(0.053)}}} \\  

    \mc{1}{l}{\scriptsize{Positive Symptom Total (PST)}} & \mc{1}{c}{\scriptsize{21}} & \mc{1}{c}{\scriptsize{-2.638}} & \mc{1}{c}{\scriptsize{-2.409}} & \mc{1}{c}{\scriptsize{-4.521}} & \mc{1}{c}{\scriptsize{-5.389}} & \mc{1}{c}{\scriptsize{-4.835}} & \mc{1}{c}{\scriptsize{-2.247}} & \mc{1}{c}{\scriptsize{-1.972}} & \mc{1}{c}{\scriptsize{-2.491}} \\  

     &  & \mc{1}{c}{\scriptsize{\textbf{(0.039)}}} & \mc{1}{c}{\scriptsize{\textbf{(0.092)}}} & \mc{1}{c}{\scriptsize{\textbf{(0.092)}}} & \mc{1}{c}{\scriptsize{\textbf{(0.066)}}} & \mc{1}{c}{\scriptsize{\textbf{(0.079)}}} & \mc{1}{c}{\scriptsize{\textbf{(0.092)}}} & \mc{1}{c}{\scriptsize{(0.158)}} & \mc{1}{c}{\scriptsize{\textbf{(0.066)}}} \\  

  \bottomrule
  \end{tabular}
	\end{table} 

	\begin{table}[H]
     \caption{Treatment Effects on Laboratory Test  - Metabolic Panel, Pooled Sample}
     \label{table:abccare_rslt_pooled_cat35}
	\input{AppResOutput/abccare/rslt_pooled_cat35}
	\end{table} 

	\begin{table}[H]
     \caption{Treatment Effects on Laboratory Test - Complete Blood Count, Pooled Sample}
     \label{table:abccare_rslt_pooled_cat36}
	  \begin{tabular}{cccccccccc}
  \toprule

    \scriptsize{Variable} & \scriptsize{Age} & \scriptsize{(1)} & \scriptsize{(2)} & \scriptsize{(3)} & \scriptsize{(4)} & \scriptsize{(5)} & \scriptsize{(6)} & \scriptsize{(7)} & \scriptsize{(8)} \\ 
    \midrule  

    \mc{1}{l}{\scriptsize{Mean Cell Volum}} & \mc{1}{c}{\scriptsize{Mid-30s}} & \mc{1}{c}{\scriptsize{-0.609}} & \mc{1}{c}{\scriptsize{-1.158}} & \mc{1}{c}{\scriptsize{0.493}} & \mc{1}{c}{\scriptsize{-0.284}} & \mc{1}{c}{\scriptsize{0.112}} & \mc{1}{c}{\scriptsize{-0.647}} & \mc{1}{c}{\scriptsize{-0.968}} & \mc{1}{c}{\scriptsize{-1.112}} \\  

     &  & \mc{1}{c}{\scriptsize{(0.684)}} & \mc{1}{c}{\scriptsize{(0.789)}} & \mc{1}{c}{\scriptsize{(0.342)}} & \mc{1}{c}{\scriptsize{(0.434)}} & \mc{1}{c}{\scriptsize{(0.408)}} & \mc{1}{c}{\scriptsize{(0.684)}} & \mc{1}{c}{\scriptsize{(0.737)}} & \mc{1}{c}{\scriptsize{(0.724)}} \\  

    \mc{1}{l}{\scriptsize{Platelets}} & \mc{1}{c}{\scriptsize{Mid-30s}} & \mc{1}{c}{\scriptsize{-8.494}} & \mc{1}{c}{\scriptsize{4.525}} & \mc{1}{c}{\scriptsize{-8.494}} & \mc{1}{c}{\scriptsize{8.562}} & \mc{1}{c}{\scriptsize{-2.563}} & \mc{1}{c}{\scriptsize{-10.082}} & \mc{1}{c}{\scriptsize{2.057}} & \mc{1}{c}{\scriptsize{-5.018}} \\  

     &  & \mc{1}{c}{\scriptsize{(0.789)}} & \mc{1}{c}{\scriptsize{(0.329)}} & \mc{1}{c}{\scriptsize{(0.632)}} & \mc{1}{c}{\scriptsize{(0.382)}} & \mc{1}{c}{\scriptsize{(0.553)}} & \mc{1}{c}{\scriptsize{(0.829)}} & \mc{1}{c}{\scriptsize{(0.368)}} & \mc{1}{c}{\scriptsize{(0.697)}} \\  

    \mc{1}{l}{\scriptsize{Eosinophils}} & \mc{1}{c}{\scriptsize{Mid-30s}} & \mc{1}{c}{\scriptsize{0.528}} & \mc{1}{c}{\scriptsize{0.646}} & \mc{1}{c}{\scriptsize{0.881}} & \mc{1}{c}{\scriptsize{1.482}} & \mc{1}{c}{\scriptsize{1.013}} & \mc{1}{c}{\scriptsize{0.476}} & \mc{1}{c}{\scriptsize{0.453}} & \mc{1}{c}{\scriptsize{0.574}} \\  

     &  & \mc{1}{c}{\scriptsize{(0.105)}} & \mc{1}{c}{\scriptsize{\textbf{(0.053)}}} & \mc{1}{c}{\scriptsize{\textbf{(0.013)}}} & \mc{1}{c}{\scriptsize{\textbf{(0.026)}}} & \mc{1}{c}{\scriptsize{\textbf{(0.013)}}} & \mc{1}{c}{\scriptsize{(0.145)}} & \mc{1}{c}{\scriptsize{(0.105)}} & \mc{1}{c}{\scriptsize{(0.118)}} \\  

    \mc{1}{l}{\scriptsize{Hemoglobin}} & \mc{1}{c}{\scriptsize{Mid-30s}} & \mc{1}{c}{\scriptsize{0.167}} & \mc{1}{c}{\scriptsize{-0.400}} & \mc{1}{c}{\scriptsize{0.291}} & \mc{1}{c}{\scriptsize{0.004}} & \mc{1}{c}{\scriptsize{0.007}} & \mc{1}{c}{\scriptsize{0.153}} & \mc{1}{c}{\scriptsize{-0.470}} & \mc{1}{c}{\scriptsize{-0.217}} \\  

     &  & \mc{1}{c}{\scriptsize{(0.368)}} & \mc{1}{c}{\scriptsize{(0.934)}} & \mc{1}{c}{\scriptsize{(0.250)}} & \mc{1}{c}{\scriptsize{(0.447)}} & \mc{1}{c}{\scriptsize{(0.447)}} & \mc{1}{c}{\scriptsize{(0.355)}} & \mc{1}{c}{\scriptsize{(0.921)}} & \mc{1}{c}{\scriptsize{(0.711)}} \\  

    \mc{1}{l}{\scriptsize{Red Cells}} & \mc{1}{c}{\scriptsize{Mid-30s}} & \mc{1}{c}{\scriptsize{0.075}} & \mc{1}{c}{\scriptsize{-0.074}} & \mc{1}{c}{\scriptsize{0.108}} & \mc{1}{c}{\scriptsize{0.052}} & \mc{1}{c}{\scriptsize{0.035}} & \mc{1}{c}{\scriptsize{0.058}} & \mc{1}{c}{\scriptsize{-0.116}} & \mc{1}{c}{\scriptsize{-0.022}} \\  

     &  & \mc{1}{c}{\scriptsize{(0.250)}} & \mc{1}{c}{\scriptsize{(0.776)}} & \mc{1}{c}{\scriptsize{(0.184)}} & \mc{1}{c}{\scriptsize{(0.329)}} & \mc{1}{c}{\scriptsize{(0.434)}} & \mc{1}{c}{\scriptsize{(0.276)}} & \mc{1}{c}{\scriptsize{(0.908)}} & \mc{1}{c}{\scriptsize{(0.605)}} \\  

    \mc{1}{l}{\scriptsize{Lymphocytes}} & \mc{1}{c}{\scriptsize{Mid-30s}} & \mc{1}{c}{\scriptsize{-2.184}} & \mc{1}{c}{\scriptsize{-2.444}} & \mc{1}{c}{\scriptsize{-1.847}} & \mc{1}{c}{\scriptsize{-1.061}} & \mc{1}{c}{\scriptsize{-1.726}} & \mc{1}{c}{\scriptsize{-1.782}} & \mc{1}{c}{\scriptsize{-1.907}} & \mc{1}{c}{\scriptsize{-1.615}} \\  

     &  & \mc{1}{c}{\scriptsize{(0.868)}} & \mc{1}{c}{\scriptsize{(0.882)}} & \mc{1}{c}{\scriptsize{(0.658)}} & \mc{1}{c}{\scriptsize{(0.605)}} & \mc{1}{c}{\scriptsize{(0.671)}} & \mc{1}{c}{\scriptsize{(0.842)}} & \mc{1}{c}{\scriptsize{(0.829)}} & \mc{1}{c}{\scriptsize{(0.750)}} \\  

    \mc{1}{l}{\scriptsize{Monocytes}} & \mc{1}{c}{\scriptsize{Mid-30s}} & \mc{1}{c}{\scriptsize{0.218}} & \mc{1}{c}{\scriptsize{0.260}} & \mc{1}{c}{\scriptsize{0.204}} & \mc{1}{c}{\scriptsize{0.360}} & \mc{1}{c}{\scriptsize{0.112}} & \mc{1}{c}{\scriptsize{0.216}} & \mc{1}{c}{\scriptsize{0.248}} & \mc{1}{c}{\scriptsize{0.095}} \\  

     &  & \mc{1}{c}{\scriptsize{(0.276)}} & \mc{1}{c}{\scriptsize{(0.303)}} & \mc{1}{c}{\scriptsize{(0.395)}} & \mc{1}{c}{\scriptsize{(0.342)}} & \mc{1}{c}{\scriptsize{(0.395)}} & \mc{1}{c}{\scriptsize{(0.303)}} & \mc{1}{c}{\scriptsize{(0.303)}} & \mc{1}{c}{\scriptsize{(0.368)}} \\  

    \mc{1}{l}{\scriptsize{Neutrophils}} & \mc{1}{c}{\scriptsize{Mid-30s}} & \mc{1}{c}{\scriptsize{1.432}} & \mc{1}{c}{\scriptsize{1.537}} & \mc{1}{c}{\scriptsize{0.684}} & \mc{1}{c}{\scriptsize{-0.901}} & \mc{1}{c}{\scriptsize{0.528}} & \mc{1}{c}{\scriptsize{1.096}} & \mc{1}{c}{\scriptsize{1.224}} & \mc{1}{c}{\scriptsize{0.959}} \\  

     &  & \mc{1}{c}{\scriptsize{(0.250)}} & \mc{1}{c}{\scriptsize{(0.197)}} & \mc{1}{c}{\scriptsize{(0.434)}} & \mc{1}{c}{\scriptsize{(0.539)}} & \mc{1}{c}{\scriptsize{(0.461)}} & \mc{1}{c}{\scriptsize{(0.263)}} & \mc{1}{c}{\scriptsize{(0.316)}} & \mc{1}{c}{\scriptsize{(0.316)}} \\  

    \mc{1}{l}{\scriptsize{Basophils}} & \mc{1}{c}{\scriptsize{Mid-30s}} & \mc{1}{c}{\scriptsize{-0.013}} & \mc{1}{c}{\scriptsize{0.002}} & \mc{1}{c}{\scriptsize{0.056}} & \mc{1}{c}{\scriptsize{0.119}} & \mc{1}{c}{\scriptsize{0.073}} & \mc{1}{c}{\scriptsize{-0.027}} & \mc{1}{c}{\scriptsize{-0.018}} & \mc{1}{c}{\scriptsize{-0.012}} \\  

     &  & \mc{1}{c}{\scriptsize{(0.592)}} & \mc{1}{c}{\scriptsize{(0.500)}} & \mc{1}{c}{\scriptsize{(0.197)}} & \mc{1}{c}{\scriptsize{\textbf{(0.026)}}} & \mc{1}{c}{\scriptsize{(0.118)}} & \mc{1}{c}{\scriptsize{(0.697)}} & \mc{1}{c}{\scriptsize{(0.671)}} & \mc{1}{c}{\scriptsize{(0.605)}} \\  

    \mc{1}{l}{\scriptsize{Mean Hemoglobin}} & \mc{1}{c}{\scriptsize{Mid-30s}} & \mc{1}{c}{\scriptsize{-0.184}} & \mc{1}{c}{\scriptsize{-0.435}} & \mc{1}{c}{\scriptsize{0.027}} & \mc{1}{c}{\scriptsize{-0.214}} & \mc{1}{c}{\scriptsize{-0.126}} & \mc{1}{c}{\scriptsize{-0.132}} & \mc{1}{c}{\scriptsize{-0.342}} & \mc{1}{c}{\scriptsize{-0.414}} \\  

     &  & \mc{1}{c}{\scriptsize{(0.658)}} & \mc{1}{c}{\scriptsize{(0.789)}} & \mc{1}{c}{\scriptsize{(0.474)}} & \mc{1}{c}{\scriptsize{(0.526)}} & \mc{1}{c}{\scriptsize{(0.526)}} & \mc{1}{c}{\scriptsize{(0.592)}} & \mc{1}{c}{\scriptsize{(0.697)}} & \mc{1}{c}{\scriptsize{(0.724)}} \\  

    \mc{1}{l}{\scriptsize{White Cells}} & \mc{1}{c}{\scriptsize{Mid-30s}} & \mc{1}{c}{\scriptsize{0.352}} & \mc{1}{c}{\scriptsize{0.199}} & \mc{1}{c}{\scriptsize{-0.568}} & \mc{1}{c}{\scriptsize{-0.488}} & \mc{1}{c}{\scriptsize{-0.494}} & \mc{1}{c}{\scriptsize{0.585}} & \mc{1}{c}{\scriptsize{0.323}} & \mc{1}{c}{\scriptsize{0.414}} \\  

     &  & \mc{1}{c}{\scriptsize{(0.184)}} & \mc{1}{c}{\scriptsize{(0.316)}} & \mc{1}{c}{\scriptsize{(0.697)}} & \mc{1}{c}{\scriptsize{(0.632)}} & \mc{1}{c}{\scriptsize{(0.671)}} & \mc{1}{c}{\scriptsize{\textbf{(0.066)}}} & \mc{1}{c}{\scriptsize{(0.211)}} & \mc{1}{c}{\scriptsize{(0.184)}} \\  

    \mc{1}{l}{\scriptsize{Hematocrit}} & \mc{1}{c}{\scriptsize{Mid-30s}} & \mc{1}{c}{\scriptsize{0.462}} & \mc{1}{c}{\scriptsize{-1.105}} & \mc{1}{c}{\scriptsize{1.086}} & \mc{1}{c}{\scriptsize{0.208}} & \mc{1}{c}{\scriptsize{0.297}} & \mc{1}{c}{\scriptsize{0.327}} & \mc{1}{c}{\scriptsize{-1.360}} & \mc{1}{c}{\scriptsize{-0.576}} \\  

     &  & \mc{1}{c}{\scriptsize{(0.329)}} & \mc{1}{c}{\scriptsize{(0.947)}} & \mc{1}{c}{\scriptsize{(0.224)}} & \mc{1}{c}{\scriptsize{(0.421)}} & \mc{1}{c}{\scriptsize{(0.382)}} & \mc{1}{c}{\scriptsize{(0.368)}} & \mc{1}{c}{\scriptsize{(0.961)}} & \mc{1}{c}{\scriptsize{(0.697)}} \\  

    \mc{1}{l}{\scriptsize{Red Cell Width}} & \mc{1}{c}{\scriptsize{Mid-30s}} & \mc{1}{c}{\scriptsize{-0.142}} & \mc{1}{c}{\scriptsize{0.001}} & \mc{1}{c}{\scriptsize{-0.673}} & \mc{1}{c}{\scriptsize{-0.261}} & \mc{1}{c}{\scriptsize{-0.414}} & \mc{1}{c}{\scriptsize{-0.029}} & \mc{1}{c}{\scriptsize{0.075}} & \mc{1}{c}{\scriptsize{0.179}} \\  

     &  & \mc{1}{c}{\scriptsize{(0.724)}} & \mc{1}{c}{\scriptsize{(0.513)}} & \mc{1}{c}{\scriptsize{(0.882)}} & \mc{1}{c}{\scriptsize{(0.697)}} & \mc{1}{c}{\scriptsize{(0.816)}} & \mc{1}{c}{\scriptsize{(0.500)}} & \mc{1}{c}{\scriptsize{(0.408)}} & \mc{1}{c}{\scriptsize{(0.289)}} \\  

    \mc{1}{l}{\scriptsize{Mean Hb Concentration}} & \mc{1}{c}{\scriptsize{Mid-30s}} & \mc{1}{c}{\scriptsize{0.029}} & \mc{1}{c}{\scriptsize{-0.079}} & \mc{1}{c}{\scriptsize{-0.196}} & \mc{1}{c}{\scriptsize{-0.211}} & \mc{1}{c}{\scriptsize{-0.263}} & \mc{1}{c}{\scriptsize{0.113}} & \mc{1}{c}{\scriptsize{-0.033}} & \mc{1}{c}{\scriptsize{-0.069}} \\  

     &  & \mc{1}{c}{\scriptsize{(0.408)}} & \mc{1}{c}{\scriptsize{(0.658)}} & \mc{1}{c}{\scriptsize{(0.711)}} & \mc{1}{c}{\scriptsize{(0.697)}} & \mc{1}{c}{\scriptsize{(0.789)}} & \mc{1}{c}{\scriptsize{(0.316)}} & \mc{1}{c}{\scriptsize{(0.566)}} & \mc{1}{c}{\scriptsize{(0.579)}} \\  

  \bottomrule
  \end{tabular}
	\end{table} 

	\begin{table}[H]
     \caption{Treatment Effects on Other Health-Related Information, Pooled Sample}
     \label{table:abccare_rslt_pooled_cat37}
	  \begin{tabular}{cccccccccc}
  \toprule

    \scriptsize{Variable} & \scriptsize{Age} & \scriptsize{(1)} & \scriptsize{(2)} & \scriptsize{(3)} & \scriptsize{(4)} & \scriptsize{(5)} & \scriptsize{(6)} & \scriptsize{(7)} & \scriptsize{(8)} \\ 
    \midrule  

    \mc{1}{l}{\scriptsize{Number of Days Very Healthy in Past 30 Days}} & \mc{1}{c}{\scriptsize{Mid-30s}} & \mc{1}{c}{\scriptsize{3.116}} & \mc{1}{c}{\scriptsize{2.181}} & \mc{1}{c}{\scriptsize{-0.027}} & \mc{1}{c}{\scriptsize{0.134}} & \mc{1}{c}{\scriptsize{-0.727}} & \mc{1}{c}{\scriptsize{3.609}} & \mc{1}{c}{\scriptsize{2.053}} & \mc{1}{c}{\scriptsize{2.966}} \\  

     &  & \mc{1}{c}{\scriptsize{\textbf{(0.053)}}} & \mc{1}{c}{\scriptsize{(0.132)}} & \mc{1}{c}{\scriptsize{(0.461)}} & \mc{1}{c}{\scriptsize{(0.526)}} & \mc{1}{c}{\scriptsize{(0.539)}} & \mc{1}{c}{\scriptsize{\textbf{(0.039)}}} & \mc{1}{c}{\scriptsize{(0.250)}} & \mc{1}{c}{\scriptsize{(0.118)}} \\  

    \mc{1}{l}{\scriptsize{How Subject Thinks of Own Weight}} & \mc{1}{c}{\scriptsize{30}} & \mc{1}{c}{\scriptsize{0.140}} & \mc{1}{c}{\scriptsize{0.281}} & \mc{1}{c}{\scriptsize{-0.036}} & \mc{1}{c}{\scriptsize{0.234}} & \mc{1}{c}{\scriptsize{0.005}} & \mc{1}{c}{\scriptsize{0.191}} & \mc{1}{c}{\scriptsize{0.275}} & \mc{1}{c}{\scriptsize{0.203}} \\  

     &  & \mc{1}{c}{\scriptsize{(0.842)}} & \mc{1}{c}{\scriptsize{(0.961)}} & \mc{1}{c}{\scriptsize{(0.434)}} & \mc{1}{c}{\scriptsize{(0.868)}} & \mc{1}{c}{\scriptsize{(0.461)}} & \mc{1}{c}{\scriptsize{(0.855)}} & \mc{1}{c}{\scriptsize{(0.961)}} & \mc{1}{c}{\scriptsize{(0.921)}} \\  

    \mc{1}{l}{\scriptsize{Number of Days in Pain in Past 30 Days}} & \mc{1}{c}{\scriptsize{Mid-30s}} & \mc{1}{c}{\scriptsize{1.139}} & \mc{1}{c}{\scriptsize{0.219}} & \mc{1}{c}{\scriptsize{0.687}} & \mc{1}{c}{\scriptsize{0.569}} & \mc{1}{c}{\scriptsize{0.539}} & \mc{1}{c}{\scriptsize{1.989}} & \mc{1}{c}{\scriptsize{0.844}} & \mc{1}{c}{\scriptsize{1.595}} \\  

     &  & \mc{1}{c}{\scriptsize{(0.803)}} & \mc{1}{c}{\scriptsize{(0.566)}} & \mc{1}{c}{\scriptsize{(0.592)}} & \mc{1}{c}{\scriptsize{(0.579)}} & \mc{1}{c}{\scriptsize{(0.618)}} & \mc{1}{c}{\scriptsize{(0.921)}} & \mc{1}{c}{\scriptsize{(0.684)}} & \mc{1}{c}{\scriptsize{(0.842)}} \\  

    \mc{1}{l}{\scriptsize{Physical/Nervous Condition Prevents Work}} & \mc{1}{c}{\scriptsize{30}} & \mc{1}{c}{\scriptsize{0.003}} & \mc{1}{c}{\scriptsize{0.015}} & \mc{1}{c}{\scriptsize{-0.056}} & \mc{1}{c}{\scriptsize{-0.056}} & \mc{1}{c}{\scriptsize{-0.059}} & \mc{1}{c}{\scriptsize{0.017}} & \mc{1}{c}{\scriptsize{0.020}} & \mc{1}{c}{\scriptsize{0.012}} \\  

     &  & \mc{1}{c}{\scriptsize{(0.566)}} & \mc{1}{c}{\scriptsize{(0.618)}} & \mc{1}{c}{\scriptsize{(0.263)}} & \mc{1}{c}{\scriptsize{(0.211)}} & \mc{1}{c}{\scriptsize{(0.276)}} & \mc{1}{c}{\scriptsize{(0.605)}} & \mc{1}{c}{\scriptsize{(0.658)}} & \mc{1}{c}{\scriptsize{(0.592)}} \\  

  \bottomrule
  \end{tabular}
	\end{table} 

	\begin{table}[H]
     \caption{Treatment Effects on Past Medical History - Diagnosis (Self-Reported), Pooled Sample}
     \label{table:abccare_rslt_pooled_cat38}
	  \begin{tabular}{cccccccccc}
  \toprule

    \scriptsize{Variable} & \scriptsize{Age} & \scriptsize{(1)} & \scriptsize{(2)} & \scriptsize{(3)} & \scriptsize{(4)} & \scriptsize{(5)} & \scriptsize{(6)} & \scriptsize{(7)} & \scriptsize{(8)} \\ 
    \midrule  

    \mc{1}{l}{\scriptsize{Ever Told Had: Arthritis/Gout/Lupus/Fibromyalgia}} & \mc{1}{c}{\scriptsize{Mid-30s}} & \mc{1}{c}{\scriptsize{0.071}} & \mc{1}{c}{\scriptsize{0.044}} & \mc{1}{c}{\scriptsize{-0.102}} & \mc{1}{c}{\scriptsize{-0.135}} & \mc{1}{c}{\scriptsize{-0.124}} & \mc{1}{c}{\scriptsize{0.120}} & \mc{1}{c}{\scriptsize{0.091}} & \mc{1}{c}{\scriptsize{0.091}} \\  

     &  & \mc{1}{c}{\scriptsize{(0.868)}} & \mc{1}{c}{\scriptsize{(0.724)}} & \mc{1}{c}{\scriptsize{(0.224)}} & \mc{1}{c}{\scriptsize{(0.211)}} & \mc{1}{c}{\scriptsize{(0.211)}} & \mc{1}{c}{\scriptsize{(0.987)}} & \mc{1}{c}{\scriptsize{(0.882)}} & \mc{1}{c}{\scriptsize{(0.934)}} \\  

    \mc{1}{l}{\scriptsize{Ever Told Had: Prediabetes}} & \mc{1}{c}{\scriptsize{Mid-30s}} & \mc{1}{c}{\scriptsize{-0.002}} & \mc{1}{c}{\scriptsize{0.029}} & \mc{1}{c}{\scriptsize{-0.152}} & \mc{1}{c}{\scriptsize{-0.027}} & \mc{1}{c}{\scriptsize{-0.146}} & \mc{1}{c}{\scriptsize{0.030}} & \mc{1}{c}{\scriptsize{0.054}} & \mc{1}{c}{\scriptsize{0.028}} \\  

     &  & \mc{1}{c}{\scriptsize{(0.526)}} & \mc{1}{c}{\scriptsize{(0.592)}} & \mc{1}{c}{\scriptsize{(0.211)}} & \mc{1}{c}{\scriptsize{(0.395)}} & \mc{1}{c}{\scriptsize{(0.237)}} & \mc{1}{c}{\scriptsize{(0.605)}} & \mc{1}{c}{\scriptsize{(0.697)}} & \mc{1}{c}{\scriptsize{(0.618)}} \\  

  \bottomrule
  \end{tabular}
	\end{table} 

	\begin{table}[H]
     \caption{Treatment Effects on Past Medical History - Surgery (Self-Reported), Pooled Sample}
     \label{table:abccare_rslt_pooled_cat39}
	  \begin{tabular}{cccccccccc}
  \toprule

    \scriptsize{Variable} & \scriptsize{Age} & \scriptsize{(1)} & \scriptsize{(2)} & \scriptsize{(3)} & \scriptsize{(4)} & \scriptsize{(5)} & \scriptsize{(6)} & \scriptsize{(7)} & \scriptsize{(8)} \\ 
    \midrule  

    \mc{1}{l}{\scriptsize{Past Surgery: Cholecystectomy}} & \mc{1}{c}{\scriptsize{Mid-30s}} & \mc{1}{c}{\scriptsize{0.019}} & \mc{1}{c}{\scriptsize{0.009}} & \mc{1}{c}{\scriptsize{0.043}} & \mc{1}{c}{\scriptsize{0.043}} & \mc{1}{c}{\scriptsize{0.048}} & \mc{1}{c}{\scriptsize{0.011}} & \mc{1}{c}{\scriptsize{-0.005}} & \mc{1}{c}{\scriptsize{0.006}} \\  

     &  & \mc{1}{c}{\scriptsize{(0.658)}} & \mc{1}{c}{\scriptsize{(0.474)}} & \mc{1}{c}{\scriptsize{(0.750)}} & \mc{1}{c}{\scriptsize{(0.697)}} & \mc{1}{c}{\scriptsize{(0.776)}} & \mc{1}{c}{\scriptsize{(0.500)}} & \mc{1}{c}{\scriptsize{(0.408)}} & \mc{1}{c}{\scriptsize{(0.434)}} \\  

    \mc{1}{l}{\scriptsize{Past Surgery: Orthopedic Surgery}} & \mc{1}{c}{\scriptsize{Mid-30s}} & \mc{1}{c}{\scriptsize{-0.048}} & \mc{1}{c}{\scriptsize{-0.046}} & \mc{1}{c}{\scriptsize{-0.111}} & \mc{1}{c}{\scriptsize{-0.107}} & \mc{1}{c}{\scriptsize{-0.107}} & \mc{1}{c}{\scriptsize{-0.031}} & \mc{1}{c}{\scriptsize{-0.033}} & \mc{1}{c}{\scriptsize{-0.031}} \\  

     &  & \mc{1}{c}{\scriptsize{\textbf{(0.053)}}} & \mc{1}{c}{\scriptsize{\textbf{(0.092)}}} & \mc{1}{c}{\scriptsize{\textbf{(0.066)}}} & \mc{1}{c}{\scriptsize{\textbf{(0.053)}}} & \mc{1}{c}{\scriptsize{\textbf{(0.066)}}} & \mc{1}{c}{\scriptsize{\textbf{(0.066)}}} & \mc{1}{c}{\scriptsize{(0.105)}} & \mc{1}{c}{\scriptsize{\textbf{(0.079)}}} \\  

    \mc{1}{l}{\scriptsize{Past Surgery: Appendectomy}} & \mc{1}{c}{\scriptsize{Mid-30s}} & \mc{1}{c}{\scriptsize{-0.026}} & \mc{1}{c}{\scriptsize{-0.008}} & \mc{1}{c}{\scriptsize{-0.090}} & \mc{1}{c}{\scriptsize{-0.044}} & \mc{1}{c}{\scriptsize{-0.080}} & \mc{1}{c}{\scriptsize{-0.010}} & \mc{1}{c}{\scriptsize{0.014}} & \mc{1}{c}{\scriptsize{-0.005}} \\  

     &  & \mc{1}{c}{\scriptsize{(0.211)}} & \mc{1}{c}{\scriptsize{(0.474)}} & \mc{1}{c}{\scriptsize{(0.145)}} & \mc{1}{c}{\scriptsize{(0.342)}} & \mc{1}{c}{\scriptsize{(0.184)}} & \mc{1}{c}{\scriptsize{(0.368)}} & \mc{1}{c}{\scriptsize{(0.553)}} & \mc{1}{c}{\scriptsize{(0.382)}} \\  

    \mc{1}{l}{\scriptsize{Past Surgery: Ectopic Pregnancy}} & \mc{1}{c}{\scriptsize{Mid-30s}} & \mc{1}{c}{\scriptsize{-0.003}} & \mc{1}{c}{\scriptsize{0.009}} & \mc{1}{c}{\scriptsize{0.021}} & \mc{1}{c}{\scriptsize{0.004}} & \mc{1}{c}{\scriptsize{0.026}} & \mc{1}{c}{\scriptsize{-0.010}} & \mc{1}{c}{\scriptsize{0.015}} & \mc{1}{c}{\scriptsize{-0.005}} \\  

     &  & \mc{1}{c}{\scriptsize{(0.329)}} & \mc{1}{c}{\scriptsize{(0.447)}} & \mc{1}{c}{\scriptsize{(0.579)}} & \mc{1}{c}{\scriptsize{(0.421)}} & \mc{1}{c}{\scriptsize{(0.592)}} & \mc{1}{c}{\scriptsize{(0.289)}} & \mc{1}{c}{\scriptsize{(0.526)}} & \mc{1}{c}{\scriptsize{(0.342)}} \\  

    \mc{1}{l}{\scriptsize{Past Surgery: Hysterectomy}} & \mc{1}{c}{\scriptsize{Mid-30s}} & \mc{1}{c}{\scriptsize{-0.003}} & \mc{1}{c}{\scriptsize{0.013}} & \mc{1}{c}{\scriptsize{-0.090}} & \mc{1}{c}{\scriptsize{-0.040}} & \mc{1}{c}{\scriptsize{-0.081}} & \mc{1}{c}{\scriptsize{0.021}} & \mc{1}{c}{\scriptsize{0.037}} & \mc{1}{c}{\scriptsize{0.025}} \\  

     &  & \mc{1}{c}{\scriptsize{(0.395)}} & \mc{1}{c}{\scriptsize{(0.539)}} & \mc{1}{c}{\scriptsize{(0.237)}} & \mc{1}{c}{\scriptsize{(0.329)}} & \mc{1}{c}{\scriptsize{(0.224)}} & \mc{1}{c}{\scriptsize{(0.671)}} & \mc{1}{c}{\scriptsize{(0.658)}} & \mc{1}{c}{\scriptsize{(0.684)}} \\  

  \bottomrule
  \end{tabular}
	\end{table} 

	\begin{table}[H]
     \caption{Treatment Effects on Physical Activity, Pooled Sample}
     \label{table:abccare_rslt_pooled_cat40}
	\input{AppResOutput/abccare/rslt_pooled_cat40}
	\end{table} 

	\begin{table}[H]
     \caption{Treatment Effects on Physical Exam - Ear, Pooled Sample}
     \label{table:abccare_rslt_pooled_cat41}
	\input{AppResOutput/abccare/rslt_pooled_cat41}
	\end{table} 

	\begin{table}[H]
     \caption{Treatment Effects on Physical Exam - General I, Pooled Sample}
     \label{table:abccare_rslt_pooled_cat42}
	  \begin{tabular}{cccccccccc}
  \toprule

    \scriptsize{Variable} & \scriptsize{Age} & \scriptsize{(1)} & \scriptsize{(2)} & \scriptsize{(3)} & \scriptsize{(4)} & \scriptsize{(5)} & \scriptsize{(6)} & \scriptsize{(7)} & \scriptsize{(8)} \\ 
    \midrule  

    \mc{1}{l}{\scriptsize{Respirations}} & \mc{1}{c}{\scriptsize{Mid-30s}} & \mc{1}{c}{\scriptsize{-0.411}} & \mc{1}{c}{\scriptsize{-0.417}} & \mc{1}{c}{\scriptsize{-0.853}} & \mc{1}{c}{\scriptsize{-0.492}} & \mc{1}{c}{\scriptsize{-0.954}} & \mc{1}{c}{\scriptsize{-0.286}} & \mc{1}{c}{\scriptsize{-0.392}} & \mc{1}{c}{\scriptsize{-0.197}} \\  

     &  & \mc{1}{c}{\scriptsize{(0.184)}} & \mc{1}{c}{\scriptsize{(0.184)}} & \mc{1}{c}{\scriptsize{\textbf{(0.000)}}} & \mc{1}{c}{\scriptsize{(0.197)}} & \mc{1}{c}{\scriptsize{\textbf{(0.000)}}} & \mc{1}{c}{\scriptsize{(0.303)}} & \mc{1}{c}{\scriptsize{(0.211)}} & \mc{1}{c}{\scriptsize{(0.329)}} \\  

    \mc{1}{l}{\scriptsize{Temp (F)}} & \mc{1}{c}{\scriptsize{Mid-30s}} & \mc{1}{c}{\scriptsize{0.089}} & \mc{1}{c}{\scriptsize{0.042}} & \mc{1}{c}{\scriptsize{-0.084}} & \mc{1}{c}{\scriptsize{-0.071}} & \mc{1}{c}{\scriptsize{-0.080}} & \mc{1}{c}{\scriptsize{0.144}} & \mc{1}{c}{\scriptsize{0.107}} & \mc{1}{c}{\scriptsize{0.115}} \\  

     &  & \mc{1}{c}{\scriptsize{(0.803)}} & \mc{1}{c}{\scriptsize{(0.671)}} & \mc{1}{c}{\scriptsize{(0.289)}} & \mc{1}{c}{\scriptsize{(0.355)}} & \mc{1}{c}{\scriptsize{(0.316)}} & \mc{1}{c}{\scriptsize{(0.895)}} & \mc{1}{c}{\scriptsize{(0.816)}} & \mc{1}{c}{\scriptsize{(0.855)}} \\  

    \mc{1}{l}{\scriptsize{Pulse}} & \mc{1}{c}{\scriptsize{Mid-30s}} & \mc{1}{c}{\scriptsize{-1.097}} & \mc{1}{c}{\scriptsize{-1.017}} & \mc{1}{c}{\scriptsize{-9.009}} & \mc{1}{c}{\scriptsize{-7.600}} & \mc{1}{c}{\scriptsize{-8.686}} & \mc{1}{c}{\scriptsize{1.119}} & \mc{1}{c}{\scriptsize{0.467}} & \mc{1}{c}{\scriptsize{1.072}} \\  

     &  & \mc{1}{c}{\scriptsize{(0.316)}} & \mc{1}{c}{\scriptsize{(0.355)}} & \mc{1}{c}{\scriptsize{\textbf{(0.026)}}} & \mc{1}{c}{\scriptsize{\textbf{(0.092)}}} & \mc{1}{c}{\scriptsize{\textbf{(0.053)}}} & \mc{1}{c}{\scriptsize{(0.684)}} & \mc{1}{c}{\scriptsize{(0.618)}} & \mc{1}{c}{\scriptsize{(0.684)}} \\  

    \mc{1}{l}{\scriptsize{Voice}} & \mc{1}{c}{\scriptsize{Mid-30s}} &  &  &  &  &  &  &  &  \\  

     &  &  &  &  &  &  &  &  &  \\  

    \mc{1}{l}{\scriptsize{Orientation}} & \mc{1}{c}{\scriptsize{Mid-30s}} &  &  &  &  &  &  &  &  \\  

     &  &  &  &  &  &  &  &  &  \\  

    \mc{1}{l}{\scriptsize{Nutrition}} & \mc{1}{c}{\scriptsize{Mid-30s}} & \mc{1}{c}{\scriptsize{-0.075}} & \mc{1}{c}{\scriptsize{0.041}} & \mc{1}{c}{\scriptsize{-0.099}} & \mc{1}{c}{\scriptsize{-0.059}} & \mc{1}{c}{\scriptsize{-0.065}} & \mc{1}{c}{\scriptsize{-0.079}} & \mc{1}{c}{\scriptsize{0.052}} & \mc{1}{c}{\scriptsize{0.004}} \\  

     &  & \mc{1}{c}{\scriptsize{(0.145)}} & \mc{1}{c}{\scriptsize{(0.632)}} & \mc{1}{c}{\scriptsize{(0.303)}} & \mc{1}{c}{\scriptsize{(0.342)}} & \mc{1}{c}{\scriptsize{(0.342)}} & \mc{1}{c}{\scriptsize{(0.224)}} & \mc{1}{c}{\scriptsize{(0.671)}} & \mc{1}{c}{\scriptsize{(0.461)}} \\  

    \mc{1}{l}{\scriptsize{Hydration}} & \mc{1}{c}{\scriptsize{Mid-30s}} &  &  &  &  &  &  &  &  \\  

     &  &  &  &  &  &  &  &  &  \\  

    \mc{1}{l}{\scriptsize{Posture}} & \mc{1}{c}{\scriptsize{Mid-30s}} & \mc{1}{c}{\scriptsize{0.043}} & \mc{1}{c}{\scriptsize{0.067}} & \mc{1}{c}{\scriptsize{0.043}} & \mc{1}{c}{\scriptsize{0.003}} & \mc{1}{c}{\scriptsize{0.050}} & \mc{1}{c}{\scriptsize{0.043}} & \mc{1}{c}{\scriptsize{0.087}} & \mc{1}{c}{\scriptsize{0.050}} \\  

     &  & \mc{1}{c}{\scriptsize{(0.579)}} & \mc{1}{c}{\scriptsize{(0.579)}} & \mc{1}{c}{\scriptsize{(0.579)}} & \mc{1}{c}{\scriptsize{(0.355)}} & \mc{1}{c}{\scriptsize{(0.605)}} & \mc{1}{c}{\scriptsize{(0.579)}} & \mc{1}{c}{\scriptsize{(0.579)}} & \mc{1}{c}{\scriptsize{(0.605)}} \\  

  \bottomrule
  \end{tabular}
	\end{table} 

	\begin{table}[H]
     \caption{Treatment Effects on Physical Exam - General II, Pooled Sample}
     \label{table:abccare_rslt_pooled_cat43}
	  \begin{tabular}{cccccccccc}
  \toprule

    \scriptsize{Variable} & \scriptsize{Age} & \scriptsize{(1)} & \scriptsize{(2)} & \scriptsize{(3)} & \scriptsize{(4)} & \scriptsize{(5)} & \scriptsize{(6)} & \scriptsize{(7)} & \scriptsize{(8)} \\ 
    \midrule  

    \mc{1}{l}{\scriptsize{Chest and Lung General}} & \mc{1}{c}{\scriptsize{Mid-30s}} & \mc{1}{c}{\scriptsize{0.043}} & \mc{1}{c}{\scriptsize{0.058}} & \mc{1}{c}{\scriptsize{0.043}} & \mc{1}{c}{\scriptsize{0.014}} & \mc{1}{c}{\scriptsize{0.051}} & \mc{1}{c}{\scriptsize{0.043}} & \mc{1}{c}{\scriptsize{0.064}} & \mc{1}{c}{\scriptsize{0.052}} \\  

     &  & \mc{1}{c}{\scriptsize{(0.539)}} & \mc{1}{c}{\scriptsize{(0.605)}} & \mc{1}{c}{\scriptsize{(0.539)}} & \mc{1}{c}{\scriptsize{(0.382)}} & \mc{1}{c}{\scriptsize{(0.539)}} & \mc{1}{c}{\scriptsize{(0.539)}} & \mc{1}{c}{\scriptsize{(0.605)}} & \mc{1}{c}{\scriptsize{(0.539)}} \\  

    \mc{1}{l}{\scriptsize{Cardiovascular General}} & \mc{1}{c}{\scriptsize{Mid-30s}} & \mc{1}{c}{\scriptsize{-0.024}} &  &  &  &  & \mc{1}{c}{\scriptsize{-0.031}} &  &  \\  

     &  & \mc{1}{c}{\scriptsize{\textbf{(0.053)}}} &  &  &  &  & \mc{1}{c}{\scriptsize{\textbf{(0.066)}}} &  &  \\  

    \mc{1}{l}{\scriptsize{Skin General}} & \mc{1}{c}{\scriptsize{Mid-30s}} & \mc{1}{c}{\scriptsize{-0.058}} & \mc{1}{c}{\scriptsize{-0.054}} & \mc{1}{c}{\scriptsize{-0.137}} & \mc{1}{c}{\scriptsize{-0.102}} & \mc{1}{c}{\scriptsize{-0.138}} & \mc{1}{c}{\scriptsize{-0.040}} & \mc{1}{c}{\scriptsize{-0.067}} & \mc{1}{c}{\scriptsize{-0.052}} \\  

     &  & \mc{1}{c}{\scriptsize{(0.171)}} & \mc{1}{c}{\scriptsize{(0.184)}} & \mc{1}{c}{\scriptsize{(0.211)}} & \mc{1}{c}{\scriptsize{(0.224)}} & \mc{1}{c}{\scriptsize{(0.184)}} & \mc{1}{c}{\scriptsize{(0.211)}} & \mc{1}{c}{\scriptsize{(0.171)}} & \mc{1}{c}{\scriptsize{(0.237)}} \\  

    \mc{1}{l}{\scriptsize{Musculoskeletal General}} & \mc{1}{c}{\scriptsize{Mid-30s}} & \mc{1}{c}{\scriptsize{0.021}} & \mc{1}{c}{\scriptsize{0.025}} & \mc{1}{c}{\scriptsize{0.021}} & \mc{1}{c}{\scriptsize{0.042}} & \mc{1}{c}{\scriptsize{0.018}} & \mc{1}{c}{\scriptsize{0.021}} & \mc{1}{c}{\scriptsize{0.025}} & \mc{1}{c}{\scriptsize{0.018}} \\  

     &  & \mc{1}{c}{\scriptsize{(0.579)}} & \mc{1}{c}{\scriptsize{(0.526)}} & \mc{1}{c}{\scriptsize{(0.579)}} & \mc{1}{c}{\scriptsize{(0.513)}} & \mc{1}{c}{\scriptsize{(0.592)}} & \mc{1}{c}{\scriptsize{(0.579)}} & \mc{1}{c}{\scriptsize{(0.539)}} & \mc{1}{c}{\scriptsize{(0.592)}} \\  

    \mc{1}{l}{\scriptsize{Head General}} & \mc{1}{c}{\scriptsize{Mid-30s}} & \mc{1}{c}{\scriptsize{-0.024}} & \mc{1}{c}{\scriptsize{-0.014}} & \mc{1}{c}{\scriptsize{-0.111}} & \mc{1}{c}{\scriptsize{-0.111}} & \mc{1}{c}{\scriptsize{-0.106}} &  &  &  \\  

     &  & \mc{1}{c}{\scriptsize{\textbf{(0.092)}}} & \mc{1}{c}{\scriptsize{(0.132)}} & \mc{1}{c}{\scriptsize{\textbf{(0.079)}}} & \mc{1}{c}{\scriptsize{\textbf{(0.053)}}} & \mc{1}{c}{\scriptsize{\textbf{(0.079)}}} &  &  &  \\  

  \bottomrule
  \end{tabular}
	\end{table} 

	\begin{table}[H]
     \caption{Treatment Effects on Physical Exam (Part II), Pooled Sample}
     \label{table:abccare_rslt_pooled_cat44}
	  \begin{tabular}{cccccccccc}
  \toprule

    \scriptsize{Variable} & \scriptsize{Age} & \scriptsize{(1)} & \scriptsize{(2)} & \scriptsize{(3)} & \scriptsize{(4)} & \scriptsize{(5)} & \scriptsize{(6)} & \scriptsize{(7)} & \scriptsize{(8)} \\ 
    \midrule  

    \mc{1}{l}{\scriptsize{Mouth and Throat: Upper Teeth}} & \mc{1}{c}{\scriptsize{Mid-30s}} & \mc{1}{c}{\scriptsize{0.089}} & \mc{1}{c}{\scriptsize{0.117}} & \mc{1}{c}{\scriptsize{0.033}} & \mc{1}{c}{\scriptsize{0.122}} & \mc{1}{c}{\scriptsize{0.042}} & \mc{1}{c}{\scriptsize{0.099}} & \mc{1}{c}{\scriptsize{0.138}} & \mc{1}{c}{\scriptsize{0.170}} \\  

     &  & \mc{1}{c}{\scriptsize{(0.855)}} & \mc{1}{c}{\scriptsize{(0.895)}} & \mc{1}{c}{\scriptsize{(0.566)}} & \mc{1}{c}{\scriptsize{(0.816)}} & \mc{1}{c}{\scriptsize{(0.605)}} & \mc{1}{c}{\scriptsize{(0.855)}} & \mc{1}{c}{\scriptsize{(0.947)}} & \mc{1}{c}{\scriptsize{(0.974)}} \\  

    \mc{1}{l}{\scriptsize{Muscle Strength: Reflexes}} & \mc{1}{c}{\scriptsize{Mid-30s}} & \mc{1}{c}{\scriptsize{0.085}} & \mc{1}{c}{\scriptsize{0.128}} & \mc{1}{c}{\scriptsize{0.085}} & \mc{1}{c}{\scriptsize{0.108}} & \mc{1}{c}{\scriptsize{0.092}} & \mc{1}{c}{\scriptsize{0.085}} & \mc{1}{c}{\scriptsize{0.136}} & \mc{1}{c}{\scriptsize{0.092}} \\  

     &  & \mc{1}{c}{\scriptsize{(1.000)}} & \mc{1}{c}{\scriptsize{(0.987)}} & \mc{1}{c}{\scriptsize{(1.000)}} & \mc{1}{c}{\scriptsize{(0.921)}} & \mc{1}{c}{\scriptsize{(1.000)}} & \mc{1}{c}{\scriptsize{(1.000)}} & \mc{1}{c}{\scriptsize{(1.000)}} & \mc{1}{c}{\scriptsize{(1.000)}} \\  

    \mc{1}{l}{\scriptsize{Mouth and Throat: Lower Teeth}} & \mc{1}{c}{\scriptsize{Mid-30s}} & \mc{1}{c}{\scriptsize{0.006}} & \mc{1}{c}{\scriptsize{-0.010}} & \mc{1}{c}{\scriptsize{-0.073}} & \mc{1}{c}{\scriptsize{-0.101}} & \mc{1}{c}{\scriptsize{-0.076}} & \mc{1}{c}{\scriptsize{0.024}} & \mc{1}{c}{\scriptsize{0.027}} & \mc{1}{c}{\scriptsize{0.071}} \\  

     &  & \mc{1}{c}{\scriptsize{(0.553)}} & \mc{1}{c}{\scriptsize{(0.474)}} & \mc{1}{c}{\scriptsize{(0.250)}} & \mc{1}{c}{\scriptsize{(0.224)}} & \mc{1}{c}{\scriptsize{(0.289)}} & \mc{1}{c}{\scriptsize{(0.605)}} & \mc{1}{c}{\scriptsize{(0.632)}} & \mc{1}{c}{\scriptsize{(0.803)}} \\  

    \mc{1}{l}{\scriptsize{Muscle Strength: Coordination}} & \mc{1}{c}{\scriptsize{Mid-30s}} &  &  &  &  &  &  &  &  \\  

     &  &  &  &  &  &  &  &  &  \\  

  \bottomrule
  \end{tabular}
	\end{table} 

	\begin{table}[H]
     \caption{Treatment Effects on Age 21 Brief Symptom Inventory, Pooled Sample}
     \label{table:abccare_rslt_pooled_cat45}
	\input{AppResOutput/abccare/rslt_pooled_cat45}
	\end{table} 

	\begin{table}[H]
     \caption{Treatment Effects on Age 30 Adult Self Report DSM Scale $t$-Score, Pooled Sample}
     \label{table:abccare_rslt_pooled_cat46}
	  \begin{tabular}{cccccccccc}
  \toprule

    \scriptsize{Variable} & \scriptsize{Age} & \scriptsize{(1)} & \scriptsize{(2)} & \scriptsize{(3)} & \scriptsize{(4)} & \scriptsize{(5)} & \scriptsize{(6)} & \scriptsize{(7)} & \scriptsize{(8)} \\ 
    \midrule  

    \mc{1}{l}{\scriptsize{Somatic Problems}} & \mc{1}{c}{\scriptsize{30}} & \mc{1}{c}{\scriptsize{1.139}} & \mc{1}{c}{\scriptsize{0.323}} & \mc{1}{c}{\scriptsize{0.882}} & \mc{1}{c}{\scriptsize{-0.518}} & \mc{1}{c}{\scriptsize{0.363}} & \mc{1}{c}{\scriptsize{1.055}} & \mc{1}{c}{\scriptsize{0.344}} & \mc{1}{c}{\scriptsize{0.537}} \\  

     &  & \mc{1}{c}{\scriptsize{(0.868)}} & \mc{1}{c}{\scriptsize{(0.566)}} & \mc{1}{c}{\scriptsize{(0.724)}} & \mc{1}{c}{\scriptsize{(0.316)}} & \mc{1}{c}{\scriptsize{(0.579)}} & \mc{1}{c}{\scriptsize{(0.803)}} & \mc{1}{c}{\scriptsize{(0.618)}} & \mc{1}{c}{\scriptsize{(0.671)}} \\  

    \mc{1}{l}{\scriptsize{AD/H Problems}} & \mc{1}{c}{\scriptsize{30}} & \mc{1}{c}{\scriptsize{-0.170}} & \mc{1}{c}{\scriptsize{-0.953}} & \mc{1}{c}{\scriptsize{0.339}} & \mc{1}{c}{\scriptsize{-0.755}} & \mc{1}{c}{\scriptsize{0.011}} & \mc{1}{c}{\scriptsize{-0.421}} & \mc{1}{c}{\scriptsize{-1.068}} & \mc{1}{c}{\scriptsize{-0.873}} \\  

     &  & \mc{1}{c}{\scriptsize{(0.382)}} & \mc{1}{c}{\scriptsize{(0.132)}} & \mc{1}{c}{\scriptsize{(0.645)}} & \mc{1}{c}{\scriptsize{(0.263)}} & \mc{1}{c}{\scriptsize{(0.434)}} & \mc{1}{c}{\scriptsize{(0.303)}} & \mc{1}{c}{\scriptsize{(0.105)}} & \mc{1}{c}{\scriptsize{(0.118)}} \\  

    \mc{1}{l}{\scriptsize{Depressive Problems}} & \mc{1}{c}{\scriptsize{30}} & \mc{1}{c}{\scriptsize{1.015}} & \mc{1}{c}{\scriptsize{0.747}} & \mc{1}{c}{\scriptsize{1.210}} & \mc{1}{c}{\scriptsize{0.661}} & \mc{1}{c}{\scriptsize{1.125}} & \mc{1}{c}{\scriptsize{0.917}} & \mc{1}{c}{\scriptsize{0.679}} & \mc{1}{c}{\scriptsize{0.783}} \\  

     &  & \mc{1}{c}{\scriptsize{(0.934)}} & \mc{1}{c}{\scriptsize{(0.921)}} & \mc{1}{c}{\scriptsize{(0.855)}} & \mc{1}{c}{\scriptsize{(0.750)}} & \mc{1}{c}{\scriptsize{(0.895)}} & \mc{1}{c}{\scriptsize{(0.908)}} & \mc{1}{c}{\scriptsize{(0.895)}} & \mc{1}{c}{\scriptsize{(0.921)}} \\  

    \mc{1}{l}{\scriptsize{Avoidant Personality Problems}} & \mc{1}{c}{\scriptsize{30}} & \mc{1}{c}{\scriptsize{0.536}} & \mc{1}{c}{\scriptsize{0.286}} & \mc{1}{c}{\scriptsize{1.292}} & \mc{1}{c}{\scriptsize{0.505}} & \mc{1}{c}{\scriptsize{0.949}} & \mc{1}{c}{\scriptsize{0.192}} & \mc{1}{c}{\scriptsize{-0.118}} & \mc{1}{c}{\scriptsize{-0.021}} \\  

     &  & \mc{1}{c}{\scriptsize{(0.711)}} & \mc{1}{c}{\scriptsize{(0.618)}} & \mc{1}{c}{\scriptsize{(0.816)}} & \mc{1}{c}{\scriptsize{(0.645)}} & \mc{1}{c}{\scriptsize{(0.750)}} & \mc{1}{c}{\scriptsize{(0.605)}} & \mc{1}{c}{\scriptsize{(0.447)}} & \mc{1}{c}{\scriptsize{(0.461)}} \\  

    \mc{1}{l}{\scriptsize{Anxiety Problems}} & \mc{1}{c}{\scriptsize{30}} & \mc{1}{c}{\scriptsize{0.596}} & \mc{1}{c}{\scriptsize{0.059}} & \mc{1}{c}{\scriptsize{0.636}} & \mc{1}{c}{\scriptsize{-0.594}} & \mc{1}{c}{\scriptsize{0.520}} & \mc{1}{c}{\scriptsize{0.469}} & \mc{1}{c}{\scriptsize{0.096}} & \mc{1}{c}{\scriptsize{0.281}} \\  

     &  & \mc{1}{c}{\scriptsize{(0.750)}} & \mc{1}{c}{\scriptsize{(0.553)}} & \mc{1}{c}{\scriptsize{(0.671)}} & \mc{1}{c}{\scriptsize{(0.263)}} & \mc{1}{c}{\scriptsize{(0.671)}} & \mc{1}{c}{\scriptsize{(0.724)}} & \mc{1}{c}{\scriptsize{(0.539)}} & \mc{1}{c}{\scriptsize{(0.632)}} \\  

    \mc{1}{l}{\scriptsize{Antisocial Personality Problems}} & \mc{1}{c}{\scriptsize{30}} & \mc{1}{c}{\scriptsize{-0.564}} & \mc{1}{c}{\scriptsize{-0.554}} & \mc{1}{c}{\scriptsize{-0.851}} & \mc{1}{c}{\scriptsize{-1.819}} & \mc{1}{c}{\scriptsize{-0.846}} & \mc{1}{c}{\scriptsize{-0.465}} & \mc{1}{c}{\scriptsize{-0.561}} & \mc{1}{c}{\scriptsize{-0.396}} \\  

     &  & \mc{1}{c}{\scriptsize{(0.329)}} & \mc{1}{c}{\scriptsize{(0.276)}} & \mc{1}{c}{\scriptsize{(0.303)}} & \mc{1}{c}{\scriptsize{(0.132)}} & \mc{1}{c}{\scriptsize{(0.316)}} & \mc{1}{c}{\scriptsize{(0.368)}} & \mc{1}{c}{\scriptsize{(0.250)}} & \mc{1}{c}{\scriptsize{(0.395)}} \\  

  \bottomrule
  \end{tabular}
	\end{table} 

	\begin{table}[H]
     \caption{Treatment Effects on Age 30 Adult Self Report Syndrome Scale $t$-Score, Pooled Sample}
     \label{table:abccare_rslt_pooled_cat47}
	  \begin{tabular}{cccccccccc}
  \toprule

    \scriptsize{Variable} & \scriptsize{Age} & \scriptsize{(1)} & \scriptsize{(2)} & \scriptsize{(3)} & \scriptsize{(4)} & \scriptsize{(5)} & \scriptsize{(6)} & \scriptsize{(7)} & \scriptsize{(8)} \\ 
    \midrule  

    \mc{1}{l}{\scriptsize{Views Self as Not Liked}} & \mc{1}{c}{\scriptsize{12}} & \mc{1}{c}{\scriptsize{-0.092}} & \mc{1}{c}{\scriptsize{-0.117}} & \mc{1}{c}{\scriptsize{-0.142}} & \mc{1}{c}{\scriptsize{-0.147}} & \mc{1}{c}{\scriptsize{-0.158}} & \mc{1}{c}{\scriptsize{-0.083}} & \mc{1}{c}{\scriptsize{-0.110}} & \mc{1}{c}{\scriptsize{-0.112}} \\  

     &  & \mc{1}{c}{\scriptsize{(0.118)}} & \mc{1}{c}{\scriptsize{(0.132)}} & \mc{1}{c}{\scriptsize{(0.197)}} & \mc{1}{c}{\scriptsize{(0.145)}} & \mc{1}{c}{\scriptsize{(0.171)}} & \mc{1}{c}{\scriptsize{(0.158)}} & \mc{1}{c}{\scriptsize{(0.145)}} & \mc{1}{c}{\scriptsize{(0.105)}} \\  

    \mc{1}{l}{\scriptsize{Proud about Self}} & \mc{1}{c}{\scriptsize{12}} & \mc{1}{c}{\scriptsize{-0.045}} & \mc{1}{c}{\scriptsize{-0.060}} & \mc{1}{c}{\scriptsize{0.034}} & \mc{1}{c}{\scriptsize{0.068}} & \mc{1}{c}{\scriptsize{0.036}} & \mc{1}{c}{\scriptsize{-0.068}} & \mc{1}{c}{\scriptsize{-0.091}} & \mc{1}{c}{\scriptsize{-0.083}} \\  

     &  & \mc{1}{c}{\scriptsize{(0.697)}} & \mc{1}{c}{\scriptsize{(0.724)}} & \mc{1}{c}{\scriptsize{(0.421)}} & \mc{1}{c}{\scriptsize{(0.276)}} & \mc{1}{c}{\scriptsize{(0.382)}} & \mc{1}{c}{\scriptsize{(0.789)}} & \mc{1}{c}{\scriptsize{(0.816)}} & \mc{1}{c}{\scriptsize{(0.776)}} \\  

    \mc{1}{l}{\scriptsize{Ignores Situation}} & \mc{1}{c}{\scriptsize{12}} & \mc{1}{c}{\scriptsize{-0.164}} & \mc{1}{c}{\scriptsize{-0.233}} & \mc{1}{c}{\scriptsize{-0.135}} & \mc{1}{c}{\scriptsize{-0.255}} & \mc{1}{c}{\scriptsize{-0.160}} & \mc{1}{c}{\scriptsize{-0.167}} & \mc{1}{c}{\scriptsize{-0.209}} & \mc{1}{c}{\scriptsize{-0.191}} \\  

     &  & \mc{1}{c}{\scriptsize{\textbf{(0.053)}}} & \mc{1}{c}{\scriptsize{\textbf{(0.013)}}} & \mc{1}{c}{\scriptsize{(0.197)}} & \mc{1}{c}{\scriptsize{(0.105)}} & \mc{1}{c}{\scriptsize{(0.145)}} & \mc{1}{c}{\scriptsize{\textbf{(0.066)}}} & \mc{1}{c}{\scriptsize{\textbf{(0.066)}}} & \mc{1}{c}{\scriptsize{\textbf{(0.053)}}} \\  

    \mc{1}{l}{\scriptsize{Not Cope with Problem}} & \mc{1}{c}{\scriptsize{12}} & \mc{1}{c}{\scriptsize{-0.023}} & \mc{1}{c}{\scriptsize{-0.033}} & \mc{1}{c}{\scriptsize{-0.049}} & \mc{1}{c}{\scriptsize{-0.006}} & \mc{1}{c}{\scriptsize{-0.007}} & \mc{1}{c}{\scriptsize{-0.006}} & \mc{1}{c}{\scriptsize{-0.019}} & \mc{1}{c}{\scriptsize{0.007}} \\  

     &  & \mc{1}{c}{\scriptsize{(0.421)}} & \mc{1}{c}{\scriptsize{(0.316)}} & \mc{1}{c}{\scriptsize{(0.421)}} & \mc{1}{c}{\scriptsize{(0.513)}} & \mc{1}{c}{\scriptsize{(0.526)}} & \mc{1}{c}{\scriptsize{(0.487)}} & \mc{1}{c}{\scriptsize{(0.434)}} & \mc{1}{c}{\scriptsize{(0.526)}} \\  

    \mc{1}{l}{\scriptsize{Significant Fears}} & \mc{1}{c}{\scriptsize{12}} & \mc{1}{c}{\scriptsize{-0.108}} & \mc{1}{c}{\scriptsize{-0.090}} & \mc{1}{c}{\scriptsize{0.036}} & \mc{1}{c}{\scriptsize{0.112}} & \mc{1}{c}{\scriptsize{0.024}} & \mc{1}{c}{\scriptsize{-0.165}} & \mc{1}{c}{\scriptsize{-0.128}} & \mc{1}{c}{\scriptsize{-0.169}} \\  

     &  & \mc{1}{c}{\scriptsize{\textbf{(0.053)}}} & \mc{1}{c}{\scriptsize{(0.105)}} & \mc{1}{c}{\scriptsize{(0.618)}} & \mc{1}{c}{\scriptsize{(0.842)}} & \mc{1}{c}{\scriptsize{(0.539)}} & \mc{1}{c}{\scriptsize{\textbf{(0.013)}}} & \mc{1}{c}{\scriptsize{\textbf{(0.039)}}} & \mc{1}{c}{\scriptsize{\textbf{(0.013)}}} \\  

    \mc{1}{l}{\scriptsize{Denies Any Worries}} & \mc{1}{c}{\scriptsize{12}} & \mc{1}{c}{\scriptsize{-0.140}} & \mc{1}{c}{\scriptsize{-0.195}} & \mc{1}{c}{\scriptsize{-0.071}} & \mc{1}{c}{\scriptsize{-0.076}} & \mc{1}{c}{\scriptsize{-0.061}} & \mc{1}{c}{\scriptsize{-0.164}} & \mc{1}{c}{\scriptsize{-0.221}} & \mc{1}{c}{\scriptsize{-0.168}} \\  

     &  & \mc{1}{c}{\scriptsize{\textbf{(0.013)}}} & \mc{1}{c}{\scriptsize{\textbf{(0.013)}}} & \mc{1}{c}{\scriptsize{(0.197)}} & \mc{1}{c}{\scriptsize{(0.145)}} & \mc{1}{c}{\scriptsize{(0.211)}} & \mc{1}{c}{\scriptsize{\textbf{(0.000)}}} & \mc{1}{c}{\scriptsize{\textbf{(0.013)}}} & \mc{1}{c}{\scriptsize{\textbf{(0.000)}}} \\  

    \mc{1}{l}{\scriptsize{Participates in Activity}} & \mc{1}{c}{\scriptsize{12}} & \mc{1}{c}{\scriptsize{0.155}} & \mc{1}{c}{\scriptsize{0.116}} & \mc{1}{c}{\scriptsize{0.201}} & \mc{1}{c}{\scriptsize{0.198}} & \mc{1}{c}{\scriptsize{0.174}} & \mc{1}{c}{\scriptsize{0.149}} & \mc{1}{c}{\scriptsize{0.120}} & \mc{1}{c}{\scriptsize{0.089}} \\  

     &  & \mc{1}{c}{\scriptsize{\textbf{(0.039)}}} & \mc{1}{c}{\scriptsize{\textbf{(0.053)}}} & \mc{1}{c}{\scriptsize{\textbf{(0.079)}}} & \mc{1}{c}{\scriptsize{(0.118)}} & \mc{1}{c}{\scriptsize{\textbf{(0.079)}}} & \mc{1}{c}{\scriptsize{\textbf{(0.039)}}} & \mc{1}{c}{\scriptsize{\textbf{(0.092)}}} & \mc{1}{c}{\scriptsize{(0.105)}} \\  

    \mc{1}{l}{\scriptsize{Views Self as Dumb}} & \mc{1}{c}{\scriptsize{12}} & \mc{1}{c}{\scriptsize{0.043}} & \mc{1}{c}{\scriptsize{0.047}} & \mc{1}{c}{\scriptsize{-0.009}} & \mc{1}{c}{\scriptsize{-0.110}} & \mc{1}{c}{\scriptsize{-0.001}} & \mc{1}{c}{\scriptsize{0.070}} & \mc{1}{c}{\scriptsize{0.092}} & \mc{1}{c}{\scriptsize{0.079}} \\  

     &  & \mc{1}{c}{\scriptsize{(0.671)}} & \mc{1}{c}{\scriptsize{(0.697)}} & \mc{1}{c}{\scriptsize{(0.579)}} & \mc{1}{c}{\scriptsize{(0.316)}} & \mc{1}{c}{\scriptsize{(0.553)}} & \mc{1}{c}{\scriptsize{(0.750)}} & \mc{1}{c}{\scriptsize{(0.763)}} & \mc{1}{c}{\scriptsize{(0.724)}} \\  

    \mc{1}{l}{\scriptsize{Impulsivity}} & \mc{1}{c}{\scriptsize{12}} & \mc{1}{c}{\scriptsize{-0.017}} & \mc{1}{c}{\scriptsize{0.028}} & \mc{1}{c}{\scriptsize{-0.034}} & \mc{1}{c}{\scriptsize{-0.057}} & \mc{1}{c}{\scriptsize{-0.020}} & \mc{1}{c}{\scriptsize{-0.027}} & \mc{1}{c}{\scriptsize{0.046}} & \mc{1}{c}{\scriptsize{0.001}} \\  

     &  & \mc{1}{c}{\scriptsize{(0.408)}} & \mc{1}{c}{\scriptsize{(0.645)}} & \mc{1}{c}{\scriptsize{(0.382)}} & \mc{1}{c}{\scriptsize{(0.329)}} & \mc{1}{c}{\scriptsize{(0.474)}} & \mc{1}{c}{\scriptsize{(0.355)}} & \mc{1}{c}{\scriptsize{(0.671)}} & \mc{1}{c}{\scriptsize{(0.487)}} \\  

    \mc{1}{l}{\scriptsize{Withdraws Excessively}} & \mc{1}{c}{\scriptsize{12}} & \mc{1}{c}{\scriptsize{-0.001}} & \mc{1}{c}{\scriptsize{-0.026}} & \mc{1}{c}{\scriptsize{-0.009}} & \mc{1}{c}{\scriptsize{-0.085}} & \mc{1}{c}{\scriptsize{0.018}} & \mc{1}{c}{\scriptsize{0.013}} & \mc{1}{c}{\scriptsize{-0.021}} & \mc{1}{c}{\scriptsize{0.038}} \\  

     &  & \mc{1}{c}{\scriptsize{(0.553)}} & \mc{1}{c}{\scriptsize{(0.395)}} & \mc{1}{c}{\scriptsize{(0.474)}} & \mc{1}{c}{\scriptsize{(0.289)}} & \mc{1}{c}{\scriptsize{(0.553)}} & \mc{1}{c}{\scriptsize{(0.566)}} & \mc{1}{c}{\scriptsize{(0.421)}} & \mc{1}{c}{\scriptsize{(0.671)}} \\  

    \mc{1}{l}{\scriptsize{Views Self as Clumsy}} & \mc{1}{c}{\scriptsize{12}} & \mc{1}{c}{\scriptsize{-0.017}} & \mc{1}{c}{\scriptsize{-0.033}} & \mc{1}{c}{\scriptsize{0.105}} & \mc{1}{c}{\scriptsize{0.178}} & \mc{1}{c}{\scriptsize{0.097}} & \mc{1}{c}{\scriptsize{-0.054}} & \mc{1}{c}{\scriptsize{-0.081}} & \mc{1}{c}{\scriptsize{-0.077}} \\  

     &  & \mc{1}{c}{\scriptsize{(0.395)}} & \mc{1}{c}{\scriptsize{(0.355)}} & \mc{1}{c}{\scriptsize{(0.855)}} & \mc{1}{c}{\scriptsize{(0.934)}} & \mc{1}{c}{\scriptsize{(0.842)}} & \mc{1}{c}{\scriptsize{(0.276)}} & \mc{1}{c}{\scriptsize{(0.250)}} & \mc{1}{c}{\scriptsize{(0.237)}} \\  

    \mc{1}{l}{\scriptsize{Time spent reading}} & \mc{1}{c}{\scriptsize{12}} & \mc{1}{c}{\scriptsize{1.889}} & \mc{1}{c}{\scriptsize{1.768}} & \mc{1}{c}{\scriptsize{0.548}} & \mc{1}{c}{\scriptsize{0.571}} & \mc{1}{c}{\scriptsize{0.904}} & \mc{1}{c}{\scriptsize{2.203}} & \mc{1}{c}{\scriptsize{2.035}} & \mc{1}{c}{\scriptsize{2.492}} \\  

     &  & \mc{1}{c}{\scriptsize{\textbf{(0.013)}}} & \mc{1}{c}{\scriptsize{\textbf{(0.053)}}} & \mc{1}{c}{\scriptsize{(0.447)}} & \mc{1}{c}{\scriptsize{(0.447)}} & \mc{1}{c}{\scriptsize{(0.368)}} & \mc{1}{c}{\scriptsize{\textbf{(0.013)}}} & \mc{1}{c}{\scriptsize{\textbf{(0.039)}}} & \mc{1}{c}{\scriptsize{\textbf{(0.013)}}} \\  

    \mc{1}{l}{\scriptsize{Often Mad or Angry}} & \mc{1}{c}{\scriptsize{12}} & \mc{1}{c}{\scriptsize{-0.112}} & \mc{1}{c}{\scriptsize{-0.164}} & \mc{1}{c}{\scriptsize{0.122}} & \mc{1}{c}{\scriptsize{0.075}} & \mc{1}{c}{\scriptsize{0.130}} & \mc{1}{c}{\scriptsize{-0.175}} & \mc{1}{c}{\scriptsize{-0.218}} & \mc{1}{c}{\scriptsize{-0.162}} \\  

     &  & \mc{1}{c}{\scriptsize{(0.158)}} & \mc{1}{c}{\scriptsize{\textbf{(0.092)}}} & \mc{1}{c}{\scriptsize{(0.934)}} & \mc{1}{c}{\scriptsize{(0.829)}} & \mc{1}{c}{\scriptsize{(0.934)}} & \mc{1}{c}{\scriptsize{(0.118)}} & \mc{1}{c}{\scriptsize{\textbf{(0.066)}}} & \mc{1}{c}{\scriptsize{\textbf{(0.079)}}} \\  

    \mc{1}{l}{\scriptsize{Family Proud of You}} & \mc{1}{c}{\scriptsize{12}} & \mc{1}{c}{\scriptsize{0.001}} & \mc{1}{c}{\scriptsize{-0.041}} & \mc{1}{c}{\scriptsize{0.085}} & \mc{1}{c}{\scriptsize{0.016}} & \mc{1}{c}{\scriptsize{0.066}} & \mc{1}{c}{\scriptsize{-0.026}} & \mc{1}{c}{\scriptsize{-0.083}} & \mc{1}{c}{\scriptsize{-0.060}} \\  

     &  & \mc{1}{c}{\scriptsize{(0.487)}} & \mc{1}{c}{\scriptsize{(0.711)}} & \mc{1}{c}{\scriptsize{(0.171)}} & \mc{1}{c}{\scriptsize{(0.487)}} & \mc{1}{c}{\scriptsize{(0.237)}} & \mc{1}{c}{\scriptsize{(0.645)}} & \mc{1}{c}{\scriptsize{(0.842)}} & \mc{1}{c}{\scriptsize{(0.737)}} \\  

    \mc{1}{l}{\scriptsize{Feels Inadequate, Inferior}} & \mc{1}{c}{\scriptsize{12}} & \mc{1}{c}{\scriptsize{0.004}} & \mc{1}{c}{\scriptsize{-0.000}} & \mc{1}{c}{\scriptsize{0.050}} & \mc{1}{c}{\scriptsize{0.038}} & \mc{1}{c}{\scriptsize{0.072}} & \mc{1}{c}{\scriptsize{0.024}} & \mc{1}{c}{\scriptsize{-0.029}} & \mc{1}{c}{\scriptsize{0.052}} \\  

     &  & \mc{1}{c}{\scriptsize{(0.487)}} & \mc{1}{c}{\scriptsize{(0.447)}} & \mc{1}{c}{\scriptsize{(0.592)}} & \mc{1}{c}{\scriptsize{(0.592)}} & \mc{1}{c}{\scriptsize{(0.632)}} & \mc{1}{c}{\scriptsize{(0.553)}} & \mc{1}{c}{\scriptsize{(0.342)}} & \mc{1}{c}{\scriptsize{(0.671)}} \\  

    \mc{1}{l}{\scriptsize{Good Description of Self}} & \mc{1}{c}{\scriptsize{12}} & \mc{1}{c}{\scriptsize{0.023}} & \mc{1}{c}{\scriptsize{-0.060}} & \mc{1}{c}{\scriptsize{-0.087}} & \mc{1}{c}{\scriptsize{-0.283}} & \mc{1}{c}{\scriptsize{-0.147}} & \mc{1}{c}{\scriptsize{0.067}} & \mc{1}{c}{\scriptsize{0.008}} & \mc{1}{c}{\scriptsize{-0.006}} \\  

     &  & \mc{1}{c}{\scriptsize{(0.355)}} & \mc{1}{c}{\scriptsize{(0.724)}} & \mc{1}{c}{\scriptsize{(0.711)}} & \mc{1}{c}{\scriptsize{(1.000)}} & \mc{1}{c}{\scriptsize{(0.829)}} & \mc{1}{c}{\scriptsize{(0.237)}} & \mc{1}{c}{\scriptsize{(0.461)}} & \mc{1}{c}{\scriptsize{(0.553)}} \\  

  \bottomrule
  \end{tabular}
	\end{table} 

	\begin{table}[H]
     \caption{Treatment Effects on BSI 18 $t$-Score, Pooled Sample}
     \label{table:abccare_rslt_pooled_cat48}
	  \begin{tabular}{cccccccccc}
  \toprule

    \scriptsize{Variable} & \scriptsize{Age} & \scriptsize{(1)} & \scriptsize{(2)} & \scriptsize{(3)} & \scriptsize{(4)} & \scriptsize{(5)} & \scriptsize{(6)} & \scriptsize{(7)} & \scriptsize{(8)} \\ 
    \midrule  

    \mc{1}{l}{\scriptsize{Global Severity Index}} & \mc{1}{c}{\scriptsize{Mid-30s}} & \mc{1}{c}{\scriptsize{-2.516}} & \mc{1}{c}{\scriptsize{-2.683}} & \mc{1}{c}{\scriptsize{-0.151}} & \mc{1}{c}{\scriptsize{0.428}} & \mc{1}{c}{\scriptsize{-0.521}} & \mc{1}{c}{\scriptsize{-3.478}} & \mc{1}{c}{\scriptsize{-4.043}} & \mc{1}{c}{\scriptsize{-3.449}} \\  

     &  & \mc{1}{c}{\scriptsize{(0.158)}} & \mc{1}{c}{\scriptsize{(0.171)}} & \mc{1}{c}{\scriptsize{(0.447)}} & \mc{1}{c}{\scriptsize{(0.447)}} & \mc{1}{c}{\scriptsize{(0.408)}} & \mc{1}{c}{\scriptsize{\textbf{(0.053)}}} & \mc{1}{c}{\scriptsize{\textbf{(0.066)}}} & \mc{1}{c}{\scriptsize{\textbf{(0.066)}}} \\  

    \mc{1}{l}{\scriptsize{Somatization}} & \mc{1}{c}{\scriptsize{Mid-30s}} & \mc{1}{c}{\scriptsize{-1.211}} & \mc{1}{c}{\scriptsize{-0.947}} & \mc{1}{c}{\scriptsize{-1.529}} & \mc{1}{c}{\scriptsize{-2.007}} & \mc{1}{c}{\scriptsize{-1.536}} & \mc{1}{c}{\scriptsize{-1.296}} & \mc{1}{c}{\scriptsize{-0.810}} & \mc{1}{c}{\scriptsize{-1.094}} \\  

     &  & \mc{1}{c}{\scriptsize{(0.263)}} & \mc{1}{c}{\scriptsize{(0.355)}} & \mc{1}{c}{\scriptsize{(0.316)}} & \mc{1}{c}{\scriptsize{(0.316)}} & \mc{1}{c}{\scriptsize{(0.289)}} & \mc{1}{c}{\scriptsize{(0.276)}} & \mc{1}{c}{\scriptsize{(0.329)}} & \mc{1}{c}{\scriptsize{(0.263)}} \\  

    \mc{1}{l}{\scriptsize{Anxiety}} & \mc{1}{c}{\scriptsize{Mid-30s}} & \mc{1}{c}{\scriptsize{-3.343}} & \mc{1}{c}{\scriptsize{-3.827}} & \mc{1}{c}{\scriptsize{-2.478}} & \mc{1}{c}{\scriptsize{-2.320}} & \mc{1}{c}{\scriptsize{-2.936}} & \mc{1}{c}{\scriptsize{-3.825}} & \mc{1}{c}{\scriptsize{-4.487}} & \mc{1}{c}{\scriptsize{-4.164}} \\  

     &  & \mc{1}{c}{\scriptsize{\textbf{(0.026)}}} & \mc{1}{c}{\scriptsize{\textbf{(0.039)}}} & \mc{1}{c}{\scriptsize{(0.250)}} & \mc{1}{c}{\scriptsize{(0.289)}} & \mc{1}{c}{\scriptsize{(0.171)}} & \mc{1}{c}{\scriptsize{\textbf{(0.013)}}} & \mc{1}{c}{\scriptsize{\textbf{(0.039)}}} & \mc{1}{c}{\scriptsize{\textbf{(0.026)}}} \\  

    \mc{1}{l}{\scriptsize{Depression}} & \mc{1}{c}{\scriptsize{Mid-30s}} & \mc{1}{c}{\scriptsize{-1.787}} & \mc{1}{c}{\scriptsize{-2.357}} & \mc{1}{c}{\scriptsize{0.380}} & \mc{1}{c}{\scriptsize{1.336}} & \mc{1}{c}{\scriptsize{-0.166}} & \mc{1}{c}{\scriptsize{-2.589}} & \mc{1}{c}{\scriptsize{-3.890}} & \mc{1}{c}{\scriptsize{-2.728}} \\  

     &  & \mc{1}{c}{\scriptsize{(0.197)}} & \mc{1}{c}{\scriptsize{(0.145)}} & \mc{1}{c}{\scriptsize{(0.461)}} & \mc{1}{c}{\scriptsize{(0.539)}} & \mc{1}{c}{\scriptsize{(0.421)}} & \mc{1}{c}{\scriptsize{(0.145)}} & \mc{1}{c}{\scriptsize{\textbf{(0.039)}}} & \mc{1}{c}{\scriptsize{(0.105)}} \\  

  \bottomrule
  \end{tabular}
	\end{table} 

	\begin{table}[H]
     \caption{Treatment Effects on BSI Raw Score, Pooled Sample}
     \label{table:abccare_rslt_pooled_cat49}
	\input{AppResOutput/abccare/rslt_pooled_cat49}
	\end{table} 

	\begin{table}[H]
     \caption{Treatment Effects on BSI $t$-Score, Pooled Sample}
     \label{table:abccare_rslt_pooled_cat50}
	\input{AppResOutput/abccare/rslt_pooled_cat50}
	\end{table} 

	\begin{table}[H]
     \caption{Treatment Effects on Mid-30s Mental Health Conditions, Pooled Sample}
     \label{table:abccare_rslt_pooled_cat51}
	\input{AppResOutput/abccare/rslt_pooled_cat51}
	\end{table} 

	\begin{table}[H]
     \caption{Treatment Effects on Smoking and Drinking Behavior, Pooled Sample}
     \label{table:abccare_rslt_pooled_cat52}
	  \begin{tabular}{cccccccccc}
  \toprule

    \scriptsize{Variable} & \scriptsize{Age} & \scriptsize{(1)} & \scriptsize{(2)} & \scriptsize{(3)} & \scriptsize{(4)} & \scriptsize{(5)} & \scriptsize{(6)} & \scriptsize{(7)} & \scriptsize{(8)} \\ 
    \midrule  

    \mc{1}{l}{\scriptsize{Problems Due to Alcohol or Drugs}} & \mc{1}{c}{\scriptsize{12}} & \mc{1}{c}{\scriptsize{-0.039}} & \mc{1}{c}{\scriptsize{-0.041}} & \mc{1}{c}{\scriptsize{-0.139}} & \mc{1}{c}{\scriptsize{-0.146}} & \mc{1}{c}{\scriptsize{-0.152}} & \mc{1}{c}{\scriptsize{-0.007}} & \mc{1}{c}{\scriptsize{-0.015}} & \mc{1}{c}{\scriptsize{-0.009}} \\  

     &  & \mc{1}{c}{\scriptsize{(0.316)}} & \mc{1}{c}{\scriptsize{(0.316)}} & \mc{1}{c}{\scriptsize{\textbf{(0.079)}}} & \mc{1}{c}{\scriptsize{(0.158)}} & \mc{1}{c}{\scriptsize{(0.105)}} & \mc{1}{c}{\scriptsize{(0.382)}} & \mc{1}{c}{\scriptsize{(0.421)}} & \mc{1}{c}{\scriptsize{(0.408)}} \\  

    \mc{1}{l}{\scriptsize{Used Alcohol and/or Drugs}} & \mc{1}{c}{\scriptsize{12}} & \mc{1}{c}{\scriptsize{0.036}} & \mc{1}{c}{\scriptsize{0.040}} & \mc{1}{c}{\scriptsize{0.027}} & \mc{1}{c}{\scriptsize{0.027}} & \mc{1}{c}{\scriptsize{0.042}} & \mc{1}{c}{\scriptsize{0.037}} & \mc{1}{c}{\scriptsize{0.036}} & \mc{1}{c}{\scriptsize{0.049}} \\  

     &  & \mc{1}{c}{\scriptsize{(0.711)}} & \mc{1}{c}{\scriptsize{(0.763)}} & \mc{1}{c}{\scriptsize{(0.645)}} & \mc{1}{c}{\scriptsize{(0.632)}} & \mc{1}{c}{\scriptsize{(0.724)}} & \mc{1}{c}{\scriptsize{(0.750)}} & \mc{1}{c}{\scriptsize{(0.750)}} & \mc{1}{c}{\scriptsize{(0.776)}} \\  

  \bottomrule
  \end{tabular}
	\end{table} 

	\begin{table}[H]
     \caption{Treatment Effects on Tobacco, Drugs, Alcohol, Pooled Sample}
     \label{table:abccare_rslt_pooled_cat53}
	\input{AppResOutput/abccare/rslt_pooled_cat53}
	\end{table} 
\section{Treatment Effects for Male Sample}


	\begin{table}[H]
     \caption{Treatment Effects on IQ Scores, Male Sample}
     \label{table:abccare_rslt_male_cat0}
	  \begin{tabular}{cccccccccc}
  \toprule

    \scriptsize{Variable} & \scriptsize{Age} & \scriptsize{(1)} & \scriptsize{(2)} & \scriptsize{(3)} & \scriptsize{(4)} & \scriptsize{(5)} & \scriptsize{(6)} & \scriptsize{(7)} & \scriptsize{(8)} \\ 
    \midrule  

    \mc{1}{l}{\scriptsize{Std. IQ Test}} & \mc{1}{c}{\scriptsize{2}} & \mc{1}{c}{\scriptsize{9.528}} & \mc{1}{c}{\scriptsize{11.036}} & \mc{1}{c}{\scriptsize{6.875}} & \mc{1}{c}{\scriptsize{10.704}} & \mc{1}{c}{\scriptsize{7.944}} & \mc{1}{c}{\scriptsize{10.286}} & \mc{1}{c}{\scriptsize{11.497}} & \mc{1}{c}{\scriptsize{11.080}} \\  

     &  & \mc{1}{c}{\scriptsize{\textbf{(0.000)}}} & \mc{1}{c}{\scriptsize{\textbf{(0.000)}}} & \mc{1}{c}{\scriptsize{\textbf{(0.026)}}} & \mc{1}{c}{\scriptsize{\textbf{(0.026)}}} & \mc{1}{c}{\scriptsize{\textbf{(0.013)}}} & \mc{1}{c}{\scriptsize{\textbf{(0.000)}}} & \mc{1}{c}{\scriptsize{\textbf{(0.000)}}} & \mc{1}{c}{\scriptsize{\textbf{(0.000)}}} \\  

     & \mc{1}{c}{\scriptsize{3}} & \mc{1}{c}{\scriptsize{13.410}} & \mc{1}{c}{\scriptsize{14.873}} & \mc{1}{c}{\scriptsize{13.896}} & \mc{1}{c}{\scriptsize{17.254}} & \mc{1}{c}{\scriptsize{15.474}} & \mc{1}{c}{\scriptsize{13.271}} & \mc{1}{c}{\scriptsize{14.229}} & \mc{1}{c}{\scriptsize{14.302}} \\  

     &  & \mc{1}{c}{\scriptsize{\textbf{(0.000)}}} & \mc{1}{c}{\scriptsize{\textbf{(0.000)}}} & \mc{1}{c}{\scriptsize{\textbf{(0.000)}}} & \mc{1}{c}{\scriptsize{\textbf{(0.000)}}} & \mc{1}{c}{\scriptsize{\textbf{(0.000)}}} & \mc{1}{c}{\scriptsize{\textbf{(0.000)}}} & \mc{1}{c}{\scriptsize{\textbf{(0.000)}}} & \mc{1}{c}{\scriptsize{\textbf{(0.000)}}} \\  

     & \mc{1}{c}{\scriptsize{3.5}} & \mc{1}{c}{\scriptsize{8.756}} & \mc{1}{c}{\scriptsize{7.994}} & \mc{1}{c}{\scriptsize{6.354}} & \mc{1}{c}{\scriptsize{5.904}} & \mc{1}{c}{\scriptsize{6.784}} & \mc{1}{c}{\scriptsize{9.443}} & \mc{1}{c}{\scriptsize{8.690}} & \mc{1}{c}{\scriptsize{9.043}} \\  

     &  & \mc{1}{c}{\scriptsize{\textbf{(0.000)}}} & \mc{1}{c}{\scriptsize{\textbf{(0.000)}}} & \mc{1}{c}{\scriptsize{\textbf{(0.066)}}} & \mc{1}{c}{\scriptsize{(0.105)}} & \mc{1}{c}{\scriptsize{\textbf{(0.053)}}} & \mc{1}{c}{\scriptsize{\textbf{(0.000)}}} & \mc{1}{c}{\scriptsize{\textbf{(0.000)}}} & \mc{1}{c}{\scriptsize{\textbf{(0.000)}}} \\  

     & \mc{1}{c}{\scriptsize{4}} & \mc{1}{c}{\scriptsize{12.089}} & \mc{1}{c}{\scriptsize{11.867}} & \mc{1}{c}{\scriptsize{8.950}} & \mc{1}{c}{\scriptsize{7.899}} & \mc{1}{c}{\scriptsize{9.668}} & \mc{1}{c}{\scriptsize{12.986}} & \mc{1}{c}{\scriptsize{12.604}} & \mc{1}{c}{\scriptsize{13.487}} \\  

     &  & \mc{1}{c}{\scriptsize{\textbf{(0.000)}}} & \mc{1}{c}{\scriptsize{\textbf{(0.000)}}} & \mc{1}{c}{\scriptsize{\textbf{(0.000)}}} & \mc{1}{c}{\scriptsize{(0.118)}} & \mc{1}{c}{\scriptsize{\textbf{(0.000)}}} & \mc{1}{c}{\scriptsize{\textbf{(0.000)}}} & \mc{1}{c}{\scriptsize{\textbf{(0.000)}}} & \mc{1}{c}{\scriptsize{\textbf{(0.000)}}} \\  

     & \mc{1}{c}{\scriptsize{4.5}} & \mc{1}{c}{\scriptsize{8.508}} & \mc{1}{c}{\scriptsize{8.914}} & \mc{1}{c}{\scriptsize{10.411}} & \mc{1}{c}{\scriptsize{11.349}} & \mc{1}{c}{\scriptsize{10.639}} & \mc{1}{c}{\scriptsize{7.964}} & \mc{1}{c}{\scriptsize{8.518}} & \mc{1}{c}{\scriptsize{7.804}} \\  

     &  & \mc{1}{c}{\scriptsize{\textbf{(0.000)}}} & \mc{1}{c}{\scriptsize{\textbf{(0.000)}}} & \mc{1}{c}{\scriptsize{\textbf{(0.000)}}} & \mc{1}{c}{\scriptsize{\textbf{(0.013)}}} & \mc{1}{c}{\scriptsize{\textbf{(0.000)}}} & \mc{1}{c}{\scriptsize{\textbf{(0.000)}}} & \mc{1}{c}{\scriptsize{\textbf{(0.000)}}} & \mc{1}{c}{\scriptsize{\textbf{(0.000)}}} \\  

     & \mc{1}{c}{\scriptsize{5}} & \mc{1}{c}{\scriptsize{7.697}} & \mc{1}{c}{\scriptsize{7.130}} & \mc{1}{c}{\scriptsize{4.643}} & \mc{1}{c}{\scriptsize{3.469}} & \mc{1}{c}{\scriptsize{4.992}} & \mc{1}{c}{\scriptsize{8.679}} & \mc{1}{c}{\scriptsize{8.241}} & \mc{1}{c}{\scriptsize{8.181}} \\  

     &  & \mc{1}{c}{\scriptsize{\textbf{(0.000)}}} & \mc{1}{c}{\scriptsize{\textbf{(0.000)}}} & \mc{1}{c}{\scriptsize{(0.158)}} & \mc{1}{c}{\scriptsize{(0.316)}} & \mc{1}{c}{\scriptsize{(0.158)}} & \mc{1}{c}{\scriptsize{\textbf{(0.000)}}} & \mc{1}{c}{\scriptsize{\textbf{(0.000)}}} & \mc{1}{c}{\scriptsize{\textbf{(0.000)}}} \\  

     & \mc{1}{c}{\scriptsize{6.6}} & \mc{1}{c}{\scriptsize{5.803}} & \mc{1}{c}{\scriptsize{5.157}} & \mc{1}{c}{\scriptsize{0.831}} & \mc{1}{c}{\scriptsize{-1.794}} & \mc{1}{c}{\scriptsize{1.479}} & \mc{1}{c}{\scriptsize{5.916}} & \mc{1}{c}{\scriptsize{5.542}} & \mc{1}{c}{\scriptsize{5.886}} \\  

     &  & \mc{1}{c}{\scriptsize{\textbf{(0.026)}}} & \mc{1}{c}{\scriptsize{\textbf{(0.000)}}} & \mc{1}{c}{\scriptsize{(0.500)}} & \mc{1}{c}{\scriptsize{(0.579)}} & \mc{1}{c}{\scriptsize{(0.474)}} & \mc{1}{c}{\scriptsize{\textbf{(0.013)}}} & \mc{1}{c}{\scriptsize{\textbf{(0.000)}}} & \mc{1}{c}{\scriptsize{\textbf{(0.013)}}} \\  

     & \mc{1}{c}{\scriptsize{7}} & \mc{1}{c}{\scriptsize{4.390}} & \mc{1}{c}{\scriptsize{5.812}} & \mc{1}{c}{\scriptsize{5.323}} & \mc{1}{c}{\scriptsize{8.155}} & \mc{1}{c}{\scriptsize{5.750}} & \mc{1}{c}{\scriptsize{4.156}} & \mc{1}{c}{\scriptsize{5.481}} & \mc{1}{c}{\scriptsize{4.418}} \\  

     &  & \mc{1}{c}{\scriptsize{\textbf{(0.053)}}} & \mc{1}{c}{\scriptsize{\textbf{(0.026)}}} & \mc{1}{c}{\scriptsize{(0.237)}} & \mc{1}{c}{\scriptsize{(0.132)}} & \mc{1}{c}{\scriptsize{(0.145)}} & \mc{1}{c}{\scriptsize{\textbf{(0.092)}}} & \mc{1}{c}{\scriptsize{\textbf{(0.039)}}} & \mc{1}{c}{\scriptsize{(0.105)}} \\  

     & \mc{1}{c}{\scriptsize{8}} & \mc{1}{c}{\scriptsize{4.160}} & \mc{1}{c}{\scriptsize{2.644}} & \mc{1}{c}{\scriptsize{-2.514}} & \mc{1}{c}{\scriptsize{-4.445}} & \mc{1}{c}{\scriptsize{-2.651}} & \mc{1}{c}{\scriptsize{4.754}} & \mc{1}{c}{\scriptsize{3.391}} & \mc{1}{c}{\scriptsize{4.207}} \\  

     &  & \mc{1}{c}{\scriptsize{\textbf{(0.079)}}} & \mc{1}{c}{\scriptsize{(0.211)}} & \mc{1}{c}{\scriptsize{(0.645)}} & \mc{1}{c}{\scriptsize{(0.684)}} & \mc{1}{c}{\scriptsize{(0.632)}} & \mc{1}{c}{\scriptsize{\textbf{(0.053)}}} & \mc{1}{c}{\scriptsize{\textbf{(0.092)}}} & \mc{1}{c}{\scriptsize{\textbf{(0.066)}}} \\  

     & \mc{1}{c}{\scriptsize{12}} & \mc{1}{c}{\scriptsize{0.686}} & \mc{1}{c}{\scriptsize{-0.856}} & \mc{1}{c}{\scriptsize{-0.343}} & \mc{1}{c}{\scriptsize{-4.076}} & \mc{1}{c}{\scriptsize{-1.018}} & \mc{1}{c}{\scriptsize{0.943}} & \mc{1}{c}{\scriptsize{-0.116}} & \mc{1}{c}{\scriptsize{-0.794}} \\  

     &  & \mc{1}{c}{\scriptsize{(0.382)}} & \mc{1}{c}{\scriptsize{(0.592)}} & \mc{1}{c}{\scriptsize{(0.566)}} & \mc{1}{c}{\scriptsize{(0.658)}} & \mc{1}{c}{\scriptsize{(0.605)}} & \mc{1}{c}{\scriptsize{(0.355)}} & \mc{1}{c}{\scriptsize{(0.461)}} & \mc{1}{c}{\scriptsize{(0.592)}} \\  

    \mc{1}{l}{\scriptsize{IQ Factor}} & \mc{1}{c}{\scriptsize{2 to 5}} & \mc{1}{c}{\scriptsize{0.831}} & \mc{1}{c}{\scriptsize{0.845}} & \mc{1}{c}{\scriptsize{0.708}} & \mc{1}{c}{\scriptsize{0.766}} & \mc{1}{c}{\scriptsize{0.759}} & \mc{1}{c}{\scriptsize{0.867}} & \mc{1}{c}{\scriptsize{0.870}} & \mc{1}{c}{\scriptsize{0.874}} \\  

     &  & \mc{1}{c}{\scriptsize{\textbf{(0.000)}}} & \mc{1}{c}{\scriptsize{\textbf{(0.000)}}} & \mc{1}{c}{\scriptsize{\textbf{(0.000)}}} & \mc{1}{c}{\scriptsize{\textbf{(0.039)}}} & \mc{1}{c}{\scriptsize{\textbf{(0.000)}}} & \mc{1}{c}{\scriptsize{\textbf{(0.000)}}} & \mc{1}{c}{\scriptsize{\textbf{(0.000)}}} & \mc{1}{c}{\scriptsize{\textbf{(0.000)}}} \\  

     & \mc{1}{c}{\scriptsize{6 to 12}} & \mc{1}{c}{\scriptsize{0.268}} & \mc{1}{c}{\scriptsize{0.227}} & \mc{1}{c}{\scriptsize{0.308}} & \mc{1}{c}{\scriptsize{0.251}} & \mc{1}{c}{\scriptsize{0.304}} & \mc{1}{c}{\scriptsize{0.258}} & \mc{1}{c}{\scriptsize{0.245}} & \mc{1}{c}{\scriptsize{0.226}} \\  

     &  & \mc{1}{c}{\scriptsize{(0.105)}} & \mc{1}{c}{\scriptsize{(0.184)}} & \mc{1}{c}{\scriptsize{(0.237)}} & \mc{1}{c}{\scriptsize{(0.303)}} & \mc{1}{c}{\scriptsize{(0.263)}} & \mc{1}{c}{\scriptsize{(0.105)}} & \mc{1}{c}{\scriptsize{(0.105)}} & \mc{1}{c}{\scriptsize{(0.184)}} \\ 
    \midrule  

    \mc{2}{l}{\scriptsize{\% of Pos. TE ($H_0$: $\le$ 50\%)}} & \mc{1}{c}{\scriptsize{100}} & \mc{1}{c}{\scriptsize{92}} & \mc{1}{c}{\scriptsize{83}} & \mc{1}{c}{\scriptsize{75}} & \mc{1}{c}{\scriptsize{83}} & \mc{1}{c}{\scriptsize{100}} & \mc{1}{c}{\scriptsize{92}} & \mc{1}{c}{\scriptsize{92}} \\  

     &  & \mc{1}{c}{\scriptsize{\textbf{(0.000)}}} & \mc{1}{c}{\scriptsize{\textbf{(0.000)}}} & \mc{1}{c}{\scriptsize{\textbf{(0.000)}}} & \mc{1}{c}{\scriptsize{\textbf{(0.000)}}} & \mc{1}{c}{\scriptsize{\textbf{(0.000)}}} & \mc{1}{c}{\scriptsize{\textbf{(0.000)}}} & \mc{1}{c}{\scriptsize{\textbf{(0.000)}}} & \mc{1}{c}{\scriptsize{\textbf{(0.000)}}} \\  

    \mc{2}{l}{\scriptsize{\% of Pos. TE ($H_0$: $\le$ 10\% $|$ 10\% Significance)}} & \mc{1}{c}{\scriptsize{75}} & \mc{1}{c}{\scriptsize{75}} & \mc{1}{c}{\scriptsize{50}} & \mc{1}{c}{\scriptsize{33}} & \mc{1}{c}{\scriptsize{50}} & \mc{1}{c}{\scriptsize{83}} & \mc{1}{c}{\scriptsize{75}} & \mc{1}{c}{\scriptsize{83}} \\  

     &  & \mc{1}{c}{\scriptsize{\textbf{(0.000)}}} & \mc{1}{c}{\scriptsize{\textbf{(0.000)}}} & \mc{1}{c}{\scriptsize{\textbf{(0.053)}}} & \mc{1}{c}{\scriptsize{(0.184)}} & \mc{1}{c}{\scriptsize{\textbf{(0.026)}}} & \mc{1}{c}{\scriptsize{\textbf{(0.000)}}} & \mc{1}{c}{\scriptsize{\textbf{(0.000)}}} & \mc{1}{c}{\scriptsize{\textbf{(0.000)}}} \\  

  \bottomrule
  \end{tabular}
	\end{table} 

	\begin{table}[H]
     \caption{Treatment Effects on Achievement Scores, Male Sample}
     \label{table:abccare_rslt_male_cat1}
	  \begin{tabular}{cccccccccc}
  \toprule

    \scriptsize{Variable} & \scriptsize{Age} & \scriptsize{(1)} & \scriptsize{(2)} & \scriptsize{(3)} & \scriptsize{(4)} & \scriptsize{(5)} & \scriptsize{(6)} & \scriptsize{(7)} & \scriptsize{(8)} \\ 
    \midrule  

    \mc{1}{l}{\scriptsize{Heart Attack}} & \mc{1}{c}{\scriptsize{Mid-30s}} &  &  &  &  &  &  &  &  \\  

     &  &  &  &  &  &  &  &  &  \\  

    \mc{1}{l}{\scriptsize{Sickle Cell Anemia}} & \mc{1}{c}{\scriptsize{Mid-30s}} &  &  &  &  &  &  &  &  \\  

     &  &  &  &  &  &  &  &  &  \\  

    \mc{1}{l}{\scriptsize{Asthma}} & \mc{1}{c}{\scriptsize{Mid-30s}} & \mc{1}{c}{\scriptsize{-0.034}} & \mc{1}{c}{\scriptsize{-0.027}} & \mc{1}{c}{\scriptsize{0.037}} & \mc{1}{c}{\scriptsize{0.028}} & \mc{1}{c}{\scriptsize{0.048}} & \mc{1}{c}{\scriptsize{-0.063}} & \mc{1}{c}{\scriptsize{-0.042}} & \mc{1}{c}{\scriptsize{-0.034}} \\  

     &  & \mc{1}{c}{\scriptsize{(0.237)}} & \mc{1}{c}{\scriptsize{(0.250)}} & \mc{1}{c}{\scriptsize{(0.382)}} & \mc{1}{c}{\scriptsize{(0.289)}} & \mc{1}{c}{\scriptsize{(0.382)}} & \mc{1}{c}{\scriptsize{(0.211)}} & \mc{1}{c}{\scriptsize{(0.211)}} & \mc{1}{c}{\scriptsize{(0.263)}} \\  

    \mc{1}{l}{\scriptsize{Stroke}} & \mc{1}{c}{\scriptsize{Mid-30s}} &  &  &  &  &  &  &  &  \\  

     &  &  &  &  &  &  &  &  &  \\  

    \mc{1}{l}{\scriptsize{High Blood Pressure (Hypertension)}} & \mc{1}{c}{\scriptsize{Mid-30s}} & \mc{1}{c}{\scriptsize{0.040}} & \mc{1}{c}{\scriptsize{0.039}} & \mc{1}{c}{\scriptsize{0.040}} & \mc{1}{c}{\scriptsize{0.019}} & \mc{1}{c}{\scriptsize{0.050}} & \mc{1}{c}{\scriptsize{0.040}} & \mc{1}{c}{\scriptsize{0.045}} & \mc{1}{c}{\scriptsize{0.050}} \\  

     &  & \mc{1}{c}{\scriptsize{(0.566)}} & \mc{1}{c}{\scriptsize{(0.526)}} & \mc{1}{c}{\scriptsize{(0.513)}} & \mc{1}{c}{\scriptsize{(0.382)}} & \mc{1}{c}{\scriptsize{(0.487)}} & \mc{1}{c}{\scriptsize{(0.553)}} & \mc{1}{c}{\scriptsize{(0.487)}} & \mc{1}{c}{\scriptsize{(0.539)}} \\  

    \mc{1}{l}{\scriptsize{Arthritis or Generative Disease}} & \mc{1}{c}{\scriptsize{Mid-30s}} &  &  &  &  &  &  &  &  \\  

     &  &  &  &  &  &  &  &  &  \\  

    \mc{1}{l}{\scriptsize{Diabetes}} & \mc{1}{c}{\scriptsize{Mid-30s}} & \mc{1}{c}{\scriptsize{0.040}} & \mc{1}{c}{\scriptsize{0.033}} & \mc{1}{c}{\scriptsize{0.040}} & \mc{1}{c}{\scriptsize{-0.046}} & \mc{1}{c}{\scriptsize{0.050}} & \mc{1}{c}{\scriptsize{0.040}} & \mc{1}{c}{\scriptsize{0.051}} & \mc{1}{c}{\scriptsize{0.050}} \\  

     &  & \mc{1}{c}{\scriptsize{(0.566)}} & \mc{1}{c}{\scriptsize{(0.513)}} & \mc{1}{c}{\scriptsize{(0.487)}} & \mc{1}{c}{\scriptsize{(0.171)}} & \mc{1}{c}{\scriptsize{(0.487)}} & \mc{1}{c}{\scriptsize{(0.553)}} & \mc{1}{c}{\scriptsize{(0.539)}} & \mc{1}{c}{\scriptsize{(0.553)}} \\  

    \mc{1}{l}{\scriptsize{Cancer}} & \mc{1}{c}{\scriptsize{Mid-30s}} &  &  &  &  &  &  &  &  \\  

     &  &  &  &  &  &  &  &  &  \\  

    \mc{1}{l}{\scriptsize{Heart Attack or Coronary Disease}} & \mc{1}{c}{\scriptsize{Mid-30s}} &  &  &  &  &  &  &  &  \\  

     &  &  &  &  &  &  &  &  &  \\  

    \mc{1}{l}{\scriptsize{High Cholesterol}} & \mc{1}{c}{\scriptsize{Mid-30s}} &  &  &  &  &  &  &  &  \\  

     &  &  &  &  &  &  &  &  &  \\  

    \mc{1}{l}{\scriptsize{Dementia}} & \mc{1}{c}{\scriptsize{Mid-30s}} &  &  &  &  &  &  &  &  \\  

     &  &  &  &  &  &  &  &  &  \\  

  \bottomrule
  \end{tabular}
	\end{table} 

	\begin{table}[H]
     \caption{Treatment Effects on Infant Behavior Record, Male Sample}
     \label{table:abccare_rslt_male_cat2}
	  \begin{tabular}{cccccccccc}
  \toprule

    \scriptsize{Variable} & \scriptsize{Age} & \scriptsize{(1)} & \scriptsize{(2)} & \scriptsize{(3)} & \scriptsize{(4)} & \scriptsize{(5)} & \scriptsize{(6)} & \scriptsize{(7)} & \scriptsize{(8)} \\ 
    \midrule  

    \mc{1}{l}{\scriptsize{Prediabetes}} & \mc{1}{c}{\scriptsize{Mid-30s}} & \mc{1}{c}{\scriptsize{-0.129}} & \mc{1}{c}{\scriptsize{-0.190}} & \mc{1}{c}{\scriptsize{-0.267}} & \mc{1}{c}{\scriptsize{-0.478}} & \mc{1}{c}{\scriptsize{-0.338}} & \mc{1}{c}{\scriptsize{-0.139}} & \mc{1}{c}{\scriptsize{-0.182}} & \mc{1}{c}{\scriptsize{-0.173}} \\  

     &  & \mc{1}{c}{\scriptsize{(0.237)}} & \mc{1}{c}{\scriptsize{(0.184)}} & \mc{1}{c}{\scriptsize{(0.184)}} & \mc{1}{c}{\scriptsize{(0.105)}} & \mc{1}{c}{\scriptsize{(0.132)}} & \mc{1}{c}{\scriptsize{(0.237)}} & \mc{1}{c}{\scriptsize{(0.224)}} & \mc{1}{c}{\scriptsize{(0.224)}} \\  

    \mc{1}{l}{\scriptsize{Hemoglobin Level (\%)}} & \mc{1}{c}{\scriptsize{Mid-30s}} & \mc{1}{c}{\scriptsize{0.322}} & \mc{1}{c}{\scriptsize{0.435}} & \mc{1}{c}{\scriptsize{0.240}} & \mc{1}{c}{\scriptsize{0.469}} & \mc{1}{c}{\scriptsize{0.355}} & \mc{1}{c}{\scriptsize{0.286}} & \mc{1}{c}{\scriptsize{0.386}} & \mc{1}{c}{\scriptsize{0.396}} \\  

     &  & \mc{1}{c}{\scriptsize{(0.816)}} & \mc{1}{c}{\scriptsize{(0.724)}} & \mc{1}{c}{\scriptsize{(0.632)}} & \mc{1}{c}{\scriptsize{(0.592)}} & \mc{1}{c}{\scriptsize{(0.697)}} & \mc{1}{c}{\scriptsize{(0.776)}} & \mc{1}{c}{\scriptsize{(0.737)}} & \mc{1}{c}{\scriptsize{(0.803)}} \\  

    \mc{1}{l}{\scriptsize{Diabetes}} & \mc{1}{c}{\scriptsize{Mid-30s}} & \mc{1}{c}{\scriptsize{0.080}} & \mc{1}{c}{\scriptsize{0.081}} & \mc{1}{c}{\scriptsize{0.080}} & \mc{1}{c}{\scriptsize{0.027}} & \mc{1}{c}{\scriptsize{0.098}} & \mc{1}{c}{\scriptsize{0.080}} & \mc{1}{c}{\scriptsize{0.083}} & \mc{1}{c}{\scriptsize{0.098}} \\  

     &  & \mc{1}{c}{\scriptsize{(0.829)}} & \mc{1}{c}{\scriptsize{(0.789)}} & \mc{1}{c}{\scriptsize{(0.711)}} & \mc{1}{c}{\scriptsize{(0.447)}} & \mc{1}{c}{\scriptsize{(0.750)}} & \mc{1}{c}{\scriptsize{(0.816)}} & \mc{1}{c}{\scriptsize{(0.776)}} & \mc{1}{c}{\scriptsize{(0.855)}} \\  

  \bottomrule
  \end{tabular}
	\end{table} 

	\begin{table}[H]
     \caption{Treatment Effects on Kohn and Rosman: Attentive/Cooperative, Male Sample}
     \label{table:abccare_rslt_male_cat3}
	  \begin{tabular}{cccccccccc}
  \toprule

    \scriptsize{Variable} & \scriptsize{Age} & \scriptsize{(1)} & \scriptsize{(2)} & \scriptsize{(3)} & \scriptsize{(4)} & \scriptsize{(5)} & \scriptsize{(6)} & \scriptsize{(7)} & \scriptsize{(8)} \\ 
    \midrule  

    \mc{1}{l}{\scriptsize{Attentive/Cooperative}} & \mc{1}{c}{\scriptsize{2}} & \mc{1}{c}{\scriptsize{0.908}} & \mc{1}{c}{\scriptsize{1.301}} & \mc{1}{c}{\scriptsize{0.117}} & \mc{1}{c}{\scriptsize{0.506}} & \mc{1}{c}{\scriptsize{0.356}} & \mc{1}{c}{\scriptsize{1.091}} & \mc{1}{c}{\scriptsize{1.429}} & \mc{1}{c}{\scriptsize{1.356}} \\  

     &  & \mc{1}{c}{\scriptsize{\textbf{(0.000)}}} & \mc{1}{c}{\scriptsize{\textbf{(0.000)}}} & \mc{1}{c}{\scriptsize{(0.342)}} & \mc{1}{c}{\scriptsize{\textbf{(0.066)}}} & \mc{1}{c}{\scriptsize{(0.145)}} & \mc{1}{c}{\scriptsize{\textbf{(0.000)}}} & \mc{1}{c}{\scriptsize{\textbf{(0.000)}}} & \mc{1}{c}{\scriptsize{\textbf{(0.000)}}} \\  

     & \mc{1}{c}{\scriptsize{12}} & \mc{1}{c}{\scriptsize{0.156}} & \mc{1}{c}{\scriptsize{0.148}} & \mc{1}{c}{\scriptsize{0.266}} & \mc{1}{c}{\scriptsize{0.386}} & \mc{1}{c}{\scriptsize{0.245}} & \mc{1}{c}{\scriptsize{0.129}} & \mc{1}{c}{\scriptsize{0.108}} & \mc{1}{c}{\scriptsize{0.106}} \\  

     &  & \mc{1}{c}{\scriptsize{(0.250)}} & \mc{1}{c}{\scriptsize{(0.263)}} & \mc{1}{c}{\scriptsize{(0.329)}} & \mc{1}{c}{\scriptsize{(0.263)}} & \mc{1}{c}{\scriptsize{(0.329)}} & \mc{1}{c}{\scriptsize{(0.276)}} & \mc{1}{c}{\scriptsize{(0.355)}} & \mc{1}{c}{\scriptsize{(0.303)}} \\  

  \bottomrule
  \end{tabular}
	\end{table} 

	\begin{table}[H]
     \caption{Treatment Effects on Classroom Behavior Inventory (Part I), Male Sample}
     \label{table:abccare_rslt_male_cat4}
	  \begin{tabular}{cccccccccc}
  \toprule

    \scriptsize{Variable} & \scriptsize{Age} & \scriptsize{(1)} & \scriptsize{(2)} & \scriptsize{(3)} & \scriptsize{(4)} & \scriptsize{(5)} & \scriptsize{(6)} & \scriptsize{(7)} & \scriptsize{(8)} \\ 
    \midrule  

    \mc{1}{l}{\scriptsize{Mother Works}} & \mc{1}{c}{\scriptsize{2}} & \mc{1}{c}{\scriptsize{0.056}} & \mc{1}{c}{\scriptsize{0.038}} & \mc{1}{c}{\scriptsize{0.264}} & \mc{1}{c}{\scriptsize{0.184}} & \mc{1}{c}{\scriptsize{0.240}} & \mc{1}{c}{\scriptsize{-0.004}} & \mc{1}{c}{\scriptsize{0.008}} & \mc{1}{c}{\scriptsize{-0.018}} \\  

     &  & \mc{1}{c}{\scriptsize{(0.289)}} & \mc{1}{c}{\scriptsize{(0.368)}} & \mc{1}{c}{\scriptsize{\textbf{(0.066)}}} & \mc{1}{c}{\scriptsize{(0.145)}} & \mc{1}{c}{\scriptsize{\textbf{(0.079)}}} & \mc{1}{c}{\scriptsize{(0.447)}} & \mc{1}{c}{\scriptsize{(0.487)}} & \mc{1}{c}{\scriptsize{(0.539)}} \\  

     & \mc{1}{c}{\scriptsize{3}} & \mc{1}{c}{\scriptsize{0.150}} & \mc{1}{c}{\scriptsize{0.125}} & \mc{1}{c}{\scriptsize{0.261}} & \mc{1}{c}{\scriptsize{0.184}} & \mc{1}{c}{\scriptsize{0.240}} & \mc{1}{c}{\scriptsize{0.117}} & \mc{1}{c}{\scriptsize{0.117}} & \mc{1}{c}{\scriptsize{0.117}} \\  

     &  & \mc{1}{c}{\scriptsize{\textbf{(0.066)}}} & \mc{1}{c}{\scriptsize{(0.184)}} & \mc{1}{c}{\scriptsize{\textbf{(0.079)}}} & \mc{1}{c}{\scriptsize{(0.145)}} & \mc{1}{c}{\scriptsize{\textbf{(0.079)}}} & \mc{1}{c}{\scriptsize{(0.158)}} & \mc{1}{c}{\scriptsize{(0.197)}} & \mc{1}{c}{\scriptsize{(0.171)}} \\  

     & \mc{1}{c}{\scriptsize{4}} & \mc{1}{c}{\scriptsize{0.134}} & \mc{1}{c}{\scriptsize{0.147}} & \mc{1}{c}{\scriptsize{0.287}} & \mc{1}{c}{\scriptsize{0.262}} & \mc{1}{c}{\scriptsize{0.270}} & \mc{1}{c}{\scriptsize{0.090}} & \mc{1}{c}{\scriptsize{0.124}} & \mc{1}{c}{\scriptsize{0.089}} \\  

     &  & \mc{1}{c}{\scriptsize{\textbf{(0.079)}}} & \mc{1}{c}{\scriptsize{(0.105)}} & \mc{1}{c}{\scriptsize{\textbf{(0.053)}}} & \mc{1}{c}{\scriptsize{(0.132)}} & \mc{1}{c}{\scriptsize{\textbf{(0.066)}}} & \mc{1}{c}{\scriptsize{(0.184)}} & \mc{1}{c}{\scriptsize{(0.158)}} & \mc{1}{c}{\scriptsize{(0.197)}} \\  

     & \mc{1}{c}{\scriptsize{5}} & \mc{1}{c}{\scriptsize{0.111}} & \mc{1}{c}{\scriptsize{0.095}} & \mc{1}{c}{\scriptsize{0.311}} & \mc{1}{c}{\scriptsize{0.259}} & \mc{1}{c}{\scriptsize{0.289}} & \mc{1}{c}{\scriptsize{0.061}} & \mc{1}{c}{\scriptsize{0.071}} & \mc{1}{c}{\scriptsize{0.054}} \\  

     &  & \mc{1}{c}{\scriptsize{\textbf{(0.092)}}} & \mc{1}{c}{\scriptsize{(0.211)}} & \mc{1}{c}{\scriptsize{\textbf{(0.053)}}} & \mc{1}{c}{\scriptsize{\textbf{(0.079)}}} & \mc{1}{c}{\scriptsize{\textbf{(0.066)}}} & \mc{1}{c}{\scriptsize{(0.276)}} & \mc{1}{c}{\scriptsize{(0.303)}} & \mc{1}{c}{\scriptsize{(0.316)}} \\  

    \mc{1}{l}{\scriptsize{Mother Works Factor}} & \mc{1}{c}{\scriptsize{2 to 21}} & \mc{1}{c}{\scriptsize{0.309}} & \mc{1}{c}{\scriptsize{0.280}} & \mc{1}{c}{\scriptsize{0.865}} & \mc{1}{c}{\scriptsize{0.730}} & \mc{1}{c}{\scriptsize{0.805}} & \mc{1}{c}{\scriptsize{0.160}} & \mc{1}{c}{\scriptsize{0.196}} & \mc{1}{c}{\scriptsize{0.144}} \\  

     &  & \mc{1}{c}{\scriptsize{(0.118)}} & \mc{1}{c}{\scriptsize{(0.211)}} & \mc{1}{c}{\scriptsize{\textbf{(0.066)}}} & \mc{1}{c}{\scriptsize{(0.184)}} & \mc{1}{c}{\scriptsize{\textbf{(0.079)}}} & \mc{1}{c}{\scriptsize{(0.237)}} & \mc{1}{c}{\scriptsize{(0.250)}} & \mc{1}{c}{\scriptsize{(0.250)}} \\ 
    \midrule  

    \mc{2}{l}{\scriptsize{\% of Pos. TE ($H_0$: $\le$ 50\%)}} & \mc{1}{c}{\scriptsize{100}} & \mc{1}{c}{\scriptsize{100}} & \mc{1}{c}{\scriptsize{100}} & \mc{1}{c}{\scriptsize{100}} & \mc{1}{c}{\scriptsize{100}} & \mc{1}{c}{\scriptsize{80}} & \mc{1}{c}{\scriptsize{100}} & \mc{1}{c}{\scriptsize{80}} \\  

     &  & \mc{1}{c}{\scriptsize{\textbf{(0.000)}}} & \mc{1}{c}{\scriptsize{\textbf{(0.000)}}} & \mc{1}{c}{\scriptsize{\textbf{(0.000)}}} & \mc{1}{c}{\scriptsize{\textbf{(0.000)}}} & \mc{1}{c}{\scriptsize{\textbf{(0.000)}}} & \mc{1}{c}{\scriptsize{\textbf{(0.000)}}} & \mc{1}{c}{\scriptsize{\textbf{(0.000)}}} & \mc{1}{c}{\scriptsize{\textbf{(0.000)}}} \\  

    \mc{2}{l}{\scriptsize{\% of Pos. TE ($H_0$: $\le$ 10\% $|$ 10\% Significance)}} & \mc{1}{c}{\scriptsize{60}} & \mc{1}{c}{\scriptsize{20}} & \mc{1}{c}{\scriptsize{40}} & \mc{1}{c}{\scriptsize{0}} & \mc{1}{c}{\scriptsize{40}} & \mc{1}{c}{\scriptsize{0}} & \mc{1}{c}{\scriptsize{0}} & \mc{1}{c}{\scriptsize{0}} \\  

     &  & \mc{1}{c}{\scriptsize{\textbf{(0.092)}}} & \mc{1}{c}{\scriptsize{(0.316)}} & \mc{1}{c}{\scriptsize{(0.342)}} & \mc{1}{c}{\scriptsize{(0.237)}} & \mc{1}{c}{\scriptsize{(0.289)}} & \mc{1}{c}{\scriptsize{(0.513)}} & \mc{1}{c}{\scriptsize{(0.487)}} & \mc{1}{c}{\scriptsize{(0.474)}} \\  

  \bottomrule
  \end{tabular}
	\end{table} 

	\begin{table}[H]
     \caption{Treatment Effects on Classroom Behavior Inventory (Part II), Male Sample}
     \label{table:abccare_rslt_male_cat5}
	  \begin{tabular}{cccccccccc}
  \toprule

    \scriptsize{Variable} & \scriptsize{Age} & \scriptsize{(1)} & \scriptsize{(2)} & \scriptsize{(3)} & \scriptsize{(4)} & \scriptsize{(5)} & \scriptsize{(6)} & \scriptsize{(7)} & \scriptsize{(8)} \\ 
    \midrule  

    \mc{1}{l}{\scriptsize{Dependence}} & \mc{1}{c}{\scriptsize{6}} & \mc{1}{c}{\scriptsize{1.316}} & \mc{1}{c}{\scriptsize{0.579}} & \mc{1}{c}{\scriptsize{2.009}} & \mc{1}{c}{\scriptsize{1.425}} & \mc{1}{c}{\scriptsize{1.360}} & \mc{1}{c}{\scriptsize{1.130}} & \mc{1}{c}{\scriptsize{0.409}} & \mc{1}{c}{\scriptsize{0.530}} \\  

     &  & \mc{1}{c}{\scriptsize{(0.934)}} & \mc{1}{c}{\scriptsize{(0.737)}} & \mc{1}{c}{\scriptsize{(0.974)}} & \mc{1}{c}{\scriptsize{(0.908)}} & \mc{1}{c}{\scriptsize{(0.934)}} & \mc{1}{c}{\scriptsize{(0.921)}} & \mc{1}{c}{\scriptsize{(0.645)}} & \mc{1}{c}{\scriptsize{(0.776)}} \\  

     & \mc{1}{c}{\scriptsize{7}} & \mc{1}{c}{\scriptsize{1.130}} & \mc{1}{c}{\scriptsize{1.678}} & \mc{1}{c}{\scriptsize{3.645}} & \mc{1}{c}{\scriptsize{5.228}} & \mc{1}{c}{\scriptsize{3.832}} & \mc{1}{c}{\scriptsize{0.530}} & \mc{1}{c}{\scriptsize{1.137}} & \mc{1}{c}{\scriptsize{0.631}} \\  

     &  & \mc{1}{c}{\scriptsize{(0.961)}} & \mc{1}{c}{\scriptsize{(0.947)}} & \mc{1}{c}{\scriptsize{(0.987)}} & \mc{1}{c}{\scriptsize{(0.974)}} & \mc{1}{c}{\scriptsize{(0.987)}} & \mc{1}{c}{\scriptsize{(0.671)}} & \mc{1}{c}{\scriptsize{(0.829)}} & \mc{1}{c}{\scriptsize{(0.684)}} \\  

     & \mc{1}{c}{\scriptsize{8}} & \mc{1}{c}{\scriptsize{1.818}} & \mc{1}{c}{\scriptsize{1.939}} & \mc{1}{c}{\scriptsize{2.857}} & \mc{1}{c}{\scriptsize{3.565}} & \mc{1}{c}{\scriptsize{2.616}} & \mc{1}{c}{\scriptsize{1.560}} & \mc{1}{c}{\scriptsize{1.669}} & \mc{1}{c}{\scriptsize{1.428}} \\  

     &  & \mc{1}{c}{\scriptsize{(0.987)}} & \mc{1}{c}{\scriptsize{(0.974)}} & \mc{1}{c}{\scriptsize{(0.987)}} & \mc{1}{c}{\scriptsize{(0.961)}} & \mc{1}{c}{\scriptsize{(0.961)}} & \mc{1}{c}{\scriptsize{(0.974)}} & \mc{1}{c}{\scriptsize{(0.947)}} & \mc{1}{c}{\scriptsize{(0.947)}} \\  

     & \mc{1}{c}{\scriptsize{12}} & \mc{1}{c}{\scriptsize{1.979}} & \mc{1}{c}{\scriptsize{1.435}} & \mc{1}{c}{\scriptsize{2.523}} & \mc{1}{c}{\scriptsize{2.704}} & \mc{1}{c}{\scriptsize{2.599}} & \mc{1}{c}{\scriptsize{1.769}} & \mc{1}{c}{\scriptsize{1.153}} & \mc{1}{c}{\scriptsize{1.620}} \\  

     &  & \mc{1}{c}{\scriptsize{(0.947)}} & \mc{1}{c}{\scriptsize{(0.842)}} & \mc{1}{c}{\scriptsize{(0.947)}} & \mc{1}{c}{\scriptsize{\textbf{(0.026)}}} & \mc{1}{c}{\scriptsize{(0.961)}} & \mc{1}{c}{\scriptsize{(0.947)}} & \mc{1}{c}{\scriptsize{(0.789)}} & \mc{1}{c}{\scriptsize{(0.961)}} \\  

    \mc{1}{l}{\scriptsize{Distractibility}} & \mc{1}{c}{\scriptsize{6}} & \mc{1}{c}{\scriptsize{0.758}} & \mc{1}{c}{\scriptsize{0.117}} & \mc{1}{c}{\scriptsize{1.714}} & \mc{1}{c}{\scriptsize{1.528}} & \mc{1}{c}{\scriptsize{1.018}} & \mc{1}{c}{\scriptsize{0.500}} & \mc{1}{c}{\scriptsize{-0.074}} & \mc{1}{c}{\scriptsize{-0.318}} \\  

     &  & \mc{1}{c}{\scriptsize{(0.842)}} & \mc{1}{c}{\scriptsize{(0.605)}} & \mc{1}{c}{\scriptsize{(0.895)}} & \mc{1}{c}{\scriptsize{(0.803)}} & \mc{1}{c}{\scriptsize{(0.750)}} & \mc{1}{c}{\scriptsize{(0.737)}} & \mc{1}{c}{\scriptsize{(0.513)}} & \mc{1}{c}{\scriptsize{(0.342)}} \\  

     & \mc{1}{c}{\scriptsize{7}} & \mc{1}{c}{\scriptsize{0.764}} & \mc{1}{c}{\scriptsize{0.939}} & \mc{1}{c}{\scriptsize{2.613}} & \mc{1}{c}{\scriptsize{3.917}} & \mc{1}{c}{\scriptsize{2.261}} & \mc{1}{c}{\scriptsize{0.267}} & \mc{1}{c}{\scriptsize{0.507}} & \mc{1}{c}{\scriptsize{0.003}} \\  

     &  & \mc{1}{c}{\scriptsize{(0.829)}} & \mc{1}{c}{\scriptsize{(0.855)}} & \mc{1}{c}{\scriptsize{(0.987)}} & \mc{1}{c}{\scriptsize{(0.974)}} & \mc{1}{c}{\scriptsize{(0.947)}} & \mc{1}{c}{\scriptsize{(0.632)}} & \mc{1}{c}{\scriptsize{(0.763)}} & \mc{1}{c}{\scriptsize{(0.474)}} \\  

     & \mc{1}{c}{\scriptsize{8}} & \mc{1}{c}{\scriptsize{0.602}} & \mc{1}{c}{\scriptsize{0.012}} & \mc{1}{c}{\scriptsize{1.290}} & \mc{1}{c}{\scriptsize{1.175}} & \mc{1}{c}{\scriptsize{0.818}} & \mc{1}{c}{\scriptsize{0.387}} & \mc{1}{c}{\scriptsize{-0.176}} & \mc{1}{c}{\scriptsize{-0.063}} \\  

     &  & \mc{1}{c}{\scriptsize{(0.855)}} & \mc{1}{c}{\scriptsize{(0.566)}} & \mc{1}{c}{\scriptsize{(0.816)}} & \mc{1}{c}{\scriptsize{(0.750)}} & \mc{1}{c}{\scriptsize{(0.724)}} & \mc{1}{c}{\scriptsize{(0.724)}} & \mc{1}{c}{\scriptsize{(0.461)}} & \mc{1}{c}{\scriptsize{(0.513)}} \\  

     & \mc{1}{c}{\scriptsize{12}} & \mc{1}{c}{\scriptsize{1.534}} & \mc{1}{c}{\scriptsize{0.704}} & \mc{1}{c}{\scriptsize{3.292}} & \mc{1}{c}{\scriptsize{2.945}} & \mc{1}{c}{\scriptsize{2.654}} & \mc{1}{c}{\scriptsize{0.907}} & \mc{1}{c}{\scriptsize{0.099}} & \mc{1}{c}{\scriptsize{0.989}} \\  

     &  & \mc{1}{c}{\scriptsize{(0.961)}} & \mc{1}{c}{\scriptsize{(0.645)}} & \mc{1}{c}{\scriptsize{(0.961)}} & \mc{1}{c}{\scriptsize{\textbf{(0.026)}}} & \mc{1}{c}{\scriptsize{(0.921)}} & \mc{1}{c}{\scriptsize{(0.816)}} & \mc{1}{c}{\scriptsize{(0.553)}} & \mc{1}{c}{\scriptsize{(0.868)}} \\  

    \mc{1}{l}{\scriptsize{Hostility}} & \mc{1}{c}{\scriptsize{6}} & \mc{1}{c}{\scriptsize{2.532}} & \mc{1}{c}{\scriptsize{2.575}} & \mc{1}{c}{\scriptsize{3.848}} & \mc{1}{c}{\scriptsize{4.068}} & \mc{1}{c}{\scriptsize{3.749}} & \mc{1}{c}{\scriptsize{2.178}} & \mc{1}{c}{\scriptsize{2.303}} & \mc{1}{c}{\scriptsize{1.865}} \\  

     &  & \mc{1}{c}{\scriptsize{(1.000)}} & \mc{1}{c}{\scriptsize{(1.000)}} & \mc{1}{c}{\scriptsize{(0.987)}} & \mc{1}{c}{\scriptsize{(0.974)}} & \mc{1}{c}{\scriptsize{(0.987)}} & \mc{1}{c}{\scriptsize{(1.000)}} & \mc{1}{c}{\scriptsize{(1.000)}} & \mc{1}{c}{\scriptsize{(0.987)}} \\  

     & \mc{1}{c}{\scriptsize{7}} & \mc{1}{c}{\scriptsize{1.351}} & \mc{1}{c}{\scriptsize{1.897}} & \mc{1}{c}{\scriptsize{1.457}} & \mc{1}{c}{\scriptsize{3.575}} & \mc{1}{c}{\scriptsize{1.132}} & \mc{1}{c}{\scriptsize{1.367}} & \mc{1}{c}{\scriptsize{1.692}} & \mc{1}{c}{\scriptsize{0.915}} \\  

     &  & \mc{1}{c}{\scriptsize{(0.961)}} & \mc{1}{c}{\scriptsize{(0.987)}} & \mc{1}{c}{\scriptsize{(0.882)}} & \mc{1}{c}{\scriptsize{(0.987)}} & \mc{1}{c}{\scriptsize{(0.803)}} & \mc{1}{c}{\scriptsize{(0.974)}} & \mc{1}{c}{\scriptsize{(0.934)}} & \mc{1}{c}{\scriptsize{(0.829)}} \\  

     & \mc{1}{c}{\scriptsize{8}} & \mc{1}{c}{\scriptsize{2.790}} & \mc{1}{c}{\scriptsize{3.086}} & \mc{1}{c}{\scriptsize{2.941}} & \mc{1}{c}{\scriptsize{4.163}} & \mc{1}{c}{\scriptsize{3.212}} & \mc{1}{c}{\scriptsize{2.781}} & \mc{1}{c}{\scriptsize{2.906}} & \mc{1}{c}{\scriptsize{2.825}} \\  

     &  & \mc{1}{c}{\scriptsize{(1.000)}} & \mc{1}{c}{\scriptsize{(1.000)}} & \mc{1}{c}{\scriptsize{(0.987)}} & \mc{1}{c}{\scriptsize{(0.987)}} & \mc{1}{c}{\scriptsize{(0.987)}} & \mc{1}{c}{\scriptsize{(1.000)}} & \mc{1}{c}{\scriptsize{(1.000)}} & \mc{1}{c}{\scriptsize{(1.000)}} \\  

     & \mc{1}{c}{\scriptsize{12}} & \mc{1}{c}{\scriptsize{3.923}} & \mc{1}{c}{\scriptsize{3.384}} & \mc{1}{c}{\scriptsize{4.923}} & \mc{1}{c}{\scriptsize{4.178}} & \mc{1}{c}{\scriptsize{4.197}} & \mc{1}{c}{\scriptsize{3.566}} & \mc{1}{c}{\scriptsize{3.181}} & \mc{1}{c}{\scriptsize{2.732}} \\  

     &  & \mc{1}{c}{\scriptsize{(1.000)}} & \mc{1}{c}{\scriptsize{(0.947)}} & \mc{1}{c}{\scriptsize{(0.974)}} & \mc{1}{c}{\scriptsize{(0.947)}} & \mc{1}{c}{\scriptsize{(0.974)}} & \mc{1}{c}{\scriptsize{(0.987)}} & \mc{1}{c}{\scriptsize{(0.895)}} & \mc{1}{c}{\scriptsize{(0.987)}} \\  

    \mc{1}{l}{\scriptsize{Introversion}} & \mc{1}{c}{\scriptsize{6}} & \mc{1}{c}{\scriptsize{0.544}} & \mc{1}{c}{\scriptsize{0.891}} & \mc{1}{c}{\scriptsize{0.540}} & \mc{1}{c}{\scriptsize{1.107}} & \mc{1}{c}{\scriptsize{0.347}} & \mc{1}{c}{\scriptsize{0.546}} & \mc{1}{c}{\scriptsize{0.961}} & \mc{1}{c}{\scriptsize{0.367}} \\  

     &  & \mc{1}{c}{\scriptsize{(0.829)}} & \mc{1}{c}{\scriptsize{(0.895)}} & \mc{1}{c}{\scriptsize{(0.697)}} & \mc{1}{c}{\scriptsize{(0.776)}} & \mc{1}{c}{\scriptsize{(0.645)}} & \mc{1}{c}{\scriptsize{(0.803)}} & \mc{1}{c}{\scriptsize{(0.895)}} & \mc{1}{c}{\scriptsize{(0.763)}} \\  

     & \mc{1}{c}{\scriptsize{7}} & \mc{1}{c}{\scriptsize{1.380}} & \mc{1}{c}{\scriptsize{1.668}} & \mc{1}{c}{\scriptsize{0.941}} & \mc{1}{c}{\scriptsize{1.315}} & \mc{1}{c}{\scriptsize{0.796}} & \mc{1}{c}{\scriptsize{1.543}} & \mc{1}{c}{\scriptsize{1.715}} & \mc{1}{c}{\scriptsize{1.508}} \\  

     &  & \mc{1}{c}{\scriptsize{(1.000)}} & \mc{1}{c}{\scriptsize{(0.987)}} & \mc{1}{c}{\scriptsize{(0.829)}} & \mc{1}{c}{\scriptsize{(0.855)}} & \mc{1}{c}{\scriptsize{(0.789)}} & \mc{1}{c}{\scriptsize{(0.987)}} & \mc{1}{c}{\scriptsize{(1.000)}} & \mc{1}{c}{\scriptsize{(0.974)}} \\  

     & \mc{1}{c}{\scriptsize{8}} & \mc{1}{c}{\scriptsize{-0.208}} & \mc{1}{c}{\scriptsize{-0.492}} & \mc{1}{c}{\scriptsize{0.252}} & \mc{1}{c}{\scriptsize{-0.880}} & \mc{1}{c}{\scriptsize{0.152}} & \mc{1}{c}{\scriptsize{-0.218}} & \mc{1}{c}{\scriptsize{-0.320}} & \mc{1}{c}{\scriptsize{-0.320}} \\  

     &  & \mc{1}{c}{\scriptsize{(0.329)}} & \mc{1}{c}{\scriptsize{(0.250)}} & \mc{1}{c}{\scriptsize{(0.658)}} & \mc{1}{c}{\scriptsize{(0.118)}} & \mc{1}{c}{\scriptsize{(0.579)}} & \mc{1}{c}{\scriptsize{(0.355)}} & \mc{1}{c}{\scriptsize{(0.329)}} & \mc{1}{c}{\scriptsize{(0.316)}} \\  

     & \mc{1}{c}{\scriptsize{12}} & \mc{1}{c}{\scriptsize{0.830}} & \mc{1}{c}{\scriptsize{0.117}} & \mc{1}{c}{\scriptsize{1.062}} & \mc{1}{c}{\scriptsize{1.354}} & \mc{1}{c}{\scriptsize{0.204}} & \mc{1}{c}{\scriptsize{0.747}} & \mc{1}{c}{\scriptsize{0.574}} & \mc{1}{c}{\scriptsize{0.059}} \\  

     &  & \mc{1}{c}{\scriptsize{(0.763)}} & \mc{1}{c}{\scriptsize{(0.553)}} & \mc{1}{c}{\scriptsize{(0.803)}} & \mc{1}{c}{\scriptsize{(0.947)}} & \mc{1}{c}{\scriptsize{(0.539)}} & \mc{1}{c}{\scriptsize{(0.763)}} & \mc{1}{c}{\scriptsize{(0.632)}} & \mc{1}{c}{\scriptsize{(0.487)}} \\  

  \bottomrule
  \end{tabular}
	\end{table} 

	\begin{table}[H]
     \caption{Treatment Effects on Emotional, Activity, Sociability, Impulsivity Survey, Male Sample}
     \label{table:abccare_rslt_male_cat6}
	  \begin{tabular}{cccccccccc}
  \toprule

    \scriptsize{Variable} & \scriptsize{Age} & \scriptsize{(1)} & \scriptsize{(2)} & \scriptsize{(3)} & \scriptsize{(4)} & \scriptsize{(5)} & \scriptsize{(6)} & \scriptsize{(7)} & \scriptsize{(8)} \\ 
    \midrule  

    \mc{1}{l}{\scriptsize{Activity - tempo}} & \mc{1}{c}{\scriptsize{8}} & \mc{1}{c}{\scriptsize{0.281}} & \mc{1}{c}{\scriptsize{-0.949}} & \mc{1}{c}{\scriptsize{-1.170}} & \mc{1}{c}{\scriptsize{-2.661}} & \mc{1}{c}{\scriptsize{-1.476}} & \mc{1}{c}{\scriptsize{0.896}} & \mc{1}{c}{\scriptsize{-0.360}} & \mc{1}{c}{\scriptsize{0.395}} \\  

     &  & \mc{1}{c}{\scriptsize{(0.645)}} & \mc{1}{c}{\scriptsize{(0.250)}} & \mc{1}{c}{\scriptsize{(0.250)}} & \mc{1}{c}{\scriptsize{(0.118)}} & \mc{1}{c}{\scriptsize{(0.237)}} & \mc{1}{c}{\scriptsize{(0.789)}} & \mc{1}{c}{\scriptsize{(0.368)}} & \mc{1}{c}{\scriptsize{(0.632)}} \\  

    \mc{1}{l}{\scriptsize{Activity - vigor}} & \mc{1}{c}{\scriptsize{8}} & \mc{1}{c}{\scriptsize{0.062}} & \mc{1}{c}{\scriptsize{-0.509}} & \mc{1}{c}{\scriptsize{-1.326}} & \mc{1}{c}{\scriptsize{-1.246}} & \mc{1}{c}{\scriptsize{-1.509}} & \mc{1}{c}{\scriptsize{0.448}} & \mc{1}{c}{\scriptsize{-0.182}} & \mc{1}{c}{\scriptsize{0.313}} \\  

     &  & \mc{1}{c}{\scriptsize{(0.461)}} & \mc{1}{c}{\scriptsize{(0.697)}} & \mc{1}{c}{\scriptsize{(0.934)}} & \mc{1}{c}{\scriptsize{(0.789)}} & \mc{1}{c}{\scriptsize{(0.961)}} & \mc{1}{c}{\scriptsize{(0.329)}} & \mc{1}{c}{\scriptsize{(0.579)}} & \mc{1}{c}{\scriptsize{(0.342)}} \\  

    \mc{1}{l}{\scriptsize{Emotionality - anger}} & \mc{1}{c}{\scriptsize{8}} & \mc{1}{c}{\scriptsize{0.438}} & \mc{1}{c}{\scriptsize{0.433}} & \mc{1}{c}{\scriptsize{0.120}} & \mc{1}{c}{\scriptsize{-0.522}} & \mc{1}{c}{\scriptsize{0.382}} & \mc{1}{c}{\scriptsize{0.865}} & \mc{1}{c}{\scriptsize{0.837}} & \mc{1}{c}{\scriptsize{0.993}} \\  

     &  & \mc{1}{c}{\scriptsize{(0.671)}} & \mc{1}{c}{\scriptsize{(0.632)}} & \mc{1}{c}{\scriptsize{(0.500)}} & \mc{1}{c}{\scriptsize{(0.421)}} & \mc{1}{c}{\scriptsize{(0.618)}} & \mc{1}{c}{\scriptsize{(0.776)}} & \mc{1}{c}{\scriptsize{(0.724)}} & \mc{1}{c}{\scriptsize{(0.789)}} \\  

    \mc{1}{l}{\scriptsize{Emotionality - fear}} & \mc{1}{c}{\scriptsize{8}} & \mc{1}{c}{\scriptsize{1.094}} & \mc{1}{c}{\scriptsize{0.906}} & \mc{1}{c}{\scriptsize{0.464}} & \mc{1}{c}{\scriptsize{-0.635}} & \mc{1}{c}{\scriptsize{0.721}} & \mc{1}{c}{\scriptsize{1.500}} & \mc{1}{c}{\scriptsize{1.464}} & \mc{1}{c}{\scriptsize{1.769}} \\  

     &  & \mc{1}{c}{\scriptsize{(0.829)}} & \mc{1}{c}{\scriptsize{(0.737)}} & \mc{1}{c}{\scriptsize{(0.605)}} & \mc{1}{c}{\scriptsize{(0.342)}} & \mc{1}{c}{\scriptsize{(0.684)}} & \mc{1}{c}{\scriptsize{(0.921)}} & \mc{1}{c}{\scriptsize{(0.842)}} & \mc{1}{c}{\scriptsize{(0.947)}} \\  

    \mc{1}{l}{\scriptsize{Emotionality - general}} & \mc{1}{c}{\scriptsize{8}} & \mc{1}{c}{\scriptsize{0.969}} & \mc{1}{c}{\scriptsize{0.827}} & \mc{1}{c}{\scriptsize{3.634}} & \mc{1}{c}{\scriptsize{3.741}} & \mc{1}{c}{\scriptsize{3.617}} & \mc{1}{c}{\scriptsize{0.229}} & \mc{1}{c}{\scriptsize{-0.017}} & \mc{1}{c}{\scriptsize{-0.011}} \\  

     &  & \mc{1}{c}{\scriptsize{(0.855)}} & \mc{1}{c}{\scriptsize{(0.763)}} & \mc{1}{c}{\scriptsize{(0.974)}} & \mc{1}{c}{\scriptsize{(0.934)}} & \mc{1}{c}{\scriptsize{(0.974)}} & \mc{1}{c}{\scriptsize{(0.566)}} & \mc{1}{c}{\scriptsize{(0.539)}} & \mc{1}{c}{\scriptsize{(0.526)}} \\  

    \mc{1}{l}{\scriptsize{Impulsitivity - control}} & \mc{1}{c}{\scriptsize{8}} & \mc{1}{c}{\scriptsize{0.938}} & \mc{1}{c}{\scriptsize{1.254}} & \mc{1}{c}{\scriptsize{1.585}} & \mc{1}{c}{\scriptsize{1.912}} & \mc{1}{c}{\scriptsize{1.605}} & \mc{1}{c}{\scriptsize{1.073}} & \mc{1}{c}{\scriptsize{1.311}} & \mc{1}{c}{\scriptsize{1.355}} \\  

     &  & \mc{1}{c}{\scriptsize{(0.921)}} & \mc{1}{c}{\scriptsize{(0.961)}} & \mc{1}{c}{\scriptsize{(0.921)}} & \mc{1}{c}{\scriptsize{(0.895)}} & \mc{1}{c}{\scriptsize{(0.921)}} & \mc{1}{c}{\scriptsize{(0.947)}} & \mc{1}{c}{\scriptsize{(0.961)}} & \mc{1}{c}{\scriptsize{(0.974)}} \\  

    \mc{1}{l}{\scriptsize{Impulsitivity - decisive}} & \mc{1}{c}{\scriptsize{8}} & \mc{1}{c}{\scriptsize{0.344}} & \mc{1}{c}{\scriptsize{-0.517}} & \mc{1}{c}{\scriptsize{2.013}} & \mc{1}{c}{\scriptsize{1.645}} & \mc{1}{c}{\scriptsize{1.679}} & \mc{1}{c}{\scriptsize{0.115}} & \mc{1}{c}{\scriptsize{-0.917}} & \mc{1}{c}{\scriptsize{-0.279}} \\  

     &  & \mc{1}{c}{\scriptsize{(0.658)}} & \mc{1}{c}{\scriptsize{(0.276)}} & \mc{1}{c}{\scriptsize{(0.987)}} & \mc{1}{c}{\scriptsize{(0.934)}} & \mc{1}{c}{\scriptsize{(0.974)}} & \mc{1}{c}{\scriptsize{(0.566)}} & \mc{1}{c}{\scriptsize{(0.132)}} & \mc{1}{c}{\scriptsize{(0.421)}} \\  

    \mc{1}{l}{\scriptsize{Impulsitivity - perservere}} & \mc{1}{c}{\scriptsize{8}} & \mc{1}{c}{\scriptsize{-0.406}} & \mc{1}{c}{\scriptsize{-0.255}} & \mc{1}{c}{\scriptsize{0.746}} & \mc{1}{c}{\scriptsize{1.570}} & \mc{1}{c}{\scriptsize{0.497}} & \mc{1}{c}{\scriptsize{-0.677}} & \mc{1}{c}{\scriptsize{-0.680}} & \mc{1}{c}{\scriptsize{-0.681}} \\  

     &  & \mc{1}{c}{\scriptsize{(0.329)}} & \mc{1}{c}{\scriptsize{(0.408)}} & \mc{1}{c}{\scriptsize{(0.711)}} & \mc{1}{c}{\scriptsize{(0.750)}} & \mc{1}{c}{\scriptsize{(0.684)}} & \mc{1}{c}{\scriptsize{(0.211)}} & \mc{1}{c}{\scriptsize{(0.237)}} & \mc{1}{c}{\scriptsize{(0.224)}} \\  

    \mc{1}{l}{\scriptsize{Impulsitivity - sensation}} & \mc{1}{c}{\scriptsize{8}} & \mc{1}{c}{\scriptsize{-0.500}} & \mc{1}{c}{\scriptsize{-0.396}} & \mc{1}{c}{\scriptsize{-1.085}} & \mc{1}{c}{\scriptsize{-0.529}} & \mc{1}{c}{\scriptsize{-1.019}} & \mc{1}{c}{\scriptsize{-0.240}} & \mc{1}{c}{\scriptsize{-0.234}} & \mc{1}{c}{\scriptsize{-0.240}} \\  

     &  & \mc{1}{c}{\scriptsize{(0.237)}} & \mc{1}{c}{\scriptsize{(0.342)}} & \mc{1}{c}{\scriptsize{(0.316)}} & \mc{1}{c}{\scriptsize{(0.250)}} & \mc{1}{c}{\scriptsize{(0.329)}} & \mc{1}{c}{\scriptsize{(0.382)}} & \mc{1}{c}{\scriptsize{(0.395)}} & \mc{1}{c}{\scriptsize{(0.408)}} \\  

    \mc{1}{l}{\scriptsize{Sociablity}} & \mc{1}{c}{\scriptsize{8}} & \mc{1}{c}{\scriptsize{0.094}} & \mc{1}{c}{\scriptsize{0.133}} & \mc{1}{c}{\scriptsize{-0.844}} & \mc{1}{c}{\scriptsize{0.198}} & \mc{1}{c}{\scriptsize{-0.738}} & \mc{1}{c}{\scriptsize{0.115}} & \mc{1}{c}{\scriptsize{0.140}} & \mc{1}{c}{\scriptsize{0.244}} \\  

     &  & \mc{1}{c}{\scriptsize{(0.408)}} & \mc{1}{c}{\scriptsize{(0.434)}} & \mc{1}{c}{\scriptsize{(0.737)}} & \mc{1}{c}{\scriptsize{(0.434)}} & \mc{1}{c}{\scriptsize{(0.737)}} & \mc{1}{c}{\scriptsize{(0.434)}} & \mc{1}{c}{\scriptsize{(0.447)}} & \mc{1}{c}{\scriptsize{(0.368)}} \\  

  \bottomrule
  \end{tabular}
	\end{table} 

	\begin{table}[H]
     \caption{Treatment Effects on Harter Importance, Male Sample}
     \label{table:abccare_rslt_male_cat7}
	  \begin{tabular}{cccccccccc}
  \toprule

    \scriptsize{Variable} & \scriptsize{Age} & \scriptsize{(1)} & \scriptsize{(2)} & \scriptsize{(3)} & \scriptsize{(4)} & \scriptsize{(5)} & \scriptsize{(6)} & \scriptsize{(7)} & \scriptsize{(8)} \\ 
    \midrule  

    \mc{1}{l}{\scriptsize{Employed}} & \mc{1}{c}{\scriptsize{30}} & \mc{1}{c}{\scriptsize{0.119}} & \mc{1}{c}{\scriptsize{0.179}} & \mc{1}{c}{\scriptsize{-0.029}} & \mc{1}{c}{\scriptsize{-0.050}} & \mc{1}{c}{\scriptsize{0.041}} & \mc{1}{c}{\scriptsize{0.176}} & \mc{1}{c}{\scriptsize{0.245}} & \mc{1}{c}{\scriptsize{0.262}} \\  

     &  & \mc{1}{c}{\scriptsize{\textbf{(0.079)}}} & \mc{1}{c}{\scriptsize{\textbf{(0.039)}}} & \mc{1}{c}{\scriptsize{(0.487)}} & \mc{1}{c}{\scriptsize{(0.579)}} & \mc{1}{c}{\scriptsize{(0.355)}} & \mc{1}{c}{\scriptsize{\textbf{(0.053)}}} & \mc{1}{c}{\scriptsize{\textbf{(0.013)}}} & \mc{1}{c}{\scriptsize{\textbf{(0.000)}}} \\  

    \mc{1}{l}{\scriptsize{Labor Income}} & \mc{1}{c}{\scriptsize{21}} & \mc{1}{c}{\scriptsize{-1,672}} & \mc{1}{c}{\scriptsize{-4,977}} & \mc{1}{c}{\scriptsize{-3,951}} & \mc{1}{c}{\scriptsize{-13,951}} & \mc{1}{c}{\scriptsize{-4,585}} & \mc{1}{c}{\scriptsize{-1,527}} & \mc{1}{c}{\scriptsize{-3,973}} & \mc{1}{c}{\scriptsize{-3,746}} \\  

     &  & \mc{1}{c}{\scriptsize{(0.711)}} & \mc{1}{c}{\scriptsize{(0.895)}} & \mc{1}{c}{\scriptsize{(0.724)}} & \mc{1}{c}{\scriptsize{(0.974)}} & \mc{1}{c}{\scriptsize{(0.737)}} & \mc{1}{c}{\scriptsize{(0.724)}} & \mc{1}{c}{\scriptsize{(0.842)}} & \mc{1}{c}{\scriptsize{(0.855)}} \\  

     & \mc{1}{c}{\scriptsize{30}} & \mc{1}{c}{\scriptsize{19,810}} & \mc{1}{c}{\scriptsize{24,902}} & \mc{1}{c}{\scriptsize{17,909}} & \mc{1}{c}{\scriptsize{21,069}} & \mc{1}{c}{\scriptsize{24,012}} & \mc{1}{c}{\scriptsize{20,065}} & \mc{1}{c}{\scriptsize{28,483}} & \mc{1}{c}{\scriptsize{21,170}} \\  

     &  & \mc{1}{c}{\scriptsize{\textbf{(0.079)}}} & \mc{1}{c}{\scriptsize{(0.171)}} & \mc{1}{c}{\scriptsize{(0.132)}} & \mc{1}{c}{\scriptsize{(0.263)}} & \mc{1}{c}{\scriptsize{(0.105)}} & \mc{1}{c}{\scriptsize{\textbf{(0.066)}}} & \mc{1}{c}{\scriptsize{(0.132)}} & \mc{1}{c}{\scriptsize{(0.158)}} \\  

    \mc{1}{l}{\scriptsize{Public-Transfer Income}} & \mc{1}{c}{\scriptsize{21}} & \mc{1}{c}{\scriptsize{315}} & \mc{1}{c}{\scriptsize{456}} & \mc{1}{c}{\scriptsize{1,376}} & \mc{1}{c}{\scriptsize{2,654}} & \mc{1}{c}{\scriptsize{1,543}} & \mc{1}{c}{\scriptsize{-58.901}} & \mc{1}{c}{\scriptsize{144}} & \mc{1}{c}{\scriptsize{97.591}} \\  

     &  & \mc{1}{c}{\scriptsize{(0.684)}} & \mc{1}{c}{\scriptsize{(0.658)}} & \mc{1}{c}{\scriptsize{(0.868)}} & \mc{1}{c}{\scriptsize{(0.763)}} & \mc{1}{c}{\scriptsize{(0.803)}} & \mc{1}{c}{\scriptsize{(0.513)}} & \mc{1}{c}{\scriptsize{(0.618)}} & \mc{1}{c}{\scriptsize{(0.645)}} \\  

     & \mc{1}{c}{\scriptsize{30}} & \mc{1}{c}{\scriptsize{-530}} & \mc{1}{c}{\scriptsize{-176}} & \mc{1}{c}{\scriptsize{287}} & \mc{1}{c}{\scriptsize{722}} & \mc{1}{c}{\scriptsize{548}} & \mc{1}{c}{\scriptsize{-279}} & \mc{1}{c}{\scriptsize{-82.612}} & \mc{1}{c}{\scriptsize{-155}} \\  

     &  & \mc{1}{c}{\scriptsize{(0.118)}} & \mc{1}{c}{\scriptsize{(0.355)}} & \mc{1}{c}{\scriptsize{(0.539)}} & \mc{1}{c}{\scriptsize{(0.539)}} & \mc{1}{c}{\scriptsize{(0.579)}} & \mc{1}{c}{\scriptsize{(0.237)}} & \mc{1}{c}{\scriptsize{(0.395)}} & \mc{1}{c}{\scriptsize{(0.355)}} \\  

    \mc{1}{l}{\scriptsize{Employment Factor}} & \mc{1}{c}{\scriptsize{21 to 30}} & \mc{1}{c}{\scriptsize{0.351}} & \mc{1}{c}{\scriptsize{0.433}} & \mc{1}{c}{\scriptsize{0.053}} & \mc{1}{c}{\scriptsize{-0.119}} & \mc{1}{c}{\scriptsize{0.086}} & \mc{1}{c}{\scriptsize{0.450}} & \mc{1}{c}{\scriptsize{0.595}} & \mc{1}{c}{\scriptsize{0.486}} \\  

     &  & \mc{1}{c}{\scriptsize{\textbf{(0.026)}}} & \mc{1}{c}{\scriptsize{\textbf{(0.053)}}} & \mc{1}{c}{\scriptsize{(0.408)}} & \mc{1}{c}{\scriptsize{(0.539)}} & \mc{1}{c}{\scriptsize{(0.382)}} & \mc{1}{c}{\scriptsize{\textbf{(0.000)}}} & \mc{1}{c}{\scriptsize{\textbf{(0.013)}}} & \mc{1}{c}{\scriptsize{\textbf{(0.000)}}} \\ 
    \midrule  

    \mc{2}{l}{\scriptsize{\% of Pos. TE ($H_0$: $\le$ 50\%)}} & \mc{1}{c}{\scriptsize{67}} & \mc{1}{c}{\scriptsize{67}} & \mc{1}{c}{\scriptsize{33}} & \mc{1}{c}{\scriptsize{17}} & \mc{1}{c}{\scriptsize{50}} & \mc{1}{c}{\scriptsize{83}} & \mc{1}{c}{\scriptsize{67}} & \mc{1}{c}{\scriptsize{67}} \\  

     &  & \mc{1}{c}{\scriptsize{(0.118)}} & \mc{1}{c}{\scriptsize{\textbf{(0.053)}}} & \mc{1}{c}{\scriptsize{(0.671)}} & \mc{1}{c}{\scriptsize{(1.000)}} & \mc{1}{c}{\scriptsize{(0.474)}} & \mc{1}{c}{\scriptsize{\textbf{(0.000)}}} & \mc{1}{c}{\scriptsize{\textbf{(0.092)}}} & \mc{1}{c}{\scriptsize{(0.105)}} \\  

    \mc{2}{l}{\scriptsize{\% of Pos. TE ($H_0$: $\le$ 10\% $|$ 10\% Significance)}} & \mc{1}{c}{\scriptsize{33}} & \mc{1}{c}{\scriptsize{33}} & \mc{1}{c}{\scriptsize{0}} & \mc{1}{c}{\scriptsize{0}} & \mc{1}{c}{\scriptsize{0}} & \mc{1}{c}{\scriptsize{33}} & \mc{1}{c}{\scriptsize{33}} & \mc{1}{c}{\scriptsize{33}} \\  

     &  & \mc{1}{c}{\scriptsize{\textbf{(0.053)}}} & \mc{1}{c}{\scriptsize{(0.105)}} & \mc{1}{c}{\scriptsize{(1.000)}} & \mc{1}{c}{\scriptsize{(1.000)}} & \mc{1}{c}{\scriptsize{(1.000)}} & \mc{1}{c}{\scriptsize{\textbf{(0.066)}}} & \mc{1}{c}{\scriptsize{(0.158)}} & \mc{1}{c}{\scriptsize{\textbf{(0.079)}}} \\  

  \bottomrule
  \end{tabular}
	\end{table} 

	\begin{table}[H]
     \caption{Treatment Effects on Achenbach Behavior, Male Sample}
     \label{table:abccare_rslt_male_cat8}
	  \begin{tabular}{cccccccccc}
  \toprule

    \scriptsize{Variable} & \scriptsize{Age} & \scriptsize{(1)} & \scriptsize{(2)} & \scriptsize{(3)} & \scriptsize{(4)} & \scriptsize{(5)} & \scriptsize{(6)} & \scriptsize{(7)} & \scriptsize{(8)} \\ 
    \midrule  

    \mc{1}{l}{\scriptsize{Total Felony Arrests}} & \mc{1}{c}{\scriptsize{Mid-30s}} & \mc{1}{c}{\scriptsize{0.196}} & \mc{1}{c}{\scriptsize{0.683}} & \mc{1}{c}{\scriptsize{0.946}} & \mc{1}{c}{\scriptsize{2.132}} & \mc{1}{c}{\scriptsize{1.337}} & \mc{1}{c}{\scriptsize{0.017}} & \mc{1}{c}{\scriptsize{0.466}} & \mc{1}{c}{\scriptsize{0.183}} \\  

     &  & \mc{1}{c}{\scriptsize{(0.618)}} & \mc{1}{c}{\scriptsize{(0.711)}} & \mc{1}{c}{\scriptsize{(0.934)}} & \mc{1}{c}{\scriptsize{(0.895)}} & \mc{1}{c}{\scriptsize{(0.974)}} & \mc{1}{c}{\scriptsize{(0.487)}} & \mc{1}{c}{\scriptsize{(0.645)}} & \mc{1}{c}{\scriptsize{(0.553)}} \\  

    \mc{1}{l}{\scriptsize{Total Misdemeanor Arrests}} & \mc{1}{c}{\scriptsize{Mid-30s}} & \mc{1}{c}{\scriptsize{-0.501}} & \mc{1}{c}{\scriptsize{-0.239}} & \mc{1}{c}{\scriptsize{-0.251}} & \mc{1}{c}{\scriptsize{0.085}} & \mc{1}{c}{\scriptsize{-0.040}} & \mc{1}{c}{\scriptsize{-0.665}} & \mc{1}{c}{\scriptsize{-0.343}} & \mc{1}{c}{\scriptsize{-0.512}} \\  

     &  & \mc{1}{c}{\scriptsize{(0.132)}} & \mc{1}{c}{\scriptsize{(0.316)}} & \mc{1}{c}{\scriptsize{(0.408)}} & \mc{1}{c}{\scriptsize{(0.500)}} & \mc{1}{c}{\scriptsize{(0.395)}} & \mc{1}{c}{\scriptsize{(0.105)}} & \mc{1}{c}{\scriptsize{(0.224)}} & \mc{1}{c}{\scriptsize{(0.118)}} \\  

    \mc{1}{l}{\scriptsize{Total Years Incarcerated}} & \mc{1}{c}{\scriptsize{30}} & \mc{1}{c}{\scriptsize{0.347}} & \mc{1}{c}{\scriptsize{0.489}} & \mc{1}{c}{\scriptsize{0.553}} & \mc{1}{c}{\scriptsize{0.811}} & \mc{1}{c}{\scriptsize{0.701}} & \mc{1}{c}{\scriptsize{0.338}} & \mc{1}{c}{\scriptsize{0.489}} & \mc{1}{c}{\scriptsize{0.470}} \\  

     &  & \mc{1}{c}{\scriptsize{(0.934)}} & \mc{1}{c}{\scriptsize{(0.934)}} & \mc{1}{c}{\scriptsize{(1.000)}} & \mc{1}{c}{\scriptsize{(0.974)}} & \mc{1}{c}{\scriptsize{(1.000)}} & \mc{1}{c}{\scriptsize{(0.882)}} & \mc{1}{c}{\scriptsize{(0.934)}} & \mc{1}{c}{\scriptsize{(0.947)}} \\  

    \mc{1}{l}{\scriptsize{Crime Factor}} & \mc{1}{c}{\scriptsize{30 to Mid-30s}} & \mc{1}{c}{\scriptsize{0.169}} & \mc{1}{c}{\scriptsize{0.326}} & \mc{1}{c}{\scriptsize{0.469}} & \mc{1}{c}{\scriptsize{0.835}} & \mc{1}{c}{\scriptsize{0.545}} & \mc{1}{c}{\scriptsize{0.109}} & \mc{1}{c}{\scriptsize{0.272}} & \mc{1}{c}{\scriptsize{0.182}} \\  

     &  & \mc{1}{c}{\scriptsize{(0.776)}} & \mc{1}{c}{\scriptsize{(0.776)}} & \mc{1}{c}{\scriptsize{(1.000)}} & \mc{1}{c}{\scriptsize{(0.921)}} & \mc{1}{c}{\scriptsize{(1.000)}} & \mc{1}{c}{\scriptsize{(0.658)}} & \mc{1}{c}{\scriptsize{(0.684)}} & \mc{1}{c}{\scriptsize{(0.737)}} \\  

  \bottomrule
  \end{tabular}
	\end{table} 

	\begin{table}[H]
     \caption{Treatment Effects on Achenbach Symptom T Score (Reported by Mother), Male Sample}
     \label{table:abccare_rslt_male_cat9}
	  \begin{tabular}{cccccccccc}
  \toprule

    \scriptsize{Variable} & \scriptsize{Age} & \scriptsize{(1)} & \scriptsize{(2)} & \scriptsize{(3)} & \scriptsize{(4)} & \scriptsize{(5)} & \scriptsize{(6)} & \scriptsize{(7)} & \scriptsize{(8)} \\ 
    \midrule  

    \mc{1}{l}{\scriptsize{Cig. Smoked per day last month}} & \mc{1}{c}{\scriptsize{30}} & \mc{1}{c}{\scriptsize{0.826}} & \mc{1}{c}{\scriptsize{0.245}} & \mc{1}{c}{\scriptsize{0.757}} & \mc{1}{c}{\scriptsize{-2.228}} & \mc{1}{c}{\scriptsize{0.624}} & \mc{1}{c}{\scriptsize{1.429}} & \mc{1}{c}{\scriptsize{1.083}} & \mc{1}{c}{\scriptsize{1.215}} \\  

     &  & \mc{1}{c}{\scriptsize{(0.789)}} & \mc{1}{c}{\scriptsize{(0.539)}} & \mc{1}{c}{\scriptsize{(0.618)}} & \mc{1}{c}{\scriptsize{(0.184)}} & \mc{1}{c}{\scriptsize{(0.579)}} & \mc{1}{c}{\scriptsize{(0.908)}} & \mc{1}{c}{\scriptsize{(0.789)}} & \mc{1}{c}{\scriptsize{(0.789)}} \\  

    \mc{1}{l}{\scriptsize{Days drank alcohol last month}} & \mc{1}{c}{\scriptsize{30}} & \mc{1}{c}{\scriptsize{0.805}} & \mc{1}{c}{\scriptsize{1.083}} & \mc{1}{c}{\scriptsize{-0.186}} & \mc{1}{c}{\scriptsize{-0.648}} & \mc{1}{c}{\scriptsize{0.060}} & \mc{1}{c}{\scriptsize{0.944}} & \mc{1}{c}{\scriptsize{1.348}} & \mc{1}{c}{\scriptsize{1.341}} \\  

     &  & \mc{1}{c}{\scriptsize{(0.711)}} & \mc{1}{c}{\scriptsize{(0.645)}} & \mc{1}{c}{\scriptsize{(0.408)}} & \mc{1}{c}{\scriptsize{(0.355)}} & \mc{1}{c}{\scriptsize{(0.421)}} & \mc{1}{c}{\scriptsize{(0.711)}} & \mc{1}{c}{\scriptsize{(0.724)}} & \mc{1}{c}{\scriptsize{(0.776)}} \\  

    \mc{1}{l}{\scriptsize{Days binge drank alcohol last month}} & \mc{1}{c}{\scriptsize{30}} & \mc{1}{c}{\scriptsize{0.500}} & \mc{1}{c}{\scriptsize{0.474}} & \mc{1}{c}{\scriptsize{0.543}} & \mc{1}{c}{\scriptsize{-0.360}} & \mc{1}{c}{\scriptsize{0.692}} & \mc{1}{c}{\scriptsize{0.491}} & \mc{1}{c}{\scriptsize{0.625}} & \mc{1}{c}{\scriptsize{0.702}} \\  

     &  & \mc{1}{c}{\scriptsize{(0.816)}} & \mc{1}{c}{\scriptsize{(0.816)}} & \mc{1}{c}{\scriptsize{(0.724)}} & \mc{1}{c}{\scriptsize{(0.368)}} & \mc{1}{c}{\scriptsize{(0.829)}} & \mc{1}{c}{\scriptsize{(0.789)}} & \mc{1}{c}{\scriptsize{(0.803)}} & \mc{1}{c}{\scriptsize{(0.868)}} \\  

    \mc{1}{l}{\scriptsize{Self-reported drug user}} & \mc{1}{c}{\scriptsize{Mid-30s}} & \mc{1}{c}{\scriptsize{-0.333}} & \mc{1}{c}{\scriptsize{-0.432}} & \mc{1}{c}{\scriptsize{-0.500}} & \mc{1}{c}{\scriptsize{-0.788}} & \mc{1}{c}{\scriptsize{-0.555}} & \mc{1}{c}{\scriptsize{-0.233}} & \mc{1}{c}{\scriptsize{-0.326}} & \mc{1}{c}{\scriptsize{-0.330}} \\  

     &  & \mc{1}{c}{\scriptsize{\textbf{(0.026)}}} & \mc{1}{c}{\scriptsize{\textbf{(0.000)}}} & \mc{1}{c}{\scriptsize{\textbf{(0.000)}}} & \mc{1}{c}{\scriptsize{\textbf{(0.000)}}} & \mc{1}{c}{\scriptsize{\textbf{(0.000)}}} & \mc{1}{c}{\scriptsize{(0.118)}} & \mc{1}{c}{\scriptsize{\textbf{(0.053)}}} & \mc{1}{c}{\scriptsize{\textbf{(0.039)}}} \\  

    \mc{1}{l}{\scriptsize{Substance Use Factor}} & \mc{1}{c}{\scriptsize{30 to Mid-30s}} & \mc{1}{c}{\scriptsize{0.337}} & \mc{1}{c}{\scriptsize{0.232}} & \mc{1}{c}{\scriptsize{0.180}} & \mc{1}{c}{\scriptsize{0.014}} & \mc{1}{c}{\scriptsize{0.219}} & \mc{1}{c}{\scriptsize{0.403}} & \mc{1}{c}{\scriptsize{0.273}} & \mc{1}{c}{\scriptsize{0.292}} \\  

     &  & \mc{1}{c}{\scriptsize{(0.908)}} & \mc{1}{c}{\scriptsize{(0.737)}} & \mc{1}{c}{\scriptsize{(0.553)}} & \mc{1}{c}{\scriptsize{(0.395)}} & \mc{1}{c}{\scriptsize{(0.566)}} & \mc{1}{c}{\scriptsize{(0.934)}} & \mc{1}{c}{\scriptsize{(0.724)}} & \mc{1}{c}{\scriptsize{(0.855)}} \\ 
    \midrule  

    \mc{2}{l}{\scriptsize{\% of Pos. TE ($H_0$: $\le$ 50\%)}} & \mc{1}{c}{\scriptsize{20}} & \mc{1}{c}{\scriptsize{20}} & \mc{1}{c}{\scriptsize{40}} & \mc{1}{c}{\scriptsize{80}} & \mc{1}{c}{\scriptsize{20}} & \mc{1}{c}{\scriptsize{20}} & \mc{1}{c}{\scriptsize{20}} & \mc{1}{c}{\scriptsize{20}} \\  

     &  & \mc{1}{c}{\scriptsize{(1.000)}} & \mc{1}{c}{\scriptsize{(1.000)}} & \mc{1}{c}{\scriptsize{(0.605)}} & \mc{1}{c}{\scriptsize{(0.237)}} & \mc{1}{c}{\scriptsize{(0.921)}} & \mc{1}{c}{\scriptsize{(1.000)}} & \mc{1}{c}{\scriptsize{(1.000)}} & \mc{1}{c}{\scriptsize{(0.974)}} \\  

    \mc{2}{l}{\scriptsize{\% of Pos. TE ($H_0$: $\le$ 10\% $|$ 10\% Significance)}} & \mc{1}{c}{\scriptsize{20}} & \mc{1}{c}{\scriptsize{20}} & \mc{1}{c}{\scriptsize{20}} & \mc{1}{c}{\scriptsize{20}} & \mc{1}{c}{\scriptsize{20}} & \mc{1}{c}{\scriptsize{20}} & \mc{1}{c}{\scriptsize{20}} & \mc{1}{c}{\scriptsize{20}} \\  

     &  & \mc{1}{c}{\scriptsize{\textbf{(0.039)}}} & \mc{1}{c}{\scriptsize{\textbf{(0.026)}}} & \mc{1}{c}{\scriptsize{\textbf{(0.066)}}} & \mc{1}{c}{\scriptsize{(0.158)}} & \mc{1}{c}{\scriptsize{\textbf{(0.079)}}} & \mc{1}{c}{\scriptsize{(0.395)}} & \mc{1}{c}{\scriptsize{(0.500)}} & \mc{1}{c}{\scriptsize{(0.474)}} \\  

  \bottomrule
  \end{tabular}
	\end{table} 

	\begin{table}[H]
     \caption{Treatment Effects on Achenbach Symptom T Score (Reported by Teacher), Male Sample}
     \label{table:abccare_rslt_male_cat10}
	  \begin{tabular}{cccccccccc}
  \toprule

    \scriptsize{Variable} & \scriptsize{Age} & \scriptsize{(1)} & \scriptsize{(2)} & \scriptsize{(3)} & \scriptsize{(4)} & \scriptsize{(5)} & \scriptsize{(6)} & \scriptsize{(7)} & \scriptsize{(8)} \\ 
    \midrule  

    \mc{1}{l}{\scriptsize{Self-reported Health}} & \mc{1}{c}{\scriptsize{30}} & \mc{1}{c}{\scriptsize{-0.004}} & \mc{1}{c}{\scriptsize{0.226}} & \mc{1}{c}{\scriptsize{0.457}} & \mc{1}{c}{\scriptsize{0.755}} & \mc{1}{c}{\scriptsize{0.568}} & \mc{1}{c}{\scriptsize{-0.145}} & \mc{1}{c}{\scriptsize{0.096}} & \mc{1}{c}{\scriptsize{-0.000}} \\  

     &  & \mc{1}{c}{\scriptsize{(0.539)}} & \mc{1}{c}{\scriptsize{(0.789)}} & \mc{1}{c}{\scriptsize{(0.868)}} & \mc{1}{c}{\scriptsize{(0.934)}} & \mc{1}{c}{\scriptsize{(0.961)}} & \mc{1}{c}{\scriptsize{(0.303)}} & \mc{1}{c}{\scriptsize{(0.632)}} & \mc{1}{c}{\scriptsize{(0.500)}} \\  

     & \mc{1}{c}{\scriptsize{Mid-30s}} & \mc{1}{c}{\scriptsize{0.196}} & \mc{1}{c}{\scriptsize{0.010}} & \mc{1}{c}{\scriptsize{-0.852}} & \mc{1}{c}{\scriptsize{-1.130}} & \mc{1}{c}{\scriptsize{-0.925}} & \mc{1}{c}{\scriptsize{0.382}} & \mc{1}{c}{\scriptsize{0.207}} & \mc{1}{c}{\scriptsize{0.170}} \\  

     &  & \mc{1}{c}{\scriptsize{(0.684)}} & \mc{1}{c}{\scriptsize{(0.539)}} & \mc{1}{c}{\scriptsize{\textbf{(0.000)}}} & \mc{1}{c}{\scriptsize{\textbf{(0.026)}}} & \mc{1}{c}{\scriptsize{\textbf{(0.000)}}} & \mc{1}{c}{\scriptsize{(0.934)}} & \mc{1}{c}{\scriptsize{(0.697)}} & \mc{1}{c}{\scriptsize{(0.711)}} \\  

    \mc{1}{l}{\scriptsize{Self-reported Health Factor}} & \mc{1}{c}{\scriptsize{30 to Mid-30s}} & \mc{1}{c}{\scriptsize{-0.035}} & \mc{1}{c}{\scriptsize{-0.136}} & \mc{1}{c}{\scriptsize{-0.563}} & \mc{1}{c}{\scriptsize{-0.758}} & \mc{1}{c}{\scriptsize{-0.596}} & \mc{1}{c}{\scriptsize{0.065}} & \mc{1}{c}{\scriptsize{-0.035}} & \mc{1}{c}{\scriptsize{-0.044}} \\  

     &  & \mc{1}{c}{\scriptsize{(0.368)}} & \mc{1}{c}{\scriptsize{(0.224)}} & \mc{1}{c}{\scriptsize{\textbf{(0.039)}}} & \mc{1}{c}{\scriptsize{\textbf{(0.026)}}} & \mc{1}{c}{\scriptsize{\textbf{(0.039)}}} & \mc{1}{c}{\scriptsize{(0.684)}} & \mc{1}{c}{\scriptsize{(0.395)}} & \mc{1}{c}{\scriptsize{(0.355)}} \\ 
    \midrule  

    \mc{2}{l}{\scriptsize{\% of Pos. TE ($H_0$: $\le$ 50\%)}} & \mc{1}{c}{\scriptsize{67}} & \mc{1}{c}{\scriptsize{33}} & \mc{1}{c}{\scriptsize{67}} & \mc{1}{c}{\scriptsize{67}} & \mc{1}{c}{\scriptsize{67}} & \mc{1}{c}{\scriptsize{33}} & \mc{1}{c}{\scriptsize{33}} & \mc{1}{c}{\scriptsize{67}} \\  

     &  & \mc{1}{c}{\scriptsize{(0.434)}} & \mc{1}{c}{\scriptsize{(0.724)}} & \mc{1}{c}{\scriptsize{\textbf{(0.053)}}} & \mc{1}{c}{\scriptsize{\textbf{(0.039)}}} & \mc{1}{c}{\scriptsize{\textbf{(0.053)}}} & \mc{1}{c}{\scriptsize{(0.895)}} & \mc{1}{c}{\scriptsize{(0.763)}} & \mc{1}{c}{\scriptsize{(0.355)}} \\  

    \mc{2}{l}{\scriptsize{\% of Pos. TE ($H_0$: $\le$ 10\% $|$ 10\% Significance)}} & \mc{1}{c}{\scriptsize{0}} & \mc{1}{c}{\scriptsize{0}} & \mc{1}{c}{\scriptsize{67}} & \mc{1}{c}{\scriptsize{67}} & \mc{1}{c}{\scriptsize{67}} & \mc{1}{c}{\scriptsize{0}} & \mc{1}{c}{\scriptsize{0}} & \mc{1}{c}{\scriptsize{0}} \\  

     &  & \mc{1}{c}{\scriptsize{(1.000)}} & \mc{1}{c}{\scriptsize{(1.000)}} & \mc{1}{c}{\scriptsize{\textbf{(0.000)}}} & \mc{1}{c}{\scriptsize{\textbf{(0.000)}}} & \mc{1}{c}{\scriptsize{\textbf{(0.000)}}} & \mc{1}{c}{\scriptsize{(1.000)}} & \mc{1}{c}{\scriptsize{(1.000)}} & \mc{1}{c}{\scriptsize{(1.000)}} \\  

  \bottomrule
  \end{tabular}
	\end{table} 

	\begin{table}[H]
     \caption{Treatment Effects on Child Assessment Schedule (CAS), Male Sample}
     \label{table:abccare_rslt_male_cat11}
	  \begin{tabular}{cccccccccc}
  \toprule

    \scriptsize{Variable} & \scriptsize{Age} & \scriptsize{(1)} & \scriptsize{(2)} & \scriptsize{(3)} & \scriptsize{(4)} & \scriptsize{(5)} & \scriptsize{(6)} & \scriptsize{(7)} & \scriptsize{(8)} \\ 
    \midrule  

    \mc{1}{l}{\scriptsize{Denies Any Worries}} & \mc{1}{c}{\scriptsize{12}} & \mc{1}{c}{\scriptsize{-0.170}} & \mc{1}{c}{\scriptsize{-0.234}} & \mc{1}{c}{\scriptsize{-0.084}} & \mc{1}{c}{\scriptsize{-0.021}} & \mc{1}{c}{\scriptsize{-0.075}} & \mc{1}{c}{\scriptsize{-0.191}} & \mc{1}{c}{\scriptsize{-0.273}} & \mc{1}{c}{\scriptsize{-0.201}} \\  

     &  & \mc{1}{c}{\scriptsize{\textbf{(0.039)}}} & \mc{1}{c}{\scriptsize{\textbf{(0.039)}}} & \mc{1}{c}{\scriptsize{(0.224)}} & \mc{1}{c}{\scriptsize{(0.355)}} & \mc{1}{c}{\scriptsize{(0.224)}} & \mc{1}{c}{\scriptsize{\textbf{(0.026)}}} & \mc{1}{c}{\scriptsize{\textbf{(0.026)}}} & \mc{1}{c}{\scriptsize{\textbf{(0.039)}}} \\  

    \mc{1}{l}{\scriptsize{Family Proud of You}} & \mc{1}{c}{\scriptsize{12}} & \mc{1}{c}{\scriptsize{-0.110}} & \mc{1}{c}{\scriptsize{-0.103}} & \mc{1}{c}{\scriptsize{0.004}} & \mc{1}{c}{\scriptsize{-0.006}} & \mc{1}{c}{\scriptsize{-0.040}} & \mc{1}{c}{\scriptsize{-0.139}} & \mc{1}{c}{\scriptsize{-0.155}} & \mc{1}{c}{\scriptsize{-0.204}} \\  

     &  & \mc{1}{c}{\scriptsize{(0.895)}} & \mc{1}{c}{\scriptsize{(0.921)}} & \mc{1}{c}{\scriptsize{(0.553)}} & \mc{1}{c}{\scriptsize{(0.592)}} & \mc{1}{c}{\scriptsize{(0.645)}} & \mc{1}{c}{\scriptsize{(0.908)}} & \mc{1}{c}{\scriptsize{(0.921)}} & \mc{1}{c}{\scriptsize{(0.974)}} \\  

    \mc{1}{l}{\scriptsize{Feels Inadequate, Inferior}} & \mc{1}{c}{\scriptsize{12}} & \mc{1}{c}{\scriptsize{0.153}} & \mc{1}{c}{\scriptsize{0.216}} & \mc{1}{c}{\scriptsize{0.353}} & \mc{1}{c}{\scriptsize{0.608}} & \mc{1}{c}{\scriptsize{0.380}} & \mc{1}{c}{\scriptsize{0.103}} & \mc{1}{c}{\scriptsize{0.131}} & \mc{1}{c}{\scriptsize{0.149}} \\  

     &  & \mc{1}{c}{\scriptsize{(0.829)}} & \mc{1}{c}{\scriptsize{(0.855)}} & \mc{1}{c}{\scriptsize{(0.987)}} & \mc{1}{c}{\scriptsize{(0.921)}} & \mc{1}{c}{\scriptsize{(0.987)}} & \mc{1}{c}{\scriptsize{(0.711)}} & \mc{1}{c}{\scriptsize{(0.697)}} & \mc{1}{c}{\scriptsize{(0.776)}} \\  

    \mc{1}{l}{\scriptsize{Good Description of Self}} & \mc{1}{c}{\scriptsize{12}} & \mc{1}{c}{\scriptsize{0.010}} & \mc{1}{c}{\scriptsize{-0.086}} & \mc{1}{c}{\scriptsize{-0.076}} & \mc{1}{c}{\scriptsize{-0.323}} & \mc{1}{c}{\scriptsize{-0.086}} & \mc{1}{c}{\scriptsize{0.032}} & \mc{1}{c}{\scriptsize{-0.029}} & \mc{1}{c}{\scriptsize{-0.002}} \\  

     &  & \mc{1}{c}{\scriptsize{(0.447)}} & \mc{1}{c}{\scriptsize{(0.658)}} & \mc{1}{c}{\scriptsize{(0.618)}} & \mc{1}{c}{\scriptsize{(0.908)}} & \mc{1}{c}{\scriptsize{(0.605)}} & \mc{1}{c}{\scriptsize{(0.434)}} & \mc{1}{c}{\scriptsize{(0.539)}} & \mc{1}{c}{\scriptsize{(0.461)}} \\  

    \mc{1}{l}{\scriptsize{Ignores Situation}} & \mc{1}{c}{\scriptsize{12}} & \mc{1}{c}{\scriptsize{-0.154}} & \mc{1}{c}{\scriptsize{-0.306}} & \mc{1}{c}{\scriptsize{-0.382}} & \mc{1}{c}{\scriptsize{-0.573}} & \mc{1}{c}{\scriptsize{-0.420}} & \mc{1}{c}{\scriptsize{-0.097}} & \mc{1}{c}{\scriptsize{-0.227}} & \mc{1}{c}{\scriptsize{-0.146}} \\  

     &  & \mc{1}{c}{\scriptsize{(0.145)}} & \mc{1}{c}{\scriptsize{\textbf{(0.026)}}} & \mc{1}{c}{\scriptsize{(0.105)}} & \mc{1}{c}{\scriptsize{\textbf{(0.092)}}} & \mc{1}{c}{\scriptsize{\textbf{(0.079)}}} & \mc{1}{c}{\scriptsize{(0.250)}} & \mc{1}{c}{\scriptsize{\textbf{(0.092)}}} & \mc{1}{c}{\scriptsize{(0.171)}} \\  

    \mc{1}{l}{\scriptsize{Impulsivity}} & \mc{1}{c}{\scriptsize{12}} & \mc{1}{c}{\scriptsize{-0.034}} & \mc{1}{c}{\scriptsize{0.007}} & \mc{1}{c}{\scriptsize{0.109}} & \mc{1}{c}{\scriptsize{-0.117}} & \mc{1}{c}{\scriptsize{0.109}} & \mc{1}{c}{\scriptsize{-0.069}} & \mc{1}{c}{\scriptsize{0.011}} & \mc{1}{c}{\scriptsize{-0.058}} \\  

     &  & \mc{1}{c}{\scriptsize{(0.434)}} & \mc{1}{c}{\scriptsize{(0.566)}} & \mc{1}{c}{\scriptsize{(0.724)}} & \mc{1}{c}{\scriptsize{(0.237)}} & \mc{1}{c}{\scriptsize{(0.737)}} & \mc{1}{c}{\scriptsize{(0.224)}} & \mc{1}{c}{\scriptsize{(0.526)}} & \mc{1}{c}{\scriptsize{(0.342)}} \\  

    \mc{1}{l}{\scriptsize{Not Cope with Problem}} & \mc{1}{c}{\scriptsize{12}} & \mc{1}{c}{\scriptsize{-0.014}} & \mc{1}{c}{\scriptsize{-0.088}} & \mc{1}{c}{\scriptsize{0.071}} & \mc{1}{c}{\scriptsize{0.203}} & \mc{1}{c}{\scriptsize{0.122}} & \mc{1}{c}{\scriptsize{-0.036}} & \mc{1}{c}{\scriptsize{-0.093}} & \mc{1}{c}{\scriptsize{-0.027}} \\  

     &  & \mc{1}{c}{\scriptsize{(0.474)}} & \mc{1}{c}{\scriptsize{(0.395)}} & \mc{1}{c}{\scriptsize{(0.645)}} & \mc{1}{c}{\scriptsize{(0.776)}} & \mc{1}{c}{\scriptsize{(0.750)}} & \mc{1}{c}{\scriptsize{(0.434)}} & \mc{1}{c}{\scriptsize{(0.395)}} & \mc{1}{c}{\scriptsize{(0.447)}} \\  

    \mc{1}{l}{\scriptsize{Often Mad or Angry}} & \mc{1}{c}{\scriptsize{12}} & \mc{1}{c}{\scriptsize{-0.105}} & \mc{1}{c}{\scriptsize{-0.104}} & \mc{1}{c}{\scriptsize{0.077}} & \mc{1}{c}{\scriptsize{0.059}} & \mc{1}{c}{\scriptsize{0.207}} & \mc{1}{c}{\scriptsize{-0.158}} & \mc{1}{c}{\scriptsize{-0.131}} & \mc{1}{c}{\scriptsize{-0.003}} \\  

     &  & \mc{1}{c}{\scriptsize{(0.237)}} & \mc{1}{c}{\scriptsize{(0.263)}} & \mc{1}{c}{\scriptsize{(0.855)}} & \mc{1}{c}{\scriptsize{(0.724)}} & \mc{1}{c}{\scriptsize{(0.895)}} & \mc{1}{c}{\scriptsize{(0.197)}} & \mc{1}{c}{\scriptsize{(0.263)}} & \mc{1}{c}{\scriptsize{(0.513)}} \\  

    \mc{1}{l}{\scriptsize{Participates in Activity}} & \mc{1}{c}{\scriptsize{12}} & \mc{1}{c}{\scriptsize{0.194}} & \mc{1}{c}{\scriptsize{0.093}} & \mc{1}{c}{\scriptsize{0.080}} & \mc{1}{c}{\scriptsize{0.004}} & \mc{1}{c}{\scriptsize{0.041}} & \mc{1}{c}{\scriptsize{0.223}} & \mc{1}{c}{\scriptsize{0.141}} & \mc{1}{c}{\scriptsize{0.143}} \\  

     &  & \mc{1}{c}{\scriptsize{\textbf{(0.039)}}} & \mc{1}{c}{\scriptsize{(0.250)}} & \mc{1}{c}{\scriptsize{(0.237)}} & \mc{1}{c}{\scriptsize{(0.434)}} & \mc{1}{c}{\scriptsize{(0.303)}} & \mc{1}{c}{\scriptsize{\textbf{(0.053)}}} & \mc{1}{c}{\scriptsize{(0.184)}} & \mc{1}{c}{\scriptsize{\textbf{(0.079)}}} \\  

    \mc{1}{l}{\scriptsize{Proud about Self}} & \mc{1}{c}{\scriptsize{12}} & \mc{1}{c}{\scriptsize{-0.109}} & \mc{1}{c}{\scriptsize{-0.157}} & \mc{1}{c}{\scriptsize{0.034}} & \mc{1}{c}{\scriptsize{-0.017}} & \mc{1}{c}{\scriptsize{0.029}} & \mc{1}{c}{\scriptsize{-0.145}} & \mc{1}{c}{\scriptsize{-0.203}} & \mc{1}{c}{\scriptsize{-0.173}} \\  

     &  & \mc{1}{c}{\scriptsize{(0.803)}} & \mc{1}{c}{\scriptsize{(0.855)}} & \mc{1}{c}{\scriptsize{(0.487)}} & \mc{1}{c}{\scriptsize{(0.592)}} & \mc{1}{c}{\scriptsize{(0.421)}} & \mc{1}{c}{\scriptsize{(0.882)}} & \mc{1}{c}{\scriptsize{(0.908)}} & \mc{1}{c}{\scriptsize{(0.882)}} \\  

    \mc{1}{l}{\scriptsize{Significant Fears}} & \mc{1}{c}{\scriptsize{12}} & \mc{1}{c}{\scriptsize{-0.038}} & \mc{1}{c}{\scriptsize{-0.004}} & \mc{1}{c}{\scriptsize{0.105}} & \mc{1}{c}{\scriptsize{0.017}} & \mc{1}{c}{\scriptsize{0.116}} & \mc{1}{c}{\scriptsize{-0.073}} & \mc{1}{c}{\scriptsize{0.005}} & \mc{1}{c}{\scriptsize{-0.040}} \\  

     &  & \mc{1}{c}{\scriptsize{(0.342)}} & \mc{1}{c}{\scriptsize{(0.513)}} & \mc{1}{c}{\scriptsize{(0.632)}} & \mc{1}{c}{\scriptsize{(0.579)}} & \mc{1}{c}{\scriptsize{(0.711)}} & \mc{1}{c}{\scriptsize{(0.289)}} & \mc{1}{c}{\scriptsize{(0.487)}} & \mc{1}{c}{\scriptsize{(0.355)}} \\  

    \mc{1}{l}{\scriptsize{Time spent reading}} & \mc{1}{c}{\scriptsize{12}} & \mc{1}{c}{\scriptsize{-0.076}} & \mc{1}{c}{\scriptsize{-0.288}} & \mc{1}{c}{\scriptsize{-2.367}} & \mc{1}{c}{\scriptsize{-3.591}} & \mc{1}{c}{\scriptsize{-2.224}} & \mc{1}{c}{\scriptsize{0.414}} & \mc{1}{c}{\scriptsize{0.156}} & \mc{1}{c}{\scriptsize{0.503}} \\  

     &  & \mc{1}{c}{\scriptsize{(0.500)}} & \mc{1}{c}{\scriptsize{(0.566)}} & \mc{1}{c}{\scriptsize{(0.737)}} & \mc{1}{c}{\scriptsize{(0.776)}} & \mc{1}{c}{\scriptsize{(0.724)}} & \mc{1}{c}{\scriptsize{(0.342)}} & \mc{1}{c}{\scriptsize{(0.447)}} & \mc{1}{c}{\scriptsize{(0.329)}} \\  

    \mc{1}{l}{\scriptsize{Views Self as Clumsy}} & \mc{1}{c}{\scriptsize{12}} & \mc{1}{c}{\scriptsize{-0.086}} & \mc{1}{c}{\scriptsize{-0.056}} &  &  &  & \mc{1}{c}{\scriptsize{-0.107}} & \mc{1}{c}{\scriptsize{-0.081}} & \mc{1}{c}{\scriptsize{-0.115}} \\  

     &  & \mc{1}{c}{\scriptsize{\textbf{(0.000)}}} & \mc{1}{c}{\scriptsize{\textbf{(0.053)}}} &  &  &  & \mc{1}{c}{\scriptsize{\textbf{(0.013)}}} & \mc{1}{c}{\scriptsize{\textbf{(0.026)}}} & \mc{1}{c}{\scriptsize{\textbf{(0.013)}}} \\  

    \mc{1}{l}{\scriptsize{Views Self as Dumb}} & \mc{1}{c}{\scriptsize{12}} & \mc{1}{c}{\scriptsize{-0.018}} & \mc{1}{c}{\scriptsize{-0.012}} & \mc{1}{c}{\scriptsize{-0.046}} & \mc{1}{c}{\scriptsize{-0.060}} & \mc{1}{c}{\scriptsize{-0.017}} & \mc{1}{c}{\scriptsize{-0.011}} & \mc{1}{c}{\scriptsize{-0.000}} & \mc{1}{c}{\scriptsize{0.027}} \\  

     &  & \mc{1}{c}{\scriptsize{(0.487)}} & \mc{1}{c}{\scriptsize{(0.526)}} & \mc{1}{c}{\scriptsize{(0.434)}} & \mc{1}{c}{\scriptsize{(0.368)}} & \mc{1}{c}{\scriptsize{(0.434)}} & \mc{1}{c}{\scriptsize{(0.487)}} & \mc{1}{c}{\scriptsize{(0.461)}} & \mc{1}{c}{\scriptsize{(0.539)}} \\  

    \mc{1}{l}{\scriptsize{Views Self as Not Liked}} & \mc{1}{c}{\scriptsize{12}} & \mc{1}{c}{\scriptsize{-0.024}} & \mc{1}{c}{\scriptsize{-0.033}} & \mc{1}{c}{\scriptsize{0.004}} & \mc{1}{c}{\scriptsize{-0.015}} & \mc{1}{c}{\scriptsize{-0.005}} & \mc{1}{c}{\scriptsize{-0.032}} & \mc{1}{c}{\scriptsize{-0.042}} & \mc{1}{c}{\scriptsize{-0.055}} \\  

     &  & \mc{1}{c}{\scriptsize{(0.447)}} & \mc{1}{c}{\scriptsize{(0.408)}} & \mc{1}{c}{\scriptsize{(0.461)}} & \mc{1}{c}{\scriptsize{(0.434)}} & \mc{1}{c}{\scriptsize{(0.474)}} & \mc{1}{c}{\scriptsize{(0.421)}} & \mc{1}{c}{\scriptsize{(0.408)}} & \mc{1}{c}{\scriptsize{(0.382)}} \\  

    \mc{1}{l}{\scriptsize{Withdraws Excessively}} & \mc{1}{c}{\scriptsize{12}} & \mc{1}{c}{\scriptsize{0.128}} & \mc{1}{c}{\scriptsize{0.020}} & \mc{1}{c}{\scriptsize{0.185}} & \mc{1}{c}{\scriptsize{0.143}} & \mc{1}{c}{\scriptsize{0.197}} & \mc{1}{c}{\scriptsize{0.113}} & \mc{1}{c}{\scriptsize{0.003}} & \mc{1}{c}{\scriptsize{0.099}} \\  

     &  & \mc{1}{c}{\scriptsize{(0.829)}} & \mc{1}{c}{\scriptsize{(0.553)}} & \mc{1}{c}{\scriptsize{(0.763)}} & \mc{1}{c}{\scriptsize{(0.711)}} & \mc{1}{c}{\scriptsize{(0.776)}} & \mc{1}{c}{\scriptsize{(0.816)}} & \mc{1}{c}{\scriptsize{(0.487)}} & \mc{1}{c}{\scriptsize{(0.763)}} \\  

  \bottomrule
  \end{tabular}
	\end{table} 

	\begin{table}[H]
     \caption{Treatment Effects on Mother's Income, Male Sample}
     \label{table:abccare_rslt_male_cat12}
	  \begin{tabular}{cccccccccc}
  \toprule

    \scriptsize{Variable} & \scriptsize{Age} & \scriptsize{(1)} & \scriptsize{(2)} & \scriptsize{(3)} & \scriptsize{(4)} & \scriptsize{(5)} & \scriptsize{(6)} & \scriptsize{(7)} & \scriptsize{(8)} \\ 
    \midrule  

    \mc{1}{l}{\scriptsize{High-Density Lipoprotein Chol. (mg/dL)}} & \mc{1}{c}{\scriptsize{Mid-30s}} & \mc{1}{c}{\scriptsize{7.753}} & \mc{1}{c}{\scriptsize{4.139}} & \mc{1}{c}{\scriptsize{-0.267}} & \mc{1}{c}{\scriptsize{-8.079}} & \mc{1}{c}{\scriptsize{-4.138}} & \mc{1}{c}{\scriptsize{9.015}} & \mc{1}{c}{\scriptsize{5.332}} & \mc{1}{c}{\scriptsize{6.830}} \\  

     &  & \mc{1}{c}{\scriptsize{\textbf{(0.013)}}} & \mc{1}{c}{\scriptsize{(0.197)}} & \mc{1}{c}{\scriptsize{(0.461)}} & \mc{1}{c}{\scriptsize{(0.763)}} & \mc{1}{c}{\scriptsize{(0.645)}} & \mc{1}{c}{\scriptsize{\textbf{(0.013)}}} & \mc{1}{c}{\scriptsize{(0.118)}} & \mc{1}{c}{\scriptsize{\textbf{(0.066)}}} \\  

    \mc{1}{l}{\scriptsize{Dyslipidemia}} & \mc{1}{c}{\scriptsize{Mid-30s}} & \mc{1}{c}{\scriptsize{-0.094}} & \mc{1}{c}{\scriptsize{-0.029}} & \mc{1}{c}{\scriptsize{0.200}} & \mc{1}{c}{\scriptsize{0.365}} & \mc{1}{c}{\scriptsize{0.207}} & \mc{1}{c}{\scriptsize{-0.108}} & \mc{1}{c}{\scriptsize{-0.056}} & \mc{1}{c}{\scriptsize{-0.130}} \\  

     &  & \mc{1}{c}{\scriptsize{(0.289)}} & \mc{1}{c}{\scriptsize{(0.421)}} & \mc{1}{c}{\scriptsize{(0.868)}} & \mc{1}{c}{\scriptsize{(0.789)}} & \mc{1}{c}{\scriptsize{(0.855)}} & \mc{1}{c}{\scriptsize{(0.289)}} & \mc{1}{c}{\scriptsize{(0.342)}} & \mc{1}{c}{\scriptsize{(0.237)}} \\  

    \mc{1}{l}{\scriptsize{Cholesterol Factor}} & \mc{1}{c}{\scriptsize{Mid-30s}} & \mc{1}{c}{\scriptsize{-0.333}} & \mc{1}{c}{\scriptsize{-0.153}} & \mc{1}{c}{\scriptsize{0.240}} & \mc{1}{c}{\scriptsize{0.657}} & \mc{1}{c}{\scriptsize{0.359}} & \mc{1}{c}{\scriptsize{-0.385}} & \mc{1}{c}{\scriptsize{-0.219}} & \mc{1}{c}{\scriptsize{-0.348}} \\  

     &  & \mc{1}{c}{\scriptsize{\textbf{(0.092)}}} & \mc{1}{c}{\scriptsize{(0.316)}} & \mc{1}{c}{\scriptsize{(0.750)}} & \mc{1}{c}{\scriptsize{(0.882)}} & \mc{1}{c}{\scriptsize{(0.816)}} & \mc{1}{c}{\scriptsize{(0.105)}} & \mc{1}{c}{\scriptsize{(0.303)}} & \mc{1}{c}{\scriptsize{(0.145)}} \\ 
    \midrule  

    \mc{2}{l}{\scriptsize{\% of Pos. TE ($H_0$: $\le$ 50\%)}} & \mc{1}{c}{\scriptsize{100}} & \mc{1}{c}{\scriptsize{100}} & \mc{1}{c}{\scriptsize{0}} & \mc{1}{c}{\scriptsize{0}} & \mc{1}{c}{\scriptsize{0}} & \mc{1}{c}{\scriptsize{100}} & \mc{1}{c}{\scriptsize{100}} & \mc{1}{c}{\scriptsize{100}} \\  

     &  & \mc{1}{c}{\scriptsize{\textbf{(0.000)}}} & \mc{1}{c}{\scriptsize{\textbf{(0.000)}}} & \mc{1}{c}{\scriptsize{(0.908)}} & \mc{1}{c}{\scriptsize{(0.908)}} & \mc{1}{c}{\scriptsize{(0.908)}} & \mc{1}{c}{\scriptsize{\textbf{(0.000)}}} & \mc{1}{c}{\scriptsize{\textbf{(0.000)}}} & \mc{1}{c}{\scriptsize{\textbf{(0.000)}}} \\  

    \mc{2}{l}{\scriptsize{\% of Pos. TE ($H_0$: $\le$ 10\% $|$ 10\% Significance)}} & \mc{1}{c}{\scriptsize{67}} & \mc{1}{c}{\scriptsize{0}} & \mc{1}{c}{\scriptsize{0}} & \mc{1}{c}{\scriptsize{0}} & \mc{1}{c}{\scriptsize{0}} & \mc{1}{c}{\scriptsize{67}} & \mc{1}{c}{\scriptsize{33}} & \mc{1}{c}{\scriptsize{67}} \\  

     &  & \mc{1}{c}{\scriptsize{\textbf{(0.000)}}} & \mc{1}{c}{\scriptsize{(0.289)}} & \mc{1}{c}{\scriptsize{(0.908)}} & \mc{1}{c}{\scriptsize{(0.908)}} & \mc{1}{c}{\scriptsize{(0.908)}} & \mc{1}{c}{\scriptsize{\textbf{(0.000)}}} & \mc{1}{c}{\scriptsize{(0.145)}} & \mc{1}{c}{\scriptsize{\textbf{(0.000)}}} \\  

  \bottomrule
  \end{tabular}
	\end{table} 

	\begin{table}[H]
     \caption{Treatment Effects on Parental Labor Income, Male Sample}
     \label{table:abccare_rslt_male_cat13}
	  \begin{tabular}{cccccccccc}
  \toprule

    \scriptsize{Variable} & \scriptsize{Age} & \scriptsize{(1)} & \scriptsize{(2)} & \scriptsize{(3)} & \scriptsize{(4)} & \scriptsize{(5)} & \scriptsize{(6)} & \scriptsize{(7)} & \scriptsize{(8)} \\ 
    \midrule  

    \mc{1}{l}{\scriptsize{Hemoglobin Level (\%)}} & \mc{1}{c}{\scriptsize{Mid-30s}} & \mc{1}{c}{\scriptsize{0.322}} & \mc{1}{c}{\scriptsize{0.435}} & \mc{1}{c}{\scriptsize{0.240}} & \mc{1}{c}{\scriptsize{0.469}} & \mc{1}{c}{\scriptsize{0.355}} & \mc{1}{c}{\scriptsize{0.286}} & \mc{1}{c}{\scriptsize{0.386}} & \mc{1}{c}{\scriptsize{0.396}} \\  

     &  & \mc{1}{c}{\scriptsize{(0.829)}} & \mc{1}{c}{\scriptsize{(0.737)}} & \mc{1}{c}{\scriptsize{(0.658)}} & \mc{1}{c}{\scriptsize{(0.618)}} & \mc{1}{c}{\scriptsize{(0.724)}} & \mc{1}{c}{\scriptsize{(0.803)}} & \mc{1}{c}{\scriptsize{(0.763)}} & \mc{1}{c}{\scriptsize{(0.829)}} \\  

    \mc{1}{l}{\scriptsize{Prediabetes}} & \mc{1}{c}{\scriptsize{Mid-30s}} & \mc{1}{c}{\scriptsize{-0.129}} & \mc{1}{c}{\scriptsize{-0.190}} & \mc{1}{c}{\scriptsize{-0.267}} & \mc{1}{c}{\scriptsize{-0.478}} & \mc{1}{c}{\scriptsize{-0.338}} & \mc{1}{c}{\scriptsize{-0.139}} & \mc{1}{c}{\scriptsize{-0.182}} & \mc{1}{c}{\scriptsize{-0.173}} \\  

     &  & \mc{1}{c}{\scriptsize{(0.250)}} & \mc{1}{c}{\scriptsize{(0.184)}} & \mc{1}{c}{\scriptsize{(0.197)}} & \mc{1}{c}{\scriptsize{(0.118)}} & \mc{1}{c}{\scriptsize{(0.132)}} & \mc{1}{c}{\scriptsize{(0.237)}} & \mc{1}{c}{\scriptsize{(0.224)}} & \mc{1}{c}{\scriptsize{(0.224)}} \\  

    \mc{1}{l}{\scriptsize{Diabetes}} & \mc{1}{c}{\scriptsize{Mid-30s}} & \mc{1}{c}{\scriptsize{0.080}} & \mc{1}{c}{\scriptsize{0.081}} & \mc{1}{c}{\scriptsize{0.080}} & \mc{1}{c}{\scriptsize{0.027}} & \mc{1}{c}{\scriptsize{0.098}} & \mc{1}{c}{\scriptsize{0.080}} & \mc{1}{c}{\scriptsize{0.083}} & \mc{1}{c}{\scriptsize{0.098}} \\  

     &  & \mc{1}{c}{\scriptsize{(0.842)}} & \mc{1}{c}{\scriptsize{(0.776)}} & \mc{1}{c}{\scriptsize{(0.750)}} & \mc{1}{c}{\scriptsize{(0.461)}} & \mc{1}{c}{\scriptsize{(0.789)}} & \mc{1}{c}{\scriptsize{(0.842)}} & \mc{1}{c}{\scriptsize{(0.803)}} & \mc{1}{c}{\scriptsize{(0.882)}} \\  

    \mc{1}{l}{\scriptsize{Diabetes Factor}} & \mc{1}{c}{\scriptsize{Mid-30s}} & \mc{1}{c}{\scriptsize{0.283}} & \mc{1}{c}{\scriptsize{0.338}} & \mc{1}{c}{\scriptsize{0.225}} & \mc{1}{c}{\scriptsize{0.244}} & \mc{1}{c}{\scriptsize{0.307}} & \mc{1}{c}{\scriptsize{0.262}} & \mc{1}{c}{\scriptsize{0.317}} & \mc{1}{c}{\scriptsize{0.345}} \\  

     &  & \mc{1}{c}{\scriptsize{(0.908)}} & \mc{1}{c}{\scriptsize{(0.763)}} & \mc{1}{c}{\scriptsize{(0.711)}} & \mc{1}{c}{\scriptsize{(0.579)}} & \mc{1}{c}{\scriptsize{(0.737)}} & \mc{1}{c}{\scriptsize{(0.882)}} & \mc{1}{c}{\scriptsize{(0.789)}} & \mc{1}{c}{\scriptsize{(0.882)}} \\ 
    \midrule  

    \mc{2}{l}{\scriptsize{\% of Pos. TE ($H_0$: $\le$ 50\%)}} & \mc{1}{c}{\scriptsize{25}} & \mc{1}{c}{\scriptsize{25}} & \mc{1}{c}{\scriptsize{25}} & \mc{1}{c}{\scriptsize{25}} & \mc{1}{c}{\scriptsize{25}} & \mc{1}{c}{\scriptsize{25}} & \mc{1}{c}{\scriptsize{25}} & \mc{1}{c}{\scriptsize{25}} \\  

     &  & \mc{1}{c}{\scriptsize{(0.750)}} & \mc{1}{c}{\scriptsize{(0.763)}} & \mc{1}{c}{\scriptsize{(0.711)}} & \mc{1}{c}{\scriptsize{(0.474)}} & \mc{1}{c}{\scriptsize{(0.789)}} & \mc{1}{c}{\scriptsize{(0.763)}} & \mc{1}{c}{\scriptsize{(0.763)}} & \mc{1}{c}{\scriptsize{(0.829)}} \\  

    \mc{2}{l}{\scriptsize{\% of Pos. TE ($H_0$: $\le$ 10\% $|$ 10\% Significance)}} & \mc{1}{c}{\scriptsize{0}} & \mc{1}{c}{\scriptsize{0}} & \mc{1}{c}{\scriptsize{0}} & \mc{1}{c}{\scriptsize{0}} & \mc{1}{c}{\scriptsize{0}} & \mc{1}{c}{\scriptsize{0}} & \mc{1}{c}{\scriptsize{0}} & \mc{1}{c}{\scriptsize{0}} \\  

     &  & \mc{1}{c}{\scriptsize{(1.000)}} & \mc{1}{c}{\scriptsize{(1.000)}} & \mc{1}{c}{\scriptsize{(0.368)}} & \mc{1}{c}{\scriptsize{(0.908)}} & \mc{1}{c}{\scriptsize{(0.395)}} & \mc{1}{c}{\scriptsize{(1.000)}} & \mc{1}{c}{\scriptsize{(1.000)}} & \mc{1}{c}{\scriptsize{(1.000)}} \\  

  \bottomrule
  \end{tabular}
	\end{table} 

	\begin{table}[H]
     \caption{Treatment Effects on Parental Public Transfer Income, Male Sample}
     \label{table:abccare_rslt_male_cat14}
	  \begin{tabular}{cccccccccc}
  \toprule

    \scriptsize{Variable} & \scriptsize{Age} & \scriptsize{(1)} & \scriptsize{(2)} & \scriptsize{(3)} & \scriptsize{(4)} & \scriptsize{(5)} & \scriptsize{(6)} & \scriptsize{(7)} & \scriptsize{(8)} \\ 
    \midrule  

    \mc{1}{l}{\scriptsize{Parental Public-Transfer Income}} & \mc{1}{c}{\scriptsize{12}} & \mc{1}{c}{\scriptsize{233}} & \mc{1}{c}{\scriptsize{1,082}} & \mc{1}{c}{\scriptsize{1,074}} & \mc{1}{c}{\scriptsize{390}} & \mc{1}{c}{\scriptsize{2,582}} & \mc{1}{c}{\scriptsize{-7.224}} & \mc{1}{c}{\scriptsize{1,625}} & \mc{1}{c}{\scriptsize{1,870}} \\  

     &  & \mc{1}{c}{\scriptsize{(0.447)}} & \mc{1}{c}{\scriptsize{(0.368)}} & \mc{1}{c}{\scriptsize{(0.408)}} & \mc{1}{c}{\scriptsize{(0.368)}} & \mc{1}{c}{\scriptsize{(0.276)}} & \mc{1}{c}{\scriptsize{(0.474)}} & \mc{1}{c}{\scriptsize{(0.303)}} & \mc{1}{c}{\scriptsize{(0.289)}} \\  

  \bottomrule
  \end{tabular}
	\end{table} 

	\begin{table}[H]
     \caption{Treatment Effects on Adoption, Male Sample}
     \label{table:abccare_rslt_male_cat15}
	  \begin{tabular}{cccccccccc}
  \toprule

    \scriptsize{Variable} & \scriptsize{Age} & \scriptsize{(1)} & \scriptsize{(2)} & \scriptsize{(3)} & \scriptsize{(4)} & \scriptsize{(5)} & \scriptsize{(6)} & \scriptsize{(7)} & \scriptsize{(8)} \\ 
    \midrule  

    \mc{1}{l}{\scriptsize{Measured BMI}} & \mc{1}{c}{\scriptsize{Mid-30s}} & \mc{1}{c}{\scriptsize{-0.125}} & \mc{1}{c}{\scriptsize{-0.617}} & \mc{1}{c}{\scriptsize{-0.684}} & \mc{1}{c}{\scriptsize{-3.505}} & \mc{1}{c}{\scriptsize{0.867}} & \mc{1}{c}{\scriptsize{-0.627}} & \mc{1}{c}{\scriptsize{0.025}} & \mc{1}{c}{\scriptsize{-0.487}} \\  

     &  & \mc{1}{c}{\scriptsize{(0.500)}} & \mc{1}{c}{\scriptsize{(0.408)}} & \mc{1}{c}{\scriptsize{(0.382)}} & \mc{1}{c}{\scriptsize{(0.303)}} & \mc{1}{c}{\scriptsize{(0.447)}} & \mc{1}{c}{\scriptsize{(0.447)}} & \mc{1}{c}{\scriptsize{(0.539)}} & \mc{1}{c}{\scriptsize{(0.421)}} \\  

    \mc{1}{l}{\scriptsize{Obesity}} & \mc{1}{c}{\scriptsize{Mid-30s}} &  & \mc{1}{c}{\scriptsize{-0.103}} & \mc{1}{c}{\scriptsize{-0.128}} & \mc{1}{c}{\scriptsize{-0.331}} & \mc{1}{c}{\scriptsize{0.031}} & \mc{1}{c}{\scriptsize{-0.017}} & \mc{1}{c}{\scriptsize{-0.045}} & \mc{1}{c}{\scriptsize{-0.060}} \\  

     &  &  & \mc{1}{c}{\scriptsize{(0.303)}} & \mc{1}{c}{\scriptsize{(0.329)}} & \mc{1}{c}{\scriptsize{\textbf{(0.079)}}} & \mc{1}{c}{\scriptsize{(0.487)}} & \mc{1}{c}{\scriptsize{(0.487)}} & \mc{1}{c}{\scriptsize{(0.395)}} & \mc{1}{c}{\scriptsize{(0.395)}} \\  

    \mc{1}{l}{\scriptsize{Severe Obesity}} & \mc{1}{c}{\scriptsize{Mid-30s}} & \mc{1}{c}{\scriptsize{-0.160}} & \mc{1}{c}{\scriptsize{-0.132}} & \mc{1}{c}{\scriptsize{-0.185}} & \mc{1}{c}{\scriptsize{-0.319}} & \mc{1}{c}{\scriptsize{-0.126}} & \mc{1}{c}{\scriptsize{-0.185}} & \mc{1}{c}{\scriptsize{-0.108}} & \mc{1}{c}{\scriptsize{-0.131}} \\  

     &  & \mc{1}{c}{\scriptsize{(0.132)}} & \mc{1}{c}{\scriptsize{(0.211)}} & \mc{1}{c}{\scriptsize{(0.224)}} & \mc{1}{c}{\scriptsize{(0.145)}} & \mc{1}{c}{\scriptsize{(0.276)}} & \mc{1}{c}{\scriptsize{(0.171)}} & \mc{1}{c}{\scriptsize{(0.289)}} & \mc{1}{c}{\scriptsize{(0.250)}} \\  

    \mc{1}{l}{\scriptsize{Waist-hip Ratio}} & \mc{1}{c}{\scriptsize{Mid-30s}} & \mc{1}{c}{\scriptsize{0.004}} & \mc{1}{c}{\scriptsize{-0.014}} & \mc{1}{c}{\scriptsize{0.018}} & \mc{1}{c}{\scriptsize{-0.043}} & \mc{1}{c}{\scriptsize{0.030}} & \mc{1}{c}{\scriptsize{-0.002}} & \mc{1}{c}{\scriptsize{-0.008}} & \mc{1}{c}{\scriptsize{-0.004}} \\  

     &  & \mc{1}{c}{\scriptsize{(0.553)}} & \mc{1}{c}{\scriptsize{(0.237)}} & \mc{1}{c}{\scriptsize{(0.526)}} & \mc{1}{c}{\scriptsize{(0.171)}} & \mc{1}{c}{\scriptsize{(0.553)}} & \mc{1}{c}{\scriptsize{(0.461)}} & \mc{1}{c}{\scriptsize{(0.329)}} & \mc{1}{c}{\scriptsize{(0.421)}} \\  

    \mc{1}{l}{\scriptsize{Abdominal Obesity}} & \mc{1}{c}{\scriptsize{Mid-30s}} & \mc{1}{c}{\scriptsize{0.003}} & \mc{1}{c}{\scriptsize{-0.137}} & \mc{1}{c}{\scriptsize{0.029}} & \mc{1}{c}{\scriptsize{-0.377}} & \mc{1}{c}{\scriptsize{0.100}} & \mc{1}{c}{\scriptsize{0.029}} & \mc{1}{c}{\scriptsize{-0.062}} & \mc{1}{c}{\scriptsize{-0.031}} \\  

     &  & \mc{1}{c}{\scriptsize{(0.461)}} & \mc{1}{c}{\scriptsize{(0.211)}} & \mc{1}{c}{\scriptsize{(0.487)}} & \mc{1}{c}{\scriptsize{(0.171)}} & \mc{1}{c}{\scriptsize{(0.566)}} & \mc{1}{c}{\scriptsize{(0.605)}} & \mc{1}{c}{\scriptsize{(0.382)}} & \mc{1}{c}{\scriptsize{(0.421)}} \\  

    \mc{1}{l}{\scriptsize{Framingham Risk Score}} & \mc{1}{c}{\scriptsize{Mid-30s}} & \mc{1}{c}{\scriptsize{-0.766}} & \mc{1}{c}{\scriptsize{-0.472}} & \mc{1}{c}{\scriptsize{1.491}} & \mc{1}{c}{\scriptsize{2.437}} & \mc{1}{c}{\scriptsize{1.802}} & \mc{1}{c}{\scriptsize{-1.202}} & \mc{1}{c}{\scriptsize{-1.036}} & \mc{1}{c}{\scriptsize{-0.705}} \\  

     &  & \mc{1}{c}{\scriptsize{(0.224)}} & \mc{1}{c}{\scriptsize{(0.368)}} & \mc{1}{c}{\scriptsize{(0.737)}} & \mc{1}{c}{\scriptsize{(0.684)}} & \mc{1}{c}{\scriptsize{(0.816)}} & \mc{1}{c}{\scriptsize{(0.184)}} & \mc{1}{c}{\scriptsize{(0.250)}} & \mc{1}{c}{\scriptsize{(0.316)}} \\  

    \mc{1}{l}{\scriptsize{Obesity Factor}} & \mc{1}{c}{\scriptsize{Mid-30s}} & \mc{1}{c}{\scriptsize{-0.040}} & \mc{1}{c}{\scriptsize{-0.197}} & \mc{1}{c}{\scriptsize{-0.036}} & \mc{1}{c}{\scriptsize{-0.524}} & \mc{1}{c}{\scriptsize{0.211}} & \mc{1}{c}{\scriptsize{-0.101}} & \mc{1}{c}{\scriptsize{-0.091}} & \mc{1}{c}{\scriptsize{-0.116}} \\  

     &  & \mc{1}{c}{\scriptsize{(0.461)}} & \mc{1}{c}{\scriptsize{(0.263)}} & \mc{1}{c}{\scriptsize{(0.408)}} & \mc{1}{c}{\scriptsize{(0.211)}} & \mc{1}{c}{\scriptsize{(0.474)}} & \mc{1}{c}{\scriptsize{(0.395)}} & \mc{1}{c}{\scriptsize{(0.434)}} & \mc{1}{c}{\scriptsize{(0.355)}} \\ 
    \midrule  

    \mc{2}{l}{\scriptsize{\% of Pos. TE ($H_0$: $\le$ 50\%)}} & \mc{1}{c}{\scriptsize{67}} & \mc{1}{c}{\scriptsize{100}} & \mc{1}{c}{\scriptsize{57}} & \mc{1}{c}{\scriptsize{86}} & \mc{1}{c}{\scriptsize{14}} & \mc{1}{c}{\scriptsize{86}} & \mc{1}{c}{\scriptsize{86}} & \mc{1}{c}{\scriptsize{100}} \\  

     &  & \mc{1}{c}{\scriptsize{(0.395)}} & \mc{1}{c}{\scriptsize{\textbf{(0.000)}}} & \mc{1}{c}{\scriptsize{(0.421)}} & \mc{1}{c}{\scriptsize{\textbf{(0.000)}}} & \mc{1}{c}{\scriptsize{(0.737)}} & \mc{1}{c}{\scriptsize{\textbf{(0.000)}}} & \mc{1}{c}{\scriptsize{\textbf{(0.000)}}} & \mc{1}{c}{\scriptsize{\textbf{(0.000)}}} \\  

    \mc{2}{l}{\scriptsize{\% of Pos. TE ($H_0$: $\le$ 10\% $|$ 10\% Significance)}} & \mc{1}{c}{\scriptsize{17}} & \mc{1}{c}{\scriptsize{0}} & \mc{1}{c}{\scriptsize{0}} & \mc{1}{c}{\scriptsize{0}} & \mc{1}{c}{\scriptsize{0}} & \mc{1}{c}{\scriptsize{14}} & \mc{1}{c}{\scriptsize{0}} & \mc{1}{c}{\scriptsize{0}} \\  

     &  & \mc{1}{c}{\scriptsize{(0.145)}} & \mc{1}{c}{\scriptsize{(0.395)}} & \mc{1}{c}{\scriptsize{(0.329)}} & \mc{1}{c}{\scriptsize{(0.500)}} & \mc{1}{c}{\scriptsize{(0.342)}} & \mc{1}{c}{\scriptsize{(0.184)}} & \mc{1}{c}{\scriptsize{(1.000)}} & \mc{1}{c}{\scriptsize{(0.368)}} \\  

  \bottomrule
  \end{tabular}
	\end{table} 

	\begin{table}[H]
     \caption{Treatment Effects on Childhood Household Income, Male Sample}
     \label{table:abccare_rslt_male_cat16}
	  \begin{tabular}{cccccccccc}
  \toprule

    \scriptsize{Variable} & \scriptsize{Age} & \scriptsize{(1)} & \scriptsize{(2)} & \scriptsize{(3)} & \scriptsize{(4)} & \scriptsize{(5)} & \scriptsize{(6)} & \scriptsize{(7)} & \scriptsize{(8)} \\ 
    \midrule  

    \mc{1}{l}{\scriptsize{Vitamin D Deficiency}} & \mc{1}{c}{\scriptsize{Mid-30s}} & \mc{1}{c}{\scriptsize{-0.382}} & \mc{1}{c}{\scriptsize{-0.255}} & \mc{1}{c}{\scriptsize{-0.632}} & \mc{1}{c}{\scriptsize{-1.113}} & \mc{1}{c}{\scriptsize{-0.679}} & \mc{1}{c}{\scriptsize{-0.332}} & \mc{1}{c}{\scriptsize{-0.139}} & \mc{1}{c}{\scriptsize{-0.307}} \\  

     &  & \mc{1}{c}{\scriptsize{\textbf{(0.000)}}} & \mc{1}{c}{\scriptsize{(0.105)}} & \mc{1}{c}{\scriptsize{\textbf{(0.000)}}} & \mc{1}{c}{\scriptsize{\textbf{(0.000)}}} & \mc{1}{c}{\scriptsize{\textbf{(0.000)}}} & \mc{1}{c}{\scriptsize{\textbf{(0.066)}}} & \mc{1}{c}{\scriptsize{(0.263)}} & \mc{1}{c}{\scriptsize{\textbf{(0.079)}}} \\ 
    \midrule  

    \mc{2}{l}{\scriptsize{\% of Pos. TE ($H_0$: $\le$ 50\%)}} & \mc{1}{c}{\scriptsize{100}} & \mc{1}{c}{\scriptsize{100}} & \mc{1}{c}{\scriptsize{100}} & \mc{1}{c}{\scriptsize{100}} & \mc{1}{c}{\scriptsize{100}} & \mc{1}{c}{\scriptsize{100}} & \mc{1}{c}{\scriptsize{100}} & \mc{1}{c}{\scriptsize{100}} \\  

     &  & \mc{1}{c}{\scriptsize{\textbf{(0.000)}}} & \mc{1}{c}{\scriptsize{\textbf{(0.000)}}} & \mc{1}{c}{\scriptsize{\textbf{(0.000)}}} & \mc{1}{c}{\scriptsize{\textbf{(0.000)}}} & \mc{1}{c}{\scriptsize{\textbf{(0.000)}}} & \mc{1}{c}{\scriptsize{\textbf{(0.000)}}} & \mc{1}{c}{\scriptsize{\textbf{(0.000)}}} & \mc{1}{c}{\scriptsize{\textbf{(0.000)}}} \\  

    \mc{2}{l}{\scriptsize{\% of Pos. TE ($H_0$: $\le$ 10\% $|$ 10\% Significance)}} & \mc{1}{c}{\scriptsize{100}} & \mc{1}{c}{\scriptsize{0}} & \mc{1}{c}{\scriptsize{100}} & \mc{1}{c}{\scriptsize{100}} & \mc{1}{c}{\scriptsize{100}} & \mc{1}{c}{\scriptsize{100}} & \mc{1}{c}{\scriptsize{0}} & \mc{1}{c}{\scriptsize{100}} \\  

     &  & \mc{1}{c}{\scriptsize{\textbf{(0.000)}}} & \mc{1}{c}{\scriptsize{(0.184)}} & \mc{1}{c}{\scriptsize{\textbf{(0.000)}}} & \mc{1}{c}{\scriptsize{\textbf{(0.000)}}} & \mc{1}{c}{\scriptsize{\textbf{(0.000)}}} & \mc{1}{c}{\scriptsize{\textbf{(0.000)}}} & \mc{1}{c}{\scriptsize{(0.132)}} & \mc{1}{c}{\scriptsize{\textbf{(0.000)}}} \\  

  \bottomrule
  \end{tabular}
	\end{table} 

	\begin{table}[H]
     \caption{Treatment Effects on Father at Home, Male Sample}
     \label{table:abccare_rslt_male_cat17}
	\begin{table}[H]
\captionsetup{singlelinecheck=false,justification=centering}
\caption{ABC Average Treatment Effects, Males \\ Obesity \label{tab:ate_male_apx17}}

  \begin{threeparttable}
  \begin{tabular}{cccccccccc}
  \hline\hline

     &  & \scriptsize{(1)} & \scriptsize{(2)} & \scriptsize{(3)} & \scriptsize{(4)} & \scriptsize{(5)} & \scriptsize{(6)} & \scriptsize{(7)} & \scriptsize{(8)} \\  

     &  &  &  & \mc{3}{c}{\scriptsize{$P=0$}} & \mc{3}{c}{\scriptsize{$P=1$}} \\ 
    \cmidrule(lr){5-7} \cmidrule(lr){8-10} 

    \scriptsize{Variable} & \scriptsize{Age} & \scriptsize{ITT} & \scriptsize{ITT$|X,W$} & \scriptsize{ITT} & \scriptsize{ITT$|X,W$} & \scriptsize{KE$|X,W$} & \scriptsize{ITT} & \scriptsize{ITT$|X,W$} & \scriptsize{KE$|X,W$} \\ 
    \hline  

    \mc{1}{l}{\scriptsize{Measured BMI}} & \mc{1}{c}{\scriptsize{Mid-30s}} & \mc{1}{c}{\scriptsize{-0.041}} & \mc{1}{c}{\scriptsize{-2.966}} & \mc{1}{c}{\scriptsize{4.219}} & \mc{1}{c}{\scriptsize{2.774}} &  & \mc{1}{c}{\scriptsize{-6.856}} & \mc{1}{c}{\scriptsize{-9.309}} & \mc{1}{c}{\scriptsize{-7.135}} \\  

     &  & \mc{1}{c}{\scriptsize{(0.529)}} & \mc{1}{c}{\scriptsize{(0.216)}} & \mc{1}{c}{\scriptsize{(0.941)}} & \mc{1}{c}{\scriptsize{(0.765)}} &  & \mc{1}{c}{\scriptsize{\textbf{(0.059)}}} & \mc{1}{c}{\scriptsize{\textbf{(0.078)}}} & \mc{1}{c}{\scriptsize{\textbf{(0.059)}}} \\  

    \mc{1}{l}{\scriptsize{Obesity}} & \mc{1}{c}{\scriptsize{Mid-30s}} & \mc{1}{c}{\scriptsize{0.238}} & \mc{1}{c}{\scriptsize{-0.019}} & \mc{1}{c}{\scriptsize{0.444}} & \mc{1}{c}{\scriptsize{0.361}} &  & \mc{1}{c}{\scriptsize{-0.133}} & \mc{1}{c}{\scriptsize{-0.347}} & \mc{1}{c}{\scriptsize{-0.225}} \\  

     &  & \mc{1}{c}{\scriptsize{(0.863)}} & \mc{1}{c}{\scriptsize{(0.490)}} & \mc{1}{c}{\scriptsize{(0.980)}} & \mc{1}{c}{\scriptsize{(0.824)}} &  & \mc{1}{c}{\scriptsize{(0.294)}} & \mc{1}{c}{\scriptsize{(0.118)}} & \mc{1}{c}{\scriptsize{(0.157)}} \\  

    \mc{1}{l}{\scriptsize{Severe Obesity}} & \mc{1}{c}{\scriptsize{Mid-30s}} & \mc{1}{c}{\scriptsize{-0.033}} & \mc{1}{c}{\scriptsize{-0.097}} & \mc{1}{c}{\scriptsize{0.167}} & \mc{1}{c}{\scriptsize{0.182}} & \mc{1}{c}{\scriptsize{0.171}} & \mc{1}{c}{\scriptsize{-0.433}} & \mc{1}{c}{\scriptsize{-0.450}} & \mc{1}{c}{\scriptsize{-0.410}} \\  

     &  & \mc{1}{c}{\scriptsize{(0.373)}} & \mc{1}{c}{\scriptsize{(0.314)}} & \mc{1}{c}{\scriptsize{(0.706)}} & \mc{1}{c}{\scriptsize{(0.588)}} & \mc{1}{c}{\scriptsize{(0.647)}} & \mc{1}{c}{\scriptsize{\textbf{(0.059)}}} & \mc{1}{c}{\scriptsize{(0.216)}} & \mc{1}{c}{\scriptsize{\textbf{(0.078)}}} \\  

    \mc{1}{l}{\scriptsize{Waist-hip Ratio}} & \mc{1}{c}{\scriptsize{Mid-30s}} & \mc{1}{c}{\scriptsize{0.015}} & \mc{1}{c}{\scriptsize{-0.005}} & \mc{1}{c}{\scriptsize{0.049}} & \mc{1}{c}{\scriptsize{0.044}} & \mc{1}{c}{\scriptsize{0.045}} & \mc{1}{c}{\scriptsize{-0.052}} & \mc{1}{c}{\scriptsize{-0.059}} & \mc{1}{c}{\scriptsize{-0.049}} \\  

     &  & \mc{1}{c}{\scriptsize{(0.667)}} & \mc{1}{c}{\scriptsize{(0.412)}} & \mc{1}{c}{\scriptsize{(0.863)}} & \mc{1}{c}{\scriptsize{(0.745)}} & \mc{1}{c}{\scriptsize{(0.647)}} & \mc{1}{c}{\scriptsize{(0.118)}} & \mc{1}{c}{\scriptsize{(0.294)}} & \mc{1}{c}{\scriptsize{(0.235)}} \\  

    \mc{1}{l}{\scriptsize{Abdominal Obesity}} & \mc{1}{c}{\scriptsize{Mid-30s}} & \mc{1}{c}{\scriptsize{-0.023}} & \mc{1}{c}{\scriptsize{-0.159}} & \mc{1}{c}{\scriptsize{0.102}} & \mc{1}{c}{\scriptsize{-0.030}} & \mc{1}{c}{\scriptsize{0.105}} & \mc{1}{c}{\scriptsize{-0.273}} & \mc{1}{c}{\scriptsize{-0.298}} & \mc{1}{c}{\scriptsize{-0.296}} \\  

     &  & \mc{1}{c}{\scriptsize{(0.510)}} & \mc{1}{c}{\scriptsize{(0.255)}} & \mc{1}{c}{\scriptsize{(0.647)}} & \mc{1}{c}{\scriptsize{(0.412)}} & \mc{1}{c}{\scriptsize{(0.451)}} & \mc{1}{c}{\scriptsize{\textbf{(0.059)}}} & \mc{1}{c}{\scriptsize{(0.255)}} & \mc{1}{c}{\scriptsize{\textbf{(0.059)}}} \\  

    \mc{1}{l}{\scriptsize{Framingham Risk Score}} & \mc{1}{c}{\scriptsize{Mid-30s}} & \mc{1}{c}{\scriptsize{-0.469}} & \mc{1}{c}{\scriptsize{0.565}} & \mc{1}{c}{\scriptsize{1.022}} & \mc{1}{c}{\scriptsize{2.145}} &  & \mc{1}{c}{\scriptsize{-2.854}} & \mc{1}{c}{\scriptsize{-2.133}} & \mc{1}{c}{\scriptsize{-2.114}} \\  

     &  & \mc{1}{c}{\scriptsize{(0.294)}} & \mc{1}{c}{\scriptsize{(0.608)}} & \mc{1}{c}{\scriptsize{(0.725)}} & \mc{1}{c}{\scriptsize{(0.824)}} &  & \mc{1}{c}{\scriptsize{\textbf{(0.039)}}} & \mc{1}{c}{\scriptsize{(0.196)}} & \mc{1}{c}{\scriptsize{(0.118)}} \\  

    \mc{1}{l}{\scriptsize{Obesity Factor}} & \mc{1}{c}{\scriptsize{Mid-30s}} & \mc{1}{c}{\scriptsize{0.076}} & \mc{1}{c}{\scriptsize{-0.263}} & \mc{1}{c}{\scriptsize{0.647}} & \mc{1}{c}{\scriptsize{0.442}} & \mc{1}{c}{\scriptsize{0.591}} & \mc{1}{c}{\scriptsize{-1.065}} & \mc{1}{c}{\scriptsize{-1.169}} & \mc{1}{c}{\scriptsize{-1.038}} \\  

     &  & \mc{1}{c}{\scriptsize{(0.608)}} & \mc{1}{c}{\scriptsize{(0.314)}} & \mc{1}{c}{\scriptsize{(0.961)}} & \mc{1}{c}{\scriptsize{(0.784)}} & \mc{1}{c}{\scriptsize{(0.647)}} & \mc{1}{c}{\scriptsize{\textbf{(0.000)}}} & \mc{1}{c}{\scriptsize{(0.235)}} & \mc{1}{c}{\scriptsize{\textbf{(0.000)}}} \\ 
    \hline  

    \\[0.1cm]
    \mc{2}{l}{\scriptsize{\% of Pos. TE ($H_0$: $\le$ 25\% $|$ 10\% Significance)}} & \mc{1}{c}{\scriptsize{0}} & \mc{1}{c}{\scriptsize{0}} & \mc{1}{c}{\scriptsize{0}} & \mc{1}{c}{\scriptsize{0}} & \mc{1}{c}{\scriptsize{0}} & \mc{1}{c}{\scriptsize{71}} & \mc{1}{c}{\scriptsize{14}} & \mc{1}{c}{\scriptsize{57}} \\  

     &  & \mc{1}{c}{\scriptsize{(1.000)}} & \mc{1}{c}{\scriptsize{(1.000)}} & \mc{1}{c}{\scriptsize{(1.000)}} & \mc{1}{c}{\scriptsize{(1.000)}} & \mc{1}{c}{\scriptsize{(0.980)}} & \mc{1}{c}{\scriptsize{\textbf{(0.000)}}} & \mc{1}{c}{\scriptsize{(0.490)}} & \mc{1}{c}{\scriptsize{(0.176)}} \\  

    \mc{2}{l}{\scriptsize{\% of Pos. TE ($H_0$: $\le$ 50\% $|$ 10\% Significance)}} & \mc{1}{c}{\scriptsize{0}} & \mc{1}{c}{\scriptsize{0}} & \mc{1}{c}{\scriptsize{0}} & \mc{1}{c}{\scriptsize{0}} & \mc{1}{c}{\scriptsize{0}} & \mc{1}{c}{\scriptsize{71}} & \mc{1}{c}{\scriptsize{14}} & \mc{1}{c}{\scriptsize{57}} \\  

     &  & \mc{1}{c}{\scriptsize{(1.000)}} & \mc{1}{c}{\scriptsize{(1.000)}} & \mc{1}{c}{\scriptsize{(1.000)}} & \mc{1}{c}{\scriptsize{(1.000)}} & \mc{1}{c}{\scriptsize{(0.980)}} & \mc{1}{c}{\scriptsize{(0.157)}} & \mc{1}{c}{\scriptsize{(0.863)}} & \mc{1}{c}{\scriptsize{(0.490)}} \\  

    \mc{2}{l}{\scriptsize{\% of Pos. TE ($H_0$: $\le$ 75\% $|$ 10\% Significance)}} & \mc{1}{c}{\scriptsize{0}} & \mc{1}{c}{\scriptsize{0}} & \mc{1}{c}{\scriptsize{0}} & \mc{1}{c}{\scriptsize{0}} & \mc{1}{c}{\scriptsize{0}} & \mc{1}{c}{\scriptsize{71}} & \mc{1}{c}{\scriptsize{14}} & \mc{1}{c}{\scriptsize{57}} \\  

     &  & \mc{1}{c}{\scriptsize{(1.000)}} & \mc{1}{c}{\scriptsize{(1.000)}} & \mc{1}{c}{\scriptsize{(1.000)}} & \mc{1}{c}{\scriptsize{(1.000)}} & \mc{1}{c}{\scriptsize{(0.980)}} & \mc{1}{c}{\scriptsize{(0.608)}} & \mc{1}{c}{\scriptsize{(0.863)}} & \mc{1}{c}{\scriptsize{(0.706)}} \\  

  \hline\hline
  \end{tabular}
    \begin{tablenotes}
    \scriptsize
    \item 
Note: This table displays various estimates of the treatment effect of ABC's school age program.
Column (1) displays the ITT, without accounting for any controls.
Column (2) displays the ITT conditioning on vector of controls, $X$, consisting of the Apgar score 1 minute after birth, the HRI index, maternal IQ, an
indicator for teenage pregnancy of the mother, an indicator for the father being at 
home, and an indicator for having a grandmother residing in the same county. We also apply IPW weights, $W$, to account for attrition.
Columns (3)--(4) are analogous to columns (1)--(2), but we restrict the control sample to subjects
who did not enroll in any alternative care.
Column (5) displys the matching estimate, where we use the Mahalanobis metric and Epanechnikov kernel
to match on controls $X$ listed above, and restrict the control sample to subjects who did not enroll
in any alternative care. Additionally, we apply IPW weights, $W$.
Columns (6)--(8) are analogous to Columns (3)--(5), except we restrict the control sample to subejcts
who did enroll in alternative care. The final three pairs of rows display the proportion of treatment effects in the table that are 
socially positive. The first row in each pair displays the percentage of treatment effects, and the
second row presents the inference. 
Numbers in parentheses represent the $p$-value from a single hypothesis test, and are obtained from 
the empirical bootstrap distribution generated by 200 resamples of the original data. 
Bold $p$-values indicate significance at the 10\% level.
Blank point estimates indicate that we are unable to obtain estimates due to a lack of support in the data. 

    \end{tablenotes}
  \end{threeparttable}

\end{table}
	\end{table} 

	\begin{table}[H]
     \caption{Treatment Effects on HOME Scores, Male Sample}
     \label{table:abccare_rslt_male_cat18}
	  \begin{tabular}{cccccccccc}
  \toprule

    \scriptsize{Variable} & \scriptsize{Age} & \scriptsize{(1)} & \scriptsize{(2)} & \scriptsize{(3)} & \scriptsize{(4)} & \scriptsize{(5)} & \scriptsize{(6)} & \scriptsize{(7)} & \scriptsize{(8)} \\ 
    \midrule  

    \mc{1}{l}{\scriptsize{Somatization}} & \mc{1}{c}{\scriptsize{21}} & \mc{1}{c}{\scriptsize{0.035}} & \mc{1}{c}{\scriptsize{-0.051}} & \mc{1}{c}{\scriptsize{-0.021}} & \mc{1}{c}{\scriptsize{-0.259}} & \mc{1}{c}{\scriptsize{-0.019}} & \mc{1}{c}{\scriptsize{0.064}} & \mc{1}{c}{\scriptsize{-0.010}} & \mc{1}{c}{\scriptsize{0.056}} \\  

     &  & \mc{1}{c}{\scriptsize{(0.592)}} & \mc{1}{c}{\scriptsize{(0.342)}} & \mc{1}{c}{\scriptsize{(0.434)}} & \mc{1}{c}{\scriptsize{(0.197)}} & \mc{1}{c}{\scriptsize{(0.487)}} & \mc{1}{c}{\scriptsize{(0.618)}} & \mc{1}{c}{\scriptsize{(0.408)}} & \mc{1}{c}{\scriptsize{(0.618)}} \\  

     & \mc{1}{c}{\scriptsize{34}} & \mc{1}{c}{\scriptsize{-0.466}} & \mc{1}{c}{\scriptsize{-0.502}} & \mc{1}{c}{\scriptsize{0.071}} & \mc{1}{c}{\scriptsize{-0.062}} & \mc{1}{c}{\scriptsize{0.048}} & \mc{1}{c}{\scriptsize{-0.619}} & \mc{1}{c}{\scriptsize{-0.566}} & \mc{1}{c}{\scriptsize{-0.649}} \\  

     &  & \mc{1}{c}{\scriptsize{(0.118)}} & \mc{1}{c}{\scriptsize{(0.118)}} & \mc{1}{c}{\scriptsize{(0.632)}} & \mc{1}{c}{\scriptsize{\textbf{(0.092)}}} & \mc{1}{c}{\scriptsize{(0.579)}} & \mc{1}{c}{\scriptsize{\textbf{(0.079)}}} & \mc{1}{c}{\scriptsize{(0.132)}} & \mc{1}{c}{\scriptsize{\textbf{(0.079)}}} \\  

    \mc{1}{l}{\scriptsize{Depression}} & \mc{1}{c}{\scriptsize{21}} & \mc{1}{c}{\scriptsize{-0.116}} & \mc{1}{c}{\scriptsize{-0.180}} & \mc{1}{c}{\scriptsize{0.105}} & \mc{1}{c}{\scriptsize{-0.142}} & \mc{1}{c}{\scriptsize{0.119}} & \mc{1}{c}{\scriptsize{-0.078}} & \mc{1}{c}{\scriptsize{-0.180}} & \mc{1}{c}{\scriptsize{-0.117}} \\  

     &  & \mc{1}{c}{\scriptsize{(0.263)}} & \mc{1}{c}{\scriptsize{(0.224)}} & \mc{1}{c}{\scriptsize{(0.553)}} & \mc{1}{c}{\scriptsize{(0.329)}} & \mc{1}{c}{\scriptsize{(0.592)}} & \mc{1}{c}{\scriptsize{(0.368)}} & \mc{1}{c}{\scriptsize{(0.237)}} & \mc{1}{c}{\scriptsize{(0.289)}} \\  

     & \mc{1}{c}{\scriptsize{34}} & \mc{1}{c}{\scriptsize{-0.320}} & \mc{1}{c}{\scriptsize{-0.265}} & \mc{1}{c}{\scriptsize{0.254}} & \mc{1}{c}{\scriptsize{0.257}} & \mc{1}{c}{\scriptsize{0.262}} & \mc{1}{c}{\scriptsize{-0.484}} & \mc{1}{c}{\scriptsize{-0.343}} & \mc{1}{c}{\scriptsize{-0.458}} \\  

     &  & \mc{1}{c}{\scriptsize{(0.224)}} & \mc{1}{c}{\scriptsize{(0.263)}} & \mc{1}{c}{\scriptsize{(0.645)}} & \mc{1}{c}{\scriptsize{(0.592)}} & \mc{1}{c}{\scriptsize{(0.632)}} & \mc{1}{c}{\scriptsize{(0.171)}} & \mc{1}{c}{\scriptsize{(0.263)}} & \mc{1}{c}{\scriptsize{(0.171)}} \\  

    \mc{1}{l}{\scriptsize{Anxiety}} & \mc{1}{c}{\scriptsize{21}} & \mc{1}{c}{\scriptsize{0.119}} & \mc{1}{c}{\scriptsize{0.006}} & \mc{1}{c}{\scriptsize{0.286}} & \mc{1}{c}{\scriptsize{-0.023}} & \mc{1}{c}{\scriptsize{0.278}} & \mc{1}{c}{\scriptsize{0.070}} & \mc{1}{c}{\scriptsize{-0.025}} & \mc{1}{c}{\scriptsize{0.019}} \\  

     &  & \mc{1}{c}{\scriptsize{(0.868)}} & \mc{1}{c}{\scriptsize{(0.526)}} & \mc{1}{c}{\scriptsize{(0.895)}} & \mc{1}{c}{\scriptsize{(0.434)}} & \mc{1}{c}{\scriptsize{(0.868)}} & \mc{1}{c}{\scriptsize{(0.737)}} & \mc{1}{c}{\scriptsize{(0.434)}} & \mc{1}{c}{\scriptsize{(0.579)}} \\  

     & \mc{1}{c}{\scriptsize{34}} & \mc{1}{c}{\scriptsize{-0.415}} & \mc{1}{c}{\scriptsize{-0.399}} & \mc{1}{c}{\scriptsize{0.103}} & \mc{1}{c}{\scriptsize{0.102}} & \mc{1}{c}{\scriptsize{0.117}} & \mc{1}{c}{\scriptsize{-0.564}} & \mc{1}{c}{\scriptsize{-0.471}} & \mc{1}{c}{\scriptsize{-0.571}} \\  

     &  & \mc{1}{c}{\scriptsize{(0.105)}} & \mc{1}{c}{\scriptsize{(0.158)}} & \mc{1}{c}{\scriptsize{(0.658)}} & \mc{1}{c}{\scriptsize{(0.553)}} & \mc{1}{c}{\scriptsize{(0.658)}} & \mc{1}{c}{\scriptsize{(0.118)}} & \mc{1}{c}{\scriptsize{(0.158)}} & \mc{1}{c}{\scriptsize{\textbf{(0.092)}}} \\  

    \mc{1}{l}{\scriptsize{Hostility}} & \mc{1}{c}{\scriptsize{21}} & \mc{1}{c}{\scriptsize{-0.180}} & \mc{1}{c}{\scriptsize{-0.165}} & \mc{1}{c}{\scriptsize{0.118}} & \mc{1}{c}{\scriptsize{-0.073}} & \mc{1}{c}{\scriptsize{0.109}} & \mc{1}{c}{\scriptsize{-0.238}} & \mc{1}{c}{\scriptsize{-0.195}} & \mc{1}{c}{\scriptsize{-0.257}} \\  

     &  & \mc{1}{c}{\scriptsize{(0.184)}} & \mc{1}{c}{\scriptsize{(0.250)}} & \mc{1}{c}{\scriptsize{(0.632)}} & \mc{1}{c}{\scriptsize{(0.382)}} & \mc{1}{c}{\scriptsize{(0.632)}} & \mc{1}{c}{\scriptsize{(0.158)}} & \mc{1}{c}{\scriptsize{(0.237)}} & \mc{1}{c}{\scriptsize{(0.145)}} \\  

     & \mc{1}{c}{\scriptsize{34}} & \mc{1}{c}{\scriptsize{-0.368}} & \mc{1}{c}{\scriptsize{-0.335}} & \mc{1}{c}{\scriptsize{0.143}} & \mc{1}{c}{\scriptsize{0.385}} & \mc{1}{c}{\scriptsize{0.133}} & \mc{1}{c}{\scriptsize{-0.514}} & \mc{1}{c}{\scriptsize{-0.439}} & \mc{1}{c}{\scriptsize{-0.522}} \\  

     &  & \mc{1}{c}{\scriptsize{(0.105)}} & \mc{1}{c}{\scriptsize{(0.171)}} & \mc{1}{c}{\scriptsize{(0.645)}} & \mc{1}{c}{\scriptsize{(0.618)}} & \mc{1}{c}{\scriptsize{(0.645)}} & \mc{1}{c}{\scriptsize{(0.132)}} & \mc{1}{c}{\scriptsize{(0.158)}} & \mc{1}{c}{\scriptsize{\textbf{(0.092)}}} \\  

    \mc{1}{l}{\scriptsize{Global Severity Index}} & \mc{1}{c}{\scriptsize{21}} & \mc{1}{c}{\scriptsize{-0.014}} & \mc{1}{c}{\scriptsize{-0.098}} & \mc{1}{c}{\scriptsize{0.188}} & \mc{1}{c}{\scriptsize{-0.102}} & \mc{1}{c}{\scriptsize{0.189}} & \mc{1}{c}{\scriptsize{-0.046}} & \mc{1}{c}{\scriptsize{-0.107}} & \mc{1}{c}{\scriptsize{-0.075}} \\  

     &  & \mc{1}{c}{\scriptsize{(0.474)}} & \mc{1}{c}{\scriptsize{(0.276)}} & \mc{1}{c}{\scriptsize{(0.829)}} & \mc{1}{c}{\scriptsize{(0.289)}} & \mc{1}{c}{\scriptsize{(0.842)}} & \mc{1}{c}{\scriptsize{(0.368)}} & \mc{1}{c}{\scriptsize{(0.289)}} & \mc{1}{c}{\scriptsize{(0.289)}} \\  

     & \mc{1}{c}{\scriptsize{34}} & \mc{1}{c}{\scriptsize{-7.206}} & \mc{1}{c}{\scriptsize{-6.999}} & \mc{1}{c}{\scriptsize{2.571}} & \mc{1}{c}{\scriptsize{1.777}} & \mc{1}{c}{\scriptsize{2.559}} & \mc{1}{c}{\scriptsize{-10.000}} & \mc{1}{c}{\scriptsize{-8.283}} & \mc{1}{c}{\scriptsize{-10.071}} \\  

     &  & \mc{1}{c}{\scriptsize{(0.118)}} & \mc{1}{c}{\scriptsize{(0.171)}} & \mc{1}{c}{\scriptsize{(0.658)}} & \mc{1}{c}{\scriptsize{(0.566)}} & \mc{1}{c}{\scriptsize{(0.645)}} & \mc{1}{c}{\scriptsize{(0.132)}} & \mc{1}{c}{\scriptsize{(0.158)}} & \mc{1}{c}{\scriptsize{(0.105)}} \\  

    \mc{1}{l}{\scriptsize{BSI Factor}} & \mc{1}{c}{\scriptsize{21 and 34}} & \mc{1}{c}{\scriptsize{-0.260}} & \mc{1}{c}{\scriptsize{-0.083}} & \mc{1}{c}{\scriptsize{0.240}} & \mc{1}{c}{\scriptsize{-0.012}} & \mc{1}{c}{\scriptsize{0.279}} & \mc{1}{c}{\scriptsize{-0.343}} & \mc{1}{c}{\scriptsize{-0.150}} & \mc{1}{c}{\scriptsize{-0.186}} \\  

     &  & \mc{1}{c}{\scriptsize{(0.211)}} & \mc{1}{c}{\scriptsize{(0.355)}} & \mc{1}{c}{\scriptsize{(0.618)}} & \mc{1}{c}{\scriptsize{(0.316)}} & \mc{1}{c}{\scriptsize{(0.645)}} & \mc{1}{c}{\scriptsize{(0.237)}} & \mc{1}{c}{\scriptsize{(0.355)}} & \mc{1}{c}{\scriptsize{(0.263)}} \\ 
    \midrule  

    \mc{2}{l}{\scriptsize{\% of Pos. TE ($H_0$: $\le$ 50\%)}} & \mc{1}{c}{\scriptsize{82}} & \mc{1}{c}{\scriptsize{91}} & \mc{1}{c}{\scriptsize{9}} & \mc{1}{c}{\scriptsize{64}} & \mc{1}{c}{\scriptsize{9}} & \mc{1}{c}{\scriptsize{82}} & \mc{1}{c}{\scriptsize{100}} & \mc{1}{c}{\scriptsize{82}} \\  

     &  & \mc{1}{c}{\scriptsize{(0.132)}} & \mc{1}{c}{\scriptsize{\textbf{(0.000)}}} & \mc{1}{c}{\scriptsize{(1.000)}} & \mc{1}{c}{\scriptsize{(0.329)}} & \mc{1}{c}{\scriptsize{(1.000)}} & \mc{1}{c}{\scriptsize{\textbf{(0.000)}}} & \mc{1}{c}{\scriptsize{\textbf{(0.000)}}} & \mc{1}{c}{\scriptsize{\textbf{(0.000)}}} \\  

    \mc{2}{l}{\scriptsize{\% of Pos. TE ($H_0$: $\le$ 10\% $|$ 10\% Significance)}} & \mc{1}{c}{\scriptsize{0}} & \mc{1}{c}{\scriptsize{0}} & \mc{1}{c}{\scriptsize{0}} & \mc{1}{c}{\scriptsize{0}} & \mc{1}{c}{\scriptsize{0}} & \mc{1}{c}{\scriptsize{0}} & \mc{1}{c}{\scriptsize{0}} & \mc{1}{c}{\scriptsize{9}} \\  

     &  & \mc{1}{c}{\scriptsize{(0.487)}} & \mc{1}{c}{\scriptsize{(1.000)}} & \mc{1}{c}{\scriptsize{(1.000)}} & \mc{1}{c}{\scriptsize{(1.000)}} & \mc{1}{c}{\scriptsize{(1.000)}} & \mc{1}{c}{\scriptsize{(0.579)}} & \mc{1}{c}{\scriptsize{(1.000)}} & \mc{1}{c}{\scriptsize{(0.276)}} \\  

  \bottomrule
  \end{tabular}
	\end{table} 

	\begin{table}[H]
     \caption{Treatment Effects on Relation with Spouse, Male Sample}
     \label{table:abccare_rslt_male_cat19}
	  \begin{tabular}{cccccccccc}
  \toprule

    \scriptsize{Variable} & \scriptsize{Age} & \scriptsize{(1)} & \scriptsize{(2)} & \scriptsize{(3)} & \scriptsize{(4)} & \scriptsize{(5)} & \scriptsize{(6)} & \scriptsize{(7)} & \scriptsize{(8)} \\ 
    \midrule  

    \mc{1}{l}{\scriptsize{No trouble with spouse family}} & \mc{1}{c}{\scriptsize{30}} & \mc{1}{c}{\scriptsize{-0.382}} & \mc{1}{c}{\scriptsize{-0.357}} & \mc{1}{c}{\scriptsize{-0.500}} & \mc{1}{c}{\scriptsize{-0.512}} & \mc{1}{c}{\scriptsize{-0.593}} & \mc{1}{c}{\scriptsize{-0.346}} & \mc{1}{c}{\scriptsize{-0.365}} & \mc{1}{c}{\scriptsize{-0.427}} \\  

     &  & \mc{1}{c}{\scriptsize{(1.000)}} & \mc{1}{c}{\scriptsize{(1.000)}} & \mc{1}{c}{\scriptsize{(0.947)}} & \mc{1}{c}{\scriptsize{(0.145)}} & \mc{1}{c}{\scriptsize{(0.947)}} & \mc{1}{c}{\scriptsize{(1.000)}} & \mc{1}{c}{\scriptsize{(0.974)}} & \mc{1}{c}{\scriptsize{(1.000)}} \\  

    \mc{1}{l}{\scriptsize{Get along well with spouse}} & \mc{1}{c}{\scriptsize{30}} & \mc{1}{c}{\scriptsize{0.099}} & \mc{1}{c}{\scriptsize{0.159}} & \mc{1}{c}{\scriptsize{0.021}} & \mc{1}{c}{\scriptsize{0.353}} & \mc{1}{c}{\scriptsize{-0.231}} & \mc{1}{c}{\scriptsize{0.149}} & \mc{1}{c}{\scriptsize{0.160}} & \mc{1}{c}{\scriptsize{-0.186}} \\  

     &  & \mc{1}{c}{\scriptsize{(0.237)}} & \mc{1}{c}{\scriptsize{(0.184)}} & \mc{1}{c}{\scriptsize{(0.382)}} & \mc{1}{c}{\scriptsize{(0.145)}} & \mc{1}{c}{\scriptsize{(0.697)}} & \mc{1}{c}{\scriptsize{(0.184)}} & \mc{1}{c}{\scriptsize{(0.184)}} & \mc{1}{c}{\scriptsize{(0.855)}} \\  

    \mc{1}{l}{\scriptsize{No disagreement on living arrangement}} & \mc{1}{c}{\scriptsize{30}} & \mc{1}{c}{\scriptsize{0.217}} & \mc{1}{c}{\scriptsize{0.263}} & \mc{1}{c}{\scriptsize{0.688}} & \mc{1}{c}{\scriptsize{0.524}} & \mc{1}{c}{\scriptsize{0.716}} & \mc{1}{c}{\scriptsize{0.149}} & \mc{1}{c}{\scriptsize{0.254}} & \mc{1}{c}{\scriptsize{0.171}} \\  

     &  & \mc{1}{c}{\scriptsize{(0.118)}} & \mc{1}{c}{\scriptsize{(0.105)}} & \mc{1}{c}{\scriptsize{\textbf{(0.000)}}} & \mc{1}{c}{\scriptsize{(0.105)}} & \mc{1}{c}{\scriptsize{\textbf{(0.000)}}} & \mc{1}{c}{\scriptsize{(0.197)}} & \mc{1}{c}{\scriptsize{(0.158)}} & \mc{1}{c}{\scriptsize{(0.197)}} \\  

  \bottomrule
  \end{tabular}
	\end{table} 

	\begin{table}[H]
     \caption{Treatment Effects on Spouse Characteristics, Male Sample}
     \label{table:abccare_rslt_male_cat20}
	  \begin{tabular}{cccccccccc}
  \toprule

    \scriptsize{Variable} & \scriptsize{Age} & \scriptsize{(1)} & \scriptsize{(2)} & \scriptsize{(3)} & \scriptsize{(4)} & \scriptsize{(5)} & \scriptsize{(6)} & \scriptsize{(7)} & \scriptsize{(8)} \\ 
    \midrule  

    \mc{1}{l}{\scriptsize{Body Mass Index (BMI)}} & \mc{1}{c}{\scriptsize{4}} & \mc{1}{c}{\scriptsize{0.157}} & \mc{1}{c}{\scriptsize{0.140}} & \mc{1}{c}{\scriptsize{0.020}} & \mc{1}{c}{\scriptsize{0.564}} & \mc{1}{c}{\scriptsize{-0.152}} & \mc{1}{c}{\scriptsize{0.191}} & \mc{1}{c}{\scriptsize{0.174}} & \mc{1}{c}{\scriptsize{0.026}} \\  

     &  & \mc{1}{c}{\scriptsize{(0.684)}} & \mc{1}{c}{\scriptsize{(0.658)}} & \mc{1}{c}{\scriptsize{(0.500)}} & \mc{1}{c}{\scriptsize{(0.763)}} & \mc{1}{c}{\scriptsize{(0.342)}} & \mc{1}{c}{\scriptsize{(0.724)}} & \mc{1}{c}{\scriptsize{(0.671)}} & \mc{1}{c}{\scriptsize{(0.579)}} \\  

     & \mc{1}{c}{\scriptsize{3}} & \mc{1}{c}{\scriptsize{-0.498}} & \mc{1}{c}{\scriptsize{-0.566}} & \mc{1}{c}{\scriptsize{-0.810}} & \mc{1}{c}{\scriptsize{-0.726}} & \mc{1}{c}{\scriptsize{-0.946}} & \mc{1}{c}{\scriptsize{-0.373}} & \mc{1}{c}{\scriptsize{-0.521}} & \mc{1}{c}{\scriptsize{-0.434}} \\  

     &  & \mc{1}{c}{\scriptsize{(0.105)}} & \mc{1}{c}{\scriptsize{(0.145)}} & \mc{1}{c}{\scriptsize{\textbf{(0.092)}}} & \mc{1}{c}{\scriptsize{(0.329)}} & \mc{1}{c}{\scriptsize{(0.118)}} & \mc{1}{c}{\scriptsize{(0.158)}} & \mc{1}{c}{\scriptsize{(0.132)}} & \mc{1}{c}{\scriptsize{(0.184)}} \\  

     & \mc{1}{c}{\scriptsize{0.5}} & \mc{1}{c}{\scriptsize{-0.535}} & \mc{1}{c}{\scriptsize{-0.287}} & \mc{1}{c}{\scriptsize{-0.748}} & \mc{1}{c}{\scriptsize{-0.471}} & \mc{1}{c}{\scriptsize{-0.775}} & \mc{1}{c}{\scriptsize{-0.299}} & \mc{1}{c}{\scriptsize{-0.088}} & \mc{1}{c}{\scriptsize{-0.336}} \\  

     &  & \mc{1}{c}{\scriptsize{(0.105)}} & \mc{1}{c}{\scriptsize{(0.250)}} & \mc{1}{c}{\scriptsize{\textbf{(0.026)}}} & \mc{1}{c}{\scriptsize{(0.184)}} & \mc{1}{c}{\scriptsize{\textbf{(0.039)}}} & \mc{1}{c}{\scriptsize{(0.276)}} & \mc{1}{c}{\scriptsize{(0.513)}} & \mc{1}{c}{\scriptsize{(0.250)}} \\  

     & \mc{1}{c}{\scriptsize{1.5}} & \mc{1}{c}{\scriptsize{-0.833}} & \mc{1}{c}{\scriptsize{-0.717}} & \mc{1}{c}{\scriptsize{-1.365}} & \mc{1}{c}{\scriptsize{-1.156}} & \mc{1}{c}{\scriptsize{-1.643}} & \mc{1}{c}{\scriptsize{-0.689}} & \mc{1}{c}{\scriptsize{-0.525}} & \mc{1}{c}{\scriptsize{-0.946}} \\  

     &  & \mc{1}{c}{\scriptsize{\textbf{(0.026)}}} & \mc{1}{c}{\scriptsize{\textbf{(0.092)}}} & \mc{1}{c}{\scriptsize{\textbf{(0.013)}}} & \mc{1}{c}{\scriptsize{\textbf{(0.066)}}} & \mc{1}{c}{\scriptsize{\textbf{(0.000)}}} & \mc{1}{c}{\scriptsize{\textbf{(0.066)}}} & \mc{1}{c}{\scriptsize{(0.184)}} & \mc{1}{c}{\scriptsize{\textbf{(0.000)}}} \\  

     & \mc{1}{c}{\scriptsize{0}} & \mc{1}{c}{\scriptsize{0.172}} & \mc{1}{c}{\scriptsize{2.648}} & \mc{1}{c}{\scriptsize{-1.470}} & \mc{1}{c}{\scriptsize{-1.316}} & \mc{1}{c}{\scriptsize{-0.953}} & \mc{1}{c}{\scriptsize{0.761}} & \mc{1}{c}{\scriptsize{3.009}} & \mc{1}{c}{\scriptsize{1.091}} \\  

     &  & \mc{1}{c}{\scriptsize{(0.526)}} & \mc{1}{c}{\scriptsize{(0.947)}} & \mc{1}{c}{\scriptsize{(0.118)}} & \mc{1}{c}{\scriptsize{(0.250)}} & \mc{1}{c}{\scriptsize{(0.211)}} & \mc{1}{c}{\scriptsize{(0.724)}} & \mc{1}{c}{\scriptsize{(0.921)}} & \mc{1}{c}{\scriptsize{(0.776)}} \\  

     & \mc{1}{c}{\scriptsize{0.75}} & \mc{1}{c}{\scriptsize{-1.170}} & \mc{1}{c}{\scriptsize{-1.145}} & \mc{1}{c}{\scriptsize{-1.444}} & \mc{1}{c}{\scriptsize{-1.458}} & \mc{1}{c}{\scriptsize{-1.012}} & \mc{1}{c}{\scriptsize{-1.150}} & \mc{1}{c}{\scriptsize{-1.262}} & \mc{1}{c}{\scriptsize{-1.001}} \\  

     &  & \mc{1}{c}{\scriptsize{\textbf{(0.013)}}} & \mc{1}{c}{\scriptsize{\textbf{(0.039)}}} & \mc{1}{c}{\scriptsize{\textbf{(0.039)}}} & \mc{1}{c}{\scriptsize{\textbf{(0.026)}}} & \mc{1}{c}{\scriptsize{(0.105)}} & \mc{1}{c}{\scriptsize{\textbf{(0.039)}}} & \mc{1}{c}{\scriptsize{(0.118)}} & \mc{1}{c}{\scriptsize{\textbf{(0.092)}}} \\  

     & \mc{1}{c}{\scriptsize{1}} & \mc{1}{c}{\scriptsize{-0.915}} & \mc{1}{c}{\scriptsize{-1.117}} & \mc{1}{c}{\scriptsize{-1.208}} & \mc{1}{c}{\scriptsize{-1.335}} & \mc{1}{c}{\scriptsize{-1.515}} & \mc{1}{c}{\scriptsize{-0.889}} & \mc{1}{c}{\scriptsize{-1.056}} & \mc{1}{c}{\scriptsize{-1.206}} \\  

     &  & \mc{1}{c}{\scriptsize{\textbf{(0.013)}}} & \mc{1}{c}{\scriptsize{\textbf{(0.000)}}} & \mc{1}{c}{\scriptsize{\textbf{(0.053)}}} & \mc{1}{c}{\scriptsize{\textbf{(0.079)}}} & \mc{1}{c}{\scriptsize{\textbf{(0.013)}}} & \mc{1}{c}{\scriptsize{\textbf{(0.026)}}} & \mc{1}{c}{\scriptsize{\textbf{(0.000)}}} & \mc{1}{c}{\scriptsize{\textbf{(0.000)}}} \\  

     & \mc{1}{c}{\scriptsize{8}} & \mc{1}{c}{\scriptsize{0.096}} & \mc{1}{c}{\scriptsize{-0.101}} & \mc{1}{c}{\scriptsize{0.201}} & \mc{1}{c}{\scriptsize{0.380}} & \mc{1}{c}{\scriptsize{0.034}} & \mc{1}{c}{\scriptsize{-0.032}} & \mc{1}{c}{\scriptsize{-0.152}} & \mc{1}{c}{\scriptsize{-0.277}} \\  

     &  & \mc{1}{c}{\scriptsize{(0.566)}} & \mc{1}{c}{\scriptsize{(0.447)}} & \mc{1}{c}{\scriptsize{(0.605)}} & \mc{1}{c}{\scriptsize{(0.645)}} & \mc{1}{c}{\scriptsize{(0.500)}} & \mc{1}{c}{\scriptsize{(0.421)}} & \mc{1}{c}{\scriptsize{(0.421)}} & \mc{1}{c}{\scriptsize{(0.303)}} \\  

     & \mc{1}{c}{\scriptsize{5}} & \mc{1}{c}{\scriptsize{-0.085}} & \mc{1}{c}{\scriptsize{-0.136}} & \mc{1}{c}{\scriptsize{-0.153}} & \mc{1}{c}{\scriptsize{0.406}} & \mc{1}{c}{\scriptsize{-0.198}} & \mc{1}{c}{\scriptsize{-0.065}} & \mc{1}{c}{\scriptsize{-0.054}} & \mc{1}{c}{\scriptsize{-0.139}} \\  

     &  & \mc{1}{c}{\scriptsize{(0.421)}} & \mc{1}{c}{\scriptsize{(0.421)}} & \mc{1}{c}{\scriptsize{(0.447)}} & \mc{1}{c}{\scriptsize{(0.711)}} & \mc{1}{c}{\scriptsize{(0.395)}} & \mc{1}{c}{\scriptsize{(0.434)}} & \mc{1}{c}{\scriptsize{(0.487)}} & \mc{1}{c}{\scriptsize{(0.434)}} \\  

     & \mc{1}{c}{\scriptsize{2.5}} & \mc{1}{c}{\scriptsize{-0.408}} & \mc{1}{c}{\scriptsize{1.371}} &  &  &  & \mc{1}{c}{\scriptsize{-0.408}} & \mc{1}{c}{\scriptsize{1.371}} & \mc{1}{c}{\scriptsize{-0.159}} \\  

     &  & \mc{1}{c}{\scriptsize{(0.211)}} & \mc{1}{c}{\scriptsize{(0.724)}} &  &  &  & \mc{1}{c}{\scriptsize{(0.211)}} & \mc{1}{c}{\scriptsize{(0.724)}} & \mc{1}{c}{\scriptsize{(0.368)}} \\  

     & \mc{1}{c}{\scriptsize{0.25}} & \mc{1}{c}{\scriptsize{-0.799}} & \mc{1}{c}{\scriptsize{-0.346}} & \mc{1}{c}{\scriptsize{-1.343}} & \mc{1}{c}{\scriptsize{-1.229}} & \mc{1}{c}{\scriptsize{-1.166}} & \mc{1}{c}{\scriptsize{-0.775}} & \mc{1}{c}{\scriptsize{-0.299}} & \mc{1}{c}{\scriptsize{-0.592}} \\  

     &  & \mc{1}{c}{\scriptsize{\textbf{(0.039)}}} & \mc{1}{c}{\scriptsize{(0.237)}} & \mc{1}{c}{\scriptsize{\textbf{(0.000)}}} & \mc{1}{c}{\scriptsize{\textbf{(0.039)}}} & \mc{1}{c}{\scriptsize{\textbf{(0.000)}}} & \mc{1}{c}{\scriptsize{\textbf{(0.092)}}} & \mc{1}{c}{\scriptsize{(0.316)}} & \mc{1}{c}{\scriptsize{(0.158)}} \\  

     & \mc{1}{c}{\scriptsize{2}} & \mc{1}{c}{\scriptsize{-0.439}} & \mc{1}{c}{\scriptsize{-0.534}} & \mc{1}{c}{\scriptsize{-0.580}} & \mc{1}{c}{\scriptsize{-0.300}} & \mc{1}{c}{\scriptsize{-0.743}} & \mc{1}{c}{\scriptsize{-0.404}} & \mc{1}{c}{\scriptsize{-0.480}} & \mc{1}{c}{\scriptsize{-0.570}} \\  

     &  & \mc{1}{c}{\scriptsize{(0.118)}} & \mc{1}{c}{\scriptsize{(0.158)}} & \mc{1}{c}{\scriptsize{\textbf{(0.053)}}} & \mc{1}{c}{\scriptsize{(0.434)}} & \mc{1}{c}{\scriptsize{\textbf{(0.039)}}} & \mc{1}{c}{\scriptsize{(0.145)}} & \mc{1}{c}{\scriptsize{(0.197)}} & \mc{1}{c}{\scriptsize{\textbf{(0.079)}}} \\  

  \bottomrule
  \end{tabular}
	\end{table} 

	\begin{table}[H]
     \caption{Treatment Effects on Subject Home and Property, Male Sample}
     \label{table:abccare_rslt_male_cat21}
	  \begin{tabular}{cccccccccc}
  \toprule

    \scriptsize{Variable} & \scriptsize{Age} & \scriptsize{(1)} & \scriptsize{(2)} & \scriptsize{(3)} & \scriptsize{(4)} & \scriptsize{(5)} & \scriptsize{(6)} & \scriptsize{(7)} & \scriptsize{(8)} \\ 
    \midrule  

    \mc{1}{l}{\scriptsize{Room density (room/people)}} & \mc{1}{c}{\scriptsize{30}} & \mc{1}{c}{\scriptsize{0.019}} & \mc{1}{c}{\scriptsize{-0.158}} & \mc{1}{c}{\scriptsize{-0.554}} & \mc{1}{c}{\scriptsize{-1.291}} & \mc{1}{c}{\scriptsize{-0.661}} & \mc{1}{c}{\scriptsize{0.171}} & \mc{1}{c}{\scriptsize{0.117}} & \mc{1}{c}{\scriptsize{0.080}} \\  

     &  & \mc{1}{c}{\scriptsize{(0.487)}} & \mc{1}{c}{\scriptsize{(0.684)}} & \mc{1}{c}{\scriptsize{(0.776)}} & \mc{1}{c}{\scriptsize{(0.895)}} & \mc{1}{c}{\scriptsize{(0.803)}} & \mc{1}{c}{\scriptsize{(0.276)}} & \mc{1}{c}{\scriptsize{(0.421)}} & \mc{1}{c}{\scriptsize{(0.395)}} \\  

    \mc{1}{l}{\scriptsize{Own computers}} & \mc{1}{c}{\scriptsize{30}} & \mc{1}{c}{\scriptsize{0.037}} & \mc{1}{c}{\scriptsize{0.006}} &  & \mc{1}{c}{\scriptsize{-0.070}} & \mc{1}{c}{\scriptsize{-0.024}} & \mc{1}{c}{\scriptsize{0.062}} & \mc{1}{c}{\scriptsize{0.031}} & \mc{1}{c}{\scriptsize{0.053}} \\  

     &  & \mc{1}{c}{\scriptsize{(0.368)}} & \mc{1}{c}{\scriptsize{(0.461)}} &  & \mc{1}{c}{\scriptsize{(0.539)}} & \mc{1}{c}{\scriptsize{(0.500)}} & \mc{1}{c}{\scriptsize{(0.303)}} & \mc{1}{c}{\scriptsize{(0.434)}} & \mc{1}{c}{\scriptsize{(0.276)}} \\  

    \mc{1}{l}{\scriptsize{Own cars}} & \mc{1}{c}{\scriptsize{30}} & \mc{1}{c}{\scriptsize{0.026}} & \mc{1}{c}{\scriptsize{0.028}} & \mc{1}{c}{\scriptsize{0.229}} & \mc{1}{c}{\scriptsize{0.129}} & \mc{1}{c}{\scriptsize{0.257}} & \mc{1}{c}{\scriptsize{-0.026}} & \mc{1}{c}{\scriptsize{0.018}} & \mc{1}{c}{\scriptsize{0.011}} \\  

     &  & \mc{1}{c}{\scriptsize{(0.368)}} & \mc{1}{c}{\scriptsize{(0.355)}} & \mc{1}{c}{\scriptsize{(0.145)}} & \mc{1}{c}{\scriptsize{(0.289)}} & \mc{1}{c}{\scriptsize{(0.105)}} & \mc{1}{c}{\scriptsize{(0.592)}} & \mc{1}{c}{\scriptsize{(0.434)}} & \mc{1}{c}{\scriptsize{(0.474)}} \\  

    \mc{1}{l}{\scriptsize{Own residences}} & \mc{1}{c}{\scriptsize{30}} & \mc{1}{c}{\scriptsize{-0.058}} & \mc{1}{c}{\scriptsize{-0.066}} & \mc{1}{c}{\scriptsize{-0.229}} & \mc{1}{c}{\scriptsize{-0.231}} & \mc{1}{c}{\scriptsize{-0.255}} & \mc{1}{c}{\scriptsize{0.026}} & \mc{1}{c}{\scriptsize{-0.029}} & \mc{1}{c}{\scriptsize{-0.008}} \\  

     &  & \mc{1}{c}{\scriptsize{(0.697)}} & \mc{1}{c}{\scriptsize{(0.684)}} & \mc{1}{c}{\scriptsize{(0.816)}} & \mc{1}{c}{\scriptsize{(0.829)}} & \mc{1}{c}{\scriptsize{(0.842)}} & \mc{1}{c}{\scriptsize{(0.421)}} & \mc{1}{c}{\scriptsize{(0.579)}} & \mc{1}{c}{\scriptsize{(0.566)}} \\  

  \bottomrule
  \end{tabular}
	\end{table} 

	\begin{table}[H]
     \caption{Treatment Effects on Education, Male Sample}
     \label{table:abccare_rslt_male_cat22}
	  \begin{tabular}{cccccccccc}
  \toprule

    \scriptsize{Variable} & \scriptsize{Age} & \scriptsize{(1)} & \scriptsize{(2)} & \scriptsize{(3)} & \scriptsize{(4)} & \scriptsize{(5)} & \scriptsize{(6)} & \scriptsize{(7)} & \scriptsize{(8)} \\ 
    \midrule  

    \mc{1}{l}{\scriptsize{Years of Edu.}} & \mc{1}{c}{\scriptsize{30}} & \mc{1}{c}{\scriptsize{0.525}} & \mc{1}{c}{\scriptsize{0.708}} & \mc{1}{c}{\scriptsize{0.857}} & \mc{1}{c}{\scriptsize{1.302}} & \mc{1}{c}{\scriptsize{0.791}} & \mc{1}{c}{\scriptsize{0.385}} & \mc{1}{c}{\scriptsize{0.540}} & \mc{1}{c}{\scriptsize{0.347}} \\  

     &  & \mc{1}{c}{\scriptsize{\textbf{(0.079)}}} & \mc{1}{c}{\scriptsize{\textbf{(0.079)}}} & \mc{1}{c}{\scriptsize{(0.132)}} & \mc{1}{c}{\scriptsize{\textbf{(0.079)}}} & \mc{1}{c}{\scriptsize{(0.184)}} & \mc{1}{c}{\scriptsize{(0.171)}} & \mc{1}{c}{\scriptsize{(0.118)}} & \mc{1}{c}{\scriptsize{(0.263)}} \\  

    \mc{1}{l}{\scriptsize{Graduated High School}} & \mc{1}{c}{\scriptsize{30}} & \mc{1}{c}{\scriptsize{0.073}} & \mc{1}{c}{\scriptsize{0.109}} & \mc{1}{c}{\scriptsize{0.114}} & \mc{1}{c}{\scriptsize{0.113}} & \mc{1}{c}{\scriptsize{0.085}} & \mc{1}{c}{\scriptsize{0.077}} & \mc{1}{c}{\scriptsize{0.102}} & \mc{1}{c}{\scriptsize{0.064}} \\  

     &  & \mc{1}{c}{\scriptsize{(0.250)}} & \mc{1}{c}{\scriptsize{(0.250)}} & \mc{1}{c}{\scriptsize{(0.316)}} & \mc{1}{c}{\scriptsize{(0.342)}} & \mc{1}{c}{\scriptsize{(0.368)}} & \mc{1}{c}{\scriptsize{(0.224)}} & \mc{1}{c}{\scriptsize{(0.289)}} & \mc{1}{c}{\scriptsize{(0.316)}} \\  

    \mc{1}{l}{\scriptsize{Graduated 4-year College}} & \mc{1}{c}{\scriptsize{30}} & \mc{1}{c}{\scriptsize{0.170}} & \mc{1}{c}{\scriptsize{0.169}} & \mc{1}{c}{\scriptsize{0.124}} & \mc{1}{c}{\scriptsize{0.232}} & \mc{1}{c}{\scriptsize{0.102}} & \mc{1}{c}{\scriptsize{0.179}} & \mc{1}{c}{\scriptsize{0.159}} & \mc{1}{c}{\scriptsize{0.143}} \\  

     &  & \mc{1}{c}{\scriptsize{\textbf{(0.066)}}} & \mc{1}{c}{\scriptsize{(0.118)}} & \mc{1}{c}{\scriptsize{(0.303)}} & \mc{1}{c}{\scriptsize{(0.211)}} & \mc{1}{c}{\scriptsize{(0.342)}} & \mc{1}{c}{\scriptsize{\textbf{(0.013)}}} & \mc{1}{c}{\scriptsize{(0.118)}} & \mc{1}{c}{\scriptsize{(0.105)}} \\  

    \mc{1}{l}{\scriptsize{Attended Voc./Tech./Com. College}} & \mc{1}{c}{\scriptsize{30}} & \mc{1}{c}{\scriptsize{-0.099}} & \mc{1}{c}{\scriptsize{-0.124}} & \mc{1}{c}{\scriptsize{0.086}} & \mc{1}{c}{\scriptsize{0.258}} & \mc{1}{c}{\scriptsize{0.024}} & \mc{1}{c}{\scriptsize{-0.138}} & \mc{1}{c}{\scriptsize{-0.231}} & \mc{1}{c}{\scriptsize{-0.233}} \\  

     &  & \mc{1}{c}{\scriptsize{(0.803)}} & \mc{1}{c}{\scriptsize{(0.803)}} & \mc{1}{c}{\scriptsize{(0.382)}} & \mc{1}{c}{\scriptsize{(0.132)}} & \mc{1}{c}{\scriptsize{(0.434)}} & \mc{1}{c}{\scriptsize{(0.789)}} & \mc{1}{c}{\scriptsize{(0.921)}} & \mc{1}{c}{\scriptsize{(0.934)}} \\  

  \bottomrule
  \end{tabular}
	\end{table} 

	\begin{table}[H]
     \caption{Treatment Effects on Subject Employment and Income, Male Sample}
     \label{table:abccare_rslt_male_cat23}
	  \begin{tabular}{cccccccccc}
  \toprule

    \scriptsize{Variable} & \scriptsize{Age} & \scriptsize{(1)} & \scriptsize{(2)} & \scriptsize{(3)} & \scriptsize{(4)} & \scriptsize{(5)} & \scriptsize{(6)} & \scriptsize{(7)} & \scriptsize{(8)} \\ 
    \midrule  

    \mc{1}{l}{\scriptsize{Father at Home}} & \mc{1}{c}{\scriptsize{3}} & \mc{1}{c}{\scriptsize{-0.076}} & \mc{1}{c}{\scriptsize{0.007}} & \mc{1}{c}{\scriptsize{-0.243}} & \mc{1}{c}{\scriptsize{0.026}} & \mc{1}{c}{\scriptsize{-0.197}} & \mc{1}{c}{\scriptsize{-0.029}} & \mc{1}{c}{\scriptsize{0.010}} & \mc{1}{c}{\scriptsize{0.070}} \\  

     &  & \mc{1}{c}{\scriptsize{(0.737)}} & \mc{1}{c}{\scriptsize{(0.447)}} & \mc{1}{c}{\scriptsize{(0.868)}} & \mc{1}{c}{\scriptsize{(0.461)}} & \mc{1}{c}{\scriptsize{(0.803)}} & \mc{1}{c}{\scriptsize{(0.592)}} & \mc{1}{c}{\scriptsize{(0.487)}} & \mc{1}{c}{\scriptsize{(0.276)}} \\  

     & \mc{1}{c}{\scriptsize{2}} & \mc{1}{c}{\scriptsize{-0.018}} & \mc{1}{c}{\scriptsize{0.160}} & \mc{1}{c}{\scriptsize{-0.282}} & \mc{1}{c}{\scriptsize{0.111}} & \mc{1}{c}{\scriptsize{-0.223}} & \mc{1}{c}{\scriptsize{0.057}} & \mc{1}{c}{\scriptsize{0.187}} & \mc{1}{c}{\scriptsize{0.171}} \\  

     &  & \mc{1}{c}{\scriptsize{(0.553)}} & \mc{1}{c}{\scriptsize{\textbf{(0.066)}}} & \mc{1}{c}{\scriptsize{(0.868)}} & \mc{1}{c}{\scriptsize{(0.368)}} & \mc{1}{c}{\scriptsize{(0.803)}} & \mc{1}{c}{\scriptsize{(0.355)}} & \mc{1}{c}{\scriptsize{\textbf{(0.039)}}} & \mc{1}{c}{\scriptsize{\textbf{(0.066)}}} \\  

     & \mc{1}{c}{\scriptsize{8}} & \mc{1}{c}{\scriptsize{0.037}} & \mc{1}{c}{\scriptsize{0.038}} & \mc{1}{c}{\scriptsize{-0.177}} & \mc{1}{c}{\scriptsize{0.012}} & \mc{1}{c}{\scriptsize{-0.295}} & \mc{1}{c}{\scriptsize{0.123}} & \mc{1}{c}{\scriptsize{0.062}} & \mc{1}{c}{\scriptsize{0.128}} \\  

     &  & \mc{1}{c}{\scriptsize{(0.355)}} & \mc{1}{c}{\scriptsize{(0.382)}} & \mc{1}{c}{\scriptsize{(0.842)}} & \mc{1}{c}{\scriptsize{(0.513)}} & \mc{1}{c}{\scriptsize{(0.882)}} & \mc{1}{c}{\scriptsize{(0.132)}} & \mc{1}{c}{\scriptsize{(0.316)}} & \mc{1}{c}{\scriptsize{(0.145)}} \\  

     & \mc{1}{c}{\scriptsize{5}} & \mc{1}{c}{\scriptsize{-0.057}} & \mc{1}{c}{\scriptsize{0.051}} & \mc{1}{c}{\scriptsize{-0.429}} & \mc{1}{c}{\scriptsize{-0.081}} & \mc{1}{c}{\scriptsize{-0.376}} & \mc{1}{c}{\scriptsize{0.036}} & \mc{1}{c}{\scriptsize{0.089}} & \mc{1}{c}{\scriptsize{0.142}} \\  

     &  & \mc{1}{c}{\scriptsize{(0.697)}} & \mc{1}{c}{\scriptsize{(0.303)}} & \mc{1}{c}{\scriptsize{(0.974)}} & \mc{1}{c}{\scriptsize{(0.724)}} & \mc{1}{c}{\scriptsize{(0.974)}} & \mc{1}{c}{\scriptsize{(0.395)}} & \mc{1}{c}{\scriptsize{(0.211)}} & \mc{1}{c}{\scriptsize{(0.132)}} \\  

     & \mc{1}{c}{\scriptsize{4}} & \mc{1}{c}{\scriptsize{-0.075}} & \mc{1}{c}{\scriptsize{0.008}} & \mc{1}{c}{\scriptsize{-0.339}} & \mc{1}{c}{\scriptsize{-0.050}} & \mc{1}{c}{\scriptsize{-0.287}} &  & \mc{1}{c}{\scriptsize{0.036}} & \mc{1}{c}{\scriptsize{0.103}} \\  

     &  & \mc{1}{c}{\scriptsize{(0.724)}} & \mc{1}{c}{\scriptsize{(0.513)}} & \mc{1}{c}{\scriptsize{(0.961)}} & \mc{1}{c}{\scriptsize{(0.658)}} & \mc{1}{c}{\scriptsize{(0.947)}} &  & \mc{1}{c}{\scriptsize{(0.395)}} & \mc{1}{c}{\scriptsize{(0.224)}} \\  

  \bottomrule
  \end{tabular}
	\end{table} 

	\begin{table}[H]
     \caption{Treatment Effects on Job Attitude, Male Sample}
     \label{table:abccare_rslt_male_cat24}
	  \begin{tabular}{cccccccccc}
  \toprule

    \scriptsize{Variable} & \scriptsize{Age} & \scriptsize{(1)} & \scriptsize{(2)} & \scriptsize{(3)} & \scriptsize{(4)} & \scriptsize{(5)} & \scriptsize{(6)} & \scriptsize{(7)} & \scriptsize{(8)} \\ 
    \midrule  

    \mc{1}{l}{\scriptsize{Satisfied with working situation}} & \mc{1}{c}{\scriptsize{30}} & \mc{1}{c}{\scriptsize{0.191}} & \mc{1}{c}{\scriptsize{0.166}} & \mc{1}{c}{\scriptsize{0.205}} & \mc{1}{c}{\scriptsize{0.116}} & \mc{1}{c}{\scriptsize{0.220}} & \mc{1}{c}{\scriptsize{0.226}} & \mc{1}{c}{\scriptsize{0.221}} & \mc{1}{c}{\scriptsize{0.293}} \\  

     &  & \mc{1}{c}{\scriptsize{(0.105)}} & \mc{1}{c}{\scriptsize{(0.171)}} & \mc{1}{c}{\scriptsize{(0.237)}} & \mc{1}{c}{\scriptsize{(0.342)}} & \mc{1}{c}{\scriptsize{(0.158)}} & \mc{1}{c}{\scriptsize{\textbf{(0.092)}}} & \mc{1}{c}{\scriptsize{(0.132)}} & \mc{1}{c}{\scriptsize{\textbf{(0.053)}}} \\  

    \mc{1}{l}{\scriptsize{Do work well}} & \mc{1}{c}{\scriptsize{30}} & \mc{1}{c}{\scriptsize{0.015}} & \mc{1}{c}{\scriptsize{0.043}} & \mc{1}{c}{\scriptsize{-0.115}} & \mc{1}{c}{\scriptsize{-0.096}} & \mc{1}{c}{\scriptsize{-0.116}} & \mc{1}{c}{\scriptsize{0.072}} & \mc{1}{c}{\scriptsize{0.014}} & \mc{1}{c}{\scriptsize{0.094}} \\  

     &  & \mc{1}{c}{\scriptsize{(0.461)}} & \mc{1}{c}{\scriptsize{(0.263)}} & \mc{1}{c}{\scriptsize{(0.882)}} & \mc{1}{c}{\scriptsize{(0.750)}} & \mc{1}{c}{\scriptsize{(0.895)}} & \mc{1}{c}{\scriptsize{(0.250)}} & \mc{1}{c}{\scriptsize{(0.421)}} & \mc{1}{c}{\scriptsize{(0.184)}} \\  

    \mc{1}{l}{\scriptsize{Not worry too much about work}} & \mc{1}{c}{\scriptsize{30}} & \mc{1}{c}{\scriptsize{-0.096}} & \mc{1}{c}{\scriptsize{-0.051}} & \mc{1}{c}{\scriptsize{-0.013}} & \mc{1}{c}{\scriptsize{0.330}} & \mc{1}{c}{\scriptsize{0.007}} & \mc{1}{c}{\scriptsize{-0.111}} & \mc{1}{c}{\scriptsize{-0.013}} & \mc{1}{c}{\scriptsize{-0.102}} \\  

     &  & \mc{1}{c}{\scriptsize{(0.711)}} & \mc{1}{c}{\scriptsize{(0.579)}} & \mc{1}{c}{\scriptsize{(0.487)}} & \mc{1}{c}{\scriptsize{(0.132)}} & \mc{1}{c}{\scriptsize{(0.434)}} & \mc{1}{c}{\scriptsize{(0.763)}} & \mc{1}{c}{\scriptsize{(0.539)}} & \mc{1}{c}{\scriptsize{(0.750)}} \\  

    \mc{1}{l}{\scriptsize{Work well with others}} & \mc{1}{c}{\scriptsize{30}} & \mc{1}{c}{\scriptsize{-0.023}} & \mc{1}{c}{\scriptsize{-0.160}} & \mc{1}{c}{\scriptsize{0.179}} & \mc{1}{c}{\scriptsize{0.192}} & \mc{1}{c}{\scriptsize{0.059}} & \mc{1}{c}{\scriptsize{-0.091}} & \mc{1}{c}{\scriptsize{-0.176}} & \mc{1}{c}{\scriptsize{-0.206}} \\  

     &  & \mc{1}{c}{\scriptsize{(0.618)}} & \mc{1}{c}{\scriptsize{(0.934)}} & \mc{1}{c}{\scriptsize{(0.197)}} & \mc{1}{c}{\scriptsize{(0.237)}} & \mc{1}{c}{\scriptsize{(0.355)}} & \mc{1}{c}{\scriptsize{(0.882)}} & \mc{1}{c}{\scriptsize{(0.895)}} & \mc{1}{c}{\scriptsize{(0.961)}} \\  

    \mc{1}{l}{\scriptsize{Don't do things that cause to lose job}} & \mc{1}{c}{\scriptsize{30}} & \mc{1}{c}{\scriptsize{0.054}} & \mc{1}{c}{\scriptsize{0.024}} & \mc{1}{c}{\scriptsize{-0.154}} & \mc{1}{c}{\scriptsize{-0.096}} & \mc{1}{c}{\scriptsize{-0.229}} & \mc{1}{c}{\scriptsize{0.081}} & \mc{1}{c}{\scriptsize{-0.043}} & \mc{1}{c}{\scriptsize{-0.006}} \\  

     &  & \mc{1}{c}{\scriptsize{(0.289)}} & \mc{1}{c}{\scriptsize{(0.368)}} & \mc{1}{c}{\scriptsize{(0.947)}} & \mc{1}{c}{\scriptsize{(0.816)}} & \mc{1}{c}{\scriptsize{(0.987)}} & \mc{1}{c}{\scriptsize{(0.276)}} & \mc{1}{c}{\scriptsize{(0.605)}} & \mc{1}{c}{\scriptsize{(0.500)}} \\  

    \mc{1}{l}{\scriptsize{No trouble finishing work}} & \mc{1}{c}{\scriptsize{30}} & \mc{1}{c}{\scriptsize{-0.110}} & \mc{1}{c}{\scriptsize{-0.220}} & \mc{1}{c}{\scriptsize{0.013}} & \mc{1}{c}{\scriptsize{-0.153}} & \mc{1}{c}{\scriptsize{-0.060}} & \mc{1}{c}{\scriptsize{-0.154}} & \mc{1}{c}{\scriptsize{-0.171}} & \mc{1}{c}{\scriptsize{-0.215}} \\  

     &  & \mc{1}{c}{\scriptsize{(0.921)}} & \mc{1}{c}{\scriptsize{(0.921)}} & \mc{1}{c}{\scriptsize{(0.421)}} & \mc{1}{c}{\scriptsize{(0.724)}} & \mc{1}{c}{\scriptsize{(0.579)}} & \mc{1}{c}{\scriptsize{(0.934)}} & \mc{1}{c}{\scriptsize{(0.895)}} & \mc{1}{c}{\scriptsize{(0.947)}} \\  

    \mc{1}{l}{\scriptsize{Job not too stressful}} & \mc{1}{c}{\scriptsize{30}} & \mc{1}{c}{\scriptsize{-0.061}} & \mc{1}{c}{\scriptsize{-0.001}} & \mc{1}{c}{\scriptsize{-0.269}} & \mc{1}{c}{\scriptsize{-0.110}} & \mc{1}{c}{\scriptsize{-0.199}} & \mc{1}{c}{\scriptsize{0.025}} & \mc{1}{c}{\scriptsize{-0.007}} & \mc{1}{c}{\scriptsize{0.048}} \\  

     &  & \mc{1}{c}{\scriptsize{(0.605)}} & \mc{1}{c}{\scriptsize{(0.513)}} & \mc{1}{c}{\scriptsize{(1.000)}} & \mc{1}{c}{\scriptsize{(0.776)}} & \mc{1}{c}{\scriptsize{(0.987)}} & \mc{1}{c}{\scriptsize{(0.474)}} & \mc{1}{c}{\scriptsize{(0.408)}} & \mc{1}{c}{\scriptsize{(0.342)}} \\  

    \mc{1}{l}{\scriptsize{Don't stay away from job}} & \mc{1}{c}{\scriptsize{30}} & \mc{1}{c}{\scriptsize{-0.198}} & \mc{1}{c}{\scriptsize{-0.252}} & \mc{1}{c}{\scriptsize{-0.240}} & \mc{1}{c}{\scriptsize{-0.181}} & \mc{1}{c}{\scriptsize{-0.312}} & \mc{1}{c}{\scriptsize{-0.181}} & \mc{1}{c}{\scriptsize{-0.244}} & \mc{1}{c}{\scriptsize{-0.288}} \\  

     &  & \mc{1}{c}{\scriptsize{(0.987)}} & \mc{1}{c}{\scriptsize{(0.974)}} & \mc{1}{c}{\scriptsize{(1.000)}} & \mc{1}{c}{\scriptsize{(0.737)}} & \mc{1}{c}{\scriptsize{(1.000)}} & \mc{1}{c}{\scriptsize{(0.987)}} & \mc{1}{c}{\scriptsize{(0.974)}} & \mc{1}{c}{\scriptsize{(1.000)}} \\  

    \mc{1}{l}{\scriptsize{No trouble with boss}} & \mc{1}{c}{\scriptsize{30}} & \mc{1}{c}{\scriptsize{-0.149}} & \mc{1}{c}{\scriptsize{-0.117}} & \mc{1}{c}{\scriptsize{-0.240}} & \mc{1}{c}{\scriptsize{-0.137}} & \mc{1}{c}{\scriptsize{-0.258}} & \mc{1}{c}{\scriptsize{-0.107}} & \mc{1}{c}{\scriptsize{-0.105}} & \mc{1}{c}{\scriptsize{-0.147}} \\  

     &  & \mc{1}{c}{\scriptsize{(0.908)}} & \mc{1}{c}{\scriptsize{(0.816)}} & \mc{1}{c}{\scriptsize{(1.000)}} & \mc{1}{c}{\scriptsize{(0.671)}} & \mc{1}{c}{\scriptsize{(0.987)}} & \mc{1}{c}{\scriptsize{(0.776)}} & \mc{1}{c}{\scriptsize{(0.750)}} & \mc{1}{c}{\scriptsize{(0.868)}} \\  

  \bottomrule
  \end{tabular}
	\end{table} 

	\begin{table}[H]
     \caption{Treatment Effects on Job Satisfaction Score, Male Sample}
     \label{table:abccare_rslt_male_cat25}
	  \begin{tabular}{cccccccccc}
  \toprule

    \scriptsize{Variable} & \scriptsize{Age} & \scriptsize{(1)} & \scriptsize{(2)} & \scriptsize{(3)} & \scriptsize{(4)} & \scriptsize{(5)} & \scriptsize{(6)} & \scriptsize{(7)} & \scriptsize{(8)} \\ 
    \midrule  

    \mc{1}{l}{\scriptsize{Recognition for good work}} & \mc{1}{c}{\scriptsize{30}} & \mc{1}{c}{\scriptsize{-0.355}} & \mc{1}{c}{\scriptsize{-0.270}} & \mc{1}{c}{\scriptsize{-0.582}} & \mc{1}{c}{\scriptsize{-0.700}} & \mc{1}{c}{\scriptsize{-0.503}} & \mc{1}{c}{\scriptsize{-0.303}} & \mc{1}{c}{\scriptsize{-0.184}} & \mc{1}{c}{\scriptsize{-0.189}} \\  

     &  & \mc{1}{c}{\scriptsize{(0.895)}} & \mc{1}{c}{\scriptsize{(0.855)}} & \mc{1}{c}{\scriptsize{(0.882)}} & \mc{1}{c}{\scriptsize{(0.868)}} & \mc{1}{c}{\scriptsize{(0.842)}} & \mc{1}{c}{\scriptsize{(0.855)}} & \mc{1}{c}{\scriptsize{(0.684)}} & \mc{1}{c}{\scriptsize{(0.697)}} \\  

    \mc{1}{l}{\scriptsize{Total}} & \mc{1}{c}{\scriptsize{30}} & \mc{1}{c}{\scriptsize{-0.447}} & \mc{1}{c}{\scriptsize{-0.405}} & \mc{1}{c}{\scriptsize{-0.479}} & \mc{1}{c}{\scriptsize{-0.403}} & \mc{1}{c}{\scriptsize{-0.429}} & \mc{1}{c}{\scriptsize{-0.399}} & \mc{1}{c}{\scriptsize{-0.393}} & \mc{1}{c}{\scriptsize{-0.272}} \\  

     &  & \mc{1}{c}{\scriptsize{(0.934)}} & \mc{1}{c}{\scriptsize{(0.895)}} & \mc{1}{c}{\scriptsize{(0.868)}} & \mc{1}{c}{\scriptsize{(0.776)}} & \mc{1}{c}{\scriptsize{(0.882)}} & \mc{1}{c}{\scriptsize{(0.908)}} & \mc{1}{c}{\scriptsize{(0.868)}} & \mc{1}{c}{\scriptsize{(0.789)}} \\  

    \mc{1}{l}{\scriptsize{Operating policies and procedures}} & \mc{1}{c}{\scriptsize{30}} & \mc{1}{c}{\scriptsize{-0.333}} & \mc{1}{c}{\scriptsize{-0.065}} & \mc{1}{c}{\scriptsize{-0.151}} & \mc{1}{c}{\scriptsize{-0.014}} & \mc{1}{c}{\scriptsize{-0.038}} & \mc{1}{c}{\scriptsize{-0.316}} & \mc{1}{c}{\scriptsize{-0.055}} & \mc{1}{c}{\scriptsize{-0.115}} \\  

     &  & \mc{1}{c}{\scriptsize{(0.868)}} & \mc{1}{c}{\scriptsize{(0.553)}} & \mc{1}{c}{\scriptsize{(0.539)}} & \mc{1}{c}{\scriptsize{(0.487)}} & \mc{1}{c}{\scriptsize{(0.461)}} & \mc{1}{c}{\scriptsize{(0.816)}} & \mc{1}{c}{\scriptsize{(0.539)}} & \mc{1}{c}{\scriptsize{(0.592)}} \\  

    \mc{1}{l}{\scriptsize{Immediate supervisor}} & \mc{1}{c}{\scriptsize{30}} & \mc{1}{c}{\scriptsize{-0.413}} & \mc{1}{c}{\scriptsize{-0.309}} & \mc{1}{c}{\scriptsize{-0.712}} & \mc{1}{c}{\scriptsize{-0.457}} & \mc{1}{c}{\scriptsize{-0.661}} & \mc{1}{c}{\scriptsize{-0.327}} & \mc{1}{c}{\scriptsize{-0.283}} & \mc{1}{c}{\scriptsize{-0.281}} \\  

     &  & \mc{1}{c}{\scriptsize{(0.961)}} & \mc{1}{c}{\scriptsize{(0.868)}} & \mc{1}{c}{\scriptsize{(0.961)}} & \mc{1}{c}{\scriptsize{(0.829)}} & \mc{1}{c}{\scriptsize{(0.961)}} & \mc{1}{c}{\scriptsize{(0.921)}} & \mc{1}{c}{\scriptsize{(0.855)}} & \mc{1}{c}{\scriptsize{(0.842)}} \\  

    \mc{1}{l}{\scriptsize{Pay and remuneration}} & \mc{1}{c}{\scriptsize{30}} & \mc{1}{c}{\scriptsize{-0.558}} & \mc{1}{c}{\scriptsize{-0.692}} & \mc{1}{c}{\scriptsize{-0.499}} & \mc{1}{c}{\scriptsize{-0.839}} & \mc{1}{c}{\scriptsize{-0.543}} & \mc{1}{c}{\scriptsize{-0.529}} & \mc{1}{c}{\scriptsize{-0.630}} & \mc{1}{c}{\scriptsize{-0.571}} \\  

     &  & \mc{1}{c}{\scriptsize{(0.974)}} & \mc{1}{c}{\scriptsize{(0.961)}} & \mc{1}{c}{\scriptsize{(0.816)}} & \mc{1}{c}{\scriptsize{(0.855)}} & \mc{1}{c}{\scriptsize{(0.776)}} & \mc{1}{c}{\scriptsize{(0.947)}} & \mc{1}{c}{\scriptsize{(0.947)}} & \mc{1}{c}{\scriptsize{(0.921)}} \\  

    \mc{1}{l}{\scriptsize{Coworkers}} & \mc{1}{c}{\scriptsize{30}} & \mc{1}{c}{\scriptsize{-0.035}} & \mc{1}{c}{\scriptsize{0.104}} & \mc{1}{c}{\scriptsize{-0.232}} & \mc{1}{c}{\scriptsize{-0.092}} & \mc{1}{c}{\scriptsize{-0.158}} & \mc{1}{c}{\scriptsize{0.057}} & \mc{1}{c}{\scriptsize{0.128}} & \mc{1}{c}{\scriptsize{0.187}} \\  

     &  & \mc{1}{c}{\scriptsize{(0.526)}} & \mc{1}{c}{\scriptsize{(0.342)}} & \mc{1}{c}{\scriptsize{(0.697)}} & \mc{1}{c}{\scriptsize{(0.632)}} & \mc{1}{c}{\scriptsize{(0.658)}} & \mc{1}{c}{\scriptsize{(0.421)}} & \mc{1}{c}{\scriptsize{(0.355)}} & \mc{1}{c}{\scriptsize{(0.250)}} \\  

    \mc{1}{l}{\scriptsize{Job tasks}} & \mc{1}{c}{\scriptsize{30}} & \mc{1}{c}{\scriptsize{-0.248}} & \mc{1}{c}{\scriptsize{-0.237}} & \mc{1}{c}{\scriptsize{-0.438}} & \mc{1}{c}{\scriptsize{-0.455}} & \mc{1}{c}{\scriptsize{-0.398}} & \mc{1}{c}{\scriptsize{-0.144}} & \mc{1}{c}{\scriptsize{-0.113}} & \mc{1}{c}{\scriptsize{-0.034}} \\  

     &  & \mc{1}{c}{\scriptsize{(0.803)}} & \mc{1}{c}{\scriptsize{(0.789)}} & \mc{1}{c}{\scriptsize{(0.789)}} & \mc{1}{c}{\scriptsize{(0.829)}} & \mc{1}{c}{\scriptsize{(0.829)}} & \mc{1}{c}{\scriptsize{(0.658)}} & \mc{1}{c}{\scriptsize{(0.605)}} & \mc{1}{c}{\scriptsize{(0.513)}} \\  

    \mc{1}{l}{\scriptsize{Fringe benefits}} & \mc{1}{c}{\scriptsize{30}} & \mc{1}{c}{\scriptsize{-0.350}} & \mc{1}{c}{\scriptsize{-0.409}} & \mc{1}{c}{\scriptsize{-0.325}} & \mc{1}{c}{\scriptsize{-0.611}} & \mc{1}{c}{\scriptsize{-0.347}} & \mc{1}{c}{\scriptsize{-0.331}} & \mc{1}{c}{\scriptsize{-0.350}} & \mc{1}{c}{\scriptsize{-0.289}} \\  

     &  & \mc{1}{c}{\scriptsize{(0.908)}} & \mc{1}{c}{\scriptsize{(0.895)}} & \mc{1}{c}{\scriptsize{(0.724)}} & \mc{1}{c}{\scriptsize{(0.855)}} & \mc{1}{c}{\scriptsize{(0.724)}} & \mc{1}{c}{\scriptsize{(0.868)}} & \mc{1}{c}{\scriptsize{(0.868)}} & \mc{1}{c}{\scriptsize{(0.842)}} \\  

    \mc{1}{l}{\scriptsize{Communication with organization}} & \mc{1}{c}{\scriptsize{30}} & \mc{1}{c}{\scriptsize{-0.322}} & \mc{1}{c}{\scriptsize{-0.171}} & \mc{1}{c}{\scriptsize{-0.382}} & \mc{1}{c}{\scriptsize{0.068}} & \mc{1}{c}{\scriptsize{-0.448}} & \mc{1}{c}{\scriptsize{-0.300}} & \mc{1}{c}{\scriptsize{-0.238}} & \mc{1}{c}{\scriptsize{-0.295}} \\  

     &  & \mc{1}{c}{\scriptsize{(0.882)}} & \mc{1}{c}{\scriptsize{(0.711)}} & \mc{1}{c}{\scriptsize{(0.789)}} & \mc{1}{c}{\scriptsize{(0.434)}} & \mc{1}{c}{\scriptsize{(0.829)}} & \mc{1}{c}{\scriptsize{(0.855)}} & \mc{1}{c}{\scriptsize{(0.789)}} & \mc{1}{c}{\scriptsize{(0.816)}} \\  

    \mc{1}{l}{\scriptsize{Promotion opportunities}} & \mc{1}{c}{\scriptsize{30}} & \mc{1}{c}{\scriptsize{-0.144}} & \mc{1}{c}{\scriptsize{-0.293}} & \mc{1}{c}{\scriptsize{0.262}} & \mc{1}{c}{\scriptsize{0.396}} & \mc{1}{c}{\scriptsize{0.244}} & \mc{1}{c}{\scriptsize{-0.223}} & \mc{1}{c}{\scriptsize{-0.430}} & \mc{1}{c}{\scriptsize{-0.353}} \\  

     &  & \mc{1}{c}{\scriptsize{(0.671)}} & \mc{1}{c}{\scriptsize{(0.763)}} & \mc{1}{c}{\scriptsize{(0.250)}} & \mc{1}{c}{\scriptsize{(0.211)}} & \mc{1}{c}{\scriptsize{(0.289)}} & \mc{1}{c}{\scriptsize{(0.750)}} & \mc{1}{c}{\scriptsize{(0.842)}} & \mc{1}{c}{\scriptsize{(0.829)}} \\  

  \bottomrule
  \end{tabular}
	\end{table} 

	\begin{table}[H]
     \caption{Treatment Effects on Crime, Male Sample}
     \label{table:abccare_rslt_male_cat26}
	\input{AppResOutput/abccare/rslt_male_cat26}
	\end{table} 

	\begin{table}[H]
     \caption{Treatment Effects on Childhood and Adolescence Physical Health, Male Sample}
     \label{table:abccare_rslt_male_cat27}
	\input{AppResOutput/abccare/rslt_male_cat27}
	\end{table} 

	\begin{table}[H]
     \caption{Treatment Effects on Childhood Health Problems, Male Sample}
     \label{table:abccare_rslt_male_cat28}
	  \begin{tabular}{cccccccccc}
  \toprule

    \scriptsize{Variable} & \scriptsize{Age} & \scriptsize{(1)} & \scriptsize{(2)} & \scriptsize{(3)} & \scriptsize{(4)} & \scriptsize{(5)} & \scriptsize{(6)} & \scriptsize{(7)} & \scriptsize{(8)} \\ 
    \midrule  

    \mc{1}{l}{\scriptsize{Has Health Problems}} & \mc{1}{c}{\scriptsize{12}} & \mc{1}{c}{\scriptsize{-0.065}} & \mc{1}{c}{\scriptsize{-0.181}} & \mc{1}{c}{\scriptsize{-0.005}} & \mc{1}{c}{\scriptsize{-0.082}} & \mc{1}{c}{\scriptsize{0.017}} & \mc{1}{c}{\scriptsize{-0.079}} & \mc{1}{c}{\scriptsize{-0.213}} & \mc{1}{c}{\scriptsize{-0.154}} \\  

     &  & \mc{1}{c}{\scriptsize{(0.303)}} & \mc{1}{c}{\scriptsize{\textbf{(0.079)}}} & \mc{1}{c}{\scriptsize{(0.461)}} & \mc{1}{c}{\scriptsize{(0.276)}} & \mc{1}{c}{\scriptsize{(0.474)}} & \mc{1}{c}{\scriptsize{(0.263)}} & \mc{1}{c}{\scriptsize{\textbf{(0.039)}}} & \mc{1}{c}{\scriptsize{\textbf{(0.092)}}} \\  

    \mc{1}{l}{\scriptsize{Ever Hospitalized for Over 1 Week}} & \mc{1}{c}{\scriptsize{12}} & \mc{1}{c}{\scriptsize{-0.067}} & \mc{1}{c}{\scriptsize{-0.087}} & \mc{1}{c}{\scriptsize{0.067}} & \mc{1}{c}{\scriptsize{0.014}} & \mc{1}{c}{\scriptsize{0.066}} & \mc{1}{c}{\scriptsize{-0.100}} & \mc{1}{c}{\scriptsize{-0.122}} & \mc{1}{c}{\scriptsize{-0.099}} \\  

     &  & \mc{1}{c}{\scriptsize{(0.224)}} & \mc{1}{c}{\scriptsize{(0.250)}} & \mc{1}{c}{\scriptsize{(0.724)}} & \mc{1}{c}{\scriptsize{(0.513)}} & \mc{1}{c}{\scriptsize{(0.750)}} & \mc{1}{c}{\scriptsize{(0.171)}} & \mc{1}{c}{\scriptsize{(0.145)}} & \mc{1}{c}{\scriptsize{(0.171)}} \\  

  \bottomrule
  \end{tabular}
	\end{table} 

	\begin{table}[H]
     \caption{Treatment Effects on Cholesterol, Male Sample}
     \label{table:abccare_rslt_male_cat29}
	\input{AppResOutput/abccare/rslt_male_cat29}
	\end{table} 

	\begin{table}[H]
     \caption{Treatment Effects on Current Health Condition (Self-Reported), Male Sample}
     \label{table:abccare_rslt_male_cat30}
	\input{AppResOutput/abccare/rslt_male_cat30}
	\end{table} 

	\begin{table}[H]
     \caption{Treatment Effects on Diabetes, Male Sample}
     \label{table:abccare_rslt_male_cat31}
	\input{AppResOutput/abccare/rslt_male_cat31}
	\end{table} 

	\begin{table}[H]
     \caption{Treatment Effects on Drug Behavior and ASR Substance Scale, Male Sample}
     \label{table:abccare_rslt_male_cat32}
	  \begin{tabular}{cccccccccc}
  \toprule

    \scriptsize{Variable} & \scriptsize{Age} & \scriptsize{(1)} & \scriptsize{(2)} & \scriptsize{(3)} & \scriptsize{(4)} & \scriptsize{(5)} & \scriptsize{(6)} & \scriptsize{(7)} & \scriptsize{(8)} \\ 
    \midrule  

    \mc{1}{l}{\scriptsize{Cocaine: Smokes Reguarly}} & \mc{1}{c}{\scriptsize{30}} & \mc{1}{c}{\scriptsize{-0.009}} & \mc{1}{c}{\scriptsize{-0.064}} & \mc{1}{c}{\scriptsize{0.057}} & \mc{1}{c}{\scriptsize{-0.053}} & \mc{1}{c}{\scriptsize{0.034}} & \mc{1}{c}{\scriptsize{0.012}} & \mc{1}{c}{\scriptsize{-0.047}} & \mc{1}{c}{\scriptsize{-0.012}} \\  

     &  & \mc{1}{c}{\scriptsize{(0.434)}} & \mc{1}{c}{\scriptsize{(0.118)}} & \mc{1}{c}{\scriptsize{(0.789)}} & \mc{1}{c}{\scriptsize{(0.105)}} & \mc{1}{c}{\scriptsize{(0.500)}} & \mc{1}{c}{\scriptsize{(0.566)}} & \mc{1}{c}{\scriptsize{(0.250)}} & \mc{1}{c}{\scriptsize{(0.355)}} \\  

    \mc{1}{l}{\scriptsize{Marijuana: Times Used}} & \mc{1}{c}{\scriptsize{30}} & \mc{1}{c}{\scriptsize{-0.052}} & \mc{1}{c}{\scriptsize{-0.279}} & \mc{1}{c}{\scriptsize{0.229}} & \mc{1}{c}{\scriptsize{-0.219}} & \mc{1}{c}{\scriptsize{0.349}} & \mc{1}{c}{\scriptsize{-0.031}} & \mc{1}{c}{\scriptsize{-0.117}} & \mc{1}{c}{\scriptsize{0.219}} \\  

     &  & \mc{1}{c}{\scriptsize{(0.408)}} & \mc{1}{c}{\scriptsize{(0.276)}} & \mc{1}{c}{\scriptsize{(0.566)}} & \mc{1}{c}{\scriptsize{(0.355)}} & \mc{1}{c}{\scriptsize{(0.566)}} & \mc{1}{c}{\scriptsize{(0.474)}} & \mc{1}{c}{\scriptsize{(0.382)}} & \mc{1}{c}{\scriptsize{(0.566)}} \\  

    \mc{1}{l}{\scriptsize{Marijuana: Smokes Regularly}} & \mc{1}{c}{\scriptsize{30}} & \mc{1}{c}{\scriptsize{-0.245}} & \mc{1}{c}{\scriptsize{-0.284}} & \mc{1}{c}{\scriptsize{-0.340}} & \mc{1}{c}{\scriptsize{-0.486}} & \mc{1}{c}{\scriptsize{-0.328}} & \mc{1}{c}{\scriptsize{-0.184}} & \mc{1}{c}{\scriptsize{-0.202}} & \mc{1}{c}{\scriptsize{-0.134}} \\  

     &  & \mc{1}{c}{\scriptsize{\textbf{(0.000)}}} & \mc{1}{c}{\scriptsize{\textbf{(0.000)}}} & \mc{1}{c}{\scriptsize{\textbf{(0.026)}}} & \mc{1}{c}{\scriptsize{\textbf{(0.013)}}} & \mc{1}{c}{\scriptsize{\textbf{(0.026)}}} & \mc{1}{c}{\scriptsize{\textbf{(0.053)}}} & \mc{1}{c}{\scriptsize{\textbf{(0.053)}}} & \mc{1}{c}{\scriptsize{(0.118)}} \\  

    \mc{1}{l}{\scriptsize{Cocaine: Times Used}} & \mc{1}{c}{\scriptsize{30}} & \mc{1}{c}{\scriptsize{-0.233}} & \mc{1}{c}{\scriptsize{-0.432}} & \mc{1}{c}{\scriptsize{-0.086}} & \mc{1}{c}{\scriptsize{-0.515}} & \mc{1}{c}{\scriptsize{-0.152}} & \mc{1}{c}{\scriptsize{-0.073}} & \mc{1}{c}{\scriptsize{-0.326}} & \mc{1}{c}{\scriptsize{-0.142}} \\  

     &  & \mc{1}{c}{\scriptsize{(0.132)}} & \mc{1}{c}{\scriptsize{\textbf{(0.066)}}} & \mc{1}{c}{\scriptsize{(0.382)}} & \mc{1}{c}{\scriptsize{\textbf{(0.053)}}} & \mc{1}{c}{\scriptsize{(0.250)}} & \mc{1}{c}{\scriptsize{(0.368)}} & \mc{1}{c}{\scriptsize{(0.145)}} & \mc{1}{c}{\scriptsize{(0.250)}} \\  

    \mc{1}{l}{\scriptsize{Marijuana: Times Used in Past 30 Days}} & \mc{1}{c}{\scriptsize{30}} & \mc{1}{c}{\scriptsize{-1.004}} & \mc{1}{c}{\scriptsize{-1.053}} & \mc{1}{c}{\scriptsize{-1.328}} & \mc{1}{c}{\scriptsize{-1.668}} & \mc{1}{c}{\scriptsize{-1.249}} & \mc{1}{c}{\scriptsize{-0.743}} & \mc{1}{c}{\scriptsize{-0.707}} & \mc{1}{c}{\scriptsize{-0.473}} \\  

     &  & \mc{1}{c}{\scriptsize{\textbf{(0.000)}}} & \mc{1}{c}{\scriptsize{\textbf{(0.013)}}} & \mc{1}{c}{\scriptsize{\textbf{(0.039)}}} & \mc{1}{c}{\scriptsize{\textbf{(0.066)}}} & \mc{1}{c}{\scriptsize{\textbf{(0.066)}}} & \mc{1}{c}{\scriptsize{\textbf{(0.053)}}} & \mc{1}{c}{\scriptsize{(0.105)}} & \mc{1}{c}{\scriptsize{(0.118)}} \\  

    \mc{1}{l}{\scriptsize{Times Used Other Illegal Drugs in Past 30 Days}} & \mc{1}{c}{\scriptsize{21}} & \mc{1}{c}{\scriptsize{-0.194}} & \mc{1}{c}{\scriptsize{-0.202}} & \mc{1}{c}{\scriptsize{-0.114}} & \mc{1}{c}{\scriptsize{-0.174}} & \mc{1}{c}{\scriptsize{-0.110}} & \mc{1}{c}{\scriptsize{-0.221}} & \mc{1}{c}{\scriptsize{-0.233}} & \mc{1}{c}{\scriptsize{-0.235}} \\  

     &  & \mc{1}{c}{\scriptsize{\textbf{(0.039)}}} & \mc{1}{c}{\scriptsize{\textbf{(0.079)}}} & \mc{1}{c}{\scriptsize{(0.145)}} & \mc{1}{c}{\scriptsize{(0.132)}} & \mc{1}{c}{\scriptsize{(0.158)}} & \mc{1}{c}{\scriptsize{\textbf{(0.079)}}} & \mc{1}{c}{\scriptsize{\textbf{(0.079)}}} & \mc{1}{c}{\scriptsize{\textbf{(0.079)}}} \\  

    \mc{1}{l}{\scriptsize{ASR Substance Use Scale: Alcohol}} & \mc{1}{c}{\scriptsize{30}} & \mc{1}{c}{\scriptsize{-0.121}} & \mc{1}{c}{\scriptsize{-0.789}} & \mc{1}{c}{\scriptsize{1.046}} & \mc{1}{c}{\scriptsize{-0.926}} & \mc{1}{c}{\scriptsize{1.212}} & \mc{1}{c}{\scriptsize{-0.597}} & \mc{1}{c}{\scriptsize{-1.076}} & \mc{1}{c}{\scriptsize{0.345}} \\  

     &  & \mc{1}{c}{\scriptsize{(0.421)}} & \mc{1}{c}{\scriptsize{(0.303)}} & \mc{1}{c}{\scriptsize{(0.671)}} & \mc{1}{c}{\scriptsize{(0.355)}} & \mc{1}{c}{\scriptsize{(0.697)}} & \mc{1}{c}{\scriptsize{(0.368)}} & \mc{1}{c}{\scriptsize{(0.303)}} & \mc{1}{c}{\scriptsize{(0.579)}} \\  

    \mc{1}{l}{\scriptsize{Marijuana: Times Used in Past 30 Days}} & \mc{1}{c}{\scriptsize{21}} & \mc{1}{c}{\scriptsize{-0.473}} & \mc{1}{c}{\scriptsize{-0.490}} & \mc{1}{c}{\scriptsize{-0.029}} & \mc{1}{c}{\scriptsize{0.441}} & \mc{1}{c}{\scriptsize{-0.071}} & \mc{1}{c}{\scriptsize{-0.493}} & \mc{1}{c}{\scriptsize{-0.514}} & \mc{1}{c}{\scriptsize{-0.341}} \\  

     &  & \mc{1}{c}{\scriptsize{(0.132)}} & \mc{1}{c}{\scriptsize{(0.171)}} & \mc{1}{c}{\scriptsize{(0.434)}} & \mc{1}{c}{\scriptsize{(0.750)}} & \mc{1}{c}{\scriptsize{(0.421)}} & \mc{1}{c}{\scriptsize{(0.132)}} & \mc{1}{c}{\scriptsize{(0.132)}} & \mc{1}{c}{\scriptsize{(0.224)}} \\  

    \mc{1}{l}{\scriptsize{ASR Substance Use Scale: Mean Substance Abuse}} & \mc{1}{c}{\scriptsize{30}} & \mc{1}{c}{\scriptsize{-0.246}} & \mc{1}{c}{\scriptsize{-0.435}} & \mc{1}{c}{\scriptsize{-0.721}} & \mc{1}{c}{\scriptsize{-2.114}} & \mc{1}{c}{\scriptsize{-0.616}} & \mc{1}{c}{\scriptsize{0.268}} & \mc{1}{c}{\scriptsize{0.189}} & \mc{1}{c}{\scriptsize{1.137}} \\  

     &  & \mc{1}{c}{\scriptsize{(0.382)}} & \mc{1}{c}{\scriptsize{(0.342)}} & \mc{1}{c}{\scriptsize{(0.408)}} & \mc{1}{c}{\scriptsize{(0.263)}} & \mc{1}{c}{\scriptsize{(0.408)}} & \mc{1}{c}{\scriptsize{(0.513)}} & \mc{1}{c}{\scriptsize{(0.539)}} & \mc{1}{c}{\scriptsize{(0.776)}} \\  

    \mc{1}{l}{\scriptsize{Cocaine: Number of Times Used Crack Cocaine}} & \mc{1}{c}{\scriptsize{30}} & \mc{1}{c}{\scriptsize{0.029}} & \mc{1}{c}{\scriptsize{0.025}} & \mc{1}{c}{\scriptsize{0.029}} & \mc{1}{c}{\scriptsize{0.040}} & \mc{1}{c}{\scriptsize{0.035}} & \mc{1}{c}{\scriptsize{0.029}} & \mc{1}{c}{\scriptsize{0.021}} & \mc{1}{c}{\scriptsize{0.035}} \\  

     &  & \mc{1}{c}{\scriptsize{(0.592)}} & \mc{1}{c}{\scriptsize{(0.618)}} & \mc{1}{c}{\scriptsize{(0.579)}} & \mc{1}{c}{\scriptsize{(0.526)}} & \mc{1}{c}{\scriptsize{(0.592)}} & \mc{1}{c}{\scriptsize{(0.579)}} & \mc{1}{c}{\scriptsize{(0.526)}} & \mc{1}{c}{\scriptsize{(0.592)}} \\  

    \mc{1}{l}{\scriptsize{ASR Substance Use Scale: Tobacco}} & \mc{1}{c}{\scriptsize{30}} & \mc{1}{c}{\scriptsize{1.398}} & \mc{1}{c}{\scriptsize{1.019}} & \mc{1}{c}{\scriptsize{1.358}} & \mc{1}{c}{\scriptsize{-0.324}} & \mc{1}{c}{\scriptsize{1.005}} & \mc{1}{c}{\scriptsize{1.758}} & \mc{1}{c}{\scriptsize{1.349}} & \mc{1}{c}{\scriptsize{1.548}} \\  

     &  & \mc{1}{c}{\scriptsize{(0.947)}} & \mc{1}{c}{\scriptsize{(0.855)}} & \mc{1}{c}{\scriptsize{(0.658)}} & \mc{1}{c}{\scriptsize{(0.447)}} & \mc{1}{c}{\scriptsize{(0.645)}} & \mc{1}{c}{\scriptsize{(0.947)}} & \mc{1}{c}{\scriptsize{(0.908)}} & \mc{1}{c}{\scriptsize{(0.961)}} \\  

    \mc{1}{l}{\scriptsize{Marijuana: Smokes Regularly}} & \mc{1}{c}{\scriptsize{Mid-30s}} & \mc{1}{c}{\scriptsize{-0.233}} & \mc{1}{c}{\scriptsize{-0.301}} & \mc{1}{c}{\scriptsize{-0.370}} & \mc{1}{c}{\scriptsize{-0.598}} & \mc{1}{c}{\scriptsize{-0.370}} & \mc{1}{c}{\scriptsize{-0.120}} & \mc{1}{c}{\scriptsize{-0.190}} & \mc{1}{c}{\scriptsize{-0.202}} \\  

     &  & \mc{1}{c}{\scriptsize{\textbf{(0.053)}}} & \mc{1}{c}{\scriptsize{\textbf{(0.092)}}} & \mc{1}{c}{\scriptsize{(0.118)}} & \mc{1}{c}{\scriptsize{\textbf{(0.053)}}} & \mc{1}{c}{\scriptsize{(0.132)}} & \mc{1}{c}{\scriptsize{(0.197)}} & \mc{1}{c}{\scriptsize{\textbf{(0.066)}}} & \mc{1}{c}{\scriptsize{(0.171)}} \\  

    \mc{1}{l}{\scriptsize{ASR Substance Use Scale: Drugs}} & \mc{1}{c}{\scriptsize{30}} & \mc{1}{c}{\scriptsize{-3.498}} & \mc{1}{c}{\scriptsize{-3.446}} & \mc{1}{c}{\scriptsize{-10.758}} & \mc{1}{c}{\scriptsize{-9.637}} & \mc{1}{c}{\scriptsize{-10.537}} & \mc{1}{c}{\scriptsize{-0.924}} & \mc{1}{c}{\scriptsize{-1.587}} & \mc{1}{c}{\scriptsize{0.133}} \\  

     &  & \mc{1}{c}{\scriptsize{\textbf{(0.066)}}} & \mc{1}{c}{\scriptsize{\textbf{(0.092)}}} & \mc{1}{c}{\scriptsize{(0.105)}} & \mc{1}{c}{\scriptsize{(0.105)}} & \mc{1}{c}{\scriptsize{(0.105)}} & \mc{1}{c}{\scriptsize{(0.329)}} & \mc{1}{c}{\scriptsize{(0.237)}} & \mc{1}{c}{\scriptsize{(0.513)}} \\  

  \bottomrule
  \end{tabular}
	\end{table} 

	\begin{table}[H]
     \caption{Treatment Effects on Health Insurance, Male Sample}
     \label{table:abccare_rslt_male_cat33}
	  \begin{tabular}{cccccccccc}
  \toprule

    \scriptsize{Variable} & \scriptsize{Age} & \scriptsize{(1)} & \scriptsize{(2)} & \scriptsize{(3)} & \scriptsize{(4)} & \scriptsize{(5)} & \scriptsize{(6)} & \scriptsize{(7)} & \scriptsize{(8)} \\ 
    \midrule  

    \mc{1}{l}{\scriptsize{Has Health Insurance}} & \mc{1}{c}{\scriptsize{30}} & \mc{1}{c}{\scriptsize{0.266}} & \mc{1}{c}{\scriptsize{0.293}} & \mc{1}{c}{\scriptsize{0.257}} & \mc{1}{c}{\scriptsize{0.379}} & \mc{1}{c}{\scriptsize{0.265}} & \mc{1}{c}{\scriptsize{0.251}} & \mc{1}{c}{\scriptsize{0.275}} & \mc{1}{c}{\scriptsize{0.236}} \\  

     &  & \mc{1}{c}{\scriptsize{\textbf{(0.000)}}} & \mc{1}{c}{\scriptsize{\textbf{(0.000)}}} & \mc{1}{c}{\scriptsize{\textbf{(0.092)}}} & \mc{1}{c}{\scriptsize{\textbf{(0.092)}}} & \mc{1}{c}{\scriptsize{\textbf{(0.079)}}} & \mc{1}{c}{\scriptsize{\textbf{(0.000)}}} & \mc{1}{c}{\scriptsize{\textbf{(0.000)}}} & \mc{1}{c}{\scriptsize{\textbf{(0.013)}}} \\  

     & \mc{1}{c}{\scriptsize{21}} & \mc{1}{c}{\scriptsize{-0.113}} & \mc{1}{c}{\scriptsize{-0.194}} & \mc{1}{c}{\scriptsize{0.050}} & \mc{1}{c}{\scriptsize{-0.042}} & \mc{1}{c}{\scriptsize{0.044}} & \mc{1}{c}{\scriptsize{-0.180}} & \mc{1}{c}{\scriptsize{-0.232}} & \mc{1}{c}{\scriptsize{-0.208}} \\  

     &  & \mc{1}{c}{\scriptsize{(0.829)}} & \mc{1}{c}{\scriptsize{(0.921)}} & \mc{1}{c}{\scriptsize{(0.355)}} & \mc{1}{c}{\scriptsize{(0.592)}} & \mc{1}{c}{\scriptsize{(0.447)}} & \mc{1}{c}{\scriptsize{(0.921)}} & \mc{1}{c}{\scriptsize{(0.974)}} & \mc{1}{c}{\scriptsize{(0.921)}} \\  

  \bottomrule
  \end{tabular}
	\end{table} 

	\begin{table}[H]
     \caption{Treatment Effects on Hypertension, Male Sample}
     \label{table:abccare_rslt_male_cat34}
	\input{AppResOutput/abccare/rslt_male_cat34}
	\end{table} 

	\begin{table}[H]
     \caption{Treatment Effects on Laboratory Test  - Metabolic Panel, Male Sample}
     \label{table:abccare_rslt_male_cat35}
	  \begin{tabular}{cccccccccc}
  \toprule

    \scriptsize{Variable} & \scriptsize{Age} & \scriptsize{(1)} & \scriptsize{(2)} & \scriptsize{(3)} & \scriptsize{(4)} & \scriptsize{(5)} & \scriptsize{(6)} & \scriptsize{(7)} & \scriptsize{(8)} \\ 
    \midrule  

    \mc{1}{l}{\scriptsize{Somatic Problems}} & \mc{1}{c}{\scriptsize{30}} & \mc{1}{c}{\scriptsize{1.531}} & \mc{1}{c}{\scriptsize{0.945}} & \mc{1}{c}{\scriptsize{0.029}} & \mc{1}{c}{\scriptsize{-1.748}} & \mc{1}{c}{\scriptsize{-0.244}} & \mc{1}{c}{\scriptsize{1.886}} & \mc{1}{c}{\scriptsize{1.099}} & \mc{1}{c}{\scriptsize{1.533}} \\  

     &  & \mc{1}{c}{\scriptsize{(0.882)}} & \mc{1}{c}{\scriptsize{(0.803)}} & \mc{1}{c}{\scriptsize{(0.487)}} & \mc{1}{c}{\scriptsize{(0.263)}} & \mc{1}{c}{\scriptsize{(0.421)}} & \mc{1}{c}{\scriptsize{(0.947)}} & \mc{1}{c}{\scriptsize{(0.776)}} & \mc{1}{c}{\scriptsize{(0.895)}} \\  

    \mc{1}{l}{\scriptsize{AD/H Problems}} & \mc{1}{c}{\scriptsize{30}} & \mc{1}{c}{\scriptsize{1.288}} & \mc{1}{c}{\scriptsize{0.986}} & \mc{1}{c}{\scriptsize{2.200}} & \mc{1}{c}{\scriptsize{1.495}} & \mc{1}{c}{\scriptsize{2.462}} & \mc{1}{c}{\scriptsize{1.119}} & \mc{1}{c}{\scriptsize{0.872}} & \mc{1}{c}{\scriptsize{1.255}} \\  

     &  & \mc{1}{c}{\scriptsize{(0.921)}} & \mc{1}{c}{\scriptsize{(0.855)}} & \mc{1}{c}{\scriptsize{(0.987)}} & \mc{1}{c}{\scriptsize{(0.803)}} & \mc{1}{c}{\scriptsize{(0.974)}} & \mc{1}{c}{\scriptsize{(0.895)}} & \mc{1}{c}{\scriptsize{(0.776)}} & \mc{1}{c}{\scriptsize{(0.882)}} \\  

    \mc{1}{l}{\scriptsize{Inattention Subscale}} & \mc{1}{c}{\scriptsize{30}} &  &  &  &  &  &  &  &  \\  

     &  &  &  &  &  &  &  &  &  \\  

    \mc{1}{l}{\scriptsize{Depressive Problems}} & \mc{1}{c}{\scriptsize{30}} & \mc{1}{c}{\scriptsize{1.421}} & \mc{1}{c}{\scriptsize{1.275}} & \mc{1}{c}{\scriptsize{2.200}} & \mc{1}{c}{\scriptsize{1.548}} & \mc{1}{c}{\scriptsize{2.578}} & \mc{1}{c}{\scriptsize{1.225}} & \mc{1}{c}{\scriptsize{1.144}} & \mc{1}{c}{\scriptsize{1.543}} \\  

     &  & \mc{1}{c}{\scriptsize{(0.947)}} & \mc{1}{c}{\scriptsize{(0.921)}} & \mc{1}{c}{\scriptsize{(0.974)}} & \mc{1}{c}{\scriptsize{(0.842)}} & \mc{1}{c}{\scriptsize{(0.974)}} & \mc{1}{c}{\scriptsize{(0.934)}} & \mc{1}{c}{\scriptsize{(0.895)}} & \mc{1}{c}{\scriptsize{(0.947)}} \\  

    \mc{1}{l}{\scriptsize{Avoidant Personality Problems}} & \mc{1}{c}{\scriptsize{30}} & \mc{1}{c}{\scriptsize{1.391}} & \mc{1}{c}{\scriptsize{2.088}} & \mc{1}{c}{\scriptsize{1.114}} & \mc{1}{c}{\scriptsize{1.691}} & \mc{1}{c}{\scriptsize{1.535}} & \mc{1}{c}{\scriptsize{1.363}} & \mc{1}{c}{\scriptsize{1.561}} & \mc{1}{c}{\scriptsize{1.641}} \\  

     &  & \mc{1}{c}{\scriptsize{(0.842)}} & \mc{1}{c}{\scriptsize{(0.934)}} & \mc{1}{c}{\scriptsize{(0.579)}} & \mc{1}{c}{\scriptsize{(0.776)}} & \mc{1}{c}{\scriptsize{(0.632)}} & \mc{1}{c}{\scriptsize{(0.842)}} & \mc{1}{c}{\scriptsize{(0.895)}} & \mc{1}{c}{\scriptsize{(0.882)}} \\  

    \mc{1}{l}{\scriptsize{Anxiety Problems}} & \mc{1}{c}{\scriptsize{30}} & \mc{1}{c}{\scriptsize{1.591}} & \mc{1}{c}{\scriptsize{1.224}} & \mc{1}{c}{\scriptsize{1.171}} & \mc{1}{c}{\scriptsize{-0.579}} & \mc{1}{c}{\scriptsize{1.663}} & \mc{1}{c}{\scriptsize{1.606}} & \mc{1}{c}{\scriptsize{1.263}} & \mc{1}{c}{\scriptsize{1.962}} \\  

     &  & \mc{1}{c}{\scriptsize{(0.855)}} & \mc{1}{c}{\scriptsize{(0.842)}} & \mc{1}{c}{\scriptsize{(0.658)}} & \mc{1}{c}{\scriptsize{(0.447)}} & \mc{1}{c}{\scriptsize{(0.724)}} & \mc{1}{c}{\scriptsize{(0.921)}} & \mc{1}{c}{\scriptsize{(0.737)}} & \mc{1}{c}{\scriptsize{(0.961)}} \\  

    \mc{1}{l}{\scriptsize{Antisocial Personality Problems}} & \mc{1}{c}{\scriptsize{30}} & \mc{1}{c}{\scriptsize{2.349}} & \mc{1}{c}{\scriptsize{1.833}} & \mc{1}{c}{\scriptsize{3.686}} & \mc{1}{c}{\scriptsize{2.931}} & \mc{1}{c}{\scriptsize{4.180}} & \mc{1}{c}{\scriptsize{1.978}} & \mc{1}{c}{\scriptsize{1.453}} & \mc{1}{c}{\scriptsize{2.313}} \\  

     &  & \mc{1}{c}{\scriptsize{(0.934)}} & \mc{1}{c}{\scriptsize{(0.895)}} & \mc{1}{c}{\scriptsize{(0.987)}} & \mc{1}{c}{\scriptsize{(0.868)}} & \mc{1}{c}{\scriptsize{(0.987)}} & \mc{1}{c}{\scriptsize{(0.921)}} & \mc{1}{c}{\scriptsize{(0.750)}} & \mc{1}{c}{\scriptsize{(0.934)}} \\  

    \mc{1}{l}{\scriptsize{Hyperactivity-Impulsivity Subscale}} & \mc{1}{c}{\scriptsize{30}} &  &  &  &  &  &  &  &  \\  

     &  &  &  &  &  &  &  &  &  \\  

  \bottomrule
  \end{tabular}
	\end{table} 

	\begin{table}[H]
     \caption{Treatment Effects on Laboratory Test - Complete Blood Count, Male Sample}
     \label{table:abccare_rslt_male_cat36}
	\input{AppResOutput/abccare/rslt_male_cat36}
	\end{table} 

	\begin{table}[H]
     \caption{Treatment Effects on Other Health-Related Information, Male Sample}
     \label{table:abccare_rslt_male_cat37}
	  \begin{tabular}{cccccccccc}
  \toprule

    \scriptsize{Variable} & \scriptsize{Age} & \scriptsize{(1)} & \scriptsize{(2)} & \scriptsize{(3)} & \scriptsize{(4)} & \scriptsize{(5)} & \scriptsize{(6)} & \scriptsize{(7)} & \scriptsize{(8)} \\ 
    \midrule  

    \mc{1}{l}{\scriptsize{Number of Days Very Healthy in Past 30 Days}} & \mc{1}{c}{\scriptsize{Mid-30s}} & \mc{1}{c}{\scriptsize{7.619}} & \mc{1}{c}{\scriptsize{10.556}} & \mc{1}{c}{\scriptsize{-1.333}} & \mc{1}{c}{\scriptsize{4.401}} & \mc{1}{c}{\scriptsize{-1.451}} & \mc{1}{c}{\scriptsize{9.033}} & \mc{1}{c}{\scriptsize{11.084}} & \mc{1}{c}{\scriptsize{11.111}} \\  

     &  & \mc{1}{c}{\scriptsize{\textbf{(0.000)}}} & \mc{1}{c}{\scriptsize{\textbf{(0.000)}}} & \mc{1}{c}{\scriptsize{(0.605)}} & \mc{1}{c}{\scriptsize{(0.184)}} & \mc{1}{c}{\scriptsize{(0.618)}} & \mc{1}{c}{\scriptsize{\textbf{(0.000)}}} & \mc{1}{c}{\scriptsize{\textbf{(0.013)}}} & \mc{1}{c}{\scriptsize{\textbf{(0.000)}}} \\  

    \mc{1}{l}{\scriptsize{How Subject Thinks of Own Weight}} & \mc{1}{c}{\scriptsize{30}} & \mc{1}{c}{\scriptsize{-0.091}} & \mc{1}{c}{\scriptsize{0.003}} & \mc{1}{c}{\scriptsize{-0.257}} & \mc{1}{c}{\scriptsize{-0.044}} & \mc{1}{c}{\scriptsize{-0.265}} & \mc{1}{c}{\scriptsize{-0.065}} & \mc{1}{c}{\scriptsize{-0.016}} & \mc{1}{c}{\scriptsize{-0.099}} \\  

     &  & \mc{1}{c}{\scriptsize{(0.263)}} & \mc{1}{c}{\scriptsize{(0.513)}} & \mc{1}{c}{\scriptsize{\textbf{(0.079)}}} & \mc{1}{c}{\scriptsize{(0.539)}} & \mc{1}{c}{\scriptsize{\textbf{(0.066)}}} & \mc{1}{c}{\scriptsize{(0.329)}} & \mc{1}{c}{\scriptsize{(0.408)}} & \mc{1}{c}{\scriptsize{(0.316)}} \\  

    \mc{1}{l}{\scriptsize{Number of Days in Pain in Past 30 Days}} & \mc{1}{c}{\scriptsize{Mid-30s}} & \mc{1}{c}{\scriptsize{-0.669}} & \mc{1}{c}{\scriptsize{-0.440}} & \mc{1}{c}{\scriptsize{2.593}} & \mc{1}{c}{\scriptsize{2.652}} & \mc{1}{c}{\scriptsize{2.999}} & \mc{1}{c}{\scriptsize{0.459}} & \mc{1}{c}{\scriptsize{0.382}} & \mc{1}{c}{\scriptsize{1.277}} \\  

     &  & \mc{1}{c}{\scriptsize{(0.434)}} & \mc{1}{c}{\scriptsize{(0.447)}} & \mc{1}{c}{\scriptsize{(0.803)}} & \mc{1}{c}{\scriptsize{(0.658)}} & \mc{1}{c}{\scriptsize{(0.842)}} & \mc{1}{c}{\scriptsize{(0.579)}} & \mc{1}{c}{\scriptsize{(0.566)}} & \mc{1}{c}{\scriptsize{(0.592)}} \\  

    \mc{1}{l}{\scriptsize{Physical/Nervous Condition Prevents Work}} & \mc{1}{c}{\scriptsize{30}} & \mc{1}{c}{\scriptsize{0.050}} & \mc{1}{c}{\scriptsize{0.028}} & \mc{1}{c}{\scriptsize{0.114}} & \mc{1}{c}{\scriptsize{0.031}} & \mc{1}{c}{\scriptsize{0.103}} & \mc{1}{c}{\scriptsize{0.027}} & \mc{1}{c}{\scriptsize{0.020}} & \mc{1}{c}{\scriptsize{0.013}} \\  

     &  & \mc{1}{c}{\scriptsize{(0.724)}} & \mc{1}{c}{\scriptsize{(0.671)}} & \mc{1}{c}{\scriptsize{(0.947)}} & \mc{1}{c}{\scriptsize{(0.579)}} & \mc{1}{c}{\scriptsize{(0.895)}} & \mc{1}{c}{\scriptsize{(0.632)}} & \mc{1}{c}{\scriptsize{(0.592)}} & \mc{1}{c}{\scriptsize{(0.500)}} \\  

  \bottomrule
  \end{tabular}
	\end{table} 

	\begin{table}[H]
     \caption{Treatment Effects on Past Medical History - Diagnosis (Self-Reported), Male Sample}
     \label{table:abccare_rslt_male_cat38}
	  \begin{tabular}{cccccccccc}
  \toprule

    \scriptsize{Variable} & \scriptsize{Age} & \scriptsize{(1)} & \scriptsize{(2)} & \scriptsize{(3)} & \scriptsize{(4)} & \scriptsize{(5)} & \scriptsize{(6)} & \scriptsize{(7)} & \scriptsize{(8)} \\ 
    \midrule  

    \mc{1}{l}{\scriptsize{Ever Told Had: Arthritis/Gout/Lupus/Fibromyalgia}} & \mc{1}{c}{\scriptsize{Mid-30s}} & \mc{1}{c}{\scriptsize{0.034}} & \mc{1}{c}{\scriptsize{0.053}} & \mc{1}{c}{\scriptsize{-0.222}} & \mc{1}{c}{\scriptsize{-0.213}} & \mc{1}{c}{\scriptsize{-0.223}} & \mc{1}{c}{\scriptsize{0.111}} & \mc{1}{c}{\scriptsize{0.086}} & \mc{1}{c}{\scriptsize{0.083}} \\  

     &  & \mc{1}{c}{\scriptsize{(0.553)}} & \mc{1}{c}{\scriptsize{(0.539)}} & \mc{1}{c}{\scriptsize{(0.211)}} & \mc{1}{c}{\scriptsize{(0.184)}} & \mc{1}{c}{\scriptsize{(0.184)}} & \mc{1}{c}{\scriptsize{(0.895)}} & \mc{1}{c}{\scriptsize{(0.658)}} & \mc{1}{c}{\scriptsize{(0.711)}} \\  

    \mc{1}{l}{\scriptsize{Ever Told Had: Prediabetes}} & \mc{1}{c}{\scriptsize{Mid-30s}} &  &  &  &  &  &  &  &  \\  

     &  &  &  &  &  &  &  &  &  \\  

  \bottomrule
  \end{tabular}
	\end{table} 

	\begin{table}[H]
     \caption{Treatment Effects on Past Medical History - Surgery (Self-Reported), Male Sample}
     \label{table:abccare_rslt_male_cat39}
	  \begin{tabular}{cccccccccc}
  \toprule

    \scriptsize{Variable} & \scriptsize{Age} & \scriptsize{(1)} & \scriptsize{(2)} & \scriptsize{(3)} & \scriptsize{(4)} & \scriptsize{(5)} & \scriptsize{(6)} & \scriptsize{(7)} & \scriptsize{(8)} \\ 
    \midrule  

    \mc{1}{l}{\scriptsize{Past Surgery: Cholecystectomy}} & \mc{1}{c}{\scriptsize{Mid-30s}} &  &  &  &  &  &  &  &  \\  

     &  &  &  &  &  &  &  &  &  \\  

    \mc{1}{l}{\scriptsize{Past Surgery: Orthopedic Surgery}} & \mc{1}{c}{\scriptsize{Mid-30s}} & \mc{1}{c}{\scriptsize{-0.143}} & \mc{1}{c}{\scriptsize{-0.116}} & \mc{1}{c}{\scriptsize{-0.333}} & \mc{1}{c}{\scriptsize{-0.407}} & \mc{1}{c}{\scriptsize{-0.301}} & \mc{1}{c}{\scriptsize{-0.100}} & \mc{1}{c}{\scriptsize{-0.070}} & \mc{1}{c}{\scriptsize{-0.081}} \\  

     &  & \mc{1}{c}{\scriptsize{\textbf{(0.066)}}} & \mc{1}{c}{\scriptsize{(0.118)}} & \mc{1}{c}{\scriptsize{(0.118)}} & \mc{1}{c}{\scriptsize{(0.118)}} & \mc{1}{c}{\scriptsize{(0.118)}} & \mc{1}{c}{\scriptsize{\textbf{(0.092)}}} & \mc{1}{c}{\scriptsize{(0.132)}} & \mc{1}{c}{\scriptsize{(0.118)}} \\  

    \mc{1}{l}{\scriptsize{Past Surgery: Appendectomy}} & \mc{1}{c}{\scriptsize{Mid-30s}} &  &  &  &  &  &  &  &  \\  

     &  &  &  &  &  &  &  &  &  \\  

    \mc{1}{l}{\scriptsize{Past Surgery: Ectopic Pregnancy}} & \mc{1}{c}{\scriptsize{Mid-30s}} &  &  &  &  &  &  &  &  \\  

     &  &  &  &  &  &  &  &  &  \\  

    \mc{1}{l}{\scriptsize{Past Surgery: Hysterectomy}} & \mc{1}{c}{\scriptsize{Mid-30s}} &  &  &  &  &  &  &  &  \\  

     &  &  &  &  &  &  &  &  &  \\  

  \bottomrule
  \end{tabular}
	\end{table} 

	\begin{table}[H]
     \caption{Treatment Effects on Physical Activity, Male Sample}
     \label{table:abccare_rslt_male_cat40}
	  \begin{tabular}{cccccccccc}
  \toprule

    \scriptsize{Variable} & \scriptsize{Age} & \scriptsize{(1)} & \scriptsize{(2)} & \scriptsize{(3)} & \scriptsize{(4)} & \scriptsize{(5)} & \scriptsize{(6)} & \scriptsize{(7)} & \scriptsize{(8)} \\ 
    \midrule  

    \mc{1}{l}{\scriptsize{Current Condition: Any Psychiatric Concern}} & \mc{1}{c}{\scriptsize{Mid-30s}} & \mc{1}{c}{\scriptsize{0.125}} & \mc{1}{c}{\scriptsize{0.132}} & \mc{1}{c}{\scriptsize{0.125}} & \mc{1}{c}{\scriptsize{-0.073}} & \mc{1}{c}{\scriptsize{0.149}} & \mc{1}{c}{\scriptsize{0.125}} & \mc{1}{c}{\scriptsize{0.172}} & \mc{1}{c}{\scriptsize{0.150}} \\  

     &  & \mc{1}{c}{\scriptsize{(0.882)}} & \mc{1}{c}{\scriptsize{(0.776)}} & \mc{1}{c}{\scriptsize{(0.803)}} & \mc{1}{c}{\scriptsize{(0.342)}} & \mc{1}{c}{\scriptsize{(0.816)}} & \mc{1}{c}{\scriptsize{(0.882)}} & \mc{1}{c}{\scriptsize{(0.816)}} & \mc{1}{c}{\scriptsize{(0.882)}} \\  

    \mc{1}{l}{\scriptsize{Physical Exam: Mental Status Distress}} & \mc{1}{c}{\scriptsize{Mid-30s}} &  &  &  &  &  &  &  &  \\  

     &  &  &  &  &  &  &  &  &  \\  

    \mc{1}{l}{\scriptsize{Current Condition: Sad/Depressed in Past 30 Days}} & \mc{1}{c}{\scriptsize{Mid-30s}} & \mc{1}{c}{\scriptsize{-0.423}} & \mc{1}{c}{\scriptsize{-0.828}} & \mc{1}{c}{\scriptsize{1.815}} & \mc{1}{c}{\scriptsize{3.242}} & \mc{1}{c}{\scriptsize{1.730}} & \mc{1}{c}{\scriptsize{-1.352}} & \mc{1}{c}{\scriptsize{-1.598}} & \mc{1}{c}{\scriptsize{-0.913}} \\  

     &  & \mc{1}{c}{\scriptsize{(0.382)}} & \mc{1}{c}{\scriptsize{(0.303)}} & \mc{1}{c}{\scriptsize{(0.868)}} & \mc{1}{c}{\scriptsize{(0.803)}} & \mc{1}{c}{\scriptsize{(0.882)}} & \mc{1}{c}{\scriptsize{(0.263)}} & \mc{1}{c}{\scriptsize{(0.250)}} & \mc{1}{c}{\scriptsize{(0.355)}} \\  

    \mc{1}{l}{\scriptsize{Current Condition: Mental Problems}} & \mc{1}{c}{\scriptsize{Mid-30s}} & \mc{1}{c}{\scriptsize{0.045}} & \mc{1}{c}{\scriptsize{0.018}} & \mc{1}{c}{\scriptsize{-0.074}} & \mc{1}{c}{\scriptsize{-0.141}} & \mc{1}{c}{\scriptsize{-0.072}} & \mc{1}{c}{\scriptsize{0.059}} & \mc{1}{c}{\scriptsize{0.052}} & \mc{1}{c}{\scriptsize{0.058}} \\  

     &  & \mc{1}{c}{\scriptsize{(0.671)}} & \mc{1}{c}{\scriptsize{(0.592)}} & \mc{1}{c}{\scriptsize{(0.355)}} & \mc{1}{c}{\scriptsize{(0.316)}} & \mc{1}{c}{\scriptsize{(0.316)}} & \mc{1}{c}{\scriptsize{(0.618)}} & \mc{1}{c}{\scriptsize{(0.579)}} & \mc{1}{c}{\scriptsize{(0.658)}} \\  

    \mc{1}{l}{\scriptsize{Current Condition: Worried/Anxious in Past 30 Days}} & \mc{1}{c}{\scriptsize{Mid-30s}} & \mc{1}{c}{\scriptsize{1.341}} & \mc{1}{c}{\scriptsize{1.424}} & \mc{1}{c}{\scriptsize{-1.444}} & \mc{1}{c}{\scriptsize{-1.318}} & \mc{1}{c}{\scriptsize{-0.958}} & \mc{1}{c}{\scriptsize{1.956}} & \mc{1}{c}{\scriptsize{1.787}} & \mc{1}{c}{\scriptsize{2.216}} \\  

     &  & \mc{1}{c}{\scriptsize{(0.803)}} & \mc{1}{c}{\scriptsize{(0.763)}} & \mc{1}{c}{\scriptsize{(0.250)}} & \mc{1}{c}{\scriptsize{(0.329)}} & \mc{1}{c}{\scriptsize{(0.316)}} & \mc{1}{c}{\scriptsize{(0.895)}} & \mc{1}{c}{\scriptsize{(0.816)}} & \mc{1}{c}{\scriptsize{(0.921)}} \\  

    \mc{1}{l}{\scriptsize{Current Condition: Anxiety}} & \mc{1}{c}{\scriptsize{Mid-30s}} & \mc{1}{c}{\scriptsize{0.042}} & \mc{1}{c}{\scriptsize{0.034}} & \mc{1}{c}{\scriptsize{0.042}} & \mc{1}{c}{\scriptsize{-0.043}} & \mc{1}{c}{\scriptsize{0.052}} & \mc{1}{c}{\scriptsize{0.042}} & \mc{1}{c}{\scriptsize{0.051}} & \mc{1}{c}{\scriptsize{0.052}} \\  

     &  & \mc{1}{c}{\scriptsize{(0.566)}} & \mc{1}{c}{\scriptsize{(0.500)}} & \mc{1}{c}{\scriptsize{(0.500)}} & \mc{1}{c}{\scriptsize{(0.184)}} & \mc{1}{c}{\scriptsize{(0.500)}} & \mc{1}{c}{\scriptsize{(0.566)}} & \mc{1}{c}{\scriptsize{(0.553)}} & \mc{1}{c}{\scriptsize{(0.566)}} \\  

    \mc{1}{l}{\scriptsize{Physical Exam: Mental Health Alert}} & \mc{1}{c}{\scriptsize{Mid-30s}} &  &  &  &  &  &  &  &  \\  

     &  &  &  &  &  &  &  &  &  \\  

    \mc{1}{l}{\scriptsize{Current Condition: Suicidal Ideation}} & \mc{1}{c}{\scriptsize{Mid-30s}} &  &  &  &  &  &  &  &  \\  

     &  &  &  &  &  &  &  &  &  \\  

    \mc{1}{l}{\scriptsize{Current Condition: Insomnia}} & \mc{1}{c}{\scriptsize{Mid-30s}} & \mc{1}{c}{\scriptsize{0.125}} & \mc{1}{c}{\scriptsize{0.132}} & \mc{1}{c}{\scriptsize{0.125}} & \mc{1}{c}{\scriptsize{-0.073}} & \mc{1}{c}{\scriptsize{0.149}} & \mc{1}{c}{\scriptsize{0.125}} & \mc{1}{c}{\scriptsize{0.172}} & \mc{1}{c}{\scriptsize{0.150}} \\  

     &  & \mc{1}{c}{\scriptsize{(0.882)}} & \mc{1}{c}{\scriptsize{(0.776)}} & \mc{1}{c}{\scriptsize{(0.803)}} & \mc{1}{c}{\scriptsize{(0.342)}} & \mc{1}{c}{\scriptsize{(0.816)}} & \mc{1}{c}{\scriptsize{(0.882)}} & \mc{1}{c}{\scriptsize{(0.816)}} & \mc{1}{c}{\scriptsize{(0.882)}} \\  

    \mc{1}{l}{\scriptsize{Current Condition: Depression}} & \mc{1}{c}{\scriptsize{Mid-30s}} &  &  &  &  &  &  &  &  \\  

     &  &  &  &  &  &  &  &  &  \\  

  \bottomrule
  \end{tabular}
	\end{table} 

	\begin{table}[H]
     \caption{Treatment Effects on Physical Exam - Ear, Male Sample}
     \label{table:abccare_rslt_male_cat41}
	\input{AppResOutput/abccare/rslt_male_cat41}
	\end{table} 

	\begin{table}[H]
     \caption{Treatment Effects on Physical Exam - General I, Male Sample}
     \label{table:abccare_rslt_male_cat42}
	  \begin{tabular}{cccccccccc}
  \toprule

    \scriptsize{Variable} & \scriptsize{Age} & \scriptsize{(1)} & \scriptsize{(2)} & \scriptsize{(3)} & \scriptsize{(4)} & \scriptsize{(5)} & \scriptsize{(6)} & \scriptsize{(7)} & \scriptsize{(8)} \\ 
    \midrule  

    \mc{1}{l}{\scriptsize{Respirations}} & \mc{1}{c}{\scriptsize{Mid-30s}} & \mc{1}{c}{\scriptsize{-0.234}} & \mc{1}{c}{\scriptsize{-0.098}} & \mc{1}{c}{\scriptsize{-0.091}} & \mc{1}{c}{\scriptsize{1.504}} & \mc{1}{c}{\scriptsize{-0.282}} & \mc{1}{c}{\scriptsize{-0.291}} & \mc{1}{c}{\scriptsize{-0.394}} & \mc{1}{c}{\scriptsize{-0.493}} \\  

     &  & \mc{1}{c}{\scriptsize{(0.355)}} & \mc{1}{c}{\scriptsize{(0.474)}} & \mc{1}{c}{\scriptsize{(0.434)}} & \mc{1}{c}{\scriptsize{(0.829)}} & \mc{1}{c}{\scriptsize{(0.197)}} & \mc{1}{c}{\scriptsize{(0.355)}} & \mc{1}{c}{\scriptsize{(0.342)}} & \mc{1}{c}{\scriptsize{(0.184)}} \\  

    \mc{1}{l}{\scriptsize{Temp (F)}} & \mc{1}{c}{\scriptsize{Mid-30s}} & \mc{1}{c}{\scriptsize{-0.016}} & \mc{1}{c}{\scriptsize{0.005}} & \mc{1}{c}{\scriptsize{-0.287}} & \mc{1}{c}{\scriptsize{-0.168}} & \mc{1}{c}{\scriptsize{-0.232}} & \mc{1}{c}{\scriptsize{0.083}} & \mc{1}{c}{\scriptsize{0.041}} & \mc{1}{c}{\scriptsize{0.052}} \\  

     &  & \mc{1}{c}{\scriptsize{(0.408)}} & \mc{1}{c}{\scriptsize{(0.526)}} & \mc{1}{c}{\scriptsize{(0.132)}} & \mc{1}{c}{\scriptsize{(0.289)}} & \mc{1}{c}{\scriptsize{(0.197)}} & \mc{1}{c}{\scriptsize{(0.724)}} & \mc{1}{c}{\scriptsize{(0.632)}} & \mc{1}{c}{\scriptsize{(0.658)}} \\  

    \mc{1}{l}{\scriptsize{Pulse}} & \mc{1}{c}{\scriptsize{Mid-30s}} & \mc{1}{c}{\scriptsize{0.454}} & \mc{1}{c}{\scriptsize{0.416}} & \mc{1}{c}{\scriptsize{-4.427}} & \mc{1}{c}{\scriptsize{-4.030}} & \mc{1}{c}{\scriptsize{-3.150}} & \mc{1}{c}{\scriptsize{2.040}} & \mc{1}{c}{\scriptsize{1.204}} & \mc{1}{c}{\scriptsize{1.145}} \\  

     &  & \mc{1}{c}{\scriptsize{(0.500)}} & \mc{1}{c}{\scriptsize{(0.566)}} & \mc{1}{c}{\scriptsize{(0.263)}} & \mc{1}{c}{\scriptsize{(0.355)}} & \mc{1}{c}{\scriptsize{(0.250)}} & \mc{1}{c}{\scriptsize{(0.684)}} & \mc{1}{c}{\scriptsize{(0.618)}} & \mc{1}{c}{\scriptsize{(0.579)}} \\  

    \mc{1}{l}{\scriptsize{Nutrition}} & \mc{1}{c}{\scriptsize{Mid-30s}} & \mc{1}{c}{\scriptsize{-0.232}} & \mc{1}{c}{\scriptsize{-0.209}} & \mc{1}{c}{\scriptsize{-0.208}} & \mc{1}{c}{\scriptsize{-0.309}} & \mc{1}{c}{\scriptsize{-0.146}} & \mc{1}{c}{\scriptsize{-0.275}} & \mc{1}{c}{\scriptsize{-0.177}} & \mc{1}{c}{\scriptsize{-0.183}} \\  

     &  & \mc{1}{c}{\scriptsize{\textbf{(0.066)}}} & \mc{1}{c}{\scriptsize{(0.145)}} & \mc{1}{c}{\scriptsize{(0.171)}} & \mc{1}{c}{\scriptsize{(0.158)}} & \mc{1}{c}{\scriptsize{(0.197)}} & \mc{1}{c}{\scriptsize{\textbf{(0.053)}}} & \mc{1}{c}{\scriptsize{(0.184)}} & \mc{1}{c}{\scriptsize{(0.145)}} \\  

    \mc{1}{l}{\scriptsize{Posture}} & \mc{1}{c}{\scriptsize{Mid-30s}} &  &  &  &  &  &  &  &  \\  

     &  &  &  &  &  &  &  &  &  \\  

  \bottomrule
  \end{tabular}
	\end{table} 

	\begin{table}[H]
     \caption{Treatment Effects on Physical Exam - General II, Male Sample}
     \label{table:abccare_rslt_male_cat43}
	\input{AppResOutput/abccare/rslt_male_cat43}
	\end{table} 

	\begin{table}[H]
     \caption{Treatment Effects on Physical Exam (Part II), Male Sample}
     \label{table:abccare_rslt_male_cat44}
	  \begin{tabular}{cccccccccc}
  \toprule

    \scriptsize{Variable} & \scriptsize{Age} & \scriptsize{(1)} & \scriptsize{(2)} & \scriptsize{(3)} & \scriptsize{(4)} & \scriptsize{(5)} & \scriptsize{(6)} & \scriptsize{(7)} & \scriptsize{(8)} \\ 
    \midrule  

    \mc{1}{l}{\scriptsize{Behave Appropriate T Score (Reported by Teacher)}} & \mc{1}{c}{\scriptsize{12}} & \mc{1}{c}{\scriptsize{-7.611}} & \mc{1}{c}{\scriptsize{-5.971}} & \mc{1}{c}{\scriptsize{-14.446}} & \mc{1}{c}{\scriptsize{-18.741}} & \mc{1}{c}{\scriptsize{-12.410}} & \mc{1}{c}{\scriptsize{-4.763}} & \mc{1}{c}{\scriptsize{-2.124}} & \mc{1}{c}{\scriptsize{0.020}} \\  

     &  & \mc{1}{c}{\scriptsize{(0.987)}} & \mc{1}{c}{\scriptsize{(0.895)}} & \mc{1}{c}{\scriptsize{(0.974)}} & \mc{1}{c}{\scriptsize{(0.947)}} & \mc{1}{c}{\scriptsize{(0.961)}} & \mc{1}{c}{\scriptsize{(0.868)}} & \mc{1}{c}{\scriptsize{(0.632)}} & \mc{1}{c}{\scriptsize{(0.539)}} \\  

    \mc{1}{l}{\scriptsize{Activities T Score (Reported by Mother)}} & \mc{1}{c}{\scriptsize{12}} & \mc{1}{c}{\scriptsize{0.874}} & \mc{1}{c}{\scriptsize{-0.249}} & \mc{1}{c}{\scriptsize{-1.571}} & \mc{1}{c}{\scriptsize{-0.919}} & \mc{1}{c}{\scriptsize{-2.181}} & \mc{1}{c}{\scriptsize{1.508}} & \mc{1}{c}{\scriptsize{0.320}} & \mc{1}{c}{\scriptsize{0.747}} \\  

     &  & \mc{1}{c}{\scriptsize{(0.316)}} & \mc{1}{c}{\scriptsize{(0.526)}} & \mc{1}{c}{\scriptsize{(0.592)}} & \mc{1}{c}{\scriptsize{(0.539)}} & \mc{1}{c}{\scriptsize{(0.724)}} & \mc{1}{c}{\scriptsize{(0.184)}} & \mc{1}{c}{\scriptsize{(0.447)}} & \mc{1}{c}{\scriptsize{(0.342)}} \\  

    \mc{1}{l}{\scriptsize{Social T Score (Reported by Mother)}} & \mc{1}{c}{\scriptsize{8}} & \mc{1}{c}{\scriptsize{1.497}} & \mc{1}{c}{\scriptsize{2.152}} & \mc{1}{c}{\scriptsize{4.383}} & \mc{1}{c}{\scriptsize{6.379}} & \mc{1}{c}{\scriptsize{4.659}} & \mc{1}{c}{\scriptsize{0.595}} & \mc{1}{c}{\scriptsize{2.009}} & \mc{1}{c}{\scriptsize{2.274}} \\  

     &  & \mc{1}{c}{\scriptsize{(0.224)}} & \mc{1}{c}{\scriptsize{(0.171)}} & \mc{1}{c}{\scriptsize{(0.211)}} & \mc{1}{c}{\scriptsize{(0.211)}} & \mc{1}{c}{\scriptsize{(0.132)}} & \mc{1}{c}{\scriptsize{(0.355)}} & \mc{1}{c}{\scriptsize{(0.211)}} & \mc{1}{c}{\scriptsize{(0.158)}} \\  

    \mc{1}{l}{\scriptsize{Learning T Score (Reported by Teacher)}} & \mc{1}{c}{\scriptsize{12}} & \mc{1}{c}{\scriptsize{-2.244}} & \mc{1}{c}{\scriptsize{2.586}} & \mc{1}{c}{\scriptsize{-5.138}} & \mc{1}{c}{\scriptsize{-6.042}} & \mc{1}{c}{\scriptsize{-3.137}} & \mc{1}{c}{\scriptsize{-1.038}} & \mc{1}{c}{\scriptsize{4.899}} & \mc{1}{c}{\scriptsize{4.696}} \\  

     &  & \mc{1}{c}{\scriptsize{(0.750)}} & \mc{1}{c}{\scriptsize{(0.329)}} & \mc{1}{c}{\scriptsize{(0.842)}} & \mc{1}{c}{\scriptsize{(0.947)}} & \mc{1}{c}{\scriptsize{(0.724)}} & \mc{1}{c}{\scriptsize{(0.618)}} & \mc{1}{c}{\scriptsize{(0.224)}} & \mc{1}{c}{\scriptsize{(0.132)}} \\  

    \mc{1}{l}{\scriptsize{Social T Score (Reported by Mother)}} & \mc{1}{c}{\scriptsize{12}} & \mc{1}{c}{\scriptsize{0.941}} & \mc{1}{c}{\scriptsize{0.866}} & \mc{1}{c}{\scriptsize{2.319}} & \mc{1}{c}{\scriptsize{4.806}} & \mc{1}{c}{\scriptsize{1.699}} & \mc{1}{c}{\scriptsize{0.584}} & \mc{1}{c}{\scriptsize{0.426}} & \mc{1}{c}{\scriptsize{0.130}} \\  

     &  & \mc{1}{c}{\scriptsize{(0.276)}} & \mc{1}{c}{\scriptsize{(0.382)}} & \mc{1}{c}{\scriptsize{(0.263)}} & \mc{1}{c}{\scriptsize{(0.171)}} & \mc{1}{c}{\scriptsize{(0.316)}} & \mc{1}{c}{\scriptsize{(0.408)}} & \mc{1}{c}{\scriptsize{(0.447)}} & \mc{1}{c}{\scriptsize{(0.447)}} \\  

    \mc{1}{l}{\scriptsize{Activities T Score (Reported by Mother)}} & \mc{1}{c}{\scriptsize{8}} & \mc{1}{c}{\scriptsize{1.725}} & \mc{1}{c}{\scriptsize{1.441}} & \mc{1}{c}{\scriptsize{0.191}} & \mc{1}{c}{\scriptsize{-4.476}} & \mc{1}{c}{\scriptsize{1.495}} & \mc{1}{c}{\scriptsize{2.204}} & \mc{1}{c}{\scriptsize{2.458}} & \mc{1}{c}{\scriptsize{0.869}} \\  

     &  & \mc{1}{c}{\scriptsize{(0.224)}} & \mc{1}{c}{\scriptsize{(0.250)}} & \mc{1}{c}{\scriptsize{(0.461)}} & \mc{1}{c}{\scriptsize{(0.724)}} & \mc{1}{c}{\scriptsize{(0.368)}} & \mc{1}{c}{\scriptsize{(0.197)}} & \mc{1}{c}{\scriptsize{\textbf{(0.066)}}} & \mc{1}{c}{\scriptsize{(0.342)}} \\  

    \mc{1}{l}{\scriptsize{Work Hard T Score (Reported by Teacher)}} & \mc{1}{c}{\scriptsize{12}} & \mc{1}{c}{\scriptsize{-3.063}} & \mc{1}{c}{\scriptsize{2.048}} & \mc{1}{c}{\scriptsize{-8.769}} & \mc{1}{c}{\scriptsize{-9.480}} & \mc{1}{c}{\scriptsize{-6.456}} & \mc{1}{c}{\scriptsize{-0.686}} & \mc{1}{c}{\scriptsize{6.128}} & \mc{1}{c}{\scriptsize{4.633}} \\  

     &  & \mc{1}{c}{\scriptsize{(0.816)}} & \mc{1}{c}{\scriptsize{(0.316)}} & \mc{1}{c}{\scriptsize{(0.921)}} & \mc{1}{c}{\scriptsize{(0.947)}} & \mc{1}{c}{\scriptsize{(0.855)}} & \mc{1}{c}{\scriptsize{(0.605)}} & \mc{1}{c}{\scriptsize{\textbf{(0.066)}}} & \mc{1}{c}{\scriptsize{\textbf{(0.066)}}} \\  

    \mc{1}{l}{\scriptsize{Happiness T Score (Reported by Teacher)}} & \mc{1}{c}{\scriptsize{12}} & \mc{1}{c}{\scriptsize{-8.280}} & \mc{1}{c}{\scriptsize{-2.655}} & \mc{1}{c}{\scriptsize{-9.292}} & \mc{1}{c}{\scriptsize{-11.163}} & \mc{1}{c}{\scriptsize{-7.530}} & \mc{1}{c}{\scriptsize{-7.859}} & \mc{1}{c}{\scriptsize{-0.711}} & \mc{1}{c}{\scriptsize{-6.434}} \\  

     &  & \mc{1}{c}{\scriptsize{(0.987)}} & \mc{1}{c}{\scriptsize{(0.684)}} & \mc{1}{c}{\scriptsize{(0.934)}} & \mc{1}{c}{\scriptsize{(0.947)}} & \mc{1}{c}{\scriptsize{(0.855)}} & \mc{1}{c}{\scriptsize{(0.987)}} & \mc{1}{c}{\scriptsize{(0.645)}} & \mc{1}{c}{\scriptsize{(0.961)}} \\  

  \bottomrule
  \end{tabular}
	\end{table} 

	\begin{table}[H]
     \caption{Treatment Effects on Age 21 Brief Symptom Inventory, Male Sample}
     \label{table:abccare_rslt_male_cat45}
	  \begin{tabular}{cccccccccc}
  \toprule

    \scriptsize{Variable} & \scriptsize{Age} & \scriptsize{(1)} & \scriptsize{(2)} & \scriptsize{(3)} & \scriptsize{(4)} & \scriptsize{(5)} & \scriptsize{(6)} & \scriptsize{(7)} & \scriptsize{(8)} \\ 
    \midrule  

    \mc{1}{l}{\scriptsize{Paranoid Ideation}} & \mc{1}{c}{\scriptsize{21}} & \mc{1}{c}{\scriptsize{1.818}} & \mc{1}{c}{\scriptsize{2.449}} & \mc{1}{c}{\scriptsize{1.788}} & \mc{1}{c}{\scriptsize{3.955}} & \mc{1}{c}{\scriptsize{2.021}} & \mc{1}{c}{\scriptsize{2.188}} & \mc{1}{c}{\scriptsize{2.899}} & \mc{1}{c}{\scriptsize{2.022}} \\  

     &  & \mc{1}{c}{\scriptsize{(0.803)}} & \mc{1}{c}{\scriptsize{(0.842)}} & \mc{1}{c}{\scriptsize{(0.750)}} & \mc{1}{c}{\scriptsize{(0.776)}} & \mc{1}{c}{\scriptsize{(0.789)}} & \mc{1}{c}{\scriptsize{(0.803)}} & \mc{1}{c}{\scriptsize{(0.855)}} & \mc{1}{c}{\scriptsize{(0.776)}} \\  

    \mc{1}{l}{\scriptsize{Obsessive-Compulsive}} & \mc{1}{c}{\scriptsize{21}} & \mc{1}{c}{\scriptsize{-1.400}} & \mc{1}{c}{\scriptsize{-1.428}} & \mc{1}{c}{\scriptsize{1.171}} & \mc{1}{c}{\scriptsize{1.171}} & \mc{1}{c}{\scriptsize{1.434}} & \mc{1}{c}{\scriptsize{-1.983}} & \mc{1}{c}{\scriptsize{-2.333}} & \mc{1}{c}{\scriptsize{-1.789}} \\  

     &  & \mc{1}{c}{\scriptsize{(0.250)}} & \mc{1}{c}{\scriptsize{(0.224)}} & \mc{1}{c}{\scriptsize{(0.711)}} & \mc{1}{c}{\scriptsize{(0.592)}} & \mc{1}{c}{\scriptsize{(0.750)}} & \mc{1}{c}{\scriptsize{(0.171)}} & \mc{1}{c}{\scriptsize{(0.158)}} & \mc{1}{c}{\scriptsize{(0.224)}} \\  

    \mc{1}{l}{\scriptsize{Interpersonal Sense}} & \mc{1}{c}{\scriptsize{21}} & \mc{1}{c}{\scriptsize{0.184}} & \mc{1}{c}{\scriptsize{-1.189}} & \mc{1}{c}{\scriptsize{1.727}} & \mc{1}{c}{\scriptsize{-0.954}} & \mc{1}{c}{\scriptsize{0.897}} & \mc{1}{c}{\scriptsize{-0.236}} & \mc{1}{c}{\scriptsize{-1.605}} & \mc{1}{c}{\scriptsize{-1.324}} \\  

     &  & \mc{1}{c}{\scriptsize{(0.461)}} & \mc{1}{c}{\scriptsize{(0.329)}} & \mc{1}{c}{\scriptsize{(0.776)}} & \mc{1}{c}{\scriptsize{(0.421)}} & \mc{1}{c}{\scriptsize{(0.632)}} & \mc{1}{c}{\scriptsize{(0.421)}} & \mc{1}{c}{\scriptsize{(0.289)}} & \mc{1}{c}{\scriptsize{(0.263)}} \\  

    \mc{1}{l}{\scriptsize{Positive Symptom Distress Index (PSI)}} & \mc{1}{c}{\scriptsize{21}} & \mc{1}{c}{\scriptsize{0.441}} & \mc{1}{c}{\scriptsize{0.511}} & \mc{1}{c}{\scriptsize{4.571}} & \mc{1}{c}{\scriptsize{4.242}} & \mc{1}{c}{\scriptsize{4.193}} & \mc{1}{c}{\scriptsize{-0.154}} & \mc{1}{c}{\scriptsize{0.119}} & \mc{1}{c}{\scriptsize{-0.701}} \\  

     &  & \mc{1}{c}{\scriptsize{(0.539)}} & \mc{1}{c}{\scriptsize{(0.645)}} & \mc{1}{c}{\scriptsize{(0.882)}} & \mc{1}{c}{\scriptsize{(0.829)}} & \mc{1}{c}{\scriptsize{(0.868)}} & \mc{1}{c}{\scriptsize{(0.421)}} & \mc{1}{c}{\scriptsize{(0.500)}} & \mc{1}{c}{\scriptsize{(0.303)}} \\  

    \mc{1}{l}{\scriptsize{Psychoticism}} & \mc{1}{c}{\scriptsize{21}} & \mc{1}{c}{\scriptsize{-3.613}} & \mc{1}{c}{\scriptsize{-2.961}} & \mc{1}{c}{\scriptsize{-3.277}} & \mc{1}{c}{\scriptsize{-0.797}} & \mc{1}{c}{\scriptsize{-3.331}} & \mc{1}{c}{\scriptsize{-3.002}} & \mc{1}{c}{\scriptsize{-2.696}} & \mc{1}{c}{\scriptsize{-3.212}} \\  

     &  & \mc{1}{c}{\scriptsize{\textbf{(0.079)}}} & \mc{1}{c}{\scriptsize{(0.197)}} & \mc{1}{c}{\scriptsize{(0.237)}} & \mc{1}{c}{\scriptsize{(0.408)}} & \mc{1}{c}{\scriptsize{(0.184)}} & \mc{1}{c}{\scriptsize{(0.118)}} & \mc{1}{c}{\scriptsize{(0.224)}} & \mc{1}{c}{\scriptsize{\textbf{(0.066)}}} \\  

    \mc{1}{l}{\scriptsize{Phobic Anxiety}} & \mc{1}{c}{\scriptsize{21}} & \mc{1}{c}{\scriptsize{-1.702}} & \mc{1}{c}{\scriptsize{-4.387}} & \mc{1}{c}{\scriptsize{-0.445}} & \mc{1}{c}{\scriptsize{-7.741}} & \mc{1}{c}{\scriptsize{-1.248}} & \mc{1}{c}{\scriptsize{-1.810}} & \mc{1}{c}{\scriptsize{-3.622}} & \mc{1}{c}{\scriptsize{-3.322}} \\  

     &  & \mc{1}{c}{\scriptsize{(0.171)}} & \mc{1}{c}{\scriptsize{\textbf{(0.026)}}} & \mc{1}{c}{\scriptsize{(0.395)}} & \mc{1}{c}{\scriptsize{\textbf{(0.026)}}} & \mc{1}{c}{\scriptsize{(0.303)}} & \mc{1}{c}{\scriptsize{(0.237)}} & \mc{1}{c}{\scriptsize{(0.105)}} & \mc{1}{c}{\scriptsize{\textbf{(0.079)}}} \\  

    \mc{1}{l}{\scriptsize{Positive Symptom Total (PST)}} & \mc{1}{c}{\scriptsize{21}} & \mc{1}{c}{\scriptsize{0.362}} & \mc{1}{c}{\scriptsize{-0.785}} & \mc{1}{c}{\scriptsize{2.191}} & \mc{1}{c}{\scriptsize{-0.581}} & \mc{1}{c}{\scriptsize{1.528}} & \mc{1}{c}{\scriptsize{0.111}} & \mc{1}{c}{\scriptsize{-1.100}} & \mc{1}{c}{\scriptsize{-0.684}} \\  

     &  & \mc{1}{c}{\scriptsize{(0.592)}} & \mc{1}{c}{\scriptsize{(0.395)}} & \mc{1}{c}{\scriptsize{(0.803)}} & \mc{1}{c}{\scriptsize{(0.421)}} & \mc{1}{c}{\scriptsize{(0.724)}} & \mc{1}{c}{\scriptsize{(0.553)}} & \mc{1}{c}{\scriptsize{(0.368)}} & \mc{1}{c}{\scriptsize{(0.421)}} \\  

  \bottomrule
  \end{tabular}
	\end{table} 

	\begin{table}[H]
     \caption{Treatment Effects on Age 30 Adult Self Report DSM Scale $t$-Score, Male Sample}
     \label{table:abccare_rslt_male_cat46}
	\input{AppResOutput/abccare/rslt_male_cat46}
	\end{table} 

	\begin{table}[H]
     \caption{Treatment Effects on Age 30 Adult Self Report Syndrome Scale $t$-Score, Male Sample}
     \label{table:abccare_rslt_male_cat47}
	\input{AppResOutput/abccare/rslt_male_cat47}
	\end{table} 

	\begin{table}[H]
     \caption{Treatment Effects on BSI 18 $t$-Score, Male Sample}
     \label{table:abccare_rslt_male_cat48}
	  \begin{tabular}{cccccccccc}
  \toprule

    \scriptsize{Variable} & \scriptsize{Age} & \scriptsize{(1)} & \scriptsize{(2)} & \scriptsize{(3)} & \scriptsize{(4)} & \scriptsize{(5)} & \scriptsize{(6)} & \scriptsize{(7)} & \scriptsize{(8)} \\ 
    \midrule  

    \mc{1}{l}{\scriptsize{Global Severity Index}} & \mc{1}{c}{\scriptsize{Mid-30s}} & \mc{1}{c}{\scriptsize{-1.675}} & \mc{1}{c}{\scriptsize{-2.999}} & \mc{1}{c}{\scriptsize{0.111}} & \mc{1}{c}{\scriptsize{1.730}} & \mc{1}{c}{\scriptsize{-0.607}} & \mc{1}{c}{\scriptsize{-2.989}} & \mc{1}{c}{\scriptsize{-4.200}} & \mc{1}{c}{\scriptsize{-2.822}} \\  

     &  & \mc{1}{c}{\scriptsize{(0.329)}} & \mc{1}{c}{\scriptsize{(0.289)}} & \mc{1}{c}{\scriptsize{(0.447)}} & \mc{1}{c}{\scriptsize{(0.618)}} & \mc{1}{c}{\scriptsize{(0.355)}} & \mc{1}{c}{\scriptsize{(0.316)}} & \mc{1}{c}{\scriptsize{(0.276)}} & \mc{1}{c}{\scriptsize{(0.289)}} \\  

    \mc{1}{l}{\scriptsize{Somatization}} & \mc{1}{c}{\scriptsize{Mid-30s}} & \mc{1}{c}{\scriptsize{-2.823}} & \mc{1}{c}{\scriptsize{-3.958}} & \mc{1}{c}{\scriptsize{-1.704}} & \mc{1}{c}{\scriptsize{-3.058}} & \mc{1}{c}{\scriptsize{-2.144}} & \mc{1}{c}{\scriptsize{-3.737}} & \mc{1}{c}{\scriptsize{-4.107}} & \mc{1}{c}{\scriptsize{-3.567}} \\  

     &  & \mc{1}{c}{\scriptsize{(0.211)}} & \mc{1}{c}{\scriptsize{(0.184)}} & \mc{1}{c}{\scriptsize{(0.211)}} & \mc{1}{c}{\scriptsize{(0.171)}} & \mc{1}{c}{\scriptsize{(0.197)}} & \mc{1}{c}{\scriptsize{(0.184)}} & \mc{1}{c}{\scriptsize{(0.211)}} & \mc{1}{c}{\scriptsize{(0.211)}} \\  

    \mc{1}{l}{\scriptsize{Anxiety}} & \mc{1}{c}{\scriptsize{Mid-30s}} & \mc{1}{c}{\scriptsize{-1.508}} & \mc{1}{c}{\scriptsize{-2.915}} & \mc{1}{c}{\scriptsize{1.111}} & \mc{1}{c}{\scriptsize{2.708}} & \mc{1}{c}{\scriptsize{0.439}} & \mc{1}{c}{\scriptsize{-2.822}} & \mc{1}{c}{\scriptsize{-3.931}} & \mc{1}{c}{\scriptsize{-2.694}} \\  

     &  & \mc{1}{c}{\scriptsize{(0.316)}} & \mc{1}{c}{\scriptsize{(0.237)}} & \mc{1}{c}{\scriptsize{(0.605)}} & \mc{1}{c}{\scriptsize{(0.724)}} & \mc{1}{c}{\scriptsize{(0.526)}} & \mc{1}{c}{\scriptsize{(0.263)}} & \mc{1}{c}{\scriptsize{(0.171)}} & \mc{1}{c}{\scriptsize{(0.276)}} \\  

    \mc{1}{l}{\scriptsize{Depression}} & \mc{1}{c}{\scriptsize{Mid-30s}} & \mc{1}{c}{\scriptsize{-1.135}} & \mc{1}{c}{\scriptsize{-2.524}} & \mc{1}{c}{\scriptsize{3.222}} & \mc{1}{c}{\scriptsize{5.281}} & \mc{1}{c}{\scriptsize{2.240}} & \mc{1}{c}{\scriptsize{-2.978}} & \mc{1}{c}{\scriptsize{-4.227}} & \mc{1}{c}{\scriptsize{-2.873}} \\  

     &  & \mc{1}{c}{\scriptsize{(0.368)}} & \mc{1}{c}{\scriptsize{(0.316)}} & \mc{1}{c}{\scriptsize{(0.855)}} & \mc{1}{c}{\scriptsize{(0.816)}} & \mc{1}{c}{\scriptsize{(0.803)}} & \mc{1}{c}{\scriptsize{(0.303)}} & \mc{1}{c}{\scriptsize{(0.171)}} & \mc{1}{c}{\scriptsize{(0.276)}} \\  

  \bottomrule
  \end{tabular}
	\end{table} 

	\begin{table}[H]
     \caption{Treatment Effects on BSI Raw Score, Male Sample}
     \label{table:abccare_rslt_male_cat49}
	\input{AppResOutput/abccare/rslt_male_cat49}
	\end{table} 

	\begin{table}[H]
     \caption{Treatment Effects on BSI $t$-Score, Male Sample}
     \label{table:abccare_rslt_male_cat50}
	  \begin{tabular}{cccccccccc}
  \toprule

    \scriptsize{Variable} & \scriptsize{Age} & \scriptsize{(1)} & \scriptsize{(2)} & \scriptsize{(3)} & \scriptsize{(4)} & \scriptsize{(5)} & \scriptsize{(6)} & \scriptsize{(7)} & \scriptsize{(8)} \\ 
    \midrule  

    \mc{1}{l}{\scriptsize{Depression $t$-Score}} & \mc{1}{c}{\scriptsize{Mid-30s}} & \mc{1}{c}{\scriptsize{-1.042}} & \mc{1}{c}{\scriptsize{-2.585}} & \mc{1}{c}{\scriptsize{3.148}} & \mc{1}{c}{\scriptsize{6.283}} & \mc{1}{c}{\scriptsize{1.931}} & \mc{1}{c}{\scriptsize{-2.985}} & \mc{1}{c}{\scriptsize{-4.699}} & \mc{1}{c}{\scriptsize{-2.990}} \\  

     &  & \mc{1}{c}{\scriptsize{(0.382)}} & \mc{1}{c}{\scriptsize{(0.303)}} & \mc{1}{c}{\scriptsize{(0.763)}} & \mc{1}{c}{\scriptsize{(0.789)}} & \mc{1}{c}{\scriptsize{(0.684)}} & \mc{1}{c}{\scriptsize{(0.316)}} & \mc{1}{c}{\scriptsize{(0.197)}} & \mc{1}{c}{\scriptsize{(0.276)}} \\  

     & \mc{1}{c}{\scriptsize{21}} & \mc{1}{c}{\scriptsize{-2.515}} & \mc{1}{c}{\scriptsize{-0.640}} & \mc{1}{c}{\scriptsize{1.649}} & \mc{1}{c}{\scriptsize{6.173}} & \mc{1}{c}{\scriptsize{1.715}} & \mc{1}{c}{\scriptsize{-3.636}} & \mc{1}{c}{\scriptsize{-2.114}} & \mc{1}{c}{\scriptsize{-3.132}} \\  

     &  & \mc{1}{c}{\scriptsize{(0.105)}} & \mc{1}{c}{\scriptsize{(0.408)}} & \mc{1}{c}{\scriptsize{(0.566)}} & \mc{1}{c}{\scriptsize{(0.934)}} & \mc{1}{c}{\scriptsize{(0.579)}} & \mc{1}{c}{\scriptsize{\textbf{(0.066)}}} & \mc{1}{c}{\scriptsize{(0.263)}} & \mc{1}{c}{\scriptsize{\textbf{(0.092)}}} \\  

    \mc{1}{l}{\scriptsize{Somatization $t$-Score}} & \mc{1}{c}{\scriptsize{21}} & \mc{1}{c}{\scriptsize{-2.804}} & \mc{1}{c}{\scriptsize{-4.232}} & \mc{1}{c}{\scriptsize{-3.719}} & \mc{1}{c}{\scriptsize{-5.059}} & \mc{1}{c}{\scriptsize{-4.295}} & \mc{1}{c}{\scriptsize{-2.295}} & \mc{1}{c}{\scriptsize{-4.404}} & \mc{1}{c}{\scriptsize{-3.843}} \\  

     &  & \mc{1}{c}{\scriptsize{\textbf{(0.079)}}} & \mc{1}{c}{\scriptsize{\textbf{(0.026)}}} & \mc{1}{c}{\scriptsize{(0.118)}} & \mc{1}{c}{\scriptsize{\textbf{(0.079)}}} & \mc{1}{c}{\scriptsize{\textbf{(0.053)}}} & \mc{1}{c}{\scriptsize{(0.132)}} & \mc{1}{c}{\scriptsize{\textbf{(0.039)}}} & \mc{1}{c}{\scriptsize{\textbf{(0.026)}}} \\  

    \mc{1}{l}{\scriptsize{Anxiety $t$-Score}} & \mc{1}{c}{\scriptsize{21}} & \mc{1}{c}{\scriptsize{0.400}} & \mc{1}{c}{\scriptsize{-0.425}} & \mc{1}{c}{\scriptsize{3.857}} & \mc{1}{c}{\scriptsize{3.077}} & \mc{1}{c}{\scriptsize{3.446}} & \mc{1}{c}{\scriptsize{-0.333}} & \mc{1}{c}{\scriptsize{-1.670}} & \mc{1}{c}{\scriptsize{-1.393}} \\  

     &  & \mc{1}{c}{\scriptsize{(0.539)}} & \mc{1}{c}{\scriptsize{(0.474)}} & \mc{1}{c}{\scriptsize{(0.868)}} & \mc{1}{c}{\scriptsize{(0.750)}} & \mc{1}{c}{\scriptsize{(0.868)}} & \mc{1}{c}{\scriptsize{(0.487)}} & \mc{1}{c}{\scriptsize{(0.263)}} & \mc{1}{c}{\scriptsize{(0.237)}} \\  

    \mc{1}{l}{\scriptsize{Somatization $t$-Score}} & \mc{1}{c}{\scriptsize{Mid-30s}} & \mc{1}{c}{\scriptsize{-3.066}} & \mc{1}{c}{\scriptsize{-3.900}} & \mc{1}{c}{\scriptsize{-4.852}} & \mc{1}{c}{\scriptsize{-5.571}} & \mc{1}{c}{\scriptsize{-4.945}} & \mc{1}{c}{\scriptsize{-3.252}} & \mc{1}{c}{\scriptsize{-3.671}} & \mc{1}{c}{\scriptsize{-3.068}} \\  

     &  & \mc{1}{c}{\scriptsize{(0.224)}} & \mc{1}{c}{\scriptsize{(0.211)}} & \mc{1}{c}{\scriptsize{(0.105)}} & \mc{1}{c}{\scriptsize{(0.145)}} & \mc{1}{c}{\scriptsize{(0.132)}} & \mc{1}{c}{\scriptsize{(0.276)}} & \mc{1}{c}{\scriptsize{(0.316)}} & \mc{1}{c}{\scriptsize{(0.289)}} \\  

    \mc{1}{l}{\scriptsize{Hostility $t$-Score}} & \mc{1}{c}{\scriptsize{Mid-30s}} & \mc{1}{c}{\scriptsize{-1.556}} & \mc{1}{c}{\scriptsize{-2.845}} & \mc{1}{c}{\scriptsize{-1.889}} & \mc{1}{c}{\scriptsize{-1.665}} & \mc{1}{c}{\scriptsize{-2.737}} & \mc{1}{c}{\scriptsize{-2.156}} & \mc{1}{c}{\scriptsize{-3.313}} & \mc{1}{c}{\scriptsize{-2.513}} \\  

     &  & \mc{1}{c}{\scriptsize{(0.329)}} & \mc{1}{c}{\scriptsize{(0.289)}} & \mc{1}{c}{\scriptsize{(0.316)}} & \mc{1}{c}{\scriptsize{(0.316)}} & \mc{1}{c}{\scriptsize{(0.237)}} & \mc{1}{c}{\scriptsize{(0.316)}} & \mc{1}{c}{\scriptsize{(0.289)}} & \mc{1}{c}{\scriptsize{(0.303)}} \\  

    \mc{1}{l}{\scriptsize{Global Severity Index $t$-Score}} & \mc{1}{c}{\scriptsize{21}} & \mc{1}{c}{\scriptsize{0.246}} & \mc{1}{c}{\scriptsize{0.838}} & \mc{1}{c}{\scriptsize{1.978}} & \mc{1}{c}{\scriptsize{5.044}} & \mc{1}{c}{\scriptsize{1.752}} & \mc{1}{c}{\scriptsize{0.330}} & \mc{1}{c}{\scriptsize{0.510}} & \mc{1}{c}{\scriptsize{0.049}} \\  

     &  & \mc{1}{c}{\scriptsize{(0.539)}} & \mc{1}{c}{\scriptsize{(0.684)}} & \mc{1}{c}{\scriptsize{(0.750)}} & \mc{1}{c}{\scriptsize{(0.816)}} & \mc{1}{c}{\scriptsize{(0.724)}} & \mc{1}{c}{\scriptsize{(0.592)}} & \mc{1}{c}{\scriptsize{(0.553)}} & \mc{1}{c}{\scriptsize{(0.461)}} \\  

    \mc{1}{l}{\scriptsize{Anxiety $t$-Score}} & \mc{1}{c}{\scriptsize{Mid-30s}} & \mc{1}{c}{\scriptsize{-1.847}} & \mc{1}{c}{\scriptsize{-3.594}} & \mc{1}{c}{\scriptsize{1.630}} & \mc{1}{c}{\scriptsize{3.191}} & \mc{1}{c}{\scriptsize{0.711}} & \mc{1}{c}{\scriptsize{-3.504}} & \mc{1}{c}{\scriptsize{-4.841}} & \mc{1}{c}{\scriptsize{-3.410}} \\  

     &  & \mc{1}{c}{\scriptsize{(0.316)}} & \mc{1}{c}{\scriptsize{(0.237)}} & \mc{1}{c}{\scriptsize{(0.684)}} & \mc{1}{c}{\scriptsize{(0.737)}} & \mc{1}{c}{\scriptsize{(0.566)}} & \mc{1}{c}{\scriptsize{(0.263)}} & \mc{1}{c}{\scriptsize{(0.171)}} & \mc{1}{c}{\scriptsize{(0.263)}} \\  

    \mc{1}{l}{\scriptsize{Hostility $t$-Score}} & \mc{1}{c}{\scriptsize{21}} & \mc{1}{c}{\scriptsize{-1.471}} & \mc{1}{c}{\scriptsize{0.130}} & \mc{1}{c}{\scriptsize{2.941}} & \mc{1}{c}{\scriptsize{4.476}} & \mc{1}{c}{\scriptsize{2.704}} & \mc{1}{c}{\scriptsize{-2.251}} & \mc{1}{c}{\scriptsize{-0.489}} & \mc{1}{c}{\scriptsize{-1.828}} \\  

     &  & \mc{1}{c}{\scriptsize{(0.211)}} & \mc{1}{c}{\scriptsize{(0.513)}} & \mc{1}{c}{\scriptsize{(0.763)}} & \mc{1}{c}{\scriptsize{(0.763)}} & \mc{1}{c}{\scriptsize{(0.750)}} & \mc{1}{c}{\scriptsize{(0.158)}} & \mc{1}{c}{\scriptsize{(0.461)}} & \mc{1}{c}{\scriptsize{(0.237)}} \\  

  \bottomrule
  \end{tabular}
	\end{table} 

	\begin{table}[H]
     \caption{Treatment Effects on Mid-30s Mental Health Conditions, Male Sample}
     \label{table:abccare_rslt_male_cat51}
	  \begin{tabular}{cccccccccc}
  \toprule

    \scriptsize{Variable} & \scriptsize{Age} & \scriptsize{(1)} & \scriptsize{(2)} & \scriptsize{(3)} & \scriptsize{(4)} & \scriptsize{(5)} & \scriptsize{(6)} & \scriptsize{(7)} & \scriptsize{(8)} \\ 
    \midrule  

    \mc{1}{l}{\scriptsize{Social acceptance}} & \mc{1}{c}{\scriptsize{12}} & \mc{1}{c}{\scriptsize{0.424}} & \mc{1}{c}{\scriptsize{0.620}} & \mc{1}{c}{\scriptsize{-0.118}} & \mc{1}{c}{\scriptsize{-0.051}} & \mc{1}{c}{\scriptsize{-0.129}} & \mc{1}{c}{\scriptsize{0.605}} & \mc{1}{c}{\scriptsize{0.958}} & \mc{1}{c}{\scriptsize{0.838}} \\  

     &  & \mc{1}{c}{\scriptsize{\textbf{(0.066)}}} & \mc{1}{c}{\scriptsize{(0.118)}} & \mc{1}{c}{\scriptsize{(0.487)}} & \mc{1}{c}{\scriptsize{(0.421)}} & \mc{1}{c}{\scriptsize{(0.421)}} & \mc{1}{c}{\scriptsize{\textbf{(0.013)}}} & \mc{1}{c}{\scriptsize{\textbf{(0.026)}}} & \mc{1}{c}{\scriptsize{\textbf{(0.013)}}} \\  

    \mc{1}{l}{\scriptsize{Behavioral conduct}} & \mc{1}{c}{\scriptsize{12}} & \mc{1}{c}{\scriptsize{-0.017}} & \mc{1}{c}{\scriptsize{0.226}} & \mc{1}{c}{\scriptsize{-0.226}} & \mc{1}{c}{\scriptsize{0.047}} & \mc{1}{c}{\scriptsize{-0.180}} & \mc{1}{c}{\scriptsize{0.052}} & \mc{1}{c}{\scriptsize{0.356}} & \mc{1}{c}{\scriptsize{0.094}} \\  

     &  & \mc{1}{c}{\scriptsize{(0.487)}} & \mc{1}{c}{\scriptsize{(0.197)}} & \mc{1}{c}{\scriptsize{(0.632)}} & \mc{1}{c}{\scriptsize{(0.434)}} & \mc{1}{c}{\scriptsize{(0.553)}} & \mc{1}{c}{\scriptsize{(0.395)}} & \mc{1}{c}{\scriptsize{(0.118)}} & \mc{1}{c}{\scriptsize{(0.382)}} \\  

    \mc{1}{l}{\scriptsize{Physical appearance}} & \mc{1}{c}{\scriptsize{12}} & \mc{1}{c}{\scriptsize{0.449}} & \mc{1}{c}{\scriptsize{0.503}} & \mc{1}{c}{\scriptsize{0.490}} & \mc{1}{c}{\scriptsize{0.693}} & \mc{1}{c}{\scriptsize{0.533}} & \mc{1}{c}{\scriptsize{0.435}} & \mc{1}{c}{\scriptsize{0.420}} & \mc{1}{c}{\scriptsize{0.504}} \\  

     &  & \mc{1}{c}{\scriptsize{\textbf{(0.053)}}} & \mc{1}{c}{\scriptsize{(0.118)}} & \mc{1}{c}{\scriptsize{(0.145)}} & \mc{1}{c}{\scriptsize{(0.145)}} & \mc{1}{c}{\scriptsize{(0.105)}} & \mc{1}{c}{\scriptsize{\textbf{(0.053)}}} & \mc{1}{c}{\scriptsize{(0.224)}} & \mc{1}{c}{\scriptsize{\textbf{(0.066)}}} \\  

  \bottomrule
  \end{tabular}
	\end{table} 

	\begin{table}[H]
     \caption{Treatment Effects on Smoking and Drinking Behavior, Male Sample}
     \label{table:abccare_rslt_male_cat52}
	  \begin{tabular}{cccccccccc}
  \toprule

    \scriptsize{Variable} & \scriptsize{Age} & \scriptsize{(1)} & \scriptsize{(2)} & \scriptsize{(3)} & \scriptsize{(4)} & \scriptsize{(5)} & \scriptsize{(6)} & \scriptsize{(7)} & \scriptsize{(8)} \\ 
    \midrule  

    \mc{1}{l}{\scriptsize{Problems Due to Alcohol or Drugs}} & \mc{1}{c}{\scriptsize{12}} & \mc{1}{c}{\scriptsize{0.039}} & \mc{1}{c}{\scriptsize{0.042}} & \mc{1}{c}{\scriptsize{-0.076}} & \mc{1}{c}{\scriptsize{-0.147}} & \mc{1}{c}{\scriptsize{-0.083}} & \mc{1}{c}{\scriptsize{0.065}} & \mc{1}{c}{\scriptsize{0.068}} & \mc{1}{c}{\scriptsize{0.069}} \\  

     &  & \mc{1}{c}{\scriptsize{(0.632)}} & \mc{1}{c}{\scriptsize{(0.592)}} & \mc{1}{c}{\scriptsize{(0.395)}} & \mc{1}{c}{\scriptsize{(0.211)}} & \mc{1}{c}{\scriptsize{(0.368)}} & \mc{1}{c}{\scriptsize{(0.684)}} & \mc{1}{c}{\scriptsize{(0.671)}} & \mc{1}{c}{\scriptsize{(0.684)}} \\  

    \mc{1}{l}{\scriptsize{Used Alcohol and/or Drugs}} & \mc{1}{c}{\scriptsize{12}} & \mc{1}{c}{\scriptsize{0.060}} & \mc{1}{c}{\scriptsize{0.092}} & \mc{1}{c}{\scriptsize{0.118}} & \mc{1}{c}{\scriptsize{0.243}} & \mc{1}{c}{\scriptsize{0.138}} & \mc{1}{c}{\scriptsize{0.046}} & \mc{1}{c}{\scriptsize{0.073}} & \mc{1}{c}{\scriptsize{0.061}} \\  

     &  & \mc{1}{c}{\scriptsize{(0.816)}} & \mc{1}{c}{\scriptsize{(0.855)}} & \mc{1}{c}{\scriptsize{(0.934)}} & \mc{1}{c}{\scriptsize{(0.974)}} & \mc{1}{c}{\scriptsize{(0.961)}} & \mc{1}{c}{\scriptsize{(0.724)}} & \mc{1}{c}{\scriptsize{(0.816)}} & \mc{1}{c}{\scriptsize{(0.763)}} \\  

  \bottomrule
  \end{tabular}
	\end{table} 

	\begin{table}[H]
     \caption{Treatment Effects on Tobacco, Drugs, Alcohol, Male Sample}
     \label{table:abccare_rslt_male_cat53}
	  \begin{tabular}{cccccccccc}
  \toprule

    \scriptsize{Variable} & \scriptsize{Age} & \scriptsize{(1)} & \scriptsize{(2)} & \scriptsize{(3)} & \scriptsize{(4)} & \scriptsize{(5)} & \scriptsize{(6)} & \scriptsize{(7)} & \scriptsize{(8)} \\ 
    \midrule  

    \mc{1}{l}{\scriptsize{Days drank alcohol last month}} & \mc{1}{c}{\scriptsize{30}} & \mc{1}{c}{\scriptsize{0.805}} & \mc{1}{c}{\scriptsize{1.083}} & \mc{1}{c}{\scriptsize{-0.186}} & \mc{1}{c}{\scriptsize{-0.648}} & \mc{1}{c}{\scriptsize{0.060}} & \mc{1}{c}{\scriptsize{0.944}} & \mc{1}{c}{\scriptsize{1.348}} & \mc{1}{c}{\scriptsize{1.341}} \\  

     &  & \mc{1}{c}{\scriptsize{(0.711)}} & \mc{1}{c}{\scriptsize{(0.645)}} & \mc{1}{c}{\scriptsize{(0.408)}} & \mc{1}{c}{\scriptsize{(0.355)}} & \mc{1}{c}{\scriptsize{(0.421)}} & \mc{1}{c}{\scriptsize{(0.711)}} & \mc{1}{c}{\scriptsize{(0.724)}} & \mc{1}{c}{\scriptsize{(0.776)}} \\  

    \mc{1}{l}{\scriptsize{Days binge drank alcohol last month}} & \mc{1}{c}{\scriptsize{30}} & \mc{1}{c}{\scriptsize{0.500}} & \mc{1}{c}{\scriptsize{0.474}} & \mc{1}{c}{\scriptsize{0.543}} & \mc{1}{c}{\scriptsize{-0.360}} & \mc{1}{c}{\scriptsize{0.692}} & \mc{1}{c}{\scriptsize{0.491}} & \mc{1}{c}{\scriptsize{0.625}} & \mc{1}{c}{\scriptsize{0.702}} \\  

     &  & \mc{1}{c}{\scriptsize{(0.816)}} & \mc{1}{c}{\scriptsize{(0.816)}} & \mc{1}{c}{\scriptsize{(0.724)}} & \mc{1}{c}{\scriptsize{(0.368)}} & \mc{1}{c}{\scriptsize{(0.829)}} & \mc{1}{c}{\scriptsize{(0.789)}} & \mc{1}{c}{\scriptsize{(0.803)}} & \mc{1}{c}{\scriptsize{(0.868)}} \\  

  \bottomrule
  \end{tabular}
	\end{table} 
\section{Treatment Effects for Female Sample}


	\begin{table}[H]
     \caption{Treatment Effects on IQ Scores, Female Sample}
     \label{table:abccare_rslt_female_cat0}
	  \begin{tabular}{cccccccccc}
  \toprule

    \scriptsize{Variable} & \scriptsize{Age} & \scriptsize{(1)} & \scriptsize{(2)} & \scriptsize{(3)} & \scriptsize{(4)} & \scriptsize{(5)} & \scriptsize{(6)} & \scriptsize{(7)} & \scriptsize{(8)} \\ 
    \midrule  

    \mc{1}{l}{\scriptsize{Std. IQ Test}} & \mc{1}{c}{\scriptsize{2}} & \mc{1}{c}{\scriptsize{9.600}} & \mc{1}{c}{\scriptsize{9.750}} & \mc{1}{c}{\scriptsize{14.881}} & \mc{1}{c}{\scriptsize{16.647}} & \mc{1}{c}{\scriptsize{14.839}} & \mc{1}{c}{\scriptsize{7.333}} & \mc{1}{c}{\scriptsize{6.963}} & \mc{1}{c}{\scriptsize{7.193}} \\  

     &  & \mc{1}{c}{\scriptsize{\textbf{(0.000)}}} & \mc{1}{c}{\scriptsize{\textbf{(0.000)}}} & \mc{1}{c}{\scriptsize{\textbf{(0.000)}}} & \mc{1}{c}{\scriptsize{\textbf{(0.000)}}} & \mc{1}{c}{\scriptsize{\textbf{(0.000)}}} & \mc{1}{c}{\scriptsize{\textbf{(0.000)}}} & \mc{1}{c}{\scriptsize{\textbf{(0.000)}}} & \mc{1}{c}{\scriptsize{\textbf{(0.000)}}} \\  

     & \mc{1}{c}{\scriptsize{3}} & \mc{1}{c}{\scriptsize{12.148}} & \mc{1}{c}{\scriptsize{14.354}} & \mc{1}{c}{\scriptsize{23.524}} & \mc{1}{c}{\scriptsize{30.423}} & \mc{1}{c}{\scriptsize{24.033}} & \mc{1}{c}{\scriptsize{7.093}} & \mc{1}{c}{\scriptsize{9.388}} & \mc{1}{c}{\scriptsize{7.107}} \\  

     &  & \mc{1}{c}{\scriptsize{\textbf{(0.000)}}} & \mc{1}{c}{\scriptsize{\textbf{(0.000)}}} & \mc{1}{c}{\scriptsize{\textbf{(0.000)}}} & \mc{1}{c}{\scriptsize{\textbf{(0.000)}}} & \mc{1}{c}{\scriptsize{\textbf{(0.000)}}} & \mc{1}{c}{\scriptsize{\textbf{(0.053)}}} & \mc{1}{c}{\scriptsize{\textbf{(0.000)}}} & \mc{1}{c}{\scriptsize{\textbf{(0.039)}}} \\  

     & \mc{1}{c}{\scriptsize{3.5}} & \mc{1}{c}{\scriptsize{6.894}} & \mc{1}{c}{\scriptsize{7.007}} & \mc{1}{c}{\scriptsize{15.521}} & \mc{1}{c}{\scriptsize{19.445}} & \mc{1}{c}{\scriptsize{15.376}} & \mc{1}{c}{\scriptsize{3.240}} & \mc{1}{c}{\scriptsize{3.076}} & \mc{1}{c}{\scriptsize{3.092}} \\  

     &  & \mc{1}{c}{\scriptsize{\textbf{(0.000)}}} & \mc{1}{c}{\scriptsize{\textbf{(0.000)}}} & \mc{1}{c}{\scriptsize{\textbf{(0.000)}}} & \mc{1}{c}{\scriptsize{\textbf{(0.000)}}} & \mc{1}{c}{\scriptsize{\textbf{(0.000)}}} & \mc{1}{c}{\scriptsize{(0.132)}} & \mc{1}{c}{\scriptsize{(0.211)}} & \mc{1}{c}{\scriptsize{(0.158)}} \\  

     & \mc{1}{c}{\scriptsize{4}} & \mc{1}{c}{\scriptsize{5.454}} & \mc{1}{c}{\scriptsize{6.277}} & \mc{1}{c}{\scriptsize{13.729}} & \mc{1}{c}{\scriptsize{21.025}} & \mc{1}{c}{\scriptsize{14.230}} & \mc{1}{c}{\scriptsize{2.078}} & \mc{1}{c}{\scriptsize{3.088}} & \mc{1}{c}{\scriptsize{2.539}} \\  

     &  & \mc{1}{c}{\scriptsize{\textbf{(0.039)}}} & \mc{1}{c}{\scriptsize{\textbf{(0.000)}}} & \mc{1}{c}{\scriptsize{\textbf{(0.000)}}} & \mc{1}{c}{\scriptsize{\textbf{(0.000)}}} & \mc{1}{c}{\scriptsize{\textbf{(0.000)}}} & \mc{1}{c}{\scriptsize{(0.197)}} & \mc{1}{c}{\scriptsize{(0.158)}} & \mc{1}{c}{\scriptsize{(0.184)}} \\  

     & \mc{1}{c}{\scriptsize{4.5}} & \mc{1}{c}{\scriptsize{7.822}} & \mc{1}{c}{\scriptsize{7.971}} & \mc{1}{c}{\scriptsize{17.017}} & \mc{1}{c}{\scriptsize{22.208}} & \mc{1}{c}{\scriptsize{17.313}} & \mc{1}{c}{\scriptsize{4.027}} & \mc{1}{c}{\scriptsize{4.508}} & \mc{1}{c}{\scriptsize{4.460}} \\  

     &  & \mc{1}{c}{\scriptsize{\textbf{(0.000)}}} & \mc{1}{c}{\scriptsize{\textbf{(0.000)}}} & \mc{1}{c}{\scriptsize{\textbf{(0.000)}}} & \mc{1}{c}{\scriptsize{\textbf{(0.000)}}} & \mc{1}{c}{\scriptsize{\textbf{(0.000)}}} & \mc{1}{c}{\scriptsize{(0.132)}} & \mc{1}{c}{\scriptsize{\textbf{(0.066)}}} & \mc{1}{c}{\scriptsize{(0.118)}} \\  

     & \mc{1}{c}{\scriptsize{5}} & \mc{1}{c}{\scriptsize{5.006}} & \mc{1}{c}{\scriptsize{5.100}} & \mc{1}{c}{\scriptsize{13.550}} & \mc{1}{c}{\scriptsize{18.640}} & \mc{1}{c}{\scriptsize{14.082}} & \mc{1}{c}{\scriptsize{1.899}} & \mc{1}{c}{\scriptsize{2.081}} & \mc{1}{c}{\scriptsize{2.229}} \\  

     &  & \mc{1}{c}{\scriptsize{\textbf{(0.079)}}} & \mc{1}{c}{\scriptsize{\textbf{(0.079)}}} & \mc{1}{c}{\scriptsize{\textbf{(0.000)}}} & \mc{1}{c}{\scriptsize{\textbf{(0.013)}}} & \mc{1}{c}{\scriptsize{\textbf{(0.000)}}} & \mc{1}{c}{\scriptsize{(0.237)}} & \mc{1}{c}{\scriptsize{(0.237)}} & \mc{1}{c}{\scriptsize{(0.211)}} \\  

     & \mc{1}{c}{\scriptsize{6.6}} & \mc{1}{c}{\scriptsize{5.067}} & \mc{1}{c}{\scriptsize{7.731}} & \mc{1}{c}{\scriptsize{9.572}} & \mc{1}{c}{\scriptsize{12.743}} & \mc{1}{c}{\scriptsize{10.410}} & \mc{1}{c}{\scriptsize{3.415}} & \mc{1}{c}{\scriptsize{6.755}} & \mc{1}{c}{\scriptsize{4.033}} \\  

     &  & \mc{1}{c}{\scriptsize{\textbf{(0.066)}}} & \mc{1}{c}{\scriptsize{\textbf{(0.013)}}} & \mc{1}{c}{\scriptsize{\textbf{(0.000)}}} & \mc{1}{c}{\scriptsize{\textbf{(0.013)}}} & \mc{1}{c}{\scriptsize{\textbf{(0.000)}}} & \mc{1}{c}{\scriptsize{(0.158)}} & \mc{1}{c}{\scriptsize{\textbf{(0.026)}}} & \mc{1}{c}{\scriptsize{(0.118)}} \\  

     & \mc{1}{c}{\scriptsize{7}} & \mc{1}{c}{\scriptsize{4.486}} & \mc{1}{c}{\scriptsize{6.651}} & \mc{1}{c}{\scriptsize{8.096}} & \mc{1}{c}{\scriptsize{11.156}} & \mc{1}{c}{\scriptsize{8.151}} & \mc{1}{c}{\scriptsize{3.148}} & \mc{1}{c}{\scriptsize{4.663}} & \mc{1}{c}{\scriptsize{2.650}} \\  

     &  & \mc{1}{c}{\scriptsize{\textbf{(0.092)}}} & \mc{1}{c}{\scriptsize{\textbf{(0.079)}}} & \mc{1}{c}{\scriptsize{\textbf{(0.026)}}} & \mc{1}{c}{\scriptsize{\textbf{(0.026)}}} & \mc{1}{c}{\scriptsize{\textbf{(0.026)}}} & \mc{1}{c}{\scriptsize{(0.250)}} & \mc{1}{c}{\scriptsize{(0.132)}} & \mc{1}{c}{\scriptsize{(0.250)}} \\  

     & \mc{1}{c}{\scriptsize{8}} & \mc{1}{c}{\scriptsize{5.902}} & \mc{1}{c}{\scriptsize{5.645}} & \mc{1}{c}{\scriptsize{9.000}} & \mc{1}{c}{\scriptsize{13.148}} & \mc{1}{c}{\scriptsize{9.335}} & \mc{1}{c}{\scriptsize{4.767}} & \mc{1}{c}{\scriptsize{5.258}} & \mc{1}{c}{\scriptsize{4.976}} \\  

     &  & \mc{1}{c}{\scriptsize{\textbf{(0.053)}}} & \mc{1}{c}{\scriptsize{\textbf{(0.053)}}} & \mc{1}{c}{\scriptsize{\textbf{(0.026)}}} & \mc{1}{c}{\scriptsize{\textbf{(0.013)}}} & \mc{1}{c}{\scriptsize{\textbf{(0.026)}}} & \mc{1}{c}{\scriptsize{\textbf{(0.092)}}} & \mc{1}{c}{\scriptsize{\textbf{(0.079)}}} & \mc{1}{c}{\scriptsize{(0.105)}} \\  

     & \mc{1}{c}{\scriptsize{12}} & \mc{1}{c}{\scriptsize{8.259}} & \mc{1}{c}{\scriptsize{10.139}} & \mc{1}{c}{\scriptsize{8.100}} & \mc{1}{c}{\scriptsize{9.130}} & \mc{1}{c}{\scriptsize{7.690}} & \mc{1}{c}{\scriptsize{8.181}} & \mc{1}{c}{\scriptsize{10.271}} & \mc{1}{c}{\scriptsize{7.520}} \\  

     &  & \mc{1}{c}{\scriptsize{\textbf{(0.013)}}} & \mc{1}{c}{\scriptsize{\textbf{(0.000)}}} & \mc{1}{c}{\scriptsize{\textbf{(0.000)}}} & \mc{1}{c}{\scriptsize{\textbf{(0.026)}}} & \mc{1}{c}{\scriptsize{\textbf{(0.013)}}} & \mc{1}{c}{\scriptsize{\textbf{(0.026)}}} & \mc{1}{c}{\scriptsize{\textbf{(0.000)}}} & \mc{1}{c}{\scriptsize{\textbf{(0.026)}}} \\  

    \mc{1}{l}{\scriptsize{IQ Factor}} & \mc{1}{c}{\scriptsize{2 to 5}} & \mc{1}{c}{\scriptsize{0.584}} & \mc{1}{c}{\scriptsize{0.602}} & \mc{1}{c}{\scriptsize{1.319}} & \mc{1}{c}{\scriptsize{1.754}} & \mc{1}{c}{\scriptsize{1.347}} & \mc{1}{c}{\scriptsize{0.309}} & \mc{1}{c}{\scriptsize{0.310}} & \mc{1}{c}{\scriptsize{0.328}} \\  

     &  & \mc{1}{c}{\scriptsize{\textbf{(0.000)}}} & \mc{1}{c}{\scriptsize{\textbf{(0.000)}}} & \mc{1}{c}{\scriptsize{\textbf{(0.000)}}} & \mc{1}{c}{\scriptsize{\textbf{(0.000)}}} & \mc{1}{c}{\scriptsize{\textbf{(0.000)}}} & \mc{1}{c}{\scriptsize{(0.132)}} & \mc{1}{c}{\scriptsize{(0.132)}} & \mc{1}{c}{\scriptsize{(0.105)}} \\  

     & \mc{1}{c}{\scriptsize{6 to 12}} & \mc{1}{c}{\scriptsize{0.491}} & \mc{1}{c}{\scriptsize{0.666}} & \mc{1}{c}{\scriptsize{0.598}} & \mc{1}{c}{\scriptsize{0.700}} & \mc{1}{c}{\scriptsize{0.656}} & \mc{1}{c}{\scriptsize{0.451}} & \mc{1}{c}{\scriptsize{0.612}} & \mc{1}{c}{\scriptsize{0.482}} \\  

     &  & \mc{1}{c}{\scriptsize{\textbf{(0.039)}}} & \mc{1}{c}{\scriptsize{\textbf{(0.000)}}} & \mc{1}{c}{\scriptsize{\textbf{(0.039)}}} & \mc{1}{c}{\scriptsize{\textbf{(0.053)}}} & \mc{1}{c}{\scriptsize{\textbf{(0.000)}}} & \mc{1}{c}{\scriptsize{\textbf{(0.053)}}} & \mc{1}{c}{\scriptsize{\textbf{(0.013)}}} & \mc{1}{c}{\scriptsize{\textbf{(0.053)}}} \\ 
    \midrule  

    \mc{2}{l}{\scriptsize{\% of Pos. TE ($H_0$: $\le$ 50\%)}} & \mc{1}{c}{\scriptsize{100}} & \mc{1}{c}{\scriptsize{100}} & \mc{1}{c}{\scriptsize{100}} & \mc{1}{c}{\scriptsize{100}} & \mc{1}{c}{\scriptsize{100}} & \mc{1}{c}{\scriptsize{100}} & \mc{1}{c}{\scriptsize{100}} & \mc{1}{c}{\scriptsize{100}} \\  

     &  & \mc{1}{c}{\scriptsize{\textbf{(0.000)}}} & \mc{1}{c}{\scriptsize{\textbf{(0.000)}}} & \mc{1}{c}{\scriptsize{\textbf{(0.000)}}} & \mc{1}{c}{\scriptsize{\textbf{(0.000)}}} & \mc{1}{c}{\scriptsize{\textbf{(0.000)}}} & \mc{1}{c}{\scriptsize{\textbf{(0.000)}}} & \mc{1}{c}{\scriptsize{\textbf{(0.000)}}} & \mc{1}{c}{\scriptsize{\textbf{(0.000)}}} \\  

    \mc{2}{l}{\scriptsize{\% of Pos. TE ($H_0$: $\le$ 10\% $|$ 10\% Significance)}} & \mc{1}{c}{\scriptsize{92}} & \mc{1}{c}{\scriptsize{100}} & \mc{1}{c}{\scriptsize{100}} & \mc{1}{c}{\scriptsize{100}} & \mc{1}{c}{\scriptsize{100}} & \mc{1}{c}{\scriptsize{67}} & \mc{1}{c}{\scriptsize{75}} & \mc{1}{c}{\scriptsize{67}} \\  

     &  & \mc{1}{c}{\scriptsize{\textbf{(0.000)}}} & \mc{1}{c}{\scriptsize{\textbf{(0.000)}}} & \mc{1}{c}{\scriptsize{\textbf{(0.000)}}} & \mc{1}{c}{\scriptsize{\textbf{(0.000)}}} & \mc{1}{c}{\scriptsize{\textbf{(0.000)}}} & \mc{1}{c}{\scriptsize{\textbf{(0.066)}}} & \mc{1}{c}{\scriptsize{\textbf{(0.000)}}} & \mc{1}{c}{\scriptsize{\textbf{(0.039)}}} \\  

  \bottomrule
  \end{tabular}
	\end{table} 

	\begin{table}[H]
     \caption{Treatment Effects on Achievement Scores, Female Sample}
     \label{table:abccare_rslt_female_cat1}
	  \begin{tabular}{cccccccccc}
  \toprule

    \scriptsize{Variable} & \scriptsize{Age} & \scriptsize{(1)} & \scriptsize{(2)} & \scriptsize{(3)} & \scriptsize{(4)} & \scriptsize{(5)} & \scriptsize{(6)} & \scriptsize{(7)} & \scriptsize{(8)} \\ 
    \midrule  

    \mc{1}{l}{\scriptsize{Std. Achv.  Test}} & \mc{1}{c}{\scriptsize{5.5}} & \mc{1}{c}{\scriptsize{12.979}} & \mc{1}{c}{\scriptsize{11.238}} & \mc{1}{c}{\scriptsize{21.894}} & \mc{1}{c}{\scriptsize{21.119}} & \mc{1}{c}{\scriptsize{23.191}} & \mc{1}{c}{\scriptsize{9.347}} & \mc{1}{c}{\scriptsize{8.229}} & \mc{1}{c}{\scriptsize{10.973}} \\  

     &  & \mc{1}{c}{\scriptsize{\textbf{(0.000)}}} & \mc{1}{c}{\scriptsize{\textbf{(0.000)}}} & \mc{1}{c}{\scriptsize{\textbf{(0.000)}}} & \mc{1}{c}{\scriptsize{\textbf{(0.000)}}} & \mc{1}{c}{\scriptsize{\textbf{(0.000)}}} & \mc{1}{c}{\scriptsize{\textbf{(0.000)}}} & \mc{1}{c}{\scriptsize{\textbf{(0.000)}}} & \mc{1}{c}{\scriptsize{\textbf{(0.000)}}} \\  

     & \mc{1}{c}{\scriptsize{6}} & \mc{1}{c}{\scriptsize{5.941}} & \mc{1}{c}{\scriptsize{7.421}} & \mc{1}{c}{\scriptsize{11.508}} & \mc{1}{c}{\scriptsize{14.330}} & \mc{1}{c}{\scriptsize{12.114}} & \mc{1}{c}{\scriptsize{3.829}} & \mc{1}{c}{\scriptsize{5.529}} & \mc{1}{c}{\scriptsize{4.604}} \\  

     &  & \mc{1}{c}{\scriptsize{\textbf{(0.013)}}} & \mc{1}{c}{\scriptsize{\textbf{(0.000)}}} & \mc{1}{c}{\scriptsize{\textbf{(0.000)}}} & \mc{1}{c}{\scriptsize{\textbf{(0.000)}}} & \mc{1}{c}{\scriptsize{\textbf{(0.000)}}} & \mc{1}{c}{\scriptsize{\textbf{(0.039)}}} & \mc{1}{c}{\scriptsize{\textbf{(0.026)}}} & \mc{1}{c}{\scriptsize{\textbf{(0.026)}}} \\  

     & \mc{1}{c}{\scriptsize{7.5}} & \mc{1}{c}{\scriptsize{2.162}} & \mc{1}{c}{\scriptsize{5.441}} & \mc{1}{c}{\scriptsize{3.577}} & \mc{1}{c}{\scriptsize{10.873}} & \mc{1}{c}{\scriptsize{3.853}} & \mc{1}{c}{\scriptsize{1.607}} & \mc{1}{c}{\scriptsize{3.924}} & \mc{1}{c}{\scriptsize{1.784}} \\  

     &  & \mc{1}{c}{\scriptsize{(0.263)}} & \mc{1}{c}{\scriptsize{\textbf{(0.000)}}} & \mc{1}{c}{\scriptsize{(0.197)}} & \mc{1}{c}{\scriptsize{\textbf{(0.000)}}} & \mc{1}{c}{\scriptsize{(0.197)}} & \mc{1}{c}{\scriptsize{(0.329)}} & \mc{1}{c}{\scriptsize{\textbf{(0.053)}}} & \mc{1}{c}{\scriptsize{(0.276)}} \\  

     & \mc{1}{c}{\scriptsize{8}} & \mc{1}{c}{\scriptsize{5.025}} & \mc{1}{c}{\scriptsize{6.810}} & \mc{1}{c}{\scriptsize{7.721}} & \mc{1}{c}{\scriptsize{12.030}} & \mc{1}{c}{\scriptsize{8.338}} & \mc{1}{c}{\scriptsize{4.003}} & \mc{1}{c}{\scriptsize{5.410}} & \mc{1}{c}{\scriptsize{4.513}} \\  

     &  & \mc{1}{c}{\scriptsize{\textbf{(0.053)}}} & \mc{1}{c}{\scriptsize{\textbf{(0.000)}}} & \mc{1}{c}{\scriptsize{\textbf{(0.053)}}} & \mc{1}{c}{\scriptsize{\textbf{(0.000)}}} & \mc{1}{c}{\scriptsize{\textbf{(0.039)}}} & \mc{1}{c}{\scriptsize{\textbf{(0.092)}}} & \mc{1}{c}{\scriptsize{\textbf{(0.026)}}} & \mc{1}{c}{\scriptsize{\textbf{(0.066)}}} \\  

     & \mc{1}{c}{\scriptsize{8.5}} & \mc{1}{c}{\scriptsize{6.971}} & \mc{1}{c}{\scriptsize{8.386}} & \mc{1}{c}{\scriptsize{12.592}} & \mc{1}{c}{\scriptsize{15.085}} & \mc{1}{c}{\scriptsize{12.920}} & \mc{1}{c}{\scriptsize{4.839}} & \mc{1}{c}{\scriptsize{6.440}} & \mc{1}{c}{\scriptsize{5.386}} \\  

     &  & \mc{1}{c}{\scriptsize{\textbf{(0.013)}}} & \mc{1}{c}{\scriptsize{\textbf{(0.000)}}} & \mc{1}{c}{\scriptsize{\textbf{(0.000)}}} & \mc{1}{c}{\scriptsize{\textbf{(0.000)}}} & \mc{1}{c}{\scriptsize{\textbf{(0.000)}}} & \mc{1}{c}{\scriptsize{\textbf{(0.026)}}} & \mc{1}{c}{\scriptsize{\textbf{(0.026)}}} & \mc{1}{c}{\scriptsize{\textbf{(0.026)}}} \\  

    \mc{1}{l}{\scriptsize{Achievement Factor}} & \mc{1}{c}{\scriptsize{5.5 to 12}} & \mc{1}{c}{\scriptsize{0.760}} & \mc{1}{c}{\scriptsize{0.903}} & \mc{1}{c}{\scriptsize{1.212}} & \mc{1}{c}{\scriptsize{1.330}} & \mc{1}{c}{\scriptsize{1.293}} & \mc{1}{c}{\scriptsize{0.544}} & \mc{1}{c}{\scriptsize{0.709}} & \mc{1}{c}{\scriptsize{0.635}} \\  

     &  & \mc{1}{c}{\scriptsize{\textbf{(0.000)}}} & \mc{1}{c}{\scriptsize{\textbf{(0.000)}}} & \mc{1}{c}{\scriptsize{\textbf{(0.000)}}} & \mc{1}{c}{\scriptsize{\textbf{(0.013)}}} & \mc{1}{c}{\scriptsize{\textbf{(0.000)}}} & \mc{1}{c}{\scriptsize{\textbf{(0.000)}}} & \mc{1}{c}{\scriptsize{\textbf{(0.000)}}} & \mc{1}{c}{\scriptsize{\textbf{(0.000)}}} \\ 
    \midrule  

    \mc{2}{l}{\scriptsize{\% of Pos. TE ($H_0$: $\le$ 50\%)}} & \mc{1}{c}{\scriptsize{100}} & \mc{1}{c}{\scriptsize{100}} & \mc{1}{c}{\scriptsize{100}} & \mc{1}{c}{\scriptsize{100}} & \mc{1}{c}{\scriptsize{100}} & \mc{1}{c}{\scriptsize{100}} & \mc{1}{c}{\scriptsize{100}} & \mc{1}{c}{\scriptsize{100}} \\  

     &  & \mc{1}{c}{\scriptsize{\textbf{(0.000)}}} & \mc{1}{c}{\scriptsize{\textbf{(0.000)}}} & \mc{1}{c}{\scriptsize{\textbf{(0.000)}}} & \mc{1}{c}{\scriptsize{\textbf{(0.000)}}} & \mc{1}{c}{\scriptsize{\textbf{(0.000)}}} & \mc{1}{c}{\scriptsize{\textbf{(0.000)}}} & \mc{1}{c}{\scriptsize{\textbf{(0.000)}}} & \mc{1}{c}{\scriptsize{\textbf{(0.000)}}} \\  

    \mc{2}{l}{\scriptsize{\% of Pos. TE ($H_0$: $\le$ 10\% $|$ 10\% Significance)}} & \mc{1}{c}{\scriptsize{83}} & \mc{1}{c}{\scriptsize{100}} & \mc{1}{c}{\scriptsize{83}} & \mc{1}{c}{\scriptsize{100}} & \mc{1}{c}{\scriptsize{83}} & \mc{1}{c}{\scriptsize{83}} & \mc{1}{c}{\scriptsize{100}} & \mc{1}{c}{\scriptsize{83}} \\  

     &  & \mc{1}{c}{\scriptsize{\textbf{(0.000)}}} & \mc{1}{c}{\scriptsize{\textbf{(0.000)}}} & \mc{1}{c}{\scriptsize{\textbf{(0.000)}}} & \mc{1}{c}{\scriptsize{\textbf{(0.000)}}} & \mc{1}{c}{\scriptsize{\textbf{(0.000)}}} & \mc{1}{c}{\scriptsize{\textbf{(0.000)}}} & \mc{1}{c}{\scriptsize{\textbf{(0.000)}}} & \mc{1}{c}{\scriptsize{\textbf{(0.000)}}} \\  

  \bottomrule
  \end{tabular}
	\end{table} 

	\begin{table}[H]
     \caption{Treatment Effects on Infant Behavior Record, Female Sample}
     \label{table:abccare_rslt_female_cat2}
	  \begin{tabular}{cccccccccc}
  \toprule

    \scriptsize{Variable} & \scriptsize{Age} & \scriptsize{(1)} & \scriptsize{(2)} & \scriptsize{(3)} & \scriptsize{(4)} & \scriptsize{(5)} & \scriptsize{(6)} & \scriptsize{(7)} & \scriptsize{(8)} \\ 
    \midrule  

    \mc{1}{l}{\scriptsize{Prediabetes}} & \mc{1}{c}{\scriptsize{Mid-30s}} & \mc{1}{c}{\scriptsize{0.088}} & \mc{1}{c}{\scriptsize{0.163}} & \mc{1}{c}{\scriptsize{0.076}} & \mc{1}{c}{\scriptsize{0.176}} & \mc{1}{c}{\scriptsize{0.035}} & \mc{1}{c}{\scriptsize{0.091}} & \mc{1}{c}{\scriptsize{0.161}} & \mc{1}{c}{\scriptsize{0.029}} \\  

     &  & \mc{1}{c}{\scriptsize{(0.737)}} & \mc{1}{c}{\scriptsize{(0.789)}} & \mc{1}{c}{\scriptsize{(0.632)}} & \mc{1}{c}{\scriptsize{(0.763)}} & \mc{1}{c}{\scriptsize{(0.526)}} & \mc{1}{c}{\scriptsize{(0.632)}} & \mc{1}{c}{\scriptsize{(0.645)}} & \mc{1}{c}{\scriptsize{(0.487)}} \\  

    \mc{1}{l}{\scriptsize{Hemoglobin Level (\%)}} & \mc{1}{c}{\scriptsize{Mid-30s}} & \mc{1}{c}{\scriptsize{-0.277}} & \mc{1}{c}{\scriptsize{-0.101}} & \mc{1}{c}{\scriptsize{-0.176}} & \mc{1}{c}{\scriptsize{-0.088}} & \mc{1}{c}{\scriptsize{-0.190}} & \mc{1}{c}{\scriptsize{-0.304}} & \mc{1}{c}{\scriptsize{-0.037}} & \mc{1}{c}{\scriptsize{-0.355}} \\  

     &  & \mc{1}{c}{\scriptsize{(0.211)}} & \mc{1}{c}{\scriptsize{(0.316)}} & \mc{1}{c}{\scriptsize{\textbf{(0.092)}}} & \mc{1}{c}{\scriptsize{(0.355)}} & \mc{1}{c}{\scriptsize{(0.105)}} & \mc{1}{c}{\scriptsize{(0.171)}} & \mc{1}{c}{\scriptsize{(0.395)}} & \mc{1}{c}{\scriptsize{(0.158)}} \\  

    \mc{1}{l}{\scriptsize{Diabetes}} & \mc{1}{c}{\scriptsize{Mid-30s}} & \mc{1}{c}{\scriptsize{-0.071}} & \mc{1}{c}{\scriptsize{-0.032}} &  &  &  & \mc{1}{c}{\scriptsize{-0.091}} & \mc{1}{c}{\scriptsize{-0.039}} & \mc{1}{c}{\scriptsize{-0.095}} \\  

     &  & \mc{1}{c}{\scriptsize{\textbf{(0.066)}}} & \mc{1}{c}{\scriptsize{(0.171)}} &  &  &  & \mc{1}{c}{\scriptsize{\textbf{(0.053)}}} & \mc{1}{c}{\scriptsize{(0.145)}} & \mc{1}{c}{\scriptsize{\textbf{(0.039)}}} \\  

  \bottomrule
  \end{tabular}
	\end{table} 

	\begin{table}[H]
     \caption{Treatment Effects on Kohn and Rosman: Attentive/Cooperative, Female Sample}
     \label{table:abccare_rslt_female_cat3}
	  \begin{tabular}{cccccccccc}
  \toprule

    \scriptsize{Variable} & \scriptsize{Age} & \scriptsize{(1)} & \scriptsize{(2)} & \scriptsize{(3)} & \scriptsize{(4)} & \scriptsize{(5)} & \scriptsize{(6)} & \scriptsize{(7)} & \scriptsize{(8)} \\ 
    \midrule  

    \mc{1}{l}{\scriptsize{Parental Income}} & \mc{1}{c}{\scriptsize{1.5}} & \mc{1}{c}{\scriptsize{4,516}} & \mc{1}{c}{\scriptsize{7,539}} & \mc{1}{c}{\scriptsize{5,105}} & \mc{1}{c}{\scriptsize{8,682}} & \mc{1}{c}{\scriptsize{9,681}} & \mc{1}{c}{\scriptsize{4,309}} & \mc{1}{c}{\scriptsize{7,091}} & \mc{1}{c}{\scriptsize{7,337}} \\  

     &  & \mc{1}{c}{\scriptsize{\textbf{(0.066)}}} & \mc{1}{c}{\scriptsize{\textbf{(0.013)}}} & \mc{1}{c}{\scriptsize{(0.158)}} & \mc{1}{c}{\scriptsize{\textbf{(0.079)}}} & \mc{1}{c}{\scriptsize{\textbf{(0.039)}}} & \mc{1}{c}{\scriptsize{(0.105)}} & \mc{1}{c}{\scriptsize{\textbf{(0.066)}}} & \mc{1}{c}{\scriptsize{\textbf{(0.000)}}} \\  

     & \mc{1}{c}{\scriptsize{2.5}} & \mc{1}{c}{\scriptsize{222}} & \mc{1}{c}{\scriptsize{400}} & \mc{1}{c}{\scriptsize{-3,712}} & \mc{1}{c}{\scriptsize{894}} & \mc{1}{c}{\scriptsize{1,769}} & \mc{1}{c}{\scriptsize{1,598}} & \mc{1}{c}{\scriptsize{894}} & \mc{1}{c}{\scriptsize{3,227}} \\  

     &  & \mc{1}{c}{\scriptsize{(0.474)}} & \mc{1}{c}{\scriptsize{(0.434)}} & \mc{1}{c}{\scriptsize{(0.697)}} & \mc{1}{c}{\scriptsize{(0.408)}} & \mc{1}{c}{\scriptsize{(0.408)}} & \mc{1}{c}{\scriptsize{(0.355)}} & \mc{1}{c}{\scriptsize{(0.382)}} & \mc{1}{c}{\scriptsize{(0.224)}} \\  

     & \mc{1}{c}{\scriptsize{3.5}} & \mc{1}{c}{\scriptsize{2,756}} & \mc{1}{c}{\scriptsize{4,043}} & \mc{1}{c}{\scriptsize{4,469}} & \mc{1}{c}{\scriptsize{7,429}} & \mc{1}{c}{\scriptsize{8,577}} & \mc{1}{c}{\scriptsize{2,125}} & \mc{1}{c}{\scriptsize{2,120}} & \mc{1}{c}{\scriptsize{3,764}} \\  

     &  & \mc{1}{c}{\scriptsize{(0.211)}} & \mc{1}{c}{\scriptsize{(0.184)}} & \mc{1}{c}{\scriptsize{(0.184)}} & \mc{1}{c}{\scriptsize{(0.118)}} & \mc{1}{c}{\scriptsize{\textbf{(0.066)}}} & \mc{1}{c}{\scriptsize{(0.276)}} & \mc{1}{c}{\scriptsize{(0.316)}} & \mc{1}{c}{\scriptsize{(0.211)}} \\  

     & \mc{1}{c}{\scriptsize{4.5}} & \mc{1}{c}{\scriptsize{4,039}} & \mc{1}{c}{\scriptsize{8,141}} & \mc{1}{c}{\scriptsize{6,443}} & \mc{1}{c}{\scriptsize{9,836}} & \mc{1}{c}{\scriptsize{9,337}} & \mc{1}{c}{\scriptsize{3,202}} & \mc{1}{c}{\scriptsize{7,852}} & \mc{1}{c}{\scriptsize{6,022}} \\  

     &  & \mc{1}{c}{\scriptsize{\textbf{(0.092)}}} & \mc{1}{c}{\scriptsize{\textbf{(0.026)}}} & \mc{1}{c}{\scriptsize{(0.105)}} & \mc{1}{c}{\scriptsize{\textbf{(0.079)}}} & \mc{1}{c}{\scriptsize{\textbf{(0.013)}}} & \mc{1}{c}{\scriptsize{(0.118)}} & \mc{1}{c}{\scriptsize{\textbf{(0.039)}}} & \mc{1}{c}{\scriptsize{\textbf{(0.013)}}} \\  

    \mc{1}{l}{\scriptsize{Parental Income Factor}} & \mc{1}{c}{\scriptsize{1.5 to 15}} & \mc{1}{c}{\scriptsize{0.193}} & \mc{1}{c}{\scriptsize{0.209}} & \mc{1}{c}{\scriptsize{0.192}} & \mc{1}{c}{\scriptsize{0.291}} & \mc{1}{c}{\scriptsize{0.284}} & \mc{1}{c}{\scriptsize{0.193}} & \mc{1}{c}{\scriptsize{0.163}} & \mc{1}{c}{\scriptsize{0.283}} \\  

     &  & \mc{1}{c}{\scriptsize{(0.237)}} & \mc{1}{c}{\scriptsize{(0.276)}} & \mc{1}{c}{\scriptsize{(0.316)}} & \mc{1}{c}{\scriptsize{(0.211)}} & \mc{1}{c}{\scriptsize{(0.276)}} & \mc{1}{c}{\scriptsize{(0.276)}} & \mc{1}{c}{\scriptsize{(0.342)}} & \mc{1}{c}{\scriptsize{(0.184)}} \\ 
    \midrule  

    \mc{2}{l}{\scriptsize{\% of Pos. TE ($H_0$: $\le$ 50\%)}} & \mc{1}{c}{\scriptsize{100}} & \mc{1}{c}{\scriptsize{100}} & \mc{1}{c}{\scriptsize{80}} & \mc{1}{c}{\scriptsize{100}} & \mc{1}{c}{\scriptsize{100}} & \mc{1}{c}{\scriptsize{100}} & \mc{1}{c}{\scriptsize{100}} & \mc{1}{c}{\scriptsize{100}} \\  

     &  & \mc{1}{c}{\scriptsize{\textbf{(0.000)}}} & \mc{1}{c}{\scriptsize{\textbf{(0.000)}}} & \mc{1}{c}{\scriptsize{\textbf{(0.000)}}} & \mc{1}{c}{\scriptsize{\textbf{(0.000)}}} & \mc{1}{c}{\scriptsize{\textbf{(0.000)}}} & \mc{1}{c}{\scriptsize{\textbf{(0.000)}}} & \mc{1}{c}{\scriptsize{\textbf{(0.000)}}} & \mc{1}{c}{\scriptsize{\textbf{(0.000)}}} \\  

    \mc{2}{l}{\scriptsize{\% of Pos. TE ($H_0$: $\le$ 10\% $|$ 10\% Significance)}} & \mc{1}{c}{\scriptsize{40}} & \mc{1}{c}{\scriptsize{40}} & \mc{1}{c}{\scriptsize{0}} & \mc{1}{c}{\scriptsize{40}} & \mc{1}{c}{\scriptsize{60}} & \mc{1}{c}{\scriptsize{20}} & \mc{1}{c}{\scriptsize{40}} & \mc{1}{c}{\scriptsize{40}} \\  

     &  & \mc{1}{c}{\scriptsize{(0.184)}} & \mc{1}{c}{\scriptsize{(0.105)}} & \mc{1}{c}{\scriptsize{(0.434)}} & \mc{1}{c}{\scriptsize{\textbf{(0.053)}}} & \mc{1}{c}{\scriptsize{\textbf{(0.092)}}} & \mc{1}{c}{\scriptsize{(0.276)}} & \mc{1}{c}{\scriptsize{\textbf{(0.053)}}} & \mc{1}{c}{\scriptsize{(0.184)}} \\  

  \bottomrule
  \end{tabular}
	\end{table} 

	\begin{table}[H]
     \caption{Treatment Effects on Classroom Behavior Inventory (Part I), Female Sample}
     \label{table:abccare_rslt_female_cat4}
	  \begin{tabular}{cccccccccc}
  \toprule

    \scriptsize{Variable} & \scriptsize{Age} & \scriptsize{(1)} & \scriptsize{(2)} & \scriptsize{(3)} & \scriptsize{(4)} & \scriptsize{(5)} & \scriptsize{(6)} & \scriptsize{(7)} & \scriptsize{(8)} \\ 
    \midrule  

    \mc{1}{l}{\scriptsize{Mother Works}} & \mc{1}{c}{\scriptsize{2}} & \mc{1}{c}{\scriptsize{0.168}} & \mc{1}{c}{\scriptsize{0.124}} & \mc{1}{c}{\scriptsize{0.323}} & \mc{1}{c}{\scriptsize{0.292}} & \mc{1}{c}{\scriptsize{0.309}} & \mc{1}{c}{\scriptsize{0.101}} & \mc{1}{c}{\scriptsize{0.082}} & \mc{1}{c}{\scriptsize{0.096}} \\  

     &  & \mc{1}{c}{\scriptsize{\textbf{(0.026)}}} & \mc{1}{c}{\scriptsize{\textbf{(0.079)}}} & \mc{1}{c}{\scriptsize{\textbf{(0.079)}}} & \mc{1}{c}{\scriptsize{(0.145)}} & \mc{1}{c}{\scriptsize{\textbf{(0.079)}}} & \mc{1}{c}{\scriptsize{(0.145)}} & \mc{1}{c}{\scriptsize{(0.211)}} & \mc{1}{c}{\scriptsize{(0.171)}} \\  

     & \mc{1}{c}{\scriptsize{3}} & \mc{1}{c}{\scriptsize{0.087}} & \mc{1}{c}{\scriptsize{0.047}} & \mc{1}{c}{\scriptsize{0.177}} & \mc{1}{c}{\scriptsize{0.145}} & \mc{1}{c}{\scriptsize{0.160}} & \mc{1}{c}{\scriptsize{0.066}} & \mc{1}{c}{\scriptsize{0.021}} & \mc{1}{c}{\scriptsize{0.057}} \\  

     &  & \mc{1}{c}{\scriptsize{(0.237)}} & \mc{1}{c}{\scriptsize{(0.368)}} & \mc{1}{c}{\scriptsize{(0.158)}} & \mc{1}{c}{\scriptsize{(0.316)}} & \mc{1}{c}{\scriptsize{(0.184)}} & \mc{1}{c}{\scriptsize{(0.276)}} & \mc{1}{c}{\scriptsize{(0.461)}} & \mc{1}{c}{\scriptsize{(0.276)}} \\  

     & \mc{1}{c}{\scriptsize{4}} & \mc{1}{c}{\scriptsize{0.118}} & \mc{1}{c}{\scriptsize{0.080}} & \mc{1}{c}{\scriptsize{0.319}} & \mc{1}{c}{\scriptsize{0.275}} & \mc{1}{c}{\scriptsize{0.305}} & \mc{1}{c}{\scriptsize{0.060}} & \mc{1}{c}{\scriptsize{0.038}} & \mc{1}{c}{\scriptsize{0.053}} \\  

     &  & \mc{1}{c}{\scriptsize{\textbf{(0.039)}}} & \mc{1}{c}{\scriptsize{(0.197)}} & \mc{1}{c}{\scriptsize{\textbf{(0.079)}}} & \mc{1}{c}{\scriptsize{(0.171)}} & \mc{1}{c}{\scriptsize{\textbf{(0.079)}}} & \mc{1}{c}{\scriptsize{(0.276)}} & \mc{1}{c}{\scriptsize{(0.368)}} & \mc{1}{c}{\scriptsize{(0.316)}} \\  

     & \mc{1}{c}{\scriptsize{5}} & \mc{1}{c}{\scriptsize{0.067}} & \mc{1}{c}{\scriptsize{0.017}} & \mc{1}{c}{\scriptsize{0.367}} & \mc{1}{c}{\scriptsize{0.268}} & \mc{1}{c}{\scriptsize{0.397}} & \mc{1}{c}{\scriptsize{-0.056}} & \mc{1}{c}{\scriptsize{-0.097}} & \mc{1}{c}{\scriptsize{-0.024}} \\  

     &  & \mc{1}{c}{\scriptsize{(0.263)}} & \mc{1}{c}{\scriptsize{(0.434)}} & \mc{1}{c}{\scriptsize{\textbf{(0.000)}}} & \mc{1}{c}{\scriptsize{\textbf{(0.066)}}} & \mc{1}{c}{\scriptsize{\textbf{(0.000)}}} & \mc{1}{c}{\scriptsize{(0.658)}} & \mc{1}{c}{\scriptsize{(0.895)}} & \mc{1}{c}{\scriptsize{(0.605)}} \\  

    \mc{1}{l}{\scriptsize{Mother Works Factor}} & \mc{1}{c}{\scriptsize{2 to 21}} & \mc{1}{c}{\scriptsize{0.200}} & \mc{1}{c}{\scriptsize{0.076}} & \mc{1}{c}{\scriptsize{0.578}} & \mc{1}{c}{\scriptsize{0.484}} & \mc{1}{c}{\scriptsize{0.564}} & \mc{1}{c}{\scriptsize{0.101}} & \mc{1}{c}{\scriptsize{-0.037}} & \mc{1}{c}{\scriptsize{0.108}} \\  

     &  & \mc{1}{c}{\scriptsize{(0.211)}} & \mc{1}{c}{\scriptsize{(0.434)}} & \mc{1}{c}{\scriptsize{(0.132)}} & \mc{1}{c}{\scriptsize{(0.197)}} & \mc{1}{c}{\scriptsize{(0.145)}} & \mc{1}{c}{\scriptsize{(0.329)}} & \mc{1}{c}{\scriptsize{(0.553)}} & \mc{1}{c}{\scriptsize{(0.316)}} \\ 
    \midrule  

    \mc{2}{l}{\scriptsize{\% of Pos. TE ($H_0$: $\le$ 50\%)}} & \mc{1}{c}{\scriptsize{100}} & \mc{1}{c}{\scriptsize{100}} & \mc{1}{c}{\scriptsize{100}} & \mc{1}{c}{\scriptsize{100}} & \mc{1}{c}{\scriptsize{100}} & \mc{1}{c}{\scriptsize{80}} & \mc{1}{c}{\scriptsize{60}} & \mc{1}{c}{\scriptsize{80}} \\  

     &  & \mc{1}{c}{\scriptsize{\textbf{(0.000)}}} & \mc{1}{c}{\scriptsize{\textbf{(0.000)}}} & \mc{1}{c}{\scriptsize{\textbf{(0.000)}}} & \mc{1}{c}{\scriptsize{\textbf{(0.000)}}} & \mc{1}{c}{\scriptsize{\textbf{(0.000)}}} & \mc{1}{c}{\scriptsize{(0.250)}} & \mc{1}{c}{\scriptsize{(0.526)}} & \mc{1}{c}{\scriptsize{(0.368)}} \\  

    \mc{2}{l}{\scriptsize{\% of Pos. TE ($H_0$: $\le$ 10\% $|$ 10\% Significance)}} & \mc{1}{c}{\scriptsize{40}} & \mc{1}{c}{\scriptsize{0}} & \mc{1}{c}{\scriptsize{60}} & \mc{1}{c}{\scriptsize{20}} & \mc{1}{c}{\scriptsize{60}} & \mc{1}{c}{\scriptsize{0}} & \mc{1}{c}{\scriptsize{0}} & \mc{1}{c}{\scriptsize{0}} \\  

     &  & \mc{1}{c}{\scriptsize{(0.197)}} & \mc{1}{c}{\scriptsize{(0.276)}} & \mc{1}{c}{\scriptsize{(0.145)}} & \mc{1}{c}{\scriptsize{(0.197)}} & \mc{1}{c}{\scriptsize{(0.132)}} & \mc{1}{c}{\scriptsize{(0.197)}} & \mc{1}{c}{\scriptsize{(1.000)}} & \mc{1}{c}{\scriptsize{(1.000)}} \\  

  \bottomrule
  \end{tabular}
	\end{table} 

	\begin{table}[H]
     \caption{Treatment Effects on Classroom Behavior Inventory (Part II), Female Sample}
     \label{table:abccare_rslt_female_cat5}
	  \begin{tabular}{cccccccccc}
  \toprule

    \scriptsize{Variable} & \scriptsize{Age} & \scriptsize{(1)} & \scriptsize{(2)} & \scriptsize{(3)} & \scriptsize{(4)} & \scriptsize{(5)} & \scriptsize{(6)} & \scriptsize{(7)} & \scriptsize{(8)} \\ 
    \midrule  

    \mc{1}{l}{\scriptsize{Father at Home}} & \mc{1}{c}{\scriptsize{2}} & \mc{1}{c}{\scriptsize{0.018}} & \mc{1}{c}{\scriptsize{-0.166}} & \mc{1}{c}{\scriptsize{0.083}} &  & \mc{1}{c}{\scriptsize{-0.271}} & \mc{1}{c}{\scriptsize{-0.030}} & \mc{1}{c}{\scriptsize{-0.153}} & \mc{1}{c}{\scriptsize{0.067}} \\  

     &  & \mc{1}{c}{\scriptsize{(0.487)}} & \mc{1}{c}{\scriptsize{(0.816)}} & \mc{1}{c}{\scriptsize{(0.395)}} &  & \mc{1}{c}{\scriptsize{(0.750)}} & \mc{1}{c}{\scriptsize{(0.500)}} & \mc{1}{c}{\scriptsize{(0.605)}} & \mc{1}{c}{\scriptsize{(0.355)}} \\  

     & \mc{1}{c}{\scriptsize{3}} & \mc{1}{c}{\scriptsize{-0.088}} & \mc{1}{c}{\scriptsize{-0.205}} & \mc{1}{c}{\scriptsize{-0.292}} & \mc{1}{c}{\scriptsize{-0.260}} & \mc{1}{c}{\scriptsize{-0.579}} & \mc{1}{c}{\scriptsize{0.061}} & \mc{1}{c}{\scriptsize{-0.191}} & \mc{1}{c}{\scriptsize{0.089}} \\  

     &  & \mc{1}{c}{\scriptsize{(0.645)}} & \mc{1}{c}{\scriptsize{(0.842)}} & \mc{1}{c}{\scriptsize{(0.816)}} & \mc{1}{c}{\scriptsize{(0.395)}} & \mc{1}{c}{\scriptsize{(0.974)}} & \mc{1}{c}{\scriptsize{(0.342)}} & \mc{1}{c}{\scriptsize{(0.632)}} & \mc{1}{c}{\scriptsize{(0.368)}} \\  

     & \mc{1}{c}{\scriptsize{4}} & \mc{1}{c}{\scriptsize{-0.126}} & \mc{1}{c}{\scriptsize{-0.182}} & \mc{1}{c}{\scriptsize{-0.475}} & \mc{1}{c}{\scriptsize{-1.000}} & \mc{1}{c}{\scriptsize{-0.622}} & \mc{1}{c}{\scriptsize{0.127}} & \mc{1}{c}{\scriptsize{0.041}} & \mc{1}{c}{\scriptsize{0.120}} \\  

     &  & \mc{1}{c}{\scriptsize{(0.658)}} & \mc{1}{c}{\scriptsize{(0.789)}} & \mc{1}{c}{\scriptsize{(0.921)}} & \mc{1}{c}{\scriptsize{(0.697)}} & \mc{1}{c}{\scriptsize{(0.934)}} & \mc{1}{c}{\scriptsize{(0.289)}} & \mc{1}{c}{\scriptsize{(0.434)}} & \mc{1}{c}{\scriptsize{(0.329)}} \\  

     & \mc{1}{c}{\scriptsize{5}} & \mc{1}{c}{\scriptsize{-0.276}} & \mc{1}{c}{\scriptsize{-0.387}} & \mc{1}{c}{\scriptsize{-0.625}} & \mc{1}{c}{\scriptsize{-1.000}} & \mc{1}{c}{\scriptsize{-0.803}} & \mc{1}{c}{\scriptsize{-0.023}} & \mc{1}{c}{\scriptsize{0.022}} & \mc{1}{c}{\scriptsize{-0.096}} \\  

     &  & \mc{1}{c}{\scriptsize{(0.803)}} & \mc{1}{c}{\scriptsize{(0.895)}} & \mc{1}{c}{\scriptsize{(0.987)}} & \mc{1}{c}{\scriptsize{(0.763)}} & \mc{1}{c}{\scriptsize{(0.987)}} & \mc{1}{c}{\scriptsize{(0.434)}} & \mc{1}{c}{\scriptsize{(0.487)}} & \mc{1}{c}{\scriptsize{(0.526)}} \\  

     & \mc{1}{c}{\scriptsize{8}} & \mc{1}{c}{\scriptsize{-0.042}} & \mc{1}{c}{\scriptsize{-0.250}} & \mc{1}{c}{\scriptsize{-0.167}} &  & \mc{1}{c}{\scriptsize{-0.452}} & \mc{1}{c}{\scriptsize{0.333}} &  & \mc{1}{c}{\scriptsize{0.043}} \\  

     &  & \mc{1}{c}{\scriptsize{(0.447)}} & \mc{1}{c}{\scriptsize{(0.329)}} & \mc{1}{c}{\scriptsize{(0.605)}} &  & \mc{1}{c}{\scriptsize{(0.789)}} & \mc{1}{c}{\scriptsize{\textbf{(0.079)}}} &  & \mc{1}{c}{\scriptsize{\textbf{(0.066)}}} \\  

    \mc{1}{l}{\scriptsize{Father at Home Factor}} & \mc{1}{c}{\scriptsize{2 to 8}} & \mc{1}{c}{\scriptsize{-0.692}} & \mc{1}{c}{\scriptsize{-2.154}} & \mc{1}{c}{\scriptsize{-1.160}} & \mc{1}{c}{\scriptsize{-1.834}} & \mc{1}{c}{\scriptsize{-1.603}} & \mc{1}{c}{\scriptsize{0.712}} &  & \mc{1}{c}{\scriptsize{0.092}} \\  

     &  & \mc{1}{c}{\scriptsize{(0.750)}} & \mc{1}{c}{\scriptsize{(0.711)}} & \mc{1}{c}{\scriptsize{(1.000)}} & \mc{1}{c}{\scriptsize{(0.737)}} & \mc{1}{c}{\scriptsize{(0.961)}} & \mc{1}{c}{\scriptsize{\textbf{(0.079)}}} &  & \mc{1}{c}{\scriptsize{\textbf{(0.053)}}} \\ 
    \midrule  

    \mc{2}{l}{\scriptsize{\% of Pos. TE ($H_0$: $\le$ 50\%)}} & \mc{1}{c}{\scriptsize{17}} & \mc{1}{c}{\scriptsize{0}} & \mc{1}{c}{\scriptsize{17}} & \mc{1}{c}{\scriptsize{0}} & \mc{1}{c}{\scriptsize{0}} & \mc{1}{c}{\scriptsize{67}} & \mc{1}{c}{\scriptsize{50}} & \mc{1}{c}{\scriptsize{83}} \\  

     &  & \mc{1}{c}{\scriptsize{(0.697)}} & \mc{1}{c}{\scriptsize{(1.000)}} & \mc{1}{c}{\scriptsize{(1.000)}} & \mc{1}{c}{\scriptsize{(0.908)}} & \mc{1}{c}{\scriptsize{(1.000)}} & \mc{1}{c}{\scriptsize{(0.539)}} & \mc{1}{c}{\scriptsize{(0.329)}} & \mc{1}{c}{\scriptsize{(0.368)}} \\  

    \mc{2}{l}{\scriptsize{\% of Pos. TE ($H_0$: $\le$ 10\% $|$ 10\% Significance)}} & \mc{1}{c}{\scriptsize{0}} & \mc{1}{c}{\scriptsize{0}} & \mc{1}{c}{\scriptsize{0}} & \mc{1}{c}{\scriptsize{0}} & \mc{1}{c}{\scriptsize{0}} & \mc{1}{c}{\scriptsize{33}} & \mc{1}{c}{\scriptsize{0}} & \mc{1}{c}{\scriptsize{33}} \\  

     &  & \mc{1}{c}{\scriptsize{(1.000)}} & \mc{1}{c}{\scriptsize{(1.000)}} & \mc{1}{c}{\scriptsize{(1.000)}} & \mc{1}{c}{\scriptsize{(0.237)}} & \mc{1}{c}{\scriptsize{(1.000)}} & \mc{1}{c}{\scriptsize{(0.211)}} & \mc{1}{c}{\scriptsize{(0.211)}} & \mc{1}{c}{\scriptsize{(0.145)}} \\  

  \bottomrule
  \end{tabular}
	\end{table} 

	\begin{table}[H]
     \caption{Treatment Effects on Emotional, Activity, Sociability, Impulsivity Survey, Female Sample}
     \label{table:abccare_rslt_female_cat6}
	  \begin{tabular}{cccccccccc}
  \toprule

    \scriptsize{Variable} & \scriptsize{Age} & \scriptsize{(1)} & \scriptsize{(2)} & \scriptsize{(3)} & \scriptsize{(4)} & \scriptsize{(5)} & \scriptsize{(6)} & \scriptsize{(7)} & \scriptsize{(8)} \\ 
    \midrule  

    \mc{1}{l}{\scriptsize{Graduated High School}} & \mc{1}{c}{\scriptsize{30}} & \mc{1}{c}{\scriptsize{0.253}} & \mc{1}{c}{\scriptsize{0.148}} & \mc{1}{c}{\scriptsize{0.642}} & \mc{1}{c}{\scriptsize{0.519}} & \mc{1}{c}{\scriptsize{0.595}} & \mc{1}{c}{\scriptsize{0.137}} & \mc{1}{c}{\scriptsize{0.004}} & \mc{1}{c}{\scriptsize{0.066}} \\  

     &  & \mc{1}{c}{\scriptsize{\textbf{(0.000)}}} & \mc{1}{c}{\scriptsize{\textbf{(0.066)}}} & \mc{1}{c}{\scriptsize{\textbf{(0.000)}}} & \mc{1}{c}{\scriptsize{\textbf{(0.013)}}} & \mc{1}{c}{\scriptsize{\textbf{(0.000)}}} & \mc{1}{c}{\scriptsize{(0.158)}} & \mc{1}{c}{\scriptsize{(0.461)}} & \mc{1}{c}{\scriptsize{(0.263)}} \\  

    \mc{1}{l}{\scriptsize{Attended Voc./Tech./Com. College}} & \mc{1}{c}{\scriptsize{30}} & \mc{1}{c}{\scriptsize{-0.057}} & \mc{1}{c}{\scriptsize{-0.115}} & \mc{1}{c}{\scriptsize{-0.050}} & \mc{1}{c}{\scriptsize{-0.098}} & \mc{1}{c}{\scriptsize{-0.071}} & \mc{1}{c}{\scriptsize{-0.041}} & \mc{1}{c}{\scriptsize{-0.145}} & \mc{1}{c}{\scriptsize{-0.051}} \\  

     &  & \mc{1}{c}{\scriptsize{(0.684)}} & \mc{1}{c}{\scriptsize{(0.803)}} & \mc{1}{c}{\scriptsize{(0.618)}} & \mc{1}{c}{\scriptsize{(0.671)}} & \mc{1}{c}{\scriptsize{(0.697)}} & \mc{1}{c}{\scriptsize{(0.618)}} & \mc{1}{c}{\scriptsize{(0.895)}} & \mc{1}{c}{\scriptsize{(0.724)}} \\  

    \mc{1}{l}{\scriptsize{Graduated 4-year College}} & \mc{1}{c}{\scriptsize{30}} & \mc{1}{c}{\scriptsize{0.134}} & \mc{1}{c}{\scriptsize{0.102}} & \mc{1}{c}{\scriptsize{0.217}} & \mc{1}{c}{\scriptsize{0.097}} & \mc{1}{c}{\scriptsize{0.210}} & \mc{1}{c}{\scriptsize{0.106}} & \mc{1}{c}{\scriptsize{0.073}} & \mc{1}{c}{\scriptsize{0.095}} \\  

     &  & \mc{1}{c}{\scriptsize{\textbf{(0.066)}}} & \mc{1}{c}{\scriptsize{(0.158)}} & \mc{1}{c}{\scriptsize{\textbf{(0.013)}}} & \mc{1}{c}{\scriptsize{(0.184)}} & \mc{1}{c}{\scriptsize{\textbf{(0.000)}}} & \mc{1}{c}{\scriptsize{(0.145)}} & \mc{1}{c}{\scriptsize{(0.316)}} & \mc{1}{c}{\scriptsize{(0.197)}} \\  

    \mc{1}{l}{\scriptsize{Years of Edu.}} & \mc{1}{c}{\scriptsize{30}} & \mc{1}{c}{\scriptsize{2.143}} & \mc{1}{c}{\scriptsize{1.695}} & \mc{1}{c}{\scriptsize{4.025}} & \mc{1}{c}{\scriptsize{2.984}} & \mc{1}{c}{\scriptsize{3.918}} & \mc{1}{c}{\scriptsize{1.567}} & \mc{1}{c}{\scriptsize{1.155}} & \mc{1}{c}{\scriptsize{1.409}} \\  

     &  & \mc{1}{c}{\scriptsize{\textbf{(0.000)}}} & \mc{1}{c}{\scriptsize{\textbf{(0.000)}}} & \mc{1}{c}{\scriptsize{\textbf{(0.000)}}} & \mc{1}{c}{\scriptsize{\textbf{(0.000)}}} & \mc{1}{c}{\scriptsize{\textbf{(0.000)}}} & \mc{1}{c}{\scriptsize{\textbf{(0.013)}}} & \mc{1}{c}{\scriptsize{\textbf{(0.066)}}} & \mc{1}{c}{\scriptsize{\textbf{(0.026)}}} \\  

    \mc{1}{l}{\scriptsize{Education Factor}} & \mc{1}{c}{\scriptsize{30}} & \mc{1}{c}{\scriptsize{0.661}} & \mc{1}{c}{\scriptsize{0.461}} & \mc{1}{c}{\scriptsize{1.277}} & \mc{1}{c}{\scriptsize{0.799}} & \mc{1}{c}{\scriptsize{1.212}} & \mc{1}{c}{\scriptsize{0.447}} & \mc{1}{c}{\scriptsize{0.261}} & \mc{1}{c}{\scriptsize{0.361}} \\  

     &  & \mc{1}{c}{\scriptsize{\textbf{(0.000)}}} & \mc{1}{c}{\scriptsize{\textbf{(0.039)}}} & \mc{1}{c}{\scriptsize{\textbf{(0.000)}}} & \mc{1}{c}{\scriptsize{\textbf{(0.039)}}} & \mc{1}{c}{\scriptsize{\textbf{(0.000)}}} & \mc{1}{c}{\scriptsize{\textbf{(0.039)}}} & \mc{1}{c}{\scriptsize{(0.250)}} & \mc{1}{c}{\scriptsize{(0.105)}} \\ 
    \midrule  

    \mc{2}{l}{\scriptsize{\% of Pos. TE ($H_0$: $\le$ 50\%)}} & \mc{1}{c}{\scriptsize{80}} & \mc{1}{c}{\scriptsize{80}} & \mc{1}{c}{\scriptsize{80}} & \mc{1}{c}{\scriptsize{80}} & \mc{1}{c}{\scriptsize{80}} & \mc{1}{c}{\scriptsize{80}} & \mc{1}{c}{\scriptsize{80}} & \mc{1}{c}{\scriptsize{80}} \\  

     &  & \mc{1}{c}{\scriptsize{\textbf{(0.000)}}} & \mc{1}{c}{\scriptsize{\textbf{(0.000)}}} & \mc{1}{c}{\scriptsize{\textbf{(0.000)}}} & \mc{1}{c}{\scriptsize{\textbf{(0.000)}}} & \mc{1}{c}{\scriptsize{\textbf{(0.000)}}} & \mc{1}{c}{\scriptsize{\textbf{(0.000)}}} & \mc{1}{c}{\scriptsize{\textbf{(0.026)}}} & \mc{1}{c}{\scriptsize{\textbf{(0.000)}}} \\  

    \mc{2}{l}{\scriptsize{\% of Pos. TE ($H_0$: $\le$ 10\% $|$ 10\% Significance)}} & \mc{1}{c}{\scriptsize{80}} & \mc{1}{c}{\scriptsize{40}} & \mc{1}{c}{\scriptsize{80}} & \mc{1}{c}{\scriptsize{60}} & \mc{1}{c}{\scriptsize{80}} & \mc{1}{c}{\scriptsize{40}} & \mc{1}{c}{\scriptsize{20}} & \mc{1}{c}{\scriptsize{40}} \\  

     &  & \mc{1}{c}{\scriptsize{\textbf{(0.000)}}} & \mc{1}{c}{\scriptsize{(0.118)}} & \mc{1}{c}{\scriptsize{\textbf{(0.000)}}} & \mc{1}{c}{\scriptsize{\textbf{(0.000)}}} & \mc{1}{c}{\scriptsize{\textbf{(0.000)}}} & \mc{1}{c}{\scriptsize{(0.132)}} & \mc{1}{c}{\scriptsize{(0.250)}} & \mc{1}{c}{\scriptsize{(0.211)}} \\  

  \bottomrule
  \end{tabular}
	\end{table} 

	\begin{table}[H]
     \caption{Treatment Effects on Harter Importance, Female Sample}
     \label{table:abccare_rslt_female_cat7}
	  \begin{tabular}{cccccccccc}
  \toprule

    \scriptsize{Variable} & \scriptsize{Age} & \scriptsize{(1)} & \scriptsize{(2)} & \scriptsize{(3)} & \scriptsize{(4)} & \scriptsize{(5)} & \scriptsize{(6)} & \scriptsize{(7)} & \scriptsize{(8)} \\ 
    \midrule  

    \mc{1}{l}{\scriptsize{Behavioral conduct}} & \mc{1}{c}{\scriptsize{12}} & \mc{1}{c}{\scriptsize{-0.156}} & \mc{1}{c}{\scriptsize{-0.706}} & \mc{1}{c}{\scriptsize{-0.727}} & \mc{1}{c}{\scriptsize{-0.503}} & \mc{1}{c}{\scriptsize{-0.742}} & \mc{1}{c}{\scriptsize{-0.112}} & \mc{1}{c}{\scriptsize{-0.630}} & \mc{1}{c}{\scriptsize{-0.131}} \\  

     &  & \mc{1}{c}{\scriptsize{(0.711)}} & \mc{1}{c}{\scriptsize{(0.763)}} & \mc{1}{c}{\scriptsize{(0.671)}} & \mc{1}{c}{\scriptsize{(0.276)}} & \mc{1}{c}{\scriptsize{(0.658)}} & \mc{1}{c}{\scriptsize{(0.632)}} & \mc{1}{c}{\scriptsize{(0.697)}} & \mc{1}{c}{\scriptsize{(0.671)}} \\  

    \mc{1}{l}{\scriptsize{Physical appearance}} & \mc{1}{c}{\scriptsize{12}} & \mc{1}{c}{\scriptsize{-0.646}} & \mc{1}{c}{\scriptsize{-0.492}} & \mc{1}{c}{\scriptsize{-1.682}} & \mc{1}{c}{\scriptsize{-0.050}} & \mc{1}{c}{\scriptsize{-1.661}} & \mc{1}{c}{\scriptsize{-0.566}} & \mc{1}{c}{\scriptsize{-0.450}} & \mc{1}{c}{\scriptsize{-0.557}} \\  

     &  & \mc{1}{c}{\scriptsize{(0.974)}} & \mc{1}{c}{\scriptsize{(0.711)}} & \mc{1}{c}{\scriptsize{(0.671)}} & \mc{1}{c}{\scriptsize{(0.184)}} & \mc{1}{c}{\scriptsize{(0.671)}} & \mc{1}{c}{\scriptsize{(0.961)}} & \mc{1}{c}{\scriptsize{(0.724)}} & \mc{1}{c}{\scriptsize{(0.947)}} \\  

    \mc{1}{l}{\scriptsize{Social acceptance}} & \mc{1}{c}{\scriptsize{12}} & \mc{1}{c}{\scriptsize{-0.471}} & \mc{1}{c}{\scriptsize{-0.495}} & \mc{1}{c}{\scriptsize{-1.364}} & \mc{1}{c}{\scriptsize{-0.878}} & \mc{1}{c}{\scriptsize{-1.276}} & \mc{1}{c}{\scriptsize{-0.402}} & \mc{1}{c}{\scriptsize{-0.378}} & \mc{1}{c}{\scriptsize{-0.297}} \\  

     &  & \mc{1}{c}{\scriptsize{(0.934)}} & \mc{1}{c}{\scriptsize{(0.711)}} & \mc{1}{c}{\scriptsize{(0.671)}} & \mc{1}{c}{\scriptsize{(0.368)}} & \mc{1}{c}{\scriptsize{(0.671)}} & \mc{1}{c}{\scriptsize{(0.921)}} & \mc{1}{c}{\scriptsize{(0.711)}} & \mc{1}{c}{\scriptsize{(0.750)}} \\  

  \bottomrule
  \end{tabular}
	\end{table} 

	\begin{table}[H]
     \caption{Treatment Effects on Achenbach Behavior, Female Sample}
     \label{table:abccare_rslt_female_cat8}
	  \begin{tabular}{cccccccccc}
  \toprule

    \scriptsize{Variable} & \scriptsize{Age} & \scriptsize{(1)} & \scriptsize{(2)} & \scriptsize{(3)} & \scriptsize{(4)} & \scriptsize{(5)} & \scriptsize{(6)} & \scriptsize{(7)} & \scriptsize{(8)} \\ 
    \midrule  

    \mc{1}{l}{\scriptsize{Total Felony Arrests}} & \mc{1}{c}{\scriptsize{Mid-30s}} & \mc{1}{c}{\scriptsize{-0.266}} & \mc{1}{c}{\scriptsize{-0.313}} & \mc{1}{c}{\scriptsize{-0.917}} & \mc{1}{c}{\scriptsize{-0.861}} & \mc{1}{c}{\scriptsize{-0.854}} & \mc{1}{c}{\scriptsize{-0.050}} & \mc{1}{c}{\scriptsize{-0.111}} & \mc{1}{c}{\scriptsize{0.002}} \\  

     &  & \mc{1}{c}{\scriptsize{(0.171)}} & \mc{1}{c}{\scriptsize{\textbf{(0.092)}}} & \mc{1}{c}{\scriptsize{(0.105)}} & \mc{1}{c}{\scriptsize{(0.145)}} & \mc{1}{c}{\scriptsize{\textbf{(0.092)}}} & \mc{1}{c}{\scriptsize{(0.276)}} & \mc{1}{c}{\scriptsize{(0.224)}} & \mc{1}{c}{\scriptsize{(0.513)}} \\  

    \mc{1}{l}{\scriptsize{Total Misdemeanor Arrests}} & \mc{1}{c}{\scriptsize{Mid-30s}} & \mc{1}{c}{\scriptsize{-0.918}} & \mc{1}{c}{\scriptsize{-1.154}} & \mc{1}{c}{\scriptsize{-2.345}} & \mc{1}{c}{\scriptsize{-1.959}} & \mc{1}{c}{\scriptsize{-2.560}} & \mc{1}{c}{\scriptsize{-0.475}} & \mc{1}{c}{\scriptsize{-0.553}} & \mc{1}{c}{\scriptsize{-0.175}} \\  

     &  & \mc{1}{c}{\scriptsize{\textbf{(0.079)}}} & \mc{1}{c}{\scriptsize{(0.132)}} & \mc{1}{c}{\scriptsize{\textbf{(0.092)}}} & \mc{1}{c}{\scriptsize{(0.171)}} & \mc{1}{c}{\scriptsize{(0.105)}} & \mc{1}{c}{\scriptsize{(0.158)}} & \mc{1}{c}{\scriptsize{(0.197)}} & \mc{1}{c}{\scriptsize{(0.250)}} \\  

    \mc{1}{l}{\scriptsize{Total Years Incarcerated}} & \mc{1}{c}{\scriptsize{30}} & \mc{1}{c}{\scriptsize{-0.020}} & \mc{1}{c}{\scriptsize{-0.012}} &  &  &  & \mc{1}{c}{\scriptsize{-0.030}} & \mc{1}{c}{\scriptsize{-0.015}} & \mc{1}{c}{\scriptsize{-0.031}} \\  

     &  & \mc{1}{c}{\scriptsize{\textbf{(0.039)}}} & \mc{1}{c}{\scriptsize{(0.171)}} &  &  &  & \mc{1}{c}{\scriptsize{\textbf{(0.026)}}} & \mc{1}{c}{\scriptsize{(0.145)}} & \mc{1}{c}{\scriptsize{\textbf{(0.039)}}} \\  

    \mc{1}{l}{\scriptsize{Crime Factor}} & \mc{1}{c}{\scriptsize{30 to Mid-30s}} & \mc{1}{c}{\scriptsize{-0.162}} & \mc{1}{c}{\scriptsize{-0.229}} & \mc{1}{c}{\scriptsize{-0.457}} & \mc{1}{c}{\scriptsize{-0.569}} & \mc{1}{c}{\scriptsize{-0.450}} & \mc{1}{c}{\scriptsize{-0.068}} & \mc{1}{c}{\scriptsize{-0.086}} & \mc{1}{c}{\scriptsize{-0.033}} \\  

     &  & \mc{1}{c}{\scriptsize{\textbf{(0.079)}}} & \mc{1}{c}{\scriptsize{\textbf{(0.079)}}} & \mc{1}{c}{\scriptsize{\textbf{(0.079)}}} & \mc{1}{c}{\scriptsize{(0.118)}} & \mc{1}{c}{\scriptsize{\textbf{(0.092)}}} & \mc{1}{c}{\scriptsize{(0.250)}} & \mc{1}{c}{\scriptsize{(0.211)}} & \mc{1}{c}{\scriptsize{(0.303)}} \\ 
    \midrule  

    \mc{2}{l}{\scriptsize{\% of Pos. TE ($H_0$: $\le$ 50\%)}} & \mc{1}{c}{\scriptsize{100}} & \mc{1}{c}{\scriptsize{100}} & \mc{1}{c}{\scriptsize{100}} & \mc{1}{c}{\scriptsize{100}} & \mc{1}{c}{\scriptsize{100}} & \mc{1}{c}{\scriptsize{100}} & \mc{1}{c}{\scriptsize{100}} & \mc{1}{c}{\scriptsize{75}} \\  

     &  & \mc{1}{c}{\scriptsize{\textbf{(0.000)}}} & \mc{1}{c}{\scriptsize{\textbf{(0.000)}}} & \mc{1}{c}{\scriptsize{\textbf{(0.000)}}} & \mc{1}{c}{\scriptsize{\textbf{(0.000)}}} & \mc{1}{c}{\scriptsize{\textbf{(0.000)}}} & \mc{1}{c}{\scriptsize{\textbf{(0.000)}}} & \mc{1}{c}{\scriptsize{\textbf{(0.000)}}} & \mc{1}{c}{\scriptsize{(0.526)}} \\  

    \mc{2}{l}{\scriptsize{\% of Pos. TE ($H_0$: $\le$ 10\% $|$ 10\% Significance)}} & \mc{1}{c}{\scriptsize{100}} & \mc{1}{c}{\scriptsize{75}} & \mc{1}{c}{\scriptsize{100}} & \mc{1}{c}{\scriptsize{33}} & \mc{1}{c}{\scriptsize{67}} & \mc{1}{c}{\scriptsize{25}} & \mc{1}{c}{\scriptsize{0}} & \mc{1}{c}{\scriptsize{25}} \\  

     &  & \mc{1}{c}{\scriptsize{\textbf{(0.000)}}} & \mc{1}{c}{\scriptsize{\textbf{(0.053)}}} & \mc{1}{c}{\scriptsize{\textbf{(0.000)}}} & \mc{1}{c}{\scriptsize{(0.118)}} & \mc{1}{c}{\scriptsize{(0.250)}} & \mc{1}{c}{\scriptsize{(0.145)}} & \mc{1}{c}{\scriptsize{(0.237)}} & \mc{1}{c}{\scriptsize{\textbf{(0.092)}}} \\  

  \bottomrule
  \end{tabular}
	\end{table} 

	\begin{table}[H]
     \caption{Treatment Effects on Achenbach Symptom T Score (Reported by Mother), Female Sample}
     \label{table:abccare_rslt_female_cat9}
	  \begin{tabular}{cccccccccc}
  \toprule

    \scriptsize{Variable} & \scriptsize{Age} & \scriptsize{(1)} & \scriptsize{(2)} & \scriptsize{(3)} & \scriptsize{(4)} & \scriptsize{(5)} & \scriptsize{(6)} & \scriptsize{(7)} & \scriptsize{(8)} \\ 
    \midrule  

    \mc{1}{l}{\scriptsize{Cig. Smoked per day last month}} & \mc{1}{c}{\scriptsize{30}} & \mc{1}{c}{\scriptsize{-0.772}} & \mc{1}{c}{\scriptsize{1.073}} & \mc{1}{c}{\scriptsize{-2.650}} & \mc{1}{c}{\scriptsize{-0.052}} & \mc{1}{c}{\scriptsize{-2.447}} & \mc{1}{c}{\scriptsize{-0.279}} & \mc{1}{c}{\scriptsize{1.318}} & \mc{1}{c}{\scriptsize{-0.045}} \\  

     &  & \mc{1}{c}{\scriptsize{(0.303)}} & \mc{1}{c}{\scriptsize{(0.750)}} & \mc{1}{c}{\scriptsize{(0.132)}} & \mc{1}{c}{\scriptsize{(0.526)}} & \mc{1}{c}{\scriptsize{(0.132)}} & \mc{1}{c}{\scriptsize{(0.421)}} & \mc{1}{c}{\scriptsize{(0.776)}} & \mc{1}{c}{\scriptsize{(0.461)}} \\  

    \mc{1}{l}{\scriptsize{Days drank alcohol last month}} & \mc{1}{c}{\scriptsize{30}} & \mc{1}{c}{\scriptsize{-1.077}} & \mc{1}{c}{\scriptsize{0.377}} & \mc{1}{c}{\scriptsize{-2.233}} & \mc{1}{c}{\scriptsize{-1.652}} & \mc{1}{c}{\scriptsize{-1.934}} & \mc{1}{c}{\scriptsize{-0.779}} & \mc{1}{c}{\scriptsize{0.509}} & \mc{1}{c}{\scriptsize{-0.349}} \\  

     &  & \mc{1}{c}{\scriptsize{(0.211)}} & \mc{1}{c}{\scriptsize{(0.618)}} & \mc{1}{c}{\scriptsize{(0.197)}} & \mc{1}{c}{\scriptsize{(0.237)}} & \mc{1}{c}{\scriptsize{(0.224)}} & \mc{1}{c}{\scriptsize{(0.303)}} & \mc{1}{c}{\scriptsize{(0.658)}} & \mc{1}{c}{\scriptsize{(0.382)}} \\  

    \mc{1}{l}{\scriptsize{Days binge drank alcohol last month}} & \mc{1}{c}{\scriptsize{30}} & \mc{1}{c}{\scriptsize{-0.144}} & \mc{1}{c}{\scriptsize{0.360}} & \mc{1}{c}{\scriptsize{-0.542}} & \mc{1}{c}{\scriptsize{-0.470}} & \mc{1}{c}{\scriptsize{-0.402}} & \mc{1}{c}{\scriptsize{-0.053}} & \mc{1}{c}{\scriptsize{0.647}} & \mc{1}{c}{\scriptsize{0.194}} \\  

     &  & \mc{1}{c}{\scriptsize{(0.461)}} & \mc{1}{c}{\scriptsize{(0.684)}} & \mc{1}{c}{\scriptsize{(0.329)}} & \mc{1}{c}{\scriptsize{(0.408)}} & \mc{1}{c}{\scriptsize{(0.355)}} & \mc{1}{c}{\scriptsize{(0.513)}} & \mc{1}{c}{\scriptsize{(0.776)}} & \mc{1}{c}{\scriptsize{(0.645)}} \\  

    \mc{1}{l}{\scriptsize{Self-reported drug user}} & \mc{1}{c}{\scriptsize{Mid-30s}} & \mc{1}{c}{\scriptsize{0.007}} & \mc{1}{c}{\scriptsize{-0.060}} & \mc{1}{c}{\scriptsize{-0.083}} & \mc{1}{c}{\scriptsize{-0.214}} & \mc{1}{c}{\scriptsize{-0.167}} & \mc{1}{c}{\scriptsize{0.039}} & \mc{1}{c}{\scriptsize{0.028}} & \mc{1}{c}{\scriptsize{-0.024}} \\  

     &  & \mc{1}{c}{\scriptsize{(0.566)}} & \mc{1}{c}{\scriptsize{(0.289)}} & \mc{1}{c}{\scriptsize{(0.263)}} & \mc{1}{c}{\scriptsize{(0.224)}} & \mc{1}{c}{\scriptsize{(0.171)}} & \mc{1}{c}{\scriptsize{(0.605)}} & \mc{1}{c}{\scriptsize{(0.592)}} & \mc{1}{c}{\scriptsize{(0.447)}} \\  

    \mc{1}{l}{\scriptsize{Substance Use Factor}} & \mc{1}{c}{\scriptsize{30 to Mid-30s}} & \mc{1}{c}{\scriptsize{0.017}} & \mc{1}{c}{\scriptsize{0.405}} & \mc{1}{c}{\scriptsize{0.090}} & \mc{1}{c}{\scriptsize{0.474}} & \mc{1}{c}{\scriptsize{0.065}} & \mc{1}{c}{\scriptsize{-0.009}} & \mc{1}{c}{\scriptsize{0.407}} & \mc{1}{c}{\scriptsize{0.005}} \\  

     &  & \mc{1}{c}{\scriptsize{(0.579)}} & \mc{1}{c}{\scriptsize{(0.934)}} & \mc{1}{c}{\scriptsize{(0.579)}} & \mc{1}{c}{\scriptsize{(0.921)}} & \mc{1}{c}{\scriptsize{(0.539)}} & \mc{1}{c}{\scriptsize{(0.526)}} & \mc{1}{c}{\scriptsize{(0.934)}} & \mc{1}{c}{\scriptsize{(0.513)}} \\ 
    \midrule  

    \mc{2}{l}{\scriptsize{\% of Pos. TE ($H_0$: $\le$ 50\%)}} & \mc{1}{c}{\scriptsize{60}} & \mc{1}{c}{\scriptsize{20}} & \mc{1}{c}{\scriptsize{80}} & \mc{1}{c}{\scriptsize{80}} & \mc{1}{c}{\scriptsize{80}} & \mc{1}{c}{\scriptsize{80}} & \mc{1}{c}{\scriptsize{0}} & \mc{1}{c}{\scriptsize{60}} \\  

     &  & \mc{1}{c}{\scriptsize{(0.474)}} & \mc{1}{c}{\scriptsize{(0.763)}} & \mc{1}{c}{\scriptsize{(0.171)}} & \mc{1}{c}{\scriptsize{(0.118)}} & \mc{1}{c}{\scriptsize{(0.171)}} & \mc{1}{c}{\scriptsize{(0.145)}} & \mc{1}{c}{\scriptsize{(1.000)}} & \mc{1}{c}{\scriptsize{(0.513)}} \\  

    \mc{2}{l}{\scriptsize{\% of Pos. TE ($H_0$: $\le$ 10\% $|$ 10\% Significance)}} & \mc{1}{c}{\scriptsize{0}} & \mc{1}{c}{\scriptsize{0}} & \mc{1}{c}{\scriptsize{0}} & \mc{1}{c}{\scriptsize{0}} & \mc{1}{c}{\scriptsize{0}} & \mc{1}{c}{\scriptsize{0}} & \mc{1}{c}{\scriptsize{0}} & \mc{1}{c}{\scriptsize{0}} \\  

     &  & \mc{1}{c}{\scriptsize{(1.000)}} & \mc{1}{c}{\scriptsize{(1.000)}} & \mc{1}{c}{\scriptsize{(0.316)}} & \mc{1}{c}{\scriptsize{(1.000)}} & \mc{1}{c}{\scriptsize{(0.382)}} & \mc{1}{c}{\scriptsize{(1.000)}} & \mc{1}{c}{\scriptsize{(1.000)}} & \mc{1}{c}{\scriptsize{(1.000)}} \\  

  \bottomrule
  \end{tabular}
	\end{table} 

	\begin{table}[H]
     \caption{Treatment Effects on Achenbach Symptom T Score (Reported by Teacher), Female Sample}
     \label{table:abccare_rslt_female_cat10}
	  \begin{tabular}{cccccccccc}
  \toprule

    \scriptsize{Variable} & \scriptsize{Age} & \scriptsize{(1)} & \scriptsize{(2)} & \scriptsize{(3)} & \scriptsize{(4)} & \scriptsize{(5)} & \scriptsize{(6)} & \scriptsize{(7)} & \scriptsize{(8)} \\ 
    \midrule  

    \mc{1}{l}{\scriptsize{Self-reported Health}} & \mc{1}{c}{\scriptsize{30}} & \mc{1}{c}{\scriptsize{-0.038}} & \mc{1}{c}{\scriptsize{-0.199}} & \mc{1}{c}{\scriptsize{-0.567}} & \mc{1}{c}{\scriptsize{-1.034}} & \mc{1}{c}{\scriptsize{-0.603}} & \mc{1}{c}{\scriptsize{0.115}} & \mc{1}{c}{\scriptsize{-0.038}} & \mc{1}{c}{\scriptsize{0.060}} \\  

     &  & \mc{1}{c}{\scriptsize{(0.395)}} & \mc{1}{c}{\scriptsize{(0.158)}} & \mc{1}{c}{\scriptsize{(0.132)}} & \mc{1}{c}{\scriptsize{\textbf{(0.013)}}} & \mc{1}{c}{\scriptsize{\textbf{(0.092)}}} & \mc{1}{c}{\scriptsize{(0.724)}} & \mc{1}{c}{\scriptsize{(0.474)}} & \mc{1}{c}{\scriptsize{(0.684)}} \\  

     & \mc{1}{c}{\scriptsize{Mid-30s}} & \mc{1}{c}{\scriptsize{0.066}} & \mc{1}{c}{\scriptsize{0.051}} & \mc{1}{c}{\scriptsize{-0.265}} & \mc{1}{c}{\scriptsize{-0.098}} & \mc{1}{c}{\scriptsize{-0.284}} & \mc{1}{c}{\scriptsize{0.185}} & \mc{1}{c}{\scriptsize{0.183}} & \mc{1}{c}{\scriptsize{0.172}} \\  

     &  & \mc{1}{c}{\scriptsize{(0.618)}} & \mc{1}{c}{\scriptsize{(0.566)}} & \mc{1}{c}{\scriptsize{(0.158)}} & \mc{1}{c}{\scriptsize{(0.382)}} & \mc{1}{c}{\scriptsize{(0.145)}} & \mc{1}{c}{\scriptsize{(0.697)}} & \mc{1}{c}{\scriptsize{(0.684)}} & \mc{1}{c}{\scriptsize{(0.750)}} \\  

    \mc{1}{l}{\scriptsize{Self-reported Health Factor}} & \mc{1}{c}{\scriptsize{30 to Mid-30s}} & \mc{1}{c}{\scriptsize{0.011}} & \mc{1}{c}{\scriptsize{0.075}} & \mc{1}{c}{\scriptsize{-0.014}} & \mc{1}{c}{\scriptsize{0.225}} & \mc{1}{c}{\scriptsize{0.033}} & \mc{1}{c}{\scriptsize{0.020}} & \mc{1}{c}{\scriptsize{0.039}} & \mc{1}{c}{\scriptsize{0.008}} \\  

     &  & \mc{1}{c}{\scriptsize{(0.526)}} & \mc{1}{c}{\scriptsize{(0.618)}} & \mc{1}{c}{\scriptsize{(0.513)}} & \mc{1}{c}{\scriptsize{(0.789)}} & \mc{1}{c}{\scriptsize{(0.566)}} & \mc{1}{c}{\scriptsize{(0.605)}} & \mc{1}{c}{\scriptsize{(0.566)}} & \mc{1}{c}{\scriptsize{(0.539)}} \\ 
    \midrule  

    \mc{2}{l}{\scriptsize{\% of Pos. TE ($H_0$: $\le$ 50\%)}} & \mc{1}{c}{\scriptsize{33}} & \mc{1}{c}{\scriptsize{33}} & \mc{1}{c}{\scriptsize{100}} & \mc{1}{c}{\scriptsize{67}} & \mc{1}{c}{\scriptsize{67}} & \mc{1}{c}{\scriptsize{0}} & \mc{1}{c}{\scriptsize{33}} & \mc{1}{c}{\scriptsize{0}} \\  

     &  & \mc{1}{c}{\scriptsize{(0.474)}} & \mc{1}{c}{\scriptsize{(0.447)}} & \mc{1}{c}{\scriptsize{\textbf{(0.000)}}} & \mc{1}{c}{\scriptsize{(0.158)}} & \mc{1}{c}{\scriptsize{(0.421)}} & \mc{1}{c}{\scriptsize{(1.000)}} & \mc{1}{c}{\scriptsize{(0.816)}} & \mc{1}{c}{\scriptsize{(1.000)}} \\  

    \mc{2}{l}{\scriptsize{\% of Pos. TE ($H_0$: $\le$ 10\% $|$ 10\% Significance)}} & \mc{1}{c}{\scriptsize{0}} & \mc{1}{c}{\scriptsize{0}} & \mc{1}{c}{\scriptsize{0}} & \mc{1}{c}{\scriptsize{33}} & \mc{1}{c}{\scriptsize{0}} & \mc{1}{c}{\scriptsize{0}} & \mc{1}{c}{\scriptsize{0}} & \mc{1}{c}{\scriptsize{0}} \\  

     &  & \mc{1}{c}{\scriptsize{(1.000)}} & \mc{1}{c}{\scriptsize{(1.000)}} & \mc{1}{c}{\scriptsize{(0.513)}} & \mc{1}{c}{\scriptsize{\textbf{(0.013)}}} & \mc{1}{c}{\scriptsize{(0.487)}} & \mc{1}{c}{\scriptsize{(1.000)}} & \mc{1}{c}{\scriptsize{(1.000)}} & \mc{1}{c}{\scriptsize{(1.000)}} \\  

  \bottomrule
  \end{tabular}
	\end{table} 

	\begin{table}[H]
     \caption{Treatment Effects on Child Assessment Schedule (CAS), Female Sample}
     \label{table:abccare_rslt_female_cat11}
	  \begin{tabular}{cccccccccc}
  \toprule

    \scriptsize{Variable} & \scriptsize{Age} & \scriptsize{(1)} & \scriptsize{(2)} & \scriptsize{(3)} & \scriptsize{(4)} & \scriptsize{(5)} & \scriptsize{(6)} & \scriptsize{(7)} & \scriptsize{(8)} \\ 
    \midrule  

    \mc{1}{l}{\scriptsize{Cig. Smoked per day last month}} & \mc{1}{c}{\scriptsize{30}} & \mc{1}{c}{\scriptsize{-0.319}} & \mc{1}{c}{\scriptsize{-0.340}} & \mc{1}{c}{\scriptsize{-2.140}} & \mc{1}{c}{\scriptsize{-1.592}} & \mc{1}{c}{\scriptsize{-1.788}} & \mc{1}{c}{\scriptsize{-0.163}} & \mc{1}{c}{\scriptsize{-0.184}} & \mc{1}{c}{\scriptsize{-0.371}} \\  

     &  & \mc{1}{c}{\scriptsize{(0.408)}} & \mc{1}{c}{\scriptsize{(0.355)}} & \mc{1}{c}{\scriptsize{(0.250)}} & \mc{1}{c}{\scriptsize{(0.368)}} & \mc{1}{c}{\scriptsize{(0.276)}} & \mc{1}{c}{\scriptsize{(0.421)}} & \mc{1}{c}{\scriptsize{(0.382)}} & \mc{1}{c}{\scriptsize{(0.329)}} \\  

    \mc{1}{l}{\scriptsize{Days drank alcohol last month}} & \mc{1}{c}{\scriptsize{30}} & \mc{1}{c}{\scriptsize{-0.356}} & \mc{1}{c}{\scriptsize{1.389}} & \mc{1}{c}{\scriptsize{-0.820}} & \mc{1}{c}{\scriptsize{-2.538}} & \mc{1}{c}{\scriptsize{-0.075}} & \mc{1}{c}{\scriptsize{-0.411}} & \mc{1}{c}{\scriptsize{1.869}} & \mc{1}{c}{\scriptsize{0.510}} \\  

     &  & \mc{1}{c}{\scriptsize{(0.382)}} & \mc{1}{c}{\scriptsize{(0.763)}} & \mc{1}{c}{\scriptsize{(0.368)}} & \mc{1}{c}{\scriptsize{(0.250)}} & \mc{1}{c}{\scriptsize{(0.500)}} & \mc{1}{c}{\scriptsize{(0.355)}} & \mc{1}{c}{\scriptsize{(0.803)}} & \mc{1}{c}{\scriptsize{(0.539)}} \\  

    \mc{1}{l}{\scriptsize{Days binge drank alcohol last month}} & \mc{1}{c}{\scriptsize{30}} & \mc{1}{c}{\scriptsize{-0.573}} & \mc{1}{c}{\scriptsize{0.578}} & \mc{1}{c}{\scriptsize{-2.805}} & \mc{1}{c}{\scriptsize{-1.778}} & \mc{1}{c}{\scriptsize{-2.331}} & \mc{1}{c}{\scriptsize{-0.294}} & \mc{1}{c}{\scriptsize{0.863}} & \mc{1}{c}{\scriptsize{0.060}} \\  

     &  & \mc{1}{c}{\scriptsize{(0.250)}} & \mc{1}{c}{\scriptsize{(0.671)}} & \mc{1}{c}{\scriptsize{(0.158)}} & \mc{1}{c}{\scriptsize{(0.316)}} & \mc{1}{c}{\scriptsize{(0.237)}} & \mc{1}{c}{\scriptsize{(0.382)}} & \mc{1}{c}{\scriptsize{(0.711)}} & \mc{1}{c}{\scriptsize{(0.579)}} \\  

    \mc{1}{l}{\scriptsize{Self-reported drug user}} & \mc{1}{c}{\scriptsize{Mid-30s}} & \mc{1}{c}{\scriptsize{0.051}} & \mc{1}{c}{\scriptsize{0.056}} & \mc{1}{c}{\scriptsize{-0.389}} & \mc{1}{c}{\scriptsize{-0.408}} & \mc{1}{c}{\scriptsize{-0.376}} & \mc{1}{c}{\scriptsize{0.120}} & \mc{1}{c}{\scriptsize{0.207}} & \mc{1}{c}{\scriptsize{0.121}} \\  

     &  & \mc{1}{c}{\scriptsize{(0.618)}} & \mc{1}{c}{\scriptsize{(0.618)}} & \mc{1}{c}{\scriptsize{\textbf{(0.092)}}} & \mc{1}{c}{\scriptsize{(0.237)}} & \mc{1}{c}{\scriptsize{(0.158)}} & \mc{1}{c}{\scriptsize{(0.803)}} & \mc{1}{c}{\scriptsize{(0.882)}} & \mc{1}{c}{\scriptsize{(0.842)}} \\  

    \mc{1}{l}{\scriptsize{Substance Use Factor}} & \mc{1}{c}{\scriptsize{30 to Mid-30s}} & \mc{1}{c}{\scriptsize{0.037}} & \mc{1}{c}{\scriptsize{0.524}} & \mc{1}{c}{\scriptsize{0.331}} & \mc{1}{c}{\scriptsize{0.310}} & \mc{1}{c}{\scriptsize{0.346}} & \mc{1}{c}{\scriptsize{-0.009}} & \mc{1}{c}{\scriptsize{0.553}} & \mc{1}{c}{\scriptsize{0.162}} \\  

     &  & \mc{1}{c}{\scriptsize{(0.539)}} & \mc{1}{c}{\scriptsize{(0.987)}} & \mc{1}{c}{\scriptsize{(0.934)}} & \mc{1}{c}{\scriptsize{(0.645)}} & \mc{1}{c}{\scriptsize{(0.934)}} & \mc{1}{c}{\scriptsize{(0.461)}} & \mc{1}{c}{\scriptsize{(0.974)}} & \mc{1}{c}{\scriptsize{(0.711)}} \\ 
    \midrule  

    \mc{2}{l}{\scriptsize{\% of Pos. TE ($H_0$: $\le$ 50\%)}} & \mc{1}{c}{\scriptsize{60}} & \mc{1}{c}{\scriptsize{20}} & \mc{1}{c}{\scriptsize{80}} & \mc{1}{c}{\scriptsize{80}} & \mc{1}{c}{\scriptsize{80}} & \mc{1}{c}{\scriptsize{80}} & \mc{1}{c}{\scriptsize{20}} & \mc{1}{c}{\scriptsize{20}} \\  

     &  & \mc{1}{c}{\scriptsize{(0.434)}} & \mc{1}{c}{\scriptsize{(0.776)}} & \mc{1}{c}{\scriptsize{\textbf{(0.039)}}} & \mc{1}{c}{\scriptsize{(0.303)}} & \mc{1}{c}{\scriptsize{\textbf{(0.053)}}} & \mc{1}{c}{\scriptsize{(0.171)}} & \mc{1}{c}{\scriptsize{(1.000)}} & \mc{1}{c}{\scriptsize{(0.829)}} \\  

    \mc{2}{l}{\scriptsize{\% of Pos. TE ($H_0$: $\le$ 10\% $|$ 10\% Significance)}} & \mc{1}{c}{\scriptsize{0}} & \mc{1}{c}{\scriptsize{0}} & \mc{1}{c}{\scriptsize{20}} & \mc{1}{c}{\scriptsize{0}} & \mc{1}{c}{\scriptsize{0}} & \mc{1}{c}{\scriptsize{0}} & \mc{1}{c}{\scriptsize{0}} & \mc{1}{c}{\scriptsize{0}} \\  

     &  & \mc{1}{c}{\scriptsize{(0.329)}} & \mc{1}{c}{\scriptsize{(1.000)}} & \mc{1}{c}{\scriptsize{\textbf{(0.066)}}} & \mc{1}{c}{\scriptsize{(0.342)}} & \mc{1}{c}{\scriptsize{(0.539)}} & \mc{1}{c}{\scriptsize{(1.000)}} & \mc{1}{c}{\scriptsize{(1.000)}} & \mc{1}{c}{\scriptsize{(1.000)}} \\  

  \bottomrule
  \end{tabular}
	\end{table} 

	\begin{table}[H]
     \caption{Treatment Effects on Mother's Income, Female Sample}
     \label{table:abccare_rslt_female_cat12}
	  \begin{tabular}{cccccccccc}
  \toprule

    \scriptsize{Variable} & \scriptsize{Age} & \scriptsize{(1)} & \scriptsize{(2)} & \scriptsize{(3)} & \scriptsize{(4)} & \scriptsize{(5)} & \scriptsize{(6)} & \scriptsize{(7)} & \scriptsize{(8)} \\ 
    \midrule  

    \mc{1}{l}{\scriptsize{Mouth and Throat: Upper Teeth}} & \mc{1}{c}{\scriptsize{Mid-30s}} & \mc{1}{c}{\scriptsize{0.082}} & \mc{1}{c}{\scriptsize{0.106}} & \mc{1}{c}{\scriptsize{0.094}} & \mc{1}{c}{\scriptsize{0.221}} & \mc{1}{c}{\scriptsize{0.110}} & \mc{1}{c}{\scriptsize{0.079}} & \mc{1}{c}{\scriptsize{0.080}} & \mc{1}{c}{\scriptsize{0.128}} \\  

     &  & \mc{1}{c}{\scriptsize{(0.763)}} & \mc{1}{c}{\scriptsize{(0.829)}} & \mc{1}{c}{\scriptsize{(0.658)}} & \mc{1}{c}{\scriptsize{(0.829)}} & \mc{1}{c}{\scriptsize{(0.737)}} & \mc{1}{c}{\scriptsize{(0.763)}} & \mc{1}{c}{\scriptsize{(0.776)}} & \mc{1}{c}{\scriptsize{(0.868)}} \\  

    \mc{1}{l}{\scriptsize{Lymphatic: General Head and Neck}} & \mc{1}{c}{\scriptsize{Mid-30s}} &  &  &  &  &  &  &  &  \\  

     &  &  &  &  &  &  &  &  &  \\  

    \mc{1}{l}{\scriptsize{Muscle Strength: Upper Extremities}} & \mc{1}{c}{\scriptsize{Mid-30s}} &  &  &  &  &  &  &  &  \\  

     &  &  &  &  &  &  &  &  &  \\  

    \mc{1}{l}{\scriptsize{Abdomen: Auscultation}} & \mc{1}{c}{\scriptsize{Mid-30s}} &  &  &  &  &  &  &  &  \\  

     &  &  &  &  &  &  &  &  &  \\  

    \mc{1}{l}{\scriptsize{Abdomen: Inspection}} & \mc{1}{c}{\scriptsize{Mid-30s}} &  &  &  &  &  &  &  &  \\  

     &  &  &  &  &  &  &  &  &  \\  

    \mc{1}{l}{\scriptsize{Mouth and Throat: Tonsils}} & \mc{1}{c}{\scriptsize{Mid-30s}} &  &  &  &  &  &  &  &  \\  

     &  &  &  &  &  &  &  &  &  \\  

    \mc{1}{l}{\scriptsize{Muscle Strength: Brainstem Reflexes}} & \mc{1}{c}{\scriptsize{Mid-30s}} &  &  &  &  &  &  &  &  \\  

     &  &  &  &  &  &  &  &  &  \\  

    \mc{1}{l}{\scriptsize{Muscle Strength: Lower Extremities}} & \mc{1}{c}{\scriptsize{Mid-30s}} &  &  &  &  &  &  &  &  \\  

     &  &  &  &  &  &  &  &  &  \\  

    \mc{1}{l}{\scriptsize{Lymphatic: General Femoral and Inguinal Lymphatics}} & \mc{1}{c}{\scriptsize{Mid-30s}} &  &  &  &  &  &  &  &  \\  

     &  &  &  &  &  &  &  &  &  \\  

    \mc{1}{l}{\scriptsize{Muscle Strength: Reflexes}} & \mc{1}{c}{\scriptsize{Mid-30s}} & \mc{1}{c}{\scriptsize{0.087}} & \mc{1}{c}{\scriptsize{0.134}} & \mc{1}{c}{\scriptsize{0.087}} & \mc{1}{c}{\scriptsize{0.163}} & \mc{1}{c}{\scriptsize{0.087}} & \mc{1}{c}{\scriptsize{0.087}} & \mc{1}{c}{\scriptsize{0.131}} & \mc{1}{c}{\scriptsize{0.086}} \\  

     &  & \mc{1}{c}{\scriptsize{(0.803)}} & \mc{1}{c}{\scriptsize{(0.855)}} & \mc{1}{c}{\scriptsize{(0.803)}} & \mc{1}{c}{\scriptsize{(0.750)}} & \mc{1}{c}{\scriptsize{(0.803)}} & \mc{1}{c}{\scriptsize{(0.803)}} & \mc{1}{c}{\scriptsize{(0.842)}} & \mc{1}{c}{\scriptsize{(0.803)}} \\  

    \mc{1}{l}{\scriptsize{Mouth and Throat: Oropharynx}} & \mc{1}{c}{\scriptsize{Mid-30s}} &  &  &  &  &  &  &  &  \\  

     &  &  &  &  &  &  &  &  &  \\  

    \mc{1}{l}{\scriptsize{Lymphatic: General Axillary Region}} & \mc{1}{c}{\scriptsize{Mid-30s}} &  &  &  &  &  &  &  &  \\  

     &  &  &  &  &  &  &  &  &  \\  

    \mc{1}{l}{\scriptsize{Muscle Strength: Gait}} & \mc{1}{c}{\scriptsize{Mid-30s}} &  &  &  &  &  &  &  &  \\  

     &  &  &  &  &  &  &  &  &  \\  

    \mc{1}{l}{\scriptsize{Abdomen: Palpation/Percussion}} & \mc{1}{c}{\scriptsize{Mid-30s}} &  &  &  &  &  &  &  &  \\  

     &  &  &  &  &  &  &  &  &  \\  

    \mc{1}{l}{\scriptsize{Mouth and Throat: Floor of Mouth}} & \mc{1}{c}{\scriptsize{Mid-30s}} &  &  &  &  &  &  &  &  \\  

     &  &  &  &  &  &  &  &  &  \\  

    \mc{1}{l}{\scriptsize{Mouth and Throat: Lower Teeth}} & \mc{1}{c}{\scriptsize{Mid-30s}} & \mc{1}{c}{\scriptsize{0.031}} & \mc{1}{c}{\scriptsize{0.003}} & \mc{1}{c}{\scriptsize{0.007}} & \mc{1}{c}{\scriptsize{-0.062}} & \mc{1}{c}{\scriptsize{0.029}} & \mc{1}{c}{\scriptsize{0.037}} & \mc{1}{c}{\scriptsize{0.022}} & \mc{1}{c}{\scriptsize{0.095}} \\  

     &  & \mc{1}{c}{\scriptsize{(0.632)}} & \mc{1}{c}{\scriptsize{(0.566)}} & \mc{1}{c}{\scriptsize{(0.500)}} & \mc{1}{c}{\scriptsize{(0.289)}} & \mc{1}{c}{\scriptsize{(0.474)}} & \mc{1}{c}{\scriptsize{(0.658)}} & \mc{1}{c}{\scriptsize{(0.566)}} & \mc{1}{c}{\scriptsize{(0.855)}} \\  

    \mc{1}{l}{\scriptsize{Muscle Strength: Coordination}} & \mc{1}{c}{\scriptsize{Mid-30s}} &  &  &  &  &  &  &  &  \\  

     &  &  &  &  &  &  &  &  &  \\  

  \bottomrule
  \end{tabular}
	\end{table} 

	\begin{table}[H]
     \caption{Treatment Effects on Parental Labor Income, Female Sample}
     \label{table:abccare_rslt_female_cat13}
	  \begin{tabular}{cccccccccc}
  \toprule

    \scriptsize{Variable} & \scriptsize{Age} & \scriptsize{(1)} & \scriptsize{(2)} & \scriptsize{(3)} & \scriptsize{(4)} & \scriptsize{(5)} & \scriptsize{(6)} & \scriptsize{(7)} & \scriptsize{(8)} \\ 
    \midrule  

    \mc{1}{l}{\scriptsize{Hemoglobin Level (\%)}} & \mc{1}{c}{\scriptsize{Mid-30s}} & \mc{1}{c}{\scriptsize{-0.277}} & \mc{1}{c}{\scriptsize{-0.101}} & \mc{1}{c}{\scriptsize{-0.176}} & \mc{1}{c}{\scriptsize{-0.088}} & \mc{1}{c}{\scriptsize{-0.190}} & \mc{1}{c}{\scriptsize{-0.304}} & \mc{1}{c}{\scriptsize{-0.037}} & \mc{1}{c}{\scriptsize{-0.355}} \\  

     &  & \mc{1}{c}{\scriptsize{(0.211)}} & \mc{1}{c}{\scriptsize{(0.316)}} & \mc{1}{c}{\scriptsize{\textbf{(0.092)}}} & \mc{1}{c}{\scriptsize{(0.355)}} & \mc{1}{c}{\scriptsize{(0.105)}} & \mc{1}{c}{\scriptsize{(0.211)}} & \mc{1}{c}{\scriptsize{(0.474)}} & \mc{1}{c}{\scriptsize{(0.197)}} \\  

    \mc{1}{l}{\scriptsize{Prediabetes}} & \mc{1}{c}{\scriptsize{Mid-30s}} & \mc{1}{c}{\scriptsize{0.088}} & \mc{1}{c}{\scriptsize{0.163}} & \mc{1}{c}{\scriptsize{0.076}} & \mc{1}{c}{\scriptsize{0.176}} & \mc{1}{c}{\scriptsize{0.035}} & \mc{1}{c}{\scriptsize{0.091}} & \mc{1}{c}{\scriptsize{0.161}} & \mc{1}{c}{\scriptsize{0.029}} \\  

     &  & \mc{1}{c}{\scriptsize{(0.737)}} & \mc{1}{c}{\scriptsize{(0.789)}} & \mc{1}{c}{\scriptsize{(0.632)}} & \mc{1}{c}{\scriptsize{(0.763)}} & \mc{1}{c}{\scriptsize{(0.526)}} & \mc{1}{c}{\scriptsize{(0.697)}} & \mc{1}{c}{\scriptsize{(0.750)}} & \mc{1}{c}{\scriptsize{(0.579)}} \\  

    \mc{1}{l}{\scriptsize{Diabetes}} & \mc{1}{c}{\scriptsize{Mid-30s}} & \mc{1}{c}{\scriptsize{-0.071}} & \mc{1}{c}{\scriptsize{-0.032}} &  &  &  & \mc{1}{c}{\scriptsize{-0.091}} & \mc{1}{c}{\scriptsize{-0.039}} & \mc{1}{c}{\scriptsize{-0.095}} \\  

     &  & \mc{1}{c}{\scriptsize{\textbf{(0.066)}}} & \mc{1}{c}{\scriptsize{(0.171)}} &  &  &  & \mc{1}{c}{\scriptsize{\textbf{(0.066)}}} & \mc{1}{c}{\scriptsize{(0.171)}} & \mc{1}{c}{\scriptsize{\textbf{(0.039)}}} \\  

    \mc{1}{l}{\scriptsize{Diabetes Factor}} & \mc{1}{c}{\scriptsize{Mid-30s}} & \mc{1}{c}{\scriptsize{-0.249}} & \mc{1}{c}{\scriptsize{-0.086}} & \mc{1}{c}{\scriptsize{-0.086}} & \mc{1}{c}{\scriptsize{-0.029}} & \mc{1}{c}{\scriptsize{-0.098}} & \mc{1}{c}{\scriptsize{-0.294}} & \mc{1}{c}{\scriptsize{-0.064}} & \mc{1}{c}{\scriptsize{-0.334}} \\  

     &  & \mc{1}{c}{\scriptsize{(0.145)}} & \mc{1}{c}{\scriptsize{(0.250)}} & \mc{1}{c}{\scriptsize{(0.211)}} & \mc{1}{c}{\scriptsize{(0.382)}} & \mc{1}{c}{\scriptsize{(0.158)}} & \mc{1}{c}{\scriptsize{(0.145)}} & \mc{1}{c}{\scriptsize{(0.408)}} & \mc{1}{c}{\scriptsize{(0.145)}} \\ 
    \midrule  

    \mc{2}{l}{\scriptsize{\% of Pos. TE ($H_0$: $\le$ 50\%)}} & \mc{1}{c}{\scriptsize{75}} & \mc{1}{c}{\scriptsize{75}} & \mc{1}{c}{\scriptsize{67}} & \mc{1}{c}{\scriptsize{67}} & \mc{1}{c}{\scriptsize{67}} & \mc{1}{c}{\scriptsize{75}} & \mc{1}{c}{\scriptsize{75}} & \mc{1}{c}{\scriptsize{75}} \\  

     &  & \mc{1}{c}{\scriptsize{\textbf{(0.000)}}} & \mc{1}{c}{\scriptsize{(0.158)}} & \mc{1}{c}{\scriptsize{(0.395)}} & \mc{1}{c}{\scriptsize{(0.592)}} & \mc{1}{c}{\scriptsize{(0.395)}} & \mc{1}{c}{\scriptsize{(0.276)}} & \mc{1}{c}{\scriptsize{(0.158)}} & \mc{1}{c}{\scriptsize{\textbf{(0.000)}}} \\  

    \mc{2}{l}{\scriptsize{\% of Pos. TE ($H_0$: $\le$ 10\% $|$ 10\% Significance)}} & \mc{1}{c}{\scriptsize{50}} & \mc{1}{c}{\scriptsize{0}} & \mc{1}{c}{\scriptsize{0}} & \mc{1}{c}{\scriptsize{0}} & \mc{1}{c}{\scriptsize{0}} & \mc{1}{c}{\scriptsize{25}} & \mc{1}{c}{\scriptsize{0}} & \mc{1}{c}{\scriptsize{50}} \\  

     &  & \mc{1}{c}{\scriptsize{\textbf{(0.066)}}} & \mc{1}{c}{\scriptsize{(1.000)}} & \mc{1}{c}{\scriptsize{(0.368)}} & \mc{1}{c}{\scriptsize{(1.000)}} & \mc{1}{c}{\scriptsize{(0.342)}} & \mc{1}{c}{\scriptsize{(0.237)}} & \mc{1}{c}{\scriptsize{(1.000)}} & \mc{1}{c}{\scriptsize{\textbf{(0.066)}}} \\  

  \bottomrule
  \end{tabular}
	\end{table} 

	\begin{table}[H]
     \caption{Treatment Effects on Parental Public Transfer Income, Female Sample}
     \label{table:abccare_rslt_female_cat14}
	  \begin{tabular}{cccccccccc}
  \toprule

    \scriptsize{Variable} & \scriptsize{Age} & \scriptsize{(1)} & \scriptsize{(2)} & \scriptsize{(3)} & \scriptsize{(4)} & \scriptsize{(5)} & \scriptsize{(6)} & \scriptsize{(7)} & \scriptsize{(8)} \\ 
    \midrule  

    \mc{1}{l}{\scriptsize{Vitamin D Deficiency}} & \mc{1}{c}{\scriptsize{Mid-30s}} & \mc{1}{c}{\scriptsize{-0.047}} & \mc{1}{c}{\scriptsize{0.061}} & \mc{1}{c}{\scriptsize{-0.094}} & \mc{1}{c}{\scriptsize{-0.007}} & \mc{1}{c}{\scriptsize{-0.034}} & \mc{1}{c}{\scriptsize{-0.034}} & \mc{1}{c}{\scriptsize{0.093}} & \mc{1}{c}{\scriptsize{0.038}} \\  

     &  & \mc{1}{c}{\scriptsize{(0.303)}} & \mc{1}{c}{\scriptsize{(0.658)}} & \mc{1}{c}{\scriptsize{(0.316)}} & \mc{1}{c}{\scriptsize{(0.368)}} & \mc{1}{c}{\scriptsize{(0.434)}} & \mc{1}{c}{\scriptsize{(0.355)}} & \mc{1}{c}{\scriptsize{(0.711)}} & \mc{1}{c}{\scriptsize{(0.553)}} \\  

  \bottomrule
  \end{tabular}
	\end{table} 

	\begin{table}[H]
     \caption{Treatment Effects on Adoption, Female Sample}
     \label{table:abccare_rslt_female_cat15}
	  \begin{tabular}{cccccccccc}
  \toprule

    \scriptsize{Variable} & \scriptsize{Age} & \scriptsize{(1)} & \scriptsize{(2)} & \scriptsize{(3)} & \scriptsize{(4)} & \scriptsize{(5)} & \scriptsize{(6)} & \scriptsize{(7)} & \scriptsize{(8)} \\ 
    \midrule  

    \mc{1}{l}{\scriptsize{Nose General}} & \mc{1}{c}{\scriptsize{Mid-30s}} &  &  &  &  &  &  &  &  \\  

     &  &  &  &  &  &  &  &  &  \\  

    \mc{1}{l}{\scriptsize{Breast General}} & \mc{1}{c}{\scriptsize{Mid-30s}} &  &  &  &  &  &  &  &  \\  

     &  &  &  &  &  &  &  &  &  \\  

    \mc{1}{l}{\scriptsize{Chest and Lung General}} & \mc{1}{c}{\scriptsize{Mid-30s}} & \mc{1}{c}{\scriptsize{0.087}} & \mc{1}{c}{\scriptsize{0.112}} & \mc{1}{c}{\scriptsize{0.087}} & \mc{1}{c}{\scriptsize{0.011}} & \mc{1}{c}{\scriptsize{0.102}} & \mc{1}{c}{\scriptsize{0.087}} & \mc{1}{c}{\scriptsize{0.139}} & \mc{1}{c}{\scriptsize{0.103}} \\  

     &  & \mc{1}{c}{\scriptsize{(0.579)}} & \mc{1}{c}{\scriptsize{(0.579)}} & \mc{1}{c}{\scriptsize{(0.579)}} & \mc{1}{c}{\scriptsize{(0.382)}} & \mc{1}{c}{\scriptsize{(0.566)}} & \mc{1}{c}{\scriptsize{(0.579)}} & \mc{1}{c}{\scriptsize{(0.592)}} & \mc{1}{c}{\scriptsize{(0.566)}} \\  

    \mc{1}{l}{\scriptsize{Cardiovascular General}} & \mc{1}{c}{\scriptsize{Mid-30s}} & \mc{1}{c}{\scriptsize{-0.036}} &  &  &  &  & \mc{1}{c}{\scriptsize{-0.045}} &  &  \\  

     &  & \mc{1}{c}{\scriptsize{\textbf{(0.053)}}} &  &  &  &  & \mc{1}{c}{\scriptsize{\textbf{(0.066)}}} &  &  \\  

    \mc{1}{l}{\scriptsize{Skin General}} & \mc{1}{c}{\scriptsize{Mid-30s}} & \mc{1}{c}{\scriptsize{-0.056}} & \mc{1}{c}{\scriptsize{-0.072}} & \mc{1}{c}{\scriptsize{-0.246}} & \mc{1}{c}{\scriptsize{-0.177}} & \mc{1}{c}{\scriptsize{-0.233}} & \mc{1}{c}{\scriptsize{-0.004}} & \mc{1}{c}{\scriptsize{-0.054}} & \mc{1}{c}{\scriptsize{-0.006}} \\  

     &  & \mc{1}{c}{\scriptsize{(0.276)}} & \mc{1}{c}{\scriptsize{(0.224)}} & \mc{1}{c}{\scriptsize{(0.118)}} & \mc{1}{c}{\scriptsize{(0.132)}} & \mc{1}{c}{\scriptsize{(0.105)}} & \mc{1}{c}{\scriptsize{(0.461)}} & \mc{1}{c}{\scriptsize{(0.329)}} & \mc{1}{c}{\scriptsize{(0.461)}} \\  

    \mc{1}{l}{\scriptsize{Musculoskeletal General}} & \mc{1}{c}{\scriptsize{Mid-30s}} & \mc{1}{c}{\scriptsize{0.043}} & \mc{1}{c}{\scriptsize{0.081}} & \mc{1}{c}{\scriptsize{0.043}} & \mc{1}{c}{\scriptsize{0.115}} & \mc{1}{c}{\scriptsize{0.036}} & \mc{1}{c}{\scriptsize{0.043}} & \mc{1}{c}{\scriptsize{0.076}} & \mc{1}{c}{\scriptsize{0.035}} \\  

     &  & \mc{1}{c}{\scriptsize{(0.566)}} & \mc{1}{c}{\scriptsize{(0.579)}} & \mc{1}{c}{\scriptsize{(0.566)}} & \mc{1}{c}{\scriptsize{(0.605)}} & \mc{1}{c}{\scriptsize{(0.553)}} & \mc{1}{c}{\scriptsize{(0.566)}} & \mc{1}{c}{\scriptsize{(0.579)}} & \mc{1}{c}{\scriptsize{(0.553)}} \\  

    \mc{1}{l}{\scriptsize{Neurologic General}} & \mc{1}{c}{\scriptsize{Mid-30s}} &  &  &  &  &  &  &  &  \\  

     &  &  &  &  &  &  &  &  &  \\  

    \mc{1}{l}{\scriptsize{Neck General}} & \mc{1}{c}{\scriptsize{Mid-30s}} &  &  &  &  &  &  &  &  \\  

     &  &  &  &  &  &  &  &  &  \\  

    \mc{1}{l}{\scriptsize{Head General}} & \mc{1}{c}{\scriptsize{Mid-30s}} & \mc{1}{c}{\scriptsize{-0.036}} & \mc{1}{c}{\scriptsize{-0.026}} & \mc{1}{c}{\scriptsize{-0.167}} & \mc{1}{c}{\scriptsize{-0.162}} & \mc{1}{c}{\scriptsize{-0.165}} &  &  &  \\  

     &  & \mc{1}{c}{\scriptsize{(0.105)}} & \mc{1}{c}{\scriptsize{(0.145)}} & \mc{1}{c}{\scriptsize{\textbf{(0.053)}}} & \mc{1}{c}{\scriptsize{\textbf{(0.092)}}} & \mc{1}{c}{\scriptsize{\textbf{(0.053)}}} &  &  &  \\  

  \bottomrule
  \end{tabular}
	\end{table} 

	\begin{table}[H]
     \caption{Treatment Effects on Childhood Household Income, Female Sample}
     \label{table:abccare_rslt_female_cat16}
	  \begin{tabular}{cccccccccc}
  \toprule

    \scriptsize{Variable} & \scriptsize{Age} & \scriptsize{(1)} & \scriptsize{(2)} & \scriptsize{(3)} & \scriptsize{(4)} & \scriptsize{(5)} & \scriptsize{(6)} & \scriptsize{(7)} & \scriptsize{(8)} \\ 
    \midrule  

    \mc{1}{l}{\scriptsize{Somatization}} & \mc{1}{c}{\scriptsize{21}} & \mc{1}{c}{\scriptsize{-0.023}} & \mc{1}{c}{\scriptsize{-0.169}} & \mc{1}{c}{\scriptsize{-0.186}} & \mc{1}{c}{\scriptsize{-0.436}} & \mc{1}{c}{\scriptsize{-0.274}} & \mc{1}{c}{\scriptsize{0.027}} & \mc{1}{c}{\scriptsize{-0.096}} & \mc{1}{c}{\scriptsize{-0.064}} \\  

     &  & \mc{1}{c}{\scriptsize{(0.434)}} & \mc{1}{c}{\scriptsize{(0.132)}} & \mc{1}{c}{\scriptsize{(0.224)}} & \mc{1}{c}{\scriptsize{\textbf{(0.079)}}} & \mc{1}{c}{\scriptsize{(0.118)}} & \mc{1}{c}{\scriptsize{(0.474)}} & \mc{1}{c}{\scriptsize{(0.329)}} & \mc{1}{c}{\scriptsize{(0.382)}} \\  

     & \mc{1}{c}{\scriptsize{34}} & \mc{1}{c}{\scriptsize{0.020}} & \mc{1}{c}{\scriptsize{-0.051}} & \mc{1}{c}{\scriptsize{-0.099}} & \mc{1}{c}{\scriptsize{-0.149}} & \mc{1}{c}{\scriptsize{-0.251}} & \mc{1}{c}{\scriptsize{0.063}} & \mc{1}{c}{\scriptsize{0.026}} & \mc{1}{c}{\scriptsize{-0.052}} \\  

     &  & \mc{1}{c}{\scriptsize{(0.566)}} & \mc{1}{c}{\scriptsize{(0.382)}} & \mc{1}{c}{\scriptsize{(0.303)}} & \mc{1}{c}{\scriptsize{(0.316)}} & \mc{1}{c}{\scriptsize{(0.197)}} & \mc{1}{c}{\scriptsize{(0.605)}} & \mc{1}{c}{\scriptsize{(0.553)}} & \mc{1}{c}{\scriptsize{(0.408)}} \\  

    \mc{1}{l}{\scriptsize{Depression}} & \mc{1}{c}{\scriptsize{21}} & \mc{1}{c}{\scriptsize{-0.341}} & \mc{1}{c}{\scriptsize{-0.446}} & \mc{1}{c}{\scriptsize{-0.526}} & \mc{1}{c}{\scriptsize{-1.010}} & \mc{1}{c}{\scriptsize{-0.521}} & \mc{1}{c}{\scriptsize{-0.245}} & \mc{1}{c}{\scriptsize{-0.293}} & \mc{1}{c}{\scriptsize{-0.216}} \\  

     &  & \mc{1}{c}{\scriptsize{\textbf{(0.013)}}} & \mc{1}{c}{\scriptsize{\textbf{(0.039)}}} & \mc{1}{c}{\scriptsize{\textbf{(0.026)}}} & \mc{1}{c}{\scriptsize{\textbf{(0.000)}}} & \mc{1}{c}{\scriptsize{\textbf{(0.026)}}} & \mc{1}{c}{\scriptsize{\textbf{(0.092)}}} & \mc{1}{c}{\scriptsize{(0.145)}} & \mc{1}{c}{\scriptsize{(0.132)}} \\  

     & \mc{1}{c}{\scriptsize{34}} & \mc{1}{c}{\scriptsize{-0.087}} & \mc{1}{c}{\scriptsize{-0.146}} & \mc{1}{c}{\scriptsize{-0.015}} & \mc{1}{c}{\scriptsize{-0.155}} & \mc{1}{c}{\scriptsize{-0.223}} & \mc{1}{c}{\scriptsize{-0.113}} & \mc{1}{c}{\scriptsize{-0.160}} & \mc{1}{c}{\scriptsize{-0.325}} \\  

     &  & \mc{1}{c}{\scriptsize{(0.355)}} & \mc{1}{c}{\scriptsize{(0.263)}} & \mc{1}{c}{\scriptsize{(0.421)}} & \mc{1}{c}{\scriptsize{(0.197)}} & \mc{1}{c}{\scriptsize{(0.197)}} & \mc{1}{c}{\scriptsize{(0.368)}} & \mc{1}{c}{\scriptsize{(0.237)}} & \mc{1}{c}{\scriptsize{(0.105)}} \\  

    \mc{1}{l}{\scriptsize{Anxiety}} & \mc{1}{c}{\scriptsize{21}} & \mc{1}{c}{\scriptsize{-0.324}} & \mc{1}{c}{\scriptsize{-0.333}} & \mc{1}{c}{\scriptsize{-0.512}} & \mc{1}{c}{\scriptsize{-0.670}} & \mc{1}{c}{\scriptsize{-0.473}} & \mc{1}{c}{\scriptsize{-0.222}} & \mc{1}{c}{\scriptsize{-0.187}} & \mc{1}{c}{\scriptsize{-0.172}} \\  

     &  & \mc{1}{c}{\scriptsize{\textbf{(0.013)}}} & \mc{1}{c}{\scriptsize{\textbf{(0.026)}}} & \mc{1}{c}{\scriptsize{\textbf{(0.000)}}} & \mc{1}{c}{\scriptsize{\textbf{(0.000)}}} & \mc{1}{c}{\scriptsize{\textbf{(0.013)}}} & \mc{1}{c}{\scriptsize{\textbf{(0.066)}}} & \mc{1}{c}{\scriptsize{(0.171)}} & \mc{1}{c}{\scriptsize{(0.184)}} \\  

     & \mc{1}{c}{\scriptsize{34}} & \mc{1}{c}{\scriptsize{-0.150}} & \mc{1}{c}{\scriptsize{-0.198}} & \mc{1}{c}{\scriptsize{-0.186}} & \mc{1}{c}{\scriptsize{-0.397}} & \mc{1}{c}{\scriptsize{-0.365}} & \mc{1}{c}{\scriptsize{-0.137}} & \mc{1}{c}{\scriptsize{-0.108}} & \mc{1}{c}{\scriptsize{-0.304}} \\  

     &  & \mc{1}{c}{\scriptsize{(0.224)}} & \mc{1}{c}{\scriptsize{(0.184)}} & \mc{1}{c}{\scriptsize{(0.237)}} & \mc{1}{c}{\scriptsize{(0.132)}} & \mc{1}{c}{\scriptsize{(0.118)}} & \mc{1}{c}{\scriptsize{(0.276)}} & \mc{1}{c}{\scriptsize{(0.197)}} & \mc{1}{c}{\scriptsize{\textbf{(0.053)}}} \\  

    \mc{1}{l}{\scriptsize{Hostility}} & \mc{1}{c}{\scriptsize{21}} & \mc{1}{c}{\scriptsize{-0.323}} & \mc{1}{c}{\scriptsize{-0.513}} & \mc{1}{c}{\scriptsize{-0.695}} & \mc{1}{c}{\scriptsize{-1.060}} & \mc{1}{c}{\scriptsize{-0.767}} & \mc{1}{c}{\scriptsize{-0.124}} & \mc{1}{c}{\scriptsize{-0.312}} & \mc{1}{c}{\scriptsize{-0.186}} \\  

     &  & \mc{1}{c}{\scriptsize{\textbf{(0.053)}}} & \mc{1}{c}{\scriptsize{\textbf{(0.026)}}} & \mc{1}{c}{\scriptsize{\textbf{(0.000)}}} & \mc{1}{c}{\scriptsize{\textbf{(0.000)}}} & \mc{1}{c}{\scriptsize{\textbf{(0.000)}}} & \mc{1}{c}{\scriptsize{(0.263)}} & \mc{1}{c}{\scriptsize{\textbf{(0.079)}}} & \mc{1}{c}{\scriptsize{(0.145)}} \\  

     & \mc{1}{c}{\scriptsize{34}} & \mc{1}{c}{\scriptsize{-0.025}} & \mc{1}{c}{\scriptsize{-0.027}} & \mc{1}{c}{\scriptsize{-0.069}} & \mc{1}{c}{\scriptsize{-0.217}} & \mc{1}{c}{\scriptsize{-0.206}} & \mc{1}{c}{\scriptsize{-0.009}} & \mc{1}{c}{\scriptsize{0.066}} & \mc{1}{c}{\scriptsize{-0.135}} \\  

     &  & \mc{1}{c}{\scriptsize{(0.434)}} & \mc{1}{c}{\scriptsize{(0.434)}} & \mc{1}{c}{\scriptsize{(0.368)}} & \mc{1}{c}{\scriptsize{(0.263)}} & \mc{1}{c}{\scriptsize{(0.132)}} & \mc{1}{c}{\scriptsize{(0.434)}} & \mc{1}{c}{\scriptsize{(0.671)}} & \mc{1}{c}{\scriptsize{(0.303)}} \\  

    \mc{1}{l}{\scriptsize{Global Severity Index}} & \mc{1}{c}{\scriptsize{21}} & \mc{1}{c}{\scriptsize{-0.235}} & \mc{1}{c}{\scriptsize{-0.305}} & \mc{1}{c}{\scriptsize{-0.413}} & \mc{1}{c}{\scriptsize{-0.639}} & \mc{1}{c}{\scriptsize{-0.418}} & \mc{1}{c}{\scriptsize{-0.151}} & \mc{1}{c}{\scriptsize{-0.178}} & \mc{1}{c}{\scriptsize{-0.141}} \\  

     &  & \mc{1}{c}{\scriptsize{\textbf{(0.026)}}} & \mc{1}{c}{\scriptsize{\textbf{(0.039)}}} & \mc{1}{c}{\scriptsize{\textbf{(0.000)}}} & \mc{1}{c}{\scriptsize{\textbf{(0.000)}}} & \mc{1}{c}{\scriptsize{\textbf{(0.000)}}} & \mc{1}{c}{\scriptsize{(0.132)}} & \mc{1}{c}{\scriptsize{(0.132)}} & \mc{1}{c}{\scriptsize{(0.118)}} \\  

     & \mc{1}{c}{\scriptsize{34}} & \mc{1}{c}{\scriptsize{-1.304}} & \mc{1}{c}{\scriptsize{-2.369}} & \mc{1}{c}{\scriptsize{-1.804}} & \mc{1}{c}{\scriptsize{-4.207}} & \mc{1}{c}{\scriptsize{-5.035}} & \mc{1}{c}{\scriptsize{-1.126}} & \mc{1}{c}{\scriptsize{-1.455}} & \mc{1}{c}{\scriptsize{-4.084}} \\  

     &  & \mc{1}{c}{\scriptsize{(0.408)}} & \mc{1}{c}{\scriptsize{(0.224)}} & \mc{1}{c}{\scriptsize{(0.303)}} & \mc{1}{c}{\scriptsize{(0.184)}} & \mc{1}{c}{\scriptsize{(0.158)}} & \mc{1}{c}{\scriptsize{(0.421)}} & \mc{1}{c}{\scriptsize{(0.303)}} & \mc{1}{c}{\scriptsize{(0.132)}} \\  

    \mc{1}{l}{\scriptsize{BSI Factor}} & \mc{1}{c}{\scriptsize{21 and 34}} & \mc{1}{c}{\scriptsize{-0.460}} & \mc{1}{c}{\scriptsize{-0.374}} & \mc{1}{c}{\scriptsize{-0.578}} & \mc{1}{c}{\scriptsize{-0.778}} & \mc{1}{c}{\scriptsize{-0.708}} & \mc{1}{c}{\scriptsize{-0.417}} & \mc{1}{c}{\scriptsize{-0.211}} & \mc{1}{c}{\scriptsize{-0.460}} \\  

     &  & \mc{1}{c}{\scriptsize{\textbf{(0.039)}}} & \mc{1}{c}{\scriptsize{(0.197)}} & \mc{1}{c}{\scriptsize{(0.105)}} & \mc{1}{c}{\scriptsize{(0.118)}} & \mc{1}{c}{\scriptsize{\textbf{(0.079)}}} & \mc{1}{c}{\scriptsize{\textbf{(0.066)}}} & \mc{1}{c}{\scriptsize{(0.303)}} & \mc{1}{c}{\scriptsize{\textbf{(0.079)}}} \\ 
    \midrule  

    \mc{2}{l}{\scriptsize{\% of Pos. TE ($H_0$: $\le$ 50\%)}} & \mc{1}{c}{\scriptsize{91}} & \mc{1}{c}{\scriptsize{100}} & \mc{1}{c}{\scriptsize{100}} & \mc{1}{c}{\scriptsize{100}} & \mc{1}{c}{\scriptsize{100}} & \mc{1}{c}{\scriptsize{82}} & \mc{1}{c}{\scriptsize{82}} & \mc{1}{c}{\scriptsize{100}} \\  

     &  & \mc{1}{c}{\scriptsize{\textbf{(0.000)}}} & \mc{1}{c}{\scriptsize{\textbf{(0.000)}}} & \mc{1}{c}{\scriptsize{\textbf{(0.000)}}} & \mc{1}{c}{\scriptsize{\textbf{(0.000)}}} & \mc{1}{c}{\scriptsize{\textbf{(0.000)}}} & \mc{1}{c}{\scriptsize{\textbf{(0.000)}}} & \mc{1}{c}{\scriptsize{(0.132)}} & \mc{1}{c}{\scriptsize{\textbf{(0.000)}}} \\  

    \mc{2}{l}{\scriptsize{\% of Pos. TE ($H_0$: $\le$ 10\% $|$ 10\% Significance)}} & \mc{1}{c}{\scriptsize{45}} & \mc{1}{c}{\scriptsize{45}} & \mc{1}{c}{\scriptsize{36}} & \mc{1}{c}{\scriptsize{36}} & \mc{1}{c}{\scriptsize{45}} & \mc{1}{c}{\scriptsize{18}} & \mc{1}{c}{\scriptsize{9}} & \mc{1}{c}{\scriptsize{27}} \\  

     &  & \mc{1}{c}{\scriptsize{\textbf{(0.079)}}} & \mc{1}{c}{\scriptsize{\textbf{(0.026)}}} & \mc{1}{c}{\scriptsize{\textbf{(0.026)}}} & \mc{1}{c}{\scriptsize{\textbf{(0.013)}}} & \mc{1}{c}{\scriptsize{\textbf{(0.053)}}} & \mc{1}{c}{\scriptsize{(0.263)}} & \mc{1}{c}{\scriptsize{(0.487)}} & \mc{1}{c}{\scriptsize{(0.303)}} \\  

  \bottomrule
  \end{tabular}
	\end{table} 

	\begin{table}[H]
     \caption{Treatment Effects on Father at Home, Female Sample}
     \label{table:abccare_rslt_female_cat17}
	\begin{table}[H]
\captionsetup{singlelinecheck=false,justification=centering}
\caption{ABC Average Treatment Effects, Females \\ Obesity \label{tab:ate_female_apx17}}

  \begin{threeparttable}
  \begin{tabular}{cccccccccc}
  \hline\hline

     &  & \scriptsize{(1)} & \scriptsize{(2)} & \scriptsize{(3)} & \scriptsize{(4)} & \scriptsize{(5)} & \scriptsize{(6)} & \scriptsize{(7)} & \scriptsize{(8)} \\  

     &  &  &  & \mc{3}{c}{\scriptsize{$P=0$}} & \mc{3}{c}{\scriptsize{$P=1$}} \\ 
    \cmidrule(lr){5-7} \cmidrule(lr){8-10} 

    \scriptsize{Variable} & \scriptsize{Age} & \scriptsize{ITT} & \scriptsize{ITT$|X,W$} & \scriptsize{ITT} & \scriptsize{ITT$|X,W$} & \scriptsize{KE$|X,W$} & \scriptsize{ITT} & \scriptsize{ITT$|X,W$} & \scriptsize{KE$|X,W$} \\ 
    \hline  

    \mc{1}{l}{\scriptsize{Measured BMI}} & \mc{1}{c}{\scriptsize{Mid-30s}} & \mc{1}{c}{\scriptsize{1.785}} & \mc{1}{c}{\scriptsize{5.519}} & \mc{1}{c}{\scriptsize{1.479}} & \mc{1}{c}{\scriptsize{-6.567}} & \mc{1}{c}{\scriptsize{0.725}} & \mc{1}{c}{\scriptsize{1.853}} & \mc{1}{c}{\scriptsize{9.134}} & \mc{1}{c}{\scriptsize{3.874}} \\  

     &  & \mc{1}{c}{\scriptsize{(0.745)}} & \mc{1}{c}{\scriptsize{(0.902)}} & \mc{1}{c}{\scriptsize{(0.608)}} & \mc{1}{c}{\scriptsize{(0.235)}} & \mc{1}{c}{\scriptsize{(0.471)}} & \mc{1}{c}{\scriptsize{(0.706)}} & \mc{1}{c}{\scriptsize{(1.000)}} & \mc{1}{c}{\scriptsize{(0.784)}} \\  

    \mc{1}{l}{\scriptsize{Obesity}} & \mc{1}{c}{\scriptsize{Mid-30s}} & \mc{1}{c}{\scriptsize{-0.061}} & \mc{1}{c}{\scriptsize{0.183}} & \mc{1}{c}{\scriptsize{-0.083}} & \mc{1}{c}{\scriptsize{-0.068}} & \mc{1}{c}{\scriptsize{-0.141}} & \mc{1}{c}{\scriptsize{-0.056}} & \mc{1}{c}{\scriptsize{0.293}} & \mc{1}{c}{\scriptsize{0.065}} \\  

     &  & \mc{1}{c}{\scriptsize{(0.235)}} & \mc{1}{c}{\scriptsize{(0.863)}} & \mc{1}{c}{\scriptsize{(0.294)}} & \mc{1}{c}{\scriptsize{(0.373)}} & \mc{1}{c}{\scriptsize{(0.118)}} & \mc{1}{c}{\scriptsize{(0.275)}} & \mc{1}{c}{\scriptsize{(0.863)}} & \mc{1}{c}{\scriptsize{(0.647)}} \\  

    \mc{1}{l}{\scriptsize{Severe Obesity}} & \mc{1}{c}{\scriptsize{Mid-30s}} & \mc{1}{c}{\scriptsize{-0.141}} & \mc{1}{c}{\scriptsize{0.011}} & \mc{1}{c}{\scriptsize{-0.028}} & \mc{1}{c}{\scriptsize{-0.411}} & \mc{1}{c}{\scriptsize{-0.054}} & \mc{1}{c}{\scriptsize{-0.167}} & \mc{1}{c}{\scriptsize{0.138}} & \mc{1}{c}{\scriptsize{-0.035}} \\  

     &  & \mc{1}{c}{\scriptsize{(0.157)}} & \mc{1}{c}{\scriptsize{(0.431)}} & \mc{1}{c}{\scriptsize{(0.373)}} & \mc{1}{c}{\scriptsize{(0.176)}} & \mc{1}{c}{\scriptsize{(0.353)}} & \mc{1}{c}{\scriptsize{\textbf{(0.059)}}} & \mc{1}{c}{\scriptsize{(0.725)}} & \mc{1}{c}{\scriptsize{(0.412)}} \\  

    \mc{1}{l}{\scriptsize{Waist-hip Ratio}} & \mc{1}{c}{\scriptsize{Mid-30s}} & \mc{1}{c}{\scriptsize{-0.057}} & \mc{1}{c}{\scriptsize{-0.059}} & \mc{1}{c}{\scriptsize{-0.137}} & \mc{1}{c}{\scriptsize{-0.193}} & \mc{1}{c}{\scriptsize{-0.137}} & \mc{1}{c}{\scriptsize{-0.037}} & \mc{1}{c}{\scriptsize{-0.029}} & \mc{1}{c}{\scriptsize{-0.021}} \\  

     &  & \mc{1}{c}{\scriptsize{\textbf{(0.039)}}} & \mc{1}{c}{\scriptsize{(0.137)}} & \mc{1}{c}{\scriptsize{\textbf{(0.000)}}} & \mc{1}{c}{\scriptsize{\textbf{(0.020)}}} & \mc{1}{c}{\scriptsize{\textbf{(0.039)}}} & \mc{1}{c}{\scriptsize{(0.118)}} & \mc{1}{c}{\scriptsize{(0.353)}} & \mc{1}{c}{\scriptsize{(0.255)}} \\  

    \mc{1}{l}{\scriptsize{Abdominal Obesity}} & \mc{1}{c}{\scriptsize{Mid-30s}} & \mc{1}{c}{\scriptsize{-0.199}} & \mc{1}{c}{\scriptsize{-0.150}} & \mc{1}{c}{\scriptsize{-0.438}} & \mc{1}{c}{\scriptsize{-0.198}} & \mc{1}{c}{\scriptsize{-0.386}} & \mc{1}{c}{\scriptsize{-0.143}} & \mc{1}{c}{\scriptsize{-0.124}} & \mc{1}{c}{\scriptsize{-0.086}} \\  

     &  & \mc{1}{c}{\scriptsize{(0.118)}} & \mc{1}{c}{\scriptsize{(0.294)}} & \mc{1}{c}{\scriptsize{\textbf{(0.000)}}} & \mc{1}{c}{\scriptsize{(0.333)}} & \mc{1}{c}{\scriptsize{\textbf{(0.000)}}} & \mc{1}{c}{\scriptsize{(0.157)}} & \mc{1}{c}{\scriptsize{(0.353)}} & \mc{1}{c}{\scriptsize{(0.314)}} \\  

    \mc{1}{l}{\scriptsize{Framingham Risk Score}} & \mc{1}{c}{\scriptsize{Mid-30s}} & \mc{1}{c}{\scriptsize{-0.471}} & \mc{1}{c}{\scriptsize{-0.729}} & \mc{1}{c}{\scriptsize{-1.005}} & \mc{1}{c}{\scriptsize{-1.527}} & \mc{1}{c}{\scriptsize{-1.263}} & \mc{1}{c}{\scriptsize{-0.353}} & \mc{1}{c}{\scriptsize{-0.481}} & \mc{1}{c}{\scriptsize{-0.387}} \\  

     &  & \mc{1}{c}{\scriptsize{\textbf{(0.059)}}} & \mc{1}{c}{\scriptsize{(0.137)}} & \mc{1}{c}{\scriptsize{\textbf{(0.000)}}} & \mc{1}{c}{\scriptsize{\textbf{(0.020)}}} & \mc{1}{c}{\scriptsize{\textbf{(0.000)}}} & \mc{1}{c}{\scriptsize{(0.137)}} & \mc{1}{c}{\scriptsize{(0.216)}} & \mc{1}{c}{\scriptsize{(0.118)}} \\  

    \mc{1}{l}{\scriptsize{Obesity Factor}} & \mc{1}{c}{\scriptsize{Mid-30s}} & \mc{1}{c}{\scriptsize{-0.185}} & \mc{1}{c}{\scriptsize{-0.022}} & \mc{1}{c}{\scriptsize{-0.495}} & \mc{1}{c}{\scriptsize{-1.355}} & \mc{1}{c}{\scriptsize{-0.567}} & \mc{1}{c}{\scriptsize{-0.112}} & \mc{1}{c}{\scriptsize{0.364}} & \mc{1}{c}{\scriptsize{0.119}} \\  

     &  & \mc{1}{c}{\scriptsize{(0.294)}} & \mc{1}{c}{\scriptsize{(0.373)}} & \mc{1}{c}{\scriptsize{(0.157)}} & \mc{1}{c}{\scriptsize{\textbf{(0.098)}}} & \mc{1}{c}{\scriptsize{\textbf{(0.078)}}} & \mc{1}{c}{\scriptsize{(0.392)}} & \mc{1}{c}{\scriptsize{(0.784)}} & \mc{1}{c}{\scriptsize{(0.569)}} \\ 
    \hline  

    \\[0.1cm]
    \mc{2}{l}{\scriptsize{\% of Pos. TE ($H_0$: $\le$ 25\% $|$ 10\% Significance)}} & \mc{1}{c}{\scriptsize{29}} & \mc{1}{c}{\scriptsize{0}} & \mc{1}{c}{\scriptsize{43}} & \mc{1}{c}{\scriptsize{43}} & \mc{1}{c}{\scriptsize{57}} & \mc{1}{c}{\scriptsize{14}} & \mc{1}{c}{\scriptsize{0}} & \mc{1}{c}{\scriptsize{0}} \\  

     &  & \mc{1}{c}{\scriptsize{(0.373)}} & \mc{1}{c}{\scriptsize{(0.765)}} & \mc{1}{c}{\scriptsize{(0.157)}} & \mc{1}{c}{\scriptsize{(0.216)}} & \mc{1}{c}{\scriptsize{\textbf{(0.098)}}} & \mc{1}{c}{\scriptsize{(0.569)}} & \mc{1}{c}{\scriptsize{(1.000)}} & \mc{1}{c}{\scriptsize{(0.980)}} \\  

    \mc{2}{l}{\scriptsize{\% of Pos. TE ($H_0$: $\le$ 50\% $|$ 10\% Significance)}} & \mc{1}{c}{\scriptsize{29}} & \mc{1}{c}{\scriptsize{0}} & \mc{1}{c}{\scriptsize{43}} & \mc{1}{c}{\scriptsize{43}} & \mc{1}{c}{\scriptsize{57}} & \mc{1}{c}{\scriptsize{14}} & \mc{1}{c}{\scriptsize{0}} & \mc{1}{c}{\scriptsize{0}} \\  

     &  & \mc{1}{c}{\scriptsize{(0.745)}} & \mc{1}{c}{\scriptsize{(1.000)}} & \mc{1}{c}{\scriptsize{(0.510)}} & \mc{1}{c}{\scriptsize{(0.412)}} & \mc{1}{c}{\scriptsize{(0.294)}} & \mc{1}{c}{\scriptsize{(1.000)}} & \mc{1}{c}{\scriptsize{(1.000)}} & \mc{1}{c}{\scriptsize{(0.980)}} \\  

    \mc{2}{l}{\scriptsize{\% of Pos. TE ($H_0$: $\le$ 75\% $|$ 10\% Significance)}} & \mc{1}{c}{\scriptsize{29}} & \mc{1}{c}{\scriptsize{0}} & \mc{1}{c}{\scriptsize{43}} & \mc{1}{c}{\scriptsize{43}} & \mc{1}{c}{\scriptsize{57}} & \mc{1}{c}{\scriptsize{14}} & \mc{1}{c}{\scriptsize{0}} & \mc{1}{c}{\scriptsize{0}} \\  

     &  & \mc{1}{c}{\scriptsize{(1.000)}} & \mc{1}{c}{\scriptsize{(1.000)}} & \mc{1}{c}{\scriptsize{(0.882)}} & \mc{1}{c}{\scriptsize{(0.725)}} & \mc{1}{c}{\scriptsize{(0.647)}} & \mc{1}{c}{\scriptsize{(1.000)}} & \mc{1}{c}{\scriptsize{(1.000)}} & \mc{1}{c}{\scriptsize{(0.980)}} \\  

  \hline\hline
  \end{tabular}
    \begin{tablenotes}
    \scriptsize
    \item 
Note: This table displays various estimates of the treatment effect of ABC's center-based care.
Column (1) displays the ITT, without accounting for any controls.
Column (2) displays the ITT conditioning on vector of controls, $X$, consisting of the Apgar score 1 minute after birth, the HRI index, maternal IQ, an
indicator for teenage pregnancy of the mother, an indicator for the father being at 
home, and an indicator for having a grandmother residing in the same county. We also apply IPW weights, $W$, to account for attrition.
Columns (3)--(4) are analogous to columns (1)--(2), but we restrict the control sample to subjects
who did not enroll in any alternative care.
Column (5) displys the matching estimate, where we use the Mahalanobis metric and Epanechnikov kernel
to match on controls $X$ listed above, and restrict the control sample to subjects who did not enroll
in any alternative care. Additionally, we apply IPW weights, $W$.
Columns (6)--(8) are analogous to Columns (3)--(5), except we restrict the control sample to subejcts
who did enroll in alternative care. The final three pairs of rows display the proportion of treatment effects in the table that are 
socially positive. The first row in each pair displays the percentage of treatment effects, and the
second row presents the inference. 
Numbers in parentheses represent the $p$-value from a single hypothesis test, and are obtained from 
the empirical bootstrap distribution generated by 200 resamples of the original data. 
Bold $p$-values indicate significance at the 10\% level.
Blank point estimates indicate that we are unable to obtain estimates due to a lack of support in the data. 

    \end{tablenotes}
  \end{threeparttable}

\end{table}
	\end{table} 

	\begin{table}[H]
     \caption{Treatment Effects on HOME Scores, Female Sample}
     \label{table:abccare_rslt_female_cat18}
	\begin{table}[H]
\captionsetup{singlelinecheck=false,justification=centering}
\caption{ABC Average Treatment Effects, Females \\ Mental Health \label{tab:ate_female_apx18}}

  \begin{threeparttable}
  \begin{tabular}{cccccccccc}
  \hline\hline

     &  & \scriptsize{(1)} & \scriptsize{(2)} & \scriptsize{(3)} & \scriptsize{(4)} & \scriptsize{(5)} & \scriptsize{(6)} & \scriptsize{(7)} & \scriptsize{(8)} \\  

     &  &  &  & \mc{3}{c}{\scriptsize{$P=0$}} & \mc{3}{c}{\scriptsize{$P=1$}} \\ 
    \cmidrule(lr){5-7} \cmidrule(lr){8-10} 

    \scriptsize{Variable} & \scriptsize{Age} & \scriptsize{ITT} & \scriptsize{ITT$|X,W$} & \scriptsize{ITT} & \scriptsize{ITT$|X,W$} & \scriptsize{KE$|X,W$} & \scriptsize{ITT} & \scriptsize{ITT$|X,W$} & \scriptsize{KE$|X,W$} \\ 
    \hline  

    \mc{1}{l}{\scriptsize{Somatization}} & \mc{1}{c}{\scriptsize{21}} & \mc{1}{c}{\scriptsize{-0.054}} & \mc{1}{c}{\scriptsize{-0.211}} & \mc{1}{c}{\scriptsize{-0.320}} & \mc{1}{c}{\scriptsize{-0.925}} & \mc{1}{c}{\scriptsize{-0.624}} & \mc{1}{c}{\scriptsize{0.034}} & \mc{1}{c}{\scriptsize{-0.054}} & \mc{1}{c}{\scriptsize{-0.046}} \\  

     &  & \mc{1}{c}{\scriptsize{(0.353)}} & \mc{1}{c}{\scriptsize{(0.196)}} & \mc{1}{c}{\scriptsize{(0.157)}} & \mc{1}{c}{\scriptsize{\textbf{(0.078)}}} & \mc{1}{c}{\scriptsize{(0.118)}} & \mc{1}{c}{\scriptsize{(0.549)}} & \mc{1}{c}{\scriptsize{(0.431)}} & \mc{1}{c}{\scriptsize{(0.471)}} \\  

     & \mc{1}{c}{\scriptsize{34}} & \mc{1}{c}{\scriptsize{-0.013}} & \mc{1}{c}{\scriptsize{-0.159}} & \mc{1}{c}{\scriptsize{-0.736}} & \mc{1}{c}{\scriptsize{-1.110}} & \mc{1}{c}{\scriptsize{-0.840}} & \mc{1}{c}{\scriptsize{0.148}} & \mc{1}{c}{\scriptsize{0.221}} & \mc{1}{c}{\scriptsize{-0.103}} \\  

     &  & \mc{1}{c}{\scriptsize{(0.510)}} & \mc{1}{c}{\scriptsize{(0.275)}} & \mc{1}{c}{\scriptsize{\textbf{(0.098)}}} & \mc{1}{c}{\scriptsize{(0.196)}} & \mc{1}{c}{\scriptsize{\textbf{(0.078)}}} & \mc{1}{c}{\scriptsize{(0.686)}} & \mc{1}{c}{\scriptsize{(0.784)}} & \mc{1}{c}{\scriptsize{(0.294)}} \\  

    \mc{1}{l}{\scriptsize{Depression}} & \mc{1}{c}{\scriptsize{21}} & \mc{1}{c}{\scriptsize{-0.460}} & \mc{1}{c}{\scriptsize{-0.459}} & \mc{1}{c}{\scriptsize{-0.841}} & \mc{1}{c}{\scriptsize{-0.840}} & \mc{1}{c}{\scriptsize{-0.922}} & \mc{1}{c}{\scriptsize{-0.333}} & \mc{1}{c}{\scriptsize{-0.337}} & \mc{1}{c}{\scriptsize{-0.261}} \\  

     &  & \mc{1}{c}{\scriptsize{\textbf{(0.000)}}} & \mc{1}{c}{\scriptsize{\textbf{(0.078)}}} & \mc{1}{c}{\scriptsize{\textbf{(0.020)}}} & \mc{1}{c}{\scriptsize{\textbf{(0.059)}}} & \mc{1}{c}{\scriptsize{\textbf{(0.000)}}} & \mc{1}{c}{\scriptsize{\textbf{(0.098)}}} & \mc{1}{c}{\scriptsize{(0.137)}} & \mc{1}{c}{\scriptsize{(0.176)}} \\  

     & \mc{1}{c}{\scriptsize{34}} & \mc{1}{c}{\scriptsize{-0.048}} & \mc{1}{c}{\scriptsize{-0.191}} & \mc{1}{c}{\scriptsize{-0.431}} & \mc{1}{c}{\scriptsize{-0.960}} & \mc{1}{c}{\scriptsize{-0.623}} & \mc{1}{c}{\scriptsize{0.037}} & \mc{1}{c}{\scriptsize{0.076}} & \mc{1}{c}{\scriptsize{-0.257}} \\  

     &  & \mc{1}{c}{\scriptsize{(0.471)}} & \mc{1}{c}{\scriptsize{(0.294)}} & \mc{1}{c}{\scriptsize{(0.255)}} & \mc{1}{c}{\scriptsize{(0.157)}} & \mc{1}{c}{\scriptsize{(0.196)}} & \mc{1}{c}{\scriptsize{(0.608)}} & \mc{1}{c}{\scriptsize{(0.627)}} & \mc{1}{c}{\scriptsize{(0.176)}} \\  

    \mc{1}{l}{\scriptsize{Anxiety}} & \mc{1}{c}{\scriptsize{21}} & \mc{1}{c}{\scriptsize{-0.371}} & \mc{1}{c}{\scriptsize{-0.241}} & \mc{1}{c}{\scriptsize{-0.734}} & \mc{1}{c}{\scriptsize{-0.791}} & \mc{1}{c}{\scriptsize{-0.642}} & \mc{1}{c}{\scriptsize{-0.250}} & \mc{1}{c}{\scriptsize{-0.038}} & \mc{1}{c}{\scriptsize{-0.131}} \\  

     &  & \mc{1}{c}{\scriptsize{\textbf{(0.059)}}} & \mc{1}{c}{\scriptsize{(0.157)}} & \mc{1}{c}{\scriptsize{\textbf{(0.020)}}} & \mc{1}{c}{\scriptsize{\textbf{(0.059)}}} & \mc{1}{c}{\scriptsize{\textbf{(0.000)}}} & \mc{1}{c}{\scriptsize{\textbf{(0.098)}}} & \mc{1}{c}{\scriptsize{(0.392)}} & \mc{1}{c}{\scriptsize{(0.216)}} \\  

     & \mc{1}{c}{\scriptsize{34}} & \mc{1}{c}{\scriptsize{-0.221}} & \mc{1}{c}{\scriptsize{-0.370}} & \mc{1}{c}{\scriptsize{-0.671}} & \mc{1}{c}{\scriptsize{-1.125}} & \mc{1}{c}{\scriptsize{-0.901}} & \mc{1}{c}{\scriptsize{-0.120}} & \mc{1}{c}{\scriptsize{-0.033}} & \mc{1}{c}{\scriptsize{-0.363}} \\  

     &  & \mc{1}{c}{\scriptsize{(0.196)}} & \mc{1}{c}{\scriptsize{(0.118)}} & \mc{1}{c}{\scriptsize{\textbf{(0.098)}}} & \mc{1}{c}{\scriptsize{(0.137)}} & \mc{1}{c}{\scriptsize{\textbf{(0.059)}}} & \mc{1}{c}{\scriptsize{(0.294)}} & \mc{1}{c}{\scriptsize{(0.529)}} & \mc{1}{c}{\scriptsize{(0.118)}} \\  

    \mc{1}{l}{\scriptsize{Hostility}} & \mc{1}{c}{\scriptsize{21}} & \mc{1}{c}{\scriptsize{-0.437}} & \mc{1}{c}{\scriptsize{-0.409}} & \mc{1}{c}{\scriptsize{-1.108}} & \mc{1}{c}{\scriptsize{-1.283}} & \mc{1}{c}{\scriptsize{-1.152}} & \mc{1}{c}{\scriptsize{-0.213}} & \mc{1}{c}{\scriptsize{-0.238}} & \mc{1}{c}{\scriptsize{-0.223}} \\  

     &  & \mc{1}{c}{\scriptsize{\textbf{(0.039)}}} & \mc{1}{c}{\scriptsize{\textbf{(0.078)}}} & \mc{1}{c}{\scriptsize{\textbf{(0.020)}}} & \mc{1}{c}{\scriptsize{\textbf{(0.059)}}} & \mc{1}{c}{\scriptsize{\textbf{(0.000)}}} & \mc{1}{c}{\scriptsize{(0.235)}} & \mc{1}{c}{\scriptsize{(0.216)}} & \mc{1}{c}{\scriptsize{(0.196)}} \\  

     & \mc{1}{c}{\scriptsize{34}} & \mc{1}{c}{\scriptsize{-0.118}} & \mc{1}{c}{\scriptsize{-0.115}} & \mc{1}{c}{\scriptsize{-0.400}} & \mc{1}{c}{\scriptsize{-0.886}} & \mc{1}{c}{\scriptsize{-0.635}} & \mc{1}{c}{\scriptsize{-0.056}} & \mc{1}{c}{\scriptsize{0.197}} & \mc{1}{c}{\scriptsize{-0.315}} \\  

     &  & \mc{1}{c}{\scriptsize{(0.275)}} & \mc{1}{c}{\scriptsize{(0.392)}} & \mc{1}{c}{\scriptsize{\textbf{(0.098)}}} & \mc{1}{c}{\scriptsize{\textbf{(0.020)}}} & \mc{1}{c}{\scriptsize{\textbf{(0.039)}}} & \mc{1}{c}{\scriptsize{(0.353)}} & \mc{1}{c}{\scriptsize{(0.784)}} & \mc{1}{c}{\scriptsize{(0.157)}} \\  

    \mc{1}{l}{\scriptsize{Global Severity Index}} & \mc{1}{c}{\scriptsize{21}} & \mc{1}{c}{\scriptsize{-0.331}} & \mc{1}{c}{\scriptsize{-0.324}} & \mc{1}{c}{\scriptsize{-0.653}} & \mc{1}{c}{\scriptsize{-0.804}} & \mc{1}{c}{\scriptsize{-0.724}} & \mc{1}{c}{\scriptsize{-0.224}} & \mc{1}{c}{\scriptsize{-0.159}} & \mc{1}{c}{\scriptsize{-0.165}} \\  

     &  & \mc{1}{c}{\scriptsize{\textbf{(0.000)}}} & \mc{1}{c}{\scriptsize{\textbf{(0.078)}}} & \mc{1}{c}{\scriptsize{\textbf{(0.020)}}} & \mc{1}{c}{\scriptsize{\textbf{(0.020)}}} & \mc{1}{c}{\scriptsize{\textbf{(0.000)}}} & \mc{1}{c}{\scriptsize{(0.118)}} & \mc{1}{c}{\scriptsize{(0.255)}} & \mc{1}{c}{\scriptsize{(0.196)}} \\  

     & \mc{1}{c}{\scriptsize{34}} & \mc{1}{c}{\scriptsize{-1.687}} & \mc{1}{c}{\scriptsize{-4.320}} & \mc{1}{c}{\scriptsize{-11.028}} & \mc{1}{c}{\scriptsize{-19.171}} & \mc{1}{c}{\scriptsize{-14.180}} & \mc{1}{c}{\scriptsize{0.389}} & \mc{1}{c}{\scriptsize{1.583}} & \mc{1}{c}{\scriptsize{-4.342}} \\  

     &  & \mc{1}{c}{\scriptsize{(0.353)}} & \mc{1}{c}{\scriptsize{(0.235)}} & \mc{1}{c}{\scriptsize{(0.137)}} & \mc{1}{c}{\scriptsize{(0.157)}} & \mc{1}{c}{\scriptsize{\textbf{(0.059)}}} & \mc{1}{c}{\scriptsize{(0.529)}} & \mc{1}{c}{\scriptsize{(0.686)}} & \mc{1}{c}{\scriptsize{(0.157)}} \\  

    \mc{1}{l}{\scriptsize{BSI Factor}} & \mc{1}{c}{\scriptsize{21 and 34}} & \mc{1}{c}{\scriptsize{-0.564}} & \mc{1}{c}{\scriptsize{-0.554}} & \mc{1}{c}{\scriptsize{-1.377}} & \mc{1}{c}{\scriptsize{-2.119}} & \mc{1}{c}{\scriptsize{-1.406}} & \mc{1}{c}{\scriptsize{-0.383}} & \mc{1}{c}{\scriptsize{-0.065}} & \mc{1}{c}{\scriptsize{-0.475}} \\  

     &  & \mc{1}{c}{\scriptsize{\textbf{(0.039)}}} & \mc{1}{c}{\scriptsize{\textbf{(0.078)}}} & \mc{1}{c}{\scriptsize{\textbf{(0.039)}}} & \mc{1}{c}{\scriptsize{\textbf{(0.020)}}} & \mc{1}{c}{\scriptsize{\textbf{(0.039)}}} & \mc{1}{c}{\scriptsize{(0.118)}} & \mc{1}{c}{\scriptsize{(0.490)}} & \mc{1}{c}{\scriptsize{\textbf{(0.059)}}} \\ 
    \hline  

    \\[0.1cm]
    \mc{2}{l}{\scriptsize{\% of Pos. TE ($H_0$: $\le$ 25\% $|$ 10\% Significance)}} & \mc{1}{c}{\scriptsize{45}} & \mc{1}{c}{\scriptsize{36}} & \mc{1}{c}{\scriptsize{73}} & \mc{1}{c}{\scriptsize{64}} & \mc{1}{c}{\scriptsize{82}} & \mc{1}{c}{\scriptsize{18}} & \mc{1}{c}{\scriptsize{0}} & \mc{1}{c}{\scriptsize{9}} \\  

     &  & \mc{1}{c}{\scriptsize{(0.196)}} & \mc{1}{c}{\scriptsize{(0.235)}} & \mc{1}{c}{\scriptsize{\textbf{(0.000)}}} & \mc{1}{c}{\scriptsize{(0.196)}} & \mc{1}{c}{\scriptsize{\textbf{(0.000)}}} & \mc{1}{c}{\scriptsize{(0.529)}} & \mc{1}{c}{\scriptsize{(1.000)}} & \mc{1}{c}{\scriptsize{(0.588)}} \\  

    \mc{2}{l}{\scriptsize{\% of Pos. TE ($H_0$: $\le$ 50\% $|$ 10\% Significance)}} & \mc{1}{c}{\scriptsize{45}} & \mc{1}{c}{\scriptsize{36}} & \mc{1}{c}{\scriptsize{73}} & \mc{1}{c}{\scriptsize{64}} & \mc{1}{c}{\scriptsize{82}} & \mc{1}{c}{\scriptsize{18}} & \mc{1}{c}{\scriptsize{0}} & \mc{1}{c}{\scriptsize{9}} \\  

     &  & \mc{1}{c}{\scriptsize{(0.510)}} & \mc{1}{c}{\scriptsize{(0.686)}} & \mc{1}{c}{\scriptsize{(0.235)}} & \mc{1}{c}{\scriptsize{(0.294)}} & \mc{1}{c}{\scriptsize{\textbf{(0.000)}}} & \mc{1}{c}{\scriptsize{(1.000)}} & \mc{1}{c}{\scriptsize{(1.000)}} & \mc{1}{c}{\scriptsize{(1.000)}} \\  

    \mc{2}{l}{\scriptsize{\% of Pos. TE ($H_0$: $\le$ 75\% $|$ 10\% Significance)}} & \mc{1}{c}{\scriptsize{45}} & \mc{1}{c}{\scriptsize{36}} & \mc{1}{c}{\scriptsize{73}} & \mc{1}{c}{\scriptsize{64}} & \mc{1}{c}{\scriptsize{82}} & \mc{1}{c}{\scriptsize{18}} & \mc{1}{c}{\scriptsize{0}} & \mc{1}{c}{\scriptsize{9}} \\  

     &  & \mc{1}{c}{\scriptsize{(0.863)}} & \mc{1}{c}{\scriptsize{(1.000)}} & \mc{1}{c}{\scriptsize{(0.471)}} & \mc{1}{c}{\scriptsize{(0.627)}} & \mc{1}{c}{\scriptsize{(0.471)}} & \mc{1}{c}{\scriptsize{(1.000)}} & \mc{1}{c}{\scriptsize{(1.000)}} & \mc{1}{c}{\scriptsize{(1.000)}} \\  

  \hline\hline
  \end{tabular}
    \begin{tablenotes}
    \scriptsize
    \item 
Note: This table displays various estimates of the treatment effect of ABC's center-based care.
Column (1) displays the ITT, without accounting for any controls.
Column (2) displays the ITT conditioning on vector of controls, $X$, consisting of APGAR scores 1 
minute after birth, an indicator for the subject being born prematurely, and an indicator for the 
father being home at baseline. We also apply IPW weights, $W$, to account for attrition.
Columns (3)--(4) are analogous to columns (1)--(2), but we restrict the control sample to subjects
who did not enroll in any alternative care.
Column (5) displys the matching estimate, where we use the Mahalanobis metric and Epanechnikov kernel
to match on controls $X$ listed above, and restrict the control sample to subjects who did not enroll
in any alternative care. Additionally, we apply IPW weights, $W$.
Columns (6)--(8) are analogous to Columns (3)--(5), except we restrict the control sample to subejcts
who did enroll in alternative care. 
The final three pairs of rows display the proportion of treatment effects in the table that are 
socially positive. The first row in each pair displays the percentage of treatment effects, and the
second row presents the inference.

Numbers in parentheses represent the $p$-value from a single hypothesis test, and are obtained from 
the empirical bootstrap distribution generated by 200 resamples of the original data. 
Bold $p$-values indicate significance at the 10\% level.
Blank point estimates indicate that we are unable to obtain estimates due to a lack of support in the data. 

    \end{tablenotes}
  \end{threeparttable}

\end{table}
	\end{table} 

	\begin{table}[H]
     \caption{Treatment Effects on Relation with Spouse, Female Sample}
     \label{table:abccare_rslt_female_cat19}
	  \begin{tabular}{cccccccccc}
  \toprule

    \scriptsize{Variable} & \scriptsize{Age} & \scriptsize{(1)} & \scriptsize{(2)} & \scriptsize{(3)} & \scriptsize{(4)} & \scriptsize{(5)} & \scriptsize{(6)} & \scriptsize{(7)} & \scriptsize{(8)} \\ 
    \midrule  

    \mc{1}{l}{\scriptsize{No trouble with spouse family}} & \mc{1}{c}{\scriptsize{30}} & \mc{1}{c}{\scriptsize{0.150}} & \mc{1}{c}{\scriptsize{0.107}} & \mc{1}{c}{\scriptsize{0.233}} & \mc{1}{c}{\scriptsize{0.624}} & \mc{1}{c}{\scriptsize{0.206}} & \mc{1}{c}{\scriptsize{0.133}} & \mc{1}{c}{\scriptsize{0.040}} & \mc{1}{c}{\scriptsize{-0.050}} \\  

     &  & \mc{1}{c}{\scriptsize{(0.197)}} & \mc{1}{c}{\scriptsize{(0.303)}} & \mc{1}{c}{\scriptsize{(0.263)}} & \mc{1}{c}{\scriptsize{\textbf{(0.053)}}} & \mc{1}{c}{\scriptsize{(0.263)}} & \mc{1}{c}{\scriptsize{(0.197)}} & \mc{1}{c}{\scriptsize{(0.434)}} & \mc{1}{c}{\scriptsize{(0.566)}} \\  

    \mc{1}{l}{\scriptsize{Get along well with spouse}} & \mc{1}{c}{\scriptsize{30}} & \mc{1}{c}{\scriptsize{-0.050}} & \mc{1}{c}{\scriptsize{-0.080}} & \mc{1}{c}{\scriptsize{0.033}} & \mc{1}{c}{\scriptsize{0.097}} & \mc{1}{c}{\scriptsize{0.078}} & \mc{1}{c}{\scriptsize{-0.067}} & \mc{1}{c}{\scriptsize{-0.131}} & \mc{1}{c}{\scriptsize{-0.033}} \\  

     &  & \mc{1}{c}{\scriptsize{(0.618)}} & \mc{1}{c}{\scriptsize{(0.658)}} & \mc{1}{c}{\scriptsize{(0.434)}} & \mc{1}{c}{\scriptsize{(0.474)}} & \mc{1}{c}{\scriptsize{(0.329)}} & \mc{1}{c}{\scriptsize{(0.566)}} & \mc{1}{c}{\scriptsize{(0.671)}} & \mc{1}{c}{\scriptsize{(0.579)}} \\  

    \mc{1}{l}{\scriptsize{No disagreement on living arrangement}} & \mc{1}{c}{\scriptsize{30}} & \mc{1}{c}{\scriptsize{0.217}} & \mc{1}{c}{\scriptsize{-0.136}} & \mc{1}{c}{\scriptsize{0.300}} & \mc{1}{c}{\scriptsize{0.238}} & \mc{1}{c}{\scriptsize{0.288}} & \mc{1}{c}{\scriptsize{0.200}} & \mc{1}{c}{\scriptsize{-0.255}} & \mc{1}{c}{\scriptsize{0.004}} \\  

     &  & \mc{1}{c}{\scriptsize{(0.132)}} & \mc{1}{c}{\scriptsize{(0.645)}} & \mc{1}{c}{\scriptsize{(0.250)}} & \mc{1}{c}{\scriptsize{(0.539)}} & \mc{1}{c}{\scriptsize{(0.263)}} & \mc{1}{c}{\scriptsize{(0.145)}} & \mc{1}{c}{\scriptsize{(0.803)}} & \mc{1}{c}{\scriptsize{(0.447)}} \\  

  \bottomrule
  \end{tabular}
	\end{table} 

	\begin{table}[H]
     \caption{Treatment Effects on Spouse Characteristics, Female Sample}
     \label{table:abccare_rslt_female_cat20}
	\begin{table}[H]
\captionsetup{singlelinecheck=false,justification=centering}
\caption{ABC Average Treatment Effects, Females \\ Child Behavior \label{tab:ate_female_apx20}}

  \begin{threeparttable}
  \begin{tabular}{cccccccccc}
  \hline\hline

     &  & \scriptsize{(1)} & \scriptsize{(2)} & \scriptsize{(3)} & \scriptsize{(4)} & \scriptsize{(5)} & \scriptsize{(6)} & \scriptsize{(7)} & \scriptsize{(8)} \\  

     &  &  &  & \mc{3}{c}{\scriptsize{$P=0$}} & \mc{3}{c}{\scriptsize{$P=1$}} \\ 
    \cmidrule(lr){5-7} \cmidrule(lr){8-10} 

    \scriptsize{Variable} & \scriptsize{Age} & \scriptsize{ITT} & \scriptsize{ITT$|X,W$} & \scriptsize{ITT} & \scriptsize{ITT$|X,W$} & \scriptsize{KE$|X,W$} & \scriptsize{ITT} & \scriptsize{ITT$|X,W$} & \scriptsize{KE$|X,W$} \\ 
    \hline  

    \mc{1}{l}{\scriptsize{Participates in Activity}} & \mc{1}{c}{\scriptsize{12}} & \mc{1}{c}{\scriptsize{-0.112}} & \mc{1}{c}{\scriptsize{-0.030}} & \mc{1}{c}{\scriptsize{-0.162}} & \mc{1}{c}{\scriptsize{-0.033}} & \mc{1}{c}{\scriptsize{-0.100}} & \mc{1}{c}{\scriptsize{-0.061}} & \mc{1}{c}{\scriptsize{0.044}} & \mc{1}{c}{\scriptsize{-0.005}} \\  

     &  & \mc{1}{c}{\scriptsize{(0.765)}} & \mc{1}{c}{\scriptsize{(0.569)}} & \mc{1}{c}{\scriptsize{(0.804)}} & \mc{1}{c}{\scriptsize{(0.529)}} & \mc{1}{c}{\scriptsize{(0.667)}} & \mc{1}{c}{\scriptsize{(0.529)}} & \mc{1}{c}{\scriptsize{(0.294)}} & \mc{1}{c}{\scriptsize{(0.451)}} \\  

    \mc{1}{l}{\scriptsize{Time spent reading}} & \mc{1}{c}{\scriptsize{12}} & \mc{1}{c}{\scriptsize{2.512}} & \mc{1}{c}{\scriptsize{1.980}} & \mc{1}{c}{\scriptsize{2.959}} & \mc{1}{c}{\scriptsize{1.880}} & \mc{1}{c}{\scriptsize{3.996}} & \mc{1}{c}{\scriptsize{2.064}} & \mc{1}{c}{\scriptsize{1.174}} & \mc{1}{c}{\scriptsize{2.872}} \\  

     &  & \mc{1}{c}{\scriptsize{(0.176)}} & \mc{1}{c}{\scriptsize{(0.216)}} & \mc{1}{c}{\scriptsize{(0.137)}} & \mc{1}{c}{\scriptsize{(0.235)}} & \mc{1}{c}{\scriptsize{\textbf{(0.078)}}} & \mc{1}{c}{\scriptsize{(0.275)}} & \mc{1}{c}{\scriptsize{(0.333)}} & \mc{1}{c}{\scriptsize{(0.157)}} \\  

    \mc{1}{l}{\scriptsize{Good Description of Self}} & \mc{1}{c}{\scriptsize{12}} & \mc{1}{c}{\scriptsize{-0.146}} & \mc{1}{c}{\scriptsize{-0.140}} & \mc{1}{c}{\scriptsize{-0.146}} & \mc{1}{c}{\scriptsize{-0.019}} & \mc{1}{c}{\scriptsize{-0.133}} & \mc{1}{c}{\scriptsize{-0.146}} & \mc{1}{c}{\scriptsize{-0.226}} & \mc{1}{c}{\scriptsize{-0.195}} \\  

     &  & \mc{1}{c}{\scriptsize{(0.804)}} & \mc{1}{c}{\scriptsize{(0.804)}} & \mc{1}{c}{\scriptsize{(0.765)}} & \mc{1}{c}{\scriptsize{(0.510)}} & \mc{1}{c}{\scriptsize{(0.784)}} & \mc{1}{c}{\scriptsize{(0.765)}} & \mc{1}{c}{\scriptsize{(0.863)}} & \mc{1}{c}{\scriptsize{(0.824)}} \\  

    \mc{1}{l}{\scriptsize{Views Self as Dumb}} & \mc{1}{c}{\scriptsize{12}} & \mc{1}{c}{\scriptsize{0.123}} & \mc{1}{c}{\scriptsize{0.114}} & \mc{1}{c}{\scriptsize{0.023}} & \mc{1}{c}{\scriptsize{-0.063}} & \mc{1}{c}{\scriptsize{-0.034}} & \mc{1}{c}{\scriptsize{0.223}} & \mc{1}{c}{\scriptsize{0.277}} & \mc{1}{c}{\scriptsize{0.175}} \\  

     &  & \mc{1}{c}{\scriptsize{(0.843)}} & \mc{1}{c}{\scriptsize{(0.745)}} & \mc{1}{c}{\scriptsize{(0.510)}} & \mc{1}{c}{\scriptsize{(0.392)}} & \mc{1}{c}{\scriptsize{(0.333)}} & \mc{1}{c}{\scriptsize{(0.980)}} & \mc{1}{c}{\scriptsize{(0.902)}} & \mc{1}{c}{\scriptsize{(0.863)}} \\  

    \mc{1}{l}{\scriptsize{Views Self as Clumsy}} & \mc{1}{c}{\scriptsize{12}} & \mc{1}{c}{\scriptsize{0.339}} & \mc{1}{c}{\scriptsize{0.418}} & \mc{1}{c}{\scriptsize{0.439}} & \mc{1}{c}{\scriptsize{0.613}} & \mc{1}{c}{\scriptsize{0.463}} & \mc{1}{c}{\scriptsize{0.238}} & \mc{1}{c}{\scriptsize{0.265}} & \mc{1}{c}{\scriptsize{0.173}} \\  

     &  & \mc{1}{c}{\scriptsize{(1.000)}} & \mc{1}{c}{\scriptsize{(1.000)}} & \mc{1}{c}{\scriptsize{(1.000)}} & \mc{1}{c}{\scriptsize{(0.980)}} & \mc{1}{c}{\scriptsize{(1.000)}} & \mc{1}{c}{\scriptsize{(0.902)}} & \mc{1}{c}{\scriptsize{(0.902)}} & \mc{1}{c}{\scriptsize{(0.765)}} \\  

    \mc{1}{l}{\scriptsize{Views Self as Not Liked}} & \mc{1}{c}{\scriptsize{12}} & \mc{1}{c}{\scriptsize{-0.085}} & \mc{1}{c}{\scriptsize{-0.109}} & \mc{1}{c}{\scriptsize{0.115}} & \mc{1}{c}{\scriptsize{0.148}} & \mc{1}{c}{\scriptsize{0.086}} & \mc{1}{c}{\scriptsize{-0.285}} & \mc{1}{c}{\scriptsize{-0.441}} & \mc{1}{c}{\scriptsize{-0.374}} \\  

     &  & \mc{1}{c}{\scriptsize{(0.235)}} & \mc{1}{c}{\scriptsize{(0.255)}} & \mc{1}{c}{\scriptsize{(0.941)}} & \mc{1}{c}{\scriptsize{(0.686)}} & \mc{1}{c}{\scriptsize{(0.784)}} & \mc{1}{c}{\scriptsize{(0.157)}} & \mc{1}{c}{\scriptsize{\textbf{(0.098)}}} & \mc{1}{c}{\scriptsize{(0.137)}} \\  

    \mc{1}{l}{\scriptsize{Proud about Self}} & \mc{1}{c}{\scriptsize{12}} & \mc{1}{c}{\scriptsize{0.196}} & \mc{1}{c}{\scriptsize{0.224}} & \mc{1}{c}{\scriptsize{0.346}} & \mc{1}{c}{\scriptsize{0.352}} & \mc{1}{c}{\scriptsize{0.326}} & \mc{1}{c}{\scriptsize{0.046}} & \mc{1}{c}{\scriptsize{0.048}} & \mc{1}{c}{\scriptsize{0.010}} \\  

     &  & \mc{1}{c}{\scriptsize{\textbf{(0.059)}}} & \mc{1}{c}{\scriptsize{\textbf{(0.078)}}} & \mc{1}{c}{\scriptsize{\textbf{(0.000)}}} & \mc{1}{c}{\scriptsize{\textbf{(0.020)}}} & \mc{1}{c}{\scriptsize{\textbf{(0.000)}}} & \mc{1}{c}{\scriptsize{(0.451)}} & \mc{1}{c}{\scriptsize{(0.490)}} & \mc{1}{c}{\scriptsize{(0.510)}} \\  

    \mc{1}{l}{\scriptsize{Family Proud of You}} & \mc{1}{c}{\scriptsize{12}} & \mc{1}{c}{\scriptsize{-0.042}} & \mc{1}{c}{\scriptsize{0.016}} & \mc{1}{c}{\scriptsize{0.008}} & \mc{1}{c}{\scriptsize{0.022}} & \mc{1}{c}{\scriptsize{-0.006}} & \mc{1}{c}{\scriptsize{-0.092}} & \mc{1}{c}{\scriptsize{-0.043}} & \mc{1}{c}{\scriptsize{-0.074}} \\  

     &  & \mc{1}{c}{\scriptsize{(0.588)}} & \mc{1}{c}{\scriptsize{(0.490)}} & \mc{1}{c}{\scriptsize{(0.471)}} & \mc{1}{c}{\scriptsize{(0.412)}} & \mc{1}{c}{\scriptsize{(0.529)}} & \mc{1}{c}{\scriptsize{(0.667)}} & \mc{1}{c}{\scriptsize{(0.667)}} & \mc{1}{c}{\scriptsize{(0.627)}} \\  

    \mc{1}{l}{\scriptsize{Feels Inadequate, Inferior}} & \mc{1}{c}{\scriptsize{12}} & \mc{1}{c}{\scriptsize{0.169}} & \mc{1}{c}{\scriptsize{0.209}} & \mc{1}{c}{\scriptsize{0.269}} & \mc{1}{c}{\scriptsize{0.289}} & \mc{1}{c}{\scriptsize{0.306}} & \mc{1}{c}{\scriptsize{0.069}} & \mc{1}{c}{\scriptsize{0.137}} & \mc{1}{c}{\scriptsize{0.137}} \\  

     &  & \mc{1}{c}{\scriptsize{(0.941)}} & \mc{1}{c}{\scriptsize{(0.902)}} & \mc{1}{c}{\scriptsize{(0.980)}} & \mc{1}{c}{\scriptsize{(0.922)}} & \mc{1}{c}{\scriptsize{(1.000)}} & \mc{1}{c}{\scriptsize{(0.569)}} & \mc{1}{c}{\scriptsize{(0.667)}} & \mc{1}{c}{\scriptsize{(0.627)}} \\  

    \mc{1}{l}{\scriptsize{Withdraws Excessively}} & \mc{1}{c}{\scriptsize{12}} & \mc{1}{c}{\scriptsize{0.288}} & \mc{1}{c}{\scriptsize{0.296}} & \mc{1}{c}{\scriptsize{0.339}} & \mc{1}{c}{\scriptsize{0.267}} & \mc{1}{c}{\scriptsize{0.326}} & \mc{1}{c}{\scriptsize{0.238}} & \mc{1}{c}{\scriptsize{0.320}} & \mc{1}{c}{\scriptsize{0.226}} \\  

     &  & \mc{1}{c}{\scriptsize{(1.000)}} & \mc{1}{c}{\scriptsize{(1.000)}} & \mc{1}{c}{\scriptsize{(1.000)}} & \mc{1}{c}{\scriptsize{(0.902)}} & \mc{1}{c}{\scriptsize{(1.000)}} & \mc{1}{c}{\scriptsize{(0.902)}} & \mc{1}{c}{\scriptsize{(0.824)}} & \mc{1}{c}{\scriptsize{(0.902)}} \\  

    \mc{1}{l}{\scriptsize{Ignores Situation}} & \mc{1}{c}{\scriptsize{12}} & \mc{1}{c}{\scriptsize{0.088}} & \mc{1}{c}{\scriptsize{0.144}} & \mc{1}{c}{\scriptsize{0.238}} & \mc{1}{c}{\scriptsize{0.338}} & \mc{1}{c}{\scriptsize{0.239}} & \mc{1}{c}{\scriptsize{-0.061}} & \mc{1}{c}{\scriptsize{-0.037}} & \mc{1}{c}{\scriptsize{-0.081}} \\  

     &  & \mc{1}{c}{\scriptsize{(0.706)}} & \mc{1}{c}{\scriptsize{(0.863)}} & \mc{1}{c}{\scriptsize{(0.902)}} & \mc{1}{c}{\scriptsize{(0.961)}} & \mc{1}{c}{\scriptsize{(0.922)}} & \mc{1}{c}{\scriptsize{(0.392)}} & \mc{1}{c}{\scriptsize{(0.471)}} & \mc{1}{c}{\scriptsize{(0.294)}} \\  

    \mc{1}{l}{\scriptsize{Not Cope with Prob.}} & \mc{1}{c}{\scriptsize{12}} & \mc{1}{c}{\scriptsize{0.088}} & \mc{1}{c}{\scriptsize{0.146}} & \mc{1}{c}{\scriptsize{0.038}} & \mc{1}{c}{\scriptsize{0.087}} & \mc{1}{c}{\scriptsize{0.047}} & \mc{1}{c}{\scriptsize{0.139}} & \mc{1}{c}{\scriptsize{0.165}} & \mc{1}{c}{\scriptsize{0.192}} \\  

     &  & \mc{1}{c}{\scriptsize{(0.647)}} & \mc{1}{c}{\scriptsize{(0.824)}} & \mc{1}{c}{\scriptsize{(0.510)}} & \mc{1}{c}{\scriptsize{(0.706)}} & \mc{1}{c}{\scriptsize{(0.529)}} & \mc{1}{c}{\scriptsize{(0.745)}} & \mc{1}{c}{\scriptsize{(0.745)}} & \mc{1}{c}{\scriptsize{(0.804)}} \\  

    \mc{1}{l}{\scriptsize{Often Mad of Angry}} & \mc{1}{c}{\scriptsize{12}} & \mc{1}{c}{\scriptsize{-0.051}} & \mc{1}{c}{\scriptsize{0.073}} & \mc{1}{c}{\scriptsize{0.160}} & \mc{1}{c}{\scriptsize{0.178}} & \mc{1}{c}{\scriptsize{0.182}} & \mc{1}{c}{\scriptsize{-0.240}} & \mc{1}{c}{\scriptsize{-0.115}} & \mc{1}{c}{\scriptsize{-0.213}} \\  

     &  & \mc{1}{c}{\scriptsize{(0.353)}} & \mc{1}{c}{\scriptsize{(0.706)}} & \mc{1}{c}{\scriptsize{(0.980)}} & \mc{1}{c}{\scriptsize{(0.902)}} & \mc{1}{c}{\scriptsize{(0.961)}} & \mc{1}{c}{\scriptsize{(0.216)}} & \mc{1}{c}{\scriptsize{(0.294)}} & \mc{1}{c}{\scriptsize{(0.294)}} \\  

    \mc{1}{l}{\scriptsize{Impulsivity}} & \mc{1}{c}{\scriptsize{12}} & \mc{1}{c}{\scriptsize{0.081}} & \mc{1}{c}{\scriptsize{0.040}} & \mc{1}{c}{\scriptsize{0.131}} & \mc{1}{c}{\scriptsize{-0.111}} & \mc{1}{c}{\scriptsize{0.052}} & \mc{1}{c}{\scriptsize{0.031}} & \mc{1}{c}{\scriptsize{0.183}} & \mc{1}{c}{\scriptsize{0.096}} \\  

     &  & \mc{1}{c}{\scriptsize{(0.706)}} & \mc{1}{c}{\scriptsize{(0.490)}} & \mc{1}{c}{\scriptsize{(0.784)}} & \mc{1}{c}{\scriptsize{(0.255)}} & \mc{1}{c}{\scriptsize{(0.569)}} & \mc{1}{c}{\scriptsize{(0.490)}} & \mc{1}{c}{\scriptsize{(0.725)}} & \mc{1}{c}{\scriptsize{(0.667)}} \\  

    \mc{1}{l}{\scriptsize{Significant Fears}} & \mc{1}{c}{\scriptsize{12}} & \mc{1}{c}{\scriptsize{-0.031}} & \mc{1}{c}{\scriptsize{0.021}} & \mc{1}{c}{\scriptsize{0.069}} & \mc{1}{c}{\scriptsize{0.043}} & \mc{1}{c}{\scriptsize{0.017}} & \mc{1}{c}{\scriptsize{-0.131}} & \mc{1}{c}{\scriptsize{0.076}} & \mc{1}{c}{\scriptsize{-0.165}} \\  

     &  & \mc{1}{c}{\scriptsize{(0.431)}} & \mc{1}{c}{\scriptsize{(0.588)}} & \mc{1}{c}{\scriptsize{(0.647)}} & \mc{1}{c}{\scriptsize{(0.667)}} & \mc{1}{c}{\scriptsize{(0.627)}} & \mc{1}{c}{\scriptsize{(0.196)}} & \mc{1}{c}{\scriptsize{(0.667)}} & \mc{1}{c}{\scriptsize{\textbf{(0.059)}}} \\  

    \mc{1}{l}{\scriptsize{Denies Any Worries}} & \mc{1}{c}{\scriptsize{12}} & \mc{1}{c}{\scriptsize{0.181}} & \mc{1}{c}{\scriptsize{0.224}} & \mc{1}{c}{\scriptsize{0.231}} & \mc{1}{c}{\scriptsize{0.314}} & \mc{1}{c}{\scriptsize{0.256}} & \mc{1}{c}{\scriptsize{0.131}} & \mc{1}{c}{\scriptsize{0.080}} & \mc{1}{c}{\scriptsize{0.078}} \\  

     &  & \mc{1}{c}{\scriptsize{(0.922)}} & \mc{1}{c}{\scriptsize{(1.000)}} & \mc{1}{c}{\scriptsize{(0.961)}} & \mc{1}{c}{\scriptsize{(0.863)}} & \mc{1}{c}{\scriptsize{(0.961)}} & \mc{1}{c}{\scriptsize{(0.824)}} & \mc{1}{c}{\scriptsize{(0.745)}} & \mc{1}{c}{\scriptsize{(0.745)}} \\ 
    \hline  

    \\[0.1cm]
    \mc{2}{l}{\scriptsize{\% of Pos. TE ($H_0$: $\le$ 25\% $|$ 10\% Significance)}} & \mc{1}{c}{\scriptsize{6}} & \mc{1}{c}{\scriptsize{6}} & \mc{1}{c}{\scriptsize{6}} & \mc{1}{c}{\scriptsize{6}} & \mc{1}{c}{\scriptsize{12}} & \mc{1}{c}{\scriptsize{0}} & \mc{1}{c}{\scriptsize{6}} & \mc{1}{c}{\scriptsize{6}} \\  

     &  & \mc{1}{c}{\scriptsize{(1.000)}} & \mc{1}{c}{\scriptsize{(1.000)}} & \mc{1}{c}{\scriptsize{(1.000)}} & \mc{1}{c}{\scriptsize{(1.000)}} & \mc{1}{c}{\scriptsize{(0.922)}} & \mc{1}{c}{\scriptsize{(1.000)}} & \mc{1}{c}{\scriptsize{(1.000)}} & \mc{1}{c}{\scriptsize{(1.000)}} \\  

    \mc{2}{l}{\scriptsize{\% of Pos. TE ($H_0$: $\le$ 50\% $|$ 10\% Significance)}} & \mc{1}{c}{\scriptsize{6}} & \mc{1}{c}{\scriptsize{6}} & \mc{1}{c}{\scriptsize{6}} & \mc{1}{c}{\scriptsize{6}} & \mc{1}{c}{\scriptsize{12}} & \mc{1}{c}{\scriptsize{0}} & \mc{1}{c}{\scriptsize{6}} & \mc{1}{c}{\scriptsize{6}} \\  

     &  & \mc{1}{c}{\scriptsize{(1.000)}} & \mc{1}{c}{\scriptsize{(1.000)}} & \mc{1}{c}{\scriptsize{(1.000)}} & \mc{1}{c}{\scriptsize{(1.000)}} & \mc{1}{c}{\scriptsize{(1.000)}} & \mc{1}{c}{\scriptsize{(1.000)}} & \mc{1}{c}{\scriptsize{(1.000)}} & \mc{1}{c}{\scriptsize{(1.000)}} \\  

    \mc{2}{l}{\scriptsize{\% of Pos. TE ($H_0$: $\le$ 75\% $|$ 10\% Significance)}} & \mc{1}{c}{\scriptsize{6}} & \mc{1}{c}{\scriptsize{6}} & \mc{1}{c}{\scriptsize{6}} & \mc{1}{c}{\scriptsize{6}} & \mc{1}{c}{\scriptsize{12}} & \mc{1}{c}{\scriptsize{0}} & \mc{1}{c}{\scriptsize{6}} & \mc{1}{c}{\scriptsize{6}} \\  

     &  & \mc{1}{c}{\scriptsize{(1.000)}} & \mc{1}{c}{\scriptsize{(1.000)}} & \mc{1}{c}{\scriptsize{(1.000)}} & \mc{1}{c}{\scriptsize{(1.000)}} & \mc{1}{c}{\scriptsize{(1.000)}} & \mc{1}{c}{\scriptsize{(1.000)}} & \mc{1}{c}{\scriptsize{(1.000)}} & \mc{1}{c}{\scriptsize{(1.000)}} \\  

  \hline\hline
  \end{tabular}
    \begin{tablenotes}
    \scriptsize
    \item 
Note: This table displays various estimates of the treatment effect of ABC's school age program.
Column (1) displays the ITT, without accounting for any controls.
Column (2) displays the ITT conditioning on vector of controls, $X$, consisting of APGAR scores 1 
minute after birth, an indicator for the subject being born prematurely, and an indicator for the 
father being home at baseline. We also apply IPW weights, $W$, to account for attrition.
Columns (3)--(4) are analogous to columns (1)--(2), but we restrict the control sample to subjects
who did not enroll in any alternative care.
Column (5) displys the matching estimate, where we use the Mahalanobis metric and Epanechnikov kernel
to match on controls $X$ listed above, and restrict the control sample to subjects who did not enroll
in any alternative care. Additionally, we apply IPW weights, $W$.
Columns (6)--(8) are analogous to Columns (3)--(5), except we restrict the control sample to subejcts
who did enroll in alternative care. 
The final three pairs of rows display the proportion of treatment effects in the table that are 
socially positive. The first row in each pair displays the percentage of treatment effects, and the
second row presents the inference.

Numbers in parentheses represent the $p$-value from a single hypothesis test, and are obtained from 
the empirical bootstrap distribution generated by 200 resamples of the original data. 
Bold $p$-values indicate significance at the 10\% level.
Blank point estimates indicate that we are unable to obtain estimates due to a lack of support in the data. 

    \end{tablenotes}
  \end{threeparttable}

\end{table}
	\end{table} 

	\begin{table}[H]
     \caption{Treatment Effects on Subject Home and Property, Female Sample}
     \label{table:abccare_rslt_female_cat21}
	  \begin{tabular}{cccccccccc}
  \toprule

    \scriptsize{Variable} & \scriptsize{Age} & \scriptsize{(1)} & \scriptsize{(2)} & \scriptsize{(3)} & \scriptsize{(4)} & \scriptsize{(5)} & \scriptsize{(6)} & \scriptsize{(7)} & \scriptsize{(8)} \\ 
    \midrule  

    \mc{1}{l}{\scriptsize{Room density (room/people)}} & \mc{1}{c}{\scriptsize{30}} & \mc{1}{c}{\scriptsize{0.130}} & \mc{1}{c}{\scriptsize{0.255}} & \mc{1}{c}{\scriptsize{0.286}} & \mc{1}{c}{\scriptsize{0.476}} & \mc{1}{c}{\scriptsize{0.340}} & \mc{1}{c}{\scriptsize{0.120}} & \mc{1}{c}{\scriptsize{0.150}} & \mc{1}{c}{\scriptsize{0.162}} \\  

     &  & \mc{1}{c}{\scriptsize{(0.342)}} & \mc{1}{c}{\scriptsize{(0.197)}} & \mc{1}{c}{\scriptsize{(0.250)}} & \mc{1}{c}{\scriptsize{(0.184)}} & \mc{1}{c}{\scriptsize{(0.250)}} & \mc{1}{c}{\scriptsize{(0.382)}} & \mc{1}{c}{\scriptsize{(0.368)}} & \mc{1}{c}{\scriptsize{(0.329)}} \\  

    \mc{1}{l}{\scriptsize{Own computers}} & \mc{1}{c}{\scriptsize{30}} & \mc{1}{c}{\scriptsize{0.058}} & \mc{1}{c}{\scriptsize{0.009}} & \mc{1}{c}{\scriptsize{0.108}} & \mc{1}{c}{\scriptsize{0.143}} & \mc{1}{c}{\scriptsize{0.095}} & \mc{1}{c}{\scriptsize{0.030}} & \mc{1}{c}{\scriptsize{-0.044}} & \mc{1}{c}{\scriptsize{-0.012}} \\  

     &  & \mc{1}{c}{\scriptsize{(0.316)}} & \mc{1}{c}{\scriptsize{(0.447)}} & \mc{1}{c}{\scriptsize{(0.237)}} & \mc{1}{c}{\scriptsize{(0.224)}} & \mc{1}{c}{\scriptsize{(0.289)}} & \mc{1}{c}{\scriptsize{(0.421)}} & \mc{1}{c}{\scriptsize{(0.658)}} & \mc{1}{c}{\scriptsize{(0.553)}} \\  

    \mc{1}{l}{\scriptsize{Own cars}} & \mc{1}{c}{\scriptsize{30}} & \mc{1}{c}{\scriptsize{0.158}} & \mc{1}{c}{\scriptsize{0.191}} & \mc{1}{c}{\scriptsize{0.208}} & \mc{1}{c}{\scriptsize{0.199}} & \mc{1}{c}{\scriptsize{0.254}} & \mc{1}{c}{\scriptsize{0.167}} & \mc{1}{c}{\scriptsize{0.170}} & \mc{1}{c}{\scriptsize{0.187}} \\  

     &  & \mc{1}{c}{\scriptsize{\textbf{(0.066)}}} & \mc{1}{c}{\scriptsize{\textbf{(0.013)}}} & \mc{1}{c}{\scriptsize{(0.132)}} & \mc{1}{c}{\scriptsize{(0.158)}} & \mc{1}{c}{\scriptsize{(0.132)}} & \mc{1}{c}{\scriptsize{\textbf{(0.079)}}} & \mc{1}{c}{\scriptsize{(0.118)}} & \mc{1}{c}{\scriptsize{\textbf{(0.039)}}} \\  

    \mc{1}{l}{\scriptsize{Own residences}} & \mc{1}{c}{\scriptsize{30}} & \mc{1}{c}{\scriptsize{0.165}} & \mc{1}{c}{\scriptsize{0.176}} & \mc{1}{c}{\scriptsize{0.300}} & \mc{1}{c}{\scriptsize{0.304}} & \mc{1}{c}{\scriptsize{0.320}} & \mc{1}{c}{\scriptsize{0.115}} & \mc{1}{c}{\scriptsize{0.119}} & \mc{1}{c}{\scriptsize{0.128}} \\  

     &  & \mc{1}{c}{\scriptsize{\textbf{(0.039)}}} & \mc{1}{c}{\scriptsize{\textbf{(0.026)}}} & \mc{1}{c}{\scriptsize{\textbf{(0.000)}}} & \mc{1}{c}{\scriptsize{\textbf{(0.000)}}} & \mc{1}{c}{\scriptsize{\textbf{(0.000)}}} & \mc{1}{c}{\scriptsize{\textbf{(0.066)}}} & \mc{1}{c}{\scriptsize{(0.145)}} & \mc{1}{c}{\scriptsize{\textbf{(0.053)}}} \\  

  \bottomrule
  \end{tabular}
	\end{table} 

	\begin{table}[H]
     \caption{Treatment Effects on Education, Female Sample}
     \label{table:abccare_rslt_female_cat22}
	  \begin{tabular}{cccccccccc}
  \toprule

    \scriptsize{Variable} & \scriptsize{Age} & \scriptsize{(1)} & \scriptsize{(2)} & \scriptsize{(3)} & \scriptsize{(4)} & \scriptsize{(5)} & \scriptsize{(6)} & \scriptsize{(7)} & \scriptsize{(8)} \\ 
    \midrule  

    \mc{1}{l}{\scriptsize{Household Earned Income}} & \mc{1}{c}{\scriptsize{9}} & \mc{1}{c}{\scriptsize{17,749}} & \mc{1}{c}{\scriptsize{39,936}} & \mc{1}{c}{\scriptsize{17,749}} & \mc{1}{c}{\scriptsize{39,936}} & \mc{1}{c}{\scriptsize{9,005}} &  &  &  \\  

     &  & \mc{1}{c}{\scriptsize{\textbf{(0.092)}}} & \mc{1}{c}{\scriptsize{\textbf{(0.000)}}} & \mc{1}{c}{\scriptsize{\textbf{(0.092)}}} & \mc{1}{c}{\scriptsize{\textbf{(0.000)}}} & \mc{1}{c}{\scriptsize{(0.171)}} &  &  &  \\  

     & \mc{1}{c}{\scriptsize{0}} & \mc{1}{c}{\scriptsize{3,156}} & \mc{1}{c}{\scriptsize{8,599}} & \mc{1}{c}{\scriptsize{5,175}} & \mc{1}{c}{\scriptsize{21,776}} & \mc{1}{c}{\scriptsize{9,760}} & \mc{1}{c}{\scriptsize{1,540}} & \mc{1}{c}{\scriptsize{6,550}} & \mc{1}{c}{\scriptsize{6,107}} \\  

     &  & \mc{1}{c}{\scriptsize{(0.237)}} & \mc{1}{c}{\scriptsize{(0.184)}} & \mc{1}{c}{\scriptsize{(0.184)}} & \mc{1}{c}{\scriptsize{(0.118)}} & \mc{1}{c}{\scriptsize{\textbf{(0.053)}}} & \mc{1}{c}{\scriptsize{(0.382)}} & \mc{1}{c}{\scriptsize{(0.303)}} & \mc{1}{c}{\scriptsize{(0.171)}} \\  

     & \mc{1}{c}{\scriptsize{12}} & \mc{1}{c}{\scriptsize{11,182}} & \mc{1}{c}{\scriptsize{12,209}} & \mc{1}{c}{\scriptsize{24,645}} & \mc{1}{c}{\scriptsize{32,766}} & \mc{1}{c}{\scriptsize{23,251}} & \mc{1}{c}{\scriptsize{8,122}} & \mc{1}{c}{\scriptsize{5,571}} & \mc{1}{c}{\scriptsize{8,482}} \\  

     &  & \mc{1}{c}{\scriptsize{\textbf{(0.039)}}} & \mc{1}{c}{\scriptsize{\textbf{(0.092)}}} & \mc{1}{c}{\scriptsize{\textbf{(0.000)}}} & \mc{1}{c}{\scriptsize{\textbf{(0.013)}}} & \mc{1}{c}{\scriptsize{\textbf{(0.000)}}} & \mc{1}{c}{\scriptsize{(0.132)}} & \mc{1}{c}{\scriptsize{(0.289)}} & \mc{1}{c}{\scriptsize{\textbf{(0.092)}}} \\  

  \bottomrule
  \end{tabular}
	\end{table} 

	\begin{table}[H]
     \caption{Treatment Effects on Subject Employment and Income, Female Sample}
     \label{table:abccare_rslt_female_cat23}
	\input{AppResOutput/abccare/rslt_female_cat23}
	\end{table} 

	\begin{table}[H]
     \caption{Treatment Effects on Job Attitude, Female Sample}
     \label{table:abccare_rslt_female_cat24}
	  \begin{tabular}{cccccccccc}
  \toprule

    \scriptsize{Variable} & \scriptsize{Age} & \scriptsize{(1)} & \scriptsize{(2)} & \scriptsize{(3)} & \scriptsize{(4)} & \scriptsize{(5)} & \scriptsize{(6)} & \scriptsize{(7)} & \scriptsize{(8)} \\ 
    \midrule  

    \mc{1}{l}{\scriptsize{Satisfied with working situation}} & \mc{1}{c}{\scriptsize{30}} & \mc{1}{c}{\scriptsize{-0.164}} & \mc{1}{c}{\scriptsize{-0.005}} & \mc{1}{c}{\scriptsize{-0.359}} & \mc{1}{c}{\scriptsize{-0.452}} & \mc{1}{c}{\scriptsize{-0.175}} & \mc{1}{c}{\scriptsize{-0.133}} & \mc{1}{c}{\scriptsize{0.059}} & \mc{1}{c}{\scriptsize{0.043}} \\  

     &  & \mc{1}{c}{\scriptsize{(0.882)}} & \mc{1}{c}{\scriptsize{(0.553)}} & \mc{1}{c}{\scriptsize{(0.934)}} & \mc{1}{c}{\scriptsize{(0.947)}} & \mc{1}{c}{\scriptsize{(0.763)}} & \mc{1}{c}{\scriptsize{(0.816)}} & \mc{1}{c}{\scriptsize{(0.355)}} & \mc{1}{c}{\scriptsize{(0.382)}} \\  

    \mc{1}{l}{\scriptsize{Do work well}} & \mc{1}{c}{\scriptsize{30}} & \mc{1}{c}{\scriptsize{0.037}} & \mc{1}{c}{\scriptsize{0.007}} &  &  &  & \mc{1}{c}{\scriptsize{0.048}} & \mc{1}{c}{\scriptsize{0.013}} & \mc{1}{c}{\scriptsize{0.019}} \\  

     &  & \mc{1}{c}{\scriptsize{\textbf{(0.053)}}} & \mc{1}{c}{\scriptsize{(0.197)}} &  &  &  & \mc{1}{c}{\scriptsize{\textbf{(0.066)}}} & \mc{1}{c}{\scriptsize{(0.132)}} & \mc{1}{c}{\scriptsize{\textbf{(0.053)}}} \\  

    \mc{1}{l}{\scriptsize{Not worry too much about work}} & \mc{1}{c}{\scriptsize{30}} & \mc{1}{c}{\scriptsize{0.153}} & \mc{1}{c}{\scriptsize{0.211}} & \mc{1}{c}{\scriptsize{0.283}} & \mc{1}{c}{\scriptsize{0.216}} & \mc{1}{c}{\scriptsize{0.342}} & \mc{1}{c}{\scriptsize{0.116}} & \mc{1}{c}{\scriptsize{0.190}} & \mc{1}{c}{\scriptsize{0.142}} \\  

     &  & \mc{1}{c}{\scriptsize{(0.118)}} & \mc{1}{c}{\scriptsize{\textbf{(0.066)}}} & \mc{1}{c}{\scriptsize{(0.145)}} & \mc{1}{c}{\scriptsize{(0.289)}} & \mc{1}{c}{\scriptsize{(0.105)}} & \mc{1}{c}{\scriptsize{(0.158)}} & \mc{1}{c}{\scriptsize{(0.118)}} & \mc{1}{c}{\scriptsize{(0.145)}} \\  

    \mc{1}{l}{\scriptsize{Work well with others}} & \mc{1}{c}{\scriptsize{30}} & \mc{1}{c}{\scriptsize{0.092}} & \mc{1}{c}{\scriptsize{0.108}} & \mc{1}{c}{\scriptsize{-0.130}} & \mc{1}{c}{\scriptsize{-0.164}} & \mc{1}{c}{\scriptsize{-0.109}} & \mc{1}{c}{\scriptsize{0.108}} & \mc{1}{c}{\scriptsize{0.181}} & \mc{1}{c}{\scriptsize{0.129}} \\  

     &  & \mc{1}{c}{\scriptsize{(0.211)}} & \mc{1}{c}{\scriptsize{(0.211)}} & \mc{1}{c}{\scriptsize{(0.855)}} & \mc{1}{c}{\scriptsize{(0.632)}} & \mc{1}{c}{\scriptsize{(0.750)}} & \mc{1}{c}{\scriptsize{(0.184)}} & \mc{1}{c}{\scriptsize{(0.145)}} & \mc{1}{c}{\scriptsize{(0.158)}} \\  

    \mc{1}{l}{\scriptsize{Don't do things that cause to lose job}} & \mc{1}{c}{\scriptsize{30}} & \mc{1}{c}{\scriptsize{-0.019}} & \mc{1}{c}{\scriptsize{0.159}} & \mc{1}{c}{\scriptsize{-0.130}} & \mc{1}{c}{\scriptsize{-0.068}} & \mc{1}{c}{\scriptsize{-0.033}} & \mc{1}{c}{\scriptsize{0.012}} & \mc{1}{c}{\scriptsize{0.196}} & \mc{1}{c}{\scriptsize{0.157}} \\  

     &  & \mc{1}{c}{\scriptsize{(0.566)}} & \mc{1}{c}{\scriptsize{\textbf{(0.053)}}} & \mc{1}{c}{\scriptsize{(0.882)}} & \mc{1}{c}{\scriptsize{(0.487)}} & \mc{1}{c}{\scriptsize{(0.500)}} & \mc{1}{c}{\scriptsize{(0.434)}} & \mc{1}{c}{\scriptsize{\textbf{(0.013)}}} & \mc{1}{c}{\scriptsize{\textbf{(0.079)}}} \\  

    \mc{1}{l}{\scriptsize{No trouble finishing work}} & \mc{1}{c}{\scriptsize{30}} & \mc{1}{c}{\scriptsize{-0.050}} & \mc{1}{c}{\scriptsize{-0.079}} & \mc{1}{c}{\scriptsize{-0.087}} & \mc{1}{c}{\scriptsize{-0.256}} & \mc{1}{c}{\scriptsize{-0.171}} & \mc{1}{c}{\scriptsize{-0.039}} & \mc{1}{c}{\scriptsize{-0.052}} & \mc{1}{c}{\scriptsize{-0.093}} \\  

     &  & \mc{1}{c}{\scriptsize{(0.789)}} & \mc{1}{c}{\scriptsize{(0.724)}} & \mc{1}{c}{\scriptsize{(0.789)}} & \mc{1}{c}{\scriptsize{(0.658)}} & \mc{1}{c}{\scriptsize{(0.803)}} & \mc{1}{c}{\scriptsize{(0.671)}} & \mc{1}{c}{\scriptsize{(0.605)}} & \mc{1}{c}{\scriptsize{(0.789)}} \\  

    \mc{1}{l}{\scriptsize{Job not too stressful}} & \mc{1}{c}{\scriptsize{30}} & \mc{1}{c}{\scriptsize{0.005}} & \mc{1}{c}{\scriptsize{0.048}} & \mc{1}{c}{\scriptsize{-0.217}} & \mc{1}{c}{\scriptsize{-0.331}} & \mc{1}{c}{\scriptsize{-0.171}} & \mc{1}{c}{\scriptsize{0.021}} & \mc{1}{c}{\scriptsize{0.109}} & \mc{1}{c}{\scriptsize{0.090}} \\  

     &  & \mc{1}{c}{\scriptsize{(0.487)}} & \mc{1}{c}{\scriptsize{(0.382)}} & \mc{1}{c}{\scriptsize{(0.974)}} & \mc{1}{c}{\scriptsize{(0.882)}} & \mc{1}{c}{\scriptsize{(0.947)}} & \mc{1}{c}{\scriptsize{(0.434)}} & \mc{1}{c}{\scriptsize{(0.211)}} & \mc{1}{c}{\scriptsize{(0.250)}} \\  

    \mc{1}{l}{\scriptsize{Don't stay away from job}} & \mc{1}{c}{\scriptsize{30}} & \mc{1}{c}{\scriptsize{0.090}} & \mc{1}{c}{\scriptsize{0.244}} & \mc{1}{c}{\scriptsize{0.283}} & \mc{1}{c}{\scriptsize{0.200}} & \mc{1}{c}{\scriptsize{0.374}} & \mc{1}{c}{\scriptsize{0.083}} & \mc{1}{c}{\scriptsize{0.201}} & \mc{1}{c}{\scriptsize{0.113}} \\  

     &  & \mc{1}{c}{\scriptsize{(0.211)}} & \mc{1}{c}{\scriptsize{\textbf{(0.039)}}} & \mc{1}{c}{\scriptsize{(0.132)}} & \mc{1}{c}{\scriptsize{(0.263)}} & \mc{1}{c}{\scriptsize{(0.105)}} & \mc{1}{c}{\scriptsize{(0.250)}} & \mc{1}{c}{\scriptsize{(0.132)}} & \mc{1}{c}{\scriptsize{(0.250)}} \\  

    \mc{1}{l}{\scriptsize{No trouble with boss}} & \mc{1}{c}{\scriptsize{30}} & \mc{1}{c}{\scriptsize{0.351}} & \mc{1}{c}{\scriptsize{0.430}} & \mc{1}{c}{\scriptsize{0.120}} & \mc{1}{c}{\scriptsize{0.144}} & \mc{1}{c}{\scriptsize{0.202}} & \mc{1}{c}{\scriptsize{0.393}} & \mc{1}{c}{\scriptsize{0.523}} & \mc{1}{c}{\scriptsize{0.449}} \\  

     &  & \mc{1}{c}{\scriptsize{\textbf{(0.000)}}} & \mc{1}{c}{\scriptsize{\textbf{(0.000)}}} & \mc{1}{c}{\scriptsize{(0.289)}} & \mc{1}{c}{\scriptsize{(0.382)}} & \mc{1}{c}{\scriptsize{(0.118)}} & \mc{1}{c}{\scriptsize{\textbf{(0.000)}}} & \mc{1}{c}{\scriptsize{\textbf{(0.000)}}} & \mc{1}{c}{\scriptsize{\textbf{(0.000)}}} \\  

  \bottomrule
  \end{tabular}
	\end{table} 

	\begin{table}[H]
     \caption{Treatment Effects on Job Satisfaction Score, Female Sample}
     \label{table:abccare_rslt_female_cat25}
	  \begin{tabular}{cccccccccc}
  \toprule

    \scriptsize{Variable} & \scriptsize{Age} & \scriptsize{(1)} & \scriptsize{(2)} & \scriptsize{(3)} & \scriptsize{(4)} & \scriptsize{(5)} & \scriptsize{(6)} & \scriptsize{(7)} & \scriptsize{(8)} \\ 
    \midrule  

    \mc{1}{l}{\scriptsize{Std. Achv.  Test}} & \mc{1}{c}{\scriptsize{8}} & \mc{1}{c}{\scriptsize{6.619}} & \mc{1}{c}{\scriptsize{5.350}} & \mc{1}{c}{\scriptsize{7.125}} & \mc{1}{c}{\scriptsize{11.989}} & \mc{1}{c}{\scriptsize{7.738}} & \mc{1}{c}{\scriptsize{6.465}} & \mc{1}{c}{\scriptsize{3.618}} & \mc{1}{c}{\scriptsize{7.036}} \\  

     &  & \mc{1}{c}{\scriptsize{\textbf{(0.000)}}} & \mc{1}{c}{\scriptsize{\textbf{(0.026)}}} & \mc{1}{c}{\scriptsize{\textbf{(0.053)}}} & \mc{1}{c}{\scriptsize{\textbf{(0.000)}}} & \mc{1}{c}{\scriptsize{\textbf{(0.039)}}} & \mc{1}{c}{\scriptsize{\textbf{(0.039)}}} & \mc{1}{c}{\scriptsize{(0.171)}} & \mc{1}{c}{\scriptsize{\textbf{(0.039)}}} \\  

     & \mc{1}{c}{\scriptsize{21}} & \mc{1}{c}{\scriptsize{9.116}} & \mc{1}{c}{\scriptsize{4.052}} & \mc{1}{c}{\scriptsize{8.420}} & \mc{1}{c}{\scriptsize{3.292}} & \mc{1}{c}{\scriptsize{6.500}} & \mc{1}{c}{\scriptsize{9.420}} & \mc{1}{c}{\scriptsize{4.176}} & \mc{1}{c}{\scriptsize{7.489}} \\  

     &  & \mc{1}{c}{\scriptsize{\textbf{(0.013)}}} & \mc{1}{c}{\scriptsize{(0.132)}} & \mc{1}{c}{\scriptsize{\textbf{(0.013)}}} & \mc{1}{c}{\scriptsize{(0.145)}} & \mc{1}{c}{\scriptsize{\textbf{(0.026)}}} & \mc{1}{c}{\scriptsize{\textbf{(0.026)}}} & \mc{1}{c}{\scriptsize{(0.158)}} & \mc{1}{c}{\scriptsize{\textbf{(0.039)}}} \\  

     & \mc{1}{c}{\scriptsize{15}} & \mc{1}{c}{\scriptsize{8.275}} & \mc{1}{c}{\scriptsize{4.962}} & \mc{1}{c}{\scriptsize{9.618}} & \mc{1}{c}{\scriptsize{6.485}} & \mc{1}{c}{\scriptsize{8.389}} & \mc{1}{c}{\scriptsize{8.477}} & \mc{1}{c}{\scriptsize{4.299}} & \mc{1}{c}{\scriptsize{7.428}} \\  

     &  & \mc{1}{c}{\scriptsize{\textbf{(0.000)}}} & \mc{1}{c}{\scriptsize{\textbf{(0.039)}}} & \mc{1}{c}{\scriptsize{\textbf{(0.000)}}} & \mc{1}{c}{\scriptsize{\textbf{(0.026)}}} & \mc{1}{c}{\scriptsize{\textbf{(0.000)}}} & \mc{1}{c}{\scriptsize{\textbf{(0.000)}}} & \mc{1}{c}{\scriptsize{(0.118)}} & \mc{1}{c}{\scriptsize{\textbf{(0.026)}}} \\  

     & \mc{1}{c}{\scriptsize{6.5}} & \mc{1}{c}{\scriptsize{3.909}} & \mc{1}{c}{\scriptsize{5.045}} & \mc{1}{c}{\scriptsize{6.394}} & \mc{1}{c}{\scriptsize{7.718}} & \mc{1}{c}{\scriptsize{6.022}} & \mc{1}{c}{\scriptsize{3.517}} & \mc{1}{c}{\scriptsize{4.431}} & \mc{1}{c}{\scriptsize{4.934}} \\  

     &  & \mc{1}{c}{\scriptsize{\textbf{(0.026)}}} & \mc{1}{c}{\scriptsize{\textbf{(0.013)}}} & \mc{1}{c}{\scriptsize{\textbf{(0.000)}}} & \mc{1}{c}{\scriptsize{\textbf{(0.039)}}} & \mc{1}{c}{\scriptsize{\textbf{(0.026)}}} & \mc{1}{c}{\scriptsize{\textbf{(0.053)}}} & \mc{1}{c}{\scriptsize{\textbf{(0.026)}}} & \mc{1}{c}{\scriptsize{\textbf{(0.013)}}} \\  

     & \mc{1}{c}{\scriptsize{12}} & \mc{1}{c}{\scriptsize{5.684}} & \mc{1}{c}{\scriptsize{-3.610}} & \mc{1}{c}{\scriptsize{12.790}} & \mc{1}{c}{\scriptsize{12.821}} & \mc{1}{c}{\scriptsize{13.783}} &  & \mc{1}{c}{\scriptsize{-7.942}} & \mc{1}{c}{\scriptsize{0.974}} \\  

     &  & \mc{1}{c}{\scriptsize{(0.132)}} & \mc{1}{c}{\scriptsize{(0.776)}} & \mc{1}{c}{\scriptsize{\textbf{(0.000)}}} & \mc{1}{c}{\scriptsize{\textbf{(0.053)}}} & \mc{1}{c}{\scriptsize{\textbf{(0.000)}}} &  & \mc{1}{c}{\scriptsize{(0.513)}} & \mc{1}{c}{\scriptsize{(0.395)}} \\  

     & \mc{1}{c}{\scriptsize{7.5}} & \mc{1}{c}{\scriptsize{4.133}} & \mc{1}{c}{\scriptsize{3.185}} & \mc{1}{c}{\scriptsize{4.300}} & \mc{1}{c}{\scriptsize{9.725}} & \mc{1}{c}{\scriptsize{4.578}} & \mc{1}{c}{\scriptsize{4.082}} & \mc{1}{c}{\scriptsize{1.201}} & \mc{1}{c}{\scriptsize{4.326}} \\  

     &  & \mc{1}{c}{\scriptsize{\textbf{(0.092)}}} & \mc{1}{c}{\scriptsize{\textbf{(0.092)}}} & \mc{1}{c}{\scriptsize{(0.171)}} & \mc{1}{c}{\scriptsize{\textbf{(0.013)}}} & \mc{1}{c}{\scriptsize{(0.171)}} & \mc{1}{c}{\scriptsize{(0.105)}} & \mc{1}{c}{\scriptsize{(0.329)}} & \mc{1}{c}{\scriptsize{(0.105)}} \\  

     & \mc{1}{c}{\scriptsize{7}} & \mc{1}{c}{\scriptsize{6.411}} & \mc{1}{c}{\scriptsize{7.268}} & \mc{1}{c}{\scriptsize{12.724}} & \mc{1}{c}{\scriptsize{12.736}} & \mc{1}{c}{\scriptsize{12.630}} & \mc{1}{c}{\scriptsize{5.415}} & \mc{1}{c}{\scriptsize{5.514}} & \mc{1}{c}{\scriptsize{6.478}} \\  

     &  & \mc{1}{c}{\scriptsize{\textbf{(0.013)}}} & \mc{1}{c}{\scriptsize{\textbf{(0.013)}}} & \mc{1}{c}{\scriptsize{\textbf{(0.000)}}} & \mc{1}{c}{\scriptsize{\textbf{(0.026)}}} & \mc{1}{c}{\scriptsize{\textbf{(0.000)}}} & \mc{1}{c}{\scriptsize{\textbf{(0.013)}}} & \mc{1}{c}{\scriptsize{\textbf{(0.026)}}} & \mc{1}{c}{\scriptsize{\textbf{(0.000)}}} \\  

    \mc{1}{l}{\scriptsize{PIAT Math Std. Score}} & \mc{1}{c}{\scriptsize{7}} & \mc{1}{c}{\scriptsize{2.182}} & \mc{1}{c}{\scriptsize{3.014}} & \mc{1}{c}{\scriptsize{8.773}} & \mc{1}{c}{\scriptsize{6.930}} & \mc{1}{c}{\scriptsize{8.259}} & \mc{1}{c}{\scriptsize{1.141}} & \mc{1}{c}{\scriptsize{1.003}} & \mc{1}{c}{\scriptsize{1.545}} \\  

     &  & \mc{1}{c}{\scriptsize{(0.224)}} & \mc{1}{c}{\scriptsize{(0.211)}} & \mc{1}{c}{\scriptsize{\textbf{(0.000)}}} & \mc{1}{c}{\scriptsize{\textbf{(0.039)}}} & \mc{1}{c}{\scriptsize{\textbf{(0.000)}}} & \mc{1}{c}{\scriptsize{(0.289)}} & \mc{1}{c}{\scriptsize{(0.382)}} & \mc{1}{c}{\scriptsize{(0.276)}} \\  

    \mc{1}{l}{\scriptsize{Std. Achv.  Test}} & \mc{1}{c}{\scriptsize{5.5}} & \mc{1}{c}{\scriptsize{12.314}} & \mc{1}{c}{\scriptsize{9.641}} & \mc{1}{c}{\scriptsize{19.650}} & \mc{1}{c}{\scriptsize{16.848}} & \mc{1}{c}{\scriptsize{18.482}} & \mc{1}{c}{\scriptsize{9.869}} & \mc{1}{c}{\scriptsize{3.091}} & \mc{1}{c}{\scriptsize{11.009}} \\  

     &  & \mc{1}{c}{\scriptsize{\textbf{(0.000)}}} & \mc{1}{c}{\scriptsize{\textbf{(0.000)}}} & \mc{1}{c}{\scriptsize{\textbf{(0.000)}}} & \mc{1}{c}{\scriptsize{\textbf{(0.000)}}} & \mc{1}{c}{\scriptsize{\textbf{(0.000)}}} & \mc{1}{c}{\scriptsize{\textbf{(0.026)}}} & \mc{1}{c}{\scriptsize{(0.303)}} & \mc{1}{c}{\scriptsize{\textbf{(0.000)}}} \\  

     & \mc{1}{c}{\scriptsize{8.5}} & \mc{1}{c}{\scriptsize{8.407}} & \mc{1}{c}{\scriptsize{7.607}} & \mc{1}{c}{\scriptsize{12.299}} & \mc{1}{c}{\scriptsize{15.146}} & \mc{1}{c}{\scriptsize{12.625}} & \mc{1}{c}{\scriptsize{7.223}} & \mc{1}{c}{\scriptsize{5.279}} & \mc{1}{c}{\scriptsize{7.918}} \\  

     &  & \mc{1}{c}{\scriptsize{\textbf{(0.000)}}} & \mc{1}{c}{\scriptsize{\textbf{(0.000)}}} & \mc{1}{c}{\scriptsize{\textbf{(0.000)}}} & \mc{1}{c}{\scriptsize{\textbf{(0.000)}}} & \mc{1}{c}{\scriptsize{\textbf{(0.000)}}} & \mc{1}{c}{\scriptsize{\textbf{(0.000)}}} & \mc{1}{c}{\scriptsize{\textbf{(0.066)}}} & \mc{1}{c}{\scriptsize{\textbf{(0.000)}}} \\  

     & \mc{1}{c}{\scriptsize{6}} & \mc{1}{c}{\scriptsize{6.269}} & \mc{1}{c}{\scriptsize{9.201}} & \mc{1}{c}{\scriptsize{10.379}} & \mc{1}{c}{\scriptsize{21.540}} & \mc{1}{c}{\scriptsize{13.499}} & \mc{1}{c}{\scriptsize{5.018}} & \mc{1}{c}{\scriptsize{5.994}} & \mc{1}{c}{\scriptsize{7.474}} \\  

     &  & \mc{1}{c}{\scriptsize{\textbf{(0.000)}}} & \mc{1}{c}{\scriptsize{\textbf{(0.026)}}} & \mc{1}{c}{\scriptsize{\textbf{(0.000)}}} & \mc{1}{c}{\scriptsize{\textbf{(0.000)}}} & \mc{1}{c}{\scriptsize{\textbf{(0.000)}}} & \mc{1}{c}{\scriptsize{\textbf{(0.013)}}} & \mc{1}{c}{\scriptsize{\textbf{(0.053)}}} & \mc{1}{c}{\scriptsize{\textbf{(0.066)}}} \\  

  \bottomrule
  \end{tabular}
	\end{table} 

	\begin{table}[H]
     \caption{Treatment Effects on Crime, Female Sample}
     \label{table:abccare_rslt_female_cat26}
	  \begin{tabular}{cccccccccc}
  \toprule

    \scriptsize{Variable} & \scriptsize{Age} & \scriptsize{(1)} & \scriptsize{(2)} & \scriptsize{(3)} & \scriptsize{(4)} & \scriptsize{(5)} & \scriptsize{(6)} & \scriptsize{(7)} & \scriptsize{(8)} \\ 
    \midrule  

    \mc{1}{l}{\scriptsize{Total Misdemeanor Arrests}} & \mc{1}{c}{\scriptsize{Mid-30s}} & \mc{1}{c}{\scriptsize{-0.973}} & \mc{1}{c}{\scriptsize{-0.883}} & \mc{1}{c}{\scriptsize{-2.708}} & \mc{1}{c}{\scriptsize{-2.177}} & \mc{1}{c}{\scriptsize{-2.453}} & \mc{1}{c}{\scriptsize{-0.588}} & \mc{1}{c}{\scriptsize{-0.142}} & \mc{1}{c}{\scriptsize{-0.200}} \\  

     &  & \mc{1}{c}{\scriptsize{\textbf{(0.039)}}} & \mc{1}{c}{\scriptsize{(0.145)}} & \mc{1}{c}{\scriptsize{\textbf{(0.079)}}} & \mc{1}{c}{\scriptsize{(0.184)}} & \mc{1}{c}{\scriptsize{(0.105)}} & \mc{1}{c}{\scriptsize{\textbf{(0.092)}}} & \mc{1}{c}{\scriptsize{(0.329)}} & \mc{1}{c}{\scriptsize{(0.250)}} \\  

    \mc{1}{l}{\scriptsize{Total Felony Arrests}} & \mc{1}{c}{\scriptsize{Mid-30s}} & \mc{1}{c}{\scriptsize{-0.328}} & \mc{1}{c}{\scriptsize{-0.390}} & \mc{1}{c}{\scriptsize{-1.345}} & \mc{1}{c}{\scriptsize{-1.011}} & \mc{1}{c}{\scriptsize{-0.968}} & \mc{1}{c}{\scriptsize{-0.077}} & \mc{1}{c}{\scriptsize{-0.022}} & \mc{1}{c}{\scriptsize{0.005}} \\  

     &  & \mc{1}{c}{\scriptsize{(0.105)}} & \mc{1}{c}{\scriptsize{(0.145)}} & \mc{1}{c}{\scriptsize{(0.105)}} & \mc{1}{c}{\scriptsize{(0.171)}} & \mc{1}{c}{\scriptsize{(0.118)}} & \mc{1}{c}{\scriptsize{(0.237)}} & \mc{1}{c}{\scriptsize{(0.395)}} & \mc{1}{c}{\scriptsize{(0.500)}} \\  

    \mc{1}{l}{\scriptsize{Total Years Incarcerated}} & \mc{1}{c}{\scriptsize{30}} & \mc{1}{c}{\scriptsize{-0.024}} & \mc{1}{c}{\scriptsize{-0.025}} &  &  &  & \mc{1}{c}{\scriptsize{-0.037}} & \mc{1}{c}{\scriptsize{-0.032}} & \mc{1}{c}{\scriptsize{-0.038}} \\  

     &  & \mc{1}{c}{\scriptsize{\textbf{(0.053)}}} & \mc{1}{c}{\scriptsize{\textbf{(0.066)}}} &  &  &  & \mc{1}{c}{\scriptsize{\textbf{(0.039)}}} & \mc{1}{c}{\scriptsize{\textbf{(0.039)}}} & \mc{1}{c}{\scriptsize{\textbf{(0.053)}}} \\  

  \bottomrule
  \end{tabular}
	\end{table} 

	\begin{table}[H]
     \caption{Treatment Effects on Childhood and Adolescence Physical Health, Female Sample}
     \label{table:abccare_rslt_female_cat27}
	\input{AppResOutput/abccare/rslt_female_cat27}
	\end{table} 

	\begin{table}[H]
     \caption{Treatment Effects on Childhood Health Problems, Female Sample}
     \label{table:abccare_rslt_female_cat28}
	  \begin{tabular}{cccccccccc}
  \toprule

    \scriptsize{Variable} & \scriptsize{Age} & \scriptsize{(1)} & \scriptsize{(2)} & \scriptsize{(3)} & \scriptsize{(4)} & \scriptsize{(5)} & \scriptsize{(6)} & \scriptsize{(7)} & \scriptsize{(8)} \\ 
    \midrule  

    \mc{1}{l}{\scriptsize{Has Health Problems}} & \mc{1}{c}{\scriptsize{12}} & \mc{1}{c}{\scriptsize{0.195}} & \mc{1}{c}{\scriptsize{0.076}} & \mc{1}{c}{\scriptsize{0.167}} & \mc{1}{c}{\scriptsize{0.125}} & \mc{1}{c}{\scriptsize{0.113}} & \mc{1}{c}{\scriptsize{0.197}} & \mc{1}{c}{\scriptsize{0.097}} & \mc{1}{c}{\scriptsize{0.162}} \\  

     &  & \mc{1}{c}{\scriptsize{(0.789)}} & \mc{1}{c}{\scriptsize{(0.645)}} & \mc{1}{c}{\scriptsize{(0.763)}} & \mc{1}{c}{\scriptsize{(0.763)}} & \mc{1}{c}{\scriptsize{(0.671)}} & \mc{1}{c}{\scriptsize{(0.803)}} & \mc{1}{c}{\scriptsize{(0.671)}} & \mc{1}{c}{\scriptsize{(0.816)}} \\  

    \mc{1}{l}{\scriptsize{Ever Hospitalized for Over 1 Week}} & \mc{1}{c}{\scriptsize{12}} & \mc{1}{c}{\scriptsize{0.042}} & \mc{1}{c}{\scriptsize{0.070}} & \mc{1}{c}{\scriptsize{0.111}} & \mc{1}{c}{\scriptsize{0.138}} & \mc{1}{c}{\scriptsize{0.084}} & \mc{1}{c}{\scriptsize{0.020}} & \mc{1}{c}{\scriptsize{0.060}} & \mc{1}{c}{\scriptsize{0.001}} \\  

     &  & \mc{1}{c}{\scriptsize{(0.697)}} & \mc{1}{c}{\scriptsize{(0.776)}} & \mc{1}{c}{\scriptsize{(0.908)}} & \mc{1}{c}{\scriptsize{(0.776)}} & \mc{1}{c}{\scriptsize{(0.842)}} & \mc{1}{c}{\scriptsize{(0.421)}} & \mc{1}{c}{\scriptsize{(0.592)}} & \mc{1}{c}{\scriptsize{(0.382)}} \\  

  \bottomrule
  \end{tabular}
	\end{table} 

	\begin{table}[H]
     \caption{Treatment Effects on Cholesterol, Female Sample}
     \label{table:abccare_rslt_female_cat29}
	\input{AppResOutput/abccare/rslt_female_cat29}
	\end{table} 

	\begin{table}[H]
     \caption{Treatment Effects on Current Health Condition (Self-Reported), Female Sample}
     \label{table:abccare_rslt_female_cat30}
	  \begin{tabular}{cccccccccc}
  \toprule

    \scriptsize{Variable} & \scriptsize{Age} & \scriptsize{(1)} & \scriptsize{(2)} & \scriptsize{(3)} & \scriptsize{(4)} & \scriptsize{(5)} & \scriptsize{(6)} & \scriptsize{(7)} & \scriptsize{(8)} \\ 
    \midrule  

    \mc{1}{l}{\scriptsize{Asthma}} & \mc{1}{c}{\scriptsize{Mid-30s}} & \mc{1}{c}{\scriptsize{0.043}} &  & \mc{1}{c}{\scriptsize{0.043}} &  &  & \mc{1}{c}{\scriptsize{0.043}} &  &  \\  

     &  & \mc{1}{c}{\scriptsize{(0.658)}} &  & \mc{1}{c}{\scriptsize{(0.658)}} &  &  & \mc{1}{c}{\scriptsize{(0.539)}} &  &  \\  

    \mc{1}{l}{\scriptsize{High Blood Pressure (Hypertension)}} & \mc{1}{c}{\scriptsize{Mid-30s}} & \mc{1}{c}{\scriptsize{-0.036}} & \mc{1}{c}{\scriptsize{-0.060}} &  &  &  & \mc{1}{c}{\scriptsize{-0.045}} & \mc{1}{c}{\scriptsize{-0.078}} & \mc{1}{c}{\scriptsize{-0.053}} \\  

     &  & \mc{1}{c}{\scriptsize{\textbf{(0.092)}}} & \mc{1}{c}{\scriptsize{(0.118)}} &  &  &  & \mc{1}{c}{\scriptsize{\textbf{(0.079)}}} & \mc{1}{c}{\scriptsize{\textbf{(0.066)}}} & \mc{1}{c}{\scriptsize{\textbf{(0.066)}}} \\  

    \mc{1}{l}{\scriptsize{Arthritis or Generative Disease}} & \mc{1}{c}{\scriptsize{Mid-30s}} & \mc{1}{c}{\scriptsize{0.043}} & \mc{1}{c}{\scriptsize{0.051}} & \mc{1}{c}{\scriptsize{0.043}} & \mc{1}{c}{\scriptsize{0.029}} & \mc{1}{c}{\scriptsize{0.046}} & \mc{1}{c}{\scriptsize{0.043}} & \mc{1}{c}{\scriptsize{0.057}} & \mc{1}{c}{\scriptsize{0.046}} \\  

     &  & \mc{1}{c}{\scriptsize{(0.474)}} & \mc{1}{c}{\scriptsize{(0.461)}} & \mc{1}{c}{\scriptsize{(0.474)}} & \mc{1}{c}{\scriptsize{(0.342)}} & \mc{1}{c}{\scriptsize{(0.474)}} & \mc{1}{c}{\scriptsize{(0.421)}} & \mc{1}{c}{\scriptsize{(0.408)}} & \mc{1}{c}{\scriptsize{(0.421)}} \\  

    \mc{1}{l}{\scriptsize{Diabetes}} & \mc{1}{c}{\scriptsize{Mid-30s}} &  &  &  &  &  &  &  &  \\  

     &  &  &  &  &  &  &  &  &  \\  

  \bottomrule
  \end{tabular}
	\end{table} 

	\begin{table}[H]
     \caption{Treatment Effects on Diabetes, Female Sample}
     \label{table:abccare_rslt_female_cat31}
	\input{AppResOutput/abccare/rslt_female_cat31}
	\end{table} 

	\begin{table}[H]
     \caption{Treatment Effects on Drug Behavior and ASR Substance Scale, Female Sample}
     \label{table:abccare_rslt_female_cat32}
	\input{AppResOutput/abccare/rslt_female_cat32}
	\end{table} 

	\begin{table}[H]
     \caption{Treatment Effects on Health Insurance, Female Sample}
     \label{table:abccare_rslt_female_cat33}
	\input{AppResOutput/abccare/rslt_female_cat33}
	\end{table} 

	\begin{table}[H]
     \caption{Treatment Effects on Hypertension, Female Sample}
     \label{table:abccare_rslt_female_cat34}
	\input{AppResOutput/abccare/rslt_female_cat34}
	\end{table} 

	\begin{table}[H]
     \caption{Treatment Effects on Laboratory Test  - Metabolic Panel, Female Sample}
     \label{table:abccare_rslt_female_cat35}
	  \begin{tabular}{cccccccccc}
  \toprule

    \scriptsize{Variable} & \scriptsize{Age} & \scriptsize{(1)} & \scriptsize{(2)} & \scriptsize{(3)} & \scriptsize{(4)} & \scriptsize{(5)} & \scriptsize{(6)} & \scriptsize{(7)} & \scriptsize{(8)} \\ 
    \midrule  

    \mc{1}{l}{\scriptsize{Albumin/Globulin Ratio}} & \mc{1}{c}{\scriptsize{Mid-30s}} & \mc{1}{c}{\scriptsize{-0.015}} & \mc{1}{c}{\scriptsize{-0.033}} & \mc{1}{c}{\scriptsize{0.024}} & \mc{1}{c}{\scriptsize{0.066}} & \mc{1}{c}{\scriptsize{0.022}} & \mc{1}{c}{\scriptsize{-0.026}} & \mc{1}{c}{\scriptsize{-0.067}} & \mc{1}{c}{\scriptsize{-0.052}} \\  

     &  & \mc{1}{c}{\scriptsize{(0.579)}} & \mc{1}{c}{\scriptsize{(0.711)}} & \mc{1}{c}{\scriptsize{(0.355)}} & \mc{1}{c}{\scriptsize{(0.289)}} & \mc{1}{c}{\scriptsize{(0.368)}} & \mc{1}{c}{\scriptsize{(0.645)}} & \mc{1}{c}{\scriptsize{(0.829)}} & \mc{1}{c}{\scriptsize{(0.697)}} \\  

    \mc{1}{l}{\scriptsize{ALT}} & \mc{1}{c}{\scriptsize{Mid-30s}} & \mc{1}{c}{\scriptsize{-3.537}} & \mc{1}{c}{\scriptsize{-0.029}} & \mc{1}{c}{\scriptsize{-1.775}} & \mc{1}{c}{\scriptsize{-2.208}} & \mc{1}{c}{\scriptsize{-1.827}} & \mc{1}{c}{\scriptsize{-4.018}} & \mc{1}{c}{\scriptsize{0.458}} & \mc{1}{c}{\scriptsize{-2.606}} \\  

     &  & \mc{1}{c}{\scriptsize{(0.882)}} & \mc{1}{c}{\scriptsize{(0.500)}} & \mc{1}{c}{\scriptsize{(0.789)}} & \mc{1}{c}{\scriptsize{(0.829)}} & \mc{1}{c}{\scriptsize{(0.829)}} & \mc{1}{c}{\scriptsize{(0.868)}} & \mc{1}{c}{\scriptsize{(0.447)}} & \mc{1}{c}{\scriptsize{(0.895)}} \\  

    \mc{1}{l}{\scriptsize{Albumin}} & \mc{1}{c}{\scriptsize{Mid-30s}} & \mc{1}{c}{\scriptsize{-0.054}} & \mc{1}{c}{\scriptsize{-0.140}} & \mc{1}{c}{\scriptsize{0.079}} & \mc{1}{c}{\scriptsize{0.047}} & \mc{1}{c}{\scriptsize{0.066}} & \mc{1}{c}{\scriptsize{-0.091}} & \mc{1}{c}{\scriptsize{-0.208}} & \mc{1}{c}{\scriptsize{-0.156}} \\  

     &  & \mc{1}{c}{\scriptsize{(0.763)}} & \mc{1}{c}{\scriptsize{(0.987)}} & \mc{1}{c}{\scriptsize{(0.211)}} & \mc{1}{c}{\scriptsize{(0.355)}} & \mc{1}{c}{\scriptsize{(0.197)}} & \mc{1}{c}{\scriptsize{(0.855)}} & \mc{1}{c}{\scriptsize{(0.987)}} & \mc{1}{c}{\scriptsize{(0.987)}} \\  

    \mc{1}{l}{\scriptsize{Sodium}} & \mc{1}{c}{\scriptsize{Mid-30s}} & \mc{1}{c}{\scriptsize{0.424}} & \mc{1}{c}{\scriptsize{1.060}} & \mc{1}{c}{\scriptsize{0.507}} & \mc{1}{c}{\scriptsize{1.228}} & \mc{1}{c}{\scriptsize{0.660}} & \mc{1}{c}{\scriptsize{0.401}} & \mc{1}{c}{\scriptsize{1.077}} & \mc{1}{c}{\scriptsize{0.489}} \\  

     &  & \mc{1}{c}{\scriptsize{(0.211)}} & \mc{1}{c}{\scriptsize{\textbf{(0.092)}}} & \mc{1}{c}{\scriptsize{(0.197)}} & \mc{1}{c}{\scriptsize{\textbf{(0.000)}}} & \mc{1}{c}{\scriptsize{(0.184)}} & \mc{1}{c}{\scriptsize{(0.250)}} & \mc{1}{c}{\scriptsize{(0.118)}} & \mc{1}{c}{\scriptsize{(0.197)}} \\  

    \mc{1}{l}{\scriptsize{Carbon Dioxide}} & \mc{1}{c}{\scriptsize{Mid-30s}} & \mc{1}{c}{\scriptsize{0.581}} & \mc{1}{c}{\scriptsize{0.381}} & \mc{1}{c}{\scriptsize{-0.015}} & \mc{1}{c}{\scriptsize{-0.558}} & \mc{1}{c}{\scriptsize{-0.297}} & \mc{1}{c}{\scriptsize{0.743}} & \mc{1}{c}{\scriptsize{0.880}} & \mc{1}{c}{\scriptsize{0.717}} \\  

     &  & \mc{1}{c}{\scriptsize{(0.171)}} & \mc{1}{c}{\scriptsize{(0.224)}} & \mc{1}{c}{\scriptsize{(0.539)}} & \mc{1}{c}{\scriptsize{(0.789)}} & \mc{1}{c}{\scriptsize{(0.697)}} & \mc{1}{c}{\scriptsize{(0.132)}} & \mc{1}{c}{\scriptsize{\textbf{(0.092)}}} & \mc{1}{c}{\scriptsize{(0.158)}} \\  

    \mc{1}{l}{\scriptsize{AST}} & \mc{1}{c}{\scriptsize{Mid-30s}} & \mc{1}{c}{\scriptsize{-5.120}} & \mc{1}{c}{\scriptsize{-0.860}} & \mc{1}{c}{\scriptsize{-1.536}} & \mc{1}{c}{\scriptsize{-1.926}} & \mc{1}{c}{\scriptsize{-1.272}} & \mc{1}{c}{\scriptsize{-6.097}} & \mc{1}{c}{\scriptsize{-0.527}} & \mc{1}{c}{\scriptsize{-3.584}} \\  

     &  & \mc{1}{c}{\scriptsize{(0.855)}} & \mc{1}{c}{\scriptsize{(0.684)}} & \mc{1}{c}{\scriptsize{(0.842)}} & \mc{1}{c}{\scriptsize{(0.829)}} & \mc{1}{c}{\scriptsize{(0.829)}} & \mc{1}{c}{\scriptsize{(0.842)}} & \mc{1}{c}{\scriptsize{(0.671)}} & \mc{1}{c}{\scriptsize{(0.895)}} \\  

    \mc{1}{l}{\scriptsize{Urea Nitrogen}} & \mc{1}{c}{\scriptsize{Mid-30s}} & \mc{1}{c}{\scriptsize{-0.222}} & \mc{1}{c}{\scriptsize{-0.750}} & \mc{1}{c}{\scriptsize{-0.877}} & \mc{1}{c}{\scriptsize{-1.593}} & \mc{1}{c}{\scriptsize{-0.822}} & \mc{1}{c}{\scriptsize{-0.043}} & \mc{1}{c}{\scriptsize{-0.720}} & \mc{1}{c}{\scriptsize{-0.183}} \\  

     &  & \mc{1}{c}{\scriptsize{(0.658)}} & \mc{1}{c}{\scriptsize{(0.803)}} & \mc{1}{c}{\scriptsize{(0.737)}} & \mc{1}{c}{\scriptsize{(0.763)}} & \mc{1}{c}{\scriptsize{(0.697)}} & \mc{1}{c}{\scriptsize{(0.526)}} & \mc{1}{c}{\scriptsize{(0.750)}} & \mc{1}{c}{\scriptsize{(0.592)}} \\  

    \mc{1}{l}{\scriptsize{Globulin}} & \mc{1}{c}{\scriptsize{Mid-30s}} & \mc{1}{c}{\scriptsize{0.043}} & \mc{1}{c}{\scriptsize{0.015}} & \mc{1}{c}{\scriptsize{0.060}} & \mc{1}{c}{\scriptsize{-0.039}} & \mc{1}{c}{\scriptsize{0.059}} & \mc{1}{c}{\scriptsize{0.039}} & \mc{1}{c}{\scriptsize{0.041}} & \mc{1}{c}{\scriptsize{0.048}} \\  

     &  & \mc{1}{c}{\scriptsize{(0.329)}} & \mc{1}{c}{\scriptsize{(0.474)}} & \mc{1}{c}{\scriptsize{(0.395)}} & \mc{1}{c}{\scriptsize{(0.539)}} & \mc{1}{c}{\scriptsize{(0.408)}} & \mc{1}{c}{\scriptsize{(0.329)}} & \mc{1}{c}{\scriptsize{(0.368)}} & \mc{1}{c}{\scriptsize{(0.368)}} \\  

    \mc{1}{l}{\scriptsize{Chloride}} & \mc{1}{c}{\scriptsize{Mid-30s}} & \mc{1}{c}{\scriptsize{-0.481}} & \mc{1}{c}{\scriptsize{0.049}} & \mc{1}{c}{\scriptsize{-1.362}} & \mc{1}{c}{\scriptsize{-0.151}} & \mc{1}{c}{\scriptsize{-1.292}} & \mc{1}{c}{\scriptsize{-0.241}} & \mc{1}{c}{\scriptsize{0.191}} & \mc{1}{c}{\scriptsize{-0.182}} \\  

     &  & \mc{1}{c}{\scriptsize{(0.776)}} & \mc{1}{c}{\scriptsize{(0.461)}} & \mc{1}{c}{\scriptsize{(0.961)}} & \mc{1}{c}{\scriptsize{(0.579)}} & \mc{1}{c}{\scriptsize{(0.947)}} & \mc{1}{c}{\scriptsize{(0.632)}} & \mc{1}{c}{\scriptsize{(0.382)}} & \mc{1}{c}{\scriptsize{(0.632)}} \\  

    \mc{1}{l}{\scriptsize{Glucose}} & \mc{1}{c}{\scriptsize{Mid-30s}} & \mc{1}{c}{\scriptsize{-11.975}} & \mc{1}{c}{\scriptsize{-4.817}} & \mc{1}{c}{\scriptsize{1.906}} & \mc{1}{c}{\scriptsize{8.421}} & \mc{1}{c}{\scriptsize{1.080}} & \mc{1}{c}{\scriptsize{-15.761}} & \mc{1}{c}{\scriptsize{-5.574}} & \mc{1}{c}{\scriptsize{-18.114}} \\  

     &  & \mc{1}{c}{\scriptsize{(0.763)}} & \mc{1}{c}{\scriptsize{(0.737)}} & \mc{1}{c}{\scriptsize{(0.368)}} & \mc{1}{c}{\scriptsize{(0.303)}} & \mc{1}{c}{\scriptsize{(0.461)}} & \mc{1}{c}{\scriptsize{(0.763)}} & \mc{1}{c}{\scriptsize{(0.737)}} & \mc{1}{c}{\scriptsize{(0.763)}} \\  

    \mc{1}{l}{\scriptsize{Potassium}} & \mc{1}{c}{\scriptsize{Mid-30s}} & \mc{1}{c}{\scriptsize{0.028}} & \mc{1}{c}{\scriptsize{0.040}} & \mc{1}{c}{\scriptsize{0.029}} & \mc{1}{c}{\scriptsize{-0.007}} & \mc{1}{c}{\scriptsize{0.065}} & \mc{1}{c}{\scriptsize{0.028}} & \mc{1}{c}{\scriptsize{0.048}} & \mc{1}{c}{\scriptsize{0.103}} \\  

     &  & \mc{1}{c}{\scriptsize{(0.382)}} & \mc{1}{c}{\scriptsize{(0.421)}} & \mc{1}{c}{\scriptsize{(0.461)}} & \mc{1}{c}{\scriptsize{(0.579)}} & \mc{1}{c}{\scriptsize{(0.382)}} & \mc{1}{c}{\scriptsize{(0.355)}} & \mc{1}{c}{\scriptsize{(0.355)}} & \mc{1}{c}{\scriptsize{(0.145)}} \\  

    \mc{1}{l}{\scriptsize{Creatinine}} & \mc{1}{c}{\scriptsize{Mid-30s}} & \mc{1}{c}{\scriptsize{-0.026}} & \mc{1}{c}{\scriptsize{-0.063}} & \mc{1}{c}{\scriptsize{0.025}} & \mc{1}{c}{\scriptsize{-0.018}} & \mc{1}{c}{\scriptsize{0.004}} & \mc{1}{c}{\scriptsize{-0.040}} & \mc{1}{c}{\scriptsize{-0.080}} & \mc{1}{c}{\scriptsize{-0.073}} \\  

     &  & \mc{1}{c}{\scriptsize{(0.816)}} & \mc{1}{c}{\scriptsize{(0.908)}} & \mc{1}{c}{\scriptsize{(0.263)}} & \mc{1}{c}{\scriptsize{(0.632)}} & \mc{1}{c}{\scriptsize{(0.500)}} & \mc{1}{c}{\scriptsize{(0.868)}} & \mc{1}{c}{\scriptsize{(0.921)}} & \mc{1}{c}{\scriptsize{(0.855)}} \\  

    \mc{1}{l}{\scriptsize{Calcium}} & \mc{1}{c}{\scriptsize{Mid-30s}} & \mc{1}{c}{\scriptsize{-0.115}} & \mc{1}{c}{\scriptsize{-0.239}} & \mc{1}{c}{\scriptsize{0.095}} & \mc{1}{c}{\scriptsize{0.048}} & \mc{1}{c}{\scriptsize{0.062}} & \mc{1}{c}{\scriptsize{-0.172}} & \mc{1}{c}{\scriptsize{-0.332}} & \mc{1}{c}{\scriptsize{-0.264}} \\  

     &  & \mc{1}{c}{\scriptsize{(0.895)}} & \mc{1}{c}{\scriptsize{(1.000)}} & \mc{1}{c}{\scriptsize{(0.289)}} & \mc{1}{c}{\scriptsize{(0.342)}} & \mc{1}{c}{\scriptsize{(0.368)}} & \mc{1}{c}{\scriptsize{(0.961)}} & \mc{1}{c}{\scriptsize{(1.000)}} & \mc{1}{c}{\scriptsize{(0.987)}} \\  

    \mc{1}{l}{\scriptsize{Bilirubin}} & \mc{1}{c}{\scriptsize{Mid-30s}} & \mc{1}{c}{\scriptsize{0.020}} & \mc{1}{c}{\scriptsize{0.028}} & \mc{1}{c}{\scriptsize{0.069}} & \mc{1}{c}{\scriptsize{0.063}} & \mc{1}{c}{\scriptsize{0.086}} & \mc{1}{c}{\scriptsize{0.007}} & \mc{1}{c}{\scriptsize{0.012}} & \mc{1}{c}{\scriptsize{0.034}} \\  

     &  & \mc{1}{c}{\scriptsize{(0.355)}} & \mc{1}{c}{\scriptsize{(0.303)}} & \mc{1}{c}{\scriptsize{\textbf{(0.053)}}} & \mc{1}{c}{\scriptsize{(0.197)}} & \mc{1}{c}{\scriptsize{\textbf{(0.053)}}} & \mc{1}{c}{\scriptsize{(0.434)}} & \mc{1}{c}{\scriptsize{(0.421)}} & \mc{1}{c}{\scriptsize{(0.276)}} \\  

    \mc{1}{l}{\scriptsize{Alkaline Phosp}} & \mc{1}{c}{\scriptsize{Mid-30s}} & \mc{1}{c}{\scriptsize{-11.691}} & \mc{1}{c}{\scriptsize{-4.863}} & \mc{1}{c}{\scriptsize{-18.036}} & \mc{1}{c}{\scriptsize{-16.604}} & \mc{1}{c}{\scriptsize{-18.453}} & \mc{1}{c}{\scriptsize{-9.960}} & \mc{1}{c}{\scriptsize{-0.224}} & \mc{1}{c}{\scriptsize{-8.312}} \\  

     &  & \mc{1}{c}{\scriptsize{(0.947)}} & \mc{1}{c}{\scriptsize{(0.737)}} & \mc{1}{c}{\scriptsize{(0.961)}} & \mc{1}{c}{\scriptsize{(0.855)}} & \mc{1}{c}{\scriptsize{(0.947)}} & \mc{1}{c}{\scriptsize{(0.908)}} & \mc{1}{c}{\scriptsize{(0.526)}} & \mc{1}{c}{\scriptsize{(0.803)}} \\  

    \mc{1}{l}{\scriptsize{Protein}} & \mc{1}{c}{\scriptsize{Mid-30s}} & \mc{1}{c}{\scriptsize{-0.011}} & \mc{1}{c}{\scriptsize{-0.124}} & \mc{1}{c}{\scriptsize{0.139}} & \mc{1}{c}{\scriptsize{0.008}} & \mc{1}{c}{\scriptsize{0.125}} & \mc{1}{c}{\scriptsize{-0.052}} & \mc{1}{c}{\scriptsize{-0.168}} & \mc{1}{c}{\scriptsize{-0.109}} \\  

     &  & \mc{1}{c}{\scriptsize{(0.461)}} & \mc{1}{c}{\scriptsize{(0.842)}} & \mc{1}{c}{\scriptsize{(0.276)}} & \mc{1}{c}{\scriptsize{(0.461)}} & \mc{1}{c}{\scriptsize{(0.276)}} & \mc{1}{c}{\scriptsize{(0.684)}} & \mc{1}{c}{\scriptsize{(0.895)}} & \mc{1}{c}{\scriptsize{(0.842)}} \\  

  \bottomrule
  \end{tabular}
	\end{table} 

	\begin{table}[H]
     \caption{Treatment Effects on Laboratory Test - Complete Blood Count, Female Sample}
     \label{table:abccare_rslt_female_cat36}
	\input{AppResOutput/abccare/rslt_female_cat36}
	\end{table} 

	\begin{table}[H]
     \caption{Treatment Effects on Other Health-Related Information, Female Sample}
     \label{table:abccare_rslt_female_cat37}
	  \begin{tabular}{cccccccccc}
  \toprule

    \scriptsize{Variable} & \scriptsize{Age} & \scriptsize{(1)} & \scriptsize{(2)} & \scriptsize{(3)} & \scriptsize{(4)} & \scriptsize{(5)} & \scriptsize{(6)} & \scriptsize{(7)} & \scriptsize{(8)} \\ 
    \midrule  

    \mc{1}{l}{\scriptsize{Global Severity Index}} & \mc{1}{c}{\scriptsize{Mid-30s}} & \mc{1}{c}{\scriptsize{-2.365}} & \mc{1}{c}{\scriptsize{-1.549}} & \mc{1}{c}{\scriptsize{0.290}} & \mc{1}{c}{\scriptsize{-1.464}} & \mc{1}{c}{\scriptsize{0.341}} & \mc{1}{c}{\scriptsize{-3.089}} & \mc{1}{c}{\scriptsize{-2.815}} & \mc{1}{c}{\scriptsize{-3.115}} \\  

     &  & \mc{1}{c}{\scriptsize{(0.263)}} & \mc{1}{c}{\scriptsize{(0.276)}} & \mc{1}{c}{\scriptsize{(0.434)}} & \mc{1}{c}{\scriptsize{(0.342)}} & \mc{1}{c}{\scriptsize{(0.487)}} & \mc{1}{c}{\scriptsize{(0.105)}} & \mc{1}{c}{\scriptsize{(0.224)}} & \mc{1}{c}{\scriptsize{(0.145)}} \\  

    \mc{1}{l}{\scriptsize{Somatization}} & \mc{1}{c}{\scriptsize{Mid-30s}} & \mc{1}{c}{\scriptsize{0.536}} & \mc{1}{c}{\scriptsize{1.847}} & \mc{1}{c}{\scriptsize{-0.500}} & \mc{1}{c}{\scriptsize{-3.602}} & \mc{1}{c}{\scriptsize{-0.084}} & \mc{1}{c}{\scriptsize{0.818}} & \mc{1}{c}{\scriptsize{2.724}} & \mc{1}{c}{\scriptsize{1.421}} \\  

     &  & \mc{1}{c}{\scriptsize{(0.566)}} & \mc{1}{c}{\scriptsize{(0.816)}} & \mc{1}{c}{\scriptsize{(0.461)}} & \mc{1}{c}{\scriptsize{(0.276)}} & \mc{1}{c}{\scriptsize{(0.434)}} & \mc{1}{c}{\scriptsize{(0.566)}} & \mc{1}{c}{\scriptsize{(0.882)}} & \mc{1}{c}{\scriptsize{(0.658)}} \\  

    \mc{1}{l}{\scriptsize{Anxiety}} & \mc{1}{c}{\scriptsize{Mid-30s}} & \mc{1}{c}{\scriptsize{-3.908}} & \mc{1}{c}{\scriptsize{-3.792}} & \mc{1}{c}{\scriptsize{-3.920}} & \mc{1}{c}{\scriptsize{-6.126}} & \mc{1}{c}{\scriptsize{-4.420}} & \mc{1}{c}{\scriptsize{-3.905}} & \mc{1}{c}{\scriptsize{-3.738}} & \mc{1}{c}{\scriptsize{-4.725}} \\  

     &  & \mc{1}{c}{\scriptsize{(0.105)}} & \mc{1}{c}{\scriptsize{\textbf{(0.053)}}} & \mc{1}{c}{\scriptsize{(0.289)}} & \mc{1}{c}{\scriptsize{(0.171)}} & \mc{1}{c}{\scriptsize{(0.224)}} & \mc{1}{c}{\scriptsize{\textbf{(0.079)}}} & \mc{1}{c}{\scriptsize{\textbf{(0.092)}}} & \mc{1}{c}{\scriptsize{\textbf{(0.066)}}} \\  

    \mc{1}{l}{\scriptsize{Depression}} & \mc{1}{c}{\scriptsize{Mid-30s}} & \mc{1}{c}{\scriptsize{-2.006}} & \mc{1}{c}{\scriptsize{-2.318}} & \mc{1}{c}{\scriptsize{-0.935}} & \mc{1}{c}{\scriptsize{-1.206}} & \mc{1}{c}{\scriptsize{-1.224}} & \mc{1}{c}{\scriptsize{-2.298}} & \mc{1}{c}{\scriptsize{-3.436}} & \mc{1}{c}{\scriptsize{-2.403}} \\  

     &  & \mc{1}{c}{\scriptsize{(0.224)}} & \mc{1}{c}{\scriptsize{(0.145)}} & \mc{1}{c}{\scriptsize{(0.382)}} & \mc{1}{c}{\scriptsize{(0.355)}} & \mc{1}{c}{\scriptsize{(0.382)}} & \mc{1}{c}{\scriptsize{(0.158)}} & \mc{1}{c}{\scriptsize{\textbf{(0.079)}}} & \mc{1}{c}{\scriptsize{(0.184)}} \\  

  \bottomrule
  \end{tabular}
	\end{table} 

	\begin{table}[H]
     \caption{Treatment Effects on Past Medical History - Diagnosis (Self-Reported), Female Sample}
     \label{table:abccare_rslt_female_cat38}
	  \begin{tabular}{cccccccccc}
  \toprule

    \scriptsize{Variable} & \scriptsize{Age} & \scriptsize{(1)} & \scriptsize{(2)} & \scriptsize{(3)} & \scriptsize{(4)} & \scriptsize{(5)} & \scriptsize{(6)} & \scriptsize{(7)} & \scriptsize{(8)} \\ 
    \midrule  

    \mc{1}{l}{\scriptsize{Ever Told Had: Arthritis/Gout/Lupus/Fibromyalgia}} & \mc{1}{c}{\scriptsize{Mid-30s}} & \mc{1}{c}{\scriptsize{0.095}} & \mc{1}{c}{\scriptsize{0.092}} & \mc{1}{c}{\scriptsize{-0.036}} & \mc{1}{c}{\scriptsize{-0.082}} & \mc{1}{c}{\scriptsize{-0.066}} & \mc{1}{c}{\scriptsize{0.130}} & \mc{1}{c}{\scriptsize{0.140}} & \mc{1}{c}{\scriptsize{0.100}} \\  

     &  & \mc{1}{c}{\scriptsize{(0.895)}} & \mc{1}{c}{\scriptsize{(0.789)}} & \mc{1}{c}{\scriptsize{(0.434)}} & \mc{1}{c}{\scriptsize{(0.276)}} & \mc{1}{c}{\scriptsize{(0.342)}} & \mc{1}{c}{\scriptsize{(0.882)}} & \mc{1}{c}{\scriptsize{(0.776)}} & \mc{1}{c}{\scriptsize{(0.750)}} \\  

    \mc{1}{l}{\scriptsize{Ever Told Had: Prediabetes}} & \mc{1}{c}{\scriptsize{Mid-30s}} & \mc{1}{c}{\scriptsize{0.064}} & \mc{1}{c}{\scriptsize{0.088}} & \mc{1}{c}{\scriptsize{-0.127}} & \mc{1}{c}{\scriptsize{0.047}} & \mc{1}{c}{\scriptsize{-0.118}} & \mc{1}{c}{\scriptsize{0.115}} & \mc{1}{c}{\scriptsize{0.133}} & \mc{1}{c}{\scriptsize{0.098}} \\  

     &  & \mc{1}{c}{\scriptsize{(0.671)}} & \mc{1}{c}{\scriptsize{(0.763)}} & \mc{1}{c}{\scriptsize{(0.316)}} & \mc{1}{c}{\scriptsize{(0.618)}} & \mc{1}{c}{\scriptsize{(0.303)}} & \mc{1}{c}{\scriptsize{(0.842)}} & \mc{1}{c}{\scriptsize{(0.763)}} & \mc{1}{c}{\scriptsize{(0.737)}} \\  

  \bottomrule
  \end{tabular}
	\end{table} 

	\begin{table}[H]
     \caption{Treatment Effects on Past Medical History - Surgery (Self-Reported), Female Sample}
     \label{table:abccare_rslt_female_cat39}
	  \begin{tabular}{cccccccccc}
  \toprule

    \scriptsize{Variable} & \scriptsize{Age} & \scriptsize{(1)} & \scriptsize{(2)} & \scriptsize{(3)} & \scriptsize{(4)} & \scriptsize{(5)} & \scriptsize{(6)} & \scriptsize{(7)} & \scriptsize{(8)} \\ 
    \midrule  

    \mc{1}{l}{\scriptsize{Past Surgery: Cholecystectomy}} & \mc{1}{c}{\scriptsize{Mid-30s}} & \mc{1}{c}{\scriptsize{0.051}} & \mc{1}{c}{\scriptsize{0.014}} & \mc{1}{c}{\scriptsize{0.087}} & \mc{1}{c}{\scriptsize{0.087}} & \mc{1}{c}{\scriptsize{0.096}} & \mc{1}{c}{\scriptsize{0.042}} & \mc{1}{c}{\scriptsize{-0.002}} & \mc{1}{c}{\scriptsize{0.028}} \\  

     &  & \mc{1}{c}{\scriptsize{(0.711)}} & \mc{1}{c}{\scriptsize{(0.461)}} & \mc{1}{c}{\scriptsize{(0.763)}} & \mc{1}{c}{\scriptsize{(0.684)}} & \mc{1}{c}{\scriptsize{(0.763)}} & \mc{1}{c}{\scriptsize{(0.671)}} & \mc{1}{c}{\scriptsize{(0.434)}} & \mc{1}{c}{\scriptsize{(0.474)}} \\  

    \mc{1}{l}{\scriptsize{Past Surgery: Orthopedic Surgery}} & \mc{1}{c}{\scriptsize{Mid-30s}} &  &  &  &  &  &  &  &  \\  

     &  &  &  &  &  &  &  &  &  \\  

    \mc{1}{l}{\scriptsize{Past Surgery: Appendectomy}} & \mc{1}{c}{\scriptsize{Mid-30s}} & \mc{1}{c}{\scriptsize{-0.028}} & \mc{1}{c}{\scriptsize{0.012}} & \mc{1}{c}{\scriptsize{-0.123}} & \mc{1}{c}{\scriptsize{-0.021}} & \mc{1}{c}{\scriptsize{-0.113}} & \mc{1}{c}{\scriptsize{-0.002}} & \mc{1}{c}{\scriptsize{0.054}} & \mc{1}{c}{\scriptsize{0.002}} \\  

     &  & \mc{1}{c}{\scriptsize{(0.342)}} & \mc{1}{c}{\scriptsize{(0.592)}} & \mc{1}{c}{\scriptsize{(0.132)}} & \mc{1}{c}{\scriptsize{(0.368)}} & \mc{1}{c}{\scriptsize{(0.145)}} & \mc{1}{c}{\scriptsize{(0.408)}} & \mc{1}{c}{\scriptsize{(0.697)}} & \mc{1}{c}{\scriptsize{(0.461)}} \\  

    \mc{1}{l}{\scriptsize{Past Surgery: Ectopic Pregnancy}} & \mc{1}{c}{\scriptsize{Mid-30s}} & \mc{1}{c}{\scriptsize{0.008}} & \mc{1}{c}{\scriptsize{0.036}} & \mc{1}{c}{\scriptsize{0.043}} & \mc{1}{c}{\scriptsize{0.065}} & \mc{1}{c}{\scriptsize{0.051}} & \mc{1}{c}{\scriptsize{-0.002}} & \mc{1}{c}{\scriptsize{0.040}} & \mc{1}{c}{\scriptsize{0.002}} \\  

     &  & \mc{1}{c}{\scriptsize{(0.382)}} & \mc{1}{c}{\scriptsize{(0.605)}} & \mc{1}{c}{\scriptsize{(0.592)}} & \mc{1}{c}{\scriptsize{(0.500)}} & \mc{1}{c}{\scriptsize{(0.592)}} & \mc{1}{c}{\scriptsize{(0.408)}} & \mc{1}{c}{\scriptsize{(0.618)}} & \mc{1}{c}{\scriptsize{(0.447)}} \\  

    \mc{1}{l}{\scriptsize{Past Surgery: Hysterectomy}} & \mc{1}{c}{\scriptsize{Mid-30s}} & \mc{1}{c}{\scriptsize{0.008}} & \mc{1}{c}{\scriptsize{0.020}} & \mc{1}{c}{\scriptsize{-0.123}} & \mc{1}{c}{\scriptsize{-0.044}} & \mc{1}{c}{\scriptsize{-0.114}} & \mc{1}{c}{\scriptsize{0.043}} & \mc{1}{c}{\scriptsize{0.069}} & \mc{1}{c}{\scriptsize{0.050}} \\  

     &  & \mc{1}{c}{\scriptsize{(0.539)}} & \mc{1}{c}{\scriptsize{(0.513)}} & \mc{1}{c}{\scriptsize{(0.197)}} & \mc{1}{c}{\scriptsize{(0.355)}} & \mc{1}{c}{\scriptsize{(0.211)}} & \mc{1}{c}{\scriptsize{(0.671)}} & \mc{1}{c}{\scriptsize{(0.684)}} & \mc{1}{c}{\scriptsize{(0.684)}} \\  

  \bottomrule
  \end{tabular}
	\end{table} 

	\begin{table}[H]
     \caption{Treatment Effects on Physical Activity, Female Sample}
     \label{table:abccare_rslt_female_cat40}
	\input{AppResOutput/abccare/rslt_female_cat40}
	\end{table} 

	\begin{table}[H]
     \caption{Treatment Effects on Physical Exam - Ear, Female Sample}
     \label{table:abccare_rslt_female_cat41}
	  \begin{tabular}{cccccccccc}
  \toprule

    \scriptsize{Variable} & \scriptsize{Age} & \scriptsize{(1)} & \scriptsize{(2)} & \scriptsize{(3)} & \scriptsize{(4)} & \scriptsize{(5)} & \scriptsize{(6)} & \scriptsize{(7)} & \scriptsize{(8)} \\ 
    \midrule  

    \mc{1}{l}{\scriptsize{Ear: Auditory Canal}} & \mc{1}{c}{\scriptsize{Mid-30s}} & \mc{1}{c}{\scriptsize{-0.036}} & \mc{1}{c}{\scriptsize{-0.066}} &  &  &  & \mc{1}{c}{\scriptsize{-0.045}} & \mc{1}{c}{\scriptsize{-0.086}} & \mc{1}{c}{\scriptsize{-0.050}} \\  

     &  & \mc{1}{c}{\scriptsize{\textbf{(0.079)}}} & \mc{1}{c}{\scriptsize{\textbf{(0.066)}}} &  &  &  & \mc{1}{c}{\scriptsize{\textbf{(0.079)}}} & \mc{1}{c}{\scriptsize{\textbf{(0.066)}}} & \mc{1}{c}{\scriptsize{\textbf{(0.053)}}} \\  

    \mc{1}{l}{\scriptsize{Eye: Eyeball}} & \mc{1}{c}{\scriptsize{Mid-30s}} & \mc{1}{c}{\scriptsize{-0.036}} & \mc{1}{c}{\scriptsize{-0.026}} & \mc{1}{c}{\scriptsize{-0.167}} & \mc{1}{c}{\scriptsize{-0.162}} & \mc{1}{c}{\scriptsize{-0.165}} &  &  &  \\  

     &  & \mc{1}{c}{\scriptsize{(0.105)}} & \mc{1}{c}{\scriptsize{(0.145)}} & \mc{1}{c}{\scriptsize{\textbf{(0.053)}}} & \mc{1}{c}{\scriptsize{\textbf{(0.092)}}} & \mc{1}{c}{\scriptsize{\textbf{(0.053)}}} &  &  &  \\  

    \mc{1}{l}{\scriptsize{Eye: Fundi}} & \mc{1}{c}{\scriptsize{Mid-30s}} & \mc{1}{c}{\scriptsize{-0.036}} & \mc{1}{c}{\scriptsize{-0.031}} & \mc{1}{c}{\scriptsize{-0.167}} & \mc{1}{c}{\scriptsize{-0.139}} & \mc{1}{c}{\scriptsize{-0.164}} &  &  &  \\  

     &  & \mc{1}{c}{\scriptsize{\textbf{(0.066)}}} & \mc{1}{c}{\scriptsize{\textbf{(0.092)}}} & \mc{1}{c}{\scriptsize{\textbf{(0.053)}}} & \mc{1}{c}{\scriptsize{(0.145)}} & \mc{1}{c}{\scriptsize{\textbf{(0.053)}}} &  &  &  \\  

  \bottomrule
  \end{tabular}
	\end{table} 

	\begin{table}[H]
     \caption{Treatment Effects on Physical Exam - General I, Female Sample}
     \label{table:abccare_rslt_female_cat42}
	  \begin{tabular}{cccccccccc}
  \toprule

    \scriptsize{Variable} & \scriptsize{Age} & \scriptsize{(1)} & \scriptsize{(2)} & \scriptsize{(3)} & \scriptsize{(4)} & \scriptsize{(5)} & \scriptsize{(6)} & \scriptsize{(7)} & \scriptsize{(8)} \\ 
    \midrule  

    \mc{1}{l}{\scriptsize{Respirations}} & \mc{1}{c}{\scriptsize{Mid-30s}} & \mc{1}{c}{\scriptsize{-0.555}} & \mc{1}{c}{\scriptsize{-0.341}} & \mc{1}{c}{\scriptsize{-1.286}} & \mc{1}{c}{\scriptsize{-1.240}} & \mc{1}{c}{\scriptsize{-1.379}} & \mc{1}{c}{\scriptsize{-0.336}} & \mc{1}{c}{\scriptsize{0.151}} & \mc{1}{c}{\scriptsize{-0.010}} \\  

     &  & \mc{1}{c}{\scriptsize{(0.171)}} & \mc{1}{c}{\scriptsize{(0.316)}} & \mc{1}{c}{\scriptsize{\textbf{(0.000)}}} & \mc{1}{c}{\scriptsize{\textbf{(0.066)}}} & \mc{1}{c}{\scriptsize{\textbf{(0.000)}}} & \mc{1}{c}{\scriptsize{(0.316)}} & \mc{1}{c}{\scriptsize{(0.539)}} & \mc{1}{c}{\scriptsize{(0.539)}} \\  

    \mc{1}{l}{\scriptsize{Temp (F)}} & \mc{1}{c}{\scriptsize{Mid-30s}} & \mc{1}{c}{\scriptsize{0.185}} & \mc{1}{c}{\scriptsize{0.161}} & \mc{1}{c}{\scriptsize{0.062}} & \mc{1}{c}{\scriptsize{-0.006}} & \mc{1}{c}{\scriptsize{0.045}} & \mc{1}{c}{\scriptsize{0.219}} & \mc{1}{c}{\scriptsize{0.218}} & \mc{1}{c}{\scriptsize{0.187}} \\  

     &  & \mc{1}{c}{\scriptsize{(0.829)}} & \mc{1}{c}{\scriptsize{(0.789)}} & \mc{1}{c}{\scriptsize{(0.605)}} & \mc{1}{c}{\scriptsize{(0.461)}} & \mc{1}{c}{\scriptsize{(0.605)}} & \mc{1}{c}{\scriptsize{(0.855)}} & \mc{1}{c}{\scriptsize{(0.868)}} & \mc{1}{c}{\scriptsize{(0.842)}} \\  

    \mc{1}{l}{\scriptsize{Pulse}} & \mc{1}{c}{\scriptsize{Mid-30s}} & \mc{1}{c}{\scriptsize{-1.890}} & \mc{1}{c}{\scriptsize{-4.263}} & \mc{1}{c}{\scriptsize{-11.318}} & \mc{1}{c}{\scriptsize{-12.480}} & \mc{1}{c}{\scriptsize{-11.960}} & \mc{1}{c}{\scriptsize{0.682}} & \mc{1}{c}{\scriptsize{-1.009}} & \mc{1}{c}{\scriptsize{0.867}} \\  

     &  & \mc{1}{c}{\scriptsize{(0.316)}} & \mc{1}{c}{\scriptsize{(0.145)}} & \mc{1}{c}{\scriptsize{\textbf{(0.000)}}} & \mc{1}{c}{\scriptsize{\textbf{(0.066)}}} & \mc{1}{c}{\scriptsize{\textbf{(0.000)}}} & \mc{1}{c}{\scriptsize{(0.592)}} & \mc{1}{c}{\scriptsize{(0.421)}} & \mc{1}{c}{\scriptsize{(0.579)}} \\  

    \mc{1}{l}{\scriptsize{Nutrition}} & \mc{1}{c}{\scriptsize{Mid-30s}} & \mc{1}{c}{\scriptsize{0.062}} & \mc{1}{c}{\scriptsize{0.227}} & \mc{1}{c}{\scriptsize{0.015}} & \mc{1}{c}{\scriptsize{0.136}} & \mc{1}{c}{\scriptsize{0.026}} & \mc{1}{c}{\scriptsize{0.075}} & \mc{1}{c}{\scriptsize{0.240}} & \mc{1}{c}{\scriptsize{0.157}} \\  

     &  & \mc{1}{c}{\scriptsize{(0.697)}} & \mc{1}{c}{\scriptsize{(0.947)}} & \mc{1}{c}{\scriptsize{(0.526)}} & \mc{1}{c}{\scriptsize{(0.711)}} & \mc{1}{c}{\scriptsize{(0.513)}} & \mc{1}{c}{\scriptsize{(0.737)}} & \mc{1}{c}{\scriptsize{(0.895)}} & \mc{1}{c}{\scriptsize{(0.882)}} \\  

    \mc{1}{l}{\scriptsize{Posture}} & \mc{1}{c}{\scriptsize{Mid-30s}} & \mc{1}{c}{\scriptsize{0.087}} & \mc{1}{c}{\scriptsize{0.147}} & \mc{1}{c}{\scriptsize{0.087}} & \mc{1}{c}{\scriptsize{0.026}} & \mc{1}{c}{\scriptsize{0.100}} & \mc{1}{c}{\scriptsize{0.087}} & \mc{1}{c}{\scriptsize{0.173}} & \mc{1}{c}{\scriptsize{0.101}} \\  

     &  & \mc{1}{c}{\scriptsize{(0.579)}} & \mc{1}{c}{\scriptsize{(0.592)}} & \mc{1}{c}{\scriptsize{(0.579)}} & \mc{1}{c}{\scriptsize{(0.342)}} & \mc{1}{c}{\scriptsize{(0.579)}} & \mc{1}{c}{\scriptsize{(0.579)}} & \mc{1}{c}{\scriptsize{(0.579)}} & \mc{1}{c}{\scriptsize{(0.579)}} \\  

  \bottomrule
  \end{tabular}
	\end{table} 

	\begin{table}[H]
     \caption{Treatment Effects on Physical Exam - General II, Female Sample}
     \label{table:abccare_rslt_female_cat43}
	\input{AppResOutput/abccare/rslt_female_cat43}
	\end{table} 

	\begin{table}[H]
     \caption{Treatment Effects on Physical Exam (Part II), Female Sample}
     \label{table:abccare_rslt_female_cat44}
	  \begin{tabular}{cccccccccc}
  \toprule

    \scriptsize{Variable} & \scriptsize{Age} & \scriptsize{(1)} & \scriptsize{(2)} & \scriptsize{(3)} & \scriptsize{(4)} & \scriptsize{(5)} & \scriptsize{(6)} & \scriptsize{(7)} & \scriptsize{(8)} \\ 
    \midrule  

    \mc{1}{l}{\scriptsize{Mouth and Throat: Upper Teeth}} & \mc{1}{c}{\scriptsize{Mid-30s}} & \mc{1}{c}{\scriptsize{0.082}} & \mc{1}{c}{\scriptsize{0.106}} & \mc{1}{c}{\scriptsize{0.094}} & \mc{1}{c}{\scriptsize{0.221}} & \mc{1}{c}{\scriptsize{0.110}} & \mc{1}{c}{\scriptsize{0.079}} & \mc{1}{c}{\scriptsize{0.080}} & \mc{1}{c}{\scriptsize{0.128}} \\  

     &  & \mc{1}{c}{\scriptsize{(0.763)}} & \mc{1}{c}{\scriptsize{(0.829)}} & \mc{1}{c}{\scriptsize{(0.658)}} & \mc{1}{c}{\scriptsize{(0.829)}} & \mc{1}{c}{\scriptsize{(0.737)}} & \mc{1}{c}{\scriptsize{(0.763)}} & \mc{1}{c}{\scriptsize{(0.776)}} & \mc{1}{c}{\scriptsize{(0.868)}} \\  

    \mc{1}{l}{\scriptsize{Muscle Strength: Reflexes}} & \mc{1}{c}{\scriptsize{Mid-30s}} & \mc{1}{c}{\scriptsize{0.087}} & \mc{1}{c}{\scriptsize{0.134}} & \mc{1}{c}{\scriptsize{0.087}} & \mc{1}{c}{\scriptsize{0.163}} & \mc{1}{c}{\scriptsize{0.087}} & \mc{1}{c}{\scriptsize{0.087}} & \mc{1}{c}{\scriptsize{0.131}} & \mc{1}{c}{\scriptsize{0.086}} \\  

     &  & \mc{1}{c}{\scriptsize{(0.803)}} & \mc{1}{c}{\scriptsize{(0.855)}} & \mc{1}{c}{\scriptsize{(0.803)}} & \mc{1}{c}{\scriptsize{(0.750)}} & \mc{1}{c}{\scriptsize{(0.803)}} & \mc{1}{c}{\scriptsize{(0.803)}} & \mc{1}{c}{\scriptsize{(0.842)}} & \mc{1}{c}{\scriptsize{(0.803)}} \\  

    \mc{1}{l}{\scriptsize{Mouth and Throat: Lower Teeth}} & \mc{1}{c}{\scriptsize{Mid-30s}} & \mc{1}{c}{\scriptsize{0.031}} & \mc{1}{c}{\scriptsize{0.003}} & \mc{1}{c}{\scriptsize{0.007}} & \mc{1}{c}{\scriptsize{-0.062}} & \mc{1}{c}{\scriptsize{0.029}} & \mc{1}{c}{\scriptsize{0.037}} & \mc{1}{c}{\scriptsize{0.022}} & \mc{1}{c}{\scriptsize{0.095}} \\  

     &  & \mc{1}{c}{\scriptsize{(0.632)}} & \mc{1}{c}{\scriptsize{(0.566)}} & \mc{1}{c}{\scriptsize{(0.500)}} & \mc{1}{c}{\scriptsize{(0.289)}} & \mc{1}{c}{\scriptsize{(0.474)}} & \mc{1}{c}{\scriptsize{(0.658)}} & \mc{1}{c}{\scriptsize{(0.566)}} & \mc{1}{c}{\scriptsize{(0.855)}} \\  

    \mc{1}{l}{\scriptsize{Muscle Strength: Coordination}} & \mc{1}{c}{\scriptsize{Mid-30s}} &  &  &  &  &  &  &  &  \\  

     &  &  &  &  &  &  &  &  &  \\  

  \bottomrule
  \end{tabular}
	\end{table} 

	\begin{table}[H]
     \caption{Treatment Effects on Age 21 Brief Symptom Inventory, Female Sample}
     \label{table:abccare_rslt_female_cat45}
	\input{AppResOutput/abccare/rslt_female_cat45}
	\end{table} 

	\begin{table}[H]
     \caption{Treatment Effects on Age 30 Adult Self Report DSM Scale $t$-Score, Female Sample}
     \label{table:abccare_rslt_female_cat46}
	  \begin{tabular}{cccccccccc}
  \toprule

    \scriptsize{Variable} & \scriptsize{Age} & \scriptsize{(1)} & \scriptsize{(2)} & \scriptsize{(3)} & \scriptsize{(4)} & \scriptsize{(5)} & \scriptsize{(6)} & \scriptsize{(7)} & \scriptsize{(8)} \\ 
    \midrule  

    \mc{1}{l}{\scriptsize{Aggressive}} & \mc{1}{c}{\scriptsize{12}} & \mc{1}{c}{\scriptsize{-1.471}} & \mc{1}{c}{\scriptsize{-1.566}} & \mc{1}{c}{\scriptsize{-0.706}} & \mc{1}{c}{\scriptsize{-0.381}} & \mc{1}{c}{\scriptsize{-1.499}} & \mc{1}{c}{\scriptsize{-2.373}} & \mc{1}{c}{\scriptsize{-2.426}} & \mc{1}{c}{\scriptsize{-4.128}} \\  

     &  & \mc{1}{c}{\scriptsize{(0.250)}} & \mc{1}{c}{\scriptsize{(0.276)}} & \mc{1}{c}{\scriptsize{(0.421)}} & \mc{1}{c}{\scriptsize{(0.329)}} & \mc{1}{c}{\scriptsize{(0.250)}} & \mc{1}{c}{\scriptsize{(0.197)}} & \mc{1}{c}{\scriptsize{(0.276)}} & \mc{1}{c}{\scriptsize{(0.105)}} \\  

    \mc{1}{l}{\scriptsize{Inattentive}} & \mc{1}{c}{\scriptsize{12}} & \mc{1}{c}{\scriptsize{-0.412}} & \mc{1}{c}{\scriptsize{1.433}} & \mc{1}{c}{\scriptsize{-2.971}} & \mc{1}{c}{\scriptsize{0.763}} & \mc{1}{c}{\scriptsize{-3.140}} & \mc{1}{c}{\scriptsize{1.113}} & \mc{1}{c}{\scriptsize{2.390}} & \mc{1}{c}{\scriptsize{0.584}} \\  

     &  & \mc{1}{c}{\scriptsize{(0.487)}} & \mc{1}{c}{\scriptsize{(0.632)}} & \mc{1}{c}{\scriptsize{(0.211)}} & \mc{1}{c}{\scriptsize{(0.461)}} & \mc{1}{c}{\scriptsize{(0.171)}} & \mc{1}{c}{\scriptsize{(0.684)}} & \mc{1}{c}{\scriptsize{(0.697)}} & \mc{1}{c}{\scriptsize{(0.579)}} \\  

    \mc{1}{l}{\scriptsize{Externalizing}} & \mc{1}{c}{\scriptsize{12}} & \mc{1}{c}{\scriptsize{-2.353}} & \mc{1}{c}{\scriptsize{-2.026}} & \mc{1}{c}{\scriptsize{-4.294}} & \mc{1}{c}{\scriptsize{-3.994}} & \mc{1}{c}{\scriptsize{-6.337}} & \mc{1}{c}{\scriptsize{-1.711}} & \mc{1}{c}{\scriptsize{-1.536}} & \mc{1}{c}{\scriptsize{-4.300}} \\  

     &  & \mc{1}{c}{\scriptsize{(0.171)}} & \mc{1}{c}{\scriptsize{(0.303)}} & \mc{1}{c}{\scriptsize{(0.105)}} & \mc{1}{c}{\scriptsize{(0.263)}} & \mc{1}{c}{\scriptsize{\textbf{(0.079)}}} & \mc{1}{c}{\scriptsize{(0.250)}} & \mc{1}{c}{\scriptsize{(0.368)}} & \mc{1}{c}{\scriptsize{\textbf{(0.092)}}} \\  

    \mc{1}{l}{\scriptsize{Immature}} & \mc{1}{c}{\scriptsize{12}} & \mc{1}{c}{\scriptsize{-0.471}} & \mc{1}{c}{\scriptsize{-0.144}} & \mc{1}{c}{\scriptsize{-4.132}} & \mc{1}{c}{\scriptsize{-3.795}} & \mc{1}{c}{\scriptsize{-5.273}} & \mc{1}{c}{\scriptsize{0.701}} & \mc{1}{c}{\scriptsize{0.830}} & \mc{1}{c}{\scriptsize{-0.309}} \\  

     &  & \mc{1}{c}{\scriptsize{(0.382)}} & \mc{1}{c}{\scriptsize{(0.447)}} & \mc{1}{c}{\scriptsize{\textbf{(0.092)}}} & \mc{1}{c}{\scriptsize{(0.197)}} & \mc{1}{c}{\scriptsize{\textbf{(0.053)}}} & \mc{1}{c}{\scriptsize{(0.632)}} & \mc{1}{c}{\scriptsize{(0.711)}} & \mc{1}{c}{\scriptsize{(0.434)}} \\  

    \mc{1}{l}{\scriptsize{Socially Withdrawn}} & \mc{1}{c}{\scriptsize{12}} & \mc{1}{c}{\scriptsize{-1.765}} & \mc{1}{c}{\scriptsize{-2.273}} & \mc{1}{c}{\scriptsize{-3.412}} & \mc{1}{c}{\scriptsize{-2.975}} & \mc{1}{c}{\scriptsize{-3.082}} & \mc{1}{c}{\scriptsize{-0.912}} & \mc{1}{c}{\scriptsize{-2.434}} & \mc{1}{c}{\scriptsize{-1.809}} \\  

     &  & \mc{1}{c}{\scriptsize{(0.118)}} & \mc{1}{c}{\scriptsize{(0.184)}} & \mc{1}{c}{\scriptsize{(0.132)}} & \mc{1}{c}{\scriptsize{(0.237)}} & \mc{1}{c}{\scriptsize{(0.132)}} & \mc{1}{c}{\scriptsize{(0.289)}} & \mc{1}{c}{\scriptsize{(0.197)}} & \mc{1}{c}{\scriptsize{(0.197)}} \\  

    \mc{1}{l}{\scriptsize{Anxious}} & \mc{1}{c}{\scriptsize{12}} & \mc{1}{c}{\scriptsize{-2.235}} & \mc{1}{c}{\scriptsize{-2.473}} & \mc{1}{c}{\scriptsize{-6.618}} & \mc{1}{c}{\scriptsize{-6.110}} & \mc{1}{c}{\scriptsize{-6.809}} & \mc{1}{c}{\scriptsize{-1.118}} & \mc{1}{c}{\scriptsize{-1.300}} & \mc{1}{c}{\scriptsize{-1.462}} \\  

     &  & \mc{1}{c}{\scriptsize{\textbf{(0.079)}}} & \mc{1}{c}{\scriptsize{(0.145)}} & \mc{1}{c}{\scriptsize{\textbf{(0.000)}}} & \mc{1}{c}{\scriptsize{\textbf{(0.053)}}} & \mc{1}{c}{\scriptsize{\textbf{(0.000)}}} & \mc{1}{c}{\scriptsize{(0.250)}} & \mc{1}{c}{\scriptsize{(0.303)}} & \mc{1}{c}{\scriptsize{(0.197)}} \\  

    \mc{1}{l}{\scriptsize{Unpopular}} & \mc{1}{c}{\scriptsize{12}} & \mc{1}{c}{\scriptsize{-2.706}} & \mc{1}{c}{\scriptsize{-1.675}} & \mc{1}{c}{\scriptsize{-6.074}} & \mc{1}{c}{\scriptsize{-7.048}} & \mc{1}{c}{\scriptsize{-6.449}} & \mc{1}{c}{\scriptsize{-2.324}} & \mc{1}{c}{\scriptsize{-0.852}} & \mc{1}{c}{\scriptsize{-3.082}} \\  

     &  & \mc{1}{c}{\scriptsize{(0.171)}} & \mc{1}{c}{\scriptsize{(0.303)}} & \mc{1}{c}{\scriptsize{(0.197)}} & \mc{1}{c}{\scriptsize{(0.158)}} & \mc{1}{c}{\scriptsize{(0.171)}} & \mc{1}{c}{\scriptsize{(0.211)}} & \mc{1}{c}{\scriptsize{(0.368)}} & \mc{1}{c}{\scriptsize{(0.132)}} \\  

    \mc{1}{l}{\scriptsize{Self-Destructive}} & \mc{1}{c}{\scriptsize{12}} & \mc{1}{c}{\scriptsize{-1.118}} & \mc{1}{c}{\scriptsize{-0.211}} & \mc{1}{c}{\scriptsize{-2.250}} & \mc{1}{c}{\scriptsize{-0.837}} & \mc{1}{c}{\scriptsize{-1.628}} & \mc{1}{c}{\scriptsize{-0.917}} & \mc{1}{c}{\scriptsize{-0.066}} & \mc{1}{c}{\scriptsize{-1.009}} \\  

     &  & \mc{1}{c}{\scriptsize{(0.171)}} & \mc{1}{c}{\scriptsize{(0.434)}} & \mc{1}{c}{\scriptsize{(0.211)}} & \mc{1}{c}{\scriptsize{(0.276)}} & \mc{1}{c}{\scriptsize{(0.263)}} & \mc{1}{c}{\scriptsize{(0.197)}} & \mc{1}{c}{\scriptsize{(0.447)}} & \mc{1}{c}{\scriptsize{(0.158)}} \\  

    \mc{1}{l}{\scriptsize{Internalizing}} & \mc{1}{c}{\scriptsize{12}} & \mc{1}{c}{\scriptsize{-1.529}} & \mc{1}{c}{\scriptsize{-2.592}} & \mc{1}{c}{\scriptsize{-6.838}} & \mc{1}{c}{\scriptsize{-7.523}} & \mc{1}{c}{\scriptsize{-7.951}} & \mc{1}{c}{\scriptsize{0.662}} & \mc{1}{c}{\scriptsize{-1.546}} & \mc{1}{c}{\scriptsize{-0.967}} \\  

     &  & \mc{1}{c}{\scriptsize{(0.276)}} & \mc{1}{c}{\scriptsize{(0.276)}} & \mc{1}{c}{\scriptsize{(0.132)}} & \mc{1}{c}{\scriptsize{(0.237)}} & \mc{1}{c}{\scriptsize{\textbf{(0.079)}}} & \mc{1}{c}{\scriptsize{(0.605)}} & \mc{1}{c}{\scriptsize{(0.316)}} & \mc{1}{c}{\scriptsize{(0.368)}} \\  

  \bottomrule
  \end{tabular}
	\end{table} 

	\begin{table}[H]
     \caption{Treatment Effects on Age 30 Adult Self Report Syndrome Scale $t$-Score, Female Sample}
     \label{table:abccare_rslt_female_cat47}
	\input{AppResOutput/abccare/rslt_female_cat47}
	\end{table} 

	\begin{table}[H]
     \caption{Treatment Effects on BSI 18 $t$-Score, Female Sample}
     \label{table:abccare_rslt_female_cat48}
	\input{AppResOutput/abccare/rslt_female_cat48}
	\end{table} 

	\begin{table}[H]
     \caption{Treatment Effects on BSI Raw Score, Female Sample}
     \label{table:abccare_rslt_female_cat49}
	\input{AppResOutput/abccare/rslt_female_cat49}
	\end{table} 

	\begin{table}[H]
     \caption{Treatment Effects on BSI $t$-Score, Female Sample}
     \label{table:abccare_rslt_female_cat50}
	  \begin{tabular}{cccccccccc}
  \toprule

    \scriptsize{Variable} & \scriptsize{Age} & \scriptsize{(1)} & \scriptsize{(2)} & \scriptsize{(3)} & \scriptsize{(4)} & \scriptsize{(5)} & \scriptsize{(6)} & \scriptsize{(7)} & \scriptsize{(8)} \\ 
    \midrule  

    \mc{1}{l}{\scriptsize{Impulsitivity - control}} & \mc{1}{c}{\scriptsize{8}} & \mc{1}{c}{\scriptsize{0.551}} & \mc{1}{c}{\scriptsize{0.827}} & \mc{1}{c}{\scriptsize{1.472}} & \mc{1}{c}{\scriptsize{1.686}} & \mc{1}{c}{\scriptsize{1.414}} & \mc{1}{c}{\scriptsize{0.282}} & \mc{1}{c}{\scriptsize{0.148}} & \mc{1}{c}{\scriptsize{0.296}} \\  

     &  & \mc{1}{c}{\scriptsize{(0.711)}} & \mc{1}{c}{\scriptsize{(0.803)}} & \mc{1}{c}{\scriptsize{(0.868)}} & \mc{1}{c}{\scriptsize{(0.868)}} & \mc{1}{c}{\scriptsize{(0.855)}} & \mc{1}{c}{\scriptsize{(0.605)}} & \mc{1}{c}{\scriptsize{(0.553)}} & \mc{1}{c}{\scriptsize{(0.592)}} \\  

    \mc{1}{l}{\scriptsize{Impulsitivity - decisive}} & \mc{1}{c}{\scriptsize{8}} & \mc{1}{c}{\scriptsize{1.741}} & \mc{1}{c}{\scriptsize{1.653}} & \mc{1}{c}{\scriptsize{-0.093}} & \mc{1}{c}{\scriptsize{0.160}} & \mc{1}{c}{\scriptsize{-0.128}} & \mc{1}{c}{\scriptsize{2.276}} & \mc{1}{c}{\scriptsize{1.873}} & \mc{1}{c}{\scriptsize{2.240}} \\  

     &  & \mc{1}{c}{\scriptsize{(0.974)}} & \mc{1}{c}{\scriptsize{(0.987)}} & \mc{1}{c}{\scriptsize{(0.447)}} & \mc{1}{c}{\scriptsize{(0.526)}} & \mc{1}{c}{\scriptsize{(0.434)}} & \mc{1}{c}{\scriptsize{(1.000)}} & \mc{1}{c}{\scriptsize{(1.000)}} & \mc{1}{c}{\scriptsize{(1.000)}} \\  

    \mc{1}{l}{\scriptsize{Activity - tempo}} & \mc{1}{c}{\scriptsize{8}} & \mc{1}{c}{\scriptsize{-1.783}} & \mc{1}{c}{\scriptsize{-1.736}} & \mc{1}{c}{\scriptsize{-0.225}} & \mc{1}{c}{\scriptsize{0.843}} & \mc{1}{c}{\scriptsize{-0.055}} & \mc{1}{c}{\scriptsize{-2.237}} & \mc{1}{c}{\scriptsize{-2.556}} & \mc{1}{c}{\scriptsize{-2.438}} \\  

     &  & \mc{1}{c}{\scriptsize{\textbf{(0.079)}}} & \mc{1}{c}{\scriptsize{(0.105)}} & \mc{1}{c}{\scriptsize{(0.421)}} & \mc{1}{c}{\scriptsize{(0.658)}} & \mc{1}{c}{\scriptsize{(0.474)}} & \mc{1}{c}{\scriptsize{\textbf{(0.079)}}} & \mc{1}{c}{\scriptsize{\textbf{(0.066)}}} & \mc{1}{c}{\scriptsize{\textbf{(0.026)}}} \\  

    \mc{1}{l}{\scriptsize{Emotionality - fear}} & \mc{1}{c}{\scriptsize{8}} & \mc{1}{c}{\scriptsize{1.038}} & \mc{1}{c}{\scriptsize{0.525}} & \mc{1}{c}{\scriptsize{-0.819}} & \mc{1}{c}{\scriptsize{-0.691}} & \mc{1}{c}{\scriptsize{-0.944}} & \mc{1}{c}{\scriptsize{1.580}} & \mc{1}{c}{\scriptsize{0.952}} & \mc{1}{c}{\scriptsize{1.401}} \\  

     &  & \mc{1}{c}{\scriptsize{(0.895)}} & \mc{1}{c}{\scriptsize{(0.632)}} & \mc{1}{c}{\scriptsize{(0.289)}} & \mc{1}{c}{\scriptsize{(0.355)}} & \mc{1}{c}{\scriptsize{(0.263)}} & \mc{1}{c}{\scriptsize{(0.947)}} & \mc{1}{c}{\scriptsize{(0.776)}} & \mc{1}{c}{\scriptsize{(0.895)}} \\  

    \mc{1}{l}{\scriptsize{Emotionality - anger}} & \mc{1}{c}{\scriptsize{8}} & \mc{1}{c}{\scriptsize{-0.057}} & \mc{1}{c}{\scriptsize{-0.083}} & \mc{1}{c}{\scriptsize{-0.154}} & \mc{1}{c}{\scriptsize{-0.149}} & \mc{1}{c}{\scriptsize{-0.366}} & \mc{1}{c}{\scriptsize{-0.029}} & \mc{1}{c}{\scriptsize{0.132}} & \mc{1}{c}{\scriptsize{-0.314}} \\  

     &  & \mc{1}{c}{\scriptsize{(0.500)}} & \mc{1}{c}{\scriptsize{(0.474)}} & \mc{1}{c}{\scriptsize{(0.474)}} & \mc{1}{c}{\scriptsize{(0.500)}} & \mc{1}{c}{\scriptsize{(0.395)}} & \mc{1}{c}{\scriptsize{(0.487)}} & \mc{1}{c}{\scriptsize{(0.566)}} & \mc{1}{c}{\scriptsize{(0.395)}} \\  

    \mc{1}{l}{\scriptsize{Impulsitivity - perservere}} & \mc{1}{c}{\scriptsize{8}} & \mc{1}{c}{\scriptsize{-0.833}} & \mc{1}{c}{\scriptsize{-0.993}} & \mc{1}{c}{\scriptsize{-1.104}} & \mc{1}{c}{\scriptsize{-1.267}} & \mc{1}{c}{\scriptsize{-1.231}} & \mc{1}{c}{\scriptsize{-0.753}} & \mc{1}{c}{\scriptsize{-1.081}} & \mc{1}{c}{\scriptsize{-1.043}} \\  

     &  & \mc{1}{c}{\scriptsize{(0.224)}} & \mc{1}{c}{\scriptsize{(0.145)}} & \mc{1}{c}{\scriptsize{(0.145)}} & \mc{1}{c}{\scriptsize{(0.197)}} & \mc{1}{c}{\scriptsize{(0.158)}} & \mc{1}{c}{\scriptsize{(0.263)}} & \mc{1}{c}{\scriptsize{(0.184)}} & \mc{1}{c}{\scriptsize{(0.158)}} \\  

    \mc{1}{l}{\scriptsize{Sociablity}} & \mc{1}{c}{\scriptsize{8}} & \mc{1}{c}{\scriptsize{-0.328}} & \mc{1}{c}{\scriptsize{-0.411}} & \mc{1}{c}{\scriptsize{0.626}} & \mc{1}{c}{\scriptsize{0.887}} & \mc{1}{c}{\scriptsize{0.691}} & \mc{1}{c}{\scriptsize{-0.606}} & \mc{1}{c}{\scriptsize{-0.820}} & \mc{1}{c}{\scriptsize{-0.686}} \\  

     &  & \mc{1}{c}{\scriptsize{(0.671)}} & \mc{1}{c}{\scriptsize{(0.671)}} & \mc{1}{c}{\scriptsize{(0.316)}} & \mc{1}{c}{\scriptsize{(0.276)}} & \mc{1}{c}{\scriptsize{(0.276)}} & \mc{1}{c}{\scriptsize{(0.803)}} & \mc{1}{c}{\scriptsize{(0.829)}} & \mc{1}{c}{\scriptsize{(0.803)}} \\  

    \mc{1}{l}{\scriptsize{Emotionality - general}} & \mc{1}{c}{\scriptsize{8}} & \mc{1}{c}{\scriptsize{-0.060}} & \mc{1}{c}{\scriptsize{-0.757}} & \mc{1}{c}{\scriptsize{-1.055}} & \mc{1}{c}{\scriptsize{-1.661}} & \mc{1}{c}{\scriptsize{-1.172}} & \mc{1}{c}{\scriptsize{0.231}} & \mc{1}{c}{\scriptsize{-0.647}} & \mc{1}{c}{\scriptsize{0.117}} \\  

     &  & \mc{1}{c}{\scriptsize{(0.447)}} & \mc{1}{c}{\scriptsize{(0.250)}} & \mc{1}{c}{\scriptsize{(0.303)}} & \mc{1}{c}{\scriptsize{(0.276)}} & \mc{1}{c}{\scriptsize{(0.289)}} & \mc{1}{c}{\scriptsize{(0.553)}} & \mc{1}{c}{\scriptsize{(0.276)}} & \mc{1}{c}{\scriptsize{(0.526)}} \\  

    \mc{1}{l}{\scriptsize{Activity - vigor}} & \mc{1}{c}{\scriptsize{8}} & \mc{1}{c}{\scriptsize{-1.314}} & \mc{1}{c}{\scriptsize{-0.857}} & \mc{1}{c}{\scriptsize{0.225}} & \mc{1}{c}{\scriptsize{1.496}} & \mc{1}{c}{\scriptsize{0.296}} & \mc{1}{c}{\scriptsize{-1.763}} & \mc{1}{c}{\scriptsize{-1.612}} & \mc{1}{c}{\scriptsize{-1.975}} \\  

     &  & \mc{1}{c}{\scriptsize{(0.934)}} & \mc{1}{c}{\scriptsize{(0.803)}} & \mc{1}{c}{\scriptsize{(0.395)}} & \mc{1}{c}{\scriptsize{(0.197)}} & \mc{1}{c}{\scriptsize{(0.395)}} & \mc{1}{c}{\scriptsize{(0.961)}} & \mc{1}{c}{\scriptsize{(0.895)}} & \mc{1}{c}{\scriptsize{(0.961)}} \\  

    \mc{1}{l}{\scriptsize{Impulsitivity - sensation}} & \mc{1}{c}{\scriptsize{8}} & \mc{1}{c}{\scriptsize{-0.591}} & \mc{1}{c}{\scriptsize{-0.810}} & \mc{1}{c}{\scriptsize{0.253}} & \mc{1}{c}{\scriptsize{0.458}} & \mc{1}{c}{\scriptsize{0.113}} & \mc{1}{c}{\scriptsize{-0.837}} & \mc{1}{c}{\scriptsize{-1.422}} & \mc{1}{c}{\scriptsize{-1.082}} \\  

     &  & \mc{1}{c}{\scriptsize{(0.263)}} & \mc{1}{c}{\scriptsize{(0.197)}} & \mc{1}{c}{\scriptsize{(0.618)}} & \mc{1}{c}{\scriptsize{(0.658)}} & \mc{1}{c}{\scriptsize{(0.605)}} & \mc{1}{c}{\scriptsize{(0.237)}} & \mc{1}{c}{\scriptsize{(0.145)}} & \mc{1}{c}{\scriptsize{(0.145)}} \\  

  \bottomrule
  \end{tabular}
	\end{table} 

	\begin{table}[H]
     \caption{Treatment Effects on Mid-30s Mental Health Conditions, Female Sample}
     \label{table:abccare_rslt_female_cat51}
	  \begin{tabular}{cccccccccc}
  \toprule

    \scriptsize{Variable} & \scriptsize{Age} & \scriptsize{(1)} & \scriptsize{(2)} & \scriptsize{(3)} & \scriptsize{(4)} & \scriptsize{(5)} & \scriptsize{(6)} & \scriptsize{(7)} & \scriptsize{(8)} \\ 
    \midrule  

    \mc{1}{l}{\scriptsize{Current Condition: Any Psychiatric Concern}} & \mc{1}{c}{\scriptsize{Mid-30s}} & \mc{1}{c}{\scriptsize{0.082}} & \mc{1}{c}{\scriptsize{0.082}} & \mc{1}{c}{\scriptsize{0.094}} & \mc{1}{c}{\scriptsize{0.039}} & \mc{1}{c}{\scriptsize{0.114}} & \mc{1}{c}{\scriptsize{0.079}} & \mc{1}{c}{\scriptsize{0.062}} & \mc{1}{c}{\scriptsize{0.156}} \\  

     &  & \mc{1}{c}{\scriptsize{(0.737)}} & \mc{1}{c}{\scriptsize{(0.789)}} & \mc{1}{c}{\scriptsize{(0.671)}} & \mc{1}{c}{\scriptsize{(0.553)}} & \mc{1}{c}{\scriptsize{(0.711)}} & \mc{1}{c}{\scriptsize{(0.763)}} & \mc{1}{c}{\scriptsize{(0.658)}} & \mc{1}{c}{\scriptsize{(0.908)}} \\  

    \mc{1}{l}{\scriptsize{Current Condition: Sad/Depressed in Past 30 Days}} & \mc{1}{c}{\scriptsize{Mid-30s}} & \mc{1}{c}{\scriptsize{2.679}} & \mc{1}{c}{\scriptsize{0.708}} & \mc{1}{c}{\scriptsize{3.500}} & \mc{1}{c}{\scriptsize{1.384}} & \mc{1}{c}{\scriptsize{3.681}} & \mc{1}{c}{\scriptsize{2.454}} & \mc{1}{c}{\scriptsize{0.026}} & \mc{1}{c}{\scriptsize{2.750}} \\  

     &  & \mc{1}{c}{\scriptsize{(0.947)}} & \mc{1}{c}{\scriptsize{(0.763)}} & \mc{1}{c}{\scriptsize{(0.961)}} & \mc{1}{c}{\scriptsize{(0.618)}} & \mc{1}{c}{\scriptsize{(0.974)}} & \mc{1}{c}{\scriptsize{(0.947)}} & \mc{1}{c}{\scriptsize{(0.487)}} & \mc{1}{c}{\scriptsize{(0.921)}} \\  

    \mc{1}{l}{\scriptsize{Current Condition: Mental Problems}} & \mc{1}{c}{\scriptsize{Mid-30s}} & \mc{1}{c}{\scriptsize{0.034}} & \mc{1}{c}{\scriptsize{0.104}} & \mc{1}{c}{\scriptsize{0.058}} & \mc{1}{c}{\scriptsize{-0.019}} & \mc{1}{c}{\scriptsize{0.099}} & \mc{1}{c}{\scriptsize{0.028}} & \mc{1}{c}{\scriptsize{0.113}} & \mc{1}{c}{\scriptsize{0.063}} \\  

     &  & \mc{1}{c}{\scriptsize{(0.605)}} & \mc{1}{c}{\scriptsize{(0.750)}} & \mc{1}{c}{\scriptsize{(0.566)}} & \mc{1}{c}{\scriptsize{(0.474)}} & \mc{1}{c}{\scriptsize{(0.671)}} & \mc{1}{c}{\scriptsize{(0.500)}} & \mc{1}{c}{\scriptsize{(0.592)}} & \mc{1}{c}{\scriptsize{(0.579)}} \\  

    \mc{1}{l}{\scriptsize{Current Condition: Worried/Anxious in Past 30 Days}} & \mc{1}{c}{\scriptsize{Mid-30s}} & \mc{1}{c}{\scriptsize{6.090}} & \mc{1}{c}{\scriptsize{3.380}} & \mc{1}{c}{\scriptsize{4.471}} & \mc{1}{c}{\scriptsize{5.861}} & \mc{1}{c}{\scriptsize{5.169}} & \mc{1}{c}{\scriptsize{6.532}} & \mc{1}{c}{\scriptsize{2.153}} & \mc{1}{c}{\scriptsize{7.018}} \\  

     &  & \mc{1}{c}{\scriptsize{(0.974)}} & \mc{1}{c}{\scriptsize{(0.947)}} & \mc{1}{c}{\scriptsize{(0.895)}} & \mc{1}{c}{\scriptsize{(0.921)}} & \mc{1}{c}{\scriptsize{(0.934)}} & \mc{1}{c}{\scriptsize{(0.974)}} & \mc{1}{c}{\scriptsize{(0.816)}} & \mc{1}{c}{\scriptsize{(1.000)}} \\  

    \mc{1}{l}{\scriptsize{Current Condition: Anxiety}} & \mc{1}{c}{\scriptsize{Mid-30s}} & \mc{1}{c}{\scriptsize{0.023}} & \mc{1}{c}{\scriptsize{0.016}} & \mc{1}{c}{\scriptsize{0.130}} & \mc{1}{c}{\scriptsize{0.170}} & \mc{1}{c}{\scriptsize{0.102}} & \mc{1}{c}{\scriptsize{-0.006}} & \mc{1}{c}{\scriptsize{-0.034}} & \mc{1}{c}{\scriptsize{0.027}} \\  

     &  & \mc{1}{c}{\scriptsize{(0.592)}} & \mc{1}{c}{\scriptsize{(0.592)}} & \mc{1}{c}{\scriptsize{(0.895)}} & \mc{1}{c}{\scriptsize{(0.711)}} & \mc{1}{c}{\scriptsize{(0.750)}} & \mc{1}{c}{\scriptsize{(0.526)}} & \mc{1}{c}{\scriptsize{(0.342)}} & \mc{1}{c}{\scriptsize{(0.592)}} \\  

    \mc{1}{l}{\scriptsize{Current Condition: Suicidal Ideation}} & \mc{1}{c}{\scriptsize{Mid-30s}} & \mc{1}{c}{\scriptsize{0.008}} & \mc{1}{c}{\scriptsize{-0.025}} & \mc{1}{c}{\scriptsize{0.043}} &  &  & \mc{1}{c}{\scriptsize{-0.002}} & \mc{1}{c}{\scriptsize{-0.041}} & \mc{1}{c}{\scriptsize{-0.049}} \\  

     &  & \mc{1}{c}{\scriptsize{(0.408)}} & \mc{1}{c}{\scriptsize{\textbf{(0.066)}}} & \mc{1}{c}{\scriptsize{(0.579)}} &  &  & \mc{1}{c}{\scriptsize{(0.421)}} & \mc{1}{c}{\scriptsize{\textbf{(0.079)}}} & \mc{1}{c}{\scriptsize{\textbf{(0.053)}}} \\  

    \mc{1}{l}{\scriptsize{Current Condition: Insomnia}} & \mc{1}{c}{\scriptsize{Mid-30s}} & \mc{1}{c}{\scriptsize{0.051}} & \mc{1}{c}{\scriptsize{0.047}} & \mc{1}{c}{\scriptsize{0.087}} & \mc{1}{c}{\scriptsize{0.135}} & \mc{1}{c}{\scriptsize{0.087}} & \mc{1}{c}{\scriptsize{0.042}} & \mc{1}{c}{\scriptsize{0.019}} & \mc{1}{c}{\scriptsize{0.037}} \\  

     &  & \mc{1}{c}{\scriptsize{(0.737)}} & \mc{1}{c}{\scriptsize{(0.697)}} & \mc{1}{c}{\scriptsize{(0.816)}} & \mc{1}{c}{\scriptsize{(0.750)}} & \mc{1}{c}{\scriptsize{(0.803)}} & \mc{1}{c}{\scriptsize{(0.658)}} & \mc{1}{c}{\scriptsize{(0.579)}} & \mc{1}{c}{\scriptsize{(0.645)}} \\  

    \mc{1}{l}{\scriptsize{Current Condition: Depression}} & \mc{1}{c}{\scriptsize{Mid-30s}} & \mc{1}{c}{\scriptsize{0.059}} & \mc{1}{c}{\scriptsize{0.049}} & \mc{1}{c}{\scriptsize{-0.036}} & \mc{1}{c}{\scriptsize{-0.163}} & \mc{1}{c}{\scriptsize{-0.023}} & \mc{1}{c}{\scriptsize{0.085}} & \mc{1}{c}{\scriptsize{0.093}} & \mc{1}{c}{\scriptsize{0.142}} \\  

     &  & \mc{1}{c}{\scriptsize{(0.776)}} & \mc{1}{c}{\scriptsize{(0.658)}} & \mc{1}{c}{\scriptsize{(0.408)}} & \mc{1}{c}{\scriptsize{(0.171)}} & \mc{1}{c}{\scriptsize{(0.447)}} & \mc{1}{c}{\scriptsize{(0.855)}} & \mc{1}{c}{\scriptsize{(0.724)}} & \mc{1}{c}{\scriptsize{(0.855)}} \\  

  \bottomrule
  \end{tabular}
	\end{table} 

	\begin{table}[H]
     \caption{Treatment Effects on Smoking and Drinking Behavior, Female Sample}
     \label{table:abccare_rslt_female_cat52}
	  \begin{tabular}{cccccccccc}
  \toprule

    \scriptsize{Variable} & \scriptsize{Age} & \scriptsize{(1)} & \scriptsize{(2)} & \scriptsize{(3)} & \scriptsize{(4)} & \scriptsize{(5)} & \scriptsize{(6)} & \scriptsize{(7)} & \scriptsize{(8)} \\ 
    \midrule  

    \mc{1}{l}{\scriptsize{Problems Due to Alcohol or Drugs}} & \mc{1}{c}{\scriptsize{12}} & \mc{1}{c}{\scriptsize{-0.134}} & \mc{1}{c}{\scriptsize{-0.126}} & \mc{1}{c}{\scriptsize{-0.217}} & \mc{1}{c}{\scriptsize{-0.015}} & \mc{1}{c}{\scriptsize{-0.237}} & \mc{1}{c}{\scriptsize{-0.099}} & \mc{1}{c}{\scriptsize{-0.120}} & \mc{1}{c}{\scriptsize{-0.109}} \\  

     &  & \mc{1}{c}{\scriptsize{(0.184)}} & \mc{1}{c}{\scriptsize{(0.184)}} & \mc{1}{c}{\scriptsize{(0.105)}} & \mc{1}{c}{\scriptsize{(0.474)}} & \mc{1}{c}{\scriptsize{\textbf{(0.092)}}} & \mc{1}{c}{\scriptsize{(0.224)}} & \mc{1}{c}{\scriptsize{(0.184)}} & \mc{1}{c}{\scriptsize{(0.276)}} \\  

    \mc{1}{l}{\scriptsize{Used Alcohol and/or Drugs}} & \mc{1}{c}{\scriptsize{12}} & \mc{1}{c}{\scriptsize{0.008}} & \mc{1}{c}{\scriptsize{-0.016}} & \mc{1}{c}{\scriptsize{-0.058}} & \mc{1}{c}{\scriptsize{-0.072}} & \mc{1}{c}{\scriptsize{-0.048}} & \mc{1}{c}{\scriptsize{0.027}} & \mc{1}{c}{\scriptsize{-0.002}} & \mc{1}{c}{\scriptsize{0.035}} \\  

     &  & \mc{1}{c}{\scriptsize{(0.553)}} & \mc{1}{c}{\scriptsize{(0.368)}} & \mc{1}{c}{\scriptsize{(0.263)}} & \mc{1}{c}{\scriptsize{(0.263)}} & \mc{1}{c}{\scriptsize{(0.303)}} & \mc{1}{c}{\scriptsize{(0.618)}} & \mc{1}{c}{\scriptsize{(0.368)}} & \mc{1}{c}{\scriptsize{(0.618)}} \\  

  \bottomrule
  \end{tabular}
	\end{table} 

	\begin{table}[H]
     \caption{Treatment Effects on Tobacco, Drugs, Alcohol, Female Sample}
     \label{table:abccare_rslt_female_cat53}
	  \begin{tabular}{cccccccccc}
  \toprule

    \scriptsize{Variable} & \scriptsize{Age} & \scriptsize{(1)} & \scriptsize{(2)} & \scriptsize{(3)} & \scriptsize{(4)} & \scriptsize{(5)} & \scriptsize{(6)} & \scriptsize{(7)} & \scriptsize{(8)} \\ 
    \midrule  

    \mc{1}{l}{\scriptsize{Days drank alcohol last month}} & \mc{1}{c}{\scriptsize{30}} & \mc{1}{c}{\scriptsize{-0.742}} & \mc{1}{c}{\scriptsize{0.085}} & \mc{1}{c}{\scriptsize{-0.567}} & \mc{1}{c}{\scriptsize{0.274}} & \mc{1}{c}{\scriptsize{-0.261}} & \mc{1}{c}{\scriptsize{-0.918}} & \mc{1}{c}{\scriptsize{0.043}} & \mc{1}{c}{\scriptsize{-0.464}} \\  

     &  & \mc{1}{c}{\scriptsize{(0.289)}} & \mc{1}{c}{\scriptsize{(0.526)}} & \mc{1}{c}{\scriptsize{(0.368)}} & \mc{1}{c}{\scriptsize{(0.434)}} & \mc{1}{c}{\scriptsize{(0.395)}} & \mc{1}{c}{\scriptsize{(0.303)}} & \mc{1}{c}{\scriptsize{(0.474)}} & \mc{1}{c}{\scriptsize{(0.421)}} \\  

    \mc{1}{l}{\scriptsize{Days binge drank alcohol last month}} & \mc{1}{c}{\scriptsize{30}} & \mc{1}{c}{\scriptsize{-0.358}} & \mc{1}{c}{\scriptsize{0.216}} & \mc{1}{c}{\scriptsize{-1.062}} & \mc{1}{c}{\scriptsize{-0.070}} & \mc{1}{c}{\scriptsize{-0.922}} & \mc{1}{c}{\scriptsize{-0.232}} & \mc{1}{c}{\scriptsize{0.455}} & \mc{1}{c}{\scriptsize{0.034}} \\  

     &  & \mc{1}{c}{\scriptsize{(0.276)}} & \mc{1}{c}{\scriptsize{(0.579)}} & \mc{1}{c}{\scriptsize{(0.289)}} & \mc{1}{c}{\scriptsize{(0.461)}} & \mc{1}{c}{\scriptsize{(0.303)}} & \mc{1}{c}{\scriptsize{(0.342)}} & \mc{1}{c}{\scriptsize{(0.776)}} & \mc{1}{c}{\scriptsize{(0.526)}} \\  

  \bottomrule
  \end{tabular}
	\end{table} 
\clearpage

\section{Treatment Effects for Pooled Sample, Step Down}


	\begin{table}[H]
     \caption{Treatment Effects on IQ Scores, Pooled Sample}
     \label{table:abccare_rslt_pooled_cat0_sd}
	  \begin{tabular}{cccccccccc}
  \toprule

    \scriptsize{Variable} & \scriptsize{Age} & \scriptsize{(1)} & \scriptsize{(2)} & \scriptsize{(3)} & \scriptsize{(4)} & \scriptsize{(5)} & \scriptsize{(6)} & \scriptsize{(7)} & \scriptsize{(8)} \\ 
    \midrule  

    \mc{1}{l}{\scriptsize{Std. IQ Test}} & \mc{1}{c}{\scriptsize{2}} & \mc{1}{c}{\scriptsize{9.528}} & \mc{1}{c}{\scriptsize{10.449}} & \mc{1}{c}{\scriptsize{6.875}} & \mc{1}{c}{\scriptsize{8.111}} & \mc{1}{c}{\scriptsize{11.806}} & \mc{1}{c}{\scriptsize{10.286}} & \mc{1}{c}{\scriptsize{10.853}} & \mc{1}{c}{\scriptsize{10.212}} \\  

     &  & \mc{1}{c}{\scriptsize{\textbf{(0.078)}}} & \mc{1}{c}{\scriptsize{\textbf{(0.039)}}} & \mc{1}{c}{\scriptsize{(9.824)}} & \mc{1}{c}{\scriptsize{(9.824)}} & \mc{1}{c}{\scriptsize{\textbf{(0.010)}}} & \mc{1}{c}{\scriptsize{\textbf{(0.049)}}} & \mc{1}{c}{\scriptsize{\textbf{(0.039)}}} & \mc{1}{c}{\scriptsize{\textbf{(0.010)}}} \\  

     & \mc{1}{c}{\scriptsize{3}} & \mc{1}{c}{\scriptsize{13.410}} & \mc{1}{c}{\scriptsize{14.384}} & \mc{1}{c}{\scriptsize{13.896}} & \mc{1}{c}{\scriptsize{16.827}} & \mc{1}{c}{\scriptsize{21.541}} & \mc{1}{c}{\scriptsize{13.271}} & \mc{1}{c}{\scriptsize{14.118}} & \mc{1}{c}{\scriptsize{11.775}} \\  

     &  & \mc{1}{c}{\scriptsize{\textbf{(0.010)}}} & \mc{1}{c}{\scriptsize{\textbf{(0.010)}}} & \mc{1}{c}{\scriptsize{(9.824)}} & \mc{1}{c}{\scriptsize{(9.824)}} & \mc{1}{c}{\scriptsize{\textbf{(0.010)}}} & \mc{1}{c}{\scriptsize{\textbf{(0.020)}}} & \mc{1}{c}{\scriptsize{\textbf{(0.029)}}} & \mc{1}{c}{\scriptsize{\textbf{(0.010)}}} \\  

     & \mc{1}{c}{\scriptsize{3.5}} & \mc{1}{c}{\scriptsize{8.756}} & \mc{1}{c}{\scriptsize{8.145}} & \mc{1}{c}{\scriptsize{6.354}} & \mc{1}{c}{\scriptsize{6.785}} & \mc{1}{c}{\scriptsize{12.350}} & \mc{1}{c}{\scriptsize{9.443}} & \mc{1}{c}{\scriptsize{8.911}} & \mc{1}{c}{\scriptsize{7.002}} \\  

     &  & \mc{1}{c}{\scriptsize{(0.108)}} & \mc{1}{c}{\scriptsize{(0.186)}} & \mc{1}{c}{\scriptsize{(9.824)}} & \mc{1}{c}{\scriptsize{(9.824)}} & \mc{1}{c}{\scriptsize{\textbf{(0.010)}}} & \mc{1}{c}{\scriptsize{(0.108)}} & \mc{1}{c}{\scriptsize{(0.176)}} & \mc{1}{c}{\scriptsize{(0.108)}} \\  

     & \mc{1}{c}{\scriptsize{4}} & \mc{1}{c}{\scriptsize{12.089}} & \mc{1}{c}{\scriptsize{12.375}} & \mc{1}{c}{\scriptsize{8.950}} & \mc{1}{c}{\scriptsize{10.429}} & \mc{1}{c}{\scriptsize{13.774}} & \mc{1}{c}{\scriptsize{12.986}} & \mc{1}{c}{\scriptsize{12.995}} & \mc{1}{c}{\scriptsize{8.525}} \\  

     &  & \mc{1}{c}{\scriptsize{\textbf{(0.010)}}} & \mc{1}{c}{\scriptsize{\textbf{(0.039)}}} & \mc{1}{c}{\scriptsize{(9.824)}} & \mc{1}{c}{\scriptsize{(9.824)}} & \mc{1}{c}{\scriptsize{\textbf{(0.029)}}} & \mc{1}{c}{\scriptsize{\textbf{(0.020)}}} & \mc{1}{c}{\scriptsize{\textbf{(0.039)}}} & \mc{1}{c}{\scriptsize{\textbf{(0.020)}}} \\  

     & \mc{1}{c}{\scriptsize{4.5}} & \mc{1}{c}{\scriptsize{8.508}} & \mc{1}{c}{\scriptsize{9.152}} & \mc{1}{c}{\scriptsize{10.411}} & \mc{1}{c}{\scriptsize{13.644}} & \mc{1}{c}{\scriptsize{14.410}} & \mc{1}{c}{\scriptsize{7.964}} & \mc{1}{c}{\scriptsize{8.427}} & \mc{1}{c}{\scriptsize{6.823}} \\  

     &  & \mc{1}{c}{\scriptsize{(0.108)}} & \mc{1}{c}{\scriptsize{(0.167)}} & \mc{1}{c}{\scriptsize{(9.824)}} & \mc{1}{c}{\scriptsize{(9.824)}} & \mc{1}{c}{\scriptsize{\textbf{(0.010)}}} & \mc{1}{c}{\scriptsize{(0.245)}} & \mc{1}{c}{\scriptsize{(0.353)}} & \mc{1}{c}{\scriptsize{(0.196)}} \\  

     & \mc{1}{c}{\scriptsize{5}} & \mc{1}{c}{\scriptsize{7.697}} & \mc{1}{c}{\scriptsize{7.497}} & \mc{1}{c}{\scriptsize{4.643}} & \mc{1}{c}{\scriptsize{6.615}} & \mc{1}{c}{\scriptsize{9.490}} & \mc{1}{c}{\scriptsize{8.679}} & \mc{1}{c}{\scriptsize{8.002}} & \mc{1}{c}{\scriptsize{5.593}} \\  

     &  & \mc{1}{c}{\scriptsize{(0.186)}} & \mc{1}{c}{\scriptsize{(0.441)}} & \mc{1}{c}{\scriptsize{(1.990)}} & \mc{1}{c}{\scriptsize{(2.059)}} & \mc{1}{c}{\scriptsize{(0.294)}} & \mc{1}{c}{\scriptsize{(0.108)}} & \mc{1}{c}{\scriptsize{(0.441)}} & \mc{1}{c}{\scriptsize{(0.431)}} \\  

     & \mc{1}{c}{\scriptsize{6}} & \mc{1}{c}{\scriptsize{11.595}} & \mc{1}{c}{\scriptsize{9.843}} & \mc{1}{c}{\scriptsize{5.095}} & \mc{1}{c}{\scriptsize{5.631}} & \mc{1}{c}{\scriptsize{5.523}} & \mc{1}{c}{\scriptsize{13.762}} & \mc{1}{c}{\scriptsize{12.603}} & \mc{1}{c}{\scriptsize{9.776}} \\  

     &  & \mc{1}{c}{\scriptsize{(0.922)}} & \mc{1}{c}{\scriptsize{(3.098)}} & \mc{1}{c}{\scriptsize{(9.824)}} & \mc{1}{c}{\scriptsize{(2.059)}} & \mc{1}{c}{\scriptsize{(6.176)}} & \mc{1}{c}{\scriptsize{(0.608)}} & \mc{1}{c}{\scriptsize{(3.814)}} & \mc{1}{c}{\scriptsize{(0.225)}} \\  

     & \mc{1}{c}{\scriptsize{6.6}} & \mc{1}{c}{\scriptsize{5.803}} & \mc{1}{c}{\scriptsize{5.723}} & \mc{1}{c}{\scriptsize{0.831}} & \mc{1}{c}{\scriptsize{2.665}} & \mc{1}{c}{\scriptsize{4.850}} & \mc{1}{c}{\scriptsize{5.916}} & \mc{1}{c}{\scriptsize{5.447}} & \mc{1}{c}{\scriptsize{6.140}} \\  

     &  & \mc{1}{c}{\scriptsize{(1.461)}} & \mc{1}{c}{\scriptsize{(2.088)}} & \mc{1}{c}{\scriptsize{(9.824)}} & \mc{1}{c}{\scriptsize{(9.824)}} & \mc{1}{c}{\scriptsize{(4.892)}} & \mc{1}{c}{\scriptsize{(1.255)}} & \mc{1}{c}{\scriptsize{(2.176)}} & \mc{1}{c}{\scriptsize{(0.392)}} \\  

     & \mc{1}{c}{\scriptsize{7}} & \mc{1}{c}{\scriptsize{4.390}} & \mc{1}{c}{\scriptsize{7.508}} & \mc{1}{c}{\scriptsize{5.323}} & \mc{1}{c}{\scriptsize{13.674}} & \mc{1}{c}{\scriptsize{5.195}} & \mc{1}{c}{\scriptsize{4.156}} & \mc{1}{c}{\scriptsize{7.043}} & \mc{1}{c}{\scriptsize{5.531}} \\  

     &  & \mc{1}{c}{\scriptsize{(2.902)}} & \mc{1}{c}{\scriptsize{(0.441)}} & \mc{1}{c}{\scriptsize{(9.824)}} & \mc{1}{c}{\scriptsize{(0.971)}} & \mc{1}{c}{\scriptsize{(4.216)}} & \mc{1}{c}{\scriptsize{(2.804)}} & \mc{1}{c}{\scriptsize{(0.686)}} & \mc{1}{c}{\scriptsize{(0.755)}} \\  

     & \mc{1}{c}{\scriptsize{8}} & \mc{1}{c}{\scriptsize{4.160}} & \mc{1}{c}{\scriptsize{3.046}} & \mc{1}{c}{\scriptsize{-2.514}} & \mc{1}{c}{\scriptsize{-1.415}} & \mc{1}{c}{\scriptsize{2.713}} & \mc{1}{c}{\scriptsize{4.754}} & \mc{1}{c}{\scriptsize{4.030}} & \mc{1}{c}{\scriptsize{4.799}} \\  

     &  & \mc{1}{c}{\scriptsize{(3.127)}} & \mc{1}{c}{\scriptsize{(5.578)}} & \mc{1}{c}{\scriptsize{(9.824)}} & \mc{1}{c}{\scriptsize{(9.824)}} & \mc{1}{c}{\scriptsize{(6.176)}} & \mc{1}{c}{\scriptsize{(1.775)}} & \mc{1}{c}{\scriptsize{(3.814)}} & \mc{1}{c}{\scriptsize{(0.755)}} \\  

     & \mc{1}{c}{\scriptsize{12}} & \mc{1}{c}{\scriptsize{0.686}} & \mc{1}{c}{\scriptsize{-0.034}} & \mc{1}{c}{\scriptsize{-0.343}} & \mc{1}{c}{\scriptsize{-0.908}} & \mc{1}{c}{\scriptsize{2.744}} & \mc{1}{c}{\scriptsize{0.943}} & \mc{1}{c}{\scriptsize{0.203}} & \mc{1}{c}{\scriptsize{3.579}} \\  

     &  & \mc{1}{c}{\scriptsize{(4.539)}} & \mc{1}{c}{\scriptsize{(7.510)}} & \mc{1}{c}{\scriptsize{(9.824)}} & \mc{1}{c}{\scriptsize{(9.824)}} & \mc{1}{c}{\scriptsize{(6.176)}} & \mc{1}{c}{\scriptsize{(3.667)}} & \mc{1}{c}{\scriptsize{(7.471)}} & \mc{1}{c}{\scriptsize{(1.333)}} \\  

     & \mc{1}{c}{\scriptsize{15}} & \mc{1}{c}{\scriptsize{4.447}} & \mc{1}{c}{\scriptsize{2.571}} & \mc{1}{c}{\scriptsize{-2.057}} & \mc{1}{c}{\scriptsize{-4.614}} & \mc{1}{c}{\scriptsize{0.529}} & \mc{1}{c}{\scriptsize{6.202}} & \mc{1}{c}{\scriptsize{3.969}} & \mc{1}{c}{\scriptsize{5.123}} \\  

     &  & \mc{1}{c}{\scriptsize{(2.431)}} & \mc{1}{c}{\scriptsize{(5.922)}} & \mc{1}{c}{\scriptsize{(9.824)}} & \mc{1}{c}{\scriptsize{(9.824)}} & \mc{1}{c}{\scriptsize{(6.176)}} & \mc{1}{c}{\scriptsize{(1.333)}} & \mc{1}{c}{\scriptsize{(4.412)}} & \mc{1}{c}{\scriptsize{(1.010)}} \\  

     & \mc{1}{c}{\scriptsize{21}} & \mc{1}{c}{\scriptsize{1.550}} & \mc{1}{c}{\scriptsize{-1.024}} & \mc{1}{c}{\scriptsize{0.471}} & \mc{1}{c}{\scriptsize{-1.783}} & \mc{1}{c}{\scriptsize{3.124}} & \mc{1}{c}{\scriptsize{2.307}} & \mc{1}{c}{\scriptsize{-0.720}} & \mc{1}{c}{\scriptsize{2.348}} \\  

     &  & \mc{1}{c}{\scriptsize{(4.539)}} & \mc{1}{c}{\scriptsize{(7.510)}} & \mc{1}{c}{\scriptsize{(9.824)}} & \mc{1}{c}{\scriptsize{(9.824)}} & \mc{1}{c}{\scriptsize{(3.176)}} & \mc{1}{c}{\scriptsize{(3.667)}} & \mc{1}{c}{\scriptsize{(7.471)}} & \mc{1}{c}{\scriptsize{(1.333)}} \\  

    \mc{1}{l}{\scriptsize{IQ Factor}} & \mc{1}{c}{\scriptsize{2 to 5}} & \mc{1}{c}{\scriptsize{0.865}} & \mc{1}{c}{\scriptsize{0.888}} & \mc{1}{c}{\scriptsize{0.735}} & \mc{1}{c}{\scriptsize{0.904}} & \mc{1}{c}{\scriptsize{1.177}} & \mc{1}{c}{\scriptsize{0.903}} & \mc{1}{c}{\scriptsize{0.905}} & \mc{1}{c}{\scriptsize{0.714}} \\  

     &  & \mc{1}{c}{\scriptsize{\textbf{(0.029)}}} & \mc{1}{c}{\scriptsize{\textbf{(0.039)}}} & \mc{1}{c}{\scriptsize{(9.824)}} & \mc{1}{c}{\scriptsize{(9.824)}} & \mc{1}{c}{\scriptsize{\textbf{(0.029)}}} & \mc{1}{c}{\scriptsize{\textbf{(0.020)}}} & \mc{1}{c}{\scriptsize{\textbf{(0.039)}}} & \mc{1}{c}{\scriptsize{\textbf{(0.010)}}} \\  

     & \mc{1}{c}{\scriptsize{6 to 12}} & \mc{1}{c}{\scriptsize{0.329}} & \mc{1}{c}{\scriptsize{0.355}} & \mc{1}{c}{\scriptsize{0.349}} & \mc{1}{c}{\scriptsize{0.661}} & \mc{1}{c}{\scriptsize{0.460}} & \mc{1}{c}{\scriptsize{0.323}} & \mc{1}{c}{\scriptsize{0.369}} & \mc{1}{c}{\scriptsize{0.447}} \\  

     &  & \mc{1}{c}{\scriptsize{(3.696)}} & \mc{1}{c}{\scriptsize{(4.814)}} & \mc{1}{c}{\scriptsize{(9.824)}} & \mc{1}{c}{\scriptsize{(9.824)}} & \mc{1}{c}{\scriptsize{(4.892)}} & \mc{1}{c}{\scriptsize{(3.510)}} & \mc{1}{c}{\scriptsize{(4.412)}} & \mc{1}{c}{\scriptsize{(1.333)}} \\  

     & \mc{1}{c}{\scriptsize{15 to 21}} & \mc{1}{c}{\scriptsize{-0.276}} & \mc{1}{c}{\scriptsize{-0.055}} & \mc{1}{c}{\scriptsize{0.063}} & \mc{1}{c}{\scriptsize{0.295}} & \mc{1}{c}{\scriptsize{-0.192}} & \mc{1}{c}{\scriptsize{-0.392}} & \mc{1}{c}{\scriptsize{-0.132}} & \mc{1}{c}{\scriptsize{-0.348}} \\  

     &  & \mc{1}{c}{\scriptsize{(3.696)}} & \mc{1}{c}{\scriptsize{(7.510)}} & \mc{1}{c}{\scriptsize{(9.824)}} & \mc{1}{c}{\scriptsize{(9.824)}} & \mc{1}{c}{\scriptsize{(6.176)}} & \mc{1}{c}{\scriptsize{(2.804)}} & \mc{1}{c}{\scriptsize{(7.471)}} & \mc{1}{c}{\scriptsize{(1.333)}} \\  

  \bottomrule
  \end{tabular}
	\end{table} 

	\begin{table}[H]
     \caption{Treatment Effects on Achievement Scores, Pooled Sample}
     \label{table:abccare_rslt_pooled_cat1_sd}
	  \begin{tabular}{cccccccccc}
  \toprule

    \scriptsize{Variable} & \scriptsize{Age} & \scriptsize{(1)} & \scriptsize{(2)} & \scriptsize{(3)} & \scriptsize{(4)} & \scriptsize{(5)} & \scriptsize{(6)} & \scriptsize{(7)} & \scriptsize{(8)} \\ 
    \midrule  

    \mc{1}{l}{\scriptsize{Std. Achv.  Test}} & \mc{1}{c}{\scriptsize{5.5}} & \mc{1}{c}{\scriptsize{8.029}} & \mc{1}{c}{\scriptsize{6.821}} & \mc{1}{c}{\scriptsize{14.284}} & \mc{1}{c}{\scriptsize{13.907}} & \mc{1}{c}{\scriptsize{14.177}} & \mc{1}{c}{\scriptsize{6.223}} & \mc{1}{c}{\scriptsize{4.725}} & \mc{1}{c}{\scriptsize{5.812}} \\  

     &  & \mc{1}{c}{\scriptsize{\textbf{(0.000)}}} & \mc{1}{c}{\scriptsize{\textbf{(0.079)}}} & \mc{1}{c}{\scriptsize{\textbf{(0.000)}}} & \mc{1}{c}{\scriptsize{\textbf{(0.010)}}} & \mc{1}{c}{\scriptsize{\textbf{(0.000)}}} & \mc{1}{c}{\scriptsize{\textbf{(0.059)}}} & \mc{1}{c}{\scriptsize{(0.297)}} & \mc{1}{c}{\scriptsize{(0.208)}} \\  

     & \mc{1}{c}{\scriptsize{6}} & \mc{1}{c}{\scriptsize{4.543}} & \mc{1}{c}{\scriptsize{5.225}} & \mc{1}{c}{\scriptsize{6.178}} & \mc{1}{c}{\scriptsize{7.895}} & \mc{1}{c}{\scriptsize{7.130}} & \mc{1}{c}{\scriptsize{4.075}} & \mc{1}{c}{\scriptsize{4.556}} & \mc{1}{c}{\scriptsize{4.728}} \\  

     &  & \mc{1}{c}{\scriptsize{\textbf{(0.020)}}} & \mc{1}{c}{\scriptsize{\textbf{(0.010)}}} & \mc{1}{c}{\scriptsize{(0.158)}} & \mc{1}{c}{\scriptsize{\textbf{(0.050)}}} & \mc{1}{c}{\scriptsize{\textbf{(0.069)}}} & \mc{1}{c}{\scriptsize{\textbf{(0.059)}}} & \mc{1}{c}{\scriptsize{\textbf{(0.040)}}} & \mc{1}{c}{\scriptsize{\textbf{(0.020)}}} \\  

     & \mc{1}{c}{\scriptsize{6.5}} & \mc{1}{c}{\scriptsize{2.767}} & \mc{1}{c}{\scriptsize{3.274}} & \mc{1}{c}{\scriptsize{2.049}} & \mc{1}{c}{\scriptsize{4.264}} & \mc{1}{c}{\scriptsize{2.132}} & \mc{1}{c}{\scriptsize{2.931}} & \mc{1}{c}{\scriptsize{3.066}} & \mc{1}{c}{\scriptsize{3.602}} \\  

     &  & \mc{1}{c}{\scriptsize{(0.149)}} & \mc{1}{c}{\scriptsize{\textbf{(0.079)}}} & \mc{1}{c}{\scriptsize{(0.861)}} & \mc{1}{c}{\scriptsize{(0.436)}} & \mc{1}{c}{\scriptsize{(0.812)}} & \mc{1}{c}{\scriptsize{(0.228)}} & \mc{1}{c}{\scriptsize{(0.248)}} & \mc{1}{c}{\scriptsize{(0.208)}} \\  

     & \mc{1}{c}{\scriptsize{7}} & \mc{1}{c}{\scriptsize{3.435}} & \mc{1}{c}{\scriptsize{3.147}} & \mc{1}{c}{\scriptsize{5.227}} & \mc{1}{c}{\scriptsize{6.630}} & \mc{1}{c}{\scriptsize{5.843}} & \mc{1}{c}{\scriptsize{3.025}} & \mc{1}{c}{\scriptsize{2.357}} & \mc{1}{c}{\scriptsize{3.598}} \\  

     &  & \mc{1}{c}{\scriptsize{(0.149)}} & \mc{1}{c}{\scriptsize{(0.198)}} & \mc{1}{c}{\scriptsize{(0.347)}} & \mc{1}{c}{\scriptsize{\textbf{(0.099)}}} & \mc{1}{c}{\scriptsize{(0.337)}} & \mc{1}{c}{\scriptsize{(0.366)}} & \mc{1}{c}{\scriptsize{(0.485)}} & \mc{1}{c}{\scriptsize{(0.248)}} \\  

     & \mc{1}{c}{\scriptsize{7.5}} & \mc{1}{c}{\scriptsize{1.937}} & \mc{1}{c}{\scriptsize{3.101}} & \mc{1}{c}{\scriptsize{0.667}} & \mc{1}{c}{\scriptsize{4.290}} & \mc{1}{c}{\scriptsize{0.560}} & \mc{1}{c}{\scriptsize{2.308}} & \mc{1}{c}{\scriptsize{2.946}} & \mc{1}{c}{\scriptsize{2.126}} \\  

     &  & \mc{1}{c}{\scriptsize{(0.733)}} & \mc{1}{c}{\scriptsize{\textbf{(0.099)}}} & \mc{1}{c}{\scriptsize{(0.990)}} & \mc{1}{c}{\scriptsize{(0.614)}} & \mc{1}{c}{\scriptsize{(0.980)}} & \mc{1}{c}{\scriptsize{(0.634)}} & \mc{1}{c}{\scriptsize{(0.158)}} & \mc{1}{c}{\scriptsize{(0.713)}} \\  

     & \mc{1}{c}{\scriptsize{8}} & \mc{1}{c}{\scriptsize{4.207}} & \mc{1}{c}{\scriptsize{5.146}} & \mc{1}{c}{\scriptsize{1.630}} & \mc{1}{c}{\scriptsize{5.443}} & \mc{1}{c}{\scriptsize{2.387}} & \mc{1}{c}{\scriptsize{4.959}} & \mc{1}{c}{\scriptsize{5.482}} & \mc{1}{c}{\scriptsize{5.503}} \\  

     &  & \mc{1}{c}{\scriptsize{(0.109)}} & \mc{1}{c}{\scriptsize{\textbf{(0.010)}}} & \mc{1}{c}{\scriptsize{(0.950)}} & \mc{1}{c}{\scriptsize{(0.406)}} & \mc{1}{c}{\scriptsize{(0.832)}} & \mc{1}{c}{\scriptsize{\textbf{(0.069)}}} & \mc{1}{c}{\scriptsize{\textbf{(0.020)}}} & \mc{1}{c}{\scriptsize{\textbf{(0.020)}}} \\  

     & \mc{1}{c}{\scriptsize{8.5}} & \mc{1}{c}{\scriptsize{5.938}} & \mc{1}{c}{\scriptsize{6.593}} & \mc{1}{c}{\scriptsize{5.046}} & \mc{1}{c}{\scriptsize{7.976}} & \mc{1}{c}{\scriptsize{5.379}} & \mc{1}{c}{\scriptsize{5.507}} & \mc{1}{c}{\scriptsize{5.907}} & \mc{1}{c}{\scriptsize{5.824}} \\  

     &  & \mc{1}{c}{\scriptsize{\textbf{(0.050)}}} & \mc{1}{c}{\scriptsize{\textbf{(0.010)}}} & \mc{1}{c}{\scriptsize{(0.634)}} & \mc{1}{c}{\scriptsize{(0.139)}} & \mc{1}{c}{\scriptsize{(0.515)}} & \mc{1}{c}{\scriptsize{\textbf{(0.059)}}} & \mc{1}{c}{\scriptsize{\textbf{(0.030)}}} & \mc{1}{c}{\scriptsize{\textbf{(0.050)}}} \\  

     & \mc{1}{c}{\scriptsize{12}} & \mc{1}{c}{\scriptsize{0.003}} & \mc{1}{c}{\scriptsize{3.594}} & \mc{1}{c}{\scriptsize{5.057}} & \mc{1}{c}{\scriptsize{6.123}} &  & \mc{1}{c}{\scriptsize{-1.893}} & \mc{1}{c}{\scriptsize{1.201}} &  \\  

     &  & \mc{1}{c}{\scriptsize{(1.000)}} & \mc{1}{c}{\scriptsize{(0.851)}} & \mc{1}{c}{\scriptsize{(0.743)}} & \mc{1}{c}{\scriptsize{(0.832)}} &  & \mc{1}{c}{\scriptsize{(1.000)}} & \mc{1}{c}{\scriptsize{(1.000)}} &  \\  

     & \mc{1}{c}{\scriptsize{15}} & \mc{1}{c}{\scriptsize{5.163}} & \mc{1}{c}{\scriptsize{3.641}} & \mc{1}{c}{\scriptsize{5.177}} & \mc{1}{c}{\scriptsize{5.378}} & \mc{1}{c}{\scriptsize{4.151}} & \mc{1}{c}{\scriptsize{5.424}} & \mc{1}{c}{\scriptsize{3.200}} & \mc{1}{c}{\scriptsize{4.150}} \\  

     &  & \mc{1}{c}{\scriptsize{\textbf{(0.030)}}} & \mc{1}{c}{\scriptsize{(0.238)}} & \mc{1}{c}{\scriptsize{(0.307)}} & \mc{1}{c}{\scriptsize{(0.554)}} & \mc{1}{c}{\scriptsize{(0.545)}} & \mc{1}{c}{\scriptsize{\textbf{(0.069)}}} & \mc{1}{c}{\scriptsize{(0.396)}} & \mc{1}{c}{\scriptsize{(0.218)}} \\  

     & \mc{1}{c}{\scriptsize{21}} & \mc{1}{c}{\scriptsize{5.217}} & \mc{1}{c}{\scriptsize{2.253}} & \mc{1}{c}{\scriptsize{4.504}} & \mc{1}{c}{\scriptsize{2.087}} & \mc{1}{c}{\scriptsize{2.834}} & \mc{1}{c}{\scriptsize{5.521}} & \mc{1}{c}{\scriptsize{2.027}} & \mc{1}{c}{\scriptsize{3.493}} \\  

     &  & \mc{1}{c}{\scriptsize{\textbf{(0.079)}}} & \mc{1}{c}{\scriptsize{(0.644)}} & \mc{1}{c}{\scriptsize{(0.505)}} & \mc{1}{c}{\scriptsize{(0.931)}} & \mc{1}{c}{\scriptsize{(0.812)}} & \mc{1}{c}{\scriptsize{(0.109)}} & \mc{1}{c}{\scriptsize{(0.802)}} & \mc{1}{c}{\scriptsize{(0.535)}} \\  

    \mc{1}{l}{\scriptsize{Achievement Factor}} & \mc{1}{c}{\scriptsize{5.5 to 12}} & \mc{1}{c}{\scriptsize{0.512}} & \mc{1}{c}{\scriptsize{0.621}} & \mc{1}{c}{\scriptsize{0.634}} & \mc{1}{c}{\scriptsize{0.900}} & \mc{1}{c}{\scriptsize{0.688}} & \mc{1}{c}{\scriptsize{0.474}} & \mc{1}{c}{\scriptsize{0.547}} & \mc{1}{c}{\scriptsize{0.515}} \\  

     &  & \mc{1}{c}{\scriptsize{\textbf{(0.030)}}} & \mc{1}{c}{\scriptsize{\textbf{(0.030)}}} & \mc{1}{c}{\scriptsize{(0.307)}} & \mc{1}{c}{\scriptsize{\textbf{(0.089)}}} & \mc{1}{c}{\scriptsize{(0.257)}} & \mc{1}{c}{\scriptsize{\textbf{(0.010)}}} & \mc{1}{c}{\scriptsize{\textbf{(0.040)}}} & \mc{1}{c}{\scriptsize{(0.168)}} \\  

     & \mc{1}{c}{\scriptsize{15 to 21}} & \mc{1}{c}{\scriptsize{-0.460}} & \mc{1}{c}{\scriptsize{-0.265}} & \mc{1}{c}{\scriptsize{-0.431}} & \mc{1}{c}{\scriptsize{-0.340}} & \mc{1}{c}{\scriptsize{-0.313}} & \mc{1}{c}{\scriptsize{-0.485}} & \mc{1}{c}{\scriptsize{-0.235}} & \mc{1}{c}{\scriptsize{-0.341}} \\  

     &  & \mc{1}{c}{\scriptsize{(1.000)}} & \mc{1}{c}{\scriptsize{(1.000)}} & \mc{1}{c}{\scriptsize{(1.000)}} & \mc{1}{c}{\scriptsize{(1.000)}} & \mc{1}{c}{\scriptsize{(1.000)}} & \mc{1}{c}{\scriptsize{(1.000)}} & \mc{1}{c}{\scriptsize{(1.000)}} & \mc{1}{c}{\scriptsize{(1.000)}} \\  

  \bottomrule
  \end{tabular}
	\end{table} 

	\begin{table}[H]
     \caption{Treatment Effects on Infant Behavior Record, Pooled Sample}
     \label{table:abccare_rslt_pooled_cat2_sd}
	  \begin{tabular}{cccccccccc}
  \toprule

    \scriptsize{Variable} & \scriptsize{Age} & \scriptsize{(1)} & \scriptsize{(2)} & \scriptsize{(3)} & \scriptsize{(4)} & \scriptsize{(5)} & \scriptsize{(6)} & \scriptsize{(7)} & \scriptsize{(8)} \\ 
    \midrule  

    \mc{1}{l}{\scriptsize{HOME Score}} & \mc{1}{c}{\scriptsize{0.5}} & \mc{1}{c}{\scriptsize{1.144}} & \mc{1}{c}{\scriptsize{0.848}} & \mc{1}{c}{\scriptsize{1.702}} & \mc{1}{c}{\scriptsize{1.338}} & \mc{1}{c}{\scriptsize{1.155}} & \mc{1}{c}{\scriptsize{0.693}} & \mc{1}{c}{\scriptsize{0.866}} & \mc{1}{c}{\scriptsize{0.286}} \\  

     &  & \mc{1}{c}{\scriptsize{(0.211)}} & \mc{1}{c}{\scriptsize{(0.592)}} & \mc{1}{c}{\scriptsize{\textbf{(0.066)}}} & \mc{1}{c}{\scriptsize{(0.276)}} & \mc{1}{c}{\scriptsize{(0.276)}} & \mc{1}{c}{\scriptsize{(0.645)}} & \mc{1}{c}{\scriptsize{(0.684)}} & \mc{1}{c}{\scriptsize{(0.842)}} \\  

     & \mc{1}{c}{\scriptsize{1.5}} & \mc{1}{c}{\scriptsize{0.936}} & \mc{1}{c}{\scriptsize{0.743}} & \mc{1}{c}{\scriptsize{2.251}} & \mc{1}{c}{\scriptsize{2.135}} & \mc{1}{c}{\scriptsize{2.136}} & \mc{1}{c}{\scriptsize{0.230}} & \mc{1}{c}{\scriptsize{0.691}} & \mc{1}{c}{\scriptsize{0.257}} \\  

     &  & \mc{1}{c}{\scriptsize{(0.487)}} & \mc{1}{c}{\scriptsize{(0.618)}} & \mc{1}{c}{\scriptsize{\textbf{(0.092)}}} & \mc{1}{c}{\scriptsize{(0.171)}} & \mc{1}{c}{\scriptsize{(0.158)}} & \mc{1}{c}{\scriptsize{(0.803)}} & \mc{1}{c}{\scriptsize{(0.684)}} & \mc{1}{c}{\scriptsize{(0.842)}} \\  

     & \mc{1}{c}{\scriptsize{2.5}} & \mc{1}{c}{\scriptsize{0.154}} & \mc{1}{c}{\scriptsize{0.428}} & \mc{1}{c}{\scriptsize{2.491}} & \mc{1}{c}{\scriptsize{3.294}} & \mc{1}{c}{\scriptsize{2.899}} & \mc{1}{c}{\scriptsize{-0.884}} & \mc{1}{c}{\scriptsize{-0.313}} & \mc{1}{c}{\scriptsize{-0.385}} \\  

     &  & \mc{1}{c}{\scriptsize{(0.829)}} & \mc{1}{c}{\scriptsize{(0.750)}} & \mc{1}{c}{\scriptsize{\textbf{(0.079)}}} & \mc{1}{c}{\scriptsize{\textbf{(0.079)}}} & \mc{1}{c}{\scriptsize{\textbf{(0.053)}}} & \mc{1}{c}{\scriptsize{(1.000)}} & \mc{1}{c}{\scriptsize{(0.921)}} & \mc{1}{c}{\scriptsize{(0.947)}} \\  

     & \mc{1}{c}{\scriptsize{3.5}} & \mc{1}{c}{\scriptsize{1.567}} & \mc{1}{c}{\scriptsize{1.703}} & \mc{1}{c}{\scriptsize{6.496}} & \mc{1}{c}{\scriptsize{8.096}} & \mc{1}{c}{\scriptsize{6.378}} & \mc{1}{c}{\scriptsize{-0.112}} & \mc{1}{c}{\scriptsize{0.158}} & \mc{1}{c}{\scriptsize{-0.145}} \\  

     &  & \mc{1}{c}{\scriptsize{(0.395)}} & \mc{1}{c}{\scriptsize{(0.447)}} & \mc{1}{c}{\scriptsize{\textbf{(0.039)}}} & \mc{1}{c}{\scriptsize{\textbf{(0.013)}}} & \mc{1}{c}{\scriptsize{\textbf{(0.053)}}} & \mc{1}{c}{\scriptsize{(0.855)}} & \mc{1}{c}{\scriptsize{(0.842)}} & \mc{1}{c}{\scriptsize{(0.882)}} \\  

     & \mc{1}{c}{\scriptsize{4.5}} & \mc{1}{c}{\scriptsize{1.879}} & \mc{1}{c}{\scriptsize{1.775}} & \mc{1}{c}{\scriptsize{6.739}} & \mc{1}{c}{\scriptsize{7.178}} & \mc{1}{c}{\scriptsize{6.626}} & \mc{1}{c}{\scriptsize{0.264}} & \mc{1}{c}{\scriptsize{0.807}} & \mc{1}{c}{\scriptsize{0.392}} \\  

     &  & \mc{1}{c}{\scriptsize{(0.329)}} & \mc{1}{c}{\scriptsize{(0.447)}} & \mc{1}{c}{\scriptsize{\textbf{(0.039)}}} & \mc{1}{c}{\scriptsize{\textbf{(0.079)}}} & \mc{1}{c}{\scriptsize{\textbf{(0.053)}}} & \mc{1}{c}{\scriptsize{(0.816)}} & \mc{1}{c}{\scriptsize{(0.697)}} & \mc{1}{c}{\scriptsize{(0.842)}} \\  

     & \mc{1}{c}{\scriptsize{8}} & \mc{1}{c}{\scriptsize{1.296}} & \mc{1}{c}{\scriptsize{1.200}} & \mc{1}{c}{\scriptsize{3.852}} & \mc{1}{c}{\scriptsize{2.912}} & \mc{1}{c}{\scriptsize{4.385}} & \mc{1}{c}{\scriptsize{0.682}} & \mc{1}{c}{\scriptsize{0.932}} & \mc{1}{c}{\scriptsize{1.463}} \\  

     &  & \mc{1}{c}{\scriptsize{(0.487)}} & \mc{1}{c}{\scriptsize{(0.632)}} & \mc{1}{c}{\scriptsize{\textbf{(0.066)}}} & \mc{1}{c}{\scriptsize{(0.184)}} & \mc{1}{c}{\scriptsize{\textbf{(0.066)}}} & \mc{1}{c}{\scriptsize{(0.789)}} & \mc{1}{c}{\scriptsize{(0.697)}} & \mc{1}{c}{\scriptsize{(0.487)}} \\  

    \mc{1}{l}{\scriptsize{HOME Factor}} & \mc{1}{c}{\scriptsize{0.5 to 8}} & \mc{1}{c}{\scriptsize{0.243}} & \mc{1}{c}{\scriptsize{0.152}} & \mc{1}{c}{\scriptsize{0.792}} & \mc{1}{c}{\scriptsize{0.755}} & \mc{1}{c}{\scriptsize{0.794}} & \mc{1}{c}{\scriptsize{0.101}} & \mc{1}{c}{\scriptsize{0.036}} & \mc{1}{c}{\scriptsize{0.135}} \\  

     &  & \mc{1}{c}{\scriptsize{(0.316)}} & \mc{1}{c}{\scriptsize{(0.645)}} & \mc{1}{c}{\scriptsize{\textbf{(0.026)}}} & \mc{1}{c}{\scriptsize{\textbf{(0.092)}}} & \mc{1}{c}{\scriptsize{\textbf{(0.026)}}} & \mc{1}{c}{\scriptsize{(0.724)}} & \mc{1}{c}{\scriptsize{(0.789)}} & \mc{1}{c}{\scriptsize{(0.671)}} \\  

  \bottomrule
  \end{tabular}
	\end{table} 

	\begin{table}[H]
     \caption{Treatment Effects on Kohn and Rosman: Attentive/Cooperative, Pooled Sample}
     \label{table:abccare_rslt_pooled_cat3_sd}
	  \begin{tabular}{cccccccccc}
  \toprule

    \scriptsize{Variable} & \scriptsize{Age} & \scriptsize{(1)} & \scriptsize{(2)} & \scriptsize{(3)} & \scriptsize{(4)} & \scriptsize{(5)} & \scriptsize{(6)} & \scriptsize{(7)} & \scriptsize{(8)} \\ 
    \midrule  

    \mc{1}{l}{\scriptsize{Cocaine: Smokes Reguarly}} & \mc{1}{c}{\scriptsize{30}} & \mc{1}{c}{\scriptsize{-0.028}} & \mc{1}{c}{\scriptsize{-0.022}} & \mc{1}{c}{\scriptsize{-0.139}} & \mc{1}{c}{\scriptsize{-0.137}} & \mc{1}{c}{\scriptsize{-0.145}} & \mc{1}{c}{\scriptsize{0.021}} & \mc{1}{c}{\scriptsize{0.017}} & \mc{1}{c}{\scriptsize{0.013}} \\  

     &  & \mc{1}{c}{\scriptsize{(0.803)}} & \mc{1}{c}{\scriptsize{(0.829)}} & \mc{1}{c}{\scriptsize{(0.526)}} & \mc{1}{c}{\scriptsize{(0.513)}} & \mc{1}{c}{\scriptsize{(0.553)}} & \mc{1}{c}{\scriptsize{(1.000)}} & \mc{1}{c}{\scriptsize{(1.000)}} & \mc{1}{c}{\scriptsize{(1.000)}} \\  

    \mc{1}{l}{\scriptsize{Marijuana: Times Used}} & \mc{1}{c}{\scriptsize{30}} & \mc{1}{c}{\scriptsize{0.011}} & \mc{1}{c}{\scriptsize{0.056}} & \mc{1}{c}{\scriptsize{-0.728}} & \mc{1}{c}{\scriptsize{-0.593}} & \mc{1}{c}{\scriptsize{-0.698}} & \mc{1}{c}{\scriptsize{0.343}} & \mc{1}{c}{\scriptsize{0.251}} & \mc{1}{c}{\scriptsize{0.404}} \\  

     &  & \mc{1}{c}{\scriptsize{(0.987)}} & \mc{1}{c}{\scriptsize{(0.987)}} & \mc{1}{c}{\scriptsize{(0.539)}} & \mc{1}{c}{\scriptsize{(0.816)}} & \mc{1}{c}{\scriptsize{(0.605)}} & \mc{1}{c}{\scriptsize{(1.000)}} & \mc{1}{c}{\scriptsize{(1.000)}} & \mc{1}{c}{\scriptsize{(1.000)}} \\  

    \mc{1}{l}{\scriptsize{Marijuana: Smokes Regularly}} & \mc{1}{c}{\scriptsize{30}} & \mc{1}{c}{\scriptsize{-0.116}} & \mc{1}{c}{\scriptsize{-0.147}} & \mc{1}{c}{\scriptsize{-0.189}} & \mc{1}{c}{\scriptsize{-0.235}} & \mc{1}{c}{\scriptsize{-0.194}} & \mc{1}{c}{\scriptsize{-0.085}} & \mc{1}{c}{\scriptsize{-0.092}} & \mc{1}{c}{\scriptsize{-0.075}} \\  

     &  & \mc{1}{c}{\scriptsize{(0.158)}} & \mc{1}{c}{\scriptsize{(0.118)}} & \mc{1}{c}{\scriptsize{(0.395)}} & \mc{1}{c}{\scriptsize{(0.237)}} & \mc{1}{c}{\scriptsize{(0.408)}} & \mc{1}{c}{\scriptsize{(0.487)}} & \mc{1}{c}{\scriptsize{(0.618)}} & \mc{1}{c}{\scriptsize{(0.645)}} \\  

    \mc{1}{l}{\scriptsize{Cocaine: Times Used}} & \mc{1}{c}{\scriptsize{30}} & \mc{1}{c}{\scriptsize{-0.246}} & \mc{1}{c}{\scriptsize{-0.217}} & \mc{1}{c}{\scriptsize{-0.872}} & \mc{1}{c}{\scriptsize{-0.833}} & \mc{1}{c}{\scriptsize{-0.876}} & \mc{1}{c}{\scriptsize{0.017}} & \mc{1}{c}{\scriptsize{-0.013}} & \mc{1}{c}{\scriptsize{0.006}} \\  

     &  & \mc{1}{c}{\scriptsize{(0.447)}} & \mc{1}{c}{\scriptsize{(0.724)}} & \mc{1}{c}{\scriptsize{(0.395)}} & \mc{1}{c}{\scriptsize{(0.395)}} & \mc{1}{c}{\scriptsize{(0.421)}} & \mc{1}{c}{\scriptsize{(1.000)}} & \mc{1}{c}{\scriptsize{(0.974)}} & \mc{1}{c}{\scriptsize{(0.961)}} \\  

    \mc{1}{l}{\scriptsize{Marijuana: Times Used in Past 30 Days}} & \mc{1}{c}{\scriptsize{30}} & \mc{1}{c}{\scriptsize{-0.442}} & \mc{1}{c}{\scriptsize{-0.491}} & \mc{1}{c}{\scriptsize{-0.665}} & \mc{1}{c}{\scriptsize{-0.725}} & \mc{1}{c}{\scriptsize{-0.640}} & \mc{1}{c}{\scriptsize{-0.327}} & \mc{1}{c}{\scriptsize{-0.267}} & \mc{1}{c}{\scriptsize{-0.223}} \\  

     &  & \mc{1}{c}{\scriptsize{(0.263)}} & \mc{1}{c}{\scriptsize{(0.342)}} & \mc{1}{c}{\scriptsize{(0.526)}} & \mc{1}{c}{\scriptsize{(0.513)}} & \mc{1}{c}{\scriptsize{(0.553)}} & \mc{1}{c}{\scriptsize{(0.566)}} & \mc{1}{c}{\scriptsize{(0.737)}} & \mc{1}{c}{\scriptsize{(0.763)}} \\  

    \mc{1}{l}{\scriptsize{Cocaine: Times Used in Past 30 Days}} & \mc{1}{c}{\scriptsize{21}} &  &  &  &  &  &  &  &  \\  

     &  &  &  &  &  &  &  &  &  \\  

    \mc{1}{l}{\scriptsize{Times Used Other Illegal Drugs in Past 30 Days}} & \mc{1}{c}{\scriptsize{21}} & \mc{1}{c}{\scriptsize{-0.079}} & \mc{1}{c}{\scriptsize{-0.072}} & \mc{1}{c}{\scriptsize{-0.087}} & \mc{1}{c}{\scriptsize{-0.063}} & \mc{1}{c}{\scriptsize{-0.078}} & \mc{1}{c}{\scriptsize{-0.084}} & \mc{1}{c}{\scriptsize{-0.090}} & \mc{1}{c}{\scriptsize{-0.082}} \\  

     &  & \mc{1}{c}{\scriptsize{(0.447)}} & \mc{1}{c}{\scriptsize{(0.671)}} & \mc{1}{c}{\scriptsize{(0.645)}} & \mc{1}{c}{\scriptsize{(0.816)}} & \mc{1}{c}{\scriptsize{(0.711)}} & \mc{1}{c}{\scriptsize{(0.592)}} & \mc{1}{c}{\scriptsize{(0.645)}} & \mc{1}{c}{\scriptsize{(0.671)}} \\  

    \mc{1}{l}{\scriptsize{ASR Substance Use Scale: Alcohol}} & \mc{1}{c}{\scriptsize{30}} & \mc{1}{c}{\scriptsize{0.679}} & \mc{1}{c}{\scriptsize{1.040}} & \mc{1}{c}{\scriptsize{0.794}} & \mc{1}{c}{\scriptsize{1.214}} & \mc{1}{c}{\scriptsize{1.004}} & \mc{1}{c}{\scriptsize{0.725}} & \mc{1}{c}{\scriptsize{1.100}} & \mc{1}{c}{\scriptsize{1.254}} \\  

     &  & \mc{1}{c}{\scriptsize{(1.000)}} & \mc{1}{c}{\scriptsize{(1.000)}} & \mc{1}{c}{\scriptsize{(1.000)}} & \mc{1}{c}{\scriptsize{(1.000)}} & \mc{1}{c}{\scriptsize{(1.000)}} & \mc{1}{c}{\scriptsize{(1.000)}} & \mc{1}{c}{\scriptsize{(1.000)}} & \mc{1}{c}{\scriptsize{(1.000)}} \\  

    \mc{1}{l}{\scriptsize{Marijuana: Times Used in Past 30 Days}} & \mc{1}{c}{\scriptsize{21}} & \mc{1}{c}{\scriptsize{-0.621}} & \mc{1}{c}{\scriptsize{-0.550}} & \mc{1}{c}{\scriptsize{-0.385}} & \mc{1}{c}{\scriptsize{-0.215}} & \mc{1}{c}{\scriptsize{-0.410}} & \mc{1}{c}{\scriptsize{-0.644}} & \mc{1}{c}{\scriptsize{-0.559}} & \mc{1}{c}{\scriptsize{-0.527}} \\  

     &  & \mc{1}{c}{\scriptsize{\textbf{(0.039)}}} & \mc{1}{c}{\scriptsize{(0.342)}} & \mc{1}{c}{\scriptsize{(0.789)}} & \mc{1}{c}{\scriptsize{(0.921)}} & \mc{1}{c}{\scriptsize{(0.763)}} & \mc{1}{c}{\scriptsize{\textbf{(0.079)}}} & \mc{1}{c}{\scriptsize{(0.368)}} & \mc{1}{c}{\scriptsize{(0.316)}} \\  

    \mc{1}{l}{\scriptsize{ASR Substance Use Scale: Mean Substance Abuse}} & \mc{1}{c}{\scriptsize{30}} & \mc{1}{c}{\scriptsize{0.661}} & \mc{1}{c}{\scriptsize{1.306}} & \mc{1}{c}{\scriptsize{0.495}} & \mc{1}{c}{\scriptsize{1.976}} & \mc{1}{c}{\scriptsize{0.720}} & \mc{1}{c}{\scriptsize{0.817}} & \mc{1}{c}{\scriptsize{1.601}} & \mc{1}{c}{\scriptsize{1.336}} \\  

     &  & \mc{1}{c}{\scriptsize{(1.000)}} & \mc{1}{c}{\scriptsize{(1.000)}} & \mc{1}{c}{\scriptsize{(1.000)}} & \mc{1}{c}{\scriptsize{(1.000)}} & \mc{1}{c}{\scriptsize{(1.000)}} & \mc{1}{c}{\scriptsize{(1.000)}} & \mc{1}{c}{\scriptsize{(1.000)}} & \mc{1}{c}{\scriptsize{(1.000)}} \\  

    \mc{1}{l}{\scriptsize{Cocaine: Number of Times Used Crack Cocaine}} & \mc{1}{c}{\scriptsize{30}} & \mc{1}{c}{\scriptsize{-0.162}} & \mc{1}{c}{\scriptsize{-0.037}} & \mc{1}{c}{\scriptsize{-0.923}} & \mc{1}{c}{\scriptsize{-0.815}} & \mc{1}{c}{\scriptsize{-0.908}} & \mc{1}{c}{\scriptsize{0.057}} & \mc{1}{c}{\scriptsize{0.117}} & \mc{1}{c}{\scriptsize{0.072}} \\  

     &  & \mc{1}{c}{\scriptsize{(0.566)}} & \mc{1}{c}{\scriptsize{(0.947)}} & \mc{1}{c}{\scriptsize{(0.395)}} & \mc{1}{c}{\scriptsize{(0.421)}} & \mc{1}{c}{\scriptsize{(0.395)}} & \mc{1}{c}{\scriptsize{(1.000)}} & \mc{1}{c}{\scriptsize{(1.000)}} & \mc{1}{c}{\scriptsize{(1.000)}} \\  

    \mc{1}{l}{\scriptsize{ASR Substance Use Scale: Tobacco}} & \mc{1}{c}{\scriptsize{30}} & \mc{1}{c}{\scriptsize{0.257}} & \mc{1}{c}{\scriptsize{0.341}} & \mc{1}{c}{\scriptsize{-1.226}} & \mc{1}{c}{\scriptsize{-0.522}} & \mc{1}{c}{\scriptsize{-1.372}} & \mc{1}{c}{\scriptsize{0.746}} & \mc{1}{c}{\scriptsize{0.773}} & \mc{1}{c}{\scriptsize{0.714}} \\  

     &  & \mc{1}{c}{\scriptsize{(1.000)}} & \mc{1}{c}{\scriptsize{(1.000)}} & \mc{1}{c}{\scriptsize{(0.776)}} & \mc{1}{c}{\scriptsize{(0.934)}} & \mc{1}{c}{\scriptsize{(0.711)}} & \mc{1}{c}{\scriptsize{(1.000)}} & \mc{1}{c}{\scriptsize{(1.000)}} & \mc{1}{c}{\scriptsize{(1.000)}} \\  

    \mc{1}{l}{\scriptsize{Marijuana: Smokes Regularly}} & \mc{1}{c}{\scriptsize{Mid-30s}} & \mc{1}{c}{\scriptsize{-0.026}} & \mc{1}{c}{\scriptsize{-0.085}} & \mc{1}{c}{\scriptsize{-0.112}} & \mc{1}{c}{\scriptsize{-0.134}} & \mc{1}{c}{\scriptsize{-0.111}} & \mc{1}{c}{\scriptsize{0.026}} & \mc{1}{c}{\scriptsize{-0.019}} & \mc{1}{c}{\scriptsize{-0.007}} \\  

     &  & \mc{1}{c}{\scriptsize{(0.908)}} & \mc{1}{c}{\scriptsize{(0.724)}} & \mc{1}{c}{\scriptsize{(0.816)}} & \mc{1}{c}{\scriptsize{(0.697)}} & \mc{1}{c}{\scriptsize{(0.816)}} & \mc{1}{c}{\scriptsize{(1.000)}} & \mc{1}{c}{\scriptsize{(0.947)}} & \mc{1}{c}{\scriptsize{(0.921)}} \\  

    \mc{1}{l}{\scriptsize{ASR Substance Use Scale: Drugs}} & \mc{1}{c}{\scriptsize{30}} & \mc{1}{c}{\scriptsize{-0.539}} & \mc{1}{c}{\scriptsize{-0.134}} & \mc{1}{c}{\scriptsize{-3.599}} & \mc{1}{c}{\scriptsize{-1.465}} & \mc{1}{c}{\scriptsize{-3.307}} & \mc{1}{c}{\scriptsize{0.399}} & \mc{1}{c}{\scriptsize{1.123}} & \mc{1}{c}{\scriptsize{1.023}} \\  

     &  & \mc{1}{c}{\scriptsize{(0.908)}} & \mc{1}{c}{\scriptsize{(0.961)}} & \mc{1}{c}{\scriptsize{(0.645)}} & \mc{1}{c}{\scriptsize{(0.921)}} & \mc{1}{c}{\scriptsize{(0.697)}} & \mc{1}{c}{\scriptsize{(1.000)}} & \mc{1}{c}{\scriptsize{(1.000)}} & \mc{1}{c}{\scriptsize{(1.000)}} \\  

  \bottomrule
  \end{tabular}
	\end{table} 

	\begin{table}[H]
     \caption{Treatment Effects on Classroom Behavior Inventory (Part I), Pooled Sample}
     \label{table:abccare_rslt_pooled_cat4_sd}
	  \begin{tabular}{cccccccccc}
  \toprule

    \scriptsize{Variable} & \scriptsize{Age} & \scriptsize{(1)} & \scriptsize{(2)} & \scriptsize{(3)} & \scriptsize{(4)} & \scriptsize{(5)} & \scriptsize{(6)} & \scriptsize{(7)} & \scriptsize{(8)} \\ 
    \midrule  

    \mc{1}{l}{\scriptsize{Mother Works}} & \mc{1}{c}{\scriptsize{2}} & \mc{1}{c}{\scriptsize{0.140}} & \mc{1}{c}{\scriptsize{0.122}} & \mc{1}{c}{\scriptsize{0.325}} & \mc{1}{c}{\scriptsize{0.326}} & \mc{1}{c}{\scriptsize{0.397}} & \mc{1}{c}{\scriptsize{0.075}} & \mc{1}{c}{\scriptsize{0.066}} & \mc{1}{c}{\scriptsize{0.049}} \\  

     &  & \mc{1}{c}{\scriptsize{\textbf{(0.053)}}} & \mc{1}{c}{\scriptsize{(0.224)}} & \mc{1}{c}{\scriptsize{\textbf{(0.026)}}} & \mc{1}{c}{\scriptsize{\textbf{(0.026)}}} & \mc{1}{c}{\scriptsize{\textbf{(0.013)}}} & \mc{1}{c}{\scriptsize{(0.500)}} & \mc{1}{c}{\scriptsize{(0.579)}} & \mc{1}{c}{\scriptsize{(0.658)}} \\  

     & \mc{1}{c}{\scriptsize{3}} & \mc{1}{c}{\scriptsize{0.133}} & \mc{1}{c}{\scriptsize{0.135}} & \mc{1}{c}{\scriptsize{0.323}} & \mc{1}{c}{\scriptsize{0.326}} & \mc{1}{c}{\scriptsize{0.397}} & \mc{1}{c}{\scriptsize{0.084}} & \mc{1}{c}{\scriptsize{0.078}} & \mc{1}{c}{\scriptsize{0.060}} \\  

     &  & \mc{1}{c}{\scriptsize{\textbf{(0.079)}}} & \mc{1}{c}{\scriptsize{(0.197)}} & \mc{1}{c}{\scriptsize{\textbf{(0.026)}}} & \mc{1}{c}{\scriptsize{\textbf{(0.026)}}} & \mc{1}{c}{\scriptsize{\textbf{(0.013)}}} & \mc{1}{c}{\scriptsize{(0.421)}} & \mc{1}{c}{\scriptsize{(0.526)}} & \mc{1}{c}{\scriptsize{(0.605)}} \\  

     & \mc{1}{c}{\scriptsize{4}} & \mc{1}{c}{\scriptsize{0.140}} & \mc{1}{c}{\scriptsize{0.136}} & \mc{1}{c}{\scriptsize{0.320}} & \mc{1}{c}{\scriptsize{0.326}} & \mc{1}{c}{\scriptsize{0.391}} & \mc{1}{c}{\scriptsize{0.096}} & \mc{1}{c}{\scriptsize{0.083}} & \mc{1}{c}{\scriptsize{0.069}} \\  

     &  & \mc{1}{c}{\scriptsize{\textbf{(0.079)}}} & \mc{1}{c}{\scriptsize{(0.184)}} & \mc{1}{c}{\scriptsize{\textbf{(0.039)}}} & \mc{1}{c}{\scriptsize{\textbf{(0.026)}}} & \mc{1}{c}{\scriptsize{\textbf{(0.013)}}} & \mc{1}{c}{\scriptsize{(0.368)}} & \mc{1}{c}{\scriptsize{(0.447)}} & \mc{1}{c}{\scriptsize{(0.592)}} \\  

     & \mc{1}{c}{\scriptsize{5}} & \mc{1}{c}{\scriptsize{0.128}} & \mc{1}{c}{\scriptsize{0.059}} & \mc{1}{c}{\scriptsize{0.463}} & \mc{1}{c}{\scriptsize{0.374}} & \mc{1}{c}{\scriptsize{0.545}} & \mc{1}{c}{\scriptsize{0.021}} & \mc{1}{c}{\scriptsize{-0.032}} & \mc{1}{c}{\scriptsize{0.020}} \\  

     &  & \mc{1}{c}{\scriptsize{\textbf{(0.092)}}} & \mc{1}{c}{\scriptsize{(0.605)}} & \mc{1}{c}{\scriptsize{\textbf{(0.026)}}} & \mc{1}{c}{\scriptsize{\textbf{(0.079)}}} & \mc{1}{c}{\scriptsize{\textbf{(0.000)}}} & \mc{1}{c}{\scriptsize{(0.763)}} & \mc{1}{c}{\scriptsize{(0.947)}} & \mc{1}{c}{\scriptsize{(0.803)}} \\  

     & \mc{1}{c}{\scriptsize{21}} & \mc{1}{c}{\scriptsize{-0.040}} & \mc{1}{c}{\scriptsize{-0.042}} & \mc{1}{c}{\scriptsize{0.180}} & \mc{1}{c}{\scriptsize{0.094}} & \mc{1}{c}{\scriptsize{0.100}} & \mc{1}{c}{\scriptsize{-0.075}} & \mc{1}{c}{\scriptsize{-0.084}} & \mc{1}{c}{\scriptsize{-0.093}} \\  

     &  & \mc{1}{c}{\scriptsize{(0.895)}} & \mc{1}{c}{\scriptsize{(0.921)}} & \mc{1}{c}{\scriptsize{(0.250)}} & \mc{1}{c}{\scriptsize{(0.566)}} & \mc{1}{c}{\scriptsize{(0.224)}} & \mc{1}{c}{\scriptsize{(0.961)}} & \mc{1}{c}{\scriptsize{(0.974)}} & \mc{1}{c}{\scriptsize{(0.974)}} \\  

    \mc{1}{l}{\scriptsize{Mother Works Factor}} & \mc{1}{c}{\scriptsize{2 to 21}} & \mc{1}{c}{\scriptsize{0.236}} & \mc{1}{c}{\scriptsize{0.216}} & \mc{1}{c}{\scriptsize{0.848}} & \mc{1}{c}{\scriptsize{0.696}} & \mc{1}{c}{\scriptsize{0.901}} & \mc{1}{c}{\scriptsize{0.113}} & \mc{1}{c}{\scriptsize{0.068}} & \mc{1}{c}{\scriptsize{0.092}} \\  

     &  & \mc{1}{c}{\scriptsize{(0.263)}} & \mc{1}{c}{\scriptsize{(0.526)}} & \mc{1}{c}{\scriptsize{(0.118)}} & \mc{1}{c}{\scriptsize{(0.118)}} & \mc{1}{c}{\scriptsize{\textbf{(0.079)}}} & \mc{1}{c}{\scriptsize{(0.671)}} & \mc{1}{c}{\scriptsize{(0.697)}} & \mc{1}{c}{\scriptsize{(0.763)}} \\  

  \bottomrule
  \end{tabular}
	\end{table} 

	\begin{table}[H]
     \caption{Treatment Effects on Classroom Behavior Inventory (Part II), Pooled Sample}
     \label{table:abccare_rslt_pooled_cat5_sd}
	  \begin{tabular}{cccccccccc}
  \toprule

    \scriptsize{Variable} & \scriptsize{Age} & \scriptsize{(1)} & \scriptsize{(2)} & \scriptsize{(3)} & \scriptsize{(4)} & \scriptsize{(5)} & \scriptsize{(6)} & \scriptsize{(7)} & \scriptsize{(8)} \\ 
    \midrule  

    \mc{1}{l}{\scriptsize{Father at Home}} & \mc{1}{c}{\scriptsize{2}} & \mc{1}{c}{\scriptsize{-0.010}} & \mc{1}{c}{\scriptsize{0.019}} & \mc{1}{c}{\scriptsize{-0.187}} & \mc{1}{c}{\scriptsize{-0.186}} & \mc{1}{c}{\scriptsize{-0.173}} & \mc{1}{c}{\scriptsize{0.047}} & \mc{1}{c}{\scriptsize{0.102}} & \mc{1}{c}{\scriptsize{0.130}} \\  

     &  & \mc{1}{c}{\scriptsize{(0.523)}} & \mc{1}{c}{\scriptsize{(0.714)}} & \mc{1}{c}{\scriptsize{(0.144)}} & \mc{1}{c}{\scriptsize{(0.126)}} & \mc{1}{c}{\scriptsize{(0.207)}} & \mc{1}{c}{\scriptsize{(0.667)}} & \mc{1}{c}{\scriptsize{(0.233)}} & \mc{1}{c}{\scriptsize{(0.135)}} \\  

     & \mc{1}{c}{\scriptsize{3}} & \mc{1}{c}{\scriptsize{-0.076}} & \mc{1}{c}{\scriptsize{-0.056}} & \mc{1}{c}{\scriptsize{-0.291}} & \mc{1}{c}{\scriptsize{-0.291}} & \mc{1}{c}{\scriptsize{-0.285}} & \mc{1}{c}{\scriptsize{0.002}} & \mc{1}{c}{\scriptsize{0.040}} & \mc{1}{c}{\scriptsize{0.079}} \\  

     &  & \mc{1}{c}{\scriptsize{(0.377)}} & \mc{1}{c}{\scriptsize{(0.618)}} & \mc{1}{c}{\scriptsize{\textbf{(0.035)}}} & \mc{1}{c}{\scriptsize{\textbf{(0.034)}}} & \mc{1}{c}{\scriptsize{\textbf{(0.052)}}} & \mc{1}{c}{\scriptsize{(0.855)}} & \mc{1}{c}{\scriptsize{(0.394)}} & \mc{1}{c}{\scriptsize{(0.226)}} \\  

     & \mc{1}{c}{\scriptsize{4}} & \mc{1}{c}{\scriptsize{-0.071}} & \mc{1}{c}{\scriptsize{-0.050}} & \mc{1}{c}{\scriptsize{-0.331}} & \mc{1}{c}{\scriptsize{-0.327}} & \mc{1}{c}{\scriptsize{-0.320}} & \mc{1}{c}{\scriptsize{0.021}} & \mc{1}{c}{\scriptsize{0.054}} & \mc{1}{c}{\scriptsize{0.101}} \\  

     &  & \mc{1}{c}{\scriptsize{(0.427)}} & \mc{1}{c}{\scriptsize{(0.662)}} & \mc{1}{c}{\scriptsize{\textbf{(0.025)}}} & \mc{1}{c}{\scriptsize{\textbf{(0.034)}}} & \mc{1}{c}{\scriptsize{\textbf{(0.025)}}} & \mc{1}{c}{\scriptsize{(0.801)}} & \mc{1}{c}{\scriptsize{(0.363)}} & \mc{1}{c}{\scriptsize{(0.218)}} \\  

     & \mc{1}{c}{\scriptsize{5}} & \mc{1}{c}{\scriptsize{-0.093}} & \mc{1}{c}{\scriptsize{-0.071}} & \mc{1}{c}{\scriptsize{-0.369}} & \mc{1}{c}{\scriptsize{-0.379}} & \mc{1}{c}{\scriptsize{-0.367}} & \mc{1}{c}{\scriptsize{-0.006}} & \mc{1}{c}{\scriptsize{0.029}} & \mc{1}{c}{\scriptsize{0.062}} \\  

     &  & \mc{1}{c}{\scriptsize{(0.321)}} & \mc{1}{c}{\scriptsize{(0.520)}} & \mc{1}{c}{\scriptsize{\textbf{(0.019)}}} & \mc{1}{c}{\scriptsize{\textbf{(0.019)}}} & \mc{1}{c}{\scriptsize{\textbf{(0.013)}}} & \mc{1}{c}{\scriptsize{(0.855)}} & \mc{1}{c}{\scriptsize{(0.394)}} & \mc{1}{c}{\scriptsize{(0.226)}} \\  

     & \mc{1}{c}{\scriptsize{8}} & \mc{1}{c}{\scriptsize{0.052}} & \mc{1}{c}{\scriptsize{-0.009}} & \mc{1}{c}{\scriptsize{-0.124}} & \mc{1}{c}{\scriptsize{-0.183}} & \mc{1}{c}{\scriptsize{-0.181}} & \mc{1}{c}{\scriptsize{0.113}} & \mc{1}{c}{\scriptsize{0.070}} & \mc{1}{c}{\scriptsize{0.096}} \\  

     &  & \mc{1}{c}{\scriptsize{(0.523)}} & \mc{1}{c}{\scriptsize{(0.714)}} & \mc{1}{c}{\scriptsize{(0.200)}} & \mc{1}{c}{\scriptsize{(0.126)}} & \mc{1}{c}{\scriptsize{(0.207)}} & \mc{1}{c}{\scriptsize{(0.244)}} & \mc{1}{c}{\scriptsize{(0.354)}} & \mc{1}{c}{\scriptsize{(0.218)}} \\  

    \mc{1}{l}{\scriptsize{Father at Home Factor}} & \mc{1}{c}{\scriptsize{2 to 8}} & \mc{1}{c}{\scriptsize{-0.139}} & \mc{1}{c}{\scriptsize{-0.129}} & \mc{1}{c}{\scriptsize{-0.776}} & \mc{1}{c}{\scriptsize{-0.801}} & \mc{1}{c}{\scriptsize{-0.781}} & \mc{1}{c}{\scriptsize{0.069}} & \mc{1}{c}{\scriptsize{0.114}} & \mc{1}{c}{\scriptsize{0.241}} \\  

     &  & \mc{1}{c}{\scriptsize{(0.523)}} & \mc{1}{c}{\scriptsize{(0.662)}} & \mc{1}{c}{\scriptsize{\textbf{(0.031)}}} & \mc{1}{c}{\scriptsize{\textbf{(0.034)}}} & \mc{1}{c}{\scriptsize{\textbf{(0.023)}}} & \mc{1}{c}{\scriptsize{(0.794)}} & \mc{1}{c}{\scriptsize{(0.394)}} & \mc{1}{c}{\scriptsize{(0.218)}} \\  

  \bottomrule
  \end{tabular}
	\end{table} 

	\begin{table}[H]
     \caption{Treatment Effects on Emotional, Activity, Sociability, Impulsivity Survey, Pooled Sample}
     \label{table:abccare_rslt_pooled_cat6_sd}
	  \begin{tabular}{cccccccccc}
  \toprule

    \scriptsize{Variable} & \scriptsize{Age} & \scriptsize{(1)} & \scriptsize{(2)} & \scriptsize{(3)} & \scriptsize{(4)} & \scriptsize{(5)} & \scriptsize{(6)} & \scriptsize{(7)} & \scriptsize{(8)} \\ 
    \midrule  

    \mc{1}{l}{\scriptsize{Graduated High School}} & \mc{1}{c}{\scriptsize{30}} & \mc{1}{c}{\scriptsize{0.073}} & \mc{1}{c}{\scriptsize{0.130}} & \mc{1}{c}{\scriptsize{0.114}} & \mc{1}{c}{\scriptsize{0.186}} & \mc{1}{c}{\scriptsize{0.352}} & \mc{1}{c}{\scriptsize{0.077}} & \mc{1}{c}{\scriptsize{0.136}} & \mc{1}{c}{\scriptsize{0.059}} \\  

     &  & \mc{1}{c}{\scriptsize{(6.176)}} & \mc{1}{c}{\scriptsize{(3.706)}} & \mc{1}{c}{\scriptsize{(9.824)}} & \mc{1}{c}{\scriptsize{(9.824)}} & \mc{1}{c}{\scriptsize{(0.382)}} & \mc{1}{c}{\scriptsize{(5.578)}} & \mc{1}{c}{\scriptsize{(3.706)}} & \mc{1}{c}{\scriptsize{(4.353)}} \\  

    \mc{1}{l}{\scriptsize{Attended Voc./Tech./Com. College}} & \mc{1}{c}{\scriptsize{30}} & \mc{1}{c}{\scriptsize{-0.099}} & \mc{1}{c}{\scriptsize{-0.147}} & \mc{1}{c}{\scriptsize{0.086}} & \mc{1}{c}{\scriptsize{0.188}} & \mc{1}{c}{\scriptsize{-0.044}} & \mc{1}{c}{\scriptsize{-0.138}} & \mc{1}{c}{\scriptsize{-0.229}} & \mc{1}{c}{\scriptsize{-0.152}} \\  

     &  & \mc{1}{c}{\scriptsize{(6.176)}} & \mc{1}{c}{\scriptsize{(3.706)}} & \mc{1}{c}{\scriptsize{(9.824)}} & \mc{1}{c}{\scriptsize{(9.824)}} & \mc{1}{c}{\scriptsize{(5.696)}} & \mc{1}{c}{\scriptsize{(5.333)}} & \mc{1}{c}{\scriptsize{(3.186)}} & \mc{1}{c}{\scriptsize{(2.882)}} \\  

    \mc{1}{l}{\scriptsize{Graduated 4-year College}} & \mc{1}{c}{\scriptsize{30}} & \mc{1}{c}{\scriptsize{0.170}} & \mc{1}{c}{\scriptsize{0.178}} & \mc{1}{c}{\scriptsize{0.124}} & \mc{1}{c}{\scriptsize{0.347}} & \mc{1}{c}{\scriptsize{0.176}} & \mc{1}{c}{\scriptsize{0.179}} & \mc{1}{c}{\scriptsize{0.167}} & \mc{1}{c}{\scriptsize{0.120}} \\  

     &  & \mc{1}{c}{\scriptsize{(2.676)}} & \mc{1}{c}{\scriptsize{(3.529)}} & \mc{1}{c}{\scriptsize{(9.824)}} & \mc{1}{c}{\scriptsize{(0.686)}} & \mc{1}{c}{\scriptsize{(1.235)}} & \mc{1}{c}{\scriptsize{(2.951)}} & \mc{1}{c}{\scriptsize{(3.706)}} & \mc{1}{c}{\scriptsize{(2.882)}} \\  

    \mc{1}{l}{\scriptsize{Years of Edu.}} & \mc{1}{c}{\scriptsize{30}} & \mc{1}{c}{\scriptsize{0.525}} & \mc{1}{c}{\scriptsize{0.785}} & \mc{1}{c}{\scriptsize{0.857}} & \mc{1}{c}{\scriptsize{1.619}} & \mc{1}{c}{\scriptsize{2.423}} & \mc{1}{c}{\scriptsize{0.385}} & \mc{1}{c}{\scriptsize{0.649}} & \mc{1}{c}{\scriptsize{0.886}} \\  

     &  & \mc{1}{c}{\scriptsize{(5.304)}} & \mc{1}{c}{\scriptsize{(3.451)}} & \mc{1}{c}{\scriptsize{(9.824)}} & \mc{1}{c}{\scriptsize{(9.824)}} & \mc{1}{c}{\scriptsize{\textbf{(0.010)}}} & \mc{1}{c}{\scriptsize{(5.578)}} & \mc{1}{c}{\scriptsize{(3.706)}} & \mc{1}{c}{\scriptsize{(1.108)}} \\  

    \mc{1}{l}{\scriptsize{Ever Had Special Education by Grade 5}} & \mc{1}{c}{\scriptsize{21}} & \mc{1}{c}{\scriptsize{-0.035}} & \mc{1}{c}{\scriptsize{-0.122}} & \mc{1}{c}{\scriptsize{0.158}} & \mc{1}{c}{\scriptsize{0.033}} & \mc{1}{c}{\scriptsize{0.127}} & \mc{1}{c}{\scriptsize{-0.085}} & \mc{1}{c}{\scriptsize{-0.169}} & \mc{1}{c}{\scriptsize{-0.040}} \\  

     &  & \mc{1}{c}{\scriptsize{(6.176)}} & \mc{1}{c}{\scriptsize{(3.706)}} & \mc{1}{c}{\scriptsize{(9.824)}} & \mc{1}{c}{\scriptsize{(9.824)}} & \mc{1}{c}{\scriptsize{(4.529)}} & \mc{1}{c}{\scriptsize{(5.578)}} & \mc{1}{c}{\scriptsize{(3.706)}} & \mc{1}{c}{\scriptsize{(4.353)}} \\  

    \mc{1}{l}{\scriptsize{Total Number of Special Education by Grade 5}} & \mc{1}{c}{\scriptsize{21}} & \mc{1}{c}{\scriptsize{-0.544}} & \mc{1}{c}{\scriptsize{-1.204}} & \mc{1}{c}{\scriptsize{0.019}} & \mc{1}{c}{\scriptsize{-1.713}} & \mc{1}{c}{\scriptsize{0.975}} & \mc{1}{c}{\scriptsize{-0.690}} & \mc{1}{c}{\scriptsize{-1.185}} & \mc{1}{c}{\scriptsize{-0.852}} \\  

     &  & \mc{1}{c}{\scriptsize{(6.176)}} & \mc{1}{c}{\scriptsize{(3.451)}} & \mc{1}{c}{\scriptsize{(9.824)}} & \mc{1}{c}{\scriptsize{(9.824)}} & \mc{1}{c}{\scriptsize{(3.490)}} & \mc{1}{c}{\scriptsize{(5.578)}} & \mc{1}{c}{\scriptsize{(3.706)}} & \mc{1}{c}{\scriptsize{(3.461)}} \\  

    \mc{1}{l}{\scriptsize{Ever Retained by Grade 5}} & \mc{1}{c}{\scriptsize{21}} & \mc{1}{c}{\scriptsize{-0.095}} & \mc{1}{c}{\scriptsize{-0.213}} & \mc{1}{c}{\scriptsize{-0.023}} & \mc{1}{c}{\scriptsize{-0.238}} & \mc{1}{c}{\scriptsize{-0.176}} & \mc{1}{c}{\scriptsize{-0.113}} & \mc{1}{c}{\scriptsize{-0.206}} & \mc{1}{c}{\scriptsize{-0.183}} \\  

     &  & \mc{1}{c}{\scriptsize{(6.108)}} & \mc{1}{c}{\scriptsize{(2.284)}} & \mc{1}{c}{\scriptsize{(9.824)}} & \mc{1}{c}{\scriptsize{(9.824)}} & \mc{1}{c}{\scriptsize{(3.402)}} & \mc{1}{c}{\scriptsize{(5.333)}} & \mc{1}{c}{\scriptsize{(3.118)}} & \mc{1}{c}{\scriptsize{(1.716)}} \\  

    \mc{1}{l}{\scriptsize{Total Number of Retention by Grade 5}} & \mc{1}{c}{\scriptsize{21}} & \mc{1}{c}{\scriptsize{-0.070}} & \mc{1}{c}{\scriptsize{-0.197}} & \mc{1}{c}{\scriptsize{0.031}} & \mc{1}{c}{\scriptsize{-0.215}} & \mc{1}{c}{\scriptsize{-0.068}} & \mc{1}{c}{\scriptsize{-0.096}} & \mc{1}{c}{\scriptsize{-0.190}} & \mc{1}{c}{\scriptsize{-0.156}} \\  

     &  & \mc{1}{c}{\scriptsize{(6.176)}} & \mc{1}{c}{\scriptsize{(3.529)}} & \mc{1}{c}{\scriptsize{(9.824)}} & \mc{1}{c}{\scriptsize{(9.824)}} & \mc{1}{c}{\scriptsize{(5.696)}} & \mc{1}{c}{\scriptsize{(5.578)}} & \mc{1}{c}{\scriptsize{(3.706)}} & \mc{1}{c}{\scriptsize{(3.441)}} \\  

    \mc{1}{l}{\scriptsize{Education Factor}} & \mc{1}{c}{\scriptsize{21 to 30}} & \mc{1}{c}{\scriptsize{0.344}} & \mc{1}{c}{\scriptsize{0.564}} & \mc{1}{c}{\scriptsize{0.230}} & \mc{1}{c}{\scriptsize{1.020}} & \mc{1}{c}{\scriptsize{0.504}} & \mc{1}{c}{\scriptsize{0.385}} & \mc{1}{c}{\scriptsize{0.485}} & \mc{1}{c}{\scriptsize{0.331}} \\  

     &  & \mc{1}{c}{\scriptsize{(4.363)}} & \mc{1}{c}{\scriptsize{(1.853)}} & \mc{1}{c}{\scriptsize{(9.824)}} & \mc{1}{c}{\scriptsize{(9.824)}} & \mc{1}{c}{\scriptsize{(1.843)}} & \mc{1}{c}{\scriptsize{(4.010)}} & \mc{1}{c}{\scriptsize{(2.618)}} & \mc{1}{c}{\scriptsize{(2.882)}} \\  

  \bottomrule
  \end{tabular}
	\end{table} 

	\begin{table}[H]
     \caption{Treatment Effects on Harter Importance, Pooled Sample}
     \label{table:abccare_rslt_pooled_cat7_sd}
	  \begin{tabular}{cccccccccc}
  \toprule

    \scriptsize{Variable} & \scriptsize{Age} & \scriptsize{(1)} & \scriptsize{(2)} & \scriptsize{(3)} & \scriptsize{(4)} & \scriptsize{(5)} & \scriptsize{(6)} & \scriptsize{(7)} & \scriptsize{(8)} \\ 
    \midrule  

    \mc{1}{l}{\scriptsize{Employed}} & \mc{1}{c}{\scriptsize{30}} & \mc{1}{c}{\scriptsize{0.125}} & \mc{1}{c}{\scriptsize{0.132}} & \mc{1}{c}{\scriptsize{0.164}} & \mc{1}{c}{\scriptsize{0.128}} & \mc{1}{c}{\scriptsize{0.205}} & \mc{1}{c}{\scriptsize{0.111}} & \mc{1}{c}{\scriptsize{0.148}} & \mc{1}{c}{\scriptsize{0.163}} \\  

     &  & \mc{1}{c}{\scriptsize{(0.118)}} & \mc{1}{c}{\scriptsize{(0.145)}} & \mc{1}{c}{\scriptsize{(0.355)}} & \mc{1}{c}{\scriptsize{(0.513)}} & \mc{1}{c}{\scriptsize{(0.224)}} & \mc{1}{c}{\scriptsize{(0.316)}} & \mc{1}{c}{\scriptsize{(0.197)}} & \mc{1}{c}{\scriptsize{\textbf{(0.079)}}} \\  

    \mc{1}{l}{\scriptsize{Labor Income}} & \mc{1}{c}{\scriptsize{21}} & \mc{1}{c}{\scriptsize{167}} & \mc{1}{c}{\scriptsize{-579}} & \mc{1}{c}{\scriptsize{1,577}} & \mc{1}{c}{\scriptsize{1,469}} & \mc{1}{c}{\scriptsize{1,749}} & \mc{1}{c}{\scriptsize{-429}} & \mc{1}{c}{\scriptsize{-1,211}} & \mc{1}{c}{\scriptsize{-1,527}} \\  

     &  & \mc{1}{c}{\scriptsize{(0.908)}} & \mc{1}{c}{\scriptsize{(0.961)}} & \mc{1}{c}{\scriptsize{(0.724)}} & \mc{1}{c}{\scriptsize{(0.842)}} & \mc{1}{c}{\scriptsize{(0.829)}} & \mc{1}{c}{\scriptsize{(0.987)}} & \mc{1}{c}{\scriptsize{(0.987)}} & \mc{1}{c}{\scriptsize{(1.000)}} \\  

     & \mc{1}{c}{\scriptsize{30}} & \mc{1}{c}{\scriptsize{12,377}} & \mc{1}{c}{\scriptsize{10,752}} & \mc{1}{c}{\scriptsize{17,677}} & \mc{1}{c}{\scriptsize{13,888}} & \mc{1}{c}{\scriptsize{21,198}} & \mc{1}{c}{\scriptsize{10,847}} & \mc{1}{c}{\scriptsize{9,790}} & \mc{1}{c}{\scriptsize{11,232}} \\  

     &  & \mc{1}{c}{\scriptsize{(0.145)}} & \mc{1}{c}{\scriptsize{(0.658)}} & \mc{1}{c}{\scriptsize{\textbf{(0.092)}}} & \mc{1}{c}{\scriptsize{(0.592)}} & \mc{1}{c}{\scriptsize{(0.118)}} & \mc{1}{c}{\scriptsize{(0.329)}} & \mc{1}{c}{\scriptsize{(0.737)}} & \mc{1}{c}{\scriptsize{(0.513)}} \\  

    \mc{1}{l}{\scriptsize{Public-Transfer Income}} & \mc{1}{c}{\scriptsize{21}} & \mc{1}{c}{\scriptsize{-728}} & \mc{1}{c}{\scriptsize{-923}} & \mc{1}{c}{\scriptsize{-247}} & \mc{1}{c}{\scriptsize{-1,161}} & \mc{1}{c}{\scriptsize{-1,625}} & \mc{1}{c}{\scriptsize{-1,054}} & \mc{1}{c}{\scriptsize{-824}} & \mc{1}{c}{\scriptsize{-813}} \\  

     &  & \mc{1}{c}{\scriptsize{(0.684)}} & \mc{1}{c}{\scriptsize{(0.579)}} & \mc{1}{c}{\scriptsize{(0.789)}} & \mc{1}{c}{\scriptsize{(0.566)}} & \mc{1}{c}{\scriptsize{(0.461)}} & \mc{1}{c}{\scriptsize{(0.605)}} & \mc{1}{c}{\scriptsize{(0.658)}} & \mc{1}{c}{\scriptsize{(0.645)}} \\  

     & \mc{1}{c}{\scriptsize{30}} & \mc{1}{c}{\scriptsize{-1,832}} & \mc{1}{c}{\scriptsize{-1,005}} & \mc{1}{c}{\scriptsize{-1,613}} & \mc{1}{c}{\scriptsize{-1,480}} & \mc{1}{c}{\scriptsize{-1,600}} & \mc{1}{c}{\scriptsize{-1,483}} & \mc{1}{c}{\scriptsize{-992}} & \mc{1}{c}{\scriptsize{-1,680}} \\  

     &  & \mc{1}{c}{\scriptsize{(0.118)}} & \mc{1}{c}{\scriptsize{(0.487)}} & \mc{1}{c}{\scriptsize{(0.237)}} & \mc{1}{c}{\scriptsize{(0.368)}} & \mc{1}{c}{\scriptsize{(0.382)}} & \mc{1}{c}{\scriptsize{(0.316)}} & \mc{1}{c}{\scriptsize{(0.605)}} & \mc{1}{c}{\scriptsize{(0.342)}} \\  

    \mc{1}{l}{\scriptsize{Employment Factor}} & \mc{1}{c}{\scriptsize{21 to 30}} & \mc{1}{c}{\scriptsize{0.344}} & \mc{1}{c}{\scriptsize{0.267}} & \mc{1}{c}{\scriptsize{0.393}} & \mc{1}{c}{\scriptsize{0.315}} & \mc{1}{c}{\scriptsize{0.423}} & \mc{1}{c}{\scriptsize{0.307}} & \mc{1}{c}{\scriptsize{0.291}} & \mc{1}{c}{\scriptsize{0.313}} \\  

     &  & \mc{1}{c}{\scriptsize{\textbf{(0.013)}}} & \mc{1}{c}{\scriptsize{(0.132)}} & \mc{1}{c}{\scriptsize{(0.158)}} & \mc{1}{c}{\scriptsize{(0.461)}} & \mc{1}{c}{\scriptsize{(0.118)}} & \mc{1}{c}{\scriptsize{(0.105)}} & \mc{1}{c}{\scriptsize{(0.171)}} & \mc{1}{c}{\scriptsize{\textbf{(0.079)}}} \\  

  \bottomrule
  \end{tabular}
	\end{table} 

	\begin{table}[H]
     \caption{Treatment Effects on Achenbach Behavior, Pooled Sample}
     \label{table:abccare_rslt_pooled_cat8_sd}
	  \begin{tabular}{cccccccccc}
  \toprule

    \scriptsize{Variable} & \scriptsize{Age} & \scriptsize{(1)} & \scriptsize{(2)} & \scriptsize{(3)} & \scriptsize{(4)} & \scriptsize{(5)} & \scriptsize{(6)} & \scriptsize{(7)} & \scriptsize{(8)} \\ 
    \midrule  

    \mc{1}{l}{\scriptsize{Total Felony Arrests}} & \mc{1}{c}{\scriptsize{Mid-30s}} & \mc{1}{c}{\scriptsize{0.045}} & \mc{1}{c}{\scriptsize{0.229}} & \mc{1}{c}{\scriptsize{-0.132}} & \mc{1}{c}{\scriptsize{0.264}} & \mc{1}{c}{\scriptsize{0.209}} & \mc{1}{c}{\scriptsize{0.112}} & \mc{1}{c}{\scriptsize{0.226}} & \mc{1}{c}{\scriptsize{0.187}} \\  

     &  & \mc{1}{c}{\scriptsize{(0.725)}} & \mc{1}{c}{\scriptsize{(0.833)}} & \mc{1}{c}{\scriptsize{(0.618)}} & \mc{1}{c}{\scriptsize{(0.765)}} & \mc{1}{c}{\scriptsize{(0.843)}} & \mc{1}{c}{\scriptsize{(0.833)}} & \mc{1}{c}{\scriptsize{(0.853)}} & \mc{1}{c}{\scriptsize{(0.814)}} \\  

    \mc{1}{l}{\scriptsize{Total Misdemeanor Arrests}} & \mc{1}{c}{\scriptsize{Mid-30s}} & \mc{1}{c}{\scriptsize{-0.689}} & \mc{1}{c}{\scriptsize{-0.396}} & \mc{1}{c}{\scriptsize{-1.445}} & \mc{1}{c}{\scriptsize{-1.159}} & \mc{1}{c}{\scriptsize{-1.266}} & \mc{1}{c}{\scriptsize{-0.546}} & \mc{1}{c}{\scriptsize{-0.215}} & \mc{1}{c}{\scriptsize{-0.308}} \\  

     &  & \mc{1}{c}{\scriptsize{(0.157)}} & \mc{1}{c}{\scriptsize{(0.451)}} & \mc{1}{c}{\scriptsize{(0.245)}} & \mc{1}{c}{\scriptsize{(0.304)}} & \mc{1}{c}{\scriptsize{(0.402)}} & \mc{1}{c}{\scriptsize{(0.225)}} & \mc{1}{c}{\scriptsize{(0.588)}} & \mc{1}{c}{\scriptsize{(0.490)}} \\  

    \mc{1}{l}{\scriptsize{Total Years Incarcerated}} & \mc{1}{c}{\scriptsize{30}} & \mc{1}{c}{\scriptsize{0.167}} & \mc{1}{c}{\scriptsize{0.205}} & \mc{1}{c}{\scriptsize{0.284}} & \mc{1}{c}{\scriptsize{0.266}} & \mc{1}{c}{\scriptsize{0.369}} & \mc{1}{c}{\scriptsize{0.157}} & \mc{1}{c}{\scriptsize{0.225}} & \mc{1}{c}{\scriptsize{0.216}} \\  

     &  & \mc{1}{c}{\scriptsize{(0.902)}} & \mc{1}{c}{\scriptsize{(0.902)}} & \mc{1}{c}{\scriptsize{(0.971)}} & \mc{1}{c}{\scriptsize{(0.931)}} & \mc{1}{c}{\scriptsize{(0.990)}} & \mc{1}{c}{\scriptsize{(0.882)}} & \mc{1}{c}{\scriptsize{(0.902)}} & \mc{1}{c}{\scriptsize{(0.892)}} \\  

    \mc{1}{l}{\scriptsize{Crime Factor}} & \mc{1}{c}{\scriptsize{30 to Mid-30s}} & \mc{1}{c}{\scriptsize{0.035}} & \mc{1}{c}{\scriptsize{0.102}} & \mc{1}{c}{\scriptsize{-0.048}} & \mc{1}{c}{\scriptsize{0.006}} & \mc{1}{c}{\scriptsize{0.000}} & \mc{1}{c}{\scriptsize{0.068}} & \mc{1}{c}{\scriptsize{0.175}} & \mc{1}{c}{\scriptsize{0.152}} \\  

     &  & \mc{1}{c}{\scriptsize{(0.745)}} & \mc{1}{c}{\scriptsize{(0.833)}} & \mc{1}{c}{\scriptsize{(0.618)}} & \mc{1}{c}{\scriptsize{(0.735)}} & \mc{1}{c}{\scriptsize{(0.784)}} & \mc{1}{c}{\scriptsize{(0.833)}} & \mc{1}{c}{\scriptsize{(0.853)}} & \mc{1}{c}{\scriptsize{(0.814)}} \\  

  \bottomrule
  \end{tabular}
	\end{table} 

	\begin{table}[H]
     \caption{Treatment Effects on Achenbach Symptom T Score (Reported by Mother), Pooled Sample}
     \label{table:abccare_rslt_pooled_cat9_sd}
	  \begin{tabular}{cccccccccc}
  \toprule

    \scriptsize{Variable} & \scriptsize{Age} & \scriptsize{(1)} & \scriptsize{(2)} & \scriptsize{(3)} & \scriptsize{(4)} & \scriptsize{(5)} & \scriptsize{(6)} & \scriptsize{(7)} & \scriptsize{(8)} \\ 
    \midrule  

    \mc{1}{l}{\scriptsize{Ever Told Had: Arthritis/Gout/Lupus/Fibromyalgia}} & \mc{1}{c}{\scriptsize{Mid-30s}} & \mc{1}{c}{\scriptsize{0.071}} & \mc{1}{c}{\scriptsize{0.044}} & \mc{1}{c}{\scriptsize{-0.102}} & \mc{1}{c}{\scriptsize{-0.135}} & \mc{1}{c}{\scriptsize{-0.124}} & \mc{1}{c}{\scriptsize{0.120}} & \mc{1}{c}{\scriptsize{0.091}} & \mc{1}{c}{\scriptsize{0.091}} \\  

     &  & \mc{1}{c}{\scriptsize{(1.000)}} & \mc{1}{c}{\scriptsize{(0.947)}} & \mc{1}{c}{\scriptsize{(0.421)}} & \mc{1}{c}{\scriptsize{(0.329)}} & \mc{1}{c}{\scriptsize{(0.382)}} & \mc{1}{c}{\scriptsize{(1.000)}} & \mc{1}{c}{\scriptsize{(0.987)}} & \mc{1}{c}{\scriptsize{(1.000)}} \\  

    \mc{1}{l}{\scriptsize{Ever Told Had: Prediabetes}} & \mc{1}{c}{\scriptsize{Mid-30s}} & \mc{1}{c}{\scriptsize{-0.002}} & \mc{1}{c}{\scriptsize{0.029}} & \mc{1}{c}{\scriptsize{-0.152}} & \mc{1}{c}{\scriptsize{-0.027}} & \mc{1}{c}{\scriptsize{-0.146}} & \mc{1}{c}{\scriptsize{0.030}} & \mc{1}{c}{\scriptsize{0.054}} & \mc{1}{c}{\scriptsize{0.028}} \\  

     &  & \mc{1}{c}{\scriptsize{(0.816)}} & \mc{1}{c}{\scriptsize{(0.829)}} & \mc{1}{c}{\scriptsize{(0.408)}} & \mc{1}{c}{\scriptsize{(0.592)}} & \mc{1}{c}{\scriptsize{(0.421)}} & \mc{1}{c}{\scriptsize{(0.895)}} & \mc{1}{c}{\scriptsize{(0.934)}} & \mc{1}{c}{\scriptsize{(0.868)}} \\  

  \bottomrule
  \end{tabular}
	\end{table} 

	\begin{table}[H]
     \caption{Treatment Effects on Achenbach Symptom T Score (Reported by Teacher), Pooled Sample}
     \label{table:abccare_rslt_pooled_cat10_sd}
	  \begin{tabular}{cccccccccc}
  \toprule

    \scriptsize{Variable} & \scriptsize{Age} & \scriptsize{(1)} & \scriptsize{(2)} & \scriptsize{(3)} & \scriptsize{(4)} & \scriptsize{(5)} & \scriptsize{(6)} & \scriptsize{(7)} & \scriptsize{(8)} \\ 
    \midrule  

    \mc{1}{l}{\scriptsize{Past Surgery: Cholecystectomy}} & \mc{1}{c}{\scriptsize{Mid-30s}} & \mc{1}{c}{\scriptsize{0.019}} & \mc{1}{c}{\scriptsize{0.009}} & \mc{1}{c}{\scriptsize{0.043}} & \mc{1}{c}{\scriptsize{0.043}} & \mc{1}{c}{\scriptsize{0.048}} & \mc{1}{c}{\scriptsize{0.011}} & \mc{1}{c}{\scriptsize{-0.005}} & \mc{1}{c}{\scriptsize{0.006}} \\  

     &  & \mc{1}{c}{\scriptsize{(0.987)}} & \mc{1}{c}{\scriptsize{(0.961)}} & \mc{1}{c}{\scriptsize{(1.000)}} & \mc{1}{c}{\scriptsize{(1.000)}} & \mc{1}{c}{\scriptsize{(1.000)}} & \mc{1}{c}{\scriptsize{(0.987)}} & \mc{1}{c}{\scriptsize{(0.934)}} & \mc{1}{c}{\scriptsize{(0.974)}} \\  

    \mc{1}{l}{\scriptsize{Past Surgery: Orthopedic Surgery}} & \mc{1}{c}{\scriptsize{Mid-30s}} & \mc{1}{c}{\scriptsize{-0.048}} & \mc{1}{c}{\scriptsize{-0.046}} & \mc{1}{c}{\scriptsize{-0.111}} & \mc{1}{c}{\scriptsize{-0.107}} & \mc{1}{c}{\scriptsize{-0.107}} & \mc{1}{c}{\scriptsize{-0.031}} & \mc{1}{c}{\scriptsize{-0.033}} & \mc{1}{c}{\scriptsize{-0.031}} \\  

     &  & \mc{1}{c}{\scriptsize{(0.224)}} & \mc{1}{c}{\scriptsize{(0.303)}} & \mc{1}{c}{\scriptsize{(0.145)}} & \mc{1}{c}{\scriptsize{(0.211)}} & \mc{1}{c}{\scriptsize{(0.171)}} & \mc{1}{c}{\scriptsize{(0.237)}} & \mc{1}{c}{\scriptsize{(0.434)}} & \mc{1}{c}{\scriptsize{(0.342)}} \\  

    \mc{1}{l}{\scriptsize{Past Surgery: Appendectomy}} & \mc{1}{c}{\scriptsize{Mid-30s}} & \mc{1}{c}{\scriptsize{-0.026}} & \mc{1}{c}{\scriptsize{-0.008}} & \mc{1}{c}{\scriptsize{-0.090}} & \mc{1}{c}{\scriptsize{-0.044}} & \mc{1}{c}{\scriptsize{-0.080}} & \mc{1}{c}{\scriptsize{-0.010}} & \mc{1}{c}{\scriptsize{0.014}} & \mc{1}{c}{\scriptsize{-0.005}} \\  

     &  & \mc{1}{c}{\scriptsize{(0.618)}} & \mc{1}{c}{\scriptsize{(0.895)}} & \mc{1}{c}{\scriptsize{(0.461)}} & \mc{1}{c}{\scriptsize{(0.816)}} & \mc{1}{c}{\scriptsize{(0.539)}} & \mc{1}{c}{\scriptsize{(0.934)}} & \mc{1}{c}{\scriptsize{(0.987)}} & \mc{1}{c}{\scriptsize{(0.974)}} \\  

    \mc{1}{l}{\scriptsize{Past Surgery: Ectopic Pregnancy}} & \mc{1}{c}{\scriptsize{Mid-30s}} & \mc{1}{c}{\scriptsize{-0.003}} & \mc{1}{c}{\scriptsize{0.009}} & \mc{1}{c}{\scriptsize{0.021}} & \mc{1}{c}{\scriptsize{0.004}} & \mc{1}{c}{\scriptsize{0.026}} & \mc{1}{c}{\scriptsize{-0.010}} & \mc{1}{c}{\scriptsize{0.015}} & \mc{1}{c}{\scriptsize{-0.005}} \\  

     &  & \mc{1}{c}{\scriptsize{(0.882)}} & \mc{1}{c}{\scriptsize{(0.974)}} & \mc{1}{c}{\scriptsize{(0.987)}} & \mc{1}{c}{\scriptsize{(0.961)}} & \mc{1}{c}{\scriptsize{(1.000)}} & \mc{1}{c}{\scriptsize{(0.934)}} & \mc{1}{c}{\scriptsize{(0.987)}} & \mc{1}{c}{\scriptsize{(0.961)}} \\  

    \mc{1}{l}{\scriptsize{Past Surgery: Hysterectomy}} & \mc{1}{c}{\scriptsize{Mid-30s}} & \mc{1}{c}{\scriptsize{-0.003}} & \mc{1}{c}{\scriptsize{0.013}} & \mc{1}{c}{\scriptsize{-0.090}} & \mc{1}{c}{\scriptsize{-0.040}} & \mc{1}{c}{\scriptsize{-0.081}} & \mc{1}{c}{\scriptsize{0.021}} & \mc{1}{c}{\scriptsize{0.037}} & \mc{1}{c}{\scriptsize{0.025}} \\  

     &  & \mc{1}{c}{\scriptsize{(0.908)}} & \mc{1}{c}{\scriptsize{(0.974)}} & \mc{1}{c}{\scriptsize{(0.474)}} & \mc{1}{c}{\scriptsize{(0.816)}} & \mc{1}{c}{\scriptsize{(0.539)}} & \mc{1}{c}{\scriptsize{(1.000)}} & \mc{1}{c}{\scriptsize{(1.000)}} & \mc{1}{c}{\scriptsize{(1.000)}} \\  

  \bottomrule
  \end{tabular}
	\end{table} 

	\begin{table}[H]
     \caption{Treatment Effects on Child Assessment Schedule (CAS), Pooled Sample}
     \label{table:abccare_rslt_pooled_cat11_sd}
	  \begin{tabular}{cccccccccc}
  \toprule

    \scriptsize{Variable} & \scriptsize{Age} & \scriptsize{(1)} & \scriptsize{(2)} & \scriptsize{(3)} & \scriptsize{(4)} & \scriptsize{(5)} & \scriptsize{(6)} & \scriptsize{(7)} & \scriptsize{(8)} \\ 
    \midrule  

    \mc{1}{l}{\scriptsize{Systolic Blood Pressure (mm Hg)}} & \mc{1}{c}{\scriptsize{Mid-30s}} & \mc{1}{c}{\scriptsize{-3.100}} & \mc{1}{c}{\scriptsize{-6.448}} & \mc{1}{c}{\scriptsize{3.312}} & \mc{1}{c}{\scriptsize{7.119}} & \mc{1}{c}{\scriptsize{3.303}} & \mc{1}{c}{\scriptsize{-6.030}} & \mc{1}{c}{\scriptsize{-11.018}} & \mc{1}{c}{\scriptsize{-8.312}} \\  

     &  & \mc{1}{c}{\scriptsize{(0.355)}} & \mc{1}{c}{\scriptsize{(0.263)}} & \mc{1}{c}{\scriptsize{(0.908)}} & \mc{1}{c}{\scriptsize{(0.961)}} & \mc{1}{c}{\scriptsize{(0.895)}} & \mc{1}{c}{\scriptsize{\textbf{(0.092)}}} & \mc{1}{c}{\scriptsize{\textbf{(0.079)}}} & \mc{1}{c}{\scriptsize{\textbf{(0.079)}}} \\  

    \mc{1}{l}{\scriptsize{Diastolic Blood Pressure (mm Hg)}} & \mc{1}{c}{\scriptsize{Mid-30s}} & \mc{1}{c}{\scriptsize{-3.719}} & \mc{1}{c}{\scriptsize{-5.986}} & \mc{1}{c}{\scriptsize{-2.250}} & \mc{1}{c}{\scriptsize{-0.551}} & \mc{1}{c}{\scriptsize{-2.493}} & \mc{1}{c}{\scriptsize{-4.860}} & \mc{1}{c}{\scriptsize{-7.229}} & \mc{1}{c}{\scriptsize{-6.471}} \\  

     &  & \mc{1}{c}{\scriptsize{(0.197)}} & \mc{1}{c}{\scriptsize{(0.105)}} & \mc{1}{c}{\scriptsize{(0.487)}} & \mc{1}{c}{\scriptsize{(0.671)}} & \mc{1}{c}{\scriptsize{(0.368)}} & \mc{1}{c}{\scriptsize{\textbf{(0.092)}}} & \mc{1}{c}{\scriptsize{\textbf{(0.079)}}} & \mc{1}{c}{\scriptsize{\textbf{(0.092)}}} \\  

    \mc{1}{l}{\scriptsize{Prehypertension}} & \mc{1}{c}{\scriptsize{Mid-30s}} & \mc{1}{c}{\scriptsize{-0.135}} & \mc{1}{c}{\scriptsize{-0.093}} & \mc{1}{c}{\scriptsize{-0.083}} & \mc{1}{c}{\scriptsize{-0.001}} & \mc{1}{c}{\scriptsize{-0.023}} & \mc{1}{c}{\scriptsize{-0.176}} & \mc{1}{c}{\scriptsize{-0.191}} & \mc{1}{c}{\scriptsize{-0.214}} \\  

     &  & \mc{1}{c}{\scriptsize{\textbf{(0.079)}}} & \mc{1}{c}{\scriptsize{(0.316)}} & \mc{1}{c}{\scriptsize{(0.513)}} & \mc{1}{c}{\scriptsize{(0.737)}} & \mc{1}{c}{\scriptsize{(0.684)}} & \mc{1}{c}{\scriptsize{\textbf{(0.026)}}} & \mc{1}{c}{\scriptsize{\textbf{(0.053)}}} & \mc{1}{c}{\scriptsize{\textbf{(0.026)}}} \\  

    \mc{1}{l}{\scriptsize{Hypertension}} & \mc{1}{c}{\scriptsize{Mid-30s}} & \mc{1}{c}{\scriptsize{0.027}} & \mc{1}{c}{\scriptsize{-0.021}} & \mc{1}{c}{\scriptsize{0.042}} & \mc{1}{c}{\scriptsize{0.236}} & \mc{1}{c}{\scriptsize{0.020}} & \mc{1}{c}{\scriptsize{0.012}} & \mc{1}{c}{\scriptsize{-0.044}} & \mc{1}{c}{\scriptsize{-0.046}} \\  

     &  & \mc{1}{c}{\scriptsize{(0.789)}} & \mc{1}{c}{\scriptsize{(0.684)}} & \mc{1}{c}{\scriptsize{(0.803)}} & \mc{1}{c}{\scriptsize{(0.974)}} & \mc{1}{c}{\scriptsize{(0.789)}} & \mc{1}{c}{\scriptsize{(0.513)}} & \mc{1}{c}{\scriptsize{(0.382)}} & \mc{1}{c}{\scriptsize{(0.382)}} \\  

    \mc{1}{l}{\scriptsize{Hypertension Factor}} & \mc{1}{c}{\scriptsize{Mid-30s}} & \mc{1}{c}{\scriptsize{-0.180}} & \mc{1}{c}{\scriptsize{-0.313}} & \mc{1}{c}{\scriptsize{0.004}} & \mc{1}{c}{\scriptsize{0.226}} & \mc{1}{c}{\scriptsize{0.004}} & \mc{1}{c}{\scriptsize{-0.284}} & \mc{1}{c}{\scriptsize{-0.470}} & \mc{1}{c}{\scriptsize{-0.405}} \\  

     &  & \mc{1}{c}{\scriptsize{(0.303)}} & \mc{1}{c}{\scriptsize{(0.197)}} & \mc{1}{c}{\scriptsize{(0.750)}} & \mc{1}{c}{\scriptsize{(0.934)}} & \mc{1}{c}{\scriptsize{(0.737)}} & \mc{1}{c}{\scriptsize{\textbf{(0.092)}}} & \mc{1}{c}{\scriptsize{\textbf{(0.053)}}} & \mc{1}{c}{\scriptsize{\textbf{(0.066)}}} \\  

  \bottomrule
  \end{tabular}
	\end{table} 

	\begin{table}[H]
     \caption{Treatment Effects on Mother's Income, Pooled Sample}
     \label{table:abccare_rslt_pooled_cat12_sd}
	  \begin{tabular}{cccccccccc}
  \toprule

    \scriptsize{Variable} & \scriptsize{Age} & \scriptsize{(1)} & \scriptsize{(2)} & \scriptsize{(3)} & \scriptsize{(4)} & \scriptsize{(5)} & \scriptsize{(6)} & \scriptsize{(7)} & \scriptsize{(8)} \\ 
    \midrule  

    \mc{1}{l}{\scriptsize{High-Density Lipoprotein Chol. (mg/dL)}} & \mc{1}{c}{\scriptsize{Mid-30s}} & \mc{1}{c}{\scriptsize{3.504}} & \mc{1}{c}{\scriptsize{1.920}} & \mc{1}{c}{\scriptsize{7.000}} & \mc{1}{c}{\scriptsize{1.584}} & \mc{1}{c}{\scriptsize{4.314}} & \mc{1}{c}{\scriptsize{1.967}} & \mc{1}{c}{\scriptsize{1.562}} & \mc{1}{c}{\scriptsize{0.397}} \\  

     &  & \mc{1}{c}{\scriptsize{(0.171)}} & \mc{1}{c}{\scriptsize{(0.382)}} & \mc{1}{c}{\scriptsize{\textbf{(0.039)}}} & \mc{1}{c}{\scriptsize{(0.500)}} & \mc{1}{c}{\scriptsize{(0.184)}} & \mc{1}{c}{\scriptsize{(0.368)}} & \mc{1}{c}{\scriptsize{(0.447)}} & \mc{1}{c}{\scriptsize{(0.697)}} \\  

    \mc{1}{l}{\scriptsize{Dyslipidemia}} & \mc{1}{c}{\scriptsize{Mid-30s}} & \mc{1}{c}{\scriptsize{-0.013}} & \mc{1}{c}{\scriptsize{-0.034}} & \mc{1}{c}{\scriptsize{-0.042}} & \mc{1}{c}{\scriptsize{0.031}} & \mc{1}{c}{\scriptsize{-0.050}} & \mc{1}{c}{\scriptsize{0.015}} & \mc{1}{c}{\scriptsize{-0.014}} & \mc{1}{c}{\scriptsize{0.011}} \\  

     &  & \mc{1}{c}{\scriptsize{(0.592)}} & \mc{1}{c}{\scriptsize{(0.434)}} & \mc{1}{c}{\scriptsize{(0.395)}} & \mc{1}{c}{\scriptsize{(0.763)}} & \mc{1}{c}{\scriptsize{(0.421)}} & \mc{1}{c}{\scriptsize{(0.750)}} & \mc{1}{c}{\scriptsize{(0.618)}} & \mc{1}{c}{\scriptsize{(0.829)}} \\  

    \mc{1}{l}{\scriptsize{Cholesterol Factor}} & \mc{1}{c}{\scriptsize{Mid-30s}} & \mc{1}{c}{\scriptsize{-0.113}} & \mc{1}{c}{\scriptsize{-0.092}} & \mc{1}{c}{\scriptsize{-0.243}} & \mc{1}{c}{\scriptsize{-0.009}} & \mc{1}{c}{\scriptsize{-0.176}} & \mc{1}{c}{\scriptsize{-0.038}} & \mc{1}{c}{\scriptsize{-0.059}} & \mc{1}{c}{\scriptsize{0.001}} \\  

     &  & \mc{1}{c}{\scriptsize{(0.395)}} & \mc{1}{c}{\scriptsize{(0.421)}} & \mc{1}{c}{\scriptsize{(0.118)}} & \mc{1}{c}{\scriptsize{(0.618)}} & \mc{1}{c}{\scriptsize{(0.237)}} & \mc{1}{c}{\scriptsize{(0.579)}} & \mc{1}{c}{\scriptsize{(0.539)}} & \mc{1}{c}{\scriptsize{(0.763)}} \\  

  \bottomrule
  \end{tabular}
	\end{table} 

	\begin{table}[H]
     \caption{Treatment Effects on Parental Labor Income, Pooled Sample}
     \label{table:abccare_rslt_pooled_cat13_sd}
	  \begin{tabular}{cccccccccc}
  \toprule

    \scriptsize{Variable} & \scriptsize{Age} & \scriptsize{(1)} & \scriptsize{(2)} & \scriptsize{(3)} & \scriptsize{(4)} & \scriptsize{(5)} & \scriptsize{(6)} & \scriptsize{(7)} & \scriptsize{(8)} \\ 
    \midrule  

    \mc{1}{l}{\scriptsize{Parental Labor Income}} & \mc{1}{c}{\scriptsize{0}} & \mc{1}{c}{\scriptsize{838}} & \mc{1}{c}{\scriptsize{2,028}} & \mc{1}{c}{\scriptsize{-269}} & \mc{1}{c}{\scriptsize{1,178}} & \mc{1}{c}{\scriptsize{91.145}} & \mc{1}{c}{\scriptsize{1,232}} & \mc{1}{c}{\scriptsize{2,248}} & \mc{1}{c}{\scriptsize{2,744}} \\  

     &  & \mc{1}{c}{\scriptsize{(0.329)}} & \mc{1}{c}{\scriptsize{\textbf{(0.066)}}} & \mc{1}{c}{\scriptsize{(0.513)}} & \mc{1}{c}{\scriptsize{(0.342)}} & \mc{1}{c}{\scriptsize{(0.447)}} & \mc{1}{c}{\scriptsize{(0.197)}} & \mc{1}{c}{\scriptsize{\textbf{(0.039)}}} & \mc{1}{c}{\scriptsize{\textbf{(0.079)}}} \\  

  \bottomrule
  \end{tabular}
	\end{table} 

	\begin{table}[H]
     \caption{Treatment Effects on Parental Public Transfer Income, Pooled Sample}
     \label{table:abccare_rslt_pooled_cat14_sd}
	  \begin{tabular}{cccccccccc}
  \toprule

    \scriptsize{Variable} & \scriptsize{Age} & \scriptsize{(1)} & \scriptsize{(2)} & \scriptsize{(3)} & \scriptsize{(4)} & \scriptsize{(5)} & \scriptsize{(6)} & \scriptsize{(7)} & \scriptsize{(8)} \\ 
    \midrule  

    \mc{1}{l}{\scriptsize{Measured BMI}} & \mc{1}{c}{\scriptsize{Mid-30s}} & \mc{1}{c}{\scriptsize{0.999}} & \mc{1}{c}{\scriptsize{2.781}} & \mc{1}{c}{\scriptsize{-0.202}} & \mc{1}{c}{\scriptsize{2.524}} & \mc{1}{c}{\scriptsize{0.737}} & \mc{1}{c}{\scriptsize{1.072}} & \mc{1}{c}{\scriptsize{2.924}} & \mc{1}{c}{\scriptsize{1.825}} \\  

     &  & \mc{1}{c}{\scriptsize{(0.902)}} & \mc{1}{c}{\scriptsize{(0.902)}} & \mc{1}{c}{\scriptsize{(0.745)}} & \mc{1}{c}{\scriptsize{(0.902)}} & \mc{1}{c}{\scriptsize{(0.873)}} & \mc{1}{c}{\scriptsize{(0.882)}} & \mc{1}{c}{\scriptsize{(0.902)}} & \mc{1}{c}{\scriptsize{(0.863)}} \\  

    \mc{1}{l}{\scriptsize{Obesity}} & \mc{1}{c}{\scriptsize{Mid-30s}} & \mc{1}{c}{\scriptsize{-0.050}} & \mc{1}{c}{\scriptsize{0.061}} & \mc{1}{c}{\scriptsize{-0.256}} & \mc{1}{c}{\scriptsize{-0.084}} & \mc{1}{c}{\scriptsize{-0.143}} & \mc{1}{c}{\scriptsize{-0.013}} & \mc{1}{c}{\scriptsize{0.102}} & \mc{1}{c}{\scriptsize{0.011}} \\  

     &  & \mc{1}{c}{\scriptsize{(0.696)}} & \mc{1}{c}{\scriptsize{(0.794)}} & \mc{1}{c}{\scriptsize{\textbf{(0.069)}}} & \mc{1}{c}{\scriptsize{(0.657)}} & \mc{1}{c}{\scriptsize{(0.539)}} & \mc{1}{c}{\scriptsize{(0.833)}} & \mc{1}{c}{\scriptsize{(0.863)}} & \mc{1}{c}{\scriptsize{(0.863)}} \\  

    \mc{1}{l}{\scriptsize{Severe Obesity}} & \mc{1}{c}{\scriptsize{Mid-30s}} & \mc{1}{c}{\scriptsize{-0.126}} & \mc{1}{c}{\scriptsize{-0.045}} & \mc{1}{c}{\scriptsize{-0.093}} & \mc{1}{c}{\scriptsize{0.013}} & \mc{1}{c}{\scriptsize{-0.064}} & \mc{1}{c}{\scriptsize{-0.147}} & \mc{1}{c}{\scriptsize{-0.065}} & \mc{1}{c}{\scriptsize{-0.108}} \\  

     &  & \mc{1}{c}{\scriptsize{(0.353)}} & \mc{1}{c}{\scriptsize{(0.745)}} & \mc{1}{c}{\scriptsize{(0.510)}} & \mc{1}{c}{\scriptsize{(0.863)}} & \mc{1}{c}{\scriptsize{(0.696)}} & \mc{1}{c}{\scriptsize{(0.343)}} & \mc{1}{c}{\scriptsize{(0.647)}} & \mc{1}{c}{\scriptsize{(0.549)}} \\  

    \mc{1}{l}{\scriptsize{Waist-hip Ratio}} & \mc{1}{c}{\scriptsize{Mid-30s}} & \mc{1}{c}{\scriptsize{-0.006}} & \mc{1}{c}{\scriptsize{0.000}} & \mc{1}{c}{\scriptsize{-0.037}} & \mc{1}{c}{\scriptsize{-0.035}} & \mc{1}{c}{\scriptsize{-0.039}} & \mc{1}{c}{\scriptsize{0.003}} & \mc{1}{c}{\scriptsize{0.010}} & \mc{1}{c}{\scriptsize{0.012}} \\  

     &  & \mc{1}{c}{\scriptsize{(0.725)}} & \mc{1}{c}{\scriptsize{(0.745)}} & \mc{1}{c}{\scriptsize{(0.382)}} & \mc{1}{c}{\scriptsize{(0.657)}} & \mc{1}{c}{\scriptsize{(0.539)}} & \mc{1}{c}{\scriptsize{(0.833)}} & \mc{1}{c}{\scriptsize{(0.824)}} & \mc{1}{c}{\scriptsize{(0.863)}} \\  

    \mc{1}{l}{\scriptsize{Abdominal Obesity}} & \mc{1}{c}{\scriptsize{Mid-30s}} & \mc{1}{c}{\scriptsize{-0.091}} & \mc{1}{c}{\scriptsize{-0.025}} & \mc{1}{c}{\scriptsize{-0.230}} & \mc{1}{c}{\scriptsize{-0.091}} & \mc{1}{c}{\scriptsize{-0.190}} & \mc{1}{c}{\scriptsize{-0.041}} & \mc{1}{c}{\scriptsize{0.016}} & \mc{1}{c}{\scriptsize{0.002}} \\  

     &  & \mc{1}{c}{\scriptsize{(0.578)}} & \mc{1}{c}{\scriptsize{(0.745)}} & \mc{1}{c}{\scriptsize{\textbf{(0.078)}}} & \mc{1}{c}{\scriptsize{(0.657)}} & \mc{1}{c}{\scriptsize{(0.275)}} & \mc{1}{c}{\scriptsize{(0.794)}} & \mc{1}{c}{\scriptsize{(0.824)}} & \mc{1}{c}{\scriptsize{(0.863)}} \\  

    \mc{1}{l}{\scriptsize{Framingham Risk Score}} & \mc{1}{c}{\scriptsize{Mid-30s}} & \mc{1}{c}{\scriptsize{0.348}} & \mc{1}{c}{\scriptsize{-0.403}} & \mc{1}{c}{\scriptsize{0.948}} & \mc{1}{c}{\scriptsize{0.181}} & \mc{1}{c}{\scriptsize{0.902}} & \mc{1}{c}{\scriptsize{0.351}} & \mc{1}{c}{\scriptsize{-0.481}} & \mc{1}{c}{\scriptsize{0.087}} \\  

     &  & \mc{1}{c}{\scriptsize{(0.902)}} & \mc{1}{c}{\scriptsize{(0.745)}} & \mc{1}{c}{\scriptsize{(0.971)}} & \mc{1}{c}{\scriptsize{(0.902)}} & \mc{1}{c}{\scriptsize{(0.951)}} & \mc{1}{c}{\scriptsize{(0.882)}} & \mc{1}{c}{\scriptsize{(0.647)}} & \mc{1}{c}{\scriptsize{(0.863)}} \\  

    \mc{1}{l}{\scriptsize{Obesity Factor}} & \mc{1}{c}{\scriptsize{Mid-30s}} & \mc{1}{c}{\scriptsize{0.068}} & \mc{1}{c}{\scriptsize{-0.114}} & \mc{1}{c}{\scriptsize{0.360}} & \mc{1}{c}{\scriptsize{0.088}} & \mc{1}{c}{\scriptsize{0.336}} & \mc{1}{c}{\scriptsize{0.002}} & \mc{1}{c}{\scriptsize{-0.223}} & \mc{1}{c}{\scriptsize{-0.060}} \\  

     &  & \mc{1}{c}{\scriptsize{(0.902)}} & \mc{1}{c}{\scriptsize{(0.745)}} & \mc{1}{c}{\scriptsize{(0.971)}} & \mc{1}{c}{\scriptsize{(0.902)}} & \mc{1}{c}{\scriptsize{(0.951)}} & \mc{1}{c}{\scriptsize{(0.833)}} & \mc{1}{c}{\scriptsize{(0.637)}} & \mc{1}{c}{\scriptsize{(0.833)}} \\  

  \bottomrule
  \end{tabular}
	\end{table} 

	\begin{table}[H]
     \caption{Treatment Effects on Adoption, Pooled Sample}
     \label{table:abccare_rslt_pooled_cat15_sd}
	  \begin{tabular}{cccccccccc}
  \toprule

    \scriptsize{Variable} & \scriptsize{Age} & \scriptsize{(1)} & \scriptsize{(2)} & \scriptsize{(3)} & \scriptsize{(4)} & \scriptsize{(5)} & \scriptsize{(6)} & \scriptsize{(7)} & \scriptsize{(8)} \\ 
    \midrule  

    \mc{1}{l}{\scriptsize{Somatization $t$-Score}} & \mc{1}{c}{\scriptsize{21}} & \mc{1}{c}{\scriptsize{-2.709}} & \mc{1}{c}{\scriptsize{-3.314}} & \mc{1}{c}{\scriptsize{-4.304}} & \mc{1}{c}{\scriptsize{-5.923}} & \mc{1}{c}{\scriptsize{-4.630}} & \mc{1}{c}{\scriptsize{-2.258}} & \mc{1}{c}{\scriptsize{-2.669}} & \mc{1}{c}{\scriptsize{-3.001}} \\  

     &  & \mc{1}{c}{\scriptsize{(0.206)}} & \mc{1}{c}{\scriptsize{(0.137)}} & \mc{1}{c}{\scriptsize{(0.216)}} & \mc{1}{c}{\scriptsize{\textbf{(0.078)}}} & \mc{1}{c}{\scriptsize{(0.294)}} & \mc{1}{c}{\scriptsize{(0.314)}} & \mc{1}{c}{\scriptsize{(0.333)}} & \mc{1}{c}{\scriptsize{(0.235)}} \\  

     & \mc{1}{c}{\scriptsize{Mid-30s}} & \mc{1}{c}{\scriptsize{-1.057}} & \mc{1}{c}{\scriptsize{-0.609}} & \mc{1}{c}{\scriptsize{-2.144}} & \mc{1}{c}{\scriptsize{-3.744}} & \mc{1}{c}{\scriptsize{-2.063}} & \mc{1}{c}{\scriptsize{-0.950}} & \mc{1}{c}{\scriptsize{0.363}} & \mc{1}{c}{\scriptsize{-0.687}} \\  

     &  & \mc{1}{c}{\scriptsize{(0.382)}} & \mc{1}{c}{\scriptsize{(0.451)}} & \mc{1}{c}{\scriptsize{(0.412)}} & \mc{1}{c}{\scriptsize{(0.284)}} & \mc{1}{c}{\scriptsize{(0.382)}} & \mc{1}{c}{\scriptsize{(0.422)}} & \mc{1}{c}{\scriptsize{(0.647)}} & \mc{1}{c}{\scriptsize{(0.471)}} \\  

    \mc{1}{l}{\scriptsize{Depression $t$-Score}} & \mc{1}{c}{\scriptsize{21}} & \mc{1}{c}{\scriptsize{-4.213}} & \mc{1}{c}{\scriptsize{-3.329}} & \mc{1}{c}{\scriptsize{-4.297}} & \mc{1}{c}{\scriptsize{-4.299}} & \mc{1}{c}{\scriptsize{-4.310}} & \mc{1}{c}{\scriptsize{-4.058}} & \mc{1}{c}{\scriptsize{-3.216}} & \mc{1}{c}{\scriptsize{-3.669}} \\  

     &  & \mc{1}{c}{\scriptsize{\textbf{(0.098)}}} & \mc{1}{c}{\scriptsize{(0.137)}} & \mc{1}{c}{\scriptsize{(0.235)}} & \mc{1}{c}{\scriptsize{(0.284)}} & \mc{1}{c}{\scriptsize{(0.294)}} & \mc{1}{c}{\scriptsize{(0.147)}} & \mc{1}{c}{\scriptsize{(0.275)}} & \mc{1}{c}{\scriptsize{(0.186)}} \\  

     & \mc{1}{c}{\scriptsize{Mid-30s}} & \mc{1}{c}{\scriptsize{-1.904}} & \mc{1}{c}{\scriptsize{-2.256}} & \mc{1}{c}{\scriptsize{1.064}} & \mc{1}{c}{\scriptsize{-0.701}} & \mc{1}{c}{\scriptsize{0.482}} & \mc{1}{c}{\scriptsize{-2.974}} & \mc{1}{c}{\scriptsize{-3.181}} & \mc{1}{c}{\scriptsize{-3.161}} \\  

     &  & \mc{1}{c}{\scriptsize{(0.294)}} & \mc{1}{c}{\scriptsize{(0.294)}} & \mc{1}{c}{\scriptsize{(0.618)}} & \mc{1}{c}{\scriptsize{(0.392)}} & \mc{1}{c}{\scriptsize{(0.539)}} & \mc{1}{c}{\scriptsize{(0.314)}} & \mc{1}{c}{\scriptsize{(0.333)}} & \mc{1}{c}{\scriptsize{(0.235)}} \\  

    \mc{1}{l}{\scriptsize{Anxiety $t$-Score}} & \mc{1}{c}{\scriptsize{21}} & \mc{1}{c}{\scriptsize{-2.749}} & \mc{1}{c}{\scriptsize{-2.745}} & \mc{1}{c}{\scriptsize{-2.996}} & \mc{1}{c}{\scriptsize{-4.679}} & \mc{1}{c}{\scriptsize{-2.925}} & \mc{1}{c}{\scriptsize{-2.638}} & \mc{1}{c}{\scriptsize{-2.485}} & \mc{1}{c}{\scriptsize{-2.737}} \\  

     &  & \mc{1}{c}{\scriptsize{(0.255)}} & \mc{1}{c}{\scriptsize{(0.294)}} & \mc{1}{c}{\scriptsize{(0.324)}} & \mc{1}{c}{\scriptsize{(0.284)}} & \mc{1}{c}{\scriptsize{(0.382)}} & \mc{1}{c}{\scriptsize{(0.314)}} & \mc{1}{c}{\scriptsize{(0.382)}} & \mc{1}{c}{\scriptsize{(0.235)}} \\  

     & \mc{1}{c}{\scriptsize{Mid-30s}} & \mc{1}{c}{\scriptsize{-3.399}} & \mc{1}{c}{\scriptsize{-4.109}} & \mc{1}{c}{\scriptsize{-1.502}} & \mc{1}{c}{\scriptsize{-4.550}} & \mc{1}{c}{\scriptsize{-2.084}} & \mc{1}{c}{\scriptsize{-4.155}} & \mc{1}{c}{\scriptsize{-4.213}} & \mc{1}{c}{\scriptsize{-4.724}} \\  

     &  & \mc{1}{c}{\scriptsize{(0.206)}} & \mc{1}{c}{\scriptsize{(0.137)}} & \mc{1}{c}{\scriptsize{(0.412)}} & \mc{1}{c}{\scriptsize{(0.284)}} & \mc{1}{c}{\scriptsize{(0.382)}} & \mc{1}{c}{\scriptsize{(0.245)}} & \mc{1}{c}{\scriptsize{(0.245)}} & \mc{1}{c}{\scriptsize{(0.186)}} \\  

    \mc{1}{l}{\scriptsize{Hostility $t$-Score}} & \mc{1}{c}{\scriptsize{21}} & \mc{1}{c}{\scriptsize{-3.256}} & \mc{1}{c}{\scriptsize{-2.213}} & \mc{1}{c}{\scriptsize{-4.552}} & \mc{1}{c}{\scriptsize{-4.455}} & \mc{1}{c}{\scriptsize{-4.613}} & \mc{1}{c}{\scriptsize{-2.894}} & \mc{1}{c}{\scriptsize{-1.905}} & \mc{1}{c}{\scriptsize{-2.544}} \\  

     &  & \mc{1}{c}{\scriptsize{(0.127)}} & \mc{1}{c}{\scriptsize{(0.294)}} & \mc{1}{c}{\scriptsize{(0.235)}} & \mc{1}{c}{\scriptsize{(0.284)}} & \mc{1}{c}{\scriptsize{(0.294)}} & \mc{1}{c}{\scriptsize{(0.245)}} & \mc{1}{c}{\scriptsize{(0.382)}} & \mc{1}{c}{\scriptsize{(0.235)}} \\  

     & \mc{1}{c}{\scriptsize{Mid-30s}} & \mc{1}{c}{\scriptsize{-1.091}} & \mc{1}{c}{\scriptsize{-0.694}} & \mc{1}{c}{\scriptsize{-2.076}} & \mc{1}{c}{\scriptsize{-1.664}} & \mc{1}{c}{\scriptsize{-2.411}} & \mc{1}{c}{\scriptsize{-1.082}} &  & \mc{1}{c}{\scriptsize{-0.844}} \\  

     &  & \mc{1}{c}{\scriptsize{(0.382)}} & \mc{1}{c}{\scriptsize{(0.451)}} & \mc{1}{c}{\scriptsize{(0.412)}} & \mc{1}{c}{\scriptsize{(0.392)}} & \mc{1}{c}{\scriptsize{(0.382)}} & \mc{1}{c}{\scriptsize{(0.422)}} & \mc{1}{c}{\scriptsize{(0.647)}} & \mc{1}{c}{\scriptsize{(0.471)}} \\  

    \mc{1}{l}{\scriptsize{Global Severity Index $t$-Score}} & \mc{1}{c}{\scriptsize{21}} & \mc{1}{c}{\scriptsize{-3.146}} & \mc{1}{c}{\scriptsize{-2.581}} & \mc{1}{c}{\scriptsize{-4.917}} & \mc{1}{c}{\scriptsize{-4.896}} & \mc{1}{c}{\scriptsize{-4.929}} & \mc{1}{c}{\scriptsize{-2.564}} & \mc{1}{c}{\scriptsize{-1.827}} & \mc{1}{c}{\scriptsize{-2.485}} \\  

     &  & \mc{1}{c}{\scriptsize{(0.157)}} & \mc{1}{c}{\scriptsize{(0.294)}} & \mc{1}{c}{\scriptsize{(0.127)}} & \mc{1}{c}{\scriptsize{(0.157)}} & \mc{1}{c}{\scriptsize{(0.235)}} & \mc{1}{c}{\scriptsize{(0.314)}} & \mc{1}{c}{\scriptsize{(0.382)}} & \mc{1}{c}{\scriptsize{(0.235)}} \\  

    \mc{1}{l}{\scriptsize{Global Severity Index $t$-Score (BSI 18)}} & \mc{1}{c}{\scriptsize{Mid-30s}} & \mc{1}{c}{\scriptsize{-2.516}} & \mc{1}{c}{\scriptsize{-2.306}} & \mc{1}{c}{\scriptsize{-0.151}} & \mc{1}{c}{\scriptsize{-2.444}} & \mc{1}{c}{\scriptsize{-0.514}} & \mc{1}{c}{\scriptsize{-3.477}} & \mc{1}{c}{\scriptsize{-2.589}} & \mc{1}{c}{\scriptsize{-3.446}} \\  

     &  & \mc{1}{c}{\scriptsize{(0.127)}} & \mc{1}{c}{\scriptsize{(0.196)}} & \mc{1}{c}{\scriptsize{(0.451)}} & \mc{1}{c}{\scriptsize{(0.255)}} & \mc{1}{c}{\scriptsize{(0.441)}} & \mc{1}{c}{\scriptsize{\textbf{(0.088)}}} & \mc{1}{c}{\scriptsize{(0.157)}} & \mc{1}{c}{\scriptsize{(0.118)}} \\  

    \mc{1}{l}{\scriptsize{BSI Factor}} & \mc{1}{c}{\scriptsize{21 to Mid-30s}} & \mc{1}{c}{\scriptsize{-0.507}} & \mc{1}{c}{\scriptsize{-0.390}} & \mc{1}{c}{\scriptsize{-0.527}} & \mc{1}{c}{\scriptsize{-0.615}} & \mc{1}{c}{\scriptsize{-0.476}} & \mc{1}{c}{\scriptsize{-0.500}} & \mc{1}{c}{\scriptsize{-0.355}} & \mc{1}{c}{\scriptsize{-0.468}} \\  

     &  & \mc{1}{c}{\scriptsize{\textbf{(0.039)}}} & \mc{1}{c}{\scriptsize{\textbf{(0.078)}}} & \mc{1}{c}{\scriptsize{(0.127)}} & \mc{1}{c}{\scriptsize{(0.108)}} & \mc{1}{c}{\scriptsize{\textbf{(0.098)}}} & \mc{1}{c}{\scriptsize{\textbf{(0.059)}}} & \mc{1}{c}{\scriptsize{(0.118)}} & \mc{1}{c}{\scriptsize{\textbf{(0.088)}}} \\  

  \bottomrule
  \end{tabular}
	\end{table} 

	\begin{table}[H]
     \caption{Treatment Effects on Childhood Household Income, Pooled Sample}
     \label{table:abccare_rslt_pooled_cat16_sd}
	  \begin{tabular}{cccccccccc}
  \toprule

    \scriptsize{Variable} & \scriptsize{Age} & \scriptsize{(1)} & \scriptsize{(2)} & \scriptsize{(3)} & \scriptsize{(4)} & \scriptsize{(5)} & \scriptsize{(6)} & \scriptsize{(7)} & \scriptsize{(8)} \\ 
    \midrule  

    \mc{1}{l}{\scriptsize{Somatization $t$-Score}} & \mc{1}{c}{\scriptsize{21}} & \mc{1}{c}{\scriptsize{-2.709}} & \mc{1}{c}{\scriptsize{-2.978}} & \mc{1}{c}{\scriptsize{-4.304}} & \mc{1}{c}{\scriptsize{-4.393}} & \mc{1}{c}{\scriptsize{-4.629}} & \mc{1}{c}{\scriptsize{-2.258}} & \mc{1}{c}{\scriptsize{-2.460}} & \mc{1}{c}{\scriptsize{-3.004}} \\  

     &  & \mc{1}{c}{\scriptsize{(0.181)}} & \mc{1}{c}{\scriptsize{(0.250)}} & \mc{1}{c}{\scriptsize{(0.324)}} & \mc{1}{c}{\scriptsize{(0.281)}} & \mc{1}{c}{\scriptsize{(0.277)}} & \mc{1}{c}{\scriptsize{(0.303)}} & \mc{1}{c}{\scriptsize{(0.389)}} & \mc{1}{c}{\scriptsize{(0.247)}} \\  

     & \mc{1}{c}{\scriptsize{Mid-30s}} & \mc{1}{c}{\scriptsize{-1.057}} & \mc{1}{c}{\scriptsize{-0.159}} & \mc{1}{c}{\scriptsize{-2.144}} & \mc{1}{c}{\scriptsize{-1.831}} & \mc{1}{c}{\scriptsize{-2.072}} & \mc{1}{c}{\scriptsize{-0.950}} & \mc{1}{c}{\scriptsize{-0.055}} & \mc{1}{c}{\scriptsize{-0.679}} \\  

     &  & \mc{1}{c}{\scriptsize{(0.418)}} & \mc{1}{c}{\scriptsize{(0.535)}} & \mc{1}{c}{\scriptsize{(0.727)}} & \mc{1}{c}{\scriptsize{(0.734)}} & \mc{1}{c}{\scriptsize{(0.676)}} & \mc{1}{c}{\scriptsize{(0.441)}} & \mc{1}{c}{\scriptsize{(0.508)}} & \mc{1}{c}{\scriptsize{(0.454)}} \\  

    \mc{1}{l}{\scriptsize{Depression $t$-Score}} & \mc{1}{c}{\scriptsize{21}} & \mc{1}{c}{\scriptsize{-4.213}} & \mc{1}{c}{\scriptsize{-3.221}} & \mc{1}{c}{\scriptsize{-4.297}} & \mc{1}{c}{\scriptsize{-3.969}} & \mc{1}{c}{\scriptsize{-4.334}} & \mc{1}{c}{\scriptsize{-4.058}} & \mc{1}{c}{\scriptsize{-3.061}} & \mc{1}{c}{\scriptsize{-3.668}} \\  

     &  & \mc{1}{c}{\scriptsize{\textbf{(0.061)}}} & \mc{1}{c}{\scriptsize{(0.250)}} & \mc{1}{c}{\scriptsize{(0.352)}} & \mc{1}{c}{\scriptsize{(0.415)}} & \mc{1}{c}{\scriptsize{(0.323)}} & \mc{1}{c}{\scriptsize{\textbf{(0.100)}}} & \mc{1}{c}{\scriptsize{(0.330)}} & \mc{1}{c}{\scriptsize{(0.172)}} \\  

     & \mc{1}{c}{\scriptsize{Mid-30s}} & \mc{1}{c}{\scriptsize{-1.904}} & \mc{1}{c}{\scriptsize{-1.789}} & \mc{1}{c}{\scriptsize{1.064}} & \mc{1}{c}{\scriptsize{0.448}} & \mc{1}{c}{\scriptsize{0.468}} & \mc{1}{c}{\scriptsize{-2.974}} & \mc{1}{c}{\scriptsize{-3.163}} & \mc{1}{c}{\scriptsize{-3.154}} \\  

     &  & \mc{1}{c}{\scriptsize{(0.329)}} & \mc{1}{c}{\scriptsize{(0.333)}} & \mc{1}{c}{\scriptsize{(0.738)}} & \mc{1}{c}{\scriptsize{(0.734)}} & \mc{1}{c}{\scriptsize{(0.676)}} & \mc{1}{c}{\scriptsize{(0.303)}} & \mc{1}{c}{\scriptsize{(0.335)}} & \mc{1}{c}{\scriptsize{(0.307)}} \\  

    \mc{1}{l}{\scriptsize{Anxiety $t$-Score}} & \mc{1}{c}{\scriptsize{21}} & \mc{1}{c}{\scriptsize{-2.749}} & \mc{1}{c}{\scriptsize{-2.319}} & \mc{1}{c}{\scriptsize{-2.996}} & \mc{1}{c}{\scriptsize{-2.804}} & \mc{1}{c}{\scriptsize{-2.941}} & \mc{1}{c}{\scriptsize{-2.638}} & \mc{1}{c}{\scriptsize{-2.092}} & \mc{1}{c}{\scriptsize{-2.740}} \\  

     &  & \mc{1}{c}{\scriptsize{(0.217)}} & \mc{1}{c}{\scriptsize{(0.326)}} & \mc{1}{c}{\scriptsize{(0.534)}} & \mc{1}{c}{\scriptsize{(0.559)}} & \mc{1}{c}{\scriptsize{(0.529)}} & \mc{1}{c}{\scriptsize{(0.303)}} & \mc{1}{c}{\scriptsize{(0.445)}} & \mc{1}{c}{\scriptsize{(0.307)}} \\  

     & \mc{1}{c}{\scriptsize{Mid-30s}} & \mc{1}{c}{\scriptsize{-3.399}} & \mc{1}{c}{\scriptsize{-3.378}} & \mc{1}{c}{\scriptsize{-1.502}} & \mc{1}{c}{\scriptsize{-2.337}} & \mc{1}{c}{\scriptsize{-2.102}} & \mc{1}{c}{\scriptsize{-4.155}} & \mc{1}{c}{\scriptsize{-4.473}} & \mc{1}{c}{\scriptsize{-4.712}} \\  

     &  & \mc{1}{c}{\scriptsize{(0.217)}} & \mc{1}{c}{\scriptsize{(0.250)}} & \mc{1}{c}{\scriptsize{(0.738)}} & \mc{1}{c}{\scriptsize{(0.734)}} & \mc{1}{c}{\scriptsize{(0.676)}} & \mc{1}{c}{\scriptsize{(0.230)}} & \mc{1}{c}{\scriptsize{(0.156)}} & \mc{1}{c}{\scriptsize{(0.168)}} \\  

    \mc{1}{l}{\scriptsize{Hostility $t$-Score}} & \mc{1}{c}{\scriptsize{21}} & \mc{1}{c}{\scriptsize{-3.256}} & \mc{1}{c}{\scriptsize{-2.543}} & \mc{1}{c}{\scriptsize{-4.552}} & \mc{1}{c}{\scriptsize{-4.015}} & \mc{1}{c}{\scriptsize{-4.629}} & \mc{1}{c}{\scriptsize{-2.894}} & \mc{1}{c}{\scriptsize{-1.852}} & \mc{1}{c}{\scriptsize{-2.549}} \\  

     &  & \mc{1}{c}{\scriptsize{(0.114)}} & \mc{1}{c}{\scriptsize{(0.264)}} & \mc{1}{c}{\scriptsize{(0.352)}} & \mc{1}{c}{\scriptsize{(0.415)}} & \mc{1}{c}{\scriptsize{(0.323)}} & \mc{1}{c}{\scriptsize{(0.230)}} & \mc{1}{c}{\scriptsize{(0.445)}} & \mc{1}{c}{\scriptsize{(0.307)}} \\  

     & \mc{1}{c}{\scriptsize{Mid-30s}} & \mc{1}{c}{\scriptsize{-1.091}} & \mc{1}{c}{\scriptsize{-0.375}} & \mc{1}{c}{\scriptsize{-2.076}} &  & \mc{1}{c}{\scriptsize{-2.428}} & \mc{1}{c}{\scriptsize{-1.082}} & \mc{1}{c}{\scriptsize{-0.461}} & \mc{1}{c}{\scriptsize{-0.834}} \\  

     &  & \mc{1}{c}{\scriptsize{(0.418)}} & \mc{1}{c}{\scriptsize{(0.535)}} & \mc{1}{c}{\scriptsize{(0.727)}} &  & \mc{1}{c}{\scriptsize{(0.624)}} & \mc{1}{c}{\scriptsize{(0.441)}} & \mc{1}{c}{\scriptsize{(0.508)}} & \mc{1}{c}{\scriptsize{(0.454)}} \\  

    \mc{1}{l}{\scriptsize{Global Severity Index $t$-Score}} & \mc{1}{c}{\scriptsize{21}} & \mc{1}{c}{\scriptsize{-3.146}} & \mc{1}{c}{\scriptsize{-2.736}} & \mc{1}{c}{\scriptsize{-4.917}} & \mc{1}{c}{\scriptsize{-4.235}} & \mc{1}{c}{\scriptsize{-5.096}} & \mc{1}{c}{\scriptsize{-2.564}} & \mc{1}{c}{\scriptsize{-1.870}} & \mc{1}{c}{\scriptsize{-2.851}} \\  

     &  & \mc{1}{c}{\scriptsize{(0.157)}} & \mc{1}{c}{\scriptsize{(0.264)}} & \mc{1}{c}{\scriptsize{(0.203)}} & \mc{1}{c}{\scriptsize{(0.276)}} & \mc{1}{c}{\scriptsize{(0.192)}} & \mc{1}{c}{\scriptsize{(0.303)}} & \mc{1}{c}{\scriptsize{(0.445)}} & \mc{1}{c}{\scriptsize{(0.307)}} \\  

    \mc{1}{l}{\scriptsize{Global Severity Index $t$-Score (BSI 18)}} & \mc{1}{c}{\scriptsize{Mid-30s}} & \mc{1}{c}{\scriptsize{-2.516}} & \mc{1}{c}{\scriptsize{-1.571}} & \mc{1}{c}{\scriptsize{-0.151}} & \mc{1}{c}{\scriptsize{-0.306}} & \mc{1}{c}{\scriptsize{-0.532}} & \mc{1}{c}{\scriptsize{-3.477}} & \mc{1}{c}{\scriptsize{-2.696}} & \mc{1}{c}{\scriptsize{-3.436}} \\  

     &  & \mc{1}{c}{\scriptsize{(0.166)}} & \mc{1}{c}{\scriptsize{(0.247)}} & \mc{1}{c}{\scriptsize{(0.443)}} & \mc{1}{c}{\scriptsize{(0.428)}} & \mc{1}{c}{\scriptsize{(0.398)}} & \mc{1}{c}{\scriptsize{(0.116)}} & \mc{1}{c}{\scriptsize{(0.150)}} & \mc{1}{c}{\scriptsize{(0.125)}} \\  

    \mc{1}{l}{\scriptsize{BSI Factor}} & \mc{1}{c}{\scriptsize{21 to Mid-30s}} & \mc{1}{c}{\scriptsize{-0.507}} & \mc{1}{c}{\scriptsize{-0.323}} & \mc{1}{c}{\scriptsize{-0.527}} & \mc{1}{c}{\scriptsize{-0.458}} & \mc{1}{c}{\scriptsize{-0.478}} & \mc{1}{c}{\scriptsize{-0.500}} & \mc{1}{c}{\scriptsize{-0.353}} & \mc{1}{c}{\scriptsize{-0.468}} \\  

     &  & \mc{1}{c}{\scriptsize{\textbf{(0.028)}}} & \mc{1}{c}{\scriptsize{(0.120)}} & \mc{1}{c}{\scriptsize{(0.136)}} & \mc{1}{c}{\scriptsize{(0.185)}} & \mc{1}{c}{\scriptsize{(0.165)}} & \mc{1}{c}{\scriptsize{\textbf{(0.054)}}} & \mc{1}{c}{\scriptsize{(0.134)}} & \mc{1}{c}{\scriptsize{\textbf{(0.070)}}} \\  

  \bottomrule
  \end{tabular}
	\end{table} 

	\begin{table}[H]
     \caption{Treatment Effects on Father at Home, Pooled Sample}
     \label{table:abccare_rslt_pooled_cat17_sd}
	  \begin{tabular}{cccccccccc}
  \toprule

    \scriptsize{Variable} & \scriptsize{Age} & \scriptsize{(1)} & \scriptsize{(2)} & \scriptsize{(3)} & \scriptsize{(4)} & \scriptsize{(5)} & \scriptsize{(6)} & \scriptsize{(7)} & \scriptsize{(8)} \\ 
    \midrule  

    \mc{1}{l}{\scriptsize{Measured BMI}} & \mc{1}{c}{\scriptsize{Mid-30s}} & \mc{1}{c}{\scriptsize{-1.329}} & \mc{1}{c}{\scriptsize{0.516}} & \mc{1}{c}{\scriptsize{-0.673}} & \mc{1}{c}{\scriptsize{-5.302}} & \mc{1}{c}{\scriptsize{2.854}} & \mc{1}{c}{\scriptsize{-1.790}} & \mc{1}{c}{\scriptsize{1.084}} & \mc{1}{c}{\scriptsize{-0.583}} \\  

     &  & \mc{1}{c}{\scriptsize{(0.763)}} & \mc{1}{c}{\scriptsize{(0.961)}} & \mc{1}{c}{\scriptsize{(0.813)}} & \mc{1}{c}{\scriptsize{(0.480)}} & \mc{1}{c}{\scriptsize{(0.933)}} & \mc{1}{c}{\scriptsize{(0.579)}} & \mc{1}{c}{\scriptsize{(0.947)}} & \mc{1}{c}{\scriptsize{(0.829)}} \\  

    \mc{1}{l}{\scriptsize{Obesity}} & \mc{1}{c}{\scriptsize{Mid-30s}} & \mc{1}{c}{\scriptsize{-0.121}} & \mc{1}{c}{\scriptsize{0.030}} & \mc{1}{c}{\scriptsize{-0.171}} & \mc{1}{c}{\scriptsize{-0.197}} & \mc{1}{c}{\scriptsize{0.058}} & \mc{1}{c}{\scriptsize{-0.141}} & \mc{1}{c}{\scriptsize{0.024}} & \mc{1}{c}{\scriptsize{-0.106}} \\  

     &  & \mc{1}{c}{\scriptsize{(0.500)}} & \mc{1}{c}{\scriptsize{(0.961)}} & \mc{1}{c}{\scriptsize{(0.547)}} & \mc{1}{c}{\scriptsize{(0.533)}} & \mc{1}{c}{\scriptsize{(0.813)}} & \mc{1}{c}{\scriptsize{(0.461)}} & \mc{1}{c}{\scriptsize{(0.947)}} & \mc{1}{c}{\scriptsize{(0.658)}} \\  

    \mc{1}{l}{\scriptsize{Severe Obesity}} & \mc{1}{c}{\scriptsize{Mid-30s}} & \mc{1}{c}{\scriptsize{-0.213}} & \mc{1}{c}{\scriptsize{-0.149}} & \mc{1}{c}{\scriptsize{-0.096}} & \mc{1}{c}{\scriptsize{-0.203}} & \mc{1}{c}{\scriptsize{0.036}} & \mc{1}{c}{\scriptsize{-0.246}} & \mc{1}{c}{\scriptsize{-0.145}} & \mc{1}{c}{\scriptsize{-0.211}} \\  

     &  & \mc{1}{c}{\scriptsize{(0.118)}} & \mc{1}{c}{\scriptsize{(0.461)}} & \mc{1}{c}{\scriptsize{(0.707)}} & \mc{1}{c}{\scriptsize{(0.507)}} & \mc{1}{c}{\scriptsize{(0.800)}} & \mc{1}{c}{\scriptsize{(0.105)}} & \mc{1}{c}{\scriptsize{(0.526)}} & \mc{1}{c}{\scriptsize{(0.276)}} \\  

    \mc{1}{l}{\scriptsize{Waist-hip Ratio}} & \mc{1}{c}{\scriptsize{Mid-30s}} & \mc{1}{c}{\scriptsize{-0.033}} & \mc{1}{c}{\scriptsize{-0.035}} & \mc{1}{c}{\scriptsize{-0.075}} & \mc{1}{c}{\scriptsize{-0.091}} & \mc{1}{c}{\scriptsize{-0.046}} & \mc{1}{c}{\scriptsize{-0.027}} & \mc{1}{c}{\scriptsize{-0.037}} & \mc{1}{c}{\scriptsize{-0.026}} \\  

     &  & \mc{1}{c}{\scriptsize{(0.303)}} & \mc{1}{c}{\scriptsize{(0.329)}} & \mc{1}{c}{\scriptsize{(0.413)}} & \mc{1}{c}{\scriptsize{(0.440)}} & \mc{1}{c}{\scriptsize{(0.507)}} & \mc{1}{c}{\scriptsize{(0.421)}} & \mc{1}{c}{\scriptsize{(0.421)}} & \mc{1}{c}{\scriptsize{(0.618)}} \\  

    \mc{1}{l}{\scriptsize{Abdominal Obesity}} & \mc{1}{c}{\scriptsize{Mid-30s}} & \mc{1}{c}{\scriptsize{-0.187}} & \mc{1}{c}{\scriptsize{-0.135}} & \mc{1}{c}{\scriptsize{-0.144}} & \mc{1}{c}{\scriptsize{0.112}} & \mc{1}{c}{\scriptsize{0.036}} & \mc{1}{c}{\scriptsize{-0.186}} & \mc{1}{c}{\scriptsize{-0.199}} & \mc{1}{c}{\scriptsize{-0.151}} \\  

     &  & \mc{1}{c}{\scriptsize{(0.263)}} & \mc{1}{c}{\scriptsize{(0.474)}} & \mc{1}{c}{\scriptsize{(0.640)}} & \mc{1}{c}{\scriptsize{(0.920)}} & \mc{1}{c}{\scriptsize{(0.800)}} & \mc{1}{c}{\scriptsize{(0.237)}} & \mc{1}{c}{\scriptsize{(0.316)}} & \mc{1}{c}{\scriptsize{(0.421)}} \\  

    \mc{1}{l}{\scriptsize{Framingham Risk Score}} & \mc{1}{c}{\scriptsize{Mid-30s}} & \mc{1}{c}{\scriptsize{0.133}} & \mc{1}{c}{\scriptsize{-1.118}} & \mc{1}{c}{\scriptsize{1.178}} & \mc{1}{c}{\scriptsize{0.578}} & \mc{1}{c}{\scriptsize{1.300}} & \mc{1}{c}{\scriptsize{0.171}} & \mc{1}{c}{\scriptsize{-1.411}} & \mc{1}{c}{\scriptsize{-0.134}} \\  

     &  & \mc{1}{c}{\scriptsize{(0.947)}} & \mc{1}{c}{\scriptsize{(0.237)}} & \mc{1}{c}{\scriptsize{(1.000)}} & \mc{1}{c}{\scriptsize{(0.893)}} & \mc{1}{c}{\scriptsize{(1.000)}} & \mc{1}{c}{\scriptsize{(0.961)}} & \mc{1}{c}{\scriptsize{(0.145)}} & \mc{1}{c}{\scriptsize{(0.855)}} \\  

    \mc{1}{l}{\scriptsize{Obesity Factor}} & \mc{1}{c}{\scriptsize{Mid-30s}} & \mc{1}{c}{\scriptsize{-0.329}} & \mc{1}{c}{\scriptsize{-0.247}} & \mc{1}{c}{\scriptsize{-0.343}} & \mc{1}{c}{\scriptsize{-0.601}} & \mc{1}{c}{\scriptsize{0.108}} & \mc{1}{c}{\scriptsize{-0.358}} & \mc{1}{c}{\scriptsize{-0.261}} & \mc{1}{c}{\scriptsize{-0.281}} \\  

     &  & \mc{1}{c}{\scriptsize{(0.434)}} & \mc{1}{c}{\scriptsize{(0.605)}} & \mc{1}{c}{\scriptsize{(0.640)}} & \mc{1}{c}{\scriptsize{(0.480)}} & \mc{1}{c}{\scriptsize{(0.800)}} & \mc{1}{c}{\scriptsize{(0.355)}} & \mc{1}{c}{\scriptsize{(0.632)}} & \mc{1}{c}{\scriptsize{(0.553)}} \\  

  \bottomrule
  \end{tabular}
	\end{table} 

	\begin{table}[H]
     \caption{Treatment Effects on HOME Scores, Pooled Sample}
     \label{table:abccare_rslt_pooled_cat18_sd}
	  \begin{tabular}{cccccccccc}
  \toprule

    \scriptsize{Variable} & \scriptsize{Age} & \scriptsize{(1)} & \scriptsize{(2)} & \scriptsize{(3)} & \scriptsize{(4)} & \scriptsize{(5)} & \scriptsize{(6)} & \scriptsize{(7)} & \scriptsize{(8)} \\ 
    \midrule  

    \mc{1}{l}{\scriptsize{Room density (room/people)}} & \mc{1}{c}{\scriptsize{30}} & \mc{1}{c}{\scriptsize{0.109}} & \mc{1}{c}{\scriptsize{0.070}} & \mc{1}{c}{\scriptsize{-0.053}} & \mc{1}{c}{\scriptsize{-0.318}} & \mc{1}{c}{\scriptsize{-0.078}} & \mc{1}{c}{\scriptsize{0.165}} & \mc{1}{c}{\scriptsize{0.132}} & \mc{1}{c}{\scriptsize{0.140}} \\  

     &  & \mc{1}{c}{\scriptsize{(0.645)}} & \mc{1}{c}{\scriptsize{(0.776)}} & \mc{1}{c}{\scriptsize{(0.829)}} & \mc{1}{c}{\scriptsize{(0.934)}} & \mc{1}{c}{\scriptsize{(0.842)}} & \mc{1}{c}{\scriptsize{(0.566)}} & \mc{1}{c}{\scriptsize{(0.737)}} & \mc{1}{c}{\scriptsize{(0.605)}} \\  

    \mc{1}{l}{\scriptsize{Own computers}} & \mc{1}{c}{\scriptsize{30}} & \mc{1}{c}{\scriptsize{0.047}} & \mc{1}{c}{\scriptsize{0.007}} & \mc{1}{c}{\scriptsize{0.056}} & \mc{1}{c}{\scriptsize{0.043}} & \mc{1}{c}{\scriptsize{0.037}} & \mc{1}{c}{\scriptsize{0.043}} & \mc{1}{c}{\scriptsize{0.009}} & \mc{1}{c}{\scriptsize{0.016}} \\  

     &  & \mc{1}{c}{\scriptsize{(0.579)}} & \mc{1}{c}{\scriptsize{(0.842)}} & \mc{1}{c}{\scriptsize{(0.645)}} & \mc{1}{c}{\scriptsize{(0.697)}} & \mc{1}{c}{\scriptsize{(0.684)}} & \mc{1}{c}{\scriptsize{(0.697)}} & \mc{1}{c}{\scriptsize{(0.829)}} & \mc{1}{c}{\scriptsize{(0.776)}} \\  

    \mc{1}{l}{\scriptsize{Own cars}} & \mc{1}{c}{\scriptsize{30}} & \mc{1}{c}{\scriptsize{0.095}} & \mc{1}{c}{\scriptsize{0.104}} & \mc{1}{c}{\scriptsize{0.215}} & \mc{1}{c}{\scriptsize{0.154}} & \mc{1}{c}{\scriptsize{0.252}} & \mc{1}{c}{\scriptsize{0.075}} & \mc{1}{c}{\scriptsize{0.085}} & \mc{1}{c}{\scriptsize{0.102}} \\  

     &  & \mc{1}{c}{\scriptsize{(0.329)}} & \mc{1}{c}{\scriptsize{(0.316)}} & \mc{1}{c}{\scriptsize{(0.250)}} & \mc{1}{c}{\scriptsize{(0.421)}} & \mc{1}{c}{\scriptsize{(0.158)}} & \mc{1}{c}{\scriptsize{(0.474)}} & \mc{1}{c}{\scriptsize{(0.500)}} & \mc{1}{c}{\scriptsize{(0.316)}} \\  

    \mc{1}{l}{\scriptsize{Own residences}} & \mc{1}{c}{\scriptsize{30}} & \mc{1}{c}{\scriptsize{0.055}} & \mc{1}{c}{\scriptsize{0.026}} & \mc{1}{c}{\scriptsize{0.046}} & \mc{1}{c}{\scriptsize{0.069}} & \mc{1}{c}{\scriptsize{0.042}} & \mc{1}{c}{\scriptsize{0.066}} & \mc{1}{c}{\scriptsize{0.034}} & \mc{1}{c}{\scriptsize{0.054}} \\  

     &  & \mc{1}{c}{\scriptsize{(0.487)}} & \mc{1}{c}{\scriptsize{(0.763)}} & \mc{1}{c}{\scriptsize{(0.645)}} & \mc{1}{c}{\scriptsize{(0.632)}} & \mc{1}{c}{\scriptsize{(0.671)}} & \mc{1}{c}{\scriptsize{(0.434)}} & \mc{1}{c}{\scriptsize{(0.776)}} & \mc{1}{c}{\scriptsize{(0.579)}} \\  

  \bottomrule
  \end{tabular}
	\end{table} 

	\begin{table}[H]
     \caption{Treatment Effects on Relation with Spouse, Pooled Sample}
     \label{table:abccare_rslt_pooled_cat19_sd}
	  \begin{tabular}{cccccccccc}
  \toprule

    \scriptsize{Variable} & \scriptsize{Age} & \scriptsize{(1)} & \scriptsize{(2)} & \scriptsize{(3)} & \scriptsize{(4)} & \scriptsize{(5)} & \scriptsize{(6)} & \scriptsize{(7)} & \scriptsize{(8)} \\ 
    \midrule  

    \mc{1}{l}{\scriptsize{No trouble with spouse family}} & \mc{1}{c}{\scriptsize{30}} & \mc{1}{c}{\scriptsize{-0.146}} & \mc{1}{c}{\scriptsize{-0.235}} & \mc{1}{c}{\scriptsize{-0.187}} & \mc{1}{c}{\scriptsize{-0.141}} & \mc{1}{c}{\scriptsize{-0.285}} & \mc{1}{c}{\scriptsize{-0.126}} & \mc{1}{c}{\scriptsize{-0.257}} & \mc{1}{c}{\scriptsize{-0.261}} \\  

     &  & \mc{1}{c}{\scriptsize{(1.000)}} & \mc{1}{c}{\scriptsize{(1.000)}} & \mc{1}{c}{\scriptsize{(1.000)}} & \mc{1}{c}{\scriptsize{(0.987)}} & \mc{1}{c}{\scriptsize{(1.000)}} & \mc{1}{c}{\scriptsize{(0.974)}} & \mc{1}{c}{\scriptsize{(1.000)}} & \mc{1}{c}{\scriptsize{(1.000)}} \\  

    \mc{1}{l}{\scriptsize{Get along well with spouse}} & \mc{1}{c}{\scriptsize{30}} & \mc{1}{c}{\scriptsize{0.027}} & \mc{1}{c}{\scriptsize{-0.063}} & \mc{1}{c}{\scriptsize{0.013}} & \mc{1}{c}{\scriptsize{0.156}} & \mc{1}{c}{\scriptsize{-0.105}} & \mc{1}{c}{\scriptsize{0.048}} & \mc{1}{c}{\scriptsize{-0.095}} & \mc{1}{c}{\scriptsize{-0.125}} \\  

     &  & \mc{1}{c}{\scriptsize{(0.737)}} & \mc{1}{c}{\scriptsize{(0.961)}} & \mc{1}{c}{\scriptsize{(0.907)}} & \mc{1}{c}{\scriptsize{(0.627)}} & \mc{1}{c}{\scriptsize{(0.947)}} & \mc{1}{c}{\scriptsize{(0.645)}} & \mc{1}{c}{\scriptsize{(1.000)}} & \mc{1}{c}{\scriptsize{(0.947)}} \\  

    \mc{1}{l}{\scriptsize{No disagreement on living arrangement}} & \mc{1}{c}{\scriptsize{30}} & \mc{1}{c}{\scriptsize{0.225}} & \mc{1}{c}{\scriptsize{0.168}} & \mc{1}{c}{\scriptsize{0.542}} & \mc{1}{c}{\scriptsize{0.470}} & \mc{1}{c}{\scriptsize{0.557}} & \mc{1}{c}{\scriptsize{0.177}} & \mc{1}{c}{\scriptsize{0.114}} & \mc{1}{c}{\scriptsize{0.094}} \\  

     &  & \mc{1}{c}{\scriptsize{\textbf{(0.079)}}} & \mc{1}{c}{\scriptsize{(0.342)}} & \mc{1}{c}{\scriptsize{\textbf{(0.013)}}} & \mc{1}{c}{\scriptsize{(0.133)}} & \mc{1}{c}{\scriptsize{\textbf{(0.013)}}} & \mc{1}{c}{\scriptsize{(0.250)}} & \mc{1}{c}{\scriptsize{(0.539)}} & \mc{1}{c}{\scriptsize{(0.553)}} \\  

  \bottomrule
  \end{tabular}
	\end{table} 

	\begin{table}[H]
     \caption{Treatment Effects on Spouse Characteristics, Pooled Sample}
     \label{table:abccare_rslt_pooled_cat20_sd}
	  \begin{tabular}{cccccccccc}
  \toprule

    \scriptsize{Variable} & \scriptsize{Age} & \scriptsize{(1)} & \scriptsize{(2)} & \scriptsize{(3)} & \scriptsize{(4)} & \scriptsize{(5)} & \scriptsize{(6)} & \scriptsize{(7)} & \scriptsize{(8)} \\ 
    \midrule  

    \mc{1}{l}{\scriptsize{Spouse annual income}} & \mc{1}{c}{\scriptsize{30}} & \mc{1}{c}{\scriptsize{3,221}} & \mc{1}{c}{\scriptsize{4,380}} & \mc{1}{c}{\scriptsize{-3,085}} & \mc{1}{c}{\scriptsize{-8,302}} & \mc{1}{c}{\scriptsize{-6,349}} & \mc{1}{c}{\scriptsize{3,211}} & \mc{1}{c}{\scriptsize{4,359}} & \mc{1}{c}{\scriptsize{402}} \\  

     &  & \mc{1}{c}{\scriptsize{(0.408)}} & \mc{1}{c}{\scriptsize{(0.434)}} & \mc{1}{c}{\scriptsize{(0.800)}} & \mc{1}{c}{\scriptsize{(0.760)}} & \mc{1}{c}{\scriptsize{(0.867)}} & \mc{1}{c}{\scriptsize{(0.395)}} & \mc{1}{c}{\scriptsize{(0.434)}} & \mc{1}{c}{\scriptsize{(0.579)}} \\  

    \mc{1}{l}{\scriptsize{Spouse employment status}} & \mc{1}{c}{\scriptsize{30}} & \mc{1}{c}{\scriptsize{-0.127}} & \mc{1}{c}{\scriptsize{-0.274}} & \mc{1}{c}{\scriptsize{-0.231}} & \mc{1}{c}{\scriptsize{-0.463}} & \mc{1}{c}{\scriptsize{-0.433}} & \mc{1}{c}{\scriptsize{-0.147}} & \mc{1}{c}{\scriptsize{-0.282}} & \mc{1}{c}{\scriptsize{-0.323}} \\  

     &  & \mc{1}{c}{\scriptsize{(0.987)}} & \mc{1}{c}{\scriptsize{(0.987)}} & \mc{1}{c}{\scriptsize{(1.000)}} & \mc{1}{c}{\scriptsize{(0.973)}} & \mc{1}{c}{\scriptsize{(1.000)}} & \mc{1}{c}{\scriptsize{(1.000)}} & \mc{1}{c}{\scriptsize{(0.987)}} & \mc{1}{c}{\scriptsize{(1.000)}} \\  

  \bottomrule
  \end{tabular}
	\end{table} 

	\begin{table}[H]
     \caption{Treatment Effects on Subject Home and Property, Pooled Sample}
     \label{table:abccare_rslt_pooled_cat21_sd}
	\input{AppResOutput/abccare/rslt_pooled_cat21_sd}
	\end{table} 

	\begin{table}[H]
     \caption{Treatment Effects on Education, Pooled Sample}
     \label{table:abccare_rslt_pooled_cat22_sd}
	  \begin{tabular}{cccccccccc}
  \toprule

    \scriptsize{Variable} & \scriptsize{Age} & \scriptsize{(1)} & \scriptsize{(2)} & \scriptsize{(3)} & \scriptsize{(4)} & \scriptsize{(5)} & \scriptsize{(6)} & \scriptsize{(7)} & \scriptsize{(8)} \\ 
    \midrule  

    \mc{1}{l}{\scriptsize{Household Earned Income}} & \mc{1}{c}{\scriptsize{9}} & \mc{1}{c}{\scriptsize{17,425}} & \mc{1}{c}{\scriptsize{724}} & \mc{1}{c}{\scriptsize{16,656}} & \mc{1}{c}{\scriptsize{31,008}} & \mc{1}{c}{\scriptsize{12,782}} & \mc{1}{c}{\scriptsize{20,504}} & \mc{1}{c}{\scriptsize{-65,101}} & \mc{1}{c}{\scriptsize{16,741}} \\  

     &  & \mc{1}{c}{\scriptsize{\textbf{(0.053)}}} & \mc{1}{c}{\scriptsize{(0.934)}} & \mc{1}{c}{\scriptsize{\textbf{(0.053)}}} & \mc{1}{c}{\scriptsize{(0.553)}} & \mc{1}{c}{\scriptsize{\textbf{(0.079)}}} & \mc{1}{c}{\scriptsize{\textbf{(0.000)}}} & \mc{1}{c}{\scriptsize{(0.882)}} & \mc{1}{c}{\scriptsize{\textbf{(0.000)}}} \\  

     & \mc{1}{c}{\scriptsize{0}} & \mc{1}{c}{\scriptsize{-1,594}} & \mc{1}{c}{\scriptsize{3,403}} & \mc{1}{c}{\scriptsize{-982}} & \mc{1}{c}{\scriptsize{-652}} & \mc{1}{c}{\scriptsize{2,188}} & \mc{1}{c}{\scriptsize{-2,593}} & \mc{1}{c}{\scriptsize{3,337}} & \mc{1}{c}{\scriptsize{128}} \\  

     &  & \mc{1}{c}{\scriptsize{(0.987)}} & \mc{1}{c}{\scriptsize{(0.329)}} & \mc{1}{c}{\scriptsize{(0.947)}} & \mc{1}{c}{\scriptsize{(0.618)}} & \mc{1}{c}{\scriptsize{(0.816)}} & \mc{1}{c}{\scriptsize{(0.987)}} & \mc{1}{c}{\scriptsize{(0.539)}} & \mc{1}{c}{\scriptsize{(0.816)}} \\  

     & \mc{1}{c}{\scriptsize{12}} & \mc{1}{c}{\scriptsize{8,249}} & \mc{1}{c}{\scriptsize{9,060}} & \mc{1}{c}{\scriptsize{16,420}} & \mc{1}{c}{\scriptsize{20,483}} & \mc{1}{c}{\scriptsize{15,047}} & \mc{1}{c}{\scriptsize{6,159}} & \mc{1}{c}{\scriptsize{6,399}} & \mc{1}{c}{\scriptsize{6,205}} \\  

     &  & \mc{1}{c}{\scriptsize{\textbf{(0.079)}}} & \mc{1}{c}{\scriptsize{\textbf{(0.092)}}} & \mc{1}{c}{\scriptsize{\textbf{(0.013)}}} & \mc{1}{c}{\scriptsize{\textbf{(0.066)}}} & \mc{1}{c}{\scriptsize{\textbf{(0.026)}}} & \mc{1}{c}{\scriptsize{(0.184)}} & \mc{1}{c}{\scriptsize{(0.171)}} & \mc{1}{c}{\scriptsize{(0.197)}} \\  

  \bottomrule
  \end{tabular}
	\end{table} 

	\begin{table}[H]
     \caption{Treatment Effects on Subject Employment and Income, Pooled Sample}
     \label{table:abccare_rslt_pooled_cat23_sd}
	  \begin{tabular}{cccccccccc}
  \toprule

    \scriptsize{Variable} & \scriptsize{Age} & \scriptsize{(1)} & \scriptsize{(2)} & \scriptsize{(3)} & \scriptsize{(4)} & \scriptsize{(5)} & \scriptsize{(6)} & \scriptsize{(7)} & \scriptsize{(8)} \\ 
    \midrule  

    \mc{1}{l}{\scriptsize{Father at Home}} & \mc{1}{c}{\scriptsize{3}} & \mc{1}{c}{\scriptsize{-0.076}} & \mc{1}{c}{\scriptsize{-0.015}} & \mc{1}{c}{\scriptsize{-0.291}} & \mc{1}{c}{\scriptsize{-0.226}} & \mc{1}{c}{\scriptsize{-0.258}} & \mc{1}{c}{\scriptsize{0.002}} & \mc{1}{c}{\scriptsize{0.066}} & \mc{1}{c}{\scriptsize{0.079}} \\  

     &  & \mc{1}{c}{\scriptsize{(0.961)}} & \mc{1}{c}{\scriptsize{(0.895)}} & \mc{1}{c}{\scriptsize{(1.000)}} & \mc{1}{c}{\scriptsize{(0.987)}} & \mc{1}{c}{\scriptsize{(1.000)}} & \mc{1}{c}{\scriptsize{(0.697)}} & \mc{1}{c}{\scriptsize{(0.355)}} & \mc{1}{c}{\scriptsize{(0.316)}} \\  

     & \mc{1}{c}{\scriptsize{2}} & \mc{1}{c}{\scriptsize{-0.010}} & \mc{1}{c}{\scriptsize{0.063}} & \mc{1}{c}{\scriptsize{-0.187}} & \mc{1}{c}{\scriptsize{-0.116}} & \mc{1}{c}{\scriptsize{-0.148}} & \mc{1}{c}{\scriptsize{0.047}} & \mc{1}{c}{\scriptsize{0.142}} & \mc{1}{c}{\scriptsize{0.130}} \\  

     &  & \mc{1}{c}{\scriptsize{(0.776)}} & \mc{1}{c}{\scriptsize{(0.408)}} & \mc{1}{c}{\scriptsize{(1.000)}} & \mc{1}{c}{\scriptsize{(0.947)}} & \mc{1}{c}{\scriptsize{(0.961)}} & \mc{1}{c}{\scriptsize{(0.487)}} & \mc{1}{c}{\scriptsize{\textbf{(0.092)}}} & \mc{1}{c}{\scriptsize{(0.132)}} \\  

     & \mc{1}{c}{\scriptsize{8}} & \mc{1}{c}{\scriptsize{0.052}} & \mc{1}{c}{\scriptsize{0.041}} & \mc{1}{c}{\scriptsize{-0.124}} & \mc{1}{c}{\scriptsize{-0.060}} & \mc{1}{c}{\scriptsize{-0.176}} & \mc{1}{c}{\scriptsize{0.113}} & \mc{1}{c}{\scriptsize{0.083}} & \mc{1}{c}{\scriptsize{0.096}} \\  

     &  & \mc{1}{c}{\scriptsize{(0.461)}} & \mc{1}{c}{\scriptsize{(0.579)}} & \mc{1}{c}{\scriptsize{(0.961)}} & \mc{1}{c}{\scriptsize{(0.908)}} & \mc{1}{c}{\scriptsize{(0.961)}} & \mc{1}{c}{\scriptsize{(0.184)}} & \mc{1}{c}{\scriptsize{(0.329)}} & \mc{1}{c}{\scriptsize{(0.289)}} \\  

     & \mc{1}{c}{\scriptsize{5}} & \mc{1}{c}{\scriptsize{-0.093}} & \mc{1}{c}{\scriptsize{-0.026}} & \mc{1}{c}{\scriptsize{-0.369}} & \mc{1}{c}{\scriptsize{-0.249}} & \mc{1}{c}{\scriptsize{-0.343}} & \mc{1}{c}{\scriptsize{-0.006}} & \mc{1}{c}{\scriptsize{0.042}} & \mc{1}{c}{\scriptsize{0.062}} \\  

     &  & \mc{1}{c}{\scriptsize{(0.974)}} & \mc{1}{c}{\scriptsize{(0.921)}} & \mc{1}{c}{\scriptsize{(1.000)}} & \mc{1}{c}{\scriptsize{(0.987)}} & \mc{1}{c}{\scriptsize{(1.000)}} & \mc{1}{c}{\scriptsize{(0.763)}} & \mc{1}{c}{\scriptsize{(0.566)}} & \mc{1}{c}{\scriptsize{(0.408)}} \\  

     & \mc{1}{c}{\scriptsize{4}} & \mc{1}{c}{\scriptsize{-0.071}} & \mc{1}{c}{\scriptsize{-0.009}} & \mc{1}{c}{\scriptsize{-0.331}} & \mc{1}{c}{\scriptsize{-0.252}} & \mc{1}{c}{\scriptsize{-0.293}} & \mc{1}{c}{\scriptsize{0.021}} & \mc{1}{c}{\scriptsize{0.074}} & \mc{1}{c}{\scriptsize{0.101}} \\  

     &  & \mc{1}{c}{\scriptsize{(0.934)}} & \mc{1}{c}{\scriptsize{(0.868)}} & \mc{1}{c}{\scriptsize{(1.000)}} & \mc{1}{c}{\scriptsize{(1.000)}} & \mc{1}{c}{\scriptsize{(1.000)}} & \mc{1}{c}{\scriptsize{(0.632)}} & \mc{1}{c}{\scriptsize{(0.368)}} & \mc{1}{c}{\scriptsize{(0.250)}} \\  

  \bottomrule
  \end{tabular}
	\end{table} 

	\begin{table}[H]
     \caption{Treatment Effects on Job Attitude, Pooled Sample}
     \label{table:abccare_rslt_pooled_cat24_sd}
	  \begin{tabular}{cccccccccc}
  \toprule

    \scriptsize{Variable} & \scriptsize{Age} & \scriptsize{(1)} & \scriptsize{(2)} & \scriptsize{(3)} & \scriptsize{(4)} & \scriptsize{(5)} & \scriptsize{(6)} & \scriptsize{(7)} & \scriptsize{(8)} \\ 
    \midrule  

    \mc{1}{l}{\scriptsize{Satisfied with working situation}} & \mc{1}{c}{\scriptsize{30}} & \mc{1}{c}{\scriptsize{0.009}} & \mc{1}{c}{\scriptsize{0.084}} & \mc{1}{c}{\scriptsize{-0.031}} & \mc{1}{c}{\scriptsize{-0.146}} & \mc{1}{c}{\scriptsize{0.060}} & \mc{1}{c}{\scriptsize{0.037}} & \mc{1}{c}{\scriptsize{0.159}} & \mc{1}{c}{\scriptsize{0.150}} \\  

     &  & \mc{1}{c}{\scriptsize{(0.974)}} & \mc{1}{c}{\scriptsize{(0.750)}} & \mc{1}{c}{\scriptsize{(0.961)}} & \mc{1}{c}{\scriptsize{(0.987)}} & \mc{1}{c}{\scriptsize{(0.921)}} & \mc{1}{c}{\scriptsize{(0.947)}} & \mc{1}{c}{\scriptsize{(0.513)}} & \mc{1}{c}{\scriptsize{(0.408)}} \\  

    \mc{1}{l}{\scriptsize{Do work well}} & \mc{1}{c}{\scriptsize{30}} & \mc{1}{c}{\scriptsize{0.019}} & \mc{1}{c}{\scriptsize{0.015}} & \mc{1}{c}{\scriptsize{-0.061}} & \mc{1}{c}{\scriptsize{-0.047}} & \mc{1}{c}{\scriptsize{-0.063}} & \mc{1}{c}{\scriptsize{0.047}} & \mc{1}{c}{\scriptsize{0.031}} & \mc{1}{c}{\scriptsize{0.040}} \\  

     &  & \mc{1}{c}{\scriptsize{(0.895)}} & \mc{1}{c}{\scriptsize{(0.921)}} & \mc{1}{c}{\scriptsize{(1.000)}} & \mc{1}{c}{\scriptsize{(1.000)}} & \mc{1}{c}{\scriptsize{(1.000)}} & \mc{1}{c}{\scriptsize{(0.763)}} & \mc{1}{c}{\scriptsize{(0.803)}} & \mc{1}{c}{\scriptsize{(0.855)}} \\  

    \mc{1}{l}{\scriptsize{Not worry too much about work}} & \mc{1}{c}{\scriptsize{30}} & \mc{1}{c}{\scriptsize{0.028}} & \mc{1}{c}{\scriptsize{0.090}} & \mc{1}{c}{\scriptsize{0.114}} & \mc{1}{c}{\scriptsize{0.249}} & \mc{1}{c}{\scriptsize{0.151}} & \mc{1}{c}{\scriptsize{0.004}} & \mc{1}{c}{\scriptsize{0.080}} & \mc{1}{c}{\scriptsize{0.018}} \\  

     &  & \mc{1}{c}{\scriptsize{(0.908)}} & \mc{1}{c}{\scriptsize{(0.724)}} & \mc{1}{c}{\scriptsize{(0.842)}} & \mc{1}{c}{\scriptsize{(0.618)}} & \mc{1}{c}{\scriptsize{(0.789)}} & \mc{1}{c}{\scriptsize{(0.987)}} & \mc{1}{c}{\scriptsize{(0.776)}} & \mc{1}{c}{\scriptsize{(0.947)}} \\  

    \mc{1}{l}{\scriptsize{Work well with others}} & \mc{1}{c}{\scriptsize{30}} & \mc{1}{c}{\scriptsize{0.037}} & \mc{1}{c}{\scriptsize{-0.026}} & \mc{1}{c}{\scriptsize{0.057}} & \mc{1}{c}{\scriptsize{0.012}} & \mc{1}{c}{\scriptsize{-0.002}} & \mc{1}{c}{\scriptsize{0.019}} & \mc{1}{c}{\scriptsize{-0.003}} & \mc{1}{c}{\scriptsize{-0.035}} \\  

     &  & \mc{1}{c}{\scriptsize{(0.868)}} & \mc{1}{c}{\scriptsize{(0.987)}} & \mc{1}{c}{\scriptsize{(0.921)}} & \mc{1}{c}{\scriptsize{(0.947)}} & \mc{1}{c}{\scriptsize{(0.974)}} & \mc{1}{c}{\scriptsize{(0.947)}} & \mc{1}{c}{\scriptsize{(0.987)}} & \mc{1}{c}{\scriptsize{(1.000)}} \\  

    \mc{1}{l}{\scriptsize{Don't do things that cause to lose job}} & \mc{1}{c}{\scriptsize{30}} & \mc{1}{c}{\scriptsize{0.014}} & \mc{1}{c}{\scriptsize{0.070}} & \mc{1}{c}{\scriptsize{-0.143}} & \mc{1}{c}{\scriptsize{-0.108}} & \mc{1}{c}{\scriptsize{-0.140}} & \mc{1}{c}{\scriptsize{0.041}} & \mc{1}{c}{\scriptsize{0.085}} & \mc{1}{c}{\scriptsize{0.065}} \\  

     &  & \mc{1}{c}{\scriptsize{(0.961)}} & \mc{1}{c}{\scriptsize{(0.724)}} & \mc{1}{c}{\scriptsize{(1.000)}} & \mc{1}{c}{\scriptsize{(1.000)}} & \mc{1}{c}{\scriptsize{(1.000)}} & \mc{1}{c}{\scriptsize{(0.895)}} & \mc{1}{c}{\scriptsize{(0.645)}} & \mc{1}{c}{\scriptsize{(0.842)}} \\  

    \mc{1}{l}{\scriptsize{No trouble finishing work}} & \mc{1}{c}{\scriptsize{30}} & \mc{1}{c}{\scriptsize{-0.082}} & \mc{1}{c}{\scriptsize{-0.149}} & \mc{1}{c}{\scriptsize{-0.022}} & \mc{1}{c}{\scriptsize{-0.175}} & \mc{1}{c}{\scriptsize{-0.104}} & \mc{1}{c}{\scriptsize{-0.095}} & \mc{1}{c}{\scriptsize{-0.123}} & \mc{1}{c}{\scriptsize{-0.151}} \\  

     &  & \mc{1}{c}{\scriptsize{(1.000)}} & \mc{1}{c}{\scriptsize{(1.000)}} & \mc{1}{c}{\scriptsize{(0.961)}} & \mc{1}{c}{\scriptsize{(1.000)}} & \mc{1}{c}{\scriptsize{(1.000)}} & \mc{1}{c}{\scriptsize{(1.000)}} & \mc{1}{c}{\scriptsize{(1.000)}} & \mc{1}{c}{\scriptsize{(1.000)}} \\  

    \mc{1}{l}{\scriptsize{Job not too stressful}} & \mc{1}{c}{\scriptsize{30}} & \mc{1}{c}{\scriptsize{-0.029}} & \mc{1}{c}{\scriptsize{0.003}} & \mc{1}{c}{\scriptsize{-0.245}} & \mc{1}{c}{\scriptsize{-0.190}} & \mc{1}{c}{\scriptsize{-0.186}} & \mc{1}{c}{\scriptsize{0.018}} & \mc{1}{c}{\scriptsize{0.040}} & \mc{1}{c}{\scriptsize{0.068}} \\  

     &  & \mc{1}{c}{\scriptsize{(1.000)}} & \mc{1}{c}{\scriptsize{(0.974)}} & \mc{1}{c}{\scriptsize{(1.000)}} & \mc{1}{c}{\scriptsize{(1.000)}} & \mc{1}{c}{\scriptsize{(1.000)}} & \mc{1}{c}{\scriptsize{(0.947)}} & \mc{1}{c}{\scriptsize{(0.947)}} & \mc{1}{c}{\scriptsize{(0.776)}} \\  

    \mc{1}{l}{\scriptsize{Don't stay away from job}} & \mc{1}{c}{\scriptsize{30}} & \mc{1}{c}{\scriptsize{-0.049}} & \mc{1}{c}{\scriptsize{-0.022}} & \mc{1}{c}{\scriptsize{-0.029}} & \mc{1}{c}{\scriptsize{0.006}} & \mc{1}{c}{\scriptsize{-0.008}} & \mc{1}{c}{\scriptsize{-0.040}} & \mc{1}{c}{\scriptsize{-0.025}} & \mc{1}{c}{\scriptsize{-0.073}} \\  

     &  & \mc{1}{c}{\scriptsize{(1.000)}} & \mc{1}{c}{\scriptsize{(0.987)}} & \mc{1}{c}{\scriptsize{(0.974)}} & \mc{1}{c}{\scriptsize{(0.947)}} & \mc{1}{c}{\scriptsize{(0.974)}} & \mc{1}{c}{\scriptsize{(1.000)}} & \mc{1}{c}{\scriptsize{(1.000)}} & \mc{1}{c}{\scriptsize{(1.000)}} \\  

    \mc{1}{l}{\scriptsize{No trouble with boss}} & \mc{1}{c}{\scriptsize{30}} & \mc{1}{c}{\scriptsize{0.119}} & \mc{1}{c}{\scriptsize{0.132}} & \mc{1}{c}{\scriptsize{-0.087}} & \mc{1}{c}{\scriptsize{0.015}} & \mc{1}{c}{\scriptsize{-0.059}} & \mc{1}{c}{\scriptsize{0.174}} & \mc{1}{c}{\scriptsize{0.164}} & \mc{1}{c}{\scriptsize{0.168}} \\  

     &  & \mc{1}{c}{\scriptsize{(0.382)}} & \mc{1}{c}{\scriptsize{(0.395)}} & \mc{1}{c}{\scriptsize{(1.000)}} & \mc{1}{c}{\scriptsize{(0.947)}} & \mc{1}{c}{\scriptsize{(0.987)}} & \mc{1}{c}{\scriptsize{(0.237)}} & \mc{1}{c}{\scriptsize{(0.342)}} & \mc{1}{c}{\scriptsize{(0.263)}} \\  

  \bottomrule
  \end{tabular}
	\end{table} 

	\begin{table}[H]
     \caption{Treatment Effects on Job Satisfaction Score, Pooled Sample}
     \label{table:abccare_rslt_pooled_cat25_sd}
	\input{AppResOutput/abccare/rslt_pooled_cat25_sd}
	\end{table} 

	\begin{table}[H]
     \caption{Treatment Effects on Crime, Pooled Sample}
     \label{table:abccare_rslt_pooled_cat26_sd}
	\input{AppResOutput/abccare/rslt_pooled_cat26_sd}
	\end{table} 

	\begin{table}[H]
     \caption{Treatment Effects on Childhood and Adolescence Physical Health, Pooled Sample}
     \label{table:abccare_rslt_pooled_cat27_sd}
	  \begin{tabular}{cccccccccc}
  \toprule

    \scriptsize{Variable} & \scriptsize{Age} & \scriptsize{(1)} & \scriptsize{(2)} & \scriptsize{(3)} & \scriptsize{(4)} & \scriptsize{(5)} & \scriptsize{(6)} & \scriptsize{(7)} & \scriptsize{(8)} \\ 
    \midrule  

    \mc{1}{l}{\scriptsize{Total Misdemeanor Arrests}} & \mc{1}{c}{\scriptsize{Mid-30s}} & \mc{1}{c}{\scriptsize{-0.689}} & \mc{1}{c}{\scriptsize{-0.481}} & \mc{1}{c}{\scriptsize{-1.445}} & \mc{1}{c}{\scriptsize{-1.198}} & \mc{1}{c}{\scriptsize{-1.277}} & \mc{1}{c}{\scriptsize{-0.546}} & \mc{1}{c}{\scriptsize{-0.293}} & \mc{1}{c}{\scriptsize{-0.310}} \\  

     &  & \mc{1}{c}{\scriptsize{\textbf{(0.092)}}} & \mc{1}{c}{\scriptsize{(0.395)}} & \mc{1}{c}{\scriptsize{(0.237)}} & \mc{1}{c}{\scriptsize{(0.408)}} & \mc{1}{c}{\scriptsize{(0.395)}} & \mc{1}{c}{\scriptsize{(0.197)}} & \mc{1}{c}{\scriptsize{(0.474)}} & \mc{1}{c}{\scriptsize{(0.461)}} \\  

    \mc{1}{l}{\scriptsize{Total Felony Arrests}} & \mc{1}{c}{\scriptsize{Mid-30s}} & \mc{1}{c}{\scriptsize{0.045}} & \mc{1}{c}{\scriptsize{0.237}} & \mc{1}{c}{\scriptsize{-0.132}} & \mc{1}{c}{\scriptsize{0.227}} & \mc{1}{c}{\scriptsize{0.205}} & \mc{1}{c}{\scriptsize{0.112}} & \mc{1}{c}{\scriptsize{0.224}} & \mc{1}{c}{\scriptsize{0.185}} \\  

     &  & \mc{1}{c}{\scriptsize{(0.829)}} & \mc{1}{c}{\scriptsize{(0.934)}} & \mc{1}{c}{\scriptsize{(0.697)}} & \mc{1}{c}{\scriptsize{(0.908)}} & \mc{1}{c}{\scriptsize{(0.921)}} & \mc{1}{c}{\scriptsize{(0.882)}} & \mc{1}{c}{\scriptsize{(0.947)}} & \mc{1}{c}{\scriptsize{(0.961)}} \\  

    \mc{1}{l}{\scriptsize{Total Years Incarcerated}} & \mc{1}{c}{\scriptsize{30}} & \mc{1}{c}{\scriptsize{0.168}} & \mc{1}{c}{\scriptsize{0.187}} & \mc{1}{c}{\scriptsize{0.280}} & \mc{1}{c}{\scriptsize{0.325}} & \mc{1}{c}{\scriptsize{0.350}} & \mc{1}{c}{\scriptsize{0.153}} & \mc{1}{c}{\scriptsize{0.176}} & \mc{1}{c}{\scriptsize{0.215}} \\  

     &  & \mc{1}{c}{\scriptsize{(1.000)}} & \mc{1}{c}{\scriptsize{(1.000)}} & \mc{1}{c}{\scriptsize{(1.000)}} & \mc{1}{c}{\scriptsize{(1.000)}} & \mc{1}{c}{\scriptsize{(1.000)}} & \mc{1}{c}{\scriptsize{(0.987)}} & \mc{1}{c}{\scriptsize{(1.000)}} & \mc{1}{c}{\scriptsize{(0.987)}} \\  

  \bottomrule
  \end{tabular}
	\end{table} 

	\begin{table}[H]
     \caption{Treatment Effects on Childhood Health Problems, Pooled Sample}
     \label{table:abccare_rslt_pooled_cat28_sd}
	  \begin{tabular}{cccccccccc}
  \toprule

    \scriptsize{Variable} & \scriptsize{Age} & \scriptsize{(1)} & \scriptsize{(2)} & \scriptsize{(3)} & \scriptsize{(4)} & \scriptsize{(5)} & \scriptsize{(6)} & \scriptsize{(7)} & \scriptsize{(8)} \\ 
    \midrule  

    \mc{1}{l}{\scriptsize{Has Health Problems}} & \mc{1}{c}{\scriptsize{12}} & \mc{1}{c}{\scriptsize{0.058}} & \mc{1}{c}{\scriptsize{-0.048}} & \mc{1}{c}{\scriptsize{0.075}} & \mc{1}{c}{\scriptsize{0.040}} & \mc{1}{c}{\scriptsize{0.058}} & \mc{1}{c}{\scriptsize{0.050}} & \mc{1}{c}{\scriptsize{-0.073}} & \mc{1}{c}{\scriptsize{-0.009}} \\  

     &  & \mc{1}{c}{\scriptsize{(0.842)}} & \mc{1}{c}{\scriptsize{(0.487)}} & \mc{1}{c}{\scriptsize{(0.855)}} & \mc{1}{c}{\scriptsize{(0.776)}} & \mc{1}{c}{\scriptsize{(0.868)}} & \mc{1}{c}{\scriptsize{(0.829)}} & \mc{1}{c}{\scriptsize{(0.382)}} & \mc{1}{c}{\scriptsize{(0.618)}} \\  

    \mc{1}{l}{\scriptsize{Ever Hospitalized for Over 1 Week}} & \mc{1}{c}{\scriptsize{12}} & \mc{1}{c}{\scriptsize{-0.014}} & \mc{1}{c}{\scriptsize{-0.034}} & \mc{1}{c}{\scriptsize{0.088}} & \mc{1}{c}{\scriptsize{0.058}} & \mc{1}{c}{\scriptsize{0.074}} & \mc{1}{c}{\scriptsize{-0.043}} & \mc{1}{c}{\scriptsize{-0.061}} & \mc{1}{c}{\scriptsize{-0.052}} \\  

     &  & \mc{1}{c}{\scriptsize{(0.579)}} & \mc{1}{c}{\scriptsize{(0.487)}} & \mc{1}{c}{\scriptsize{(1.000)}} & \mc{1}{c}{\scriptsize{(0.961)}} & \mc{1}{c}{\scriptsize{(1.000)}} & \mc{1}{c}{\scriptsize{(0.421)}} & \mc{1}{c}{\scriptsize{(0.329)}} & \mc{1}{c}{\scriptsize{(0.316)}} \\  

  \bottomrule
  \end{tabular}
	\end{table} 

	\begin{table}[H]
     \caption{Treatment Effects on Cholesterol, Pooled Sample}
     \label{table:abccare_rslt_pooled_cat29_sd}
	  \begin{tabular}{cccccccccc}
  \toprule

    \scriptsize{Variable} & \scriptsize{Age} & \scriptsize{(1)} & \scriptsize{(2)} & \scriptsize{(3)} & \scriptsize{(4)} & \scriptsize{(5)} & \scriptsize{(6)} & \scriptsize{(7)} & \scriptsize{(8)} \\ 
    \midrule  

    \mc{1}{l}{\scriptsize{Dyslipidemia}} & \mc{1}{c}{\scriptsize{Mid-30s}} & \mc{1}{c}{\scriptsize{0.013}} & \mc{1}{c}{\scriptsize{-0.003}} & \mc{1}{c}{\scriptsize{0.035}} & \mc{1}{c}{\scriptsize{0.082}} & \mc{1}{c}{\scriptsize{0.031}} & \mc{1}{c}{\scriptsize{0.032}} & \mc{1}{c}{\scriptsize{-0.009}} & \mc{1}{c}{\scriptsize{0.010}} \\  

     &  & \mc{1}{c}{\scriptsize{(0.763)}} & \mc{1}{c}{\scriptsize{(0.711)}} & \mc{1}{c}{\scriptsize{(0.737)}} & \mc{1}{c}{\scriptsize{(0.763)}} & \mc{1}{c}{\scriptsize{(0.789)}} & \mc{1}{c}{\scriptsize{(0.816)}} & \mc{1}{c}{\scriptsize{(0.671)}} & \mc{1}{c}{\scriptsize{(0.737)}} \\  

    \mc{1}{l}{\scriptsize{High-Density Lipoprotein Chol. (mg/dL)}} & \mc{1}{c}{\scriptsize{Mid-30s}} & \mc{1}{c}{\scriptsize{3.872}} & \mc{1}{c}{\scriptsize{4.790}} & \mc{1}{c}{\scriptsize{5.806}} & \mc{1}{c}{\scriptsize{3.305}} & \mc{1}{c}{\scriptsize{5.518}} & \mc{1}{c}{\scriptsize{2.964}} & \mc{1}{c}{\scriptsize{5.364}} & \mc{1}{c}{\scriptsize{4.456}} \\  

     &  & \mc{1}{c}{\scriptsize{(0.184)}} & \mc{1}{c}{\scriptsize{\textbf{(0.066)}}} & \mc{1}{c}{\scriptsize{(0.105)}} & \mc{1}{c}{\scriptsize{(0.395)}} & \mc{1}{c}{\scriptsize{(0.132)}} & \mc{1}{c}{\scriptsize{(0.276)}} & \mc{1}{c}{\scriptsize{\textbf{(0.053)}}} & \mc{1}{c}{\scriptsize{(0.118)}} \\  

  \bottomrule
  \end{tabular}
	\end{table} 

	\begin{table}[H]
     \caption{Treatment Effects on Current Health Condition (Self-Reported), Pooled Sample}
     \label{table:abccare_rslt_pooled_cat30_sd}
	  \begin{tabular}{cccccccccc}
  \toprule

    \scriptsize{Variable} & \scriptsize{Age} & \scriptsize{(1)} & \scriptsize{(2)} & \scriptsize{(3)} & \scriptsize{(4)} & \scriptsize{(5)} & \scriptsize{(6)} & \scriptsize{(7)} & \scriptsize{(8)} \\ 
    \midrule  

    \mc{1}{l}{\scriptsize{Asthma}} & \mc{1}{c}{\scriptsize{Mid-30s}} & \mc{1}{c}{\scriptsize{0.016}} & \mc{1}{c}{\scriptsize{-0.002}} & \mc{1}{c}{\scriptsize{0.040}} & \mc{1}{c}{\scriptsize{0.021}} & \mc{1}{c}{\scriptsize{0.025}} & \mc{1}{c}{\scriptsize{0.009}} & \mc{1}{c}{\scriptsize{-0.016}} & \mc{1}{c}{\scriptsize{-0.006}} \\  

     &  & \mc{1}{c}{\scriptsize{(0.987)}} & \mc{1}{c}{\scriptsize{(0.947)}} & \mc{1}{c}{\scriptsize{(1.000)}} & \mc{1}{c}{\scriptsize{(1.000)}} & \mc{1}{c}{\scriptsize{(1.000)}} & \mc{1}{c}{\scriptsize{(0.974)}} & \mc{1}{c}{\scriptsize{(0.789)}} & \mc{1}{c}{\scriptsize{(0.882)}} \\  

    \mc{1}{l}{\scriptsize{High Blood Pressure (Hypertension)}} & \mc{1}{c}{\scriptsize{Mid-30s}} & \mc{1}{c}{\scriptsize{-0.003}} & \mc{1}{c}{\scriptsize{-0.010}} & \mc{1}{c}{\scriptsize{0.021}} & \mc{1}{c}{\scriptsize{0.008}} & \mc{1}{c}{\scriptsize{0.026}} & \mc{1}{c}{\scriptsize{-0.010}} & \mc{1}{c}{\scriptsize{-0.017}} & \mc{1}{c}{\scriptsize{-0.007}} \\  

     &  & \mc{1}{c}{\scriptsize{(0.947)}} & \mc{1}{c}{\scriptsize{(0.921)}} & \mc{1}{c}{\scriptsize{(1.000)}} & \mc{1}{c}{\scriptsize{(0.974)}} & \mc{1}{c}{\scriptsize{(1.000)}} & \mc{1}{c}{\scriptsize{(0.921)}} & \mc{1}{c}{\scriptsize{(0.789)}} & \mc{1}{c}{\scriptsize{(0.882)}} \\  

    \mc{1}{l}{\scriptsize{Arthritis or Generative Disease}} & \mc{1}{c}{\scriptsize{Mid-30s}} & \mc{1}{c}{\scriptsize{0.021}} & \mc{1}{c}{\scriptsize{0.024}} & \mc{1}{c}{\scriptsize{0.021}} & \mc{1}{c}{\scriptsize{0.016}} & \mc{1}{c}{\scriptsize{0.022}} & \mc{1}{c}{\scriptsize{0.021}} & \mc{1}{c}{\scriptsize{0.028}} & \mc{1}{c}{\scriptsize{0.022}} \\  

     &  & \mc{1}{c}{\scriptsize{(1.000)}} & \mc{1}{c}{\scriptsize{(1.000)}} & \mc{1}{c}{\scriptsize{(1.000)}} & \mc{1}{c}{\scriptsize{(1.000)}} & \mc{1}{c}{\scriptsize{(1.000)}} & \mc{1}{c}{\scriptsize{(1.000)}} & \mc{1}{c}{\scriptsize{(1.000)}} & \mc{1}{c}{\scriptsize{(1.000)}} \\  

    \mc{1}{l}{\scriptsize{Diabetes}} & \mc{1}{c}{\scriptsize{Mid-30s}} & \mc{1}{c}{\scriptsize{0.021}} & \mc{1}{c}{\scriptsize{0.025}} & \mc{1}{c}{\scriptsize{0.021}} & \mc{1}{c}{\scriptsize{0.002}} & \mc{1}{c}{\scriptsize{0.026}} & \mc{1}{c}{\scriptsize{0.021}} & \mc{1}{c}{\scriptsize{0.030}} & \mc{1}{c}{\scriptsize{0.026}} \\  

     &  & \mc{1}{c}{\scriptsize{(1.000)}} & \mc{1}{c}{\scriptsize{(1.000)}} & \mc{1}{c}{\scriptsize{(1.000)}} & \mc{1}{c}{\scriptsize{(0.947)}} & \mc{1}{c}{\scriptsize{(1.000)}} & \mc{1}{c}{\scriptsize{(1.000)}} & \mc{1}{c}{\scriptsize{(1.000)}} & \mc{1}{c}{\scriptsize{(1.000)}} \\  

  \bottomrule
  \end{tabular}
	\end{table} 

	\begin{table}[H]
     \caption{Treatment Effects on Diabetes, Pooled Sample}
     \label{table:abccare_rslt_pooled_cat31_sd}
	\input{AppResOutput/abccare/rslt_pooled_cat31_sd}
	\end{table} 

	\begin{table}[H]
     \caption{Treatment Effects on Drug Behavior and ASR Substance Scale, Pooled Sample}
     \label{table:abccare_rslt_pooled_cat32_sd}
	\input{AppResOutput/abccare/rslt_pooled_cat32_sd}
	\end{table} 

	\begin{table}[H]
     \caption{Treatment Effects on Health Insurance, Pooled Sample}
     \label{table:abccare_rslt_pooled_cat33_sd}
	  \begin{tabular}{cccccccccc}
  \toprule

    \scriptsize{Variable} & \scriptsize{Age} & \scriptsize{(1)} & \scriptsize{(2)} & \scriptsize{(3)} & \scriptsize{(4)} & \scriptsize{(5)} & \scriptsize{(6)} & \scriptsize{(7)} & \scriptsize{(8)} \\ 
    \midrule  

    \mc{1}{l}{\scriptsize{Has Health Insurance}} & \mc{1}{c}{\scriptsize{30}} & \mc{1}{c}{\scriptsize{0.074}} & \mc{1}{c}{\scriptsize{0.079}} & \mc{1}{c}{\scriptsize{0.077}} & \mc{1}{c}{\scriptsize{0.041}} & \mc{1}{c}{\scriptsize{0.087}} & \mc{1}{c}{\scriptsize{0.077}} & \mc{1}{c}{\scriptsize{0.057}} & \mc{1}{c}{\scriptsize{0.081}} \\  

     &  & \mc{1}{c}{\scriptsize{(0.276)}} & \mc{1}{c}{\scriptsize{(0.303)}} & \mc{1}{c}{\scriptsize{(0.461)}} & \mc{1}{c}{\scriptsize{(0.618)}} & \mc{1}{c}{\scriptsize{(0.474)}} & \mc{1}{c}{\scriptsize{(0.303)}} & \mc{1}{c}{\scriptsize{(0.421)}} & \mc{1}{c}{\scriptsize{(0.355)}} \\  

     & \mc{1}{c}{\scriptsize{21}} & \mc{1}{c}{\scriptsize{0.090}} & \mc{1}{c}{\scriptsize{0.070}} & \mc{1}{c}{\scriptsize{0.166}} & \mc{1}{c}{\scriptsize{0.181}} & \mc{1}{c}{\scriptsize{0.167}} & \mc{1}{c}{\scriptsize{0.022}} & \mc{1}{c}{\scriptsize{0.026}} & \mc{1}{c}{\scriptsize{0.051}} \\  

     &  & \mc{1}{c}{\scriptsize{(0.224)}} & \mc{1}{c}{\scriptsize{(0.355)}} & \mc{1}{c}{\scriptsize{(0.145)}} & \mc{1}{c}{\scriptsize{(0.211)}} & \mc{1}{c}{\scriptsize{(0.171)}} & \mc{1}{c}{\scriptsize{(0.579)}} & \mc{1}{c}{\scriptsize{(0.539)}} & \mc{1}{c}{\scriptsize{(0.461)}} \\  

  \bottomrule
  \end{tabular}
	\end{table} 

	\begin{table}[H]
     \caption{Treatment Effects on Hypertension, Pooled Sample}
     \label{table:abccare_rslt_pooled_cat34_sd}
	\input{AppResOutput/abccare/rslt_pooled_cat34_sd}
	\end{table} 

	\begin{table}[H]
     \caption{Treatment Effects on Laboratory Test  - Metabolic Panel, Pooled Sample}
     \label{table:abccare_rslt_pooled_cat35_sd}
	\input{AppResOutput/abccare/rslt_pooled_cat35_sd}
	\end{table} 

	\begin{table}[H]
     \caption{Treatment Effects on Laboratory Test - Complete Blood Count, Pooled Sample}
     \label{table:abccare_rslt_pooled_cat36_sd}
	\input{AppResOutput/abccare/rslt_pooled_cat36_sd}
	\end{table} 

	\begin{table}[H]
     \caption{Treatment Effects on Other Health-Related Information, Pooled Sample}
     \label{table:abccare_rslt_pooled_cat37_sd}
	\input{AppResOutput/abccare/rslt_pooled_cat37_sd}
	\end{table} 

	\begin{table}[H]
     \caption{Treatment Effects on Past Medical History - Diagnosis (Self-Reported), Pooled Sample}
     \label{table:abccare_rslt_pooled_cat38_sd}
	\input{AppResOutput/abccare/rslt_pooled_cat38_sd}
	\end{table} 

	\begin{table}[H]
     \caption{Treatment Effects on Past Medical History - Surgery (Self-Reported), Pooled Sample}
     \label{table:abccare_rslt_pooled_cat39_sd}
	  \begin{tabular}{cccccccccc}
  \toprule

    \scriptsize{Variable} & \scriptsize{Age} & \scriptsize{(1)} & \scriptsize{(2)} & \scriptsize{(3)} & \scriptsize{(4)} & \scriptsize{(5)} & \scriptsize{(6)} & \scriptsize{(7)} & \scriptsize{(8)} \\ 
    \midrule  

    \mc{1}{l}{\scriptsize{Depression $t$-Score}} & \mc{1}{c}{\scriptsize{Mid-30s}} & \mc{1}{c}{\scriptsize{-1.904}} & \mc{1}{c}{\scriptsize{-2.754}} & \mc{1}{c}{\scriptsize{1.064}} & \mc{1}{c}{\scriptsize{2.099}} & \mc{1}{c}{\scriptsize{0.478}} & \mc{1}{c}{\scriptsize{-2.974}} & \mc{1}{c}{\scriptsize{-4.746}} & \mc{1}{c}{\scriptsize{-3.167}} \\  

     &  & \mc{1}{c}{\scriptsize{(0.553)}} & \mc{1}{c}{\scriptsize{(0.355)}} & \mc{1}{c}{\scriptsize{(0.987)}} & \mc{1}{c}{\scriptsize{(0.987)}} & \mc{1}{c}{\scriptsize{(0.987)}} & \mc{1}{c}{\scriptsize{(0.382)}} & \mc{1}{c}{\scriptsize{(0.158)}} & \mc{1}{c}{\scriptsize{(0.421)}} \\  

     & \mc{1}{c}{\scriptsize{21}} & \mc{1}{c}{\scriptsize{-4.213}} & \mc{1}{c}{\scriptsize{-3.221}} & \mc{1}{c}{\scriptsize{-4.297}} & \mc{1}{c}{\scriptsize{-4.116}} & \mc{1}{c}{\scriptsize{-4.332}} & \mc{1}{c}{\scriptsize{-4.058}} & \mc{1}{c}{\scriptsize{-3.344}} & \mc{1}{c}{\scriptsize{-3.671}} \\  

     &  & \mc{1}{c}{\scriptsize{\textbf{(0.026)}}} & \mc{1}{c}{\scriptsize{(0.224)}} & \mc{1}{c}{\scriptsize{(0.303)}} & \mc{1}{c}{\scriptsize{(0.303)}} & \mc{1}{c}{\scriptsize{(0.250)}} & \mc{1}{c}{\scriptsize{\textbf{(0.079)}}} & \mc{1}{c}{\scriptsize{(0.184)}} & \mc{1}{c}{\scriptsize{\textbf{(0.092)}}} \\  

    \mc{1}{l}{\scriptsize{Somatization $t$-Score}} & \mc{1}{c}{\scriptsize{21}} & \mc{1}{c}{\scriptsize{-2.709}} & \mc{1}{c}{\scriptsize{-3.216}} & \mc{1}{c}{\scriptsize{-4.304}} & \mc{1}{c}{\scriptsize{-4.641}} & \mc{1}{c}{\scriptsize{-4.612}} & \mc{1}{c}{\scriptsize{-2.258}} & \mc{1}{c}{\scriptsize{-3.056}} & \mc{1}{c}{\scriptsize{-3.015}} \\  

     &  & \mc{1}{c}{\scriptsize{(0.184)}} & \mc{1}{c}{\scriptsize{(0.211)}} & \mc{1}{c}{\scriptsize{(0.158)}} & \mc{1}{c}{\scriptsize{(0.237)}} & \mc{1}{c}{\scriptsize{(0.145)}} & \mc{1}{c}{\scriptsize{(0.303)}} & \mc{1}{c}{\scriptsize{(0.250)}} & \mc{1}{c}{\scriptsize{(0.145)}} \\  

    \mc{1}{l}{\scriptsize{Anxiety $t$-Score}} & \mc{1}{c}{\scriptsize{21}} & \mc{1}{c}{\scriptsize{-2.749}} & \mc{1}{c}{\scriptsize{-2.856}} & \mc{1}{c}{\scriptsize{-2.996}} & \mc{1}{c}{\scriptsize{-4.999}} & \mc{1}{c}{\scriptsize{-2.949}} & \mc{1}{c}{\scriptsize{-2.639}} & \mc{1}{c}{\scriptsize{-2.692}} & \mc{1}{c}{\scriptsize{-2.744}} \\  

     &  & \mc{1}{c}{\scriptsize{(0.408)}} & \mc{1}{c}{\scriptsize{(0.342)}} & \mc{1}{c}{\scriptsize{(0.513)}} & \mc{1}{c}{\scriptsize{(0.250)}} & \mc{1}{c}{\scriptsize{(0.500)}} & \mc{1}{c}{\scriptsize{(0.434)}} & \mc{1}{c}{\scriptsize{(0.382)}} & \mc{1}{c}{\scriptsize{(0.395)}} \\  

    \mc{1}{l}{\scriptsize{Somatization $t$-Score}} & \mc{1}{c}{\scriptsize{Mid-30s}} & \mc{1}{c}{\scriptsize{-1.057}} & \mc{1}{c}{\scriptsize{-0.865}} & \mc{1}{c}{\scriptsize{-2.144}} & \mc{1}{c}{\scriptsize{-2.561}} & \mc{1}{c}{\scriptsize{-2.071}} & \mc{1}{c}{\scriptsize{-0.950}} & \mc{1}{c}{\scriptsize{-0.435}} & \mc{1}{c}{\scriptsize{-0.685}} \\  

     &  & \mc{1}{c}{\scriptsize{(0.789)}} & \mc{1}{c}{\scriptsize{(0.789)}} & \mc{1}{c}{\scriptsize{(0.750)}} & \mc{1}{c}{\scriptsize{(0.750)}} & \mc{1}{c}{\scriptsize{(0.776)}} & \mc{1}{c}{\scriptsize{(0.855)}} & \mc{1}{c}{\scriptsize{(0.882)}} & \mc{1}{c}{\scriptsize{(0.895)}} \\  

    \mc{1}{l}{\scriptsize{Hostility $t$-Score}} & \mc{1}{c}{\scriptsize{Mid-30s}} & \mc{1}{c}{\scriptsize{-1.091}} & \mc{1}{c}{\scriptsize{-1.046}} & \mc{1}{c}{\scriptsize{-2.076}} & \mc{1}{c}{\scriptsize{-0.031}} & \mc{1}{c}{\scriptsize{-2.419}} & \mc{1}{c}{\scriptsize{-1.083}} & \mc{1}{c}{\scriptsize{-1.568}} & \mc{1}{c}{\scriptsize{-0.848}} \\  

     &  & \mc{1}{c}{\scriptsize{(0.789)}} & \mc{1}{c}{\scriptsize{(0.737)}} & \mc{1}{c}{\scriptsize{(0.737)}} & \mc{1}{c}{\scriptsize{(0.934)}} & \mc{1}{c}{\scriptsize{(0.697)}} & \mc{1}{c}{\scriptsize{(0.855)}} & \mc{1}{c}{\scriptsize{(0.645)}} & \mc{1}{c}{\scriptsize{(0.882)}} \\  

    \mc{1}{l}{\scriptsize{Global Severity Index $t$-Score}} & \mc{1}{c}{\scriptsize{21}} & \mc{1}{c}{\scriptsize{-3.146}} & \mc{1}{c}{\scriptsize{-2.479}} & \mc{1}{c}{\scriptsize{-4.917}} & \mc{1}{c}{\scriptsize{-4.553}} & \mc{1}{c}{\scriptsize{-4.941}} & \mc{1}{c}{\scriptsize{-2.564}} & \mc{1}{c}{\scriptsize{-1.866}} & \mc{1}{c}{\scriptsize{-2.493}} \\  

     &  & \mc{1}{c}{\scriptsize{(0.145)}} & \mc{1}{c}{\scriptsize{(0.263)}} & \mc{1}{c}{\scriptsize{(0.118)}} & \mc{1}{c}{\scriptsize{(0.237)}} & \mc{1}{c}{\scriptsize{(0.105)}} & \mc{1}{c}{\scriptsize{(0.303)}} & \mc{1}{c}{\scriptsize{(0.474)}} & \mc{1}{c}{\scriptsize{(0.289)}} \\  

    \mc{1}{l}{\scriptsize{Anxiety $t$-Score}} & \mc{1}{c}{\scriptsize{Mid-30s}} & \mc{1}{c}{\scriptsize{-3.399}} & \mc{1}{c}{\scriptsize{-4.481}} & \mc{1}{c}{\scriptsize{-1.502}} & \mc{1}{c}{\scriptsize{-1.849}} & \mc{1}{c}{\scriptsize{-2.085}} & \mc{1}{c}{\scriptsize{-4.155}} & \mc{1}{c}{\scriptsize{-5.489}} & \mc{1}{c}{\scriptsize{-4.721}} \\  

     &  & \mc{1}{c}{\scriptsize{(0.224)}} & \mc{1}{c}{\scriptsize{(0.132)}} & \mc{1}{c}{\scriptsize{(0.842)}} & \mc{1}{c}{\scriptsize{(0.803)}} & \mc{1}{c}{\scriptsize{(0.763)}} & \mc{1}{c}{\scriptsize{(0.105)}} & \mc{1}{c}{\scriptsize{(0.105)}} & \mc{1}{c}{\scriptsize{\textbf{(0.092)}}} \\  

    \mc{1}{l}{\scriptsize{Hostility $t$-Score}} & \mc{1}{c}{\scriptsize{21}} & \mc{1}{c}{\scriptsize{-3.256}} & \mc{1}{c}{\scriptsize{-2.189}} & \mc{1}{c}{\scriptsize{-4.552}} & \mc{1}{c}{\scriptsize{-4.241}} & \mc{1}{c}{\scriptsize{-4.622}} & \mc{1}{c}{\scriptsize{-2.894}} & \mc{1}{c}{\scriptsize{-1.954}} & \mc{1}{c}{\scriptsize{-2.553}} \\  

     &  & \mc{1}{c}{\scriptsize{\textbf{(0.079)}}} & \mc{1}{c}{\scriptsize{(0.342)}} & \mc{1}{c}{\scriptsize{(0.303)}} & \mc{1}{c}{\scriptsize{(0.342)}} & \mc{1}{c}{\scriptsize{(0.303)}} & \mc{1}{c}{\scriptsize{(0.184)}} & \mc{1}{c}{\scriptsize{(0.382)}} & \mc{1}{c}{\scriptsize{(0.329)}} \\  

  \bottomrule
  \end{tabular}
	\end{table} 

	\begin{table}[H]
     \caption{Treatment Effects on Physical Activity, Pooled Sample}
     \label{table:abccare_rslt_pooled_cat40_sd}
	  \begin{tabular}{cccccccccc}
  \toprule

    \scriptsize{Variable} & \scriptsize{Age} & \scriptsize{(1)} & \scriptsize{(2)} & \scriptsize{(3)} & \scriptsize{(4)} & \scriptsize{(5)} & \scriptsize{(6)} & \scriptsize{(7)} & \scriptsize{(8)} \\ 
    \midrule  

    \mc{1}{l}{\scriptsize{Level of Activity at Work}} & \mc{1}{c}{\scriptsize{Mid-30s}} & \mc{1}{c}{\scriptsize{0.205}} & \mc{1}{c}{\scriptsize{0.086}} & \mc{1}{c}{\scriptsize{-0.517}} & \mc{1}{c}{\scriptsize{-0.817}} & \mc{1}{c}{\scriptsize{-0.579}} & \mc{1}{c}{\scriptsize{0.358}} & \mc{1}{c}{\scriptsize{0.252}} & \mc{1}{c}{\scriptsize{0.249}} \\  

     &  & \mc{1}{c}{\scriptsize{(0.132)}} & \mc{1}{c}{\scriptsize{(0.382)}} & \mc{1}{c}{\scriptsize{(1.000)}} & \mc{1}{c}{\scriptsize{(0.959)}} & \mc{1}{c}{\scriptsize{(1.000)}} & \mc{1}{c}{\scriptsize{\textbf{(0.013)}}} & \mc{1}{c}{\scriptsize{\textbf{(0.092)}}} & \mc{1}{c}{\scriptsize{\textbf{(0.066)}}} \\  

  \bottomrule
  \end{tabular}
	\end{table} 

	\begin{table}[H]
     \caption{Treatment Effects on Physical Exam - Ear, Pooled Sample}
     \label{table:abccare_rslt_pooled_cat41_sd}
	  \begin{tabular}{cccccccccc}
  \toprule

    \scriptsize{Variable} & \scriptsize{Age} & \scriptsize{(1)} & \scriptsize{(2)} & \scriptsize{(3)} & \scriptsize{(4)} & \scriptsize{(5)} & \scriptsize{(6)} & \scriptsize{(7)} & \scriptsize{(8)} \\ 
    \midrule  

    \mc{1}{l}{\scriptsize{Mother's Earned Inc.}} & \mc{1}{c}{\scriptsize{0}} & \mc{1}{c}{\scriptsize{2,075}} & \mc{1}{c}{\scriptsize{1,534}} & \mc{1}{c}{\scriptsize{5,160}} & \mc{1}{c}{\scriptsize{8,802}} & \mc{1}{c}{\scriptsize{4,151}} & \mc{1}{c}{\scriptsize{447}} & \mc{1}{c}{\scriptsize{128}} & \mc{1}{c}{\scriptsize{-740}} \\  

     &  & \mc{1}{c}{\scriptsize{(0.474)}} & \mc{1}{c}{\scriptsize{(0.618)}} & \mc{1}{c}{\scriptsize{(0.145)}} & \mc{1}{c}{\scriptsize{(0.658)}} & \mc{1}{c}{\scriptsize{(0.224)}} & \mc{1}{c}{\scriptsize{(0.724)}} & \mc{1}{c}{\scriptsize{(0.724)}} & \mc{1}{c}{\scriptsize{(0.789)}} \\  

    \mc{1}{l}{\scriptsize{Mother's Public-Transfer Inc.}} & \mc{1}{c}{\scriptsize{21}} & \mc{1}{c}{\scriptsize{-4,286}} & \mc{1}{c}{\scriptsize{-5,308}} & \mc{1}{c}{\scriptsize{-5,367}} & \mc{1}{c}{\scriptsize{-5,223}} & \mc{1}{c}{\scriptsize{-8,332}} & \mc{1}{c}{\scriptsize{-4,443}} & \mc{1}{c}{\scriptsize{-4,922}} & \mc{1}{c}{\scriptsize{-5,429}} \\  

     &  & \mc{1}{c}{\scriptsize{(1.000)}} & \mc{1}{c}{\scriptsize{(1.000)}} & \mc{1}{c}{\scriptsize{(0.974)}} & \mc{1}{c}{\scriptsize{(0.974)}} & \mc{1}{c}{\scriptsize{(1.000)}} & \mc{1}{c}{\scriptsize{(1.000)}} & \mc{1}{c}{\scriptsize{(1.000)}} & \mc{1}{c}{\scriptsize{(1.000)}} \\  

  \bottomrule
  \end{tabular}
	\end{table} 

	\begin{table}[H]
     \caption{Treatment Effects on Physical Exam - General I, Pooled Sample}
     \label{table:abccare_rslt_pooled_cat42_sd}
	  \begin{tabular}{cccccccccc}
  \toprule

    \scriptsize{Variable} & \scriptsize{Age} & \scriptsize{(1)} & \scriptsize{(2)} & \scriptsize{(3)} & \scriptsize{(4)} & \scriptsize{(5)} & \scriptsize{(6)} & \scriptsize{(7)} & \scriptsize{(8)} \\ 
    \midrule  

    \mc{1}{l}{\scriptsize{Respirations}} & \mc{1}{c}{\scriptsize{Mid-30s}} & \mc{1}{c}{\scriptsize{-0.411}} & \mc{1}{c}{\scriptsize{-0.417}} & \mc{1}{c}{\scriptsize{-0.853}} & \mc{1}{c}{\scriptsize{-0.492}} & \mc{1}{c}{\scriptsize{-0.954}} & \mc{1}{c}{\scriptsize{-0.286}} & \mc{1}{c}{\scriptsize{-0.392}} & \mc{1}{c}{\scriptsize{-0.197}} \\  

     &  & \mc{1}{c}{\scriptsize{(0.553)}} & \mc{1}{c}{\scriptsize{(0.539)}} & \mc{1}{c}{\scriptsize{\textbf{(0.026)}}} & \mc{1}{c}{\scriptsize{(0.579)}} & \mc{1}{c}{\scriptsize{\textbf{(0.013)}}} & \mc{1}{c}{\scriptsize{(0.829)}} & \mc{1}{c}{\scriptsize{(0.658)}} & \mc{1}{c}{\scriptsize{(0.829)}} \\  

    \mc{1}{l}{\scriptsize{Temp (F)}} & \mc{1}{c}{\scriptsize{Mid-30s}} & \mc{1}{c}{\scriptsize{0.089}} & \mc{1}{c}{\scriptsize{0.042}} & \mc{1}{c}{\scriptsize{-0.084}} & \mc{1}{c}{\scriptsize{-0.071}} & \mc{1}{c}{\scriptsize{-0.080}} & \mc{1}{c}{\scriptsize{0.144}} & \mc{1}{c}{\scriptsize{0.107}} & \mc{1}{c}{\scriptsize{0.115}} \\  

     &  & \mc{1}{c}{\scriptsize{(1.000)}} & \mc{1}{c}{\scriptsize{(0.987)}} & \mc{1}{c}{\scriptsize{(0.842)}} & \mc{1}{c}{\scriptsize{(0.816)}} & \mc{1}{c}{\scriptsize{(0.816)}} & \mc{1}{c}{\scriptsize{(1.000)}} & \mc{1}{c}{\scriptsize{(1.000)}} & \mc{1}{c}{\scriptsize{(1.000)}} \\  

    \mc{1}{l}{\scriptsize{Pulse}} & \mc{1}{c}{\scriptsize{Mid-30s}} & \mc{1}{c}{\scriptsize{-1.097}} & \mc{1}{c}{\scriptsize{-1.017}} & \mc{1}{c}{\scriptsize{-9.009}} & \mc{1}{c}{\scriptsize{-7.600}} & \mc{1}{c}{\scriptsize{-8.686}} & \mc{1}{c}{\scriptsize{1.119}} & \mc{1}{c}{\scriptsize{0.467}} & \mc{1}{c}{\scriptsize{1.072}} \\  

     &  & \mc{1}{c}{\scriptsize{(0.829)}} & \mc{1}{c}{\scriptsize{(0.868)}} & \mc{1}{c}{\scriptsize{(0.132)}} & \mc{1}{c}{\scriptsize{(0.329)}} & \mc{1}{c}{\scriptsize{(0.145)}} & \mc{1}{c}{\scriptsize{(0.987)}} & \mc{1}{c}{\scriptsize{(0.974)}} & \mc{1}{c}{\scriptsize{(1.000)}} \\  

    \mc{1}{l}{\scriptsize{Nutrition}} & \mc{1}{c}{\scriptsize{Mid-30s}} & \mc{1}{c}{\scriptsize{-0.075}} & \mc{1}{c}{\scriptsize{0.041}} & \mc{1}{c}{\scriptsize{-0.099}} & \mc{1}{c}{\scriptsize{-0.059}} & \mc{1}{c}{\scriptsize{-0.065}} & \mc{1}{c}{\scriptsize{-0.079}} & \mc{1}{c}{\scriptsize{0.052}} & \mc{1}{c}{\scriptsize{0.004}} \\  

     &  & \mc{1}{c}{\scriptsize{(0.711)}} & \mc{1}{c}{\scriptsize{(0.987)}} & \mc{1}{c}{\scriptsize{(0.842)}} & \mc{1}{c}{\scriptsize{(0.842)}} & \mc{1}{c}{\scriptsize{(0.882)}} & \mc{1}{c}{\scriptsize{(0.763)}} & \mc{1}{c}{\scriptsize{(0.987)}} & \mc{1}{c}{\scriptsize{(0.947)}} \\  

    \mc{1}{l}{\scriptsize{Posture}} & \mc{1}{c}{\scriptsize{Mid-30s}} & \mc{1}{c}{\scriptsize{0.043}} & \mc{1}{c}{\scriptsize{0.067}} & \mc{1}{c}{\scriptsize{0.043}} & \mc{1}{c}{\scriptsize{0.003}} & \mc{1}{c}{\scriptsize{0.050}} & \mc{1}{c}{\scriptsize{0.043}} & \mc{1}{c}{\scriptsize{0.087}} & \mc{1}{c}{\scriptsize{0.050}} \\  

     &  & \mc{1}{c}{\scriptsize{(1.000)}} & \mc{1}{c}{\scriptsize{(1.000)}} & \mc{1}{c}{\scriptsize{(0.987)}} & \mc{1}{c}{\scriptsize{(0.974)}} & \mc{1}{c}{\scriptsize{(1.000)}} & \mc{1}{c}{\scriptsize{(1.000)}} & \mc{1}{c}{\scriptsize{(1.000)}} & \mc{1}{c}{\scriptsize{(1.000)}} \\  

  \bottomrule
  \end{tabular}
	\end{table} 

	\begin{table}[H]
     \caption{Treatment Effects on Physical Exam - General II, Pooled Sample}
     \label{table:abccare_rslt_pooled_cat43_sd}
	  \begin{tabular}{cccccccccc}
  \toprule

    \scriptsize{Variable} & \scriptsize{Age} & \scriptsize{(1)} & \scriptsize{(2)} & \scriptsize{(3)} & \scriptsize{(4)} & \scriptsize{(5)} & \scriptsize{(6)} & \scriptsize{(7)} & \scriptsize{(8)} \\ 
    \midrule  

    \mc{1}{l}{\scriptsize{Chest and Lung General}} & \mc{1}{c}{\scriptsize{Mid-30s}} & \mc{1}{c}{\scriptsize{0.043}} & \mc{1}{c}{\scriptsize{0.058}} & \mc{1}{c}{\scriptsize{0.043}} & \mc{1}{c}{\scriptsize{0.014}} & \mc{1}{c}{\scriptsize{0.051}} & \mc{1}{c}{\scriptsize{0.043}} & \mc{1}{c}{\scriptsize{0.064}} & \mc{1}{c}{\scriptsize{0.052}} \\  

     &  & \mc{1}{c}{\scriptsize{(0.987)}} & \mc{1}{c}{\scriptsize{(1.000)}} & \mc{1}{c}{\scriptsize{(0.961)}} & \mc{1}{c}{\scriptsize{(0.895)}} & \mc{1}{c}{\scriptsize{(0.934)}} & \mc{1}{c}{\scriptsize{(1.000)}} & \mc{1}{c}{\scriptsize{(0.974)}} & \mc{1}{c}{\scriptsize{(0.974)}} \\  

    \mc{1}{l}{\scriptsize{Cardiovascular General}} & \mc{1}{c}{\scriptsize{Mid-30s}} & \mc{1}{c}{\scriptsize{-0.024}} &  &  &  &  & \mc{1}{c}{\scriptsize{-0.031}} &  &  \\  

     &  & \mc{1}{c}{\scriptsize{(0.263)}} &  &  &  &  & \mc{1}{c}{\scriptsize{(0.211)}} &  &  \\  

    \mc{1}{l}{\scriptsize{Skin General}} & \mc{1}{c}{\scriptsize{Mid-30s}} & \mc{1}{c}{\scriptsize{-0.058}} & \mc{1}{c}{\scriptsize{-0.054}} & \mc{1}{c}{\scriptsize{-0.137}} & \mc{1}{c}{\scriptsize{-0.102}} & \mc{1}{c}{\scriptsize{-0.138}} & \mc{1}{c}{\scriptsize{-0.040}} & \mc{1}{c}{\scriptsize{-0.067}} & \mc{1}{c}{\scriptsize{-0.052}} \\  

     &  & \mc{1}{c}{\scriptsize{(0.355)}} & \mc{1}{c}{\scriptsize{(0.474)}} & \mc{1}{c}{\scriptsize{(0.368)}} & \mc{1}{c}{\scriptsize{(0.434)}} & \mc{1}{c}{\scriptsize{(0.342)}} & \mc{1}{c}{\scriptsize{(0.579)}} & \mc{1}{c}{\scriptsize{(0.303)}} & \mc{1}{c}{\scriptsize{(0.408)}} \\  

    \mc{1}{l}{\scriptsize{Musculoskeletal General}} & \mc{1}{c}{\scriptsize{Mid-30s}} & \mc{1}{c}{\scriptsize{0.021}} & \mc{1}{c}{\scriptsize{0.025}} & \mc{1}{c}{\scriptsize{0.021}} & \mc{1}{c}{\scriptsize{0.042}} & \mc{1}{c}{\scriptsize{0.018}} & \mc{1}{c}{\scriptsize{0.021}} & \mc{1}{c}{\scriptsize{0.025}} & \mc{1}{c}{\scriptsize{0.018}} \\  

     &  & \mc{1}{c}{\scriptsize{(1.000)}} & \mc{1}{c}{\scriptsize{(1.000)}} & \mc{1}{c}{\scriptsize{(0.961)}} & \mc{1}{c}{\scriptsize{(0.961)}} & \mc{1}{c}{\scriptsize{(0.961)}} & \mc{1}{c}{\scriptsize{(1.000)}} & \mc{1}{c}{\scriptsize{(0.961)}} & \mc{1}{c}{\scriptsize{(1.000)}} \\  

    \mc{1}{l}{\scriptsize{Head General}} & \mc{1}{c}{\scriptsize{Mid-30s}} & \mc{1}{c}{\scriptsize{-0.024}} & \mc{1}{c}{\scriptsize{-0.014}} & \mc{1}{c}{\scriptsize{-0.111}} & \mc{1}{c}{\scriptsize{-0.111}} & \mc{1}{c}{\scriptsize{-0.106}} &  &  &  \\  

     &  & \mc{1}{c}{\scriptsize{(0.224)}} & \mc{1}{c}{\scriptsize{(0.395)}} & \mc{1}{c}{\scriptsize{(0.158)}} & \mc{1}{c}{\scriptsize{(0.224)}} & \mc{1}{c}{\scriptsize{(0.158)}} &  &  &  \\  

  \bottomrule
  \end{tabular}
	\end{table} 

	\begin{table}[H]
     \caption{Treatment Effects on Physical Exam (Part II), Pooled Sample}
     \label{table:abccare_rslt_pooled_cat44_sd}
	\input{AppResOutput/abccare/rslt_pooled_cat44_sd}
	\end{table} 

	\begin{table}[H]
     \caption{Treatment Effects on Age 21 Brief Symptom Inventory, Pooled Sample}
     \label{table:abccare_rslt_pooled_cat45_sd}
	\input{AppResOutput/abccare/rslt_pooled_cat45_sd}
	\end{table} 

	\begin{table}[H]
     \caption{Treatment Effects on Age 30 Adult Self Report DSM Scale $t$-Score, Pooled Sample}
     \label{table:abccare_rslt_pooled_cat46_sd}
	  \begin{tabular}{cccccccccc}
  \toprule

    \scriptsize{Variable} & \scriptsize{Age} & \scriptsize{(1)} & \scriptsize{(2)} & \scriptsize{(3)} & \scriptsize{(4)} & \scriptsize{(5)} & \scriptsize{(6)} & \scriptsize{(7)} & \scriptsize{(8)} \\ 
    \midrule  

    \mc{1}{l}{\scriptsize{Somatic Problems}} & \mc{1}{c}{\scriptsize{30}} & \mc{1}{c}{\scriptsize{1.139}} & \mc{1}{c}{\scriptsize{0.323}} & \mc{1}{c}{\scriptsize{0.882}} & \mc{1}{c}{\scriptsize{-0.518}} & \mc{1}{c}{\scriptsize{0.363}} & \mc{1}{c}{\scriptsize{1.055}} & \mc{1}{c}{\scriptsize{0.344}} & \mc{1}{c}{\scriptsize{0.537}} \\  

     &  & \mc{1}{c}{\scriptsize{(0.987)}} & \mc{1}{c}{\scriptsize{(0.974)}} & \mc{1}{c}{\scriptsize{(0.961)}} & \mc{1}{c}{\scriptsize{(0.737)}} & \mc{1}{c}{\scriptsize{(0.908)}} & \mc{1}{c}{\scriptsize{(0.987)}} & \mc{1}{c}{\scriptsize{(0.974)}} & \mc{1}{c}{\scriptsize{(0.987)}} \\  

    \mc{1}{l}{\scriptsize{AD/H Problems}} & \mc{1}{c}{\scriptsize{30}} & \mc{1}{c}{\scriptsize{-0.170}} & \mc{1}{c}{\scriptsize{-0.953}} & \mc{1}{c}{\scriptsize{0.339}} & \mc{1}{c}{\scriptsize{-0.755}} & \mc{1}{c}{\scriptsize{0.011}} & \mc{1}{c}{\scriptsize{-0.421}} & \mc{1}{c}{\scriptsize{-1.068}} & \mc{1}{c}{\scriptsize{-0.873}} \\  

     &  & \mc{1}{c}{\scriptsize{(0.868)}} & \mc{1}{c}{\scriptsize{(0.500)}} & \mc{1}{c}{\scriptsize{(0.882)}} & \mc{1}{c}{\scriptsize{(0.645)}} & \mc{1}{c}{\scriptsize{(0.803)}} & \mc{1}{c}{\scriptsize{(0.803)}} & \mc{1}{c}{\scriptsize{(0.395)}} & \mc{1}{c}{\scriptsize{(0.487)}} \\  

    \mc{1}{l}{\scriptsize{Inattention Subscale}} & \mc{1}{c}{\scriptsize{30}} &  &  &  &  &  &  &  &  \\  

     &  &  &  &  &  &  &  &  &  \\  

    \mc{1}{l}{\scriptsize{Depressive Problems}} & \mc{1}{c}{\scriptsize{30}} & \mc{1}{c}{\scriptsize{1.015}} & \mc{1}{c}{\scriptsize{0.747}} & \mc{1}{c}{\scriptsize{1.210}} & \mc{1}{c}{\scriptsize{0.661}} & \mc{1}{c}{\scriptsize{1.125}} & \mc{1}{c}{\scriptsize{0.917}} & \mc{1}{c}{\scriptsize{0.679}} & \mc{1}{c}{\scriptsize{0.783}} \\  

     &  & \mc{1}{c}{\scriptsize{(0.987)}} & \mc{1}{c}{\scriptsize{(1.000)}} & \mc{1}{c}{\scriptsize{(0.987)}} & \mc{1}{c}{\scriptsize{(0.974)}} & \mc{1}{c}{\scriptsize{(1.000)}} & \mc{1}{c}{\scriptsize{(0.987)}} & \mc{1}{c}{\scriptsize{(1.000)}} & \mc{1}{c}{\scriptsize{(1.000)}} \\  

    \mc{1}{l}{\scriptsize{Avoidant Personality Problems}} & \mc{1}{c}{\scriptsize{30}} & \mc{1}{c}{\scriptsize{0.536}} & \mc{1}{c}{\scriptsize{0.286}} & \mc{1}{c}{\scriptsize{1.292}} & \mc{1}{c}{\scriptsize{0.505}} & \mc{1}{c}{\scriptsize{0.949}} & \mc{1}{c}{\scriptsize{0.192}} & \mc{1}{c}{\scriptsize{-0.118}} & \mc{1}{c}{\scriptsize{-0.021}} \\  

     &  & \mc{1}{c}{\scriptsize{(0.974)}} & \mc{1}{c}{\scriptsize{(0.987)}} & \mc{1}{c}{\scriptsize{(0.987)}} & \mc{1}{c}{\scriptsize{(0.934)}} & \mc{1}{c}{\scriptsize{(0.974)}} & \mc{1}{c}{\scriptsize{(0.974)}} & \mc{1}{c}{\scriptsize{(0.934)}} & \mc{1}{c}{\scriptsize{(0.921)}} \\  

    \mc{1}{l}{\scriptsize{Anxiety Problems}} & \mc{1}{c}{\scriptsize{30}} & \mc{1}{c}{\scriptsize{0.596}} & \mc{1}{c}{\scriptsize{0.059}} & \mc{1}{c}{\scriptsize{0.636}} & \mc{1}{c}{\scriptsize{-0.594}} & \mc{1}{c}{\scriptsize{0.520}} & \mc{1}{c}{\scriptsize{0.469}} & \mc{1}{c}{\scriptsize{0.096}} & \mc{1}{c}{\scriptsize{0.281}} \\  

     &  & \mc{1}{c}{\scriptsize{(0.974)}} & \mc{1}{c}{\scriptsize{(0.934)}} & \mc{1}{c}{\scriptsize{(0.908)}} & \mc{1}{c}{\scriptsize{(0.671)}} & \mc{1}{c}{\scriptsize{(0.947)}} & \mc{1}{c}{\scriptsize{(0.987)}} & \mc{1}{c}{\scriptsize{(0.974)}} & \mc{1}{c}{\scriptsize{(0.974)}} \\  

    \mc{1}{l}{\scriptsize{Antisocial Personality Problems}} & \mc{1}{c}{\scriptsize{30}} & \mc{1}{c}{\scriptsize{-0.564}} & \mc{1}{c}{\scriptsize{-0.554}} & \mc{1}{c}{\scriptsize{-0.851}} & \mc{1}{c}{\scriptsize{-1.819}} & \mc{1}{c}{\scriptsize{-0.846}} & \mc{1}{c}{\scriptsize{-0.465}} & \mc{1}{c}{\scriptsize{-0.561}} & \mc{1}{c}{\scriptsize{-0.396}} \\  

     &  & \mc{1}{c}{\scriptsize{(0.711)}} & \mc{1}{c}{\scriptsize{(0.697)}} & \mc{1}{c}{\scriptsize{(0.658)}} & \mc{1}{c}{\scriptsize{(0.329)}} & \mc{1}{c}{\scriptsize{(0.697)}} & \mc{1}{c}{\scriptsize{(0.829)}} & \mc{1}{c}{\scriptsize{(0.750)}} & \mc{1}{c}{\scriptsize{(0.842)}} \\  

    \mc{1}{l}{\scriptsize{Hyperactivity-Impulsivity Subscale}} & \mc{1}{c}{\scriptsize{30}} &  &  &  &  &  &  &  &  \\  

     &  &  &  &  &  &  &  &  &  \\  

  \bottomrule
  \end{tabular}
	\end{table} 

	\begin{table}[H]
     \caption{Treatment Effects on Age 30 Adult Self Report Syndrome Scale $t$-Score, Pooled Sample}
     \label{table:abccare_rslt_pooled_cat47_sd}
	  \begin{tabular}{cccccccccc}
  \toprule

    \scriptsize{Variable} & \scriptsize{Age} & \scriptsize{(1)} & \scriptsize{(2)} & \scriptsize{(3)} & \scriptsize{(4)} & \scriptsize{(5)} & \scriptsize{(6)} & \scriptsize{(7)} & \scriptsize{(8)} \\ 
    \midrule  

    \mc{1}{l}{\scriptsize{Somatic Complaints}} & \mc{1}{c}{\scriptsize{30}} & \mc{1}{c}{\scriptsize{1.134}} & \mc{1}{c}{\scriptsize{0.184}} & \mc{1}{c}{\scriptsize{1.236}} & \mc{1}{c}{\scriptsize{-0.224}} & \mc{1}{c}{\scriptsize{0.643}} & \mc{1}{c}{\scriptsize{0.969}} & \mc{1}{c}{\scriptsize{0.106}} & \mc{1}{c}{\scriptsize{0.308}} \\  

     &  & \mc{1}{c}{\scriptsize{(1.000)}} & \mc{1}{c}{\scriptsize{(1.000)}} & \mc{1}{c}{\scriptsize{(1.000)}} & \mc{1}{c}{\scriptsize{(0.882)}} & \mc{1}{c}{\scriptsize{(0.987)}} & \mc{1}{c}{\scriptsize{(1.000)}} & \mc{1}{c}{\scriptsize{(0.987)}} & \mc{1}{c}{\scriptsize{(1.000)}} \\  

    \mc{1}{l}{\scriptsize{Aggressive Behavior}} & \mc{1}{c}{\scriptsize{30}} & \mc{1}{c}{\scriptsize{0.359}} & \mc{1}{c}{\scriptsize{0.670}} & \mc{1}{c}{\scriptsize{0.667}} & \mc{1}{c}{\scriptsize{-0.109}} & \mc{1}{c}{\scriptsize{0.570}} & \mc{1}{c}{\scriptsize{0.400}} & \mc{1}{c}{\scriptsize{0.475}} & \mc{1}{c}{\scriptsize{0.398}} \\  

     &  & \mc{1}{c}{\scriptsize{(1.000)}} & \mc{1}{c}{\scriptsize{(1.000)}} & \mc{1}{c}{\scriptsize{(0.974)}} & \mc{1}{c}{\scriptsize{(0.895)}} & \mc{1}{c}{\scriptsize{(0.961)}} & \mc{1}{c}{\scriptsize{(1.000)}} & \mc{1}{c}{\scriptsize{(1.000)}} & \mc{1}{c}{\scriptsize{(1.000)}} \\  

    \mc{1}{l}{\scriptsize{Intrusive}} & \mc{1}{c}{\scriptsize{30}} & \mc{1}{c}{\scriptsize{0.245}} & \mc{1}{c}{\scriptsize{-0.102}} & \mc{1}{c}{\scriptsize{-0.441}} & \mc{1}{c}{\scriptsize{-1.150}} & \mc{1}{c}{\scriptsize{-0.434}} & \mc{1}{c}{\scriptsize{0.492}} & \mc{1}{c}{\scriptsize{-0.086}} & \mc{1}{c}{\scriptsize{0.447}} \\  

     &  & \mc{1}{c}{\scriptsize{(0.987)}} & \mc{1}{c}{\scriptsize{(0.947)}} & \mc{1}{c}{\scriptsize{(0.842)}} & \mc{1}{c}{\scriptsize{(0.605)}} & \mc{1}{c}{\scriptsize{(0.855)}} & \mc{1}{c}{\scriptsize{(1.000)}} & \mc{1}{c}{\scriptsize{(0.974)}} & \mc{1}{c}{\scriptsize{(1.000)}} \\  

    \mc{1}{l}{\scriptsize{Anxious/Depressed}} & \mc{1}{c}{\scriptsize{30}} & \mc{1}{c}{\scriptsize{0.769}} & \mc{1}{c}{\scriptsize{0.255}} & \mc{1}{c}{\scriptsize{0.969}} & \mc{1}{c}{\scriptsize{-0.244}} & \mc{1}{c}{\scriptsize{0.602}} & \mc{1}{c}{\scriptsize{0.649}} & \mc{1}{c}{\scriptsize{0.255}} & \mc{1}{c}{\scriptsize{0.404}} \\  

     &  & \mc{1}{c}{\scriptsize{(1.000)}} & \mc{1}{c}{\scriptsize{(1.000)}} & \mc{1}{c}{\scriptsize{(1.000)}} & \mc{1}{c}{\scriptsize{(0.868)}} & \mc{1}{c}{\scriptsize{(0.987)}} & \mc{1}{c}{\scriptsize{(1.000)}} & \mc{1}{c}{\scriptsize{(0.987)}} & \mc{1}{c}{\scriptsize{(1.000)}} \\  

    \mc{1}{l}{\scriptsize{Externalizing}} & \mc{1}{c}{\scriptsize{30}} & \mc{1}{c}{\scriptsize{0.045}} & \mc{1}{c}{\scriptsize{0.024}} & \mc{1}{c}{\scriptsize{0.554}} & \mc{1}{c}{\scriptsize{-1.309}} & \mc{1}{c}{\scriptsize{0.440}} & \mc{1}{c}{\scriptsize{0.134}} & \mc{1}{c}{\scriptsize{-0.153}} & \mc{1}{c}{\scriptsize{0.159}} \\  

     &  & \mc{1}{c}{\scriptsize{(0.974)}} & \mc{1}{c}{\scriptsize{(0.987)}} & \mc{1}{c}{\scriptsize{(0.934)}} & \mc{1}{c}{\scriptsize{(0.737)}} & \mc{1}{c}{\scriptsize{(0.934)}} & \mc{1}{c}{\scriptsize{(0.987)}} & \mc{1}{c}{\scriptsize{(0.974)}} & \mc{1}{c}{\scriptsize{(0.987)}} \\  

    \mc{1}{l}{\scriptsize{Thought Problems}} & \mc{1}{c}{\scriptsize{30}} & \mc{1}{c}{\scriptsize{-1.203}} & \mc{1}{c}{\scriptsize{-1.774}} & \mc{1}{c}{\scriptsize{-0.333}} & \mc{1}{c}{\scriptsize{-1.668}} & \mc{1}{c}{\scriptsize{-0.664}} & \mc{1}{c}{\scriptsize{-1.600}} & \mc{1}{c}{\scriptsize{-2.010}} & \mc{1}{c}{\scriptsize{-1.875}} \\  

     &  & \mc{1}{c}{\scriptsize{(0.474)}} & \mc{1}{c}{\scriptsize{(0.316)}} & \mc{1}{c}{\scriptsize{(0.842)}} & \mc{1}{c}{\scriptsize{(0.461)}} & \mc{1}{c}{\scriptsize{(0.803)}} & \mc{1}{c}{\scriptsize{(0.408)}} & \mc{1}{c}{\scriptsize{(0.289)}} & \mc{1}{c}{\scriptsize{(0.316)}} \\  

    \mc{1}{l}{\scriptsize{Total Problems}} & \mc{1}{c}{\scriptsize{30}} & \mc{1}{c}{\scriptsize{0.805}} & \mc{1}{c}{\scriptsize{0.004}} & \mc{1}{c}{\scriptsize{2.374}} & \mc{1}{c}{\scriptsize{0.290}} & \mc{1}{c}{\scriptsize{2.008}} & \mc{1}{c}{\scriptsize{0.368}} & \mc{1}{c}{\scriptsize{-0.509}} & \mc{1}{c}{\scriptsize{0.059}} \\  

     &  & \mc{1}{c}{\scriptsize{(1.000)}} & \mc{1}{c}{\scriptsize{(0.987)}} & \mc{1}{c}{\scriptsize{(1.000)}} & \mc{1}{c}{\scriptsize{(0.947)}} & \mc{1}{c}{\scriptsize{(1.000)}} & \mc{1}{c}{\scriptsize{(0.987)}} & \mc{1}{c}{\scriptsize{(0.895)}} & \mc{1}{c}{\scriptsize{(0.987)}} \\  

    \mc{1}{l}{\scriptsize{Withdrawn}} & \mc{1}{c}{\scriptsize{30}} & \mc{1}{c}{\scriptsize{0.676}} & \mc{1}{c}{\scriptsize{0.822}} & \mc{1}{c}{\scriptsize{1.333}} & \mc{1}{c}{\scriptsize{1.096}} & \mc{1}{c}{\scriptsize{1.264}} & \mc{1}{c}{\scriptsize{0.580}} & \mc{1}{c}{\scriptsize{0.519}} & \mc{1}{c}{\scriptsize{0.684}} \\  

     &  & \mc{1}{c}{\scriptsize{(1.000)}} & \mc{1}{c}{\scriptsize{(1.000)}} & \mc{1}{c}{\scriptsize{(1.000)}} & \mc{1}{c}{\scriptsize{(0.974)}} & \mc{1}{c}{\scriptsize{(1.000)}} & \mc{1}{c}{\scriptsize{(1.000)}} & \mc{1}{c}{\scriptsize{(1.000)}} & \mc{1}{c}{\scriptsize{(1.000)}} \\  

    \mc{1}{l}{\scriptsize{Internalizing}} & \mc{1}{c}{\scriptsize{30}} & \mc{1}{c}{\scriptsize{1.330}} & \mc{1}{c}{\scriptsize{0.373}} & \mc{1}{c}{\scriptsize{3.744}} & \mc{1}{c}{\scriptsize{1.469}} & \mc{1}{c}{\scriptsize{3.134}} & \mc{1}{c}{\scriptsize{0.517}} & \mc{1}{c}{\scriptsize{-0.267}} & \mc{1}{c}{\scriptsize{-0.009}} \\  

     &  & \mc{1}{c}{\scriptsize{(1.000)}} & \mc{1}{c}{\scriptsize{(1.000)}} & \mc{1}{c}{\scriptsize{(1.000)}} & \mc{1}{c}{\scriptsize{(0.974)}} & \mc{1}{c}{\scriptsize{(1.000)}} & \mc{1}{c}{\scriptsize{(0.987)}} & \mc{1}{c}{\scriptsize{(0.961)}} & \mc{1}{c}{\scriptsize{(0.987)}} \\  

    \mc{1}{l}{\scriptsize{Critical Items}} & \mc{1}{c}{\scriptsize{30}} & \mc{1}{c}{\scriptsize{-1.055}} & \mc{1}{c}{\scriptsize{-1.249}} & \mc{1}{c}{\scriptsize{-0.723}} & \mc{1}{c}{\scriptsize{-1.790}} & \mc{1}{c}{\scriptsize{-0.781}} & \mc{1}{c}{\scriptsize{-1.243}} & \mc{1}{c}{\scriptsize{-1.355}} & \mc{1}{c}{\scriptsize{-1.343}} \\  

     &  & \mc{1}{c}{\scriptsize{(0.513)}} & \mc{1}{c}{\scriptsize{(0.553)}} & \mc{1}{c}{\scriptsize{(0.829)}} & \mc{1}{c}{\scriptsize{(0.461)}} & \mc{1}{c}{\scriptsize{(0.803)}} & \mc{1}{c}{\scriptsize{(0.434)}} & \mc{1}{c}{\scriptsize{(0.434)}} & \mc{1}{c}{\scriptsize{(0.461)}} \\  

    \mc{1}{l}{\scriptsize{Rule Breaking}} & \mc{1}{c}{\scriptsize{30}} & \mc{1}{c}{\scriptsize{-0.121}} & \mc{1}{c}{\scriptsize{-0.444}} & \mc{1}{c}{\scriptsize{-0.339}} & \mc{1}{c}{\scriptsize{-1.068}} & \mc{1}{c}{\scriptsize{-0.260}} & \mc{1}{c}{\scriptsize{-0.079}} & \mc{1}{c}{\scriptsize{-0.302}} & \mc{1}{c}{\scriptsize{-0.063}} \\  

     &  & \mc{1}{c}{\scriptsize{(0.961)}} & \mc{1}{c}{\scriptsize{(0.829)}} & \mc{1}{c}{\scriptsize{(0.842)}} & \mc{1}{c}{\scriptsize{(0.671)}} & \mc{1}{c}{\scriptsize{(0.855)}} & \mc{1}{c}{\scriptsize{(0.947)}} & \mc{1}{c}{\scriptsize{(0.921)}} & \mc{1}{c}{\scriptsize{(0.987)}} \\  

    \mc{1}{l}{\scriptsize{Attention Problems}} & \mc{1}{c}{\scriptsize{30}} & \mc{1}{c}{\scriptsize{0.482}} & \mc{1}{c}{\scriptsize{-0.277}} & \mc{1}{c}{\scriptsize{1.385}} & \mc{1}{c}{\scriptsize{0.179}} & \mc{1}{c}{\scriptsize{1.129}} & \mc{1}{c}{\scriptsize{0.145}} & \mc{1}{c}{\scriptsize{-0.389}} & \mc{1}{c}{\scriptsize{-0.108}} \\  

     &  & \mc{1}{c}{\scriptsize{(1.000)}} & \mc{1}{c}{\scriptsize{(0.882)}} & \mc{1}{c}{\scriptsize{(1.000)}} & \mc{1}{c}{\scriptsize{(0.961)}} & \mc{1}{c}{\scriptsize{(1.000)}} & \mc{1}{c}{\scriptsize{(0.987)}} & \mc{1}{c}{\scriptsize{(0.816)}} & \mc{1}{c}{\scriptsize{(0.974)}} \\  

  \bottomrule
  \end{tabular}
	\end{table} 

	\begin{table}[H]
     \caption{Treatment Effects on BSI 18 $t$-Score, Pooled Sample}
     \label{table:abccare_rslt_pooled_cat48_sd}
	\input{AppResOutput/abccare/rslt_pooled_cat48_sd}
	\end{table} 

	\begin{table}[H]
     \caption{Treatment Effects on BSI Raw Score, Pooled Sample}
     \label{table:abccare_rslt_pooled_cat49_sd}
	  \begin{tabular}{cccccccccc}
  \toprule

    \scriptsize{Variable} & \scriptsize{Age} & \scriptsize{(1)} & \scriptsize{(2)} & \scriptsize{(3)} & \scriptsize{(4)} & \scriptsize{(5)} & \scriptsize{(6)} & \scriptsize{(7)} & \scriptsize{(8)} \\ 
    \midrule  

    \mc{1}{l}{\scriptsize{Introversion}} & \mc{1}{c}{\scriptsize{12}} & \mc{1}{c}{\scriptsize{0.813}} & \mc{1}{c}{\scriptsize{0.365}} & \mc{1}{c}{\scriptsize{0.627}} & \mc{1}{c}{\scriptsize{-0.423}} & \mc{1}{c}{\scriptsize{-0.090}} & \mc{1}{c}{\scriptsize{0.961}} & \mc{1}{c}{\scriptsize{0.719}} & \mc{1}{c}{\scriptsize{0.311}} \\  

     &  & \mc{1}{c}{\scriptsize{(1.000)}} & \mc{1}{c}{\scriptsize{(1.000)}} & \mc{1}{c}{\scriptsize{(1.000)}} & \mc{1}{c}{\scriptsize{(0.961)}} & \mc{1}{c}{\scriptsize{(1.000)}} & \mc{1}{c}{\scriptsize{(1.000)}} & \mc{1}{c}{\scriptsize{(1.000)}} & \mc{1}{c}{\scriptsize{(1.000)}} \\  

     & \mc{1}{c}{\scriptsize{7}} & \mc{1}{c}{\scriptsize{0.600}} & \mc{1}{c}{\scriptsize{0.537}} & \mc{1}{c}{\scriptsize{-0.260}} & \mc{1}{c}{\scriptsize{-0.243}} & \mc{1}{c}{\scriptsize{-0.442}} & \mc{1}{c}{\scriptsize{0.870}} & \mc{1}{c}{\scriptsize{0.769}} & \mc{1}{c}{\scriptsize{0.684}} \\  

     &  & \mc{1}{c}{\scriptsize{(1.000)}} & \mc{1}{c}{\scriptsize{(1.000)}} & \mc{1}{c}{\scriptsize{(0.961)}} & \mc{1}{c}{\scriptsize{(0.987)}} & \mc{1}{c}{\scriptsize{(0.947)}} & \mc{1}{c}{\scriptsize{(1.000)}} & \mc{1}{c}{\scriptsize{(1.000)}} & \mc{1}{c}{\scriptsize{(1.000)}} \\  

    \mc{1}{l}{\scriptsize{Dependence}} & \mc{1}{c}{\scriptsize{6}} & \mc{1}{c}{\scriptsize{0.552}} & \mc{1}{c}{\scriptsize{0.066}} & \mc{1}{c}{\scriptsize{1.337}} & \mc{1}{c}{\scriptsize{0.993}} & \mc{1}{c}{\scriptsize{1.002}} & \mc{1}{c}{\scriptsize{0.332}} & \mc{1}{c}{\scriptsize{-0.156}} & \mc{1}{c}{\scriptsize{-0.075}} \\  

     &  & \mc{1}{c}{\scriptsize{(1.000)}} & \mc{1}{c}{\scriptsize{(1.000)}} & \mc{1}{c}{\scriptsize{(1.000)}} & \mc{1}{c}{\scriptsize{(1.000)}} & \mc{1}{c}{\scriptsize{(1.000)}} & \mc{1}{c}{\scriptsize{(1.000)}} & \mc{1}{c}{\scriptsize{(0.987)}} & \mc{1}{c}{\scriptsize{(1.000)}} \\  

    \mc{1}{l}{\scriptsize{Hostility}} & \mc{1}{c}{\scriptsize{7}} & \mc{1}{c}{\scriptsize{1.191}} & \mc{1}{c}{\scriptsize{1.076}} & \mc{1}{c}{\scriptsize{2.034}} & \mc{1}{c}{\scriptsize{2.804}} & \mc{1}{c}{\scriptsize{1.497}} & \mc{1}{c}{\scriptsize{0.991}} & \mc{1}{c}{\scriptsize{0.778}} & \mc{1}{c}{\scriptsize{0.641}} \\  

     &  & \mc{1}{c}{\scriptsize{(1.000)}} & \mc{1}{c}{\scriptsize{(1.000)}} & \mc{1}{c}{\scriptsize{(1.000)}} & \mc{1}{c}{\scriptsize{(1.000)}} & \mc{1}{c}{\scriptsize{(1.000)}} & \mc{1}{c}{\scriptsize{(1.000)}} & \mc{1}{c}{\scriptsize{(1.000)}} & \mc{1}{c}{\scriptsize{(1.000)}} \\  

    \mc{1}{l}{\scriptsize{Dependence}} & \mc{1}{c}{\scriptsize{12}} & \mc{1}{c}{\scriptsize{1.151}} & \mc{1}{c}{\scriptsize{0.948}} & \mc{1}{c}{\scriptsize{0.150}} & \mc{1}{c}{\scriptsize{-0.742}} & \mc{1}{c}{\scriptsize{-0.557}} & \mc{1}{c}{\scriptsize{1.522}} & \mc{1}{c}{\scriptsize{1.650}} & \mc{1}{c}{\scriptsize{1.665}} \\  

     &  & \mc{1}{c}{\scriptsize{(1.000)}} & \mc{1}{c}{\scriptsize{(1.000)}} & \mc{1}{c}{\scriptsize{(1.000)}} & \mc{1}{c}{\scriptsize{(0.961)}} & \mc{1}{c}{\scriptsize{(0.961)}} & \mc{1}{c}{\scriptsize{(1.000)}} & \mc{1}{c}{\scriptsize{(1.000)}} & \mc{1}{c}{\scriptsize{(1.000)}} \\  

    \mc{1}{l}{\scriptsize{Distractibility}} & \mc{1}{c}{\scriptsize{6}} & \mc{1}{c}{\scriptsize{0.200}} & \mc{1}{c}{\scriptsize{-0.002}} & \mc{1}{c}{\scriptsize{1.067}} & \mc{1}{c}{\scriptsize{1.015}} & \mc{1}{c}{\scriptsize{0.700}} & \mc{1}{c}{\scriptsize{-0.042}} & \mc{1}{c}{\scriptsize{-0.233}} & \mc{1}{c}{\scriptsize{-0.597}} \\  

     &  & \mc{1}{c}{\scriptsize{(1.000)}} & \mc{1}{c}{\scriptsize{(1.000)}} & \mc{1}{c}{\scriptsize{(1.000)}} & \mc{1}{c}{\scriptsize{(1.000)}} & \mc{1}{c}{\scriptsize{(1.000)}} & \mc{1}{c}{\scriptsize{(1.000)}} & \mc{1}{c}{\scriptsize{(0.987)}} & \mc{1}{c}{\scriptsize{(0.868)}} \\  

     & \mc{1}{c}{\scriptsize{12}} & \mc{1}{c}{\scriptsize{0.806}} & \mc{1}{c}{\scriptsize{0.887}} & \mc{1}{c}{\scriptsize{0.473}} & \mc{1}{c}{\scriptsize{-0.406}} & \mc{1}{c}{\scriptsize{-0.295}} & \mc{1}{c}{\scriptsize{0.955}} & \mc{1}{c}{\scriptsize{1.200}} & \mc{1}{c}{\scriptsize{0.662}} \\  

     &  & \mc{1}{c}{\scriptsize{(1.000)}} & \mc{1}{c}{\scriptsize{(1.000)}} & \mc{1}{c}{\scriptsize{(1.000)}} & \mc{1}{c}{\scriptsize{(0.987)}} & \mc{1}{c}{\scriptsize{(0.987)}} & \mc{1}{c}{\scriptsize{(1.000)}} & \mc{1}{c}{\scriptsize{(1.000)}} & \mc{1}{c}{\scriptsize{(1.000)}} \\  

     & \mc{1}{c}{\scriptsize{8}} & \mc{1}{c}{\scriptsize{0.427}} & \mc{1}{c}{\scriptsize{0.026}} & \mc{1}{c}{\scriptsize{0.229}} & \mc{1}{c}{\scriptsize{-0.018}} & \mc{1}{c}{\scriptsize{-0.082}} & \mc{1}{c}{\scriptsize{0.487}} & \mc{1}{c}{\scriptsize{0.097}} & \mc{1}{c}{\scriptsize{0.149}} \\  

     &  & \mc{1}{c}{\scriptsize{(1.000)}} & \mc{1}{c}{\scriptsize{(1.000)}} & \mc{1}{c}{\scriptsize{(1.000)}} & \mc{1}{c}{\scriptsize{(1.000)}} & \mc{1}{c}{\scriptsize{(1.000)}} & \mc{1}{c}{\scriptsize{(1.000)}} & \mc{1}{c}{\scriptsize{(1.000)}} & \mc{1}{c}{\scriptsize{(1.000)}} \\  

    \mc{1}{l}{\scriptsize{Introversion}} & \mc{1}{c}{\scriptsize{6}} & \mc{1}{c}{\scriptsize{0.015}} & \mc{1}{c}{\scriptsize{0.017}} & \mc{1}{c}{\scriptsize{-0.202}} & \mc{1}{c}{\scriptsize{-0.659}} & \mc{1}{c}{\scriptsize{-0.398}} & \mc{1}{c}{\scriptsize{0.075}} & \mc{1}{c}{\scriptsize{0.089}} & \mc{1}{c}{\scriptsize{-0.152}} \\  

     &  & \mc{1}{c}{\scriptsize{(1.000)}} & \mc{1}{c}{\scriptsize{(1.000)}} & \mc{1}{c}{\scriptsize{(0.961)}} & \mc{1}{c}{\scriptsize{(0.934)}} & \mc{1}{c}{\scriptsize{(0.961)}} & \mc{1}{c}{\scriptsize{(1.000)}} & \mc{1}{c}{\scriptsize{(1.000)}} & \mc{1}{c}{\scriptsize{(0.987)}} \\  

    \mc{1}{l}{\scriptsize{Hostility}} & \mc{1}{c}{\scriptsize{8}} & \mc{1}{c}{\scriptsize{1.694}} & \mc{1}{c}{\scriptsize{1.584}} & \mc{1}{c}{\scriptsize{1.567}} & \mc{1}{c}{\scriptsize{1.582}} & \mc{1}{c}{\scriptsize{1.570}} & \mc{1}{c}{\scriptsize{1.754}} & \mc{1}{c}{\scriptsize{1.630}} & \mc{1}{c}{\scriptsize{1.591}} \\  

     &  & \mc{1}{c}{\scriptsize{(1.000)}} & \mc{1}{c}{\scriptsize{(1.000)}} & \mc{1}{c}{\scriptsize{(1.000)}} & \mc{1}{c}{\scriptsize{(1.000)}} & \mc{1}{c}{\scriptsize{(1.000)}} & \mc{1}{c}{\scriptsize{(1.000)}} & \mc{1}{c}{\scriptsize{(1.000)}} & \mc{1}{c}{\scriptsize{(1.000)}} \\  

    \mc{1}{l}{\scriptsize{Dependence}} & \mc{1}{c}{\scriptsize{8}} & \mc{1}{c}{\scriptsize{0.447}} & \mc{1}{c}{\scriptsize{0.294}} & \mc{1}{c}{\scriptsize{1.050}} & \mc{1}{c}{\scriptsize{1.211}} & \mc{1}{c}{\scriptsize{0.793}} & \mc{1}{c}{\scriptsize{0.279}} & \mc{1}{c}{\scriptsize{0.150}} & \mc{1}{c}{\scriptsize{0.037}} \\  

     &  & \mc{1}{c}{\scriptsize{(1.000)}} & \mc{1}{c}{\scriptsize{(1.000)}} & \mc{1}{c}{\scriptsize{(1.000)}} & \mc{1}{c}{\scriptsize{(1.000)}} & \mc{1}{c}{\scriptsize{(1.000)}} & \mc{1}{c}{\scriptsize{(1.000)}} & \mc{1}{c}{\scriptsize{(1.000)}} & \mc{1}{c}{\scriptsize{(1.000)}} \\  

    \mc{1}{l}{\scriptsize{Distractibility}} & \mc{1}{c}{\scriptsize{7}} & \mc{1}{c}{\scriptsize{0.149}} & \mc{1}{c}{\scriptsize{0.199}} & \mc{1}{c}{\scriptsize{0.651}} & \mc{1}{c}{\scriptsize{1.335}} & \mc{1}{c}{\scriptsize{0.643}} & \mc{1}{c}{\scriptsize{-0.027}} & \mc{1}{c}{\scriptsize{-0.005}} & \mc{1}{c}{\scriptsize{-0.047}} \\  

     &  & \mc{1}{c}{\scriptsize{(1.000)}} & \mc{1}{c}{\scriptsize{(1.000)}} & \mc{1}{c}{\scriptsize{(1.000)}} & \mc{1}{c}{\scriptsize{(1.000)}} & \mc{1}{c}{\scriptsize{(1.000)}} & \mc{1}{c}{\scriptsize{(1.000)}} & \mc{1}{c}{\scriptsize{(1.000)}} & \mc{1}{c}{\scriptsize{(1.000)}} \\  

    \mc{1}{l}{\scriptsize{Dependence}} & \mc{1}{c}{\scriptsize{7}} & \mc{1}{c}{\scriptsize{0.342}} & \mc{1}{c}{\scriptsize{0.490}} & \mc{1}{c}{\scriptsize{1.113}} & \mc{1}{c}{\scriptsize{2.431}} & \mc{1}{c}{\scriptsize{1.284}} & \mc{1}{c}{\scriptsize{0.119}} & \mc{1}{c}{\scriptsize{0.142}} & \mc{1}{c}{\scriptsize{0.088}} \\  

     &  & \mc{1}{c}{\scriptsize{(1.000)}} & \mc{1}{c}{\scriptsize{(1.000)}} & \mc{1}{c}{\scriptsize{(1.000)}} & \mc{1}{c}{\scriptsize{(1.000)}} & \mc{1}{c}{\scriptsize{(1.000)}} & \mc{1}{c}{\scriptsize{(1.000)}} & \mc{1}{c}{\scriptsize{(1.000)}} & \mc{1}{c}{\scriptsize{(1.000)}} \\  

    \mc{1}{l}{\scriptsize{Introversion}} & \mc{1}{c}{\scriptsize{8}} & \mc{1}{c}{\scriptsize{-0.417}} & \mc{1}{c}{\scriptsize{-0.563}} & \mc{1}{c}{\scriptsize{0.138}} & \mc{1}{c}{\scriptsize{-0.529}} & \mc{1}{c}{\scriptsize{-0.060}} & \mc{1}{c}{\scriptsize{-0.518}} & \mc{1}{c}{\scriptsize{-0.526}} & \mc{1}{c}{\scriptsize{-0.697}} \\  

     &  & \mc{1}{c}{\scriptsize{(0.882)}} & \mc{1}{c}{\scriptsize{(0.711)}} & \mc{1}{c}{\scriptsize{(1.000)}} & \mc{1}{c}{\scriptsize{(0.776)}} & \mc{1}{c}{\scriptsize{(1.000)}} & \mc{1}{c}{\scriptsize{(0.855)}} & \mc{1}{c}{\scriptsize{(0.882)}} & \mc{1}{c}{\scriptsize{(0.697)}} \\  

    \mc{1}{l}{\scriptsize{Hostility}} & \mc{1}{c}{\scriptsize{12}} & \mc{1}{c}{\scriptsize{1.827}} & \mc{1}{c}{\scriptsize{1.956}} & \mc{1}{c}{\scriptsize{2.430}} & \mc{1}{c}{\scriptsize{2.198}} & \mc{1}{c}{\scriptsize{1.867}} & \mc{1}{c}{\scriptsize{1.541}} & \mc{1}{c}{\scriptsize{1.934}} & \mc{1}{c}{\scriptsize{0.871}} \\  

     &  & \mc{1}{c}{\scriptsize{(1.000)}} & \mc{1}{c}{\scriptsize{(1.000)}} & \mc{1}{c}{\scriptsize{(1.000)}} & \mc{1}{c}{\scriptsize{(1.000)}} & \mc{1}{c}{\scriptsize{(1.000)}} & \mc{1}{c}{\scriptsize{(1.000)}} & \mc{1}{c}{\scriptsize{(1.000)}} & \mc{1}{c}{\scriptsize{(1.000)}} \\  

     & \mc{1}{c}{\scriptsize{6}} & \mc{1}{c}{\scriptsize{1.904}} & \mc{1}{c}{\scriptsize{2.008}} & \mc{1}{c}{\scriptsize{3.382}} & \mc{1}{c}{\scriptsize{3.744}} & \mc{1}{c}{\scriptsize{3.250}} & \mc{1}{c}{\scriptsize{1.490}} & \mc{1}{c}{\scriptsize{1.632}} & \mc{1}{c}{\scriptsize{1.132}} \\  

     &  & \mc{1}{c}{\scriptsize{(1.000)}} & \mc{1}{c}{\scriptsize{(1.000)}} & \mc{1}{c}{\scriptsize{(1.000)}} & \mc{1}{c}{\scriptsize{(1.000)}} & \mc{1}{c}{\scriptsize{(1.000)}} & \mc{1}{c}{\scriptsize{(1.000)}} & \mc{1}{c}{\scriptsize{(1.000)}} & \mc{1}{c}{\scriptsize{(1.000)}} \\  

  \bottomrule
  \end{tabular}
	\end{table} 

	\begin{table}[H]
     \caption{Treatment Effects on BSI $t$-Score, Pooled Sample}
     \label{table:abccare_rslt_pooled_cat50_sd}
	  \begin{tabular}{cccccccccc}
  \toprule

    \scriptsize{Variable} & \scriptsize{Age} & \scriptsize{(1)} & \scriptsize{(2)} & \scriptsize{(3)} & \scriptsize{(4)} & \scriptsize{(5)} & \scriptsize{(6)} & \scriptsize{(7)} & \scriptsize{(8)} \\ 
    \midrule  

    \mc{1}{l}{\scriptsize{Impulsitivity - control}} & \mc{1}{c}{\scriptsize{8}} & \mc{1}{c}{\scriptsize{0.771}} & \mc{1}{c}{\scriptsize{1.071}} & \mc{1}{c}{\scriptsize{1.557}} & \mc{1}{c}{\scriptsize{2.364}} & \mc{1}{c}{\scriptsize{1.528}} & \mc{1}{c}{\scriptsize{0.706}} & \mc{1}{c}{\scriptsize{0.868}} & \mc{1}{c}{\scriptsize{0.838}} \\  

     &  & \mc{1}{c}{\scriptsize{(1.000)}} & \mc{1}{c}{\scriptsize{(1.000)}} & \mc{1}{c}{\scriptsize{(1.000)}} & \mc{1}{c}{\scriptsize{(1.000)}} & \mc{1}{c}{\scriptsize{(1.000)}} & \mc{1}{c}{\scriptsize{(1.000)}} & \mc{1}{c}{\scriptsize{(1.000)}} & \mc{1}{c}{\scriptsize{(1.000)}} \\  

    \mc{1}{l}{\scriptsize{Impulsitivity - decisive}} & \mc{1}{c}{\scriptsize{8}} & \mc{1}{c}{\scriptsize{0.986}} & \mc{1}{c}{\scriptsize{0.642}} & \mc{1}{c}{\scriptsize{0.906}} & \mc{1}{c}{\scriptsize{0.891}} & \mc{1}{c}{\scriptsize{0.734}} & \mc{1}{c}{\scriptsize{1.141}} & \mc{1}{c}{\scriptsize{0.613}} & \mc{1}{c}{\scriptsize{0.957}} \\  

     &  & \mc{1}{c}{\scriptsize{(1.000)}} & \mc{1}{c}{\scriptsize{(1.000)}} & \mc{1}{c}{\scriptsize{(1.000)}} & \mc{1}{c}{\scriptsize{(1.000)}} & \mc{1}{c}{\scriptsize{(1.000)}} & \mc{1}{c}{\scriptsize{(1.000)}} & \mc{1}{c}{\scriptsize{(1.000)}} & \mc{1}{c}{\scriptsize{(1.000)}} \\  

    \mc{1}{l}{\scriptsize{Activity - tempo}} & \mc{1}{c}{\scriptsize{8}} & \mc{1}{c}{\scriptsize{-0.632}} & \mc{1}{c}{\scriptsize{-0.857}} & \mc{1}{c}{\scriptsize{-0.576}} & \mc{1}{c}{\scriptsize{-0.342}} & \mc{1}{c}{\scriptsize{-0.709}} & \mc{1}{c}{\scriptsize{-0.550}} & \mc{1}{c}{\scriptsize{-0.785}} & \mc{1}{c}{\scriptsize{-0.978}} \\  

     &  & \mc{1}{c}{\scriptsize{(0.789)}} & \mc{1}{c}{\scriptsize{(0.750)}} & \mc{1}{c}{\scriptsize{(0.947)}} & \mc{1}{c}{\scriptsize{(0.934)}} & \mc{1}{c}{\scriptsize{(0.882)}} & \mc{1}{c}{\scriptsize{(0.842)}} & \mc{1}{c}{\scriptsize{(0.816)}} & \mc{1}{c}{\scriptsize{(0.684)}} \\  

    \mc{1}{l}{\scriptsize{Emotionality - fear}} & \mc{1}{c}{\scriptsize{8}} & \mc{1}{c}{\scriptsize{1.054}} & \mc{1}{c}{\scriptsize{1.052}} & \mc{1}{c}{\scriptsize{-0.192}} & \mc{1}{c}{\scriptsize{-0.112}} & \mc{1}{c}{\scriptsize{-0.111}} & \mc{1}{c}{\scriptsize{1.525}} & \mc{1}{c}{\scriptsize{1.546}} & \mc{1}{c}{\scriptsize{1.585}} \\  

     &  & \mc{1}{c}{\scriptsize{(1.000)}} & \mc{1}{c}{\scriptsize{(1.000)}} & \mc{1}{c}{\scriptsize{(0.974)}} & \mc{1}{c}{\scriptsize{(0.961)}} & \mc{1}{c}{\scriptsize{(0.974)}} & \mc{1}{c}{\scriptsize{(1.000)}} & \mc{1}{c}{\scriptsize{(1.000)}} & \mc{1}{c}{\scriptsize{(1.000)}} \\  

    \mc{1}{l}{\scriptsize{Emotionality - anger}} & \mc{1}{c}{\scriptsize{8}} & \mc{1}{c}{\scriptsize{0.219}} & \mc{1}{c}{\scriptsize{0.490}} & \mc{1}{c}{\scriptsize{0.012}} & \mc{1}{c}{\scriptsize{0.672}} & \mc{1}{c}{\scriptsize{0.039}} & \mc{1}{c}{\scriptsize{0.447}} & \mc{1}{c}{\scriptsize{0.698}} & \mc{1}{c}{\scriptsize{0.366}} \\  

     &  & \mc{1}{c}{\scriptsize{(1.000)}} & \mc{1}{c}{\scriptsize{(1.000)}} & \mc{1}{c}{\scriptsize{(1.000)}} & \mc{1}{c}{\scriptsize{(1.000)}} & \mc{1}{c}{\scriptsize{(1.000)}} & \mc{1}{c}{\scriptsize{(1.000)}} & \mc{1}{c}{\scriptsize{(1.000)}} & \mc{1}{c}{\scriptsize{(1.000)}} \\  

    \mc{1}{l}{\scriptsize{Impulsitivity - perservere}} & \mc{1}{c}{\scriptsize{8}} & \mc{1}{c}{\scriptsize{-0.616}} & \mc{1}{c}{\scriptsize{-0.518}} & \mc{1}{c}{\scriptsize{-0.180}} & \mc{1}{c}{\scriptsize{0.461}} & \mc{1}{c}{\scriptsize{-0.373}} & \mc{1}{c}{\scriptsize{-0.716}} & \mc{1}{c}{\scriptsize{-0.587}} & \mc{1}{c}{\scriptsize{-0.871}} \\  

     &  & \mc{1}{c}{\scriptsize{(0.737)}} & \mc{1}{c}{\scriptsize{(0.868)}} & \mc{1}{c}{\scriptsize{(0.961)}} & \mc{1}{c}{\scriptsize{(1.000)}} & \mc{1}{c}{\scriptsize{(0.934)}} & \mc{1}{c}{\scriptsize{(0.737)}} & \mc{1}{c}{\scriptsize{(0.842)}} & \mc{1}{c}{\scriptsize{(0.684)}} \\  

    \mc{1}{l}{\scriptsize{Sociablity}} & \mc{1}{c}{\scriptsize{8}} & \mc{1}{c}{\scriptsize{-0.140}} & \mc{1}{c}{\scriptsize{-0.102}} & \mc{1}{c}{\scriptsize{-0.140}} & \mc{1}{c}{\scriptsize{0.220}} & \mc{1}{c}{\scriptsize{-0.038}} & \mc{1}{c}{\scriptsize{-0.277}} & \mc{1}{c}{\scriptsize{-0.115}} & \mc{1}{c}{\scriptsize{-0.252}} \\  

     &  & \mc{1}{c}{\scriptsize{(1.000)}} & \mc{1}{c}{\scriptsize{(1.000)}} & \mc{1}{c}{\scriptsize{(1.000)}} & \mc{1}{c}{\scriptsize{(0.947)}} & \mc{1}{c}{\scriptsize{(1.000)}} & \mc{1}{c}{\scriptsize{(1.000)}} & \mc{1}{c}{\scriptsize{(1.000)}} & \mc{1}{c}{\scriptsize{(1.000)}} \\  

    \mc{1}{l}{\scriptsize{Emotionality - general}} & \mc{1}{c}{\scriptsize{8}} & \mc{1}{c}{\scriptsize{0.455}} & \mc{1}{c}{\scriptsize{0.399}} & \mc{1}{c}{\scriptsize{1.281}} & \mc{1}{c}{\scriptsize{1.645}} & \mc{1}{c}{\scriptsize{1.215}} & \mc{1}{c}{\scriptsize{0.221}} & \mc{1}{c}{\scriptsize{0.011}} & \mc{1}{c}{\scriptsize{0.053}} \\  

     &  & \mc{1}{c}{\scriptsize{(1.000)}} & \mc{1}{c}{\scriptsize{(1.000)}} & \mc{1}{c}{\scriptsize{(1.000)}} & \mc{1}{c}{\scriptsize{(1.000)}} & \mc{1}{c}{\scriptsize{(1.000)}} & \mc{1}{c}{\scriptsize{(1.000)}} & \mc{1}{c}{\scriptsize{(1.000)}} & \mc{1}{c}{\scriptsize{(1.000)}} \\  

    \mc{1}{l}{\scriptsize{Activity - vigor}} & \mc{1}{c}{\scriptsize{8}} & \mc{1}{c}{\scriptsize{-0.576}} & \mc{1}{c}{\scriptsize{-0.506}} & \mc{1}{c}{\scriptsize{-0.505}} & \mc{1}{c}{\scriptsize{0.322}} & \mc{1}{c}{\scriptsize{-0.586}} & \mc{1}{c}{\scriptsize{-0.612}} & \mc{1}{c}{\scriptsize{-0.607}} & \mc{1}{c}{\scriptsize{-0.830}} \\  

     &  & \mc{1}{c}{\scriptsize{(1.000)}} & \mc{1}{c}{\scriptsize{(1.000)}} & \mc{1}{c}{\scriptsize{(1.000)}} & \mc{1}{c}{\scriptsize{(0.934)}} & \mc{1}{c}{\scriptsize{(1.000)}} & \mc{1}{c}{\scriptsize{(1.000)}} & \mc{1}{c}{\scriptsize{(1.000)}} & \mc{1}{c}{\scriptsize{(1.000)}} \\  

    \mc{1}{l}{\scriptsize{Impulsitivity - sensation}} & \mc{1}{c}{\scriptsize{8}} & \mc{1}{c}{\scriptsize{-0.509}} & \mc{1}{c}{\scriptsize{-0.353}} & \mc{1}{c}{\scriptsize{-0.374}} & \mc{1}{c}{\scriptsize{0.432}} & \mc{1}{c}{\scriptsize{-0.423}} & \mc{1}{c}{\scriptsize{-0.496}} & \mc{1}{c}{\scriptsize{-0.431}} & \mc{1}{c}{\scriptsize{-0.631}} \\  

     &  & \mc{1}{c}{\scriptsize{(0.724)}} & \mc{1}{c}{\scriptsize{(0.895)}} & \mc{1}{c}{\scriptsize{(0.961)}} & \mc{1}{c}{\scriptsize{(1.000)}} & \mc{1}{c}{\scriptsize{(0.947)}} & \mc{1}{c}{\scriptsize{(0.789)}} & \mc{1}{c}{\scriptsize{(0.882)}} & \mc{1}{c}{\scriptsize{(0.737)}} \\  

  \bottomrule
  \end{tabular}
	\end{table} 

	\begin{table}[H]
     \caption{Treatment Effects on Mid-30s Mental Health Conditions, Pooled Sample}
     \label{table:abccare_rslt_pooled_cat51_sd}
	\input{AppResOutput/abccare/rslt_pooled_cat51_sd}
	\end{table} 

	\begin{table}[H]
     \caption{Treatment Effects on Smoking and Drinking Behavior, Pooled Sample}
     \label{table:abccare_rslt_pooled_cat52_sd}
	\input{AppResOutput/abccare/rslt_pooled_cat52_sd}
	\end{table} 

	\begin{table}[H]
     \caption{Treatment Effects on Tobacco, Drugs, Alcohol, Pooled Sample}
     \label{table:abccare_rslt_pooled_cat53_sd}
	  \begin{tabular}{cccccccccc}
  \toprule

    \scriptsize{Variable} & \scriptsize{Age} & \scriptsize{(1)} & \scriptsize{(2)} & \scriptsize{(3)} & \scriptsize{(4)} & \scriptsize{(5)} & \scriptsize{(6)} & \scriptsize{(7)} & \scriptsize{(8)} \\ 
    \midrule  

    \mc{1}{l}{\scriptsize{Days drank alcohol last month}} & \mc{1}{c}{\scriptsize{30}} & \mc{1}{c}{\scriptsize{0.244}} & \mc{1}{c}{\scriptsize{0.419}} & \mc{1}{c}{\scriptsize{-0.156}} & \mc{1}{c}{\scriptsize{-0.700}} & \mc{1}{c}{\scriptsize{0.113}} & \mc{1}{c}{\scriptsize{0.207}} & \mc{1}{c}{\scriptsize{0.618}} & \mc{1}{c}{\scriptsize{0.630}} \\  

     &  & \mc{1}{c}{\scriptsize{(0.750)}} & \mc{1}{c}{\scriptsize{(0.750)}} & \mc{1}{c}{\scriptsize{(0.618)}} & \mc{1}{c}{\scriptsize{(0.461)}} & \mc{1}{c}{\scriptsize{(0.645)}} & \mc{1}{c}{\scriptsize{(0.711)}} & \mc{1}{c}{\scriptsize{(0.803)}} & \mc{1}{c}{\scriptsize{(0.789)}} \\  

    \mc{1}{l}{\scriptsize{Days binge drank alcohol last month}} & \mc{1}{c}{\scriptsize{30}} & \mc{1}{c}{\scriptsize{0.085}} & \mc{1}{c}{\scriptsize{0.412}} & \mc{1}{c}{\scriptsize{-0.267}} & \mc{1}{c}{\scriptsize{-0.218}} & \mc{1}{c}{\scriptsize{-0.126}} & \mc{1}{c}{\scriptsize{0.151}} & \mc{1}{c}{\scriptsize{0.674}} & \mc{1}{c}{\scriptsize{0.393}} \\  

     &  & \mc{1}{c}{\scriptsize{(0.737)}} & \mc{1}{c}{\scriptsize{(0.895)}} & \mc{1}{c}{\scriptsize{(0.566)}} & \mc{1}{c}{\scriptsize{(0.539)}} & \mc{1}{c}{\scriptsize{(0.605)}} & \mc{1}{c}{\scriptsize{(0.763)}} & \mc{1}{c}{\scriptsize{(0.947)}} & \mc{1}{c}{\scriptsize{(0.908)}} \\  

  \bottomrule
  \end{tabular}
	\end{table} 
\section{Treatment Effects for Male Sample, Step Down}


	\begin{table}[H]
     \caption{Treatment Effects on IQ Scores, Male Sample}
     \label{table:abccare_rslt_male_cat0_sd}
	  \begin{tabular}{cccccccccc}
  \toprule

    \scriptsize{Variable} & \scriptsize{Age} & \scriptsize{(1)} & \scriptsize{(2)} & \scriptsize{(3)} & \scriptsize{(4)} & \scriptsize{(5)} & \scriptsize{(6)} & \scriptsize{(7)} & \scriptsize{(8)} \\ 
    \midrule  

    \mc{1}{l}{\scriptsize{Dyslipidemia}} & \mc{1}{c}{\scriptsize{Mid-30s}} & \mc{1}{c}{\scriptsize{-0.094}} & \mc{1}{c}{\scriptsize{-0.029}} & \mc{1}{c}{\scriptsize{0.200}} & \mc{1}{c}{\scriptsize{0.365}} & \mc{1}{c}{\scriptsize{0.207}} & \mc{1}{c}{\scriptsize{-0.108}} & \mc{1}{c}{\scriptsize{-0.056}} & \mc{1}{c}{\scriptsize{-0.130}} \\  

     &  & \mc{1}{c}{\scriptsize{(0.400)}} & \mc{1}{c}{\scriptsize{(0.507)}} & \mc{1}{c}{\scriptsize{(1.000)}} & \mc{1}{c}{\scriptsize{(0.985)}} & \mc{1}{c}{\scriptsize{(1.000)}} & \mc{1}{c}{\scriptsize{(0.365)}} & \mc{1}{c}{\scriptsize{(0.514)}} & \mc{1}{c}{\scriptsize{(0.338)}} \\  

    \mc{1}{l}{\scriptsize{High-Density Lipoprotein Chol. (mg/dL)}} & \mc{1}{c}{\scriptsize{Mid-30s}} & \mc{1}{c}{\scriptsize{7.753}} & \mc{1}{c}{\scriptsize{4.139}} & \mc{1}{c}{\scriptsize{-0.267}} & \mc{1}{c}{\scriptsize{-8.079}} & \mc{1}{c}{\scriptsize{-4.138}} & \mc{1}{c}{\scriptsize{9.015}} & \mc{1}{c}{\scriptsize{5.332}} & \mc{1}{c}{\scriptsize{6.830}} \\  

     &  & \mc{1}{c}{\scriptsize{\textbf{(0.027)}}} & \mc{1}{c}{\scriptsize{(0.280)}} & \mc{1}{c}{\scriptsize{(0.758)}} & \mc{1}{c}{\scriptsize{(0.985)}} & \mc{1}{c}{\scriptsize{(0.939)}} & \mc{1}{c}{\scriptsize{\textbf{(0.027)}}} & \mc{1}{c}{\scriptsize{(0.203)}} & \mc{1}{c}{\scriptsize{\textbf{(0.068)}}} \\  

  \bottomrule
  \end{tabular}
	\end{table} 

	\begin{table}[H]
     \caption{Treatment Effects on Achievement Scores, Male Sample}
     \label{table:abccare_rslt_male_cat1_sd}
	  \begin{tabular}{cccccccccc}
  \toprule

    \scriptsize{Variable} & \scriptsize{Age} & \scriptsize{(1)} & \scriptsize{(2)} & \scriptsize{(3)} & \scriptsize{(4)} & \scriptsize{(5)} & \scriptsize{(6)} & \scriptsize{(7)} & \scriptsize{(8)} \\ 
    \midrule  

    \mc{1}{l}{\scriptsize{Std. Achv.  Test}} & \mc{1}{c}{\scriptsize{5.5}} & \mc{1}{c}{\scriptsize{12.314}} & \mc{1}{c}{\scriptsize{9.930}} & \mc{1}{c}{\scriptsize{19.650}} & \mc{1}{c}{\scriptsize{16.805}} & \mc{1}{c}{\scriptsize{11.707}} & \mc{1}{c}{\scriptsize{9.869}} & \mc{1}{c}{\scriptsize{5.406}} & \mc{1}{c}{\scriptsize{2.381}} \\  

     &  & \mc{1}{c}{\scriptsize{\textbf{(0.009)}}} & \mc{1}{c}{\scriptsize{\textbf{(0.032)}}} & \mc{1}{c}{\scriptsize{(1.000)}} & \mc{1}{c}{\scriptsize{\textbf{(0.051)}}} & \mc{1}{c}{\scriptsize{(0.144)}} & \mc{1}{c}{\scriptsize{\textbf{(0.057)}}} & \mc{1}{c}{\scriptsize{(0.450)}} & \mc{1}{c}{\scriptsize{(0.816)}} \\  

     & \mc{1}{c}{\scriptsize{6}} & \mc{1}{c}{\scriptsize{6.269}} & \mc{1}{c}{\scriptsize{7.345}} & \mc{1}{c}{\scriptsize{10.379}} & \mc{1}{c}{\scriptsize{15.469}} & \mc{1}{c}{\scriptsize{3.396}} & \mc{1}{c}{\scriptsize{5.018}} & \mc{1}{c}{\scriptsize{4.784}} & \mc{1}{c}{\scriptsize{3.635}} \\  

     &  & \mc{1}{c}{\scriptsize{\textbf{(0.021)}}} & \mc{1}{c}{\scriptsize{\textbf{(0.025)}}} & \mc{1}{c}{\scriptsize{(1.000)}} & \mc{1}{c}{\scriptsize{(0.987)}} & \mc{1}{c}{\scriptsize{(0.814)}} & \mc{1}{c}{\scriptsize{\textbf{(0.070)}}} & \mc{1}{c}{\scriptsize{(0.292)}} & \mc{1}{c}{\scriptsize{(0.230)}} \\  

     & \mc{1}{c}{\scriptsize{6.5}} & \mc{1}{c}{\scriptsize{3.909}} & \mc{1}{c}{\scriptsize{3.593}} & \mc{1}{c}{\scriptsize{6.394}} & \mc{1}{c}{\scriptsize{7.317}} & \mc{1}{c}{\scriptsize{-0.618}} & \mc{1}{c}{\scriptsize{3.517}} &  & \mc{1}{c}{\scriptsize{2.328}} \\  

     &  & \mc{1}{c}{\scriptsize{\textbf{(0.051)}}} & \mc{1}{c}{\scriptsize{(0.212)}} & \mc{1}{c}{\scriptsize{(1.000)}} & \mc{1}{c}{\scriptsize{\textbf{(0.081)}}} & \mc{1}{c}{\scriptsize{(0.966)}} & \mc{1}{c}{\scriptsize{(0.103)}} &  & \mc{1}{c}{\scriptsize{(0.786)}} \\  

     & \mc{1}{c}{\scriptsize{7}} & \mc{1}{c}{\scriptsize{6.411}} & \mc{1}{c}{\scriptsize{6.164}} & \mc{1}{c}{\scriptsize{12.724}} & \mc{1}{c}{\scriptsize{13.150}} & \mc{1}{c}{\scriptsize{1.107}} & \mc{1}{c}{\scriptsize{5.415}} & \mc{1}{c}{\scriptsize{4.494}} & \mc{1}{c}{\scriptsize{0.819}} \\  

     &  & \mc{1}{c}{\scriptsize{\textbf{(0.015)}}} & \mc{1}{c}{\scriptsize{\textbf{(0.052)}}} & \mc{1}{c}{\scriptsize{(1.000)}} & \mc{1}{c}{\scriptsize{\textbf{(0.056)}}} & \mc{1}{c}{\scriptsize{(0.960)}} & \mc{1}{c}{\scriptsize{\textbf{(0.057)}}} & \mc{1}{c}{\scriptsize{(0.340)}} & \mc{1}{c}{\scriptsize{(0.946)}} \\  

     & \mc{1}{c}{\scriptsize{7.5}} & \mc{1}{c}{\scriptsize{4.133}} & \mc{1}{c}{\scriptsize{5.583}} & \mc{1}{c}{\scriptsize{4.300}} & \mc{1}{c}{\scriptsize{13.683}} & \mc{1}{c}{\scriptsize{-3.232}} & \mc{1}{c}{\scriptsize{4.082}} & \mc{1}{c}{\scriptsize{3.427}} & \mc{1}{c}{\scriptsize{0.201}} \\  

     &  & \mc{1}{c}{\scriptsize{\textbf{(0.081)}}} & \mc{1}{c}{\scriptsize{\textbf{(0.046)}}} & \mc{1}{c}{\scriptsize{(1.000)}} & \mc{1}{c}{\scriptsize{(0.987)}} & \mc{1}{c}{\scriptsize{(0.849)}} & \mc{1}{c}{\scriptsize{(0.103)}} & \mc{1}{c}{\scriptsize{(0.450)}} & \mc{1}{c}{\scriptsize{(0.946)}} \\  

     & \mc{1}{c}{\scriptsize{8}} & \mc{1}{c}{\scriptsize{6.619}} & \mc{1}{c}{\scriptsize{7.230}} & \mc{1}{c}{\scriptsize{7.125}} & \mc{1}{c}{\scriptsize{14.262}} & \mc{1}{c}{\scriptsize{-2.668}} & \mc{1}{c}{\scriptsize{6.465}} & \mc{1}{c}{\scriptsize{5.342}} & \mc{1}{c}{\scriptsize{4.225}} \\  

     &  & \mc{1}{c}{\scriptsize{\textbf{(0.037)}}} & \mc{1}{c}{\scriptsize{\textbf{(0.046)}}} & \mc{1}{c}{\scriptsize{(1.000)}} & \mc{1}{c}{\scriptsize{(0.987)}} & \mc{1}{c}{\scriptsize{(0.868)}} & \mc{1}{c}{\scriptsize{\textbf{(0.070)}}} & \mc{1}{c}{\scriptsize{(0.336)}} & \mc{1}{c}{\scriptsize{(0.286)}} \\  

     & \mc{1}{c}{\scriptsize{8.5}} & \mc{1}{c}{\scriptsize{8.407}} & \mc{1}{c}{\scriptsize{8.809}} & \mc{1}{c}{\scriptsize{12.299}} & \mc{1}{c}{\scriptsize{17.510}} & \mc{1}{c}{\scriptsize{-1.456}} & \mc{1}{c}{\scriptsize{7.223}} & \mc{1}{c}{\scriptsize{6.307}} & \mc{1}{c}{\scriptsize{4.103}} \\  

     &  & \mc{1}{c}{\scriptsize{\textbf{(0.015)}}} & \mc{1}{c}{\scriptsize{\textbf{(0.023)}}} & \mc{1}{c}{\scriptsize{(1.000)}} & \mc{1}{c}{\scriptsize{(0.987)}} & \mc{1}{c}{\scriptsize{(0.960)}} & \mc{1}{c}{\scriptsize{\textbf{(0.053)}}} & \mc{1}{c}{\scriptsize{(0.246)}} & \mc{1}{c}{\scriptsize{(0.418)}} \\  

     & \mc{1}{c}{\scriptsize{15}} & \mc{1}{c}{\scriptsize{8.275}} & \mc{1}{c}{\scriptsize{5.193}} & \mc{1}{c}{\scriptsize{9.618}} & \mc{1}{c}{\scriptsize{8.704}} & \mc{1}{c}{\scriptsize{0.590}} & \mc{1}{c}{\scriptsize{8.477}} & \mc{1}{c}{\scriptsize{4.611}} & \mc{1}{c}{\scriptsize{0.925}} \\  

     &  & \mc{1}{c}{\scriptsize{\textbf{(0.021)}}} & \mc{1}{c}{\scriptsize{(0.212)}} & \mc{1}{c}{\scriptsize{(1.000)}} & \mc{1}{c}{\scriptsize{(0.987)}} & \mc{1}{c}{\scriptsize{(0.966)}} & \mc{1}{c}{\scriptsize{\textbf{(0.052)}}} & \mc{1}{c}{\scriptsize{(0.450)}} & \mc{1}{c}{\scriptsize{(0.946)}} \\  

     & \mc{1}{c}{\scriptsize{21}} & \mc{1}{c}{\scriptsize{9.116}} & \mc{1}{c}{\scriptsize{4.293}} & \mc{1}{c}{\scriptsize{8.420}} & \mc{1}{c}{\scriptsize{4.889}} & \mc{1}{c}{\scriptsize{-0.231}} & \mc{1}{c}{\scriptsize{9.420}} & \mc{1}{c}{\scriptsize{4.412}} & \mc{1}{c}{\scriptsize{-0.876}} \\  

     &  & \mc{1}{c}{\scriptsize{\textbf{(0.021)}}} & \mc{1}{c}{\scriptsize{(0.212)}} & \mc{1}{c}{\scriptsize{(1.000)}} & \mc{1}{c}{\scriptsize{(0.987)}} & \mc{1}{c}{\scriptsize{(0.966)}} & \mc{1}{c}{\scriptsize{\textbf{(0.052)}}} & \mc{1}{c}{\scriptsize{(0.450)}} & \mc{1}{c}{\scriptsize{(0.946)}} \\  

    \mc{1}{l}{\scriptsize{Achievement Factor}} & \mc{1}{c}{\scriptsize{5.5 to 12}} & \mc{1}{c}{\scriptsize{0.880}} & \mc{1}{c}{\scriptsize{1.024}} & \mc{1}{c}{\scriptsize{1.244}} & \mc{1}{c}{\scriptsize{1.547}} & \mc{1}{c}{\scriptsize{0.121}} & \mc{1}{c}{\scriptsize{0.739}} & \mc{1}{c}{\scriptsize{0.748}} & \mc{1}{c}{\scriptsize{0.293}} \\  

     &  & \mc{1}{c}{\scriptsize{\textbf{(0.021)}}} & \mc{1}{c}{\scriptsize{\textbf{(0.028)}}} & \mc{1}{c}{\scriptsize{(1.000)}} & \mc{1}{c}{\scriptsize{(0.987)}} & \mc{1}{c}{\scriptsize{(0.966)}} & \mc{1}{c}{\scriptsize{\textbf{(0.069)}}} & \mc{1}{c}{\scriptsize{(0.182)}} & \mc{1}{c}{\scriptsize{(0.622)}} \\  

     & \mc{1}{c}{\scriptsize{15 to 21}} & \mc{1}{c}{\scriptsize{-0.769}} & \mc{1}{c}{\scriptsize{-0.423}} & \mc{1}{c}{\scriptsize{-0.803}} & \mc{1}{c}{\scriptsize{-0.613}} & \mc{1}{c}{\scriptsize{-0.018}} & \mc{1}{c}{\scriptsize{-0.791}} & \mc{1}{c}{\scriptsize{-0.401}} & \mc{1}{c}{\scriptsize{-0.007}} \\  

     &  & \mc{1}{c}{\scriptsize{\textbf{(0.015)}}} & \mc{1}{c}{\scriptsize{(0.212)}} & \mc{1}{c}{\scriptsize{(1.000)}} & \mc{1}{c}{\scriptsize{(0.987)}} & \mc{1}{c}{\scriptsize{(0.966)}} & \mc{1}{c}{\scriptsize{\textbf{(0.021)}}} & \mc{1}{c}{\scriptsize{(0.450)}} & \mc{1}{c}{\scriptsize{(0.946)}} \\  

  \bottomrule
  \end{tabular}
	\end{table} 

	\begin{table}[H]
     \caption{Treatment Effects on Infant Behavior Record, Male Sample}
     \label{table:abccare_rslt_male_cat2_sd}
	  \begin{tabular}{cccccccccc}
  \toprule

    \scriptsize{Variable} & \scriptsize{Age} & \scriptsize{(1)} & \scriptsize{(2)} & \scriptsize{(3)} & \scriptsize{(4)} & \scriptsize{(5)} & \scriptsize{(6)} & \scriptsize{(7)} & \scriptsize{(8)} \\ 
    \midrule  

    \mc{1}{l}{\scriptsize{HOME Score}} & \mc{1}{c}{\scriptsize{0.5}} & \mc{1}{c}{\scriptsize{1.581}} & \mc{1}{c}{\scriptsize{0.746}} & \mc{1}{c}{\scriptsize{1.684}} & \mc{1}{c}{\scriptsize{2.248}} & \mc{1}{c}{\scriptsize{0.445}} & \mc{1}{c}{\scriptsize{0.980}} & \mc{1}{c}{\scriptsize{0.570}} & \mc{1}{c}{\scriptsize{-0.089}} \\  

     &  & \mc{1}{c}{\scriptsize{(3.176)}} & \mc{1}{c}{\scriptsize{(6.431)}} & \mc{1}{c}{\scriptsize{(1.637)}} & \mc{1}{c}{\scriptsize{(2.363)}} & \mc{1}{c}{\scriptsize{(6.431)}} & \mc{1}{c}{\scriptsize{(5.490)}} & \mc{1}{c}{\scriptsize{(6.245)}} & \mc{1}{c}{\scriptsize{(7.971)}} \\  

     & \mc{1}{c}{\scriptsize{1.5}} & \mc{1}{c}{\scriptsize{2.668}} & \mc{1}{c}{\scriptsize{1.582}} & \mc{1}{c}{\scriptsize{4.729}} & \mc{1}{c}{\scriptsize{3.111}} & \mc{1}{c}{\scriptsize{0.244}} & \mc{1}{c}{\scriptsize{1.544}} & \mc{1}{c}{\scriptsize{1.215}} & \mc{1}{c}{\scriptsize{-0.879}} \\  

     &  & \mc{1}{c}{\scriptsize{(1.206)}} & \mc{1}{c}{\scriptsize{(4.686)}} & \mc{1}{c}{\scriptsize{(0.412)}} & \mc{1}{c}{\scriptsize{(2.363)}} & \mc{1}{c}{\scriptsize{(6.431)}} & \mc{1}{c}{\scriptsize{(4.578)}} & \mc{1}{c}{\scriptsize{(6.069)}} & \mc{1}{c}{\scriptsize{(6.804)}} \\  

     & \mc{1}{c}{\scriptsize{2.5}} & \mc{1}{c}{\scriptsize{0.762}} & \mc{1}{c}{\scriptsize{0.747}} & \mc{1}{c}{\scriptsize{4.434}} & \mc{1}{c}{\scriptsize{6.109}} & \mc{1}{c}{\scriptsize{2.220}} & \mc{1}{c}{\scriptsize{-0.899}} & \mc{1}{c}{\scriptsize{-0.872}} & \mc{1}{c}{\scriptsize{0.140}} \\  

     &  & \mc{1}{c}{\scriptsize{(4.304)}} & \mc{1}{c}{\scriptsize{(6.431)}} & \mc{1}{c}{\scriptsize{(0.127)}} & \mc{1}{c}{\scriptsize{(0.667)}} & \mc{1}{c}{\scriptsize{(4.961)}} & \mc{1}{c}{\scriptsize{(5.490)}} & \mc{1}{c}{\scriptsize{(6.196)}} & \mc{1}{c}{\scriptsize{(7.971)}} \\  

     & \mc{1}{c}{\scriptsize{3.5}} & \mc{1}{c}{\scriptsize{2.858}} & \mc{1}{c}{\scriptsize{1.686}} & \mc{1}{c}{\scriptsize{13.719}} & \mc{1}{c}{\scriptsize{16.856}} & \mc{1}{c}{\scriptsize{2.873}} & \mc{1}{c}{\scriptsize{-0.309}} & \mc{1}{c}{\scriptsize{-1.673}} & \mc{1}{c}{\scriptsize{0.728}} \\  

     &  & \mc{1}{c}{\scriptsize{(3.176)}} & \mc{1}{c}{\scriptsize{(6.294)}} & \mc{1}{c}{\scriptsize{\textbf{(0.039)}}} & \mc{1}{c}{\scriptsize{(0.275)}} & \mc{1}{c}{\scriptsize{(5.333)}} & \mc{1}{c}{\scriptsize{(6.284)}} & \mc{1}{c}{\scriptsize{(6.196)}} & \mc{1}{c}{\scriptsize{(7.725)}} \\  

     & \mc{1}{c}{\scriptsize{4.5}} & \mc{1}{c}{\scriptsize{2.736}} & \mc{1}{c}{\scriptsize{1.425}} & \mc{1}{c}{\scriptsize{12.957}} & \mc{1}{c}{\scriptsize{15.238}} & \mc{1}{c}{\scriptsize{2.814}} & \mc{1}{c}{\scriptsize{-0.273}} & \mc{1}{c}{\scriptsize{-0.335}} & \mc{1}{c}{\scriptsize{0.214}} \\  

     &  & \mc{1}{c}{\scriptsize{(3.176)}} & \mc{1}{c}{\scriptsize{(6.431)}} & \mc{1}{c}{\scriptsize{(9.814)}} & \mc{1}{c}{\scriptsize{(0.549)}} & \mc{1}{c}{\scriptsize{(5.206)}} & \mc{1}{c}{\scriptsize{(6.284)}} & \mc{1}{c}{\scriptsize{(6.245)}} & \mc{1}{c}{\scriptsize{(7.971)}} \\  

     & \mc{1}{c}{\scriptsize{8}} & \mc{1}{c}{\scriptsize{0.659}} & \mc{1}{c}{\scriptsize{0.572}} & \mc{1}{c}{\scriptsize{5.909}} & \mc{1}{c}{\scriptsize{7.605}} & \mc{1}{c}{\scriptsize{-1.555}} & \mc{1}{c}{\scriptsize{-0.773}} & \mc{1}{c}{\scriptsize{-1.183}} & \mc{1}{c}{\scriptsize{0.144}} \\  

     &  & \mc{1}{c}{\scriptsize{(4.304)}} & \mc{1}{c}{\scriptsize{(6.431)}} & \mc{1}{c}{\scriptsize{(9.814)}} & \mc{1}{c}{\scriptsize{(0.882)}} & \mc{1}{c}{\scriptsize{(6.431)}} & \mc{1}{c}{\scriptsize{(6.167)}} & \mc{1}{c}{\scriptsize{(6.245)}} & \mc{1}{c}{\scriptsize{(7.971)}} \\  

    \mc{1}{l}{\scriptsize{HOME Factor}} & \mc{1}{c}{\scriptsize{0.5 to 8}} & \mc{1}{c}{\scriptsize{0.266}} & \mc{1}{c}{\scriptsize{0.125}} & \mc{1}{c}{\scriptsize{1.162}} & \mc{1}{c}{\scriptsize{1.014}} & \mc{1}{c}{\scriptsize{0.083}} & \mc{1}{c}{\scriptsize{0.010}} & \mc{1}{c}{\scriptsize{-0.054}} & \mc{1}{c}{\scriptsize{0.282}} \\  

     &  & \mc{1}{c}{\scriptsize{(3.363)}} & \mc{1}{c}{\scriptsize{(6.431)}} & \mc{1}{c}{\scriptsize{(9.814)}} & \mc{1}{c}{\scriptsize{(2.363)}} & \mc{1}{c}{\scriptsize{(6.431)}} & \mc{1}{c}{\scriptsize{(6.284)}} & \mc{1}{c}{\scriptsize{(6.245)}} & \mc{1}{c}{\scriptsize{(5.167)}} \\  

  \bottomrule
  \end{tabular}
	\end{table} 

	\begin{table}[H]
     \caption{Treatment Effects on Kohn and Rosman: Attentive/Cooperative, Male Sample}
     \label{table:abccare_rslt_male_cat3_sd}
	  \begin{tabular}{cccccccccc}
  \toprule

    \scriptsize{Variable} & \scriptsize{Age} & \scriptsize{(1)} & \scriptsize{(2)} & \scriptsize{(3)} & \scriptsize{(4)} & \scriptsize{(5)} & \scriptsize{(6)} & \scriptsize{(7)} & \scriptsize{(8)} \\ 
    \midrule  

    \mc{1}{l}{\scriptsize{Parental Income}} & \mc{1}{c}{\scriptsize{1.5}} & \mc{1}{c}{\scriptsize{330}} & \mc{1}{c}{\scriptsize{-97.199}} & \mc{1}{c}{\scriptsize{-1,046}} & \mc{1}{c}{\scriptsize{-2,384}} & \mc{1}{c}{\scriptsize{-1,168}} & \mc{1}{c}{\scriptsize{-9.245}} & \mc{1}{c}{\scriptsize{-26.663}} & \mc{1}{c}{\scriptsize{872}} \\  

     &  & \mc{1}{c}{\scriptsize{(0.658)}} & \mc{1}{c}{\scriptsize{(0.855)}} & \mc{1}{c}{\scriptsize{(0.776)}} & \mc{1}{c}{\scriptsize{(0.882)}} & \mc{1}{c}{\scriptsize{(0.816)}} & \mc{1}{c}{\scriptsize{(0.684)}} & \mc{1}{c}{\scriptsize{(0.829)}} & \mc{1}{c}{\scriptsize{(0.658)}} \\  

     & \mc{1}{c}{\scriptsize{2.5}} & \mc{1}{c}{\scriptsize{673}} & \mc{1}{c}{\scriptsize{-941}} & \mc{1}{c}{\scriptsize{-1,167}} & \mc{1}{c}{\scriptsize{-3,542}} & \mc{1}{c}{\scriptsize{-1,858}} & \mc{1}{c}{\scriptsize{478}} & \mc{1}{c}{\scriptsize{-839}} & \mc{1}{c}{\scriptsize{232}} \\  

     &  & \mc{1}{c}{\scriptsize{(0.579)}} & \mc{1}{c}{\scriptsize{(0.908)}} & \mc{1}{c}{\scriptsize{(0.776)}} & \mc{1}{c}{\scriptsize{(0.934)}} & \mc{1}{c}{\scriptsize{(0.842)}} & \mc{1}{c}{\scriptsize{(0.618)}} & \mc{1}{c}{\scriptsize{(0.908)}} & \mc{1}{c}{\scriptsize{(0.737)}} \\  

     & \mc{1}{c}{\scriptsize{3.5}} & \mc{1}{c}{\scriptsize{1,036}} & \mc{1}{c}{\scriptsize{223}} & \mc{1}{c}{\scriptsize{3,085}} & \mc{1}{c}{\scriptsize{-1,152}} & \mc{1}{c}{\scriptsize{1,448}} & \mc{1}{c}{\scriptsize{112}} & \mc{1}{c}{\scriptsize{47.496}} & \mc{1}{c}{\scriptsize{701}} \\  

     &  & \mc{1}{c}{\scriptsize{(0.539)}} & \mc{1}{c}{\scriptsize{(0.789)}} & \mc{1}{c}{\scriptsize{(0.421)}} & \mc{1}{c}{\scriptsize{(0.763)}} & \mc{1}{c}{\scriptsize{(0.553)}} & \mc{1}{c}{\scriptsize{(0.684)}} & \mc{1}{c}{\scriptsize{(0.816)}} & \mc{1}{c}{\scriptsize{(0.711)}} \\  

     & \mc{1}{c}{\scriptsize{4.5}} & \mc{1}{c}{\scriptsize{821}} & \mc{1}{c}{\scriptsize{1,677}} & \mc{1}{c}{\scriptsize{1,561}} & \mc{1}{c}{\scriptsize{3,815}} & \mc{1}{c}{\scriptsize{-2,662}} & \mc{1}{c}{\scriptsize{-81.743}} & \mc{1}{c}{\scriptsize{917}} & \mc{1}{c}{\scriptsize{-429}} \\  

     &  & \mc{1}{c}{\scriptsize{(0.539)}} & \mc{1}{c}{\scriptsize{(0.566)}} & \mc{1}{c}{\scriptsize{(0.553)}} & \mc{1}{c}{\scriptsize{(0.395)}} & \mc{1}{c}{\scriptsize{(0.961)}} & \mc{1}{c}{\scriptsize{(0.697)}} & \mc{1}{c}{\scriptsize{(0.697)}} & \mc{1}{c}{\scriptsize{(0.829)}} \\  

    \mc{1}{l}{\scriptsize{Parental Income Factor}} & \mc{1}{c}{\scriptsize{1.5 to 15}} & \mc{1}{c}{\scriptsize{0.130}} & \mc{1}{c}{\scriptsize{-0.018}} & \mc{1}{c}{\scriptsize{0.079}} & \mc{1}{c}{\scriptsize{0.039}} & \mc{1}{c}{\scriptsize{0.092}} & \mc{1}{c}{\scriptsize{0.088}} & \mc{1}{c}{\scriptsize{-0.062}} & \mc{1}{c}{\scriptsize{0.116}} \\  

     &  & \mc{1}{c}{\scriptsize{(0.421)}} & \mc{1}{c}{\scriptsize{(0.855)}} & \mc{1}{c}{\scriptsize{(0.645)}} & \mc{1}{c}{\scriptsize{(0.618)}} & \mc{1}{c}{\scriptsize{(0.618)}} & \mc{1}{c}{\scriptsize{(0.579)}} & \mc{1}{c}{\scriptsize{(0.908)}} & \mc{1}{c}{\scriptsize{(0.632)}} \\  

  \bottomrule
  \end{tabular}
	\end{table} 

	\begin{table}[H]
     \caption{Treatment Effects on Classroom Behavior Inventory (Part I), Male Sample}
     \label{table:abccare_rslt_male_cat4_sd}
	  \begin{tabular}{cccccccccc}
  \toprule

    \scriptsize{Variable} & \scriptsize{Age} & \scriptsize{(1)} & \scriptsize{(2)} & \scriptsize{(3)} & \scriptsize{(4)} & \scriptsize{(5)} & \scriptsize{(6)} & \scriptsize{(7)} & \scriptsize{(8)} \\ 
    \midrule  

    \mc{1}{l}{\scriptsize{Parental Labor Income}} & \mc{1}{c}{\scriptsize{1.5}} & \mc{1}{c}{\scriptsize{330}} & \mc{1}{c}{\scriptsize{-850}} & \mc{1}{c}{\scriptsize{-1,046}} & \mc{1}{c}{\scriptsize{-3,177}} & \mc{1}{c}{\scriptsize{-1,164}} & \mc{1}{c}{\scriptsize{-9.244}} & \mc{1}{c}{\scriptsize{-529}} & \mc{1}{c}{\scriptsize{866}} \\  

     &  & \mc{1}{c}{\scriptsize{(0.637)}} & \mc{1}{c}{\scriptsize{(0.814)}} & \mc{1}{c}{\scriptsize{(1.000)}} & \mc{1}{c}{\scriptsize{(0.980)}} & \mc{1}{c}{\scriptsize{(0.882)}} & \mc{1}{c}{\scriptsize{(0.676)}} & \mc{1}{c}{\scriptsize{(0.794)}} & \mc{1}{c}{\scriptsize{(0.765)}} \\  

     & \mc{1}{c}{\scriptsize{2.5}} & \mc{1}{c}{\scriptsize{673}} & \mc{1}{c}{\scriptsize{-1,970}} & \mc{1}{c}{\scriptsize{-1,167}} & \mc{1}{c}{\scriptsize{-4,773}} & \mc{1}{c}{\scriptsize{-1,856}} & \mc{1}{c}{\scriptsize{478}} & \mc{1}{c}{\scriptsize{-1,648}} & \mc{1}{c}{\scriptsize{228}} \\  

     &  & \mc{1}{c}{\scriptsize{(0.637)}} & \mc{1}{c}{\scriptsize{(0.814)}} & \mc{1}{c}{\scriptsize{(1.000)}} & \mc{1}{c}{\scriptsize{(0.980)}} & \mc{1}{c}{\scriptsize{(0.882)}} & \mc{1}{c}{\scriptsize{(0.647)}} & \mc{1}{c}{\scriptsize{(0.833)}} & \mc{1}{c}{\scriptsize{(0.765)}} \\  

     & \mc{1}{c}{\scriptsize{3.5}} & \mc{1}{c}{\scriptsize{1,036}} & \mc{1}{c}{\scriptsize{-1,185}} & \mc{1}{c}{\scriptsize{3,085}} & \mc{1}{c}{\scriptsize{-2,321}} & \mc{1}{c}{\scriptsize{1,452}} & \mc{1}{c}{\scriptsize{112}} & \mc{1}{c}{\scriptsize{-1,171}} & \mc{1}{c}{\scriptsize{703}} \\  

     &  & \mc{1}{c}{\scriptsize{(0.598)}} & \mc{1}{c}{\scriptsize{(0.814)}} & \mc{1}{c}{\scriptsize{(1.000)}} & \mc{1}{c}{\scriptsize{(0.980)}} & \mc{1}{c}{\scriptsize{(0.775)}} & \mc{1}{c}{\scriptsize{(0.676)}} & \mc{1}{c}{\scriptsize{(0.814)}} & \mc{1}{c}{\scriptsize{(0.765)}} \\  

     & \mc{1}{c}{\scriptsize{4.5}} & \mc{1}{c}{\scriptsize{821}} & \mc{1}{c}{\scriptsize{1,547}} & \mc{1}{c}{\scriptsize{1,561}} & \mc{1}{c}{\scriptsize{1,867}} & \mc{1}{c}{\scriptsize{-2,687}} & \mc{1}{c}{\scriptsize{-81.743}} & \mc{1}{c}{\scriptsize{723}} & \mc{1}{c}{\scriptsize{-420}} \\  

     &  & \mc{1}{c}{\scriptsize{(0.637)}} & \mc{1}{c}{\scriptsize{(0.627)}} & \mc{1}{c}{\scriptsize{(0.490)}} & \mc{1}{c}{\scriptsize{(0.716)}} & \mc{1}{c}{\scriptsize{(0.882)}} & \mc{1}{c}{\scriptsize{(0.676)}} & \mc{1}{c}{\scriptsize{(0.725)}} & \mc{1}{c}{\scriptsize{(0.765)}} \\  

     & \mc{1}{c}{\scriptsize{8}} & \mc{1}{c}{\scriptsize{11,786}} & \mc{1}{c}{\scriptsize{12,461}} & \mc{1}{c}{\scriptsize{6,832}} & \mc{1}{c}{\scriptsize{5,160}} & \mc{1}{c}{\scriptsize{4,889}} & \mc{1}{c}{\scriptsize{13,438}} & \mc{1}{c}{\scriptsize{13,460}} & \mc{1}{c}{\scriptsize{13,487}} \\  

     &  & \mc{1}{c}{\scriptsize{\textbf{(0.088)}}} & \mc{1}{c}{\scriptsize{(0.206)}} & \mc{1}{c}{\scriptsize{(0.353)}} & \mc{1}{c}{\scriptsize{(0.716)}} & \mc{1}{c}{\scriptsize{(0.775)}} & \mc{1}{c}{\scriptsize{\textbf{(0.078)}}} & \mc{1}{c}{\scriptsize{(0.294)}} & \mc{1}{c}{\scriptsize{(0.186)}} \\  

     & \mc{1}{c}{\scriptsize{12}} & \mc{1}{c}{\scriptsize{7,085}} & \mc{1}{c}{\scriptsize{10,384}} & \mc{1}{c}{\scriptsize{15,563}} & \mc{1}{c}{\scriptsize{20,007}} & \mc{1}{c}{\scriptsize{12,682}} & \mc{1}{c}{\scriptsize{4,773}} & \mc{1}{c}{\scriptsize{7,791}} & \mc{1}{c}{\scriptsize{5,411}} \\  

     &  & \mc{1}{c}{\scriptsize{(0.294)}} & \mc{1}{c}{\scriptsize{(0.167)}} & \mc{1}{c}{\scriptsize{(0.206)}} & \mc{1}{c}{\scriptsize{(0.206)}} & \mc{1}{c}{\scriptsize{(0.284)}} & \mc{1}{c}{\scriptsize{(0.431)}} & \mc{1}{c}{\scriptsize{(0.294)}} & \mc{1}{c}{\scriptsize{(0.461)}} \\  

     & \mc{1}{c}{\scriptsize{15}} & \mc{1}{c}{\scriptsize{8,488}} & \mc{1}{c}{\scriptsize{7,185}} & \mc{1}{c}{\scriptsize{6,697}} & \mc{1}{c}{\scriptsize{10,024}} & \mc{1}{c}{\scriptsize{4,915}} & \mc{1}{c}{\scriptsize{7,603}} & \mc{1}{c}{\scriptsize{5,020}} & \mc{1}{c}{\scriptsize{4,379}} \\  

     &  & \mc{1}{c}{\scriptsize{(0.294)}} & \mc{1}{c}{\scriptsize{(0.431)}} & \mc{1}{c}{\scriptsize{(0.490)}} & \mc{1}{c}{\scriptsize{(0.706)}} & \mc{1}{c}{\scriptsize{(0.775)}} & \mc{1}{c}{\scriptsize{(0.402)}} & \mc{1}{c}{\scriptsize{(0.637)}} & \mc{1}{c}{\scriptsize{(0.765)}} \\  

     & \mc{1}{c}{\scriptsize{21}} & \mc{1}{c}{\scriptsize{12,732}} & \mc{1}{c}{\scriptsize{12,650}} & \mc{1}{c}{\scriptsize{1,568}} & \mc{1}{c}{\scriptsize{-2,880}} & \mc{1}{c}{\scriptsize{-1,000}} & \mc{1}{c}{\scriptsize{15,124}} & \mc{1}{c}{\scriptsize{17,027}} & \mc{1}{c}{\scriptsize{10,323}} \\  

     &  & \mc{1}{c}{\scriptsize{\textbf{(0.078)}}} & \mc{1}{c}{\scriptsize{(0.216)}} & \mc{1}{c}{\scriptsize{(1.000)}} & \mc{1}{c}{\scriptsize{(0.980)}} & \mc{1}{c}{\scriptsize{(0.882)}} & \mc{1}{c}{\scriptsize{\textbf{(0.020)}}} & \mc{1}{c}{\scriptsize{(0.186)}} & \mc{1}{c}{\scriptsize{(0.333)}} \\  

    \mc{1}{l}{\scriptsize{Parental Income Factor}} & \mc{1}{c}{\scriptsize{1.5 to 21}} & \mc{1}{c}{\scriptsize{-0.078}} & \mc{1}{c}{\scriptsize{-0.222}} & \mc{1}{c}{\scriptsize{0.368}} & \mc{1}{c}{\scriptsize{1.127}} & \mc{1}{c}{\scriptsize{0.363}} & \mc{1}{c}{\scriptsize{-0.125}} & \mc{1}{c}{\scriptsize{-0.271}} & \mc{1}{c}{\scriptsize{-0.122}} \\  

     &  & \mc{1}{c}{\scriptsize{(0.637)}} & \mc{1}{c}{\scriptsize{(0.814)}} & \mc{1}{c}{\scriptsize{(1.000)}} & \mc{1}{c}{\scriptsize{(0.980)}} & \mc{1}{c}{\scriptsize{(0.775)}} & \mc{1}{c}{\scriptsize{(0.676)}} & \mc{1}{c}{\scriptsize{(0.833)}} & \mc{1}{c}{\scriptsize{(0.765)}} \\  

  \bottomrule
  \end{tabular}
	\end{table} 

	\begin{table}[H]
     \caption{Treatment Effects on Classroom Behavior Inventory (Part II), Male Sample}
     \label{table:abccare_rslt_male_cat5_sd}
	  \begin{tabular}{cccccccccc}
  \toprule

    \scriptsize{Variable} & \scriptsize{Age} & \scriptsize{(1)} & \scriptsize{(2)} & \scriptsize{(3)} & \scriptsize{(4)} & \scriptsize{(5)} & \scriptsize{(6)} & \scriptsize{(7)} & \scriptsize{(8)} \\ 
    \midrule  

    \mc{1}{l}{\scriptsize{Mother Works}} & \mc{1}{c}{\scriptsize{2}} & \mc{1}{c}{\scriptsize{0.056}} & \mc{1}{c}{\scriptsize{0.033}} & \mc{1}{c}{\scriptsize{0.264}} & \mc{1}{c}{\scriptsize{0.197}} & \mc{1}{c}{\scriptsize{0.241}} & \mc{1}{c}{\scriptsize{-0.004}} & \mc{1}{c}{\scriptsize{-0.019}} & \mc{1}{c}{\scriptsize{-0.018}} \\  

     &  & \mc{1}{c}{\scriptsize{(0.604)}} & \mc{1}{c}{\scriptsize{(0.762)}} & \mc{1}{c}{\scriptsize{(0.158)}} & \mc{1}{c}{\scriptsize{(0.228)}} & \mc{1}{c}{\scriptsize{(0.267)}} & \mc{1}{c}{\scriptsize{(0.901)}} & \mc{1}{c}{\scriptsize{(0.931)}} & \mc{1}{c}{\scriptsize{(0.950)}} \\  

     & \mc{1}{c}{\scriptsize{3}} & \mc{1}{c}{\scriptsize{0.150}} & \mc{1}{c}{\scriptsize{0.112}} & \mc{1}{c}{\scriptsize{0.261}} & \mc{1}{c}{\scriptsize{0.197}} & \mc{1}{c}{\scriptsize{0.241}} & \mc{1}{c}{\scriptsize{0.116}} & \mc{1}{c}{\scriptsize{0.068}} & \mc{1}{c}{\scriptsize{0.117}} \\  

     &  & \mc{1}{c}{\scriptsize{(0.168)}} & \mc{1}{c}{\scriptsize{(0.376)}} & \mc{1}{c}{\scriptsize{(0.158)}} & \mc{1}{c}{\scriptsize{(0.228)}} & \mc{1}{c}{\scriptsize{(0.267)}} & \mc{1}{c}{\scriptsize{(0.436)}} & \mc{1}{c}{\scriptsize{(0.663)}} & \mc{1}{c}{\scriptsize{(0.436)}} \\  

     & \mc{1}{c}{\scriptsize{4}} & \mc{1}{c}{\scriptsize{0.134}} & \mc{1}{c}{\scriptsize{0.146}} & \mc{1}{c}{\scriptsize{0.287}} & \mc{1}{c}{\scriptsize{0.268}} & \mc{1}{c}{\scriptsize{0.271}} & \mc{1}{c}{\scriptsize{0.090}} & \mc{1}{c}{\scriptsize{0.104}} & \mc{1}{c}{\scriptsize{0.089}} \\  

     &  & \mc{1}{c}{\scriptsize{(0.178)}} & \mc{1}{c}{\scriptsize{(0.218)}} & \mc{1}{c}{\scriptsize{(0.149)}} & \mc{1}{c}{\scriptsize{(0.139)}} & \mc{1}{c}{\scriptsize{(0.248)}} & \mc{1}{c}{\scriptsize{(0.505)}} & \mc{1}{c}{\scriptsize{(0.436)}} & \mc{1}{c}{\scriptsize{(0.475)}} \\  

     & \mc{1}{c}{\scriptsize{5}} & \mc{1}{c}{\scriptsize{0.111}} & \mc{1}{c}{\scriptsize{0.127}} & \mc{1}{c}{\scriptsize{0.311}} & \mc{1}{c}{\scriptsize{0.310}} & \mc{1}{c}{\scriptsize{0.291}} & \mc{1}{c}{\scriptsize{0.061}} & \mc{1}{c}{\scriptsize{0.090}} & \mc{1}{c}{\scriptsize{0.055}} \\  

     &  & \mc{1}{c}{\scriptsize{(0.267)}} & \mc{1}{c}{\scriptsize{(0.307)}} & \mc{1}{c}{\scriptsize{(0.149)}} & \mc{1}{c}{\scriptsize{(0.168)}} & \mc{1}{c}{\scriptsize{(0.208)}} & \mc{1}{c}{\scriptsize{(0.584)}} & \mc{1}{c}{\scriptsize{(0.515)}} & \mc{1}{c}{\scriptsize{(0.713)}} \\  

     & \mc{1}{c}{\scriptsize{21}} & \mc{1}{c}{\scriptsize{-0.058}} & \mc{1}{c}{\scriptsize{-0.005}} & \mc{1}{c}{\scriptsize{-0.086}} & \mc{1}{c}{\scriptsize{-0.131}} & \mc{1}{c}{\scriptsize{-0.139}} & \mc{1}{c}{\scriptsize{-0.036}} & \mc{1}{c}{\scriptsize{0.043}} & \mc{1}{c}{\scriptsize{-0.067}} \\  

     &  & \mc{1}{c}{\scriptsize{(0.960)}} & \mc{1}{c}{\scriptsize{(0.861)}} & \mc{1}{c}{\scriptsize{(0.950)}} & \mc{1}{c}{\scriptsize{(0.594)}} & \mc{1}{c}{\scriptsize{(0.950)}} & \mc{1}{c}{\scriptsize{(0.921)}} & \mc{1}{c}{\scriptsize{(0.772)}} & \mc{1}{c}{\scriptsize{(0.970)}} \\  

    \mc{1}{l}{\scriptsize{Mother Works Factor}} & \mc{1}{c}{\scriptsize{2 to 21}} & \mc{1}{c}{\scriptsize{-0.341}} & \mc{1}{c}{\scriptsize{-0.271}} & \mc{1}{c}{\scriptsize{-0.932}} & \mc{1}{c}{\scriptsize{-0.795}} & \mc{1}{c}{\scriptsize{-0.872}} & \mc{1}{c}{\scriptsize{-0.182}} & \mc{1}{c}{\scriptsize{-0.119}} & \mc{1}{c}{\scriptsize{-0.166}} \\  

     &  & \mc{1}{c}{\scriptsize{(1.000)}} & \mc{1}{c}{\scriptsize{(0.990)}} & \mc{1}{c}{\scriptsize{(1.000)}} & \mc{1}{c}{\scriptsize{(1.000)}} & \mc{1}{c}{\scriptsize{(1.000)}} & \mc{1}{c}{\scriptsize{(1.000)}} & \mc{1}{c}{\scriptsize{(0.960)}} & \mc{1}{c}{\scriptsize{(0.990)}} \\  

  \bottomrule
  \end{tabular}
	\end{table} 

	\begin{table}[H]
     \caption{Treatment Effects on Emotional, Activity, Sociability, Impulsivity Survey, Male Sample}
     \label{table:abccare_rslt_male_cat6_sd}
	  \begin{tabular}{cccccccccc}
  \toprule

    \scriptsize{Variable} & \scriptsize{Age} & \scriptsize{(1)} & \scriptsize{(2)} & \scriptsize{(3)} & \scriptsize{(4)} & \scriptsize{(5)} & \scriptsize{(6)} & \scriptsize{(7)} & \scriptsize{(8)} \\ 
    \midrule  

    \mc{1}{l}{\scriptsize{Graduated High School}} & \mc{1}{c}{\scriptsize{30}} & \mc{1}{c}{\scriptsize{0.073}} & \mc{1}{c}{\scriptsize{0.130}} & \mc{1}{c}{\scriptsize{0.114}} & \mc{1}{c}{\scriptsize{0.186}} & \mc{1}{c}{\scriptsize{0.084}} & \mc{1}{c}{\scriptsize{0.077}} & \mc{1}{c}{\scriptsize{0.136}} & \mc{1}{c}{\scriptsize{0.063}} \\  

     &  & \mc{1}{c}{\scriptsize{(0.629)}} & \mc{1}{c}{\scriptsize{(0.377)}} & \mc{1}{c}{\scriptsize{(1.000)}} & \mc{1}{c}{\scriptsize{(1.000)}} & \mc{1}{c}{\scriptsize{(0.763)}} & \mc{1}{c}{\scriptsize{(0.568)}} & \mc{1}{c}{\scriptsize{(0.377)}} & \mc{1}{c}{\scriptsize{(0.539)}} \\  

    \mc{1}{l}{\scriptsize{Attended Voc./Tech./Com. College}} & \mc{1}{c}{\scriptsize{30}} & \mc{1}{c}{\scriptsize{-0.099}} & \mc{1}{c}{\scriptsize{-0.147}} & \mc{1}{c}{\scriptsize{0.086}} & \mc{1}{c}{\scriptsize{0.188}} & \mc{1}{c}{\scriptsize{0.021}} & \mc{1}{c}{\scriptsize{-0.138}} & \mc{1}{c}{\scriptsize{-0.229}} & \mc{1}{c}{\scriptsize{-0.233}} \\  

     &  & \mc{1}{c}{\scriptsize{(0.629)}} & \mc{1}{c}{\scriptsize{(0.377)}} & \mc{1}{c}{\scriptsize{(1.000)}} & \mc{1}{c}{\scriptsize{(1.000)}} & \mc{1}{c}{\scriptsize{(0.810)}} & \mc{1}{c}{\scriptsize{(0.543)}} & \mc{1}{c}{\scriptsize{(0.324)}} & \mc{1}{c}{\scriptsize{(0.248)}} \\  

    \mc{1}{l}{\scriptsize{Graduated 4-year College}} & \mc{1}{c}{\scriptsize{30}} & \mc{1}{c}{\scriptsize{0.170}} & \mc{1}{c}{\scriptsize{0.178}} & \mc{1}{c}{\scriptsize{0.124}} & \mc{1}{c}{\scriptsize{0.347}} & \mc{1}{c}{\scriptsize{0.100}} & \mc{1}{c}{\scriptsize{0.179}} & \mc{1}{c}{\scriptsize{0.167}} & \mc{1}{c}{\scriptsize{0.142}} \\  

     &  & \mc{1}{c}{\scriptsize{(0.272)}} & \mc{1}{c}{\scriptsize{(0.359)}} & \mc{1}{c}{\scriptsize{(1.000)}} & \mc{1}{c}{\scriptsize{\textbf{(0.070)}}} & \mc{1}{c}{\scriptsize{(0.763)}} & \mc{1}{c}{\scriptsize{(0.300)}} & \mc{1}{c}{\scriptsize{(0.377)}} & \mc{1}{c}{\scriptsize{(0.408)}} \\  

    \mc{1}{l}{\scriptsize{Years of Edu.}} & \mc{1}{c}{\scriptsize{30}} & \mc{1}{c}{\scriptsize{0.525}} & \mc{1}{c}{\scriptsize{0.785}} & \mc{1}{c}{\scriptsize{0.857}} & \mc{1}{c}{\scriptsize{1.619}} & \mc{1}{c}{\scriptsize{0.782}} & \mc{1}{c}{\scriptsize{0.385}} & \mc{1}{c}{\scriptsize{0.649}} & \mc{1}{c}{\scriptsize{0.343}} \\  

     &  & \mc{1}{c}{\scriptsize{(0.540)}} & \mc{1}{c}{\scriptsize{(0.351)}} & \mc{1}{c}{\scriptsize{(1.000)}} & \mc{1}{c}{\scriptsize{(1.000)}} & \mc{1}{c}{\scriptsize{(0.514)}} & \mc{1}{c}{\scriptsize{(0.568)}} & \mc{1}{c}{\scriptsize{(0.377)}} & \mc{1}{c}{\scriptsize{(0.539)}} \\  

    \mc{1}{l}{\scriptsize{Ever Had Special Education by Grade 5}} & \mc{1}{c}{\scriptsize{21}} & \mc{1}{c}{\scriptsize{-0.035}} & \mc{1}{c}{\scriptsize{-0.122}} & \mc{1}{c}{\scriptsize{0.158}} & \mc{1}{c}{\scriptsize{0.033}} & \mc{1}{c}{\scriptsize{0.128}} & \mc{1}{c}{\scriptsize{-0.085}} & \mc{1}{c}{\scriptsize{-0.169}} & \mc{1}{c}{\scriptsize{-0.100}} \\  

     &  & \mc{1}{c}{\scriptsize{(0.629)}} & \mc{1}{c}{\scriptsize{(0.377)}} & \mc{1}{c}{\scriptsize{(1.000)}} & \mc{1}{c}{\scriptsize{(1.000)}} & \mc{1}{c}{\scriptsize{(0.763)}} & \mc{1}{c}{\scriptsize{(0.568)}} & \mc{1}{c}{\scriptsize{(0.377)}} & \mc{1}{c}{\scriptsize{(0.539)}} \\  

    \mc{1}{l}{\scriptsize{Total Number of Special Education by Grade 5}} & \mc{1}{c}{\scriptsize{21}} & \mc{1}{c}{\scriptsize{-0.544}} & \mc{1}{c}{\scriptsize{-1.204}} & \mc{1}{c}{\scriptsize{0.019}} & \mc{1}{c}{\scriptsize{-1.713}} & \mc{1}{c}{\scriptsize{0.154}} & \mc{1}{c}{\scriptsize{-0.690}} & \mc{1}{c}{\scriptsize{-1.185}} & \mc{1}{c}{\scriptsize{-0.459}} \\  

     &  & \mc{1}{c}{\scriptsize{(0.629)}} & \mc{1}{c}{\scriptsize{(0.351)}} & \mc{1}{c}{\scriptsize{(1.000)}} & \mc{1}{c}{\scriptsize{(1.000)}} & \mc{1}{c}{\scriptsize{(0.810)}} & \mc{1}{c}{\scriptsize{(0.568)}} & \mc{1}{c}{\scriptsize{(0.377)}} & \mc{1}{c}{\scriptsize{(0.539)}} \\  

    \mc{1}{l}{\scriptsize{Ever Retained by Grade 5}} & \mc{1}{c}{\scriptsize{21}} & \mc{1}{c}{\scriptsize{-0.095}} & \mc{1}{c}{\scriptsize{-0.213}} & \mc{1}{c}{\scriptsize{-0.023}} & \mc{1}{c}{\scriptsize{-0.238}} & \mc{1}{c}{\scriptsize{-0.061}} & \mc{1}{c}{\scriptsize{-0.113}} & \mc{1}{c}{\scriptsize{-0.206}} & \mc{1}{c}{\scriptsize{-0.154}} \\  

     &  & \mc{1}{c}{\scriptsize{(0.622)}} & \mc{1}{c}{\scriptsize{(0.233)}} & \mc{1}{c}{\scriptsize{(1.000)}} & \mc{1}{c}{\scriptsize{(1.000)}} & \mc{1}{c}{\scriptsize{(0.763)}} & \mc{1}{c}{\scriptsize{(0.543)}} & \mc{1}{c}{\scriptsize{(0.317)}} & \mc{1}{c}{\scriptsize{(0.415)}} \\  

    \mc{1}{l}{\scriptsize{Total Number of Retention by Grade 5}} & \mc{1}{c}{\scriptsize{21}} & \mc{1}{c}{\scriptsize{-0.070}} & \mc{1}{c}{\scriptsize{-0.197}} & \mc{1}{c}{\scriptsize{0.031}} & \mc{1}{c}{\scriptsize{-0.215}} & \mc{1}{c}{\scriptsize{0.006}} & \mc{1}{c}{\scriptsize{-0.096}} & \mc{1}{c}{\scriptsize{-0.190}} & \mc{1}{c}{\scriptsize{-0.127}} \\  

     &  & \mc{1}{c}{\scriptsize{(0.629)}} & \mc{1}{c}{\scriptsize{(0.359)}} & \mc{1}{c}{\scriptsize{(1.000)}} & \mc{1}{c}{\scriptsize{(1.000)}} & \mc{1}{c}{\scriptsize{(0.810)}} & \mc{1}{c}{\scriptsize{(0.568)}} & \mc{1}{c}{\scriptsize{(0.377)}} & \mc{1}{c}{\scriptsize{(0.539)}} \\  

    \mc{1}{l}{\scriptsize{Education Factor}} & \mc{1}{c}{\scriptsize{21 to 30}} & \mc{1}{c}{\scriptsize{0.344}} & \mc{1}{c}{\scriptsize{0.564}} & \mc{1}{c}{\scriptsize{0.230}} & \mc{1}{c}{\scriptsize{1.020}} & \mc{1}{c}{\scriptsize{0.222}} & \mc{1}{c}{\scriptsize{0.385}} & \mc{1}{c}{\scriptsize{0.485}} & \mc{1}{c}{\scriptsize{0.375}} \\  

     &  & \mc{1}{c}{\scriptsize{(0.444)}} & \mc{1}{c}{\scriptsize{(0.189)}} & \mc{1}{c}{\scriptsize{(1.000)}} & \mc{1}{c}{\scriptsize{(1.000)}} & \mc{1}{c}{\scriptsize{(0.763)}} & \mc{1}{c}{\scriptsize{(0.408)}} & \mc{1}{c}{\scriptsize{(0.266)}} & \mc{1}{c}{\scriptsize{(0.389)}} \\  

  \bottomrule
  \end{tabular}
	\end{table} 

	\begin{table}[H]
     \caption{Treatment Effects on Harter Importance, Male Sample}
     \label{table:abccare_rslt_male_cat7_sd}
	  \begin{tabular}{cccccccccc}
  \toprule

    \scriptsize{Variable} & \scriptsize{Age} & \scriptsize{(1)} & \scriptsize{(2)} & \scriptsize{(3)} & \scriptsize{(4)} & \scriptsize{(5)} & \scriptsize{(6)} & \scriptsize{(7)} & \scriptsize{(8)} \\ 
    \midrule  

    \mc{1}{l}{\scriptsize{Employed}} & \mc{1}{c}{\scriptsize{30}} & \mc{1}{c}{\scriptsize{0.190}} & \mc{1}{c}{\scriptsize{0.253}} & \mc{1}{c}{\scriptsize{0.162}} & \mc{1}{c}{\scriptsize{0.147}} & \mc{1}{c}{\scriptsize{0.230}} & \mc{1}{c}{\scriptsize{0.211}} & \mc{1}{c}{\scriptsize{0.296}} & \mc{1}{c}{\scriptsize{0.291}} \\  

     &  & \mc{1}{c}{\scriptsize{(0.118)}} & \mc{1}{c}{\scriptsize{\textbf{(0.039)}}} & \mc{1}{c}{\scriptsize{(0.421)}} & \mc{1}{c}{\scriptsize{(0.539)}} & \mc{1}{c}{\scriptsize{(0.263)}} & \mc{1}{c}{\scriptsize{(0.184)}} & \mc{1}{c}{\scriptsize{\textbf{(0.066)}}} & \mc{1}{c}{\scriptsize{\textbf{(0.026)}}} \\  

    \mc{1}{l}{\scriptsize{Labor Income}} & \mc{1}{c}{\scriptsize{21}} & \mc{1}{c}{\scriptsize{-837}} & \mc{1}{c}{\scriptsize{-2,170}} & \mc{1}{c}{\scriptsize{-4,421}} & \mc{1}{c}{\scriptsize{-7,907}} & \mc{1}{c}{\scriptsize{-4,409}} & \mc{1}{c}{\scriptsize{235}} & \mc{1}{c}{\scriptsize{-1,111}} & \mc{1}{c}{\scriptsize{-953}} \\  

     &  & \mc{1}{c}{\scriptsize{(1.000)}} & \mc{1}{c}{\scriptsize{(1.000)}} & \mc{1}{c}{\scriptsize{(1.000)}} & \mc{1}{c}{\scriptsize{(1.000)}} & \mc{1}{c}{\scriptsize{(1.000)}} & \mc{1}{c}{\scriptsize{(0.882)}} & \mc{1}{c}{\scriptsize{(1.000)}} & \mc{1}{c}{\scriptsize{(1.000)}} \\  

     & \mc{1}{c}{\scriptsize{30}} & \mc{1}{c}{\scriptsize{10,070}} & \mc{1}{c}{\scriptsize{19,026}} & \mc{1}{c}{\scriptsize{10,171}} & \mc{1}{c}{\scriptsize{23,093}} & \mc{1}{c}{\scriptsize{14,235}} & \mc{1}{c}{\scriptsize{9,995}} & \mc{1}{c}{\scriptsize{20,129}} & \mc{1}{c}{\scriptsize{14,808}} \\  

     &  & \mc{1}{c}{\scriptsize{(0.316)}} & \mc{1}{c}{\scriptsize{(0.263)}} & \mc{1}{c}{\scriptsize{(0.224)}} & \mc{1}{c}{\scriptsize{(0.553)}} & \mc{1}{c}{\scriptsize{(0.408)}} & \mc{1}{c}{\scriptsize{(0.329)}} & \mc{1}{c}{\scriptsize{(0.289)}} & \mc{1}{c}{\scriptsize{(0.342)}} \\  

    \mc{1}{l}{\scriptsize{Public-Transfer Income}} & \mc{1}{c}{\scriptsize{21}} & \mc{1}{c}{\scriptsize{95.964}} & \mc{1}{c}{\scriptsize{-255}} & \mc{1}{c}{\scriptsize{243}} & \mc{1}{c}{\scriptsize{-240}} & \mc{1}{c}{\scriptsize{-384}} & \mc{1}{c}{\scriptsize{11.191}} & \mc{1}{c}{\scriptsize{-139}} & \mc{1}{c}{\scriptsize{155}} \\  

     &  & \mc{1}{c}{\scriptsize{(0.974)}} & \mc{1}{c}{\scriptsize{(0.934)}} & \mc{1}{c}{\scriptsize{(0.961)}} & \mc{1}{c}{\scriptsize{(0.947)}} & \mc{1}{c}{\scriptsize{(0.829)}} & \mc{1}{c}{\scriptsize{(0.934)}} & \mc{1}{c}{\scriptsize{(0.947)}} & \mc{1}{c}{\scriptsize{(0.987)}} \\  

     & \mc{1}{c}{\scriptsize{30}} & \mc{1}{c}{\scriptsize{-422}} & \mc{1}{c}{\scriptsize{-251}} & \mc{1}{c}{\scriptsize{-269}} & \mc{1}{c}{\scriptsize{-184}} & \mc{1}{c}{\scriptsize{-157}} & \mc{1}{c}{\scriptsize{-343}} & \mc{1}{c}{\scriptsize{-269}} & \mc{1}{c}{\scriptsize{-233}} \\  

     &  & \mc{1}{c}{\scriptsize{(0.224)}} & \mc{1}{c}{\scriptsize{(0.632)}} & \mc{1}{c}{\scriptsize{(0.684)}} & \mc{1}{c}{\scriptsize{(0.895)}} & \mc{1}{c}{\scriptsize{(0.855)}} & \mc{1}{c}{\scriptsize{(0.434)}} & \mc{1}{c}{\scriptsize{(0.671)}} & \mc{1}{c}{\scriptsize{(0.776)}} \\  

    \mc{1}{l}{\scriptsize{Employment Factor}} & \mc{1}{c}{\scriptsize{21 to 30}} & \mc{1}{c}{\scriptsize{0.358}} & \mc{1}{c}{\scriptsize{0.444}} & \mc{1}{c}{\scriptsize{0.252}} & \mc{1}{c}{\scriptsize{0.221}} & \mc{1}{c}{\scriptsize{0.289}} & \mc{1}{c}{\scriptsize{0.382}} & \mc{1}{c}{\scriptsize{0.475}} & \mc{1}{c}{\scriptsize{0.422}} \\  

     &  & \mc{1}{c}{\scriptsize{(0.329)}} & \mc{1}{c}{\scriptsize{(0.224)}} & \mc{1}{c}{\scriptsize{(0.632)}} & \mc{1}{c}{\scriptsize{(0.895)}} & \mc{1}{c}{\scriptsize{(0.684)}} & \mc{1}{c}{\scriptsize{(0.329)}} & \mc{1}{c}{\scriptsize{(0.197)}} & \mc{1}{c}{\scriptsize{(0.211)}} \\  

  \bottomrule
  \end{tabular}
	\end{table} 

	\begin{table}[H]
     \caption{Treatment Effects on Achenbach Behavior, Male Sample}
     \label{table:abccare_rslt_male_cat8_sd}
	  \begin{tabular}{cccccccccc}
  \toprule

    \scriptsize{Variable} & \scriptsize{Age} & \scriptsize{(1)} & \scriptsize{(2)} & \scriptsize{(3)} & \scriptsize{(4)} & \scriptsize{(5)} & \scriptsize{(6)} & \scriptsize{(7)} & \scriptsize{(8)} \\ 
    \midrule  

    \mc{1}{l}{\scriptsize{Total Felony Arrests}} & \mc{1}{c}{\scriptsize{Mid-30s}} & \mc{1}{c}{\scriptsize{0.145}} & \mc{1}{c}{\scriptsize{1.611}} & \mc{1}{c}{\scriptsize{0.517}} & \mc{1}{c}{\scriptsize{2.767}} & \mc{1}{c}{\scriptsize{1.802}} & \mc{1}{c}{\scriptsize{0.051}} & \mc{1}{c}{\scriptsize{1.692}} & \mc{1}{c}{\scriptsize{0.980}} \\  

     &  & \mc{1}{c}{\scriptsize{(0.868)}} & \mc{1}{c}{\scriptsize{(1.000)}} & \mc{1}{c}{\scriptsize{(0.921)}} & \mc{1}{c}{\scriptsize{(1.000)}} & \mc{1}{c}{\scriptsize{(1.000)}} & \mc{1}{c}{\scriptsize{(0.737)}} & \mc{1}{c}{\scriptsize{(0.974)}} & \mc{1}{c}{\scriptsize{(1.000)}} \\  

    \mc{1}{l}{\scriptsize{Total Misdemeanor Arrests}} & \mc{1}{c}{\scriptsize{Mid-30s}} & \mc{1}{c}{\scriptsize{-0.686}} & \mc{1}{c}{\scriptsize{-0.578}} & \mc{1}{c}{\scriptsize{-0.882}} & \mc{1}{c}{\scriptsize{-0.412}} & \mc{1}{c}{\scriptsize{-0.609}} & \mc{1}{c}{\scriptsize{-0.665}} & \mc{1}{c}{\scriptsize{-0.335}} & \mc{1}{c}{\scriptsize{-0.564}} \\  

     &  & \mc{1}{c}{\scriptsize{(0.105)}} & \mc{1}{c}{\scriptsize{(0.197)}} & \mc{1}{c}{\scriptsize{(0.276)}} & \mc{1}{c}{\scriptsize{(0.645)}} & \mc{1}{c}{\scriptsize{(0.474)}} & \mc{1}{c}{\scriptsize{\textbf{(0.066)}}} & \mc{1}{c}{\scriptsize{(0.408)}} & \mc{1}{c}{\scriptsize{(0.237)}} \\  

    \mc{1}{l}{\scriptsize{Total Years Incarcerated}} & \mc{1}{c}{\scriptsize{30}} & \mc{1}{c}{\scriptsize{0.106}} & \mc{1}{c}{\scriptsize{0.364}} & \mc{1}{c}{\scriptsize{0.419}} & \mc{1}{c}{\scriptsize{0.726}} & \mc{1}{c}{\scriptsize{0.557}} & \mc{1}{c}{\scriptsize{0.014}} & \mc{1}{c}{\scriptsize{0.370}} & \mc{1}{c}{\scriptsize{0.134}} \\  

     &  & \mc{1}{c}{\scriptsize{(0.908)}} & \mc{1}{c}{\scriptsize{(1.000)}} & \mc{1}{c}{\scriptsize{(1.000)}} & \mc{1}{c}{\scriptsize{(1.000)}} & \mc{1}{c}{\scriptsize{(1.000)}} & \mc{1}{c}{\scriptsize{(0.724)}} & \mc{1}{c}{\scriptsize{(0.961)}} & \mc{1}{c}{\scriptsize{(0.882)}} \\  

    \mc{1}{l}{\scriptsize{Crime Factor}} & \mc{1}{c}{\scriptsize{30 to Mid-30s}} & \mc{1}{c}{\scriptsize{-0.001}} & \mc{1}{c}{\scriptsize{0.180}} & \mc{1}{c}{\scriptsize{0.168}} & \mc{1}{c}{\scriptsize{0.626}} & \mc{1}{c}{\scriptsize{0.230}} & \mc{1}{c}{\scriptsize{-0.050}} & \mc{1}{c}{\scriptsize{0.284}} & \mc{1}{c}{\scriptsize{0.016}} \\  

     &  & \mc{1}{c}{\scriptsize{(0.803)}} & \mc{1}{c}{\scriptsize{(0.921)}} & \mc{1}{c}{\scriptsize{(0.895)}} & \mc{1}{c}{\scriptsize{(0.974)}} & \mc{1}{c}{\scriptsize{(0.934)}} & \mc{1}{c}{\scriptsize{(0.645)}} & \mc{1}{c}{\scriptsize{(0.947)}} & \mc{1}{c}{\scriptsize{(0.789)}} \\  

  \bottomrule
  \end{tabular}
	\end{table} 

	\begin{table}[H]
     \caption{Treatment Effects on Achenbach Symptom T Score (Reported by Mother), Male Sample}
     \label{table:abccare_rslt_male_cat9_sd}
	  \begin{tabular}{cccccccccc}
  \toprule

    \scriptsize{Variable} & \scriptsize{Age} & \scriptsize{(1)} & \scriptsize{(2)} & \scriptsize{(3)} & \scriptsize{(4)} & \scriptsize{(5)} & \scriptsize{(6)} & \scriptsize{(7)} & \scriptsize{(8)} \\ 
    \midrule  

    \mc{1}{l}{\scriptsize{Aggressive}} & \mc{1}{c}{\scriptsize{8}} & \mc{1}{c}{\scriptsize{-1.585}} & \mc{1}{c}{\scriptsize{-2.228}} & \mc{1}{c}{\scriptsize{-1.995}} & \mc{1}{c}{\scriptsize{-3.574}} & \mc{1}{c}{\scriptsize{-1.478}} & \mc{1}{c}{\scriptsize{-0.543}} & \mc{1}{c}{\scriptsize{-1.744}} & \mc{1}{c}{\scriptsize{-1.896}} \\  

     &  & \mc{1}{c}{\scriptsize{(0.605)}} & \mc{1}{c}{\scriptsize{(0.513)}} & \mc{1}{c}{\scriptsize{(0.787)}} & \mc{1}{c}{\scriptsize{(0.667)}} & \mc{1}{c}{\scriptsize{(0.840)}} & \mc{1}{c}{\scriptsize{(0.908)}} & \mc{1}{c}{\scriptsize{(0.697)}} & \mc{1}{c}{\scriptsize{(0.592)}} \\  

     & \mc{1}{c}{\scriptsize{12}} & \mc{1}{c}{\scriptsize{-1.629}} & \mc{1}{c}{\scriptsize{-1.782}} & \mc{1}{c}{\scriptsize{-5.914}} & \mc{1}{c}{\scriptsize{-6.825}} & \mc{1}{c}{\scriptsize{-5.968}} & \mc{1}{c}{\scriptsize{-0.557}} & \mc{1}{c}{\scriptsize{-1.012}} & \mc{1}{c}{\scriptsize{-0.913}} \\  

     &  & \mc{1}{c}{\scriptsize{(0.461)}} & \mc{1}{c}{\scriptsize{(0.513)}} & \mc{1}{c}{\scriptsize{(0.280)}} & \mc{1}{c}{\scriptsize{(0.333)}} & \mc{1}{c}{\scriptsize{(0.280)}} & \mc{1}{c}{\scriptsize{(0.868)}} & \mc{1}{c}{\scriptsize{(0.803)}} & \mc{1}{c}{\scriptsize{(0.803)}} \\  

    \mc{1}{l}{\scriptsize{Delinquent}} & \mc{1}{c}{\scriptsize{8}} & \mc{1}{c}{\scriptsize{-1.580}} & \mc{1}{c}{\scriptsize{-2.801}} & \mc{1}{c}{\scriptsize{-0.834}} & \mc{1}{c}{\scriptsize{-1.676}} & \mc{1}{c}{\scriptsize{-0.887}} & \mc{1}{c}{\scriptsize{-0.923}} & \mc{1}{c}{\scriptsize{-2.585}} & \mc{1}{c}{\scriptsize{-1.846}} \\  

     &  & \mc{1}{c}{\scriptsize{(0.539)}} & \mc{1}{c}{\scriptsize{(0.342)}} & \mc{1}{c}{\scriptsize{(0.813)}} & \mc{1}{c}{\scriptsize{(0.773)}} & \mc{1}{c}{\scriptsize{(0.840)}} & \mc{1}{c}{\scriptsize{(0.789)}} & \mc{1}{c}{\scriptsize{(0.434)}} & \mc{1}{c}{\scriptsize{(0.566)}} \\  

     & \mc{1}{c}{\scriptsize{12}} & \mc{1}{c}{\scriptsize{-0.714}} & \mc{1}{c}{\scriptsize{-0.964}} & \mc{1}{c}{\scriptsize{-3.000}} & \mc{1}{c}{\scriptsize{-3.982}} & \mc{1}{c}{\scriptsize{-2.940}} & \mc{1}{c}{\scriptsize{-0.143}} & \mc{1}{c}{\scriptsize{-0.512}} & \mc{1}{c}{\scriptsize{-0.162}} \\  

     &  & \mc{1}{c}{\scriptsize{(0.842)}} & \mc{1}{c}{\scriptsize{(0.816)}} & \mc{1}{c}{\scriptsize{(0.453)}} & \mc{1}{c}{\scriptsize{(0.413)}} & \mc{1}{c}{\scriptsize{(0.467)}} & \mc{1}{c}{\scriptsize{(0.987)}} & \mc{1}{c}{\scriptsize{(0.921)}} & \mc{1}{c}{\scriptsize{(0.947)}} \\  

    \mc{1}{l}{\scriptsize{Depressed}} & \mc{1}{c}{\scriptsize{8}} & \mc{1}{c}{\scriptsize{-1.335}} & \mc{1}{c}{\scriptsize{-0.227}} & \mc{1}{c}{\scriptsize{-1.995}} & \mc{1}{c}{\scriptsize{-2.456}} & \mc{1}{c}{\scriptsize{-1.517}} & \mc{1}{c}{\scriptsize{-0.501}} & \mc{1}{c}{\scriptsize{0.244}} & \mc{1}{c}{\scriptsize{-0.586}} \\  

     &  & \mc{1}{c}{\scriptsize{(0.566)}} & \mc{1}{c}{\scriptsize{(0.974)}} & \mc{1}{c}{\scriptsize{(0.760)}} & \mc{1}{c}{\scriptsize{(0.693)}} & \mc{1}{c}{\scriptsize{(0.813)}} & \mc{1}{c}{\scriptsize{(0.895)}} & \mc{1}{c}{\scriptsize{(0.974)}} & \mc{1}{c}{\scriptsize{(0.895)}} \\  

    \mc{1}{l}{\scriptsize{Externalizing}} & \mc{1}{c}{\scriptsize{8}} & \mc{1}{c}{\scriptsize{-2.223}} & \mc{1}{c}{\scriptsize{-3.192}} & \mc{1}{c}{\scriptsize{-0.986}} & \mc{1}{c}{\scriptsize{-0.095}} & \mc{1}{c}{\scriptsize{-1.164}} & \mc{1}{c}{\scriptsize{-1.671}} & \mc{1}{c}{\scriptsize{-3.539}} & \mc{1}{c}{\scriptsize{-3.212}} \\  

     &  & \mc{1}{c}{\scriptsize{(0.632)}} & \mc{1}{c}{\scriptsize{(0.526)}} & \mc{1}{c}{\scriptsize{(0.867)}} & \mc{1}{c}{\scriptsize{(0.920)}} & \mc{1}{c}{\scriptsize{(0.893)}} & \mc{1}{c}{\scriptsize{(0.763)}} & \mc{1}{c}{\scriptsize{(0.368)}} & \mc{1}{c}{\scriptsize{(0.487)}} \\  

     & \mc{1}{c}{\scriptsize{12}} & \mc{1}{c}{\scriptsize{-1.257}} & \mc{1}{c}{\scriptsize{-1.757}} & \mc{1}{c}{\scriptsize{-5.743}} & \mc{1}{c}{\scriptsize{-4.919}} & \mc{1}{c}{\scriptsize{-5.654}} & \mc{1}{c}{\scriptsize{-0.136}} & \mc{1}{c}{\scriptsize{-1.263}} & \mc{1}{c}{\scriptsize{-0.706}} \\  

     &  & \mc{1}{c}{\scriptsize{(0.829)}} & \mc{1}{c}{\scriptsize{(0.816)}} & \mc{1}{c}{\scriptsize{(0.240)}} & \mc{1}{c}{\scriptsize{(0.507)}} & \mc{1}{c}{\scriptsize{(0.280)}} & \mc{1}{c}{\scriptsize{(0.987)}} & \mc{1}{c}{\scriptsize{(0.868)}} & \mc{1}{c}{\scriptsize{(0.908)}} \\  

    \mc{1}{l}{\scriptsize{Hyperactive}} & \mc{1}{c}{\scriptsize{8}} & \mc{1}{c}{\scriptsize{-4.081}} & \mc{1}{c}{\scriptsize{-4.232}} & \mc{1}{c}{\scriptsize{-5.152}} & \mc{1}{c}{\scriptsize{-5.363}} & \mc{1}{c}{\scriptsize{-4.294}} & \mc{1}{c}{\scriptsize{-2.872}} & \mc{1}{c}{\scriptsize{-3.682}} & \mc{1}{c}{\scriptsize{-2.499}} \\  

     &  & \mc{1}{c}{\scriptsize{\textbf{(0.026)}}} & \mc{1}{c}{\scriptsize{\textbf{(0.000)}}} & \mc{1}{c}{\scriptsize{(0.280)}} & \mc{1}{c}{\scriptsize{(0.413)}} & \mc{1}{c}{\scriptsize{(0.387)}} & \mc{1}{c}{\scriptsize{\textbf{(0.092)}}} & \mc{1}{c}{\scriptsize{\textbf{(0.000)}}} & \mc{1}{c}{\scriptsize{(0.250)}} \\  

    \mc{1}{l}{\scriptsize{Internalizing}} & \mc{1}{c}{\scriptsize{8}} & \mc{1}{c}{\scriptsize{-1.743}} & \mc{1}{c}{\scriptsize{-2.423}} & \mc{1}{c}{\scriptsize{1.083}} & \mc{1}{c}{\scriptsize{-0.221}} & \mc{1}{c}{\scriptsize{0.487}} & \mc{1}{c}{\scriptsize{-1.607}} & \mc{1}{c}{\scriptsize{-2.608}} & \mc{1}{c}{\scriptsize{-3.196}} \\  

     &  & \mc{1}{c}{\scriptsize{(0.776)}} & \mc{1}{c}{\scriptsize{(0.711)}} & \mc{1}{c}{\scriptsize{(0.947)}} & \mc{1}{c}{\scriptsize{(0.920)}} & \mc{1}{c}{\scriptsize{(0.947)}} & \mc{1}{c}{\scriptsize{(0.776)}} & \mc{1}{c}{\scriptsize{(0.671)}} & \mc{1}{c}{\scriptsize{(0.553)}} \\  

     & \mc{1}{c}{\scriptsize{12}} & \mc{1}{c}{\scriptsize{-1.286}} & \mc{1}{c}{\scriptsize{-2.591}} & \mc{1}{c}{\scriptsize{-4.314}} & \mc{1}{c}{\scriptsize{-6.276}} & \mc{1}{c}{\scriptsize{-4.775}} & \mc{1}{c}{\scriptsize{-0.529}} & \mc{1}{c}{\scriptsize{-1.964}} & \mc{1}{c}{\scriptsize{-1.603}} \\  

     &  & \mc{1}{c}{\scriptsize{(0.776)}} & \mc{1}{c}{\scriptsize{(0.500)}} & \mc{1}{c}{\scriptsize{(0.160)}} & \mc{1}{c}{\scriptsize{(0.253)}} & \mc{1}{c}{\scriptsize{(0.107)}} & \mc{1}{c}{\scriptsize{(0.934)}} & \mc{1}{c}{\scriptsize{(0.697)}} & \mc{1}{c}{\scriptsize{(0.737)}} \\  

    \mc{1}{l}{\scriptsize{Schizoid}} & \mc{1}{c}{\scriptsize{12}} & \mc{1}{c}{\scriptsize{0.371}} & \mc{1}{c}{\scriptsize{-0.366}} & \mc{1}{c}{\scriptsize{0.314}} & \mc{1}{c}{\scriptsize{-2.364}} & \mc{1}{c}{\scriptsize{-0.062}} & \mc{1}{c}{\scriptsize{0.386}} & \mc{1}{c}{\scriptsize{-0.078}} & \mc{1}{c}{\scriptsize{-0.031}} \\  

     &  & \mc{1}{c}{\scriptsize{(1.000)}} & \mc{1}{c}{\scriptsize{(0.974)}} & \mc{1}{c}{\scriptsize{(0.947)}} & \mc{1}{c}{\scriptsize{(0.547)}} & \mc{1}{c}{\scriptsize{(0.933)}} & \mc{1}{c}{\scriptsize{(0.987)}} & \mc{1}{c}{\scriptsize{(0.974)}} & \mc{1}{c}{\scriptsize{(0.961)}} \\  

    \mc{1}{l}{\scriptsize{Somatic Complaints}} & \mc{1}{c}{\scriptsize{8}} & \mc{1}{c}{\scriptsize{-2.169}} & \mc{1}{c}{\scriptsize{-1.312}} & \mc{1}{c}{\scriptsize{-1.562}} & \mc{1}{c}{\scriptsize{-0.444}} & \mc{1}{c}{\scriptsize{-1.794}} & \mc{1}{c}{\scriptsize{-1.836}} & \mc{1}{c}{\scriptsize{-1.163}} & \mc{1}{c}{\scriptsize{-1.975}} \\  

     &  & \mc{1}{c}{\scriptsize{(0.329)}} & \mc{1}{c}{\scriptsize{(0.737)}} & \mc{1}{c}{\scriptsize{(0.787)}} & \mc{1}{c}{\scriptsize{(0.920)}} & \mc{1}{c}{\scriptsize{(0.773)}} & \mc{1}{c}{\scriptsize{(0.461)}} & \mc{1}{c}{\scriptsize{(0.763)}} & \mc{1}{c}{\scriptsize{(0.382)}} \\  

     & \mc{1}{c}{\scriptsize{12}} & \mc{1}{c}{\scriptsize{-0.143}} & \mc{1}{c}{\scriptsize{-1.025}} & \mc{1}{c}{\scriptsize{0.171}} & \mc{1}{c}{\scriptsize{0.067}} & \mc{1}{c}{\scriptsize{-0.103}} & \mc{1}{c}{\scriptsize{-0.221}} & \mc{1}{c}{\scriptsize{-1.190}} & \mc{1}{c}{\scriptsize{-1.015}} \\  

     &  & \mc{1}{c}{\scriptsize{(0.987)}} & \mc{1}{c}{\scriptsize{(0.868)}} & \mc{1}{c}{\scriptsize{(0.947)}} & \mc{1}{c}{\scriptsize{(0.920)}} & \mc{1}{c}{\scriptsize{(0.933)}} & \mc{1}{c}{\scriptsize{(0.974)}} & \mc{1}{c}{\scriptsize{(0.803)}} & \mc{1}{c}{\scriptsize{(0.803)}} \\  

  \bottomrule
  \end{tabular}
	\end{table} 

	\begin{table}[H]
     \caption{Treatment Effects on Achenbach Symptom T Score (Reported by Teacher), Male Sample}
     \label{table:abccare_rslt_male_cat10_sd}
	  \begin{tabular}{cccccccccc}
  \toprule

    \scriptsize{Variable} & \scriptsize{Age} & \scriptsize{(1)} & \scriptsize{(2)} & \scriptsize{(3)} & \scriptsize{(4)} & \scriptsize{(5)} & \scriptsize{(6)} & \scriptsize{(7)} & \scriptsize{(8)} \\ 
    \midrule  

    \mc{1}{l}{\scriptsize{Self-reported Health}} & \mc{1}{c}{\scriptsize{30}} & \mc{1}{c}{\scriptsize{0.092}} & \mc{1}{c}{\scriptsize{0.292}} & \mc{1}{c}{\scriptsize{0.362}} & \mc{1}{c}{\scriptsize{0.753}} & \mc{1}{c}{\scriptsize{0.470}} & \mc{1}{c}{\scriptsize{-0.001}} & \mc{1}{c}{\scriptsize{0.201}} & \mc{1}{c}{\scriptsize{0.135}} \\  

     &  & \mc{1}{c}{\scriptsize{(0.961)}} & \mc{1}{c}{\scriptsize{(1.000)}} & \mc{1}{c}{\scriptsize{(1.000)}} & \mc{1}{c}{\scriptsize{(1.000)}} & \mc{1}{c}{\scriptsize{(1.000)}} & \mc{1}{c}{\scriptsize{(0.855)}} & \mc{1}{c}{\scriptsize{(0.987)}} & \mc{1}{c}{\scriptsize{(1.000)}} \\  

     & \mc{1}{c}{\scriptsize{Mid-30s}} & \mc{1}{c}{\scriptsize{0.345}} & \mc{1}{c}{\scriptsize{0.073}} & \mc{1}{c}{\scriptsize{-0.018}} & \mc{1}{c}{\scriptsize{-0.784}} & \mc{1}{c}{\scriptsize{-0.140}} & \mc{1}{c}{\scriptsize{0.415}} & \mc{1}{c}{\scriptsize{0.333}} & \mc{1}{c}{\scriptsize{0.400}} \\  

     &  & \mc{1}{c}{\scriptsize{(1.000)}} & \mc{1}{c}{\scriptsize{(0.855)}} & \mc{1}{c}{\scriptsize{(0.895)}} & \mc{1}{c}{\scriptsize{(0.395)}} & \mc{1}{c}{\scriptsize{(0.763)}} & \mc{1}{c}{\scriptsize{(1.000)}} & \mc{1}{c}{\scriptsize{(0.987)}} & \mc{1}{c}{\scriptsize{(1.000)}} \\  

    \mc{1}{l}{\scriptsize{Self-reported Health Factor}} & \mc{1}{c}{\scriptsize{30 to Mid-30s}} & \mc{1}{c}{\scriptsize{0.082}} & \mc{1}{c}{\scriptsize{-0.094}} & \mc{1}{c}{\scriptsize{-0.070}} & \mc{1}{c}{\scriptsize{-0.626}} & \mc{1}{c}{\scriptsize{-0.097}} & \mc{1}{c}{\scriptsize{0.110}} & \mc{1}{c}{\scriptsize{0.018}} & \mc{1}{c}{\scriptsize{0.085}} \\  

     &  & \mc{1}{c}{\scriptsize{(0.961)}} & \mc{1}{c}{\scriptsize{(0.684)}} & \mc{1}{c}{\scriptsize{(0.829)}} & \mc{1}{c}{\scriptsize{(0.329)}} & \mc{1}{c}{\scriptsize{(0.763)}} & \mc{1}{c}{\scriptsize{(0.961)}} & \mc{1}{c}{\scriptsize{(0.882)}} & \mc{1}{c}{\scriptsize{(0.947)}} \\  

  \bottomrule
  \end{tabular}
	\end{table} 

	\begin{table}[H]
     \caption{Treatment Effects on Child Assessment Schedule (CAS), Male Sample}
     \label{table:abccare_rslt_male_cat11_sd}
	  \begin{tabular}{cccccccccc}
  \toprule

    \scriptsize{Variable} & \scriptsize{Age} & \scriptsize{(1)} & \scriptsize{(2)} & \scriptsize{(3)} & \scriptsize{(4)} & \scriptsize{(5)} & \scriptsize{(6)} & \scriptsize{(7)} & \scriptsize{(8)} \\ 
    \midrule  

    \mc{1}{l}{\scriptsize{Systolic Blood Pressure (mm Hg)}} & \mc{1}{c}{\scriptsize{Mid-30s}} & \mc{1}{c}{\scriptsize{-5.863}} & \mc{1}{c}{\scriptsize{-8.467}} & \mc{1}{c}{\scriptsize{6.947}} & \mc{1}{c}{\scriptsize{10.664}} & \mc{1}{c}{\scriptsize{7.441}} & \mc{1}{c}{\scriptsize{-12.934}} & \mc{1}{c}{\scriptsize{-18.465}} & \mc{1}{c}{\scriptsize{-14.412}} \\  

     &  & \mc{1}{c}{\scriptsize{(0.329)}} & \mc{1}{c}{\scriptsize{(0.316)}} & \mc{1}{c}{\scriptsize{(1.000)}} & \mc{1}{c}{\scriptsize{(0.987)}} & \mc{1}{c}{\scriptsize{(0.987)}} & \mc{1}{c}{\scriptsize{\textbf{(0.092)}}} & \mc{1}{c}{\scriptsize{\textbf{(0.026)}}} & \mc{1}{c}{\scriptsize{\textbf{(0.053)}}} \\  

    \mc{1}{l}{\scriptsize{Diastolic Blood Pressure (mm Hg)}} & \mc{1}{c}{\scriptsize{Mid-30s}} & \mc{1}{c}{\scriptsize{-9.116}} & \mc{1}{c}{\scriptsize{-14.448}} & \mc{1}{c}{\scriptsize{-8.473}} & \mc{1}{c}{\scriptsize{-9.995}} & \mc{1}{c}{\scriptsize{-7.487}} & \mc{1}{c}{\scriptsize{-11.211}} & \mc{1}{c}{\scriptsize{-19.168}} & \mc{1}{c}{\scriptsize{-12.731}} \\  

     &  & \mc{1}{c}{\scriptsize{\textbf{(0.039)}}} & \mc{1}{c}{\scriptsize{\textbf{(0.000)}}} & \mc{1}{c}{\scriptsize{\textbf{(0.066)}}} & \mc{1}{c}{\scriptsize{(0.211)}} & \mc{1}{c}{\scriptsize{(0.158)}} & \mc{1}{c}{\scriptsize{\textbf{(0.053)}}} & \mc{1}{c}{\scriptsize{\textbf{(0.000)}}} & \mc{1}{c}{\scriptsize{\textbf{(0.039)}}} \\  

    \mc{1}{l}{\scriptsize{Prehypertension}} & \mc{1}{c}{\scriptsize{Mid-30s}} & \mc{1}{c}{\scriptsize{-0.089}} & \mc{1}{c}{\scriptsize{0.015}} & \mc{1}{c}{\scriptsize{0.053}} & \mc{1}{c}{\scriptsize{0.230}} & \mc{1}{c}{\scriptsize{0.147}} & \mc{1}{c}{\scriptsize{-0.209}} & \mc{1}{c}{\scriptsize{-0.230}} & \mc{1}{c}{\scriptsize{-0.262}} \\  

     &  & \mc{1}{c}{\scriptsize{(0.461)}} & \mc{1}{c}{\scriptsize{(0.750)}} & \mc{1}{c}{\scriptsize{(0.816)}} & \mc{1}{c}{\scriptsize{(0.947)}} & \mc{1}{c}{\scriptsize{(0.908)}} & \mc{1}{c}{\scriptsize{\textbf{(0.092)}}} & \mc{1}{c}{\scriptsize{(0.118)}} & \mc{1}{c}{\scriptsize{\textbf{(0.039)}}} \\  

    \mc{1}{l}{\scriptsize{Hypertension}} & \mc{1}{c}{\scriptsize{Mid-30s}} & \mc{1}{c}{\scriptsize{-0.149}} & \mc{1}{c}{\scriptsize{-0.157}} & \mc{1}{c}{\scriptsize{-0.053}} & \mc{1}{c}{\scriptsize{0.234}} & \mc{1}{c}{\scriptsize{-0.038}} & \mc{1}{c}{\scriptsize{-0.220}} & \mc{1}{c}{\scriptsize{-0.278}} & \mc{1}{c}{\scriptsize{-0.262}} \\  

     &  & \mc{1}{c}{\scriptsize{(0.382)}} & \mc{1}{c}{\scriptsize{(0.342)}} & \mc{1}{c}{\scriptsize{(0.697)}} & \mc{1}{c}{\scriptsize{(0.987)}} & \mc{1}{c}{\scriptsize{(0.658)}} & \mc{1}{c}{\scriptsize{\textbf{(0.079)}}} & \mc{1}{c}{\scriptsize{(0.224)}} & \mc{1}{c}{\scriptsize{\textbf{(0.079)}}} \\  

    \mc{1}{l}{\scriptsize{Hypertension Factor}} & \mc{1}{c}{\scriptsize{Mid-30s}} & \mc{1}{c}{\scriptsize{-0.433}} & \mc{1}{c}{\scriptsize{-0.596}} & \mc{1}{c}{\scriptsize{-0.086}} & \mc{1}{c}{\scriptsize{0.117}} & \mc{1}{c}{\scriptsize{-0.019}} & \mc{1}{c}{\scriptsize{-0.688}} & \mc{1}{c}{\scriptsize{-1.029}} & \mc{1}{c}{\scriptsize{-0.788}} \\  

     &  & \mc{1}{c}{\scriptsize{(0.105)}} & \mc{1}{c}{\scriptsize{(0.158)}} & \mc{1}{c}{\scriptsize{(0.684)}} & \mc{1}{c}{\scriptsize{(0.855)}} & \mc{1}{c}{\scriptsize{(0.697)}} & \mc{1}{c}{\scriptsize{\textbf{(0.026)}}} & \mc{1}{c}{\scriptsize{\textbf{(0.000)}}} & \mc{1}{c}{\scriptsize{\textbf{(0.026)}}} \\  

  \bottomrule
  \end{tabular}
	\end{table} 

	\begin{table}[H]
     \caption{Treatment Effects on Mother's Income, Male Sample}
     \label{table:abccare_rslt_male_cat12_sd}
	  \begin{tabular}{cccccccccc}
  \toprule

    \scriptsize{Variable} & \scriptsize{Age} & \scriptsize{(1)} & \scriptsize{(2)} & \scriptsize{(3)} & \scriptsize{(4)} & \scriptsize{(5)} & \scriptsize{(6)} & \scriptsize{(7)} & \scriptsize{(8)} \\ 
    \midrule  

    \mc{1}{l}{\scriptsize{High-Density Lipoprotein Chol. (mg/dL)}} & \mc{1}{c}{\scriptsize{Mid-30s}} & \mc{1}{c}{\scriptsize{5.080}} & \mc{1}{c}{\scriptsize{1.695}} & \mc{1}{c}{\scriptsize{4.567}} & \mc{1}{c}{\scriptsize{-7.040}} & \mc{1}{c}{\scriptsize{-2.304}} & \mc{1}{c}{\scriptsize{4.678}} & \mc{1}{c}{\scriptsize{2.650}} & \mc{1}{c}{\scriptsize{2.241}} \\  

     &  & \mc{1}{c}{\scriptsize{(0.237)}} & \mc{1}{c}{\scriptsize{(0.526)}} & \mc{1}{c}{\scriptsize{(0.342)}} & \mc{1}{c}{\scriptsize{(0.974)}} & \mc{1}{c}{\scriptsize{(0.711)}} & \mc{1}{c}{\scriptsize{(0.355)}} & \mc{1}{c}{\scriptsize{(0.474)}} & \mc{1}{c}{\scriptsize{(0.553)}} \\  

    \mc{1}{l}{\scriptsize{Dyslipidemia}} & \mc{1}{c}{\scriptsize{Mid-30s}} & \mc{1}{c}{\scriptsize{-0.120}} & \mc{1}{c}{\scriptsize{-0.053}} & \mc{1}{c}{\scriptsize{-0.133}} & \mc{1}{c}{\scriptsize{0.124}} & \mc{1}{c}{\scriptsize{-0.078}} & \mc{1}{c}{\scriptsize{-0.078}} & \mc{1}{c}{\scriptsize{-0.044}} & \mc{1}{c}{\scriptsize{-0.097}} \\  

     &  & \mc{1}{c}{\scriptsize{(0.237)}} & \mc{1}{c}{\scriptsize{(0.526)}} & \mc{1}{c}{\scriptsize{(0.368)}} & \mc{1}{c}{\scriptsize{(0.908)}} & \mc{1}{c}{\scriptsize{(0.408)}} & \mc{1}{c}{\scriptsize{(0.461)}} & \mc{1}{c}{\scriptsize{(0.605)}} & \mc{1}{c}{\scriptsize{(0.355)}} \\  

    \mc{1}{l}{\scriptsize{Cholesterol Factor}} & \mc{1}{c}{\scriptsize{Mid-30s}} & \mc{1}{c}{\scriptsize{-0.277}} & \mc{1}{c}{\scriptsize{-0.107}} & \mc{1}{c}{\scriptsize{-0.277}} & \mc{1}{c}{\scriptsize{0.337}} & \mc{1}{c}{\scriptsize{-0.023}} & \mc{1}{c}{\scriptsize{-0.218}} & \mc{1}{c}{\scriptsize{-0.124}} & \mc{1}{c}{\scriptsize{-0.172}} \\  

     &  & \mc{1}{c}{\scriptsize{(0.197)}} & \mc{1}{c}{\scriptsize{(0.513)}} & \mc{1}{c}{\scriptsize{(0.355)}} & \mc{1}{c}{\scriptsize{(0.961)}} & \mc{1}{c}{\scriptsize{(0.592)}} & \mc{1}{c}{\scriptsize{(0.355)}} & \mc{1}{c}{\scriptsize{(0.553)}} & \mc{1}{c}{\scriptsize{(0.408)}} \\  

  \bottomrule
  \end{tabular}
	\end{table} 

	\begin{table}[H]
     \caption{Treatment Effects on Parental Labor Income, Male Sample}
     \label{table:abccare_rslt_male_cat13_sd}
	  \begin{tabular}{cccccccccc}
  \toprule

    \scriptsize{Variable} & \scriptsize{Age} & \scriptsize{(1)} & \scriptsize{(2)} & \scriptsize{(3)} & \scriptsize{(4)} & \scriptsize{(5)} & \scriptsize{(6)} & \scriptsize{(7)} & \scriptsize{(8)} \\ 
    \midrule  

    \mc{1}{l}{\scriptsize{Systolic Blood Pressure (mm Hg)}} & \mc{1}{c}{\scriptsize{Mid-30s}} & \mc{1}{c}{\scriptsize{-17.544}} & \mc{1}{c}{\scriptsize{-21.728}} & \mc{1}{c}{\scriptsize{13.790}} & \mc{1}{c}{\scriptsize{16.092}} & \mc{1}{c}{\scriptsize{13.342}} & \mc{1}{c}{\scriptsize{-26.496}} & \mc{1}{c}{\scriptsize{-27.465}} & \mc{1}{c}{\scriptsize{-23.816}} \\  

     &  & \mc{1}{c}{\scriptsize{(0.145)}} & \mc{1}{c}{\scriptsize{\textbf{(0.092)}}} & \mc{1}{c}{\scriptsize{(1.000)}} & \mc{1}{c}{\scriptsize{(0.980)}} & \mc{1}{c}{\scriptsize{(1.000)}} & \mc{1}{c}{\scriptsize{\textbf{(0.013)}}} & \mc{1}{c}{\scriptsize{\textbf{(0.053)}}} & \mc{1}{c}{\scriptsize{\textbf{(0.013)}}} \\  

    \mc{1}{l}{\scriptsize{Diastolic Blood Pressure (mm Hg)}} & \mc{1}{c}{\scriptsize{Mid-30s}} & \mc{1}{c}{\scriptsize{-13.474}} & \mc{1}{c}{\scriptsize{-17.254}} & \mc{1}{c}{\scriptsize{0.526}} & \mc{1}{c}{\scriptsize{3.407}} & \mc{1}{c}{\scriptsize{0.671}} & \mc{1}{c}{\scriptsize{-19.759}} & \mc{1}{c}{\scriptsize{-20.583}} & \mc{1}{c}{\scriptsize{-15.881}} \\  

     &  & \mc{1}{c}{\scriptsize{(0.145)}} & \mc{1}{c}{\scriptsize{\textbf{(0.053)}}} & \mc{1}{c}{\scriptsize{(0.922)}} & \mc{1}{c}{\scriptsize{(0.843)}} & \mc{1}{c}{\scriptsize{(0.922)}} & \mc{1}{c}{\scriptsize{\textbf{(0.013)}}} & \mc{1}{c}{\scriptsize{\textbf{(0.039)}}} & \mc{1}{c}{\scriptsize{\textbf{(0.013)}}} \\  

    \mc{1}{l}{\scriptsize{Prehypertension}} & \mc{1}{c}{\scriptsize{Mid-30s}} & \mc{1}{c}{\scriptsize{-0.094}} & \mc{1}{c}{\scriptsize{-0.118}} & \mc{1}{c}{\scriptsize{0.684}} & \mc{1}{c}{\scriptsize{0.723}} & \mc{1}{c}{\scriptsize{0.712}} & \mc{1}{c}{\scriptsize{-0.316}} & \mc{1}{c}{\scriptsize{-0.269}} & \mc{1}{c}{\scriptsize{-0.296}} \\  

     &  & \mc{1}{c}{\scriptsize{(0.553)}} & \mc{1}{c}{\scriptsize{(0.197)}} & \mc{1}{c}{\scriptsize{(1.000)}} & \mc{1}{c}{\scriptsize{(1.000)}} & \mc{1}{c}{\scriptsize{(1.000)}} & \mc{1}{c}{\scriptsize{\textbf{(0.000)}}} & \mc{1}{c}{\scriptsize{(0.132)}} & \mc{1}{c}{\scriptsize{\textbf{(0.013)}}} \\  

    \mc{1}{l}{\scriptsize{Hypertension}} & \mc{1}{c}{\scriptsize{Mid-30s}} & \mc{1}{c}{\scriptsize{-0.345}} & \mc{1}{c}{\scriptsize{-0.431}} & \mc{1}{c}{\scriptsize{0.210}} & \mc{1}{c}{\scriptsize{0.329}} & \mc{1}{c}{\scriptsize{0.139}} & \mc{1}{c}{\scriptsize{-0.504}} & \mc{1}{c}{\scriptsize{-0.525}} & \mc{1}{c}{\scriptsize{-0.472}} \\  

     &  & \mc{1}{c}{\scriptsize{(0.105)}} & \mc{1}{c}{\scriptsize{\textbf{(0.079)}}} & \mc{1}{c}{\scriptsize{(1.000)}} & \mc{1}{c}{\scriptsize{(0.961)}} & \mc{1}{c}{\scriptsize{(1.000)}} & \mc{1}{c}{\scriptsize{\textbf{(0.013)}}} & \mc{1}{c}{\scriptsize{\textbf{(0.092)}}} & \mc{1}{c}{\scriptsize{\textbf{(0.039)}}} \\  

    \mc{1}{l}{\scriptsize{Hypertension Factor}} & \mc{1}{c}{\scriptsize{Mid-30s}} & \mc{1}{c}{\scriptsize{-0.792}} & \mc{1}{c}{\scriptsize{-0.996}} & \mc{1}{c}{\scriptsize{0.504}} & \mc{1}{c}{\scriptsize{0.663}} & \mc{1}{c}{\scriptsize{0.479}} & \mc{1}{c}{\scriptsize{-1.214}} & \mc{1}{c}{\scriptsize{-1.249}} & \mc{1}{c}{\scriptsize{-1.058}} \\  

     &  & \mc{1}{c}{\scriptsize{(0.105)}} & \mc{1}{c}{\scriptsize{\textbf{(0.079)}}} & \mc{1}{c}{\scriptsize{(1.000)}} & \mc{1}{c}{\scriptsize{(0.961)}} & \mc{1}{c}{\scriptsize{(1.000)}} & \mc{1}{c}{\scriptsize{\textbf{(0.000)}}} & \mc{1}{c}{\scriptsize{\textbf{(0.053)}}} & \mc{1}{c}{\scriptsize{\textbf{(0.000)}}} \\  

  \bottomrule
  \end{tabular}
	\end{table} 

	\begin{table}[H]
     \caption{Treatment Effects on Parental Public Transfer Income, Male Sample}
     \label{table:abccare_rslt_male_cat14_sd}
	  \begin{tabular}{cccccccccc}
  \toprule

    \scriptsize{Variable} & \scriptsize{Age} & \scriptsize{(1)} & \scriptsize{(2)} & \scriptsize{(3)} & \scriptsize{(4)} & \scriptsize{(5)} & \scriptsize{(6)} & \scriptsize{(7)} & \scriptsize{(8)} \\ 
    \midrule  

    \mc{1}{l}{\scriptsize{Hemoglobin Level (\%)}} & \mc{1}{c}{\scriptsize{Mid-30s}} & \mc{1}{c}{\scriptsize{0.322}} & \mc{1}{c}{\scriptsize{0.449}} & \mc{1}{c}{\scriptsize{0.240}} & \mc{1}{c}{\scriptsize{0.320}} & \mc{1}{c}{\scriptsize{0.359}} & \mc{1}{c}{\scriptsize{0.286}} & \mc{1}{c}{\scriptsize{0.416}} & \mc{1}{c}{\scriptsize{0.417}} \\  

     &  & \mc{1}{c}{\scriptsize{(0.366)}} & \mc{1}{c}{\scriptsize{(0.389)}} & \mc{1}{c}{\scriptsize{(0.980)}} & \mc{1}{c}{\scriptsize{(0.245)}} & \mc{1}{c}{\scriptsize{(0.445)}} & \mc{1}{c}{\scriptsize{(0.422)}} & \mc{1}{c}{\scriptsize{(0.456)}} & \mc{1}{c}{\scriptsize{(0.380)}} \\  

    \mc{1}{l}{\scriptsize{Prediabetes}} & \mc{1}{c}{\scriptsize{Mid-30s}} & \mc{1}{c}{\scriptsize{-0.129}} & \mc{1}{c}{\scriptsize{-0.149}} & \mc{1}{c}{\scriptsize{-0.267}} & \mc{1}{c}{\scriptsize{-0.358}} & \mc{1}{c}{\scriptsize{-0.309}} & \mc{1}{c}{\scriptsize{-0.138}} & \mc{1}{c}{\scriptsize{-0.161}} & \mc{1}{c}{\scriptsize{-0.143}} \\  

     &  & \mc{1}{c}{\scriptsize{(0.433)}} & \mc{1}{c}{\scriptsize{(0.389)}} & \mc{1}{c}{\scriptsize{(0.980)}} & \mc{1}{c}{\scriptsize{(0.245)}} & \mc{1}{c}{\scriptsize{(0.445)}} & \mc{1}{c}{\scriptsize{(0.439)}} & \mc{1}{c}{\scriptsize{(0.456)}} & \mc{1}{c}{\scriptsize{(0.419)}} \\  

    \mc{1}{l}{\scriptsize{Diabetes}} & \mc{1}{c}{\scriptsize{Mid-30s}} & \mc{1}{c}{\scriptsize{0.080}} & \mc{1}{c}{\scriptsize{0.093}} & \mc{1}{c}{\scriptsize{0.080}} & \mc{1}{c}{\scriptsize{0.078}} & \mc{1}{c}{\scriptsize{0.095}} & \mc{1}{c}{\scriptsize{0.080}} & \mc{1}{c}{\scriptsize{0.097}} & \mc{1}{c}{\scriptsize{0.095}} \\  

     &  & \mc{1}{c}{\scriptsize{(0.109)}} & \mc{1}{c}{\scriptsize{(0.177)}} & \mc{1}{c}{\scriptsize{(0.980)}} & \mc{1}{c}{\scriptsize{(0.245)}} & \mc{1}{c}{\scriptsize{\textbf{(0.079)}}} & \mc{1}{c}{\scriptsize{(0.124)}} & \mc{1}{c}{\scriptsize{(0.181)}} & \mc{1}{c}{\scriptsize{(0.115)}} \\  

    \mc{1}{l}{\scriptsize{Diabetes Factor}} & \mc{1}{c}{\scriptsize{Mid-30s}} & \mc{1}{c}{\scriptsize{0.218}} & \mc{1}{c}{\scriptsize{0.271}} & \mc{1}{c}{\scriptsize{0.106}} & \mc{1}{c}{\scriptsize{0.076}} & \mc{1}{c}{\scriptsize{0.163}} & \mc{1}{c}{\scriptsize{0.199}} & \mc{1}{c}{\scriptsize{0.267}} & \mc{1}{c}{\scriptsize{0.259}} \\  

     &  & \mc{1}{c}{\scriptsize{(0.433)}} & \mc{1}{c}{\scriptsize{(0.389)}} & \mc{1}{c}{\scriptsize{(0.980)}} & \mc{1}{c}{\scriptsize{(0.245)}} & \mc{1}{c}{\scriptsize{(0.445)}} & \mc{1}{c}{\scriptsize{(0.439)}} & \mc{1}{c}{\scriptsize{(0.456)}} & \mc{1}{c}{\scriptsize{(0.419)}} \\  

  \bottomrule
  \end{tabular}
	\end{table} 

	\begin{table}[H]
     \caption{Treatment Effects on Adoption, Male Sample}
     \label{table:abccare_rslt_male_cat15_sd}
	  \begin{tabular}{cccccccccc}
  \toprule

    \scriptsize{Variable} & \scriptsize{Age} & \scriptsize{(1)} & \scriptsize{(2)} & \scriptsize{(3)} & \scriptsize{(4)} & \scriptsize{(5)} & \scriptsize{(6)} & \scriptsize{(7)} & \scriptsize{(8)} \\ 
    \midrule  

    \mc{1}{l}{\scriptsize{Measured BMI}} & \mc{1}{c}{\scriptsize{Mid-30s}} & \mc{1}{c}{\scriptsize{-0.125}} & \mc{1}{c}{\scriptsize{-0.617}} & \mc{1}{c}{\scriptsize{-0.684}} & \mc{1}{c}{\scriptsize{-3.505}} & \mc{1}{c}{\scriptsize{0.867}} & \mc{1}{c}{\scriptsize{-0.627}} & \mc{1}{c}{\scriptsize{0.025}} & \mc{1}{c}{\scriptsize{-0.487}} \\  

     &  & \mc{1}{c}{\scriptsize{(0.763)}} & \mc{1}{c}{\scriptsize{(0.789)}} & \mc{1}{c}{\scriptsize{(0.594)}} & \mc{1}{c}{\scriptsize{(0.797)}} & \mc{1}{c}{\scriptsize{(0.797)}} & \mc{1}{c}{\scriptsize{(0.776)}} & \mc{1}{c}{\scriptsize{(0.855)}} & \mc{1}{c}{\scriptsize{(0.829)}} \\  

    \mc{1}{l}{\scriptsize{Obesity}} & \mc{1}{c}{\scriptsize{Mid-30s}} &  & \mc{1}{c}{\scriptsize{-0.103}} & \mc{1}{c}{\scriptsize{-0.128}} & \mc{1}{c}{\scriptsize{-0.331}} & \mc{1}{c}{\scriptsize{0.031}} & \mc{1}{c}{\scriptsize{-0.017}} & \mc{1}{c}{\scriptsize{-0.045}} & \mc{1}{c}{\scriptsize{-0.060}} \\  

     &  &  & \mc{1}{c}{\scriptsize{(0.724)}} & \mc{1}{c}{\scriptsize{(0.493)}} & \mc{1}{c}{\scriptsize{(0.406)}} & \mc{1}{c}{\scriptsize{(0.725)}} & \mc{1}{c}{\scriptsize{(0.776)}} & \mc{1}{c}{\scriptsize{(0.763)}} & \mc{1}{c}{\scriptsize{(0.737)}} \\  

    \mc{1}{l}{\scriptsize{Severe Obesity}} & \mc{1}{c}{\scriptsize{Mid-30s}} & \mc{1}{c}{\scriptsize{-0.160}} & \mc{1}{c}{\scriptsize{-0.132}} & \mc{1}{c}{\scriptsize{-0.185}} & \mc{1}{c}{\scriptsize{-0.319}} & \mc{1}{c}{\scriptsize{-0.126}} & \mc{1}{c}{\scriptsize{-0.185}} & \mc{1}{c}{\scriptsize{-0.108}} & \mc{1}{c}{\scriptsize{-0.131}} \\  

     &  & \mc{1}{c}{\scriptsize{(0.434)}} & \mc{1}{c}{\scriptsize{(0.513)}} & \mc{1}{c}{\scriptsize{(0.464)}} & \mc{1}{c}{\scriptsize{(0.522)}} & \mc{1}{c}{\scriptsize{(0.493)}} & \mc{1}{c}{\scriptsize{(0.368)}} & \mc{1}{c}{\scriptsize{(0.711)}} & \mc{1}{c}{\scriptsize{(0.553)}} \\  

    \mc{1}{l}{\scriptsize{Waist-hip Ratio}} & \mc{1}{c}{\scriptsize{Mid-30s}} & \mc{1}{c}{\scriptsize{0.004}} & \mc{1}{c}{\scriptsize{-0.014}} & \mc{1}{c}{\scriptsize{0.018}} & \mc{1}{c}{\scriptsize{-0.043}} & \mc{1}{c}{\scriptsize{0.030}} & \mc{1}{c}{\scriptsize{-0.002}} & \mc{1}{c}{\scriptsize{-0.008}} & \mc{1}{c}{\scriptsize{-0.004}} \\  

     &  & \mc{1}{c}{\scriptsize{(0.842)}} & \mc{1}{c}{\scriptsize{(0.724)}} & \mc{1}{c}{\scriptsize{(0.783)}} & \mc{1}{c}{\scriptsize{(0.739)}} & \mc{1}{c}{\scriptsize{(0.870)}} & \mc{1}{c}{\scriptsize{(0.789)}} & \mc{1}{c}{\scriptsize{(0.763)}} & \mc{1}{c}{\scriptsize{(0.829)}} \\  

    \mc{1}{l}{\scriptsize{Abdominal Obesity}} & \mc{1}{c}{\scriptsize{Mid-30s}} & \mc{1}{c}{\scriptsize{0.003}} & \mc{1}{c}{\scriptsize{-0.137}} & \mc{1}{c}{\scriptsize{0.029}} & \mc{1}{c}{\scriptsize{-0.377}} & \mc{1}{c}{\scriptsize{0.100}} & \mc{1}{c}{\scriptsize{0.029}} & \mc{1}{c}{\scriptsize{-0.062}} & \mc{1}{c}{\scriptsize{-0.031}} \\  

     &  & \mc{1}{c}{\scriptsize{(0.803)}} & \mc{1}{c}{\scriptsize{(0.500)}} & \mc{1}{c}{\scriptsize{(0.710)}} & \mc{1}{c}{\scriptsize{(0.580)}} & \mc{1}{c}{\scriptsize{(0.826)}} & \mc{1}{c}{\scriptsize{(0.882)}} & \mc{1}{c}{\scriptsize{(0.750)}} & \mc{1}{c}{\scriptsize{(0.816)}} \\  

    \mc{1}{l}{\scriptsize{Framingham Risk Score}} & \mc{1}{c}{\scriptsize{Mid-30s}} & \mc{1}{c}{\scriptsize{-0.766}} & \mc{1}{c}{\scriptsize{-0.472}} & \mc{1}{c}{\scriptsize{1.491}} & \mc{1}{c}{\scriptsize{2.437}} & \mc{1}{c}{\scriptsize{1.802}} & \mc{1}{c}{\scriptsize{-1.202}} & \mc{1}{c}{\scriptsize{-1.036}} & \mc{1}{c}{\scriptsize{-0.705}} \\  

     &  & \mc{1}{c}{\scriptsize{(0.592)}} & \mc{1}{c}{\scriptsize{(0.750)}} & \mc{1}{c}{\scriptsize{(0.942)}} & \mc{1}{c}{\scriptsize{(1.000)}} & \mc{1}{c}{\scriptsize{(0.971)}} & \mc{1}{c}{\scriptsize{(0.408)}} & \mc{1}{c}{\scriptsize{(0.592)}} & \mc{1}{c}{\scriptsize{(0.658)}} \\  

    \mc{1}{l}{\scriptsize{Obesity Factor}} & \mc{1}{c}{\scriptsize{Mid-30s}} & \mc{1}{c}{\scriptsize{-0.040}} & \mc{1}{c}{\scriptsize{-0.197}} & \mc{1}{c}{\scriptsize{-0.036}} & \mc{1}{c}{\scriptsize{-0.524}} & \mc{1}{c}{\scriptsize{0.211}} & \mc{1}{c}{\scriptsize{-0.101}} & \mc{1}{c}{\scriptsize{-0.091}} & \mc{1}{c}{\scriptsize{-0.116}} \\  

     &  & \mc{1}{c}{\scriptsize{(0.763)}} & \mc{1}{c}{\scriptsize{(0.724)}} & \mc{1}{c}{\scriptsize{(0.652)}} & \mc{1}{c}{\scriptsize{(0.739)}} & \mc{1}{c}{\scriptsize{(0.812)}} & \mc{1}{c}{\scriptsize{(0.750)}} & \mc{1}{c}{\scriptsize{(0.803)}} & \mc{1}{c}{\scriptsize{(0.750)}} \\  

  \bottomrule
  \end{tabular}
	\end{table} 

	\begin{table}[H]
     \caption{Treatment Effects on Childhood Household Income, Male Sample}
     \label{table:abccare_rslt_male_cat16_sd}
	  \begin{tabular}{cccccccccc}
  \toprule

    \scriptsize{Variable} & \scriptsize{Age} & \scriptsize{(1)} & \scriptsize{(2)} & \scriptsize{(3)} & \scriptsize{(4)} & \scriptsize{(5)} & \scriptsize{(6)} & \scriptsize{(7)} & \scriptsize{(8)} \\ 
    \midrule  

    \mc{1}{l}{\scriptsize{Somatization}} & \mc{1}{c}{\scriptsize{21}} & \mc{1}{c}{\scriptsize{0.056}} & \mc{1}{c}{\scriptsize{-0.132}} & \mc{1}{c}{\scriptsize{0.153}} & \mc{1}{c}{\scriptsize{-0.224}} & \mc{1}{c}{\scriptsize{0.108}} & \mc{1}{c}{\scriptsize{0.033}} & \mc{1}{c}{\scriptsize{-0.132}} & \mc{1}{c}{\scriptsize{-0.031}} \\  

     &  & \mc{1}{c}{\scriptsize{(0.974)}} & \mc{1}{c}{\scriptsize{(0.500)}} & \mc{1}{c}{\scriptsize{(0.987)}} & \mc{1}{c}{\scriptsize{(0.421)}} & \mc{1}{c}{\scriptsize{(0.974)}} & \mc{1}{c}{\scriptsize{(0.974)}} & \mc{1}{c}{\scriptsize{(0.487)}} & \mc{1}{c}{\scriptsize{(0.882)}} \\  

     & \mc{1}{c}{\scriptsize{34}} & \mc{1}{c}{\scriptsize{-0.252}} & \mc{1}{c}{\scriptsize{-0.327}} & \mc{1}{c}{\scriptsize{0.059}} & \mc{1}{c}{\scriptsize{-0.013}} & \mc{1}{c}{\scriptsize{0.038}} & \mc{1}{c}{\scriptsize{-0.403}} & \mc{1}{c}{\scriptsize{-0.488}} & \mc{1}{c}{\scriptsize{-0.487}} \\  

     &  & \mc{1}{c}{\scriptsize{(0.382)}} & \mc{1}{c}{\scriptsize{(0.368)}} & \mc{1}{c}{\scriptsize{(0.974)}} & \mc{1}{c}{\scriptsize{(0.961)}} & \mc{1}{c}{\scriptsize{(0.961)}} & \mc{1}{c}{\scriptsize{(0.316)}} & \mc{1}{c}{\scriptsize{(0.342)}} & \mc{1}{c}{\scriptsize{(0.276)}} \\  

    \mc{1}{l}{\scriptsize{Depression}} & \mc{1}{c}{\scriptsize{21}} & \mc{1}{c}{\scriptsize{0.032}} & \mc{1}{c}{\scriptsize{-0.127}} & \mc{1}{c}{\scriptsize{0.304}} & \mc{1}{c}{\scriptsize{-0.040}} & \mc{1}{c}{\scriptsize{0.238}} & \mc{1}{c}{\scriptsize{0.001}} & \mc{1}{c}{\scriptsize{-0.144}} & \mc{1}{c}{\scriptsize{-0.050}} \\  

     &  & \mc{1}{c}{\scriptsize{(0.947)}} & \mc{1}{c}{\scriptsize{(0.526)}} & \mc{1}{c}{\scriptsize{(1.000)}} & \mc{1}{c}{\scriptsize{(0.908)}} & \mc{1}{c}{\scriptsize{(1.000)}} & \mc{1}{c}{\scriptsize{(0.882)}} & \mc{1}{c}{\scriptsize{(0.447)}} & \mc{1}{c}{\scriptsize{(0.855)}} \\  

     & \mc{1}{c}{\scriptsize{34}} & \mc{1}{c}{\scriptsize{-0.221}} & \mc{1}{c}{\scriptsize{-0.282}} & \mc{1}{c}{\scriptsize{0.216}} & \mc{1}{c}{\scriptsize{0.282}} & \mc{1}{c}{\scriptsize{0.124}} & \mc{1}{c}{\scriptsize{-0.428}} & \mc{1}{c}{\scriptsize{-0.570}} & \mc{1}{c}{\scriptsize{-0.569}} \\  

     &  & \mc{1}{c}{\scriptsize{(0.500)}} & \mc{1}{c}{\scriptsize{(0.461)}} & \mc{1}{c}{\scriptsize{(1.000)}} & \mc{1}{c}{\scriptsize{(1.000)}} & \mc{1}{c}{\scriptsize{(0.987)}} & \mc{1}{c}{\scriptsize{(0.329)}} & \mc{1}{c}{\scriptsize{(0.224)}} & \mc{1}{c}{\scriptsize{(0.237)}} \\  

    \mc{1}{l}{\scriptsize{Anxiety}} & \mc{1}{c}{\scriptsize{21}} & \mc{1}{c}{\scriptsize{0.215}} & \mc{1}{c}{\scriptsize{-0.005}} & \mc{1}{c}{\scriptsize{0.350}} & \mc{1}{c}{\scriptsize{-0.067}} & \mc{1}{c}{\scriptsize{0.291}} & \mc{1}{c}{\scriptsize{0.178}} & \mc{1}{c}{\scriptsize{-0.050}} & \mc{1}{c}{\scriptsize{0.105}} \\  

     &  & \mc{1}{c}{\scriptsize{(1.000)}} & \mc{1}{c}{\scriptsize{(0.908)}} & \mc{1}{c}{\scriptsize{(1.000)}} & \mc{1}{c}{\scriptsize{(0.829)}} & \mc{1}{c}{\scriptsize{(1.000)}} & \mc{1}{c}{\scriptsize{(1.000)}} & \mc{1}{c}{\scriptsize{(0.750)}} & \mc{1}{c}{\scriptsize{(1.000)}} \\  

     & \mc{1}{c}{\scriptsize{34}} & \mc{1}{c}{\scriptsize{-0.247}} & \mc{1}{c}{\scriptsize{-0.313}} & \mc{1}{c}{\scriptsize{0.099}} & \mc{1}{c}{\scriptsize{0.093}} & \mc{1}{c}{\scriptsize{0.026}} & \mc{1}{c}{\scriptsize{-0.412}} & \mc{1}{c}{\scriptsize{-0.516}} & \mc{1}{c}{\scriptsize{-0.530}} \\  

     &  & \mc{1}{c}{\scriptsize{(0.408)}} & \mc{1}{c}{\scriptsize{(0.382)}} & \mc{1}{c}{\scriptsize{(0.987)}} & \mc{1}{c}{\scriptsize{(1.000)}} & \mc{1}{c}{\scriptsize{(0.961)}} & \mc{1}{c}{\scriptsize{(0.329)}} & \mc{1}{c}{\scriptsize{(0.342)}} & \mc{1}{c}{\scriptsize{(0.250)}} \\  

    \mc{1}{l}{\scriptsize{Hostility}} & \mc{1}{c}{\scriptsize{21}} & \mc{1}{c}{\scriptsize{-0.100}} & \mc{1}{c}{\scriptsize{-0.176}} & \mc{1}{c}{\scriptsize{0.206}} & \mc{1}{c}{\scriptsize{-0.071}} & \mc{1}{c}{\scriptsize{0.159}} & \mc{1}{c}{\scriptsize{-0.174}} & \mc{1}{c}{\scriptsize{-0.215}} & \mc{1}{c}{\scriptsize{-0.196}} \\  

     &  & \mc{1}{c}{\scriptsize{(0.671)}} & \mc{1}{c}{\scriptsize{(0.461)}} & \mc{1}{c}{\scriptsize{(0.987)}} & \mc{1}{c}{\scriptsize{(0.882)}} & \mc{1}{c}{\scriptsize{(0.974)}} & \mc{1}{c}{\scriptsize{(0.461)}} & \mc{1}{c}{\scriptsize{(0.368)}} & \mc{1}{c}{\scriptsize{(0.382)}} \\  

     & \mc{1}{c}{\scriptsize{34}} & \mc{1}{c}{\scriptsize{-0.227}} & \mc{1}{c}{\scriptsize{-0.272}} & \mc{1}{c}{\scriptsize{0.100}} & \mc{1}{c}{\scriptsize{0.007}} & \mc{1}{c}{\scriptsize{0.051}} & \mc{1}{c}{\scriptsize{-0.387}} & \mc{1}{c}{\scriptsize{-0.460}} & \mc{1}{c}{\scriptsize{-0.496}} \\  

     &  & \mc{1}{c}{\scriptsize{(0.474)}} & \mc{1}{c}{\scriptsize{(0.421)}} & \mc{1}{c}{\scriptsize{(0.987)}} & \mc{1}{c}{\scriptsize{(0.974)}} & \mc{1}{c}{\scriptsize{(0.974)}} & \mc{1}{c}{\scriptsize{(0.342)}} & \mc{1}{c}{\scriptsize{(0.355)}} & \mc{1}{c}{\scriptsize{(0.250)}} \\  

    \mc{1}{l}{\scriptsize{Global Severity Index}} & \mc{1}{c}{\scriptsize{21}} & \mc{1}{c}{\scriptsize{0.077}} & \mc{1}{c}{\scriptsize{-0.093}} & \mc{1}{c}{\scriptsize{0.296}} & \mc{1}{c}{\scriptsize{-0.056}} & \mc{1}{c}{\scriptsize{0.235}} & \mc{1}{c}{\scriptsize{0.026}} & \mc{1}{c}{\scriptsize{-0.133}} & \mc{1}{c}{\scriptsize{-0.043}} \\  

     &  & \mc{1}{c}{\scriptsize{(0.987)}} & \mc{1}{c}{\scriptsize{(0.632)}} & \mc{1}{c}{\scriptsize{(1.000)}} & \mc{1}{c}{\scriptsize{(0.868)}} & \mc{1}{c}{\scriptsize{(1.000)}} & \mc{1}{c}{\scriptsize{(0.947)}} & \mc{1}{c}{\scriptsize{(0.447)}} & \mc{1}{c}{\scriptsize{(0.855)}} \\  

     & \mc{1}{c}{\scriptsize{34}} & \mc{1}{c}{\scriptsize{-4.320}} & \mc{1}{c}{\scriptsize{-5.535}} & \mc{1}{c}{\scriptsize{2.241}} & \mc{1}{c}{\scriptsize{2.178}} & \mc{1}{c}{\scriptsize{1.124}} & \mc{1}{c}{\scriptsize{-7.459}} & \mc{1}{c}{\scriptsize{-9.445}} & \mc{1}{c}{\scriptsize{-9.511}} \\  

     &  & \mc{1}{c}{\scriptsize{(0.434)}} & \mc{1}{c}{\scriptsize{(0.382)}} & \mc{1}{c}{\scriptsize{(1.000)}} & \mc{1}{c}{\scriptsize{(1.000)}} & \mc{1}{c}{\scriptsize{(0.974)}} & \mc{1}{c}{\scriptsize{(0.303)}} & \mc{1}{c}{\scriptsize{(0.289)}} & \mc{1}{c}{\scriptsize{(0.250)}} \\  

    \mc{1}{l}{\scriptsize{BSI Factor}} & \mc{1}{c}{\scriptsize{21 and 34}} & \mc{1}{c}{\scriptsize{-0.405}} & \mc{1}{c}{\scriptsize{-0.186}} & \mc{1}{c}{\scriptsize{0.080}} & \mc{1}{c}{\scriptsize{0.475}} & \mc{1}{c}{\scriptsize{0.134}} & \mc{1}{c}{\scriptsize{-0.607}} & \mc{1}{c}{\scriptsize{-0.363}} & \mc{1}{c}{\scriptsize{-0.683}} \\  

     &  & \mc{1}{c}{\scriptsize{(0.368)}} & \mc{1}{c}{\scriptsize{(0.789)}} & \mc{1}{c}{\scriptsize{(0.974)}} & \mc{1}{c}{\scriptsize{(1.000)}} & \mc{1}{c}{\scriptsize{(0.974)}} & \mc{1}{c}{\scriptsize{(0.316)}} & \mc{1}{c}{\scriptsize{(0.605)}} & \mc{1}{c}{\scriptsize{(0.316)}} \\  

  \bottomrule
  \end{tabular}
	\end{table} 

	\begin{table}[H]
     \caption{Treatment Effects on Father at Home, Male Sample}
     \label{table:abccare_rslt_male_cat17_sd}
	  \begin{tabular}{cccccccccc}
  \toprule

    \scriptsize{Variable} & \scriptsize{Age} & \scriptsize{(1)} & \scriptsize{(2)} & \scriptsize{(3)} & \scriptsize{(4)} & \scriptsize{(5)} & \scriptsize{(6)} & \scriptsize{(7)} & \scriptsize{(8)} \\ 
    \midrule  

    \mc{1}{l}{\scriptsize{Measured BMI}} & \mc{1}{c}{\scriptsize{Mid-30s}} & \mc{1}{c}{\scriptsize{-4.075}} & \mc{1}{c}{\scriptsize{-7.969}} & \mc{1}{c}{\scriptsize{5.687}} & \mc{1}{c}{\scriptsize{-1.053}} & \mc{1}{c}{\scriptsize{5.843}} & \mc{1}{c}{\scriptsize{-7.091}} & \mc{1}{c}{\scriptsize{-9.553}} & \mc{1}{c}{\scriptsize{-8.546}} \\  

     &  & \mc{1}{c}{\scriptsize{(0.368)}} & \mc{1}{c}{\scriptsize{(0.171)}} & \mc{1}{c}{\scriptsize{(1.000)}} & \mc{1}{c}{\scriptsize{(0.902)}} & \mc{1}{c}{\scriptsize{(1.000)}} & \mc{1}{c}{\scriptsize{\textbf{(0.079)}}} & \mc{1}{c}{\scriptsize{\textbf{(0.079)}}} & \mc{1}{c}{\scriptsize{\textbf{(0.013)}}} \\  

    \mc{1}{l}{\scriptsize{Obesity}} & \mc{1}{c}{\scriptsize{Mid-30s}} & \mc{1}{c}{\scriptsize{-0.125}} & \mc{1}{c}{\scriptsize{-0.389}} & \mc{1}{c}{\scriptsize{0.500}} & \mc{1}{c}{\scriptsize{0.289}} & \mc{1}{c}{\scriptsize{0.528}} & \mc{1}{c}{\scriptsize{-0.333}} & \mc{1}{c}{\scriptsize{-0.526}} & \mc{1}{c}{\scriptsize{-0.371}} \\  

     &  & \mc{1}{c}{\scriptsize{(0.632)}} & \mc{1}{c}{\scriptsize{(0.184)}} & \mc{1}{c}{\scriptsize{(1.000)}} & \mc{1}{c}{\scriptsize{(1.000)}} & \mc{1}{c}{\scriptsize{(1.000)}} & \mc{1}{c}{\scriptsize{\textbf{(0.040)}}} & \mc{1}{c}{\scriptsize{\textbf{(0.039)}}} & \mc{1}{c}{\scriptsize{\textbf{(0.026)}}} \\  

    \mc{1}{l}{\scriptsize{Severe Obesity}} & \mc{1}{c}{\scriptsize{Mid-30s}} & \mc{1}{c}{\scriptsize{-0.280}} & \mc{1}{c}{\scriptsize{-0.336}} & \mc{1}{c}{\scriptsize{0.095}} & \mc{1}{c}{\scriptsize{-0.062}} & \mc{1}{c}{\scriptsize{0.133}} & \mc{1}{c}{\scriptsize{-0.405}} & \mc{1}{c}{\scriptsize{-0.404}} & \mc{1}{c}{\scriptsize{-0.424}} \\  

     &  & \mc{1}{c}{\scriptsize{(0.276)}} & \mc{1}{c}{\scriptsize{(0.316)}} & \mc{1}{c}{\scriptsize{(1.000)}} & \mc{1}{c}{\scriptsize{(0.725)}} & \mc{1}{c}{\scriptsize{(1.000)}} & \mc{1}{c}{\scriptsize{\textbf{(0.039)}}} & \mc{1}{c}{\scriptsize{(0.289)}} & \mc{1}{c}{\scriptsize{\textbf{(0.079)}}} \\  

    \mc{1}{l}{\scriptsize{Waist-hip Ratio}} & \mc{1}{c}{\scriptsize{Mid-30s}} & \mc{1}{c}{\scriptsize{-0.025}} & \mc{1}{c}{\scriptsize{-0.026}} & \mc{1}{c}{\scriptsize{0.120}} & \mc{1}{c}{\scriptsize{0.112}} & \mc{1}{c}{\scriptsize{0.121}} & \mc{1}{c}{\scriptsize{-0.057}} & \mc{1}{c}{\scriptsize{-0.064}} & \mc{1}{c}{\scriptsize{-0.047}} \\  

     &  & \mc{1}{c}{\scriptsize{(0.579)}} & \mc{1}{c}{\scriptsize{(0.618)}} & \mc{1}{c}{\scriptsize{(1.000)}} & \mc{1}{c}{\scriptsize{(1.000)}} & \mc{1}{c}{\scriptsize{(1.000)}} & \mc{1}{c}{\scriptsize{\textbf{(0.079)}}} & \mc{1}{c}{\scriptsize{(0.329)}} & \mc{1}{c}{\scriptsize{\textbf{(0.079)}}} \\  

    \mc{1}{l}{\scriptsize{Abdominal Obesity}} & \mc{1}{c}{\scriptsize{Mid-30s}} & \mc{1}{c}{\scriptsize{-0.228}} & \mc{1}{c}{\scriptsize{-0.113}} & \mc{1}{c}{\scriptsize{0.647}} & \mc{1}{c}{\scriptsize{0.498}} & \mc{1}{c}{\scriptsize{0.633}} & \mc{1}{c}{\scriptsize{-0.353}} & \mc{1}{c}{\scriptsize{-0.248}} & \mc{1}{c}{\scriptsize{-0.289}} \\  

     &  & \mc{1}{c}{\scriptsize{(0.276)}} & \mc{1}{c}{\scriptsize{(0.658)}} & \mc{1}{c}{\scriptsize{(1.000)}} & \mc{1}{c}{\scriptsize{(1.000)}} & \mc{1}{c}{\scriptsize{(1.000)}} & \mc{1}{c}{\scriptsize{\textbf{(0.000)}}} & \mc{1}{c}{\scriptsize{(0.487)}} & \mc{1}{c}{\scriptsize{\textbf{(0.013)}}} \\  

    \mc{1}{l}{\scriptsize{Framingham Risk Score}} & \mc{1}{c}{\scriptsize{Mid-30s}} & \mc{1}{c}{\scriptsize{-1.829}} & \mc{1}{c}{\scriptsize{-1.323}} & \mc{1}{c}{\scriptsize{4.255}} & \mc{1}{c}{\scriptsize{6.734}} & \mc{1}{c}{\scriptsize{4.505}} & \mc{1}{c}{\scriptsize{-2.551}} & \mc{1}{c}{\scriptsize{-2.308}} & \mc{1}{c}{\scriptsize{-3.009}} \\  

     &  & \mc{1}{c}{\scriptsize{(0.289)}} & \mc{1}{c}{\scriptsize{(0.592)}} & \mc{1}{c}{\scriptsize{(1.000)}} & \mc{1}{c}{\scriptsize{(1.000)}} & \mc{1}{c}{\scriptsize{(1.000)}} & \mc{1}{c}{\scriptsize{\textbf{(0.079)}}} & \mc{1}{c}{\scriptsize{(0.329)}} & \mc{1}{c}{\scriptsize{\textbf{(0.066)}}} \\  

    \mc{1}{l}{\scriptsize{Obesity Factor}} & \mc{1}{c}{\scriptsize{Mid-30s}} & \mc{1}{c}{\scriptsize{-0.585}} & \mc{1}{c}{\scriptsize{-0.702}} & \mc{1}{c}{\scriptsize{1.140}} & \mc{1}{c}{\scriptsize{0.545}} & \mc{1}{c}{\scriptsize{1.181}} & \mc{1}{c}{\scriptsize{-1.110}} & \mc{1}{c}{\scriptsize{-1.111}} & \mc{1}{c}{\scriptsize{-1.137}} \\  

     &  & \mc{1}{c}{\scriptsize{(0.368)}} & \mc{1}{c}{\scriptsize{(0.329)}} & \mc{1}{c}{\scriptsize{(1.000)}} & \mc{1}{c}{\scriptsize{(1.000)}} & \mc{1}{c}{\scriptsize{(1.000)}} & \mc{1}{c}{\scriptsize{\textbf{(0.000)}}} & \mc{1}{c}{\scriptsize{(0.132)}} & \mc{1}{c}{\scriptsize{\textbf{(0.013)}}} \\  

  \bottomrule
  \end{tabular}
	\end{table} 

	\begin{table}[H]
     \caption{Treatment Effects on HOME Scores, Male Sample}
     \label{table:abccare_rslt_male_cat18_sd}
	  \begin{tabular}{cccccccccc}
  \toprule

    \scriptsize{Variable} & \scriptsize{Age} & \scriptsize{(1)} & \scriptsize{(2)} & \scriptsize{(3)} & \scriptsize{(4)} & \scriptsize{(5)} & \scriptsize{(6)} & \scriptsize{(7)} & \scriptsize{(8)} \\ 
    \midrule  

    \mc{1}{l}{\scriptsize{Somatization}} & \mc{1}{c}{\scriptsize{21}} & \mc{1}{c}{\scriptsize{0.035}} & \mc{1}{c}{\scriptsize{-0.051}} & \mc{1}{c}{\scriptsize{-0.021}} & \mc{1}{c}{\scriptsize{-0.259}} & \mc{1}{c}{\scriptsize{-0.019}} & \mc{1}{c}{\scriptsize{0.064}} & \mc{1}{c}{\scriptsize{-0.010}} & \mc{1}{c}{\scriptsize{0.056}} \\  

     &  & \mc{1}{c}{\scriptsize{(0.882)}} & \mc{1}{c}{\scriptsize{(0.711)}} & \mc{1}{c}{\scriptsize{(0.816)}} & \mc{1}{c}{\scriptsize{(0.539)}} & \mc{1}{c}{\scriptsize{(0.816)}} & \mc{1}{c}{\scriptsize{(0.961)}} & \mc{1}{c}{\scriptsize{(0.868)}} & \mc{1}{c}{\scriptsize{(0.921)}} \\  

     & \mc{1}{c}{\scriptsize{34}} & \mc{1}{c}{\scriptsize{-0.466}} & \mc{1}{c}{\scriptsize{-0.502}} & \mc{1}{c}{\scriptsize{0.071}} & \mc{1}{c}{\scriptsize{-0.062}} & \mc{1}{c}{\scriptsize{0.048}} & \mc{1}{c}{\scriptsize{-0.619}} & \mc{1}{c}{\scriptsize{-0.566}} & \mc{1}{c}{\scriptsize{-0.649}} \\  

     &  & \mc{1}{c}{\scriptsize{(0.355)}} & \mc{1}{c}{\scriptsize{(0.382)}} & \mc{1}{c}{\scriptsize{(1.000)}} & \mc{1}{c}{\scriptsize{(0.513)}} & \mc{1}{c}{\scriptsize{(1.000)}} & \mc{1}{c}{\scriptsize{(0.250)}} & \mc{1}{c}{\scriptsize{(0.368)}} & \mc{1}{c}{\scriptsize{(0.289)}} \\  

    \mc{1}{l}{\scriptsize{Depression}} & \mc{1}{c}{\scriptsize{21}} & \mc{1}{c}{\scriptsize{-0.116}} & \mc{1}{c}{\scriptsize{-0.180}} & \mc{1}{c}{\scriptsize{0.105}} & \mc{1}{c}{\scriptsize{-0.142}} & \mc{1}{c}{\scriptsize{0.119}} & \mc{1}{c}{\scriptsize{-0.078}} & \mc{1}{c}{\scriptsize{-0.180}} & \mc{1}{c}{\scriptsize{-0.117}} \\  

     &  & \mc{1}{c}{\scriptsize{(0.697)}} & \mc{1}{c}{\scriptsize{(0.513)}} & \mc{1}{c}{\scriptsize{(0.908)}} & \mc{1}{c}{\scriptsize{(0.711)}} & \mc{1}{c}{\scriptsize{(0.908)}} & \mc{1}{c}{\scriptsize{(0.724)}} & \mc{1}{c}{\scriptsize{(0.566)}} & \mc{1}{c}{\scriptsize{(0.658)}} \\  

     & \mc{1}{c}{\scriptsize{34}} & \mc{1}{c}{\scriptsize{-0.320}} & \mc{1}{c}{\scriptsize{-0.265}} & \mc{1}{c}{\scriptsize{0.254}} & \mc{1}{c}{\scriptsize{0.257}} & \mc{1}{c}{\scriptsize{0.262}} & \mc{1}{c}{\scriptsize{-0.484}} & \mc{1}{c}{\scriptsize{-0.343}} & \mc{1}{c}{\scriptsize{-0.458}} \\  

     &  & \mc{1}{c}{\scriptsize{(0.605)}} & \mc{1}{c}{\scriptsize{(0.658)}} & \mc{1}{c}{\scriptsize{(1.000)}} & \mc{1}{c}{\scriptsize{(0.974)}} & \mc{1}{c}{\scriptsize{(1.000)}} & \mc{1}{c}{\scriptsize{(0.395)}} & \mc{1}{c}{\scriptsize{(0.671)}} & \mc{1}{c}{\scriptsize{(0.447)}} \\  

    \mc{1}{l}{\scriptsize{Anxiety}} & \mc{1}{c}{\scriptsize{21}} & \mc{1}{c}{\scriptsize{0.119}} & \mc{1}{c}{\scriptsize{0.006}} & \mc{1}{c}{\scriptsize{0.286}} & \mc{1}{c}{\scriptsize{-0.023}} & \mc{1}{c}{\scriptsize{0.278}} & \mc{1}{c}{\scriptsize{0.070}} & \mc{1}{c}{\scriptsize{-0.025}} & \mc{1}{c}{\scriptsize{0.019}} \\  

     &  & \mc{1}{c}{\scriptsize{(1.000)}} & \mc{1}{c}{\scriptsize{(0.895)}} & \mc{1}{c}{\scriptsize{(0.987)}} & \mc{1}{c}{\scriptsize{(0.868)}} & \mc{1}{c}{\scriptsize{(0.987)}} & \mc{1}{c}{\scriptsize{(0.974)}} & \mc{1}{c}{\scriptsize{(0.816)}} & \mc{1}{c}{\scriptsize{(0.908)}} \\  

     & \mc{1}{c}{\scriptsize{34}} & \mc{1}{c}{\scriptsize{-0.415}} & \mc{1}{c}{\scriptsize{-0.399}} & \mc{1}{c}{\scriptsize{0.103}} & \mc{1}{c}{\scriptsize{0.102}} & \mc{1}{c}{\scriptsize{0.117}} & \mc{1}{c}{\scriptsize{-0.564}} & \mc{1}{c}{\scriptsize{-0.471}} & \mc{1}{c}{\scriptsize{-0.571}} \\  

     &  & \mc{1}{c}{\scriptsize{(0.368)}} & \mc{1}{c}{\scriptsize{(0.487)}} & \mc{1}{c}{\scriptsize{(1.000)}} & \mc{1}{c}{\scriptsize{(0.974)}} & \mc{1}{c}{\scriptsize{(1.000)}} & \mc{1}{c}{\scriptsize{(0.276)}} & \mc{1}{c}{\scriptsize{(0.447)}} & \mc{1}{c}{\scriptsize{(0.329)}} \\  

    \mc{1}{l}{\scriptsize{Hostility}} & \mc{1}{c}{\scriptsize{21}} & \mc{1}{c}{\scriptsize{-0.180}} & \mc{1}{c}{\scriptsize{-0.165}} & \mc{1}{c}{\scriptsize{0.118}} & \mc{1}{c}{\scriptsize{-0.073}} & \mc{1}{c}{\scriptsize{0.109}} & \mc{1}{c}{\scriptsize{-0.238}} & \mc{1}{c}{\scriptsize{-0.195}} & \mc{1}{c}{\scriptsize{-0.257}} \\  

     &  & \mc{1}{c}{\scriptsize{(0.605)}} & \mc{1}{c}{\scriptsize{(0.539)}} & \mc{1}{c}{\scriptsize{(0.921)}} & \mc{1}{c}{\scriptsize{(0.842)}} & \mc{1}{c}{\scriptsize{(0.908)}} & \mc{1}{c}{\scriptsize{(0.461)}} & \mc{1}{c}{\scriptsize{(0.553)}} & \mc{1}{c}{\scriptsize{(0.421)}} \\  

     & \mc{1}{c}{\scriptsize{34}} & \mc{1}{c}{\scriptsize{-0.368}} & \mc{1}{c}{\scriptsize{-0.335}} & \mc{1}{c}{\scriptsize{0.143}} & \mc{1}{c}{\scriptsize{0.385}} & \mc{1}{c}{\scriptsize{0.133}} & \mc{1}{c}{\scriptsize{-0.514}} & \mc{1}{c}{\scriptsize{-0.439}} & \mc{1}{c}{\scriptsize{-0.522}} \\  

     &  & \mc{1}{c}{\scriptsize{(0.421)}} & \mc{1}{c}{\scriptsize{(0.539)}} & \mc{1}{c}{\scriptsize{(1.000)}} & \mc{1}{c}{\scriptsize{(0.987)}} & \mc{1}{c}{\scriptsize{(1.000)}} & \mc{1}{c}{\scriptsize{(0.289)}} & \mc{1}{c}{\scriptsize{(0.500)}} & \mc{1}{c}{\scriptsize{(0.355)}} \\  

    \mc{1}{l}{\scriptsize{Global Severity Index}} & \mc{1}{c}{\scriptsize{21}} & \mc{1}{c}{\scriptsize{-0.014}} & \mc{1}{c}{\scriptsize{-0.098}} & \mc{1}{c}{\scriptsize{0.188}} & \mc{1}{c}{\scriptsize{-0.102}} & \mc{1}{c}{\scriptsize{0.189}} & \mc{1}{c}{\scriptsize{-0.046}} & \mc{1}{c}{\scriptsize{-0.107}} & \mc{1}{c}{\scriptsize{-0.075}} \\  

     &  & \mc{1}{c}{\scriptsize{(0.816)}} & \mc{1}{c}{\scriptsize{(0.618)}} & \mc{1}{c}{\scriptsize{(0.974)}} & \mc{1}{c}{\scriptsize{(0.711)}} & \mc{1}{c}{\scriptsize{(0.974)}} & \mc{1}{c}{\scriptsize{(0.763)}} & \mc{1}{c}{\scriptsize{(0.671)}} & \mc{1}{c}{\scriptsize{(0.711)}} \\  

     & \mc{1}{c}{\scriptsize{34}} & \mc{1}{c}{\scriptsize{-7.206}} & \mc{1}{c}{\scriptsize{-6.999}} & \mc{1}{c}{\scriptsize{2.571}} & \mc{1}{c}{\scriptsize{1.777}} & \mc{1}{c}{\scriptsize{2.559}} & \mc{1}{c}{\scriptsize{-10.000}} & \mc{1}{c}{\scriptsize{-8.283}} & \mc{1}{c}{\scriptsize{-10.071}} \\  

     &  & \mc{1}{c}{\scriptsize{(0.421)}} & \mc{1}{c}{\scriptsize{(0.487)}} & \mc{1}{c}{\scriptsize{(1.000)}} & \mc{1}{c}{\scriptsize{(0.961)}} & \mc{1}{c}{\scriptsize{(1.000)}} & \mc{1}{c}{\scriptsize{(0.289)}} & \mc{1}{c}{\scriptsize{(0.474)}} & \mc{1}{c}{\scriptsize{(0.342)}} \\  

    \mc{1}{l}{\scriptsize{BSI Factor}} & \mc{1}{c}{\scriptsize{21 and 34}} & \mc{1}{c}{\scriptsize{-0.260}} & \mc{1}{c}{\scriptsize{-0.083}} & \mc{1}{c}{\scriptsize{0.240}} & \mc{1}{c}{\scriptsize{-0.012}} & \mc{1}{c}{\scriptsize{0.279}} & \mc{1}{c}{\scriptsize{-0.343}} & \mc{1}{c}{\scriptsize{-0.150}} & \mc{1}{c}{\scriptsize{-0.186}} \\  

     &  & \mc{1}{c}{\scriptsize{(0.645)}} & \mc{1}{c}{\scriptsize{(0.803)}} & \mc{1}{c}{\scriptsize{(1.000)}} & \mc{1}{c}{\scriptsize{(0.882)}} & \mc{1}{c}{\scriptsize{(1.000)}} & \mc{1}{c}{\scriptsize{(0.579)}} & \mc{1}{c}{\scriptsize{(0.803)}} & \mc{1}{c}{\scriptsize{(0.684)}} \\  

  \bottomrule
  \end{tabular}
	\end{table} 

	\begin{table}[H]
     \caption{Treatment Effects on Relation with Spouse, Male Sample}
     \label{table:abccare_rslt_male_cat19_sd}
	  \begin{tabular}{cccccccccc}
  \toprule

    \scriptsize{Variable} & \scriptsize{Age} & \scriptsize{(1)} & \scriptsize{(2)} & \scriptsize{(3)} & \scriptsize{(4)} & \scriptsize{(5)} & \scriptsize{(6)} & \scriptsize{(7)} & \scriptsize{(8)} \\ 
    \midrule  

    \mc{1}{l}{\scriptsize{Participates in Activity}} & \mc{1}{c}{\scriptsize{12}} & \mc{1}{c}{\scriptsize{0.231}} & \mc{1}{c}{\scriptsize{0.142}} & \mc{1}{c}{\scriptsize{0.131}} & \mc{1}{c}{\scriptsize{0.096}} & \mc{1}{c}{\scriptsize{0.109}} & \mc{1}{c}{\scriptsize{0.260}} & \mc{1}{c}{\scriptsize{0.195}} & \mc{1}{c}{\scriptsize{0.227}} \\  

     &  & \mc{1}{c}{\scriptsize{(0.500)}} & \mc{1}{c}{\scriptsize{(0.974)}} & \mc{1}{c}{\scriptsize{(1.000)}} & \mc{1}{c}{\scriptsize{(0.987)}} & \mc{1}{c}{\scriptsize{(1.000)}} & \mc{1}{c}{\scriptsize{(0.553)}} & \mc{1}{c}{\scriptsize{(0.855)}} & \mc{1}{c}{\scriptsize{(0.750)}} \\  

    \mc{1}{l}{\scriptsize{Time spent reading}} & \mc{1}{c}{\scriptsize{12}} & \mc{1}{c}{\scriptsize{0.073}} & \mc{1}{c}{\scriptsize{-0.181}} & \mc{1}{c}{\scriptsize{0.973}} & \mc{1}{c}{\scriptsize{1.308}} & \mc{1}{c}{\scriptsize{1.119}} & \mc{1}{c}{\scriptsize{-0.139}} & \mc{1}{c}{\scriptsize{-0.371}} & \mc{1}{c}{\scriptsize{-0.062}} \\  

     &  & \mc{1}{c}{\scriptsize{(0.987)}} & \mc{1}{c}{\scriptsize{(1.000)}} & \mc{1}{c}{\scriptsize{(1.000)}} & \mc{1}{c}{\scriptsize{(0.987)}} & \mc{1}{c}{\scriptsize{(1.000)}} & \mc{1}{c}{\scriptsize{(1.000)}} & \mc{1}{c}{\scriptsize{(1.000)}} & \mc{1}{c}{\scriptsize{(1.000)}} \\  

    \mc{1}{l}{\scriptsize{Good Description of Self}} & \mc{1}{c}{\scriptsize{12}} & \mc{1}{c}{\scriptsize{0.059}} & \mc{1}{c}{\scriptsize{0.011}} & \mc{1}{c}{\scriptsize{0.023}} & \mc{1}{c}{\scriptsize{-0.290}} & \mc{1}{c}{\scriptsize{-0.068}} & \mc{1}{c}{\scriptsize{0.070}} & \mc{1}{c}{\scriptsize{0.075}} & \mc{1}{c}{\scriptsize{-0.052}} \\  

     &  & \mc{1}{c}{\scriptsize{(0.974)}} & \mc{1}{c}{\scriptsize{(1.000)}} & \mc{1}{c}{\scriptsize{(1.000)}} & \mc{1}{c}{\scriptsize{(1.000)}} & \mc{1}{c}{\scriptsize{(1.000)}} & \mc{1}{c}{\scriptsize{(0.974)}} & \mc{1}{c}{\scriptsize{(0.987)}} & \mc{1}{c}{\scriptsize{(1.000)}} \\  

    \mc{1}{l}{\scriptsize{Views Self as Dumb}} & \mc{1}{c}{\scriptsize{12}} & \mc{1}{c}{\scriptsize{-0.084}} & \mc{1}{c}{\scriptsize{-0.065}} & \mc{1}{c}{\scriptsize{-0.139}} & \mc{1}{c}{\scriptsize{-0.241}} & \mc{1}{c}{\scriptsize{-0.145}} & \mc{1}{c}{\scriptsize{-0.068}} & \mc{1}{c}{\scriptsize{-0.053}} & \mc{1}{c}{\scriptsize{-0.069}} \\  

     &  & \mc{1}{c}{\scriptsize{(0.974)}} & \mc{1}{c}{\scriptsize{(0.987)}} & \mc{1}{c}{\scriptsize{(1.000)}} & \mc{1}{c}{\scriptsize{(0.987)}} & \mc{1}{c}{\scriptsize{(1.000)}} & \mc{1}{c}{\scriptsize{(0.974)}} & \mc{1}{c}{\scriptsize{(1.000)}} & \mc{1}{c}{\scriptsize{(0.987)}} \\  

    \mc{1}{l}{\scriptsize{Views Self as Clumsy}} & \mc{1}{c}{\scriptsize{12}} & \mc{1}{c}{\scriptsize{-0.091}} & \mc{1}{c}{\scriptsize{-0.020}} &  &  &  & \mc{1}{c}{\scriptsize{-0.118}} & \mc{1}{c}{\scriptsize{-0.046}} & \mc{1}{c}{\scriptsize{-0.091}} \\  

     &  & \mc{1}{c}{\scriptsize{(0.263)}} & \mc{1}{c}{\scriptsize{(0.987)}} &  &  &  & \mc{1}{c}{\scriptsize{(0.382)}} & \mc{1}{c}{\scriptsize{(0.947)}} & \mc{1}{c}{\scriptsize{(0.382)}} \\  

    \mc{1}{l}{\scriptsize{Views Self as Not Liked}} & \mc{1}{c}{\scriptsize{12}} & \mc{1}{c}{\scriptsize{-0.119}} & \mc{1}{c}{\scriptsize{0.012}} & \mc{1}{c}{\scriptsize{-0.046}} & \mc{1}{c}{\scriptsize{-0.131}} & \mc{1}{c}{\scriptsize{-0.134}} & \mc{1}{c}{\scriptsize{-0.140}} & \mc{1}{c}{\scriptsize{0.016}} & \mc{1}{c}{\scriptsize{-0.172}} \\  

     &  & \mc{1}{c}{\scriptsize{(0.974)}} & \mc{1}{c}{\scriptsize{(1.000)}} & \mc{1}{c}{\scriptsize{(1.000)}} & \mc{1}{c}{\scriptsize{(0.987)}} & \mc{1}{c}{\scriptsize{(1.000)}} & \mc{1}{c}{\scriptsize{(0.961)}} & \mc{1}{c}{\scriptsize{(1.000)}} & \mc{1}{c}{\scriptsize{(0.711)}} \\  

    \mc{1}{l}{\scriptsize{Proud about Self}} & \mc{1}{c}{\scriptsize{12}} & \mc{1}{c}{\scriptsize{-0.126}} & \mc{1}{c}{\scriptsize{-0.114}} & \mc{1}{c}{\scriptsize{-0.008}} & \mc{1}{c}{\scriptsize{0.091}} & \mc{1}{c}{\scriptsize{0.023}} & \mc{1}{c}{\scriptsize{-0.161}} & \mc{1}{c}{\scriptsize{-0.158}} & \mc{1}{c}{\scriptsize{-0.150}} \\  

     &  & \mc{1}{c}{\scriptsize{(1.000)}} & \mc{1}{c}{\scriptsize{(1.000)}} & \mc{1}{c}{\scriptsize{(1.000)}} & \mc{1}{c}{\scriptsize{(0.987)}} & \mc{1}{c}{\scriptsize{(1.000)}} & \mc{1}{c}{\scriptsize{(1.000)}} & \mc{1}{c}{\scriptsize{(1.000)}} & \mc{1}{c}{\scriptsize{(1.000)}} \\  

    \mc{1}{l}{\scriptsize{Family Proud of You}} & \mc{1}{c}{\scriptsize{12}} & \mc{1}{c}{\scriptsize{-0.073}} & \mc{1}{c}{\scriptsize{-0.021}} & \mc{1}{c}{\scriptsize{-0.046}} & \mc{1}{c}{\scriptsize{-0.019}} & \mc{1}{c}{\scriptsize{-0.063}} & \mc{1}{c}{\scriptsize{-0.081}} & \mc{1}{c}{\scriptsize{-0.043}} & \mc{1}{c}{\scriptsize{-0.080}} \\  

     &  & \mc{1}{c}{\scriptsize{(1.000)}} & \mc{1}{c}{\scriptsize{(1.000)}} & \mc{1}{c}{\scriptsize{(1.000)}} & \mc{1}{c}{\scriptsize{(1.000)}} & \mc{1}{c}{\scriptsize{(1.000)}} & \mc{1}{c}{\scriptsize{(1.000)}} & \mc{1}{c}{\scriptsize{(1.000)}} & \mc{1}{c}{\scriptsize{(1.000)}} \\  

    \mc{1}{l}{\scriptsize{Feels Inadequate, Inferior}} & \mc{1}{c}{\scriptsize{12}} & \mc{1}{c}{\scriptsize{0.105}} & \mc{1}{c}{\scriptsize{0.261}} & \mc{1}{c}{\scriptsize{0.423}} & \mc{1}{c}{\scriptsize{0.658}} & \mc{1}{c}{\scriptsize{0.441}} & \mc{1}{c}{\scriptsize{0.011}} & \mc{1}{c}{\scriptsize{0.189}} & \mc{1}{c}{\scriptsize{0.007}} \\  

     &  & \mc{1}{c}{\scriptsize{(1.000)}} & \mc{1}{c}{\scriptsize{(1.000)}} & \mc{1}{c}{\scriptsize{(1.000)}} & \mc{1}{c}{\scriptsize{(1.000)}} & \mc{1}{c}{\scriptsize{(1.000)}} & \mc{1}{c}{\scriptsize{(1.000)}} & \mc{1}{c}{\scriptsize{(1.000)}} & \mc{1}{c}{\scriptsize{(1.000)}} \\  

    \mc{1}{l}{\scriptsize{Withdraws Excessively}} & \mc{1}{c}{\scriptsize{12}} & \mc{1}{c}{\scriptsize{0.070}} & \mc{1}{c}{\scriptsize{-0.038}} & \mc{1}{c}{\scriptsize{0.215}} & \mc{1}{c}{\scriptsize{0.053}} & \mc{1}{c}{\scriptsize{0.188}} & \mc{1}{c}{\scriptsize{0.027}} & \mc{1}{c}{\scriptsize{-0.032}} & \mc{1}{c}{\scriptsize{-0.090}} \\  

     &  & \mc{1}{c}{\scriptsize{(1.000)}} & \mc{1}{c}{\scriptsize{(1.000)}} & \mc{1}{c}{\scriptsize{(1.000)}} & \mc{1}{c}{\scriptsize{(1.000)}} & \mc{1}{c}{\scriptsize{(1.000)}} & \mc{1}{c}{\scriptsize{(1.000)}} & \mc{1}{c}{\scriptsize{(1.000)}} & \mc{1}{c}{\scriptsize{(0.987)}} \\  

    \mc{1}{l}{\scriptsize{Ignores Situation}} & \mc{1}{c}{\scriptsize{12}} & \mc{1}{c}{\scriptsize{-0.248}} & \mc{1}{c}{\scriptsize{-0.373}} & \mc{1}{c}{\scriptsize{-0.585}} & \mc{1}{c}{\scriptsize{-0.730}} & \mc{1}{c}{\scriptsize{-0.509}} & \mc{1}{c}{\scriptsize{-0.149}} & \mc{1}{c}{\scriptsize{-0.279}} & \mc{1}{c}{\scriptsize{-0.189}} \\  

     &  & \mc{1}{c}{\scriptsize{(0.697)}} & \mc{1}{c}{\scriptsize{(0.408)}} & \mc{1}{c}{\scriptsize{(0.500)}} & \mc{1}{c}{\scriptsize{(0.487)}} & \mc{1}{c}{\scriptsize{(0.539)}} & \mc{1}{c}{\scriptsize{(0.974)}} & \mc{1}{c}{\scriptsize{(0.697)}} & \mc{1}{c}{\scriptsize{(0.947)}} \\  

    \mc{1}{l}{\scriptsize{Not Cope with Prob.}} & \mc{1}{c}{\scriptsize{12}} & \mc{1}{c}{\scriptsize{-0.098}} & \mc{1}{c}{\scriptsize{-0.181}} & \mc{1}{c}{\scriptsize{-0.061}} & \mc{1}{c}{\scriptsize{-0.034}} & \mc{1}{c}{\scriptsize{0.025}} & \mc{1}{c}{\scriptsize{-0.109}} & \mc{1}{c}{\scriptsize{-0.152}} & \mc{1}{c}{\scriptsize{-0.085}} \\  

     &  & \mc{1}{c}{\scriptsize{(0.974)}} & \mc{1}{c}{\scriptsize{(0.882)}} & \mc{1}{c}{\scriptsize{(1.000)}} & \mc{1}{c}{\scriptsize{(1.000)}} & \mc{1}{c}{\scriptsize{(1.000)}} & \mc{1}{c}{\scriptsize{(0.974)}} & \mc{1}{c}{\scriptsize{(0.974)}} & \mc{1}{c}{\scriptsize{(0.974)}} \\  

    \mc{1}{l}{\scriptsize{Often Mad of Angry}} & \mc{1}{c}{\scriptsize{12}} & \mc{1}{c}{\scriptsize{-0.142}} & \mc{1}{c}{\scriptsize{-0.160}} & \mc{1}{c}{\scriptsize{0.040}} & \mc{1}{c}{\scriptsize{0.045}} & \mc{1}{c}{\scriptsize{0.043}} & \mc{1}{c}{\scriptsize{-0.195}} & \mc{1}{c}{\scriptsize{-0.195}} & \mc{1}{c}{\scriptsize{-0.191}} \\  

     &  & \mc{1}{c}{\scriptsize{(0.921)}} & \mc{1}{c}{\scriptsize{(0.974)}} & \mc{1}{c}{\scriptsize{(1.000)}} & \mc{1}{c}{\scriptsize{(1.000)}} & \mc{1}{c}{\scriptsize{(1.000)}} & \mc{1}{c}{\scriptsize{(0.882)}} & \mc{1}{c}{\scriptsize{(0.934)}} & \mc{1}{c}{\scriptsize{(0.934)}} \\  

    \mc{1}{l}{\scriptsize{Impulsivity}} & \mc{1}{c}{\scriptsize{12}} & \mc{1}{c}{\scriptsize{-0.049}} & \mc{1}{c}{\scriptsize{0.006}} & \mc{1}{c}{\scriptsize{-0.031}} & \mc{1}{c}{\scriptsize{-0.274}} & \mc{1}{c}{\scriptsize{-0.012}} & \mc{1}{c}{\scriptsize{-0.054}} & \mc{1}{c}{\scriptsize{0.015}} & \mc{1}{c}{\scriptsize{-0.042}} \\  

     &  & \mc{1}{c}{\scriptsize{(0.974)}} & \mc{1}{c}{\scriptsize{(1.000)}} & \mc{1}{c}{\scriptsize{(1.000)}} & \mc{1}{c}{\scriptsize{(0.803)}} & \mc{1}{c}{\scriptsize{(1.000)}} & \mc{1}{c}{\scriptsize{(0.974)}} & \mc{1}{c}{\scriptsize{(1.000)}} & \mc{1}{c}{\scriptsize{(0.987)}} \\  

    \mc{1}{l}{\scriptsize{Significant Fears}} & \mc{1}{c}{\scriptsize{12}} & \mc{1}{c}{\scriptsize{-0.105}} & \mc{1}{c}{\scriptsize{-0.037}} & \mc{1}{c}{\scriptsize{0.177}} & \mc{1}{c}{\scriptsize{0.234}} & \mc{1}{c}{\scriptsize{0.228}} & \mc{1}{c}{\scriptsize{-0.188}} & \mc{1}{c}{\scriptsize{-0.107}} & \mc{1}{c}{\scriptsize{-0.188}} \\  

     &  & \mc{1}{c}{\scriptsize{(0.947)}} & \mc{1}{c}{\scriptsize{(0.987)}} & \mc{1}{c}{\scriptsize{(1.000)}} & \mc{1}{c}{\scriptsize{(1.000)}} & \mc{1}{c}{\scriptsize{(1.000)}} & \mc{1}{c}{\scriptsize{(0.566)}} & \mc{1}{c}{\scriptsize{(0.974)}} & \mc{1}{c}{\scriptsize{(0.645)}} \\  

    \mc{1}{l}{\scriptsize{Denies Any Worries}} & \mc{1}{c}{\scriptsize{12}} & \mc{1}{c}{\scriptsize{-0.325}} & \mc{1}{c}{\scriptsize{-0.371}} & \mc{1}{c}{\scriptsize{-0.162}} & \mc{1}{c}{\scriptsize{-0.125}} & \mc{1}{c}{\scriptsize{-0.150}} & \mc{1}{c}{\scriptsize{-0.373}} & \mc{1}{c}{\scriptsize{-0.413}} & \mc{1}{c}{\scriptsize{-0.425}} \\  

     &  & \mc{1}{c}{\scriptsize{\textbf{(0.092)}}} & \mc{1}{c}{\scriptsize{(0.118)}} & \mc{1}{c}{\scriptsize{(0.961)}} & \mc{1}{c}{\scriptsize{(0.974)}} & \mc{1}{c}{\scriptsize{(0.987)}} & \mc{1}{c}{\scriptsize{\textbf{(0.079)}}} & \mc{1}{c}{\scriptsize{(0.145)}} & \mc{1}{c}{\scriptsize{\textbf{(0.053)}}} \\  

  \bottomrule
  \end{tabular}
	\end{table} 

	\begin{table}[H]
     \caption{Treatment Effects on Spouse Characteristics, Male Sample}
     \label{table:abccare_rslt_male_cat20_sd}
	  \begin{tabular}{cccccccccc}
  \toprule

    \scriptsize{Variable} & \scriptsize{Age} & \scriptsize{(1)} & \scriptsize{(2)} & \scriptsize{(3)} & \scriptsize{(4)} & \scriptsize{(5)} & \scriptsize{(6)} & \scriptsize{(7)} & \scriptsize{(8)} \\ 
    \midrule  

    \mc{1}{l}{\scriptsize{Spouse annual income}} & \mc{1}{c}{\scriptsize{30}} & \mc{1}{c}{\scriptsize{-2,314}} & \mc{1}{c}{\scriptsize{-1,336}} & \mc{1}{c}{\scriptsize{-14,714}} & \mc{1}{c}{\scriptsize{-34,831}} & \mc{1}{c}{\scriptsize{-16,140}} & \mc{1}{c}{\scriptsize{-2,339}} & \mc{1}{c}{\scriptsize{-1,579}} & \mc{1}{c}{\scriptsize{-859}} \\  

     &  & \mc{1}{c}{\scriptsize{(0.763)}} & \mc{1}{c}{\scriptsize{(0.658)}} & \mc{1}{c}{\scriptsize{(1.000)}} & \mc{1}{c}{\scriptsize{(0.905)}} & \mc{1}{c}{\scriptsize{(1.000)}} & \mc{1}{c}{\scriptsize{(0.789)}} & \mc{1}{c}{\scriptsize{(0.645)}} & \mc{1}{c}{\scriptsize{(0.592)}} \\  

    \mc{1}{l}{\scriptsize{Spouse employment status}} & \mc{1}{c}{\scriptsize{30}} & \mc{1}{c}{\scriptsize{-0.232}} & \mc{1}{c}{\scriptsize{-0.318}} & \mc{1}{c}{\scriptsize{-0.357}} & \mc{1}{c}{\scriptsize{-0.778}} & \mc{1}{c}{\scriptsize{-0.483}} & \mc{1}{c}{\scriptsize{-0.280}} & \mc{1}{c}{\scriptsize{-0.350}} & \mc{1}{c}{\scriptsize{-0.351}} \\  

     &  & \mc{1}{c}{\scriptsize{(1.000)}} & \mc{1}{c}{\scriptsize{(1.000)}} & \mc{1}{c}{\scriptsize{(1.000)}} & \mc{1}{c}{\scriptsize{(0.968)}} & \mc{1}{c}{\scriptsize{(1.000)}} & \mc{1}{c}{\scriptsize{(1.000)}} & \mc{1}{c}{\scriptsize{(1.000)}} & \mc{1}{c}{\scriptsize{(0.974)}} \\  

  \bottomrule
  \end{tabular}
	\end{table} 

	\begin{table}[H]
     \caption{Treatment Effects on Subject Home and Property, Male Sample}
     \label{table:abccare_rslt_male_cat21_sd}
	  \begin{tabular}{cccccccccc}
  \toprule

    \scriptsize{Variable} & \scriptsize{Age} & \scriptsize{(1)} & \scriptsize{(2)} & \scriptsize{(3)} & \scriptsize{(4)} & \scriptsize{(5)} & \scriptsize{(6)} & \scriptsize{(7)} & \scriptsize{(8)} \\ 
    \midrule  

    \mc{1}{l}{\scriptsize{Ever Adopted}} & \mc{1}{c}{\scriptsize{nan}} & \mc{1}{c}{\scriptsize{0.028}} & \mc{1}{c}{\scriptsize{0.030}} & \mc{1}{c}{\scriptsize{-0.129}} & \mc{1}{c}{\scriptsize{-0.159}} & \mc{1}{c}{\scriptsize{-0.160}} & \mc{1}{c}{\scriptsize{0.071}} & \mc{1}{c}{\scriptsize{0.062}} & \mc{1}{c}{\scriptsize{0.042}} \\  

     &  & \mc{1}{c}{\scriptsize{(0.316)}} & \mc{1}{c}{\scriptsize{(0.316)}} & \mc{1}{c}{\scriptsize{(0.711)}} & \mc{1}{c}{\scriptsize{(0.737)}} & \mc{1}{c}{\scriptsize{(0.697)}} & \mc{1}{c}{\scriptsize{\textbf{(0.039)}}} & \mc{1}{c}{\scriptsize{\textbf{(0.079)}}} & \mc{1}{c}{\scriptsize{\textbf{(0.053)}}} \\  

  \bottomrule
  \end{tabular}
	\end{table} 

	\begin{table}[H]
     \caption{Treatment Effects on Education, Male Sample}
     \label{table:abccare_rslt_male_cat22_sd}
	  \begin{tabular}{cccccccccc}
  \toprule

    \scriptsize{Variable} & \scriptsize{Age} & \scriptsize{(1)} & \scriptsize{(2)} & \scriptsize{(3)} & \scriptsize{(4)} & \scriptsize{(5)} & \scriptsize{(6)} & \scriptsize{(7)} & \scriptsize{(8)} \\ 
    \midrule  

    \mc{1}{l}{\scriptsize{Years of Edu.}} & \mc{1}{c}{\scriptsize{30}} & \mc{1}{c}{\scriptsize{0.525}} & \mc{1}{c}{\scriptsize{0.708}} & \mc{1}{c}{\scriptsize{0.857}} & \mc{1}{c}{\scriptsize{1.302}} & \mc{1}{c}{\scriptsize{0.791}} & \mc{1}{c}{\scriptsize{0.385}} & \mc{1}{c}{\scriptsize{0.540}} & \mc{1}{c}{\scriptsize{0.347}} \\  

     &  & \mc{1}{c}{\scriptsize{(0.307)}} & \mc{1}{c}{\scriptsize{(0.280)}} & \mc{1}{c}{\scriptsize{(0.315)}} & \mc{1}{c}{\scriptsize{(0.151)}} & \mc{1}{c}{\scriptsize{(0.356)}} & \mc{1}{c}{\scriptsize{(0.459)}} & \mc{1}{c}{\scriptsize{(0.392)}} & \mc{1}{c}{\scriptsize{(0.608)}} \\  

    \mc{1}{l}{\scriptsize{Graduated High School}} & \mc{1}{c}{\scriptsize{30}} & \mc{1}{c}{\scriptsize{0.073}} & \mc{1}{c}{\scriptsize{0.109}} & \mc{1}{c}{\scriptsize{0.114}} & \mc{1}{c}{\scriptsize{0.113}} & \mc{1}{c}{\scriptsize{0.085}} & \mc{1}{c}{\scriptsize{0.077}} & \mc{1}{c}{\scriptsize{0.102}} & \mc{1}{c}{\scriptsize{0.064}} \\  

     &  & \mc{1}{c}{\scriptsize{(0.547)}} & \mc{1}{c}{\scriptsize{(0.533)}} & \mc{1}{c}{\scriptsize{(0.548)}} & \mc{1}{c}{\scriptsize{(0.726)}} & \mc{1}{c}{\scriptsize{(0.644)}} & \mc{1}{c}{\scriptsize{(0.500)}} & \mc{1}{c}{\scriptsize{(0.554)}} & \mc{1}{c}{\scriptsize{(0.635)}} \\  

    \mc{1}{l}{\scriptsize{Graduated 4-year College}} & \mc{1}{c}{\scriptsize{30}} & \mc{1}{c}{\scriptsize{0.170}} & \mc{1}{c}{\scriptsize{0.169}} & \mc{1}{c}{\scriptsize{0.124}} & \mc{1}{c}{\scriptsize{0.232}} & \mc{1}{c}{\scriptsize{0.102}} & \mc{1}{c}{\scriptsize{0.179}} & \mc{1}{c}{\scriptsize{0.159}} & \mc{1}{c}{\scriptsize{0.143}} \\  

     &  & \mc{1}{c}{\scriptsize{(0.133)}} & \mc{1}{c}{\scriptsize{(0.307)}} & \mc{1}{c}{\scriptsize{(0.521)}} & \mc{1}{c}{\scriptsize{(0.562)}} & \mc{1}{c}{\scriptsize{(0.548)}} & \mc{1}{c}{\scriptsize{(0.108)}} & \mc{1}{c}{\scriptsize{(0.392)}} & \mc{1}{c}{\scriptsize{(0.351)}} \\  

    \mc{1}{l}{\scriptsize{Attended Voc./Tech./Com. College}} & \mc{1}{c}{\scriptsize{30}} & \mc{1}{c}{\scriptsize{-0.099}} & \mc{1}{c}{\scriptsize{-0.124}} & \mc{1}{c}{\scriptsize{0.086}} & \mc{1}{c}{\scriptsize{0.258}} & \mc{1}{c}{\scriptsize{0.024}} & \mc{1}{c}{\scriptsize{-0.138}} & \mc{1}{c}{\scriptsize{-0.231}} & \mc{1}{c}{\scriptsize{-0.233}} \\  

     &  & \mc{1}{c}{\scriptsize{(0.987)}} & \mc{1}{c}{\scriptsize{(0.987)}} & \mc{1}{c}{\scriptsize{(0.603)}} & \mc{1}{c}{\scriptsize{(0.301)}} & \mc{1}{c}{\scriptsize{(0.767)}} & \mc{1}{c}{\scriptsize{(1.000)}} & \mc{1}{c}{\scriptsize{(1.000)}} & \mc{1}{c}{\scriptsize{(1.000)}} \\  

  \bottomrule
  \end{tabular}
	\end{table} 

	\begin{table}[H]
     \caption{Treatment Effects on Subject Employment and Income, Male Sample}
     \label{table:abccare_rslt_male_cat23_sd}
	  \begin{tabular}{cccccccccc}
  \toprule

    \scriptsize{Variable} & \scriptsize{Age} & \scriptsize{(1)} & \scriptsize{(2)} & \scriptsize{(3)} & \scriptsize{(4)} & \scriptsize{(5)} & \scriptsize{(6)} & \scriptsize{(7)} & \scriptsize{(8)} \\ 
    \midrule  

    \mc{1}{l}{\scriptsize{Father at Home}} & \mc{1}{c}{\scriptsize{3}} & \mc{1}{c}{\scriptsize{-0.076}} & \mc{1}{c}{\scriptsize{0.007}} & \mc{1}{c}{\scriptsize{-0.243}} & \mc{1}{c}{\scriptsize{0.026}} & \mc{1}{c}{\scriptsize{-0.197}} & \mc{1}{c}{\scriptsize{-0.029}} & \mc{1}{c}{\scriptsize{0.010}} & \mc{1}{c}{\scriptsize{0.070}} \\  

     &  & \mc{1}{c}{\scriptsize{(0.895)}} & \mc{1}{c}{\scriptsize{(0.803)}} & \mc{1}{c}{\scriptsize{(1.000)}} & \mc{1}{c}{\scriptsize{(0.827)}} & \mc{1}{c}{\scriptsize{(0.973)}} & \mc{1}{c}{\scriptsize{(0.763)}} & \mc{1}{c}{\scriptsize{(0.789)}} & \mc{1}{c}{\scriptsize{(0.500)}} \\  

     & \mc{1}{c}{\scriptsize{2}} & \mc{1}{c}{\scriptsize{-0.018}} & \mc{1}{c}{\scriptsize{0.160}} & \mc{1}{c}{\scriptsize{-0.282}} & \mc{1}{c}{\scriptsize{0.111}} & \mc{1}{c}{\scriptsize{-0.223}} & \mc{1}{c}{\scriptsize{0.057}} & \mc{1}{c}{\scriptsize{0.187}} & \mc{1}{c}{\scriptsize{0.171}} \\  

     &  & \mc{1}{c}{\scriptsize{(0.803)}} & \mc{1}{c}{\scriptsize{(0.184)}} & \mc{1}{c}{\scriptsize{(1.000)}} & \mc{1}{c}{\scriptsize{(0.667)}} & \mc{1}{c}{\scriptsize{(0.973)}} & \mc{1}{c}{\scriptsize{(0.539)}} & \mc{1}{c}{\scriptsize{(0.184)}} & \mc{1}{c}{\scriptsize{(0.158)}} \\  

     & \mc{1}{c}{\scriptsize{8}} & \mc{1}{c}{\scriptsize{0.037}} & \mc{1}{c}{\scriptsize{0.038}} & \mc{1}{c}{\scriptsize{-0.177}} & \mc{1}{c}{\scriptsize{0.012}} & \mc{1}{c}{\scriptsize{-0.295}} & \mc{1}{c}{\scriptsize{0.123}} & \mc{1}{c}{\scriptsize{0.062}} & \mc{1}{c}{\scriptsize{0.128}} \\  

     &  & \mc{1}{c}{\scriptsize{(0.579)}} & \mc{1}{c}{\scriptsize{(0.724)}} & \mc{1}{c}{\scriptsize{(0.973)}} & \mc{1}{c}{\scriptsize{(0.867)}} & \mc{1}{c}{\scriptsize{(1.000)}} & \mc{1}{c}{\scriptsize{(0.342)}} & \mc{1}{c}{\scriptsize{(0.579)}} & \mc{1}{c}{\scriptsize{(0.289)}} \\  

     & \mc{1}{c}{\scriptsize{5}} & \mc{1}{c}{\scriptsize{-0.057}} & \mc{1}{c}{\scriptsize{0.051}} & \mc{1}{c}{\scriptsize{-0.429}} & \mc{1}{c}{\scriptsize{-0.081}} & \mc{1}{c}{\scriptsize{-0.376}} & \mc{1}{c}{\scriptsize{0.036}} & \mc{1}{c}{\scriptsize{0.089}} & \mc{1}{c}{\scriptsize{0.142}} \\  

     &  & \mc{1}{c}{\scriptsize{(0.882)}} & \mc{1}{c}{\scriptsize{(0.645)}} & \mc{1}{c}{\scriptsize{(1.000)}} & \mc{1}{c}{\scriptsize{(0.920)}} & \mc{1}{c}{\scriptsize{(1.000)}} & \mc{1}{c}{\scriptsize{(0.618)}} & \mc{1}{c}{\scriptsize{(0.474)}} & \mc{1}{c}{\scriptsize{(0.250)}} \\  

     & \mc{1}{c}{\scriptsize{4}} & \mc{1}{c}{\scriptsize{-0.075}} & \mc{1}{c}{\scriptsize{0.008}} & \mc{1}{c}{\scriptsize{-0.339}} & \mc{1}{c}{\scriptsize{-0.050}} & \mc{1}{c}{\scriptsize{-0.287}} &  & \mc{1}{c}{\scriptsize{0.036}} & \mc{1}{c}{\scriptsize{0.103}} \\  

     &  & \mc{1}{c}{\scriptsize{(0.895)}} & \mc{1}{c}{\scriptsize{(0.803)}} & \mc{1}{c}{\scriptsize{(1.000)}} & \mc{1}{c}{\scriptsize{(0.920)}} & \mc{1}{c}{\scriptsize{(1.000)}} &  & \mc{1}{c}{\scriptsize{(0.711)}} & \mc{1}{c}{\scriptsize{(0.382)}} \\  

  \bottomrule
  \end{tabular}
	\end{table} 

	\begin{table}[H]
     \caption{Treatment Effects on Job Attitude, Male Sample}
     \label{table:abccare_rslt_male_cat24_sd}
	  \begin{tabular}{cccccccccc}
  \toprule

    \scriptsize{Variable} & \scriptsize{Age} & \scriptsize{(1)} & \scriptsize{(2)} & \scriptsize{(3)} & \scriptsize{(4)} & \scriptsize{(5)} & \scriptsize{(6)} & \scriptsize{(7)} & \scriptsize{(8)} \\ 
    \midrule  

    \mc{1}{l}{\scriptsize{HOME Score}} & \mc{1}{c}{\scriptsize{3.5}} & \mc{1}{c}{\scriptsize{1.404}} & \mc{1}{c}{\scriptsize{0.668}} & \mc{1}{c}{\scriptsize{2.897}} & \mc{1}{c}{\scriptsize{1.917}} & \mc{1}{c}{\scriptsize{2.835}} & \mc{1}{c}{\scriptsize{0.962}} & \mc{1}{c}{\scriptsize{0.762}} & \mc{1}{c}{\scriptsize{0.743}} \\  

     &  & \mc{1}{c}{\scriptsize{(0.618)}} & \mc{1}{c}{\scriptsize{(0.816)}} & \mc{1}{c}{\scriptsize{(0.527)}} & \mc{1}{c}{\scriptsize{(0.676)}} & \mc{1}{c}{\scriptsize{(0.541)}} & \mc{1}{c}{\scriptsize{(0.776)}} & \mc{1}{c}{\scriptsize{(0.737)}} & \mc{1}{c}{\scriptsize{(0.868)}} \\  

     & \mc{1}{c}{\scriptsize{1.5}} & \mc{1}{c}{\scriptsize{-0.500}} & \mc{1}{c}{\scriptsize{-0.874}} & \mc{1}{c}{\scriptsize{0.431}} & \mc{1}{c}{\scriptsize{-1.123}} & \mc{1}{c}{\scriptsize{0.216}} & \mc{1}{c}{\scriptsize{-0.766}} & \mc{1}{c}{\scriptsize{-0.865}} & \mc{1}{c}{\scriptsize{-0.872}} \\  

     &  & \mc{1}{c}{\scriptsize{(0.987)}} & \mc{1}{c}{\scriptsize{(0.987)}} & \mc{1}{c}{\scriptsize{(0.757)}} & \mc{1}{c}{\scriptsize{(1.000)}} & \mc{1}{c}{\scriptsize{(0.851)}} & \mc{1}{c}{\scriptsize{(0.987)}} & \mc{1}{c}{\scriptsize{(1.000)}} & \mc{1}{c}{\scriptsize{(1.000)}} \\  

     & \mc{1}{c}{\scriptsize{4.5}} & \mc{1}{c}{\scriptsize{1.146}} & \mc{1}{c}{\scriptsize{1.534}} & \mc{1}{c}{\scriptsize{3.312}} & \mc{1}{c}{\scriptsize{3.894}} & \mc{1}{c}{\scriptsize{2.807}} & \mc{1}{c}{\scriptsize{0.527}} & \mc{1}{c}{\scriptsize{1.279}} & \mc{1}{c}{\scriptsize{0.232}} \\  

     &  & \mc{1}{c}{\scriptsize{(0.605)}} & \mc{1}{c}{\scriptsize{(0.632)}} & \mc{1}{c}{\scriptsize{(0.419)}} & \mc{1}{c}{\scriptsize{(0.230)}} & \mc{1}{c}{\scriptsize{(0.446)}} & \mc{1}{c}{\scriptsize{(0.803)}} & \mc{1}{c}{\scriptsize{(0.618)}} & \mc{1}{c}{\scriptsize{(0.895)}} \\  

     & \mc{1}{c}{\scriptsize{2.5}} & \mc{1}{c}{\scriptsize{0.141}} & \mc{1}{c}{\scriptsize{0.886}} & \mc{1}{c}{\scriptsize{1.654}} & \mc{1}{c}{\scriptsize{3.806}} & \mc{1}{c}{\scriptsize{2.232}} & \mc{1}{c}{\scriptsize{-0.292}} & \mc{1}{c}{\scriptsize{0.543}} & \mc{1}{c}{\scriptsize{0.145}} \\  

     &  & \mc{1}{c}{\scriptsize{(0.829)}} & \mc{1}{c}{\scriptsize{(0.658)}} & \mc{1}{c}{\scriptsize{(0.527)}} & \mc{1}{c}{\scriptsize{(0.311)}} & \mc{1}{c}{\scriptsize{(0.392)}} & \mc{1}{c}{\scriptsize{(0.921)}} & \mc{1}{c}{\scriptsize{(0.803)}} & \mc{1}{c}{\scriptsize{(0.921)}} \\  

     & \mc{1}{c}{\scriptsize{8}} & \mc{1}{c}{\scriptsize{1.548}} & \mc{1}{c}{\scriptsize{0.263}} & \mc{1}{c}{\scriptsize{-0.898}} & \mc{1}{c}{\scriptsize{-5.960}} & \mc{1}{c}{\scriptsize{-1.637}} & \mc{1}{c}{\scriptsize{2.062}} & \mc{1}{c}{\scriptsize{0.801}} & \mc{1}{c}{\scriptsize{0.146}} \\  

     &  & \mc{1}{c}{\scriptsize{(0.539)}} & \mc{1}{c}{\scriptsize{(0.855)}} & \mc{1}{c}{\scriptsize{(0.878)}} & \mc{1}{c}{\scriptsize{(1.000)}} & \mc{1}{c}{\scriptsize{(0.919)}} & \mc{1}{c}{\scriptsize{(0.434)}} & \mc{1}{c}{\scriptsize{(0.737)}} & \mc{1}{c}{\scriptsize{(0.921)}} \\  

     & \mc{1}{c}{\scriptsize{0.5}} & \mc{1}{c}{\scriptsize{0.372}} & \mc{1}{c}{\scriptsize{0.039}} & \mc{1}{c}{\scriptsize{0.944}} & \mc{1}{c}{\scriptsize{-0.416}} & \mc{1}{c}{\scriptsize{0.431}} & \mc{1}{c}{\scriptsize{0.143}} & \mc{1}{c}{\scriptsize{0.069}} & \mc{1}{c}{\scriptsize{-0.080}} \\  

     &  & \mc{1}{c}{\scriptsize{(0.724)}} & \mc{1}{c}{\scriptsize{(0.868)}} & \mc{1}{c}{\scriptsize{(0.649)}} & \mc{1}{c}{\scriptsize{(1.000)}} & \mc{1}{c}{\scriptsize{(0.743)}} & \mc{1}{c}{\scriptsize{(0.868)}} & \mc{1}{c}{\scriptsize{(0.882)}} & \mc{1}{c}{\scriptsize{(0.947)}} \\  

  \bottomrule
  \end{tabular}
	\end{table} 

	\begin{table}[H]
     \caption{Treatment Effects on Job Satisfaction Score, Male Sample}
     \label{table:abccare_rslt_male_cat25_sd}
	  \begin{tabular}{cccccccccc}
  \toprule

    \scriptsize{Variable} & \scriptsize{Age} & \scriptsize{(1)} & \scriptsize{(2)} & \scriptsize{(3)} & \scriptsize{(4)} & \scriptsize{(5)} & \scriptsize{(6)} & \scriptsize{(7)} & \scriptsize{(8)} \\ 
    \midrule  

    \mc{1}{l}{\scriptsize{Std. Achv.  Test}} & \mc{1}{c}{\scriptsize{8}} & \mc{1}{c}{\scriptsize{2.309}} & \mc{1}{c}{\scriptsize{4.117}} & \mc{1}{c}{\scriptsize{-3.386}} & \mc{1}{c}{\scriptsize{-0.625}} & \mc{1}{c}{\scriptsize{-2.713}} & \mc{1}{c}{\scriptsize{3.903}} & \mc{1}{c}{\scriptsize{5.198}} & \mc{1}{c}{\scriptsize{4.240}} \\  

     &  & \mc{1}{c}{\scriptsize{(0.592)}} & \mc{1}{c}{\scriptsize{(0.158)}} & \mc{1}{c}{\scriptsize{(0.947)}} & \mc{1}{c}{\scriptsize{(0.853)}} & \mc{1}{c}{\scriptsize{(0.933)}} & \mc{1}{c}{\scriptsize{(0.263)}} & \mc{1}{c}{\scriptsize{\textbf{(0.053)}}} & \mc{1}{c}{\scriptsize{(0.250)}} \\  

     & \mc{1}{c}{\scriptsize{21}} & \mc{1}{c}{\scriptsize{1.181}} & \mc{1}{c}{\scriptsize{-0.041}} & \mc{1}{c}{\scriptsize{1.168}} & \mc{1}{c}{\scriptsize{0.655}} & \mc{1}{c}{\scriptsize{-0.287}} & \mc{1}{c}{\scriptsize{1.356}} & \mc{1}{c}{\scriptsize{-0.595}} & \mc{1}{c}{\scriptsize{-0.888}} \\  

     &  & \mc{1}{c}{\scriptsize{(0.842)}} & \mc{1}{c}{\scriptsize{(0.961)}} & \mc{1}{c}{\scriptsize{(0.707)}} & \mc{1}{c}{\scriptsize{(0.813)}} & \mc{1}{c}{\scriptsize{(0.867)}} & \mc{1}{c}{\scriptsize{(0.842)}} & \mc{1}{c}{\scriptsize{(1.000)}} & \mc{1}{c}{\scriptsize{(0.987)}} \\  

     & \mc{1}{c}{\scriptsize{15}} & \mc{1}{c}{\scriptsize{2.231}} & \mc{1}{c}{\scriptsize{2.281}} & \mc{1}{c}{\scriptsize{1.379}} & \mc{1}{c}{\scriptsize{3.026}} & \mc{1}{c}{\scriptsize{0.536}} & \mc{1}{c}{\scriptsize{2.531}} & \mc{1}{c}{\scriptsize{1.785}} & \mc{1}{c}{\scriptsize{0.913}} \\  

     &  & \mc{1}{c}{\scriptsize{(0.632)}} & \mc{1}{c}{\scriptsize{(0.671)}} & \mc{1}{c}{\scriptsize{(0.680)}} & \mc{1}{c}{\scriptsize{(0.720)}} & \mc{1}{c}{\scriptsize{(0.827)}} & \mc{1}{c}{\scriptsize{(0.684)}} & \mc{1}{c}{\scriptsize{(0.724)}} & \mc{1}{c}{\scriptsize{(0.934)}} \\  

     & \mc{1}{c}{\scriptsize{6.5}} & \mc{1}{c}{\scriptsize{1.708}} & \mc{1}{c}{\scriptsize{3.526}} & \mc{1}{c}{\scriptsize{-0.892}} & \mc{1}{c}{\scriptsize{4.797}} & \mc{1}{c}{\scriptsize{-0.660}} & \mc{1}{c}{\scriptsize{2.521}} & \mc{1}{c}{\scriptsize{3.195}} & \mc{1}{c}{\scriptsize{2.318}} \\  

     &  & \mc{1}{c}{\scriptsize{(0.671)}} & \mc{1}{c}{\scriptsize{(0.447)}} & \mc{1}{c}{\scriptsize{(0.853)}} & \mc{1}{c}{\scriptsize{(0.600)}} & \mc{1}{c}{\scriptsize{(0.893)}} & \mc{1}{c}{\scriptsize{(0.671)}} & \mc{1}{c}{\scriptsize{(0.566)}} & \mc{1}{c}{\scriptsize{(0.724)}} \\  

     & \mc{1}{c}{\scriptsize{12}} & \mc{1}{c}{\scriptsize{-3.441}} & \mc{1}{c}{\scriptsize{-3.150}} & \mc{1}{c}{\scriptsize{-5.233}} & \mc{1}{c}{\scriptsize{-8.732}} & \mc{1}{c}{\scriptsize{-1.845}} & \mc{1}{c}{\scriptsize{-3.115}} & \mc{1}{c}{\scriptsize{-2.263}} & \mc{1}{c}{\scriptsize{-0.999}} \\  

     &  & \mc{1}{c}{\scriptsize{(1.000)}} & \mc{1}{c}{\scriptsize{(0.974)}} & \mc{1}{c}{\scriptsize{(0.933)}} & \mc{1}{c}{\scriptsize{(0.960)}} & \mc{1}{c}{\scriptsize{(0.893)}} & \mc{1}{c}{\scriptsize{(1.000)}} & \mc{1}{c}{\scriptsize{(1.000)}} & \mc{1}{c}{\scriptsize{(0.987)}} \\  

     & \mc{1}{c}{\scriptsize{7.5}} & \mc{1}{c}{\scriptsize{0.019}} & \mc{1}{c}{\scriptsize{1.103}} & \mc{1}{c}{\scriptsize{-2.767}} & \mc{1}{c}{\scriptsize{-2.380}} & \mc{1}{c}{\scriptsize{-3.284}} & \mc{1}{c}{\scriptsize{0.799}} & \mc{1}{c}{\scriptsize{1.762}} & \mc{1}{c}{\scriptsize{0.215}} \\  

     &  & \mc{1}{c}{\scriptsize{(0.934)}} & \mc{1}{c}{\scriptsize{(0.816)}} & \mc{1}{c}{\scriptsize{(0.920)}} & \mc{1}{c}{\scriptsize{(0.907)}} & \mc{1}{c}{\scriptsize{(0.947)}} & \mc{1}{c}{\scriptsize{(0.882)}} & \mc{1}{c}{\scriptsize{(0.658)}} & \mc{1}{c}{\scriptsize{(0.934)}} \\  

     & \mc{1}{c}{\scriptsize{7}} & \mc{1}{c}{\scriptsize{0.622}} & \mc{1}{c}{\scriptsize{2.047}} & \mc{1}{c}{\scriptsize{0.219}} & \mc{1}{c}{\scriptsize{1.642}} & \mc{1}{c}{\scriptsize{1.063}} & \mc{1}{c}{\scriptsize{0.748}} & \mc{1}{c}{\scriptsize{1.855}} & \mc{1}{c}{\scriptsize{0.805}} \\  

     &  & \mc{1}{c}{\scriptsize{(0.895)}} & \mc{1}{c}{\scriptsize{(0.684)}} & \mc{1}{c}{\scriptsize{(0.787)}} & \mc{1}{c}{\scriptsize{(0.720)}} & \mc{1}{c}{\scriptsize{(0.733)}} & \mc{1}{c}{\scriptsize{(0.921)}} & \mc{1}{c}{\scriptsize{(0.763)}} & \mc{1}{c}{\scriptsize{(0.934)}} \\  

    \mc{1}{l}{\scriptsize{PIAT Math Std. Score}} & \mc{1}{c}{\scriptsize{7}} & \mc{1}{c}{\scriptsize{3.074}} & \mc{1}{c}{\scriptsize{5.109}} & \mc{1}{c}{\scriptsize{-1.840}} & \mc{1}{c}{\scriptsize{-1.018}} & \mc{1}{c}{\scriptsize{-0.988}} & \mc{1}{c}{\scriptsize{4.610}} & \mc{1}{c}{\scriptsize{5.905}} & \mc{1}{c}{\scriptsize{4.612}} \\  

     &  & \mc{1}{c}{\scriptsize{(0.487)}} & \mc{1}{c}{\scriptsize{(0.276)}} & \mc{1}{c}{\scriptsize{(0.867)}} & \mc{1}{c}{\scriptsize{(0.880)}} & \mc{1}{c}{\scriptsize{(0.893)}} & \mc{1}{c}{\scriptsize{(0.250)}} & \mc{1}{c}{\scriptsize{(0.329)}} & \mc{1}{c}{\scriptsize{(0.408)}} \\  

    \mc{1}{l}{\scriptsize{Std. Achv.  Test}} & \mc{1}{c}{\scriptsize{5.5}} & \mc{1}{c}{\scriptsize{5.108}} & \mc{1}{c}{\scriptsize{3.622}} & \mc{1}{c}{\scriptsize{10.088}} & \mc{1}{c}{\scriptsize{12.739}} & \mc{1}{c}{\scriptsize{11.654}} & \mc{1}{c}{\scriptsize{3.863}} & \mc{1}{c}{\scriptsize{2.465}} & \mc{1}{c}{\scriptsize{2.371}} \\  

     &  & \mc{1}{c}{\scriptsize{(0.145)}} & \mc{1}{c}{\scriptsize{(0.684)}} & \mc{1}{c}{\scriptsize{(0.253)}} & \mc{1}{c}{\scriptsize{(0.413)}} & \mc{1}{c}{\scriptsize{\textbf{(0.080)}}} & \mc{1}{c}{\scriptsize{(0.461)}} & \mc{1}{c}{\scriptsize{(0.789)}} & \mc{1}{c}{\scriptsize{(0.868)}} \\  

     & \mc{1}{c}{\scriptsize{8.5}} & \mc{1}{c}{\scriptsize{3.910}} & \mc{1}{c}{\scriptsize{5.586}} & \mc{1}{c}{\scriptsize{-1.771}} & \mc{1}{c}{\scriptsize{0.775}} & \mc{1}{c}{\scriptsize{-1.498}} & \mc{1}{c}{\scriptsize{4.199}} & \mc{1}{c}{\scriptsize{5.497}} & \mc{1}{c}{\scriptsize{4.117}} \\  

     &  & \mc{1}{c}{\scriptsize{(0.395)}} & \mc{1}{c}{\scriptsize{\textbf{(0.092)}}} & \mc{1}{c}{\scriptsize{(0.853)}} & \mc{1}{c}{\scriptsize{(0.813)}} & \mc{1}{c}{\scriptsize{(0.893)}} & \mc{1}{c}{\scriptsize{(0.316)}} & \mc{1}{c}{\scriptsize{(0.132)}} & \mc{1}{c}{\scriptsize{(0.408)}} \\  

     & \mc{1}{c}{\scriptsize{6}} & \mc{1}{c}{\scriptsize{3.091}} & \mc{1}{c}{\scriptsize{0.075}} & \mc{1}{c}{\scriptsize{2.271}} & \mc{1}{c}{\scriptsize{-7.037}} & \mc{1}{c}{\scriptsize{-1.898}} & \mc{1}{c}{\scriptsize{3.312}} & \mc{1}{c}{\scriptsize{1.216}} & \mc{1}{c}{\scriptsize{0.962}} \\  

     &  & \mc{1}{c}{\scriptsize{(0.250)}} & \mc{1}{c}{\scriptsize{(0.961)}} & \mc{1}{c}{\scriptsize{(0.600)}} & \mc{1}{c}{\scriptsize{(0.960)}} & \mc{1}{c}{\scriptsize{(0.893)}} & \mc{1}{c}{\scriptsize{(0.250)}} & \mc{1}{c}{\scriptsize{(0.789)}} & \mc{1}{c}{\scriptsize{(0.908)}} \\  

  \bottomrule
  \end{tabular}
	\end{table} 

	\begin{table}[H]
     \caption{Treatment Effects on Crime, Male Sample}
     \label{table:abccare_rslt_male_cat26_sd}
	  \begin{tabular}{cccccccccc}
  \toprule

    \scriptsize{Variable} & \scriptsize{Age} & \scriptsize{(1)} & \scriptsize{(2)} & \scriptsize{(3)} & \scriptsize{(4)} & \scriptsize{(5)} & \scriptsize{(6)} & \scriptsize{(7)} & \scriptsize{(8)} \\ 
    \midrule  

    \mc{1}{l}{\scriptsize{Std. IQ Test}} & \mc{1}{c}{\scriptsize{6.6}} & \mc{1}{c}{\scriptsize{5.803}} & \mc{1}{c}{\scriptsize{5.157}} & \mc{1}{c}{\scriptsize{0.831}} & \mc{1}{c}{\scriptsize{-1.794}} & \mc{1}{c}{\scriptsize{1.479}} & \mc{1}{c}{\scriptsize{5.916}} & \mc{1}{c}{\scriptsize{5.542}} & \mc{1}{c}{\scriptsize{5.886}} \\  

     &  & \mc{1}{c}{\scriptsize{\textbf{(0.093)}}} & \mc{1}{c}{\scriptsize{(0.253)}} & \mc{1}{c}{\scriptsize{(0.890)}} & \mc{1}{c}{\scriptsize{(0.973)}} & \mc{1}{c}{\scriptsize{(0.890)}} & \mc{1}{c}{\scriptsize{\textbf{(0.068)}}} & \mc{1}{c}{\scriptsize{(0.122)}} & \mc{1}{c}{\scriptsize{\textbf{(0.081)}}} \\  

     & \mc{1}{c}{\scriptsize{2.5}} & \mc{1}{c}{\scriptsize{4.291}} & \mc{1}{c}{\scriptsize{6.596}} & \mc{1}{c}{\scriptsize{8.444}} & \mc{1}{c}{\scriptsize{6.746}} & \mc{1}{c}{\scriptsize{12.603}} & \mc{1}{c}{\scriptsize{3.535}} & \mc{1}{c}{\scriptsize{4.805}} & \mc{1}{c}{\scriptsize{7.574}} \\  

     &  & \mc{1}{c}{\scriptsize{(0.693)}} & \mc{1}{c}{\scriptsize{(0.853)}} & \mc{1}{c}{\scriptsize{(0.603)}} & \mc{1}{c}{\scriptsize{(0.781)}} & \mc{1}{c}{\scriptsize{(0.479)}} & \mc{1}{c}{\scriptsize{(0.838)}} & \mc{1}{c}{\scriptsize{(0.986)}} & \mc{1}{c}{\scriptsize{(0.419)}} \\  

     & \mc{1}{c}{\scriptsize{4}} & \mc{1}{c}{\scriptsize{12.089}} & \mc{1}{c}{\scriptsize{11.867}} & \mc{1}{c}{\scriptsize{8.950}} & \mc{1}{c}{\scriptsize{7.899}} & \mc{1}{c}{\scriptsize{9.668}} & \mc{1}{c}{\scriptsize{12.986}} & \mc{1}{c}{\scriptsize{12.604}} & \mc{1}{c}{\scriptsize{13.487}} \\  

     &  & \mc{1}{c}{\scriptsize{\textbf{(0.000)}}} & \mc{1}{c}{\scriptsize{\textbf{(0.013)}}} & \mc{1}{c}{\scriptsize{\textbf{(0.068)}}} & \mc{1}{c}{\scriptsize{(0.438)}} & \mc{1}{c}{\scriptsize{(0.110)}} & \mc{1}{c}{\scriptsize{\textbf{(0.000)}}} & \mc{1}{c}{\scriptsize{\textbf{(0.014)}}} & \mc{1}{c}{\scriptsize{\textbf{(0.000)}}} \\  

     & \mc{1}{c}{\scriptsize{15}} & \mc{1}{c}{\scriptsize{4.447}} & \mc{1}{c}{\scriptsize{3.119}} & \mc{1}{c}{\scriptsize{-2.057}} & \mc{1}{c}{\scriptsize{-4.401}} & \mc{1}{c}{\scriptsize{-3.006}} & \mc{1}{c}{\scriptsize{6.202}} & \mc{1}{c}{\scriptsize{4.551}} & \mc{1}{c}{\scriptsize{4.499}} \\  

     &  & \mc{1}{c}{\scriptsize{(0.307)}} & \mc{1}{c}{\scriptsize{(0.693)}} & \mc{1}{c}{\scriptsize{(0.959)}} & \mc{1}{c}{\scriptsize{(0.986)}} & \mc{1}{c}{\scriptsize{(0.973)}} & \mc{1}{c}{\scriptsize{(0.135)}} & \mc{1}{c}{\scriptsize{(0.459)}} & \mc{1}{c}{\scriptsize{(0.419)}} \\  

     & \mc{1}{c}{\scriptsize{6}} & \mc{1}{c}{\scriptsize{11.595}} & \mc{1}{c}{\scriptsize{8.499}} & \mc{1}{c}{\scriptsize{5.095}} & \mc{1}{c}{\scriptsize{2.158}} & \mc{1}{c}{\scriptsize{3.027}} & \mc{1}{c}{\scriptsize{13.762}} & \mc{1}{c}{\scriptsize{11.910}} & \mc{1}{c}{\scriptsize{12.516}} \\  

     &  & \mc{1}{c}{\scriptsize{\textbf{(0.093)}}} & \mc{1}{c}{\scriptsize{(0.693)}} & \mc{1}{c}{\scriptsize{(0.658)}} & \mc{1}{c}{\scriptsize{(0.904)}} & \mc{1}{c}{\scriptsize{(0.781)}} & \mc{1}{c}{\scriptsize{\textbf{(0.054)}}} & \mc{1}{c}{\scriptsize{(0.486)}} & \mc{1}{c}{\scriptsize{\textbf{(0.068)}}} \\  

     & \mc{1}{c}{\scriptsize{3}} & \mc{1}{c}{\scriptsize{13.410}} & \mc{1}{c}{\scriptsize{14.873}} & \mc{1}{c}{\scriptsize{13.896}} & \mc{1}{c}{\scriptsize{17.254}} & \mc{1}{c}{\scriptsize{15.474}} & \mc{1}{c}{\scriptsize{13.271}} & \mc{1}{c}{\scriptsize{14.229}} & \mc{1}{c}{\scriptsize{14.302}} \\  

     &  & \mc{1}{c}{\scriptsize{\textbf{(0.000)}}} & \mc{1}{c}{\scriptsize{\textbf{(0.013)}}} & \mc{1}{c}{\scriptsize{\textbf{(0.000)}}} & \mc{1}{c}{\scriptsize{\textbf{(0.055)}}} & \mc{1}{c}{\scriptsize{\textbf{(0.000)}}} & \mc{1}{c}{\scriptsize{\textbf{(0.000)}}} & \mc{1}{c}{\scriptsize{\textbf{(0.014)}}} & \mc{1}{c}{\scriptsize{\textbf{(0.000)}}} \\  

     & \mc{1}{c}{\scriptsize{4.5}} & \mc{1}{c}{\scriptsize{8.508}} & \mc{1}{c}{\scriptsize{8.914}} & \mc{1}{c}{\scriptsize{10.411}} & \mc{1}{c}{\scriptsize{11.349}} & \mc{1}{c}{\scriptsize{10.639}} & \mc{1}{c}{\scriptsize{7.964}} & \mc{1}{c}{\scriptsize{8.518}} & \mc{1}{c}{\scriptsize{7.804}} \\  

     &  & \mc{1}{c}{\scriptsize{\textbf{(0.000)}}} & \mc{1}{c}{\scriptsize{\textbf{(0.013)}}} & \mc{1}{c}{\scriptsize{\textbf{(0.027)}}} & \mc{1}{c}{\scriptsize{(0.110)}} & \mc{1}{c}{\scriptsize{\textbf{(0.014)}}} & \mc{1}{c}{\scriptsize{\textbf{(0.027)}}} & \mc{1}{c}{\scriptsize{\textbf{(0.027)}}} & \mc{1}{c}{\scriptsize{\textbf{(0.027)}}} \\  

     & \mc{1}{c}{\scriptsize{8}} & \mc{1}{c}{\scriptsize{4.160}} & \mc{1}{c}{\scriptsize{2.644}} & \mc{1}{c}{\scriptsize{-2.514}} & \mc{1}{c}{\scriptsize{-4.445}} & \mc{1}{c}{\scriptsize{-2.651}} & \mc{1}{c}{\scriptsize{4.754}} & \mc{1}{c}{\scriptsize{3.391}} & \mc{1}{c}{\scriptsize{4.207}} \\  

     &  & \mc{1}{c}{\scriptsize{(0.427)}} & \mc{1}{c}{\scriptsize{(0.773)}} & \mc{1}{c}{\scriptsize{(0.959)}} & \mc{1}{c}{\scriptsize{(0.986)}} & \mc{1}{c}{\scriptsize{(0.945)}} & \mc{1}{c}{\scriptsize{(0.243)}} & \mc{1}{c}{\scriptsize{(0.486)}} & \mc{1}{c}{\scriptsize{(0.365)}} \\  

     & \mc{1}{c}{\scriptsize{7}} & \mc{1}{c}{\scriptsize{4.390}} & \mc{1}{c}{\scriptsize{7.662}} & \mc{1}{c}{\scriptsize{5.323}} & \mc{1}{c}{\scriptsize{8.887}} & \mc{1}{c}{\scriptsize{4.891}} & \mc{1}{c}{\scriptsize{4.156}} & \mc{1}{c}{\scriptsize{7.491}} & \mc{1}{c}{\scriptsize{6.538}} \\  

     &  & \mc{1}{c}{\scriptsize{(0.373)}} & \mc{1}{c}{\scriptsize{\textbf{(0.013)}}} & \mc{1}{c}{\scriptsize{(0.603)}} & \mc{1}{c}{\scriptsize{(0.452)}} & \mc{1}{c}{\scriptsize{(0.644)}} & \mc{1}{c}{\scriptsize{(0.459)}} & \mc{1}{c}{\scriptsize{\textbf{(0.054)}}} & \mc{1}{c}{\scriptsize{(0.162)}} \\  

     & \mc{1}{c}{\scriptsize{5}} & \mc{1}{c}{\scriptsize{7.697}} & \mc{1}{c}{\scriptsize{7.130}} & \mc{1}{c}{\scriptsize{4.643}} & \mc{1}{c}{\scriptsize{3.469}} & \mc{1}{c}{\scriptsize{4.992}} & \mc{1}{c}{\scriptsize{8.679}} & \mc{1}{c}{\scriptsize{8.241}} & \mc{1}{c}{\scriptsize{8.181}} \\  

     &  & \mc{1}{c}{\scriptsize{\textbf{(0.000)}}} & \mc{1}{c}{\scriptsize{\textbf{(0.013)}}} & \mc{1}{c}{\scriptsize{(0.589)}} & \mc{1}{c}{\scriptsize{(0.795)}} & \mc{1}{c}{\scriptsize{(0.589)}} & \mc{1}{c}{\scriptsize{\textbf{(0.000)}}} & \mc{1}{c}{\scriptsize{\textbf{(0.027)}}} & \mc{1}{c}{\scriptsize{\textbf{(0.000)}}} \\  

     & \mc{1}{c}{\scriptsize{12}} & \mc{1}{c}{\scriptsize{0.686}} & \mc{1}{c}{\scriptsize{-0.856}} & \mc{1}{c}{\scriptsize{-0.343}} & \mc{1}{c}{\scriptsize{-4.076}} & \mc{1}{c}{\scriptsize{-1.018}} & \mc{1}{c}{\scriptsize{0.943}} & \mc{1}{c}{\scriptsize{-0.116}} & \mc{1}{c}{\scriptsize{-0.794}} \\  

     &  & \mc{1}{c}{\scriptsize{(0.880)}} & \mc{1}{c}{\scriptsize{(0.987)}} & \mc{1}{c}{\scriptsize{(0.932)}} & \mc{1}{c}{\scriptsize{(0.986)}} & \mc{1}{c}{\scriptsize{(0.932)}} & \mc{1}{c}{\scriptsize{(0.919)}} & \mc{1}{c}{\scriptsize{(0.986)}} & \mc{1}{c}{\scriptsize{(0.986)}} \\  

     & \mc{1}{c}{\scriptsize{3.5}} & \mc{1}{c}{\scriptsize{8.756}} & \mc{1}{c}{\scriptsize{7.994}} & \mc{1}{c}{\scriptsize{6.354}} & \mc{1}{c}{\scriptsize{5.904}} & \mc{1}{c}{\scriptsize{6.784}} & \mc{1}{c}{\scriptsize{9.443}} & \mc{1}{c}{\scriptsize{8.690}} & \mc{1}{c}{\scriptsize{9.043}} \\  

     &  & \mc{1}{c}{\scriptsize{\textbf{(0.000)}}} & \mc{1}{c}{\scriptsize{\textbf{(0.053)}}} & \mc{1}{c}{\scriptsize{(0.274)}} & \mc{1}{c}{\scriptsize{(0.452)}} & \mc{1}{c}{\scriptsize{(0.247)}} & \mc{1}{c}{\scriptsize{\textbf{(0.000)}}} & \mc{1}{c}{\scriptsize{\textbf{(0.054)}}} & \mc{1}{c}{\scriptsize{\textbf{(0.014)}}} \\  

     & \mc{1}{c}{\scriptsize{2}} & \mc{1}{c}{\scriptsize{9.528}} & \mc{1}{c}{\scriptsize{11.036}} & \mc{1}{c}{\scriptsize{6.875}} & \mc{1}{c}{\scriptsize{10.704}} & \mc{1}{c}{\scriptsize{7.944}} & \mc{1}{c}{\scriptsize{10.286}} & \mc{1}{c}{\scriptsize{11.497}} & \mc{1}{c}{\scriptsize{11.080}} \\  

     &  & \mc{1}{c}{\scriptsize{\textbf{(0.000)}}} & \mc{1}{c}{\scriptsize{\textbf{(0.013)}}} & \mc{1}{c}{\scriptsize{(0.233)}} & \mc{1}{c}{\scriptsize{(0.233)}} & \mc{1}{c}{\scriptsize{(0.110)}} & \mc{1}{c}{\scriptsize{\textbf{(0.000)}}} & \mc{1}{c}{\scriptsize{\textbf{(0.014)}}} & \mc{1}{c}{\scriptsize{\textbf{(0.000)}}} \\  

     & \mc{1}{c}{\scriptsize{21}} & \mc{1}{c}{\scriptsize{1.550}} & \mc{1}{c}{\scriptsize{-0.645}} & \mc{1}{c}{\scriptsize{0.471}} & \mc{1}{c}{\scriptsize{-0.846}} & \mc{1}{c}{\scriptsize{-1.551}} & \mc{1}{c}{\scriptsize{2.307}} & \mc{1}{c}{\scriptsize{-0.423}} & \mc{1}{c}{\scriptsize{-0.483}} \\  

     &  & \mc{1}{c}{\scriptsize{(0.733)}} & \mc{1}{c}{\scriptsize{(0.987)}} & \mc{1}{c}{\scriptsize{(0.877)}} & \mc{1}{c}{\scriptsize{(0.973)}} & \mc{1}{c}{\scriptsize{(0.973)}} & \mc{1}{c}{\scriptsize{(0.662)}} & \mc{1}{c}{\scriptsize{(0.986)}} & \mc{1}{c}{\scriptsize{(0.986)}} \\  

  \bottomrule
  \end{tabular}
	\end{table} 

	\begin{table}[H]
     \caption{Treatment Effects on Childhood and Adolescence Physical Health, Male Sample}
     \label{table:abccare_rslt_male_cat27_sd}
	\input{AppResOutput/abccare/rslt_male_cat27_sd}
	\end{table} 

	\begin{table}[H]
     \caption{Treatment Effects on Childhood Health Problems, Male Sample}
     \label{table:abccare_rslt_male_cat28_sd}
	  \begin{tabular}{cccccccccc}
  \toprule

    \scriptsize{Variable} & \scriptsize{Age} & \scriptsize{(1)} & \scriptsize{(2)} & \scriptsize{(3)} & \scriptsize{(4)} & \scriptsize{(5)} & \scriptsize{(6)} & \scriptsize{(7)} & \scriptsize{(8)} \\ 
    \midrule  

    \mc{1}{l}{\scriptsize{Has Health Problems}} & \mc{1}{c}{\scriptsize{12}} & \mc{1}{c}{\scriptsize{-0.065}} & \mc{1}{c}{\scriptsize{-0.181}} & \mc{1}{c}{\scriptsize{-0.005}} & \mc{1}{c}{\scriptsize{-0.082}} & \mc{1}{c}{\scriptsize{0.017}} & \mc{1}{c}{\scriptsize{-0.079}} & \mc{1}{c}{\scriptsize{-0.213}} & \mc{1}{c}{\scriptsize{-0.154}} \\  

     &  & \mc{1}{c}{\scriptsize{(0.447)}} & \mc{1}{c}{\scriptsize{\textbf{(0.079)}}} & \mc{1}{c}{\scriptsize{(0.676)}} & \mc{1}{c}{\scriptsize{(0.419)}} & \mc{1}{c}{\scriptsize{(0.703)}} & \mc{1}{c}{\scriptsize{(0.382)}} & \mc{1}{c}{\scriptsize{\textbf{(0.066)}}} & \mc{1}{c}{\scriptsize{(0.158)}} \\  

    \mc{1}{l}{\scriptsize{Ever Hospitalized for Over 1 Week}} & \mc{1}{c}{\scriptsize{12}} & \mc{1}{c}{\scriptsize{-0.067}} & \mc{1}{c}{\scriptsize{-0.087}} & \mc{1}{c}{\scriptsize{0.067}} & \mc{1}{c}{\scriptsize{0.014}} & \mc{1}{c}{\scriptsize{0.066}} & \mc{1}{c}{\scriptsize{-0.100}} & \mc{1}{c}{\scriptsize{-0.122}} & \mc{1}{c}{\scriptsize{-0.099}} \\  

     &  & \mc{1}{c}{\scriptsize{(0.355)}} & \mc{1}{c}{\scriptsize{(0.474)}} & \mc{1}{c}{\scriptsize{(0.986)}} & \mc{1}{c}{\scriptsize{(0.689)}} & \mc{1}{c}{\scriptsize{(0.959)}} & \mc{1}{c}{\scriptsize{(0.289)}} & \mc{1}{c}{\scriptsize{(0.211)}} & \mc{1}{c}{\scriptsize{(0.237)}} \\  

  \bottomrule
  \end{tabular}
	\end{table} 

	\begin{table}[H]
     \caption{Treatment Effects on Cholesterol, Male Sample}
     \label{table:abccare_rslt_male_cat29_sd}
	\input{AppResOutput/abccare/rslt_male_cat29_sd}
	\end{table} 

	\begin{table}[H]
     \caption{Treatment Effects on Current Health Condition (Self-Reported), Male Sample}
     \label{table:abccare_rslt_male_cat30_sd}
	  \begin{tabular}{cccccccccc}
  \toprule

    \scriptsize{Variable} & \scriptsize{Age} & \scriptsize{(1)} & \scriptsize{(2)} & \scriptsize{(3)} & \scriptsize{(4)} & \scriptsize{(5)} & \scriptsize{(6)} & \scriptsize{(7)} & \scriptsize{(8)} \\ 
    \midrule  

    \mc{1}{l}{\scriptsize{Asthma}} & \mc{1}{c}{\scriptsize{Mid-30s}} & \mc{1}{c}{\scriptsize{-0.034}} & \mc{1}{c}{\scriptsize{-0.027}} & \mc{1}{c}{\scriptsize{0.037}} & \mc{1}{c}{\scriptsize{0.028}} & \mc{1}{c}{\scriptsize{0.048}} & \mc{1}{c}{\scriptsize{-0.063}} & \mc{1}{c}{\scriptsize{-0.042}} & \mc{1}{c}{\scriptsize{-0.034}} \\  

     &  & \mc{1}{c}{\scriptsize{(0.829)}} & \mc{1}{c}{\scriptsize{(0.789)}} & \mc{1}{c}{\scriptsize{(1.000)}} & \mc{1}{c}{\scriptsize{(0.971)}} & \mc{1}{c}{\scriptsize{(0.986)}} & \mc{1}{c}{\scriptsize{(0.724)}} & \mc{1}{c}{\scriptsize{(0.803)}} & \mc{1}{c}{\scriptsize{(0.868)}} \\  

    \mc{1}{l}{\scriptsize{High Blood Pressure (Hypertension)}} & \mc{1}{c}{\scriptsize{Mid-30s}} & \mc{1}{c}{\scriptsize{0.040}} & \mc{1}{c}{\scriptsize{0.039}} & \mc{1}{c}{\scriptsize{0.040}} & \mc{1}{c}{\scriptsize{0.019}} & \mc{1}{c}{\scriptsize{0.050}} & \mc{1}{c}{\scriptsize{0.040}} & \mc{1}{c}{\scriptsize{0.045}} & \mc{1}{c}{\scriptsize{0.050}} \\  

     &  & \mc{1}{c}{\scriptsize{(1.000)}} & \mc{1}{c}{\scriptsize{(1.000)}} & \mc{1}{c}{\scriptsize{(1.000)}} & \mc{1}{c}{\scriptsize{(0.928)}} & \mc{1}{c}{\scriptsize{(0.986)}} & \mc{1}{c}{\scriptsize{(1.000)}} & \mc{1}{c}{\scriptsize{(1.000)}} & \mc{1}{c}{\scriptsize{(1.000)}} \\  

    \mc{1}{l}{\scriptsize{Arthritis or Generative Disease}} & \mc{1}{c}{\scriptsize{Mid-30s}} &  &  &  &  &  &  &  &  \\  

     &  &  &  &  &  &  &  &  &  \\  

    \mc{1}{l}{\scriptsize{Diabetes}} & \mc{1}{c}{\scriptsize{Mid-30s}} & \mc{1}{c}{\scriptsize{0.040}} & \mc{1}{c}{\scriptsize{0.033}} & \mc{1}{c}{\scriptsize{0.040}} & \mc{1}{c}{\scriptsize{-0.046}} & \mc{1}{c}{\scriptsize{0.050}} & \mc{1}{c}{\scriptsize{0.040}} & \mc{1}{c}{\scriptsize{0.051}} & \mc{1}{c}{\scriptsize{0.050}} \\  

     &  & \mc{1}{c}{\scriptsize{(1.000)}} & \mc{1}{c}{\scriptsize{(1.000)}} & \mc{1}{c}{\scriptsize{(1.000)}} & \mc{1}{c}{\scriptsize{(0.725)}} & \mc{1}{c}{\scriptsize{(1.000)}} & \mc{1}{c}{\scriptsize{(1.000)}} & \mc{1}{c}{\scriptsize{(1.000)}} & \mc{1}{c}{\scriptsize{(1.000)}} \\  

  \bottomrule
  \end{tabular}
	\end{table} 

	\begin{table}[H]
     \caption{Treatment Effects on Diabetes, Male Sample}
     \label{table:abccare_rslt_male_cat31_sd}
	  \begin{tabular}{cccccccccc}
  \toprule

    \scriptsize{Variable} & \scriptsize{Age} & \scriptsize{(1)} & \scriptsize{(2)} & \scriptsize{(3)} & \scriptsize{(4)} & \scriptsize{(5)} & \scriptsize{(6)} & \scriptsize{(7)} & \scriptsize{(8)} \\ 
    \midrule  

    \mc{1}{l}{\scriptsize{Prediabetes}} & \mc{1}{c}{\scriptsize{Mid-30s}} & \mc{1}{c}{\scriptsize{-0.129}} & \mc{1}{c}{\scriptsize{-0.190}} & \mc{1}{c}{\scriptsize{-0.267}} & \mc{1}{c}{\scriptsize{-0.478}} & \mc{1}{c}{\scriptsize{-0.338}} & \mc{1}{c}{\scriptsize{-0.139}} & \mc{1}{c}{\scriptsize{-0.182}} & \mc{1}{c}{\scriptsize{-0.173}} \\  

     &  & \mc{1}{c}{\scriptsize{(0.507)}} & \mc{1}{c}{\scriptsize{(0.320)}} & \mc{1}{c}{\scriptsize{(0.455)}} & \mc{1}{c}{\scriptsize{(0.273)}} & \mc{1}{c}{\scriptsize{(0.273)}} & \mc{1}{c}{\scriptsize{(0.514)}} & \mc{1}{c}{\scriptsize{(0.392)}} & \mc{1}{c}{\scriptsize{(0.459)}} \\  

    \mc{1}{l}{\scriptsize{Hemoglobin Level (\%)}} & \mc{1}{c}{\scriptsize{Mid-30s}} & \mc{1}{c}{\scriptsize{0.322}} & \mc{1}{c}{\scriptsize{0.435}} & \mc{1}{c}{\scriptsize{0.240}} & \mc{1}{c}{\scriptsize{0.469}} & \mc{1}{c}{\scriptsize{0.355}} & \mc{1}{c}{\scriptsize{0.286}} & \mc{1}{c}{\scriptsize{0.386}} & \mc{1}{c}{\scriptsize{0.396}} \\  

     &  & \mc{1}{c}{\scriptsize{(0.960)}} & \mc{1}{c}{\scriptsize{(0.920)}} & \mc{1}{c}{\scriptsize{(0.939)}} & \mc{1}{c}{\scriptsize{(0.818)}} & \mc{1}{c}{\scriptsize{(0.955)}} & \mc{1}{c}{\scriptsize{(0.946)}} & \mc{1}{c}{\scriptsize{(0.905)}} & \mc{1}{c}{\scriptsize{(0.959)}} \\  

    \mc{1}{l}{\scriptsize{Diabetes}} & \mc{1}{c}{\scriptsize{Mid-30s}} & \mc{1}{c}{\scriptsize{0.080}} & \mc{1}{c}{\scriptsize{0.081}} & \mc{1}{c}{\scriptsize{0.080}} & \mc{1}{c}{\scriptsize{0.027}} & \mc{1}{c}{\scriptsize{0.098}} & \mc{1}{c}{\scriptsize{0.080}} & \mc{1}{c}{\scriptsize{0.083}} & \mc{1}{c}{\scriptsize{0.098}} \\  

     &  & \mc{1}{c}{\scriptsize{(0.987)}} & \mc{1}{c}{\scriptsize{(0.947)}} & \mc{1}{c}{\scriptsize{(1.000)}} & \mc{1}{c}{\scriptsize{(0.788)}} & \mc{1}{c}{\scriptsize{(1.000)}} & \mc{1}{c}{\scriptsize{(0.986)}} & \mc{1}{c}{\scriptsize{(0.919)}} & \mc{1}{c}{\scriptsize{(0.986)}} \\  

  \bottomrule
  \end{tabular}
	\end{table} 

	\begin{table}[H]
     \caption{Treatment Effects on Drug Behavior and ASR Substance Scale, Male Sample}
     \label{table:abccare_rslt_male_cat32_sd}
	\input{AppResOutput/abccare/rslt_male_cat32_sd}
	\end{table} 

	\begin{table}[H]
     \caption{Treatment Effects on Health Insurance, Male Sample}
     \label{table:abccare_rslt_male_cat33_sd}
	  \begin{tabular}{cccccccccc}
  \toprule

    \scriptsize{Variable} & \scriptsize{Age} & \scriptsize{(1)} & \scriptsize{(2)} & \scriptsize{(3)} & \scriptsize{(4)} & \scriptsize{(5)} & \scriptsize{(6)} & \scriptsize{(7)} & \scriptsize{(8)} \\ 
    \midrule  

    \mc{1}{l}{\scriptsize{Labor Income}} & \mc{1}{c}{\scriptsize{30}} & \mc{1}{c}{\scriptsize{19,810}} & \mc{1}{c}{\scriptsize{24,902}} & \mc{1}{c}{\scriptsize{17,909}} & \mc{1}{c}{\scriptsize{21,069}} & \mc{1}{c}{\scriptsize{24,012}} & \mc{1}{c}{\scriptsize{20,065}} & \mc{1}{c}{\scriptsize{28,483}} & \mc{1}{c}{\scriptsize{21,170}} \\  

     &  & \mc{1}{c}{\scriptsize{(0.347)}} & \mc{1}{c}{\scriptsize{(0.640)}} & \mc{1}{c}{\scriptsize{(0.342)}} & \mc{1}{c}{\scriptsize{(0.685)}} & \mc{1}{c}{\scriptsize{(0.274)}} & \mc{1}{c}{\scriptsize{(0.324)}} & \mc{1}{c}{\scriptsize{(0.595)}} & \mc{1}{c}{\scriptsize{(0.500)}} \\  

    \mc{1}{l}{\scriptsize{Employed}} & \mc{1}{c}{\scriptsize{30}} & \mc{1}{c}{\scriptsize{0.119}} & \mc{1}{c}{\scriptsize{0.179}} & \mc{1}{c}{\scriptsize{-0.029}} & \mc{1}{c}{\scriptsize{-0.050}} & \mc{1}{c}{\scriptsize{0.041}} & \mc{1}{c}{\scriptsize{0.176}} & \mc{1}{c}{\scriptsize{0.245}} & \mc{1}{c}{\scriptsize{0.262}} \\  

     &  & \mc{1}{c}{\scriptsize{(0.387)}} & \mc{1}{c}{\scriptsize{(0.227)}} & \mc{1}{c}{\scriptsize{(0.890)}} & \mc{1}{c}{\scriptsize{(0.973)}} & \mc{1}{c}{\scriptsize{(0.863)}} & \mc{1}{c}{\scriptsize{(0.270)}} & \mc{1}{c}{\scriptsize{\textbf{(0.095)}}} & \mc{1}{c}{\scriptsize{\textbf{(0.068)}}} \\  

    \mc{1}{l}{\scriptsize{Public-Transfer Income}} & \mc{1}{c}{\scriptsize{21}} & \mc{1}{c}{\scriptsize{315}} & \mc{1}{c}{\scriptsize{456}} & \mc{1}{c}{\scriptsize{1,376}} & \mc{1}{c}{\scriptsize{2,654}} & \mc{1}{c}{\scriptsize{1,543}} & \mc{1}{c}{\scriptsize{-58.901}} & \mc{1}{c}{\scriptsize{144}} & \mc{1}{c}{\scriptsize{97.591}} \\  

     &  & \mc{1}{c}{\scriptsize{(0.987)}} & \mc{1}{c}{\scriptsize{(1.000)}} & \mc{1}{c}{\scriptsize{(1.000)}} & \mc{1}{c}{\scriptsize{(0.973)}} & \mc{1}{c}{\scriptsize{(1.000)}} & \mc{1}{c}{\scriptsize{(0.946)}} & \mc{1}{c}{\scriptsize{(0.986)}} & \mc{1}{c}{\scriptsize{(1.000)}} \\  

    \mc{1}{l}{\scriptsize{Labor Income}} & \mc{1}{c}{\scriptsize{21}} & \mc{1}{c}{\scriptsize{-1,672}} & \mc{1}{c}{\scriptsize{-4,977}} & \mc{1}{c}{\scriptsize{-3,951}} & \mc{1}{c}{\scriptsize{-13,951}} & \mc{1}{c}{\scriptsize{-4,585}} & \mc{1}{c}{\scriptsize{-1,527}} & \mc{1}{c}{\scriptsize{-3,973}} & \mc{1}{c}{\scriptsize{-3,746}} \\  

     &  & \mc{1}{c}{\scriptsize{(1.000)}} & \mc{1}{c}{\scriptsize{(1.000)}} & \mc{1}{c}{\scriptsize{(0.973)}} & \mc{1}{c}{\scriptsize{(1.000)}} & \mc{1}{c}{\scriptsize{(0.973)}} & \mc{1}{c}{\scriptsize{(1.000)}} & \mc{1}{c}{\scriptsize{(1.000)}} & \mc{1}{c}{\scriptsize{(1.000)}} \\  

    \mc{1}{l}{\scriptsize{Public-Transfer Income}} & \mc{1}{c}{\scriptsize{30}} & \mc{1}{c}{\scriptsize{-530}} & \mc{1}{c}{\scriptsize{-176}} & \mc{1}{c}{\scriptsize{287}} & \mc{1}{c}{\scriptsize{722}} & \mc{1}{c}{\scriptsize{548}} & \mc{1}{c}{\scriptsize{-279}} & \mc{1}{c}{\scriptsize{-82.612}} & \mc{1}{c}{\scriptsize{-155}} \\  

     &  & \mc{1}{c}{\scriptsize{(0.573)}} & \mc{1}{c}{\scriptsize{(0.960)}} & \mc{1}{c}{\scriptsize{(1.000)}} & \mc{1}{c}{\scriptsize{(1.000)}} & \mc{1}{c}{\scriptsize{(1.000)}} & \mc{1}{c}{\scriptsize{(0.851)}} & \mc{1}{c}{\scriptsize{(0.959)}} & \mc{1}{c}{\scriptsize{(0.919)}} \\  

  \bottomrule
  \end{tabular}
	\end{table} 

	\begin{table}[H]
     \caption{Treatment Effects on Hypertension, Male Sample}
     \label{table:abccare_rslt_male_cat34_sd}
	\input{AppResOutput/abccare/rslt_male_cat34_sd}
	\end{table} 

	\begin{table}[H]
     \caption{Treatment Effects on Laboratory Test  - Metabolic Panel, Male Sample}
     \label{table:abccare_rslt_male_cat35_sd}
	  \begin{tabular}{cccccccccc}
  \toprule

    \scriptsize{Variable} & \scriptsize{Age} & \scriptsize{(1)} & \scriptsize{(2)} & \scriptsize{(3)} & \scriptsize{(4)} & \scriptsize{(5)} & \scriptsize{(6)} & \scriptsize{(7)} & \scriptsize{(8)} \\ 
    \midrule  

    \mc{1}{l}{\scriptsize{Albumin/Globulin Ratio}} & \mc{1}{c}{\scriptsize{Mid-30s}} & \mc{1}{c}{\scriptsize{0.062}} & \mc{1}{c}{\scriptsize{0.106}} & \mc{1}{c}{\scriptsize{-0.023}} & \mc{1}{c}{\scriptsize{0.086}} & \mc{1}{c}{\scriptsize{-0.019}} & \mc{1}{c}{\scriptsize{0.098}} & \mc{1}{c}{\scriptsize{0.122}} & \mc{1}{c}{\scriptsize{0.087}} \\  

     &  & \mc{1}{c}{\scriptsize{(0.829)}} & \mc{1}{c}{\scriptsize{(0.697)}} & \mc{1}{c}{\scriptsize{(1.000)}} & \mc{1}{c}{\scriptsize{(0.971)}} & \mc{1}{c}{\scriptsize{(1.000)}} & \mc{1}{c}{\scriptsize{(0.539)}} & \mc{1}{c}{\scriptsize{(0.645)}} & \mc{1}{c}{\scriptsize{(0.750)}} \\  

    \mc{1}{l}{\scriptsize{ALT}} & \mc{1}{c}{\scriptsize{Mid-30s}} & \mc{1}{c}{\scriptsize{3.908}} & \mc{1}{c}{\scriptsize{-0.240}} & \mc{1}{c}{\scriptsize{-3.013}} & \mc{1}{c}{\scriptsize{-6.002}} & \mc{1}{c}{\scriptsize{-3.471}} & \mc{1}{c}{\scriptsize{4.782}} & \mc{1}{c}{\scriptsize{1.748}} & \mc{1}{c}{\scriptsize{1.816}} \\  

     &  & \mc{1}{c}{\scriptsize{(0.974)}} & \mc{1}{c}{\scriptsize{(1.000)}} & \mc{1}{c}{\scriptsize{(1.000)}} & \mc{1}{c}{\scriptsize{(1.000)}} & \mc{1}{c}{\scriptsize{(1.000)}} & \mc{1}{c}{\scriptsize{(0.921)}} & \mc{1}{c}{\scriptsize{(1.000)}} & \mc{1}{c}{\scriptsize{(0.987)}} \\  

    \mc{1}{l}{\scriptsize{Albumin}} & \mc{1}{c}{\scriptsize{Mid-30s}} & \mc{1}{c}{\scriptsize{0.209}} & \mc{1}{c}{\scriptsize{0.293}} & \mc{1}{c}{\scriptsize{0.001}} & \mc{1}{c}{\scriptsize{0.170}} & \mc{1}{c}{\scriptsize{0.017}} & \mc{1}{c}{\scriptsize{0.268}} & \mc{1}{c}{\scriptsize{0.324}} & \mc{1}{c}{\scriptsize{0.269}} \\  

     &  & \mc{1}{c}{\scriptsize{\textbf{(0.026)}}} & \mc{1}{c}{\scriptsize{\textbf{(0.092)}}} & \mc{1}{c}{\scriptsize{(1.000)}} & \mc{1}{c}{\scriptsize{(0.928)}} & \mc{1}{c}{\scriptsize{(1.000)}} & \mc{1}{c}{\scriptsize{\textbf{(0.000)}}} & \mc{1}{c}{\scriptsize{(0.118)}} & \mc{1}{c}{\scriptsize{\textbf{(0.000)}}} \\  

    \mc{1}{l}{\scriptsize{Sodium}} & \mc{1}{c}{\scriptsize{Mid-30s}} & \mc{1}{c}{\scriptsize{-0.462}} & \mc{1}{c}{\scriptsize{-0.035}} & \mc{1}{c}{\scriptsize{0.872}} & \mc{1}{c}{\scriptsize{2.006}} & \mc{1}{c}{\scriptsize{0.711}} & \mc{1}{c}{\scriptsize{-0.692}} & \mc{1}{c}{\scriptsize{-0.335}} & \mc{1}{c}{\scriptsize{-0.757}} \\  

     &  & \mc{1}{c}{\scriptsize{(1.000)}} & \mc{1}{c}{\scriptsize{(1.000)}} & \mc{1}{c}{\scriptsize{(0.841)}} & \mc{1}{c}{\scriptsize{(0.609)}} & \mc{1}{c}{\scriptsize{(0.928)}} & \mc{1}{c}{\scriptsize{(1.000)}} & \mc{1}{c}{\scriptsize{(1.000)}} & \mc{1}{c}{\scriptsize{(1.000)}} \\  

    \mc{1}{l}{\scriptsize{Carbon Dioxide}} & \mc{1}{c}{\scriptsize{Mid-30s}} & \mc{1}{c}{\scriptsize{0.627}} & \mc{1}{c}{\scriptsize{1.275}} & \mc{1}{c}{\scriptsize{1.705}} & \mc{1}{c}{\scriptsize{2.905}} & \mc{1}{c}{\scriptsize{0.936}} & \mc{1}{c}{\scriptsize{0.500}} & \mc{1}{c}{\scriptsize{1.150}} & \mc{1}{c}{\scriptsize{0.598}} \\  

     &  & \mc{1}{c}{\scriptsize{(0.842)}} & \mc{1}{c}{\scriptsize{(0.276)}} & \mc{1}{c}{\scriptsize{(0.928)}} & \mc{1}{c}{\scriptsize{(0.391)}} & \mc{1}{c}{\scriptsize{(0.971)}} & \mc{1}{c}{\scriptsize{(0.855)}} & \mc{1}{c}{\scriptsize{(0.250)}} & \mc{1}{c}{\scriptsize{(0.750)}} \\  

    \mc{1}{l}{\scriptsize{AST}} & \mc{1}{c}{\scriptsize{Mid-30s}} & \mc{1}{c}{\scriptsize{3.299}} & \mc{1}{c}{\scriptsize{1.328}} & \mc{1}{c}{\scriptsize{0.907}} & \mc{1}{c}{\scriptsize{2.026}} & \mc{1}{c}{\scriptsize{-0.842}} & \mc{1}{c}{\scriptsize{3.471}} & \mc{1}{c}{\scriptsize{1.712}} & \mc{1}{c}{\scriptsize{1.260}} \\  

     &  & \mc{1}{c}{\scriptsize{(0.842)}} & \mc{1}{c}{\scriptsize{(1.000)}} & \mc{1}{c}{\scriptsize{(1.000)}} & \mc{1}{c}{\scriptsize{(1.000)}} & \mc{1}{c}{\scriptsize{(1.000)}} & \mc{1}{c}{\scriptsize{(0.868)}} & \mc{1}{c}{\scriptsize{(1.000)}} & \mc{1}{c}{\scriptsize{(0.974)}} \\  

    \mc{1}{l}{\scriptsize{Urea Nitrogen}} & \mc{1}{c}{\scriptsize{Mid-30s}} & \mc{1}{c}{\scriptsize{1.155}} & \mc{1}{c}{\scriptsize{0.576}} & \mc{1}{c}{\scriptsize{3.253}} & \mc{1}{c}{\scriptsize{1.758}} & \mc{1}{c}{\scriptsize{3.883}} & \mc{1}{c}{\scriptsize{0.228}} & \mc{1}{c}{\scriptsize{0.019}} & \mc{1}{c}{\scriptsize{0.227}} \\  

     &  & \mc{1}{c}{\scriptsize{(0.816)}} & \mc{1}{c}{\scriptsize{(0.987)}} & \mc{1}{c}{\scriptsize{(0.681)}} & \mc{1}{c}{\scriptsize{(0.913)}} & \mc{1}{c}{\scriptsize{(0.580)}} & \mc{1}{c}{\scriptsize{(1.000)}} & \mc{1}{c}{\scriptsize{(1.000)}} & \mc{1}{c}{\scriptsize{(1.000)}} \\  

    \mc{1}{l}{\scriptsize{Globulin}} & \mc{1}{c}{\scriptsize{Mid-30s}} & \mc{1}{c}{\scriptsize{0.023}} & \mc{1}{c}{\scriptsize{0.000}} & \mc{1}{c}{\scriptsize{0.043}} & \mc{1}{c}{\scriptsize{-0.021}} & \mc{1}{c}{\scriptsize{0.052}} & \mc{1}{c}{\scriptsize{-0.001}} & \mc{1}{c}{\scriptsize{-0.007}} & \mc{1}{c}{\scriptsize{0.021}} \\  

     &  & \mc{1}{c}{\scriptsize{(1.000)}} & \mc{1}{c}{\scriptsize{(1.000)}} & \mc{1}{c}{\scriptsize{(0.928)}} & \mc{1}{c}{\scriptsize{(1.000)}} & \mc{1}{c}{\scriptsize{(0.928)}} & \mc{1}{c}{\scriptsize{(1.000)}} & \mc{1}{c}{\scriptsize{(1.000)}} & \mc{1}{c}{\scriptsize{(1.000)}} \\  

    \mc{1}{l}{\scriptsize{Chloride}} & \mc{1}{c}{\scriptsize{Mid-30s}} & \mc{1}{c}{\scriptsize{-1.041}} & \mc{1}{c}{\scriptsize{-0.871}} & \mc{1}{c}{\scriptsize{-2.256}} & \mc{1}{c}{\scriptsize{-2.020}} & \mc{1}{c}{\scriptsize{-2.203}} & \mc{1}{c}{\scriptsize{-0.538}} & \mc{1}{c}{\scriptsize{-0.575}} & \mc{1}{c}{\scriptsize{-0.795}} \\  

     &  & \mc{1}{c}{\scriptsize{(1.000)}} & \mc{1}{c}{\scriptsize{(1.000)}} & \mc{1}{c}{\scriptsize{(1.000)}} & \mc{1}{c}{\scriptsize{(1.000)}} & \mc{1}{c}{\scriptsize{(1.000)}} & \mc{1}{c}{\scriptsize{(1.000)}} & \mc{1}{c}{\scriptsize{(1.000)}} & \mc{1}{c}{\scriptsize{(1.000)}} \\  

    \mc{1}{l}{\scriptsize{Glucose}} & \mc{1}{c}{\scriptsize{Mid-30s}} & \mc{1}{c}{\scriptsize{15.855}} & \mc{1}{c}{\scriptsize{14.392}} & \mc{1}{c}{\scriptsize{11.385}} & \mc{1}{c}{\scriptsize{-1.251}} & \mc{1}{c}{\scriptsize{16.161}} & \mc{1}{c}{\scriptsize{16.769}} & \mc{1}{c}{\scriptsize{15.579}} & \mc{1}{c}{\scriptsize{19.138}} \\  

     &  & \mc{1}{c}{\scriptsize{(0.618)}} & \mc{1}{c}{\scriptsize{(0.921)}} & \mc{1}{c}{\scriptsize{(0.870)}} & \mc{1}{c}{\scriptsize{(1.000)}} & \mc{1}{c}{\scriptsize{(0.797)}} & \mc{1}{c}{\scriptsize{(0.566)}} & \mc{1}{c}{\scriptsize{(0.908)}} & \mc{1}{c}{\scriptsize{(0.539)}} \\  

    \mc{1}{l}{\scriptsize{Potassium}} & \mc{1}{c}{\scriptsize{Mid-30s}} & \mc{1}{c}{\scriptsize{0.170}} & \mc{1}{c}{\scriptsize{0.164}} & \mc{1}{c}{\scriptsize{0.178}} & \mc{1}{c}{\scriptsize{0.062}} & \mc{1}{c}{\scriptsize{0.213}} & \mc{1}{c}{\scriptsize{0.189}} & \mc{1}{c}{\scriptsize{0.182}} & \mc{1}{c}{\scriptsize{0.187}} \\  

     &  & \mc{1}{c}{\scriptsize{(0.184)}} & \mc{1}{c}{\scriptsize{(0.395)}} & \mc{1}{c}{\scriptsize{(0.739)}} & \mc{1}{c}{\scriptsize{(1.000)}} & \mc{1}{c}{\scriptsize{(0.696)}} & \mc{1}{c}{\scriptsize{\textbf{(0.026)}}} & \mc{1}{c}{\scriptsize{(0.263)}} & \mc{1}{c}{\scriptsize{\textbf{(0.053)}}} \\  

    \mc{1}{l}{\scriptsize{Creatinine}} & \mc{1}{c}{\scriptsize{Mid-30s}} & \mc{1}{c}{\scriptsize{0.086}} & \mc{1}{c}{\scriptsize{0.013}} & \mc{1}{c}{\scriptsize{0.254}} & \mc{1}{c}{\scriptsize{0.108}} & \mc{1}{c}{\scriptsize{0.240}} & \mc{1}{c}{\scriptsize{0.040}} & \mc{1}{c}{\scriptsize{-0.015}} & \mc{1}{c}{\scriptsize{0.023}} \\  

     &  & \mc{1}{c}{\scriptsize{(0.671)}} & \mc{1}{c}{\scriptsize{(1.000)}} & \mc{1}{c}{\scriptsize{\textbf{(0.000)}}} & \mc{1}{c}{\scriptsize{(0.899)}} & \mc{1}{c}{\scriptsize{\textbf{(0.000)}}} & \mc{1}{c}{\scriptsize{(0.947)}} & \mc{1}{c}{\scriptsize{(1.000)}} & \mc{1}{c}{\scriptsize{(0.974)}} \\  

    \mc{1}{l}{\scriptsize{Calcium}} & \mc{1}{c}{\scriptsize{Mid-30s}} & \mc{1}{c}{\scriptsize{0.120}} & \mc{1}{c}{\scriptsize{0.183}} & \mc{1}{c}{\scriptsize{0.096}} & \mc{1}{c}{\scriptsize{0.229}} & \mc{1}{c}{\scriptsize{0.160}} & \mc{1}{c}{\scriptsize{0.127}} & \mc{1}{c}{\scriptsize{0.181}} & \mc{1}{c}{\scriptsize{0.184}} \\  

     &  & \mc{1}{c}{\scriptsize{(0.618)}} & \mc{1}{c}{\scriptsize{(0.276)}} & \mc{1}{c}{\scriptsize{(0.928)}} & \mc{1}{c}{\scriptsize{(0.783)}} & \mc{1}{c}{\scriptsize{(0.754)}} & \mc{1}{c}{\scriptsize{(0.645)}} & \mc{1}{c}{\scriptsize{(0.316)}} & \mc{1}{c}{\scriptsize{(0.105)}} \\  

    \mc{1}{l}{\scriptsize{Bilirubin}} & \mc{1}{c}{\scriptsize{Mid-30s}} & \mc{1}{c}{\scriptsize{0.200}} & \mc{1}{c}{\scriptsize{0.180}} & \mc{1}{c}{\scriptsize{0.245}} & \mc{1}{c}{\scriptsize{0.211}} & \mc{1}{c}{\scriptsize{0.234}} & \mc{1}{c}{\scriptsize{0.197}} & \mc{1}{c}{\scriptsize{0.178}} & \mc{1}{c}{\scriptsize{0.186}} \\  

     &  & \mc{1}{c}{\scriptsize{(0.158)}} & \mc{1}{c}{\scriptsize{(0.237)}} & \mc{1}{c}{\scriptsize{\textbf{(0.000)}}} & \mc{1}{c}{\scriptsize{(0.696)}} & \mc{1}{c}{\scriptsize{\textbf{(0.000)}}} & \mc{1}{c}{\scriptsize{(0.132)}} & \mc{1}{c}{\scriptsize{(0.329)}} & \mc{1}{c}{\scriptsize{(0.316)}} \\  

    \mc{1}{l}{\scriptsize{Alkaline Phosp}} & \mc{1}{c}{\scriptsize{Mid-30s}} & \mc{1}{c}{\scriptsize{12.358}} & \mc{1}{c}{\scriptsize{8.133}} & \mc{1}{c}{\scriptsize{13.907}} & \mc{1}{c}{\scriptsize{0.823}} & \mc{1}{c}{\scriptsize{15.787}} & \mc{1}{c}{\scriptsize{11.163}} & \mc{1}{c}{\scriptsize{7.637}} & \mc{1}{c}{\scriptsize{11.522}} \\  

     &  & \mc{1}{c}{\scriptsize{(0.500)}} & \mc{1}{c}{\scriptsize{(0.961)}} & \mc{1}{c}{\scriptsize{(0.652)}} & \mc{1}{c}{\scriptsize{(1.000)}} & \mc{1}{c}{\scriptsize{(0.551)}} & \mc{1}{c}{\scriptsize{(0.645)}} & \mc{1}{c}{\scriptsize{(0.987)}} & \mc{1}{c}{\scriptsize{(0.684)}} \\  

    \mc{1}{l}{\scriptsize{Protein}} & \mc{1}{c}{\scriptsize{Mid-30s}} & \mc{1}{c}{\scriptsize{0.232}} & \mc{1}{c}{\scriptsize{0.294}} & \mc{1}{c}{\scriptsize{0.044}} & \mc{1}{c}{\scriptsize{0.149}} & \mc{1}{c}{\scriptsize{0.069}} & \mc{1}{c}{\scriptsize{0.267}} & \mc{1}{c}{\scriptsize{0.317}} & \mc{1}{c}{\scriptsize{0.290}} \\  

     &  & \mc{1}{c}{\scriptsize{(0.158)}} & \mc{1}{c}{\scriptsize{(0.237)}} & \mc{1}{c}{\scriptsize{(0.971)}} & \mc{1}{c}{\scriptsize{(0.913)}} & \mc{1}{c}{\scriptsize{(0.971)}} & \mc{1}{c}{\scriptsize{(0.145)}} & \mc{1}{c}{\scriptsize{(0.342)}} & \mc{1}{c}{\scriptsize{(0.105)}} \\  

  \bottomrule
  \end{tabular}
	\end{table} 

	\begin{table}[H]
     \caption{Treatment Effects on Laboratory Test - Complete Blood Count, Male Sample}
     \label{table:abccare_rslt_male_cat36_sd}
	  \begin{tabular}{cccccccccc}
  \toprule

    \scriptsize{Variable} & \scriptsize{Age} & \scriptsize{(1)} & \scriptsize{(2)} & \scriptsize{(3)} & \scriptsize{(4)} & \scriptsize{(5)} & \scriptsize{(6)} & \scriptsize{(7)} & \scriptsize{(8)} \\ 
    \midrule  

    \mc{1}{l}{\scriptsize{Mean Cell Volum}} & \mc{1}{c}{\scriptsize{Mid-30s}} & \mc{1}{c}{\scriptsize{-0.781}} & \mc{1}{c}{\scriptsize{-1.460}} & \mc{1}{c}{\scriptsize{-0.047}} & \mc{1}{c}{\scriptsize{-1.126}} & \mc{1}{c}{\scriptsize{0.617}} & \mc{1}{c}{\scriptsize{-0.435}} & \mc{1}{c}{\scriptsize{-1.059}} & \mc{1}{c}{\scriptsize{-0.130}} \\  

     &  & \mc{1}{c}{\scriptsize{(1.000)}} & \mc{1}{c}{\scriptsize{(1.000)}} & \mc{1}{c}{\scriptsize{(1.000)}} & \mc{1}{c}{\scriptsize{(1.000)}} & \mc{1}{c}{\scriptsize{(0.986)}} & \mc{1}{c}{\scriptsize{(1.000)}} & \mc{1}{c}{\scriptsize{(1.000)}} & \mc{1}{c}{\scriptsize{(1.000)}} \\  

    \mc{1}{l}{\scriptsize{Platelets}} & \mc{1}{c}{\scriptsize{Mid-30s}} & \mc{1}{c}{\scriptsize{4.665}} & \mc{1}{c}{\scriptsize{0.226}} & \mc{1}{c}{\scriptsize{6.744}} & \mc{1}{c}{\scriptsize{-12.809}} & \mc{1}{c}{\scriptsize{14.780}} & \mc{1}{c}{\scriptsize{2.077}} & \mc{1}{c}{\scriptsize{0.356}} & \mc{1}{c}{\scriptsize{4.516}} \\  

     &  & \mc{1}{c}{\scriptsize{(0.987)}} & \mc{1}{c}{\scriptsize{(1.000)}} & \mc{1}{c}{\scriptsize{(0.986)}} & \mc{1}{c}{\scriptsize{(1.000)}} & \mc{1}{c}{\scriptsize{(1.000)}} & \mc{1}{c}{\scriptsize{(1.000)}} & \mc{1}{c}{\scriptsize{(1.000)}} & \mc{1}{c}{\scriptsize{(0.987)}} \\  

    \mc{1}{l}{\scriptsize{Eosinophils}} & \mc{1}{c}{\scriptsize{Mid-30s}} & \mc{1}{c}{\scriptsize{0.371}} & \mc{1}{c}{\scriptsize{0.779}} & \mc{1}{c}{\scriptsize{0.991}} & \mc{1}{c}{\scriptsize{1.877}} & \mc{1}{c}{\scriptsize{1.176}} & \mc{1}{c}{\scriptsize{0.301}} & \mc{1}{c}{\scriptsize{0.512}} & \mc{1}{c}{\scriptsize{0.476}} \\  

     &  & \mc{1}{c}{\scriptsize{(0.961)}} & \mc{1}{c}{\scriptsize{(0.803)}} & \mc{1}{c}{\scriptsize{(0.406)}} & \mc{1}{c}{\scriptsize{(0.580)}} & \mc{1}{c}{\scriptsize{(0.159)}} & \mc{1}{c}{\scriptsize{(0.974)}} & \mc{1}{c}{\scriptsize{(0.961)}} & \mc{1}{c}{\scriptsize{(0.934)}} \\  

    \mc{1}{l}{\scriptsize{Hemoglobin}} & \mc{1}{c}{\scriptsize{Mid-30s}} & \mc{1}{c}{\scriptsize{-0.038}} & \mc{1}{c}{\scriptsize{-0.222}} & \mc{1}{c}{\scriptsize{-0.173}} & \mc{1}{c}{\scriptsize{0.043}} & \mc{1}{c}{\scriptsize{-0.166}} & \mc{1}{c}{\scriptsize{-0.050}} & \mc{1}{c}{\scriptsize{-0.278}} & \mc{1}{c}{\scriptsize{-0.169}} \\  

     &  & \mc{1}{c}{\scriptsize{(1.000)}} & \mc{1}{c}{\scriptsize{(1.000)}} & \mc{1}{c}{\scriptsize{(1.000)}} & \mc{1}{c}{\scriptsize{(1.000)}} & \mc{1}{c}{\scriptsize{(1.000)}} & \mc{1}{c}{\scriptsize{(1.000)}} & \mc{1}{c}{\scriptsize{(1.000)}} & \mc{1}{c}{\scriptsize{(1.000)}} \\  

    \mc{1}{l}{\scriptsize{Red Cells}} & \mc{1}{c}{\scriptsize{Mid-30s}} & \mc{1}{c}{\scriptsize{0.015}} & \mc{1}{c}{\scriptsize{-0.008}} & \mc{1}{c}{\scriptsize{0.002}} & \mc{1}{c}{\scriptsize{0.159}} & \mc{1}{c}{\scriptsize{-0.059}} & \mc{1}{c}{\scriptsize{-0.029}} & \mc{1}{c}{\scriptsize{-0.062}} & \mc{1}{c}{\scriptsize{-0.080}} \\  

     &  & \mc{1}{c}{\scriptsize{(1.000)}} & \mc{1}{c}{\scriptsize{(1.000)}} & \mc{1}{c}{\scriptsize{(1.000)}} & \mc{1}{c}{\scriptsize{(0.899)}} & \mc{1}{c}{\scriptsize{(1.000)}} & \mc{1}{c}{\scriptsize{(1.000)}} & \mc{1}{c}{\scriptsize{(1.000)}} & \mc{1}{c}{\scriptsize{(1.000)}} \\  

    \mc{1}{l}{\scriptsize{Lymphocytes}} & \mc{1}{c}{\scriptsize{Mid-30s}} & \mc{1}{c}{\scriptsize{-2.867}} & \mc{1}{c}{\scriptsize{-0.766}} & \mc{1}{c}{\scriptsize{-7.875}} & \mc{1}{c}{\scriptsize{-4.114}} & \mc{1}{c}{\scriptsize{-7.658}} & \mc{1}{c}{\scriptsize{-0.439}} & \mc{1}{c}{\scriptsize{0.347}} & \mc{1}{c}{\scriptsize{0.065}} \\  

     &  & \mc{1}{c}{\scriptsize{(1.000)}} & \mc{1}{c}{\scriptsize{(1.000)}} & \mc{1}{c}{\scriptsize{(1.000)}} & \mc{1}{c}{\scriptsize{(1.000)}} & \mc{1}{c}{\scriptsize{(1.000)}} & \mc{1}{c}{\scriptsize{(1.000)}} & \mc{1}{c}{\scriptsize{(1.000)}} & \mc{1}{c}{\scriptsize{(1.000)}} \\  

    \mc{1}{l}{\scriptsize{Monocytes}} & \mc{1}{c}{\scriptsize{Mid-30s}} & \mc{1}{c}{\scriptsize{0.839}} & \mc{1}{c}{\scriptsize{1.198}} & \mc{1}{c}{\scriptsize{0.868}} & \mc{1}{c}{\scriptsize{0.888}} & \mc{1}{c}{\scriptsize{0.742}} & \mc{1}{c}{\scriptsize{0.822}} & \mc{1}{c}{\scriptsize{1.135}} & \mc{1}{c}{\scriptsize{0.782}} \\  

     &  & \mc{1}{c}{\scriptsize{(0.434)}} & \mc{1}{c}{\scriptsize{(0.368)}} & \mc{1}{c}{\scriptsize{(0.522)}} & \mc{1}{c}{\scriptsize{(0.942)}} & \mc{1}{c}{\scriptsize{(0.681)}} & \mc{1}{c}{\scriptsize{(0.618)}} & \mc{1}{c}{\scriptsize{(0.632)}} & \mc{1}{c}{\scriptsize{(0.776)}} \\  

    \mc{1}{l}{\scriptsize{Neutrophils}} & \mc{1}{c}{\scriptsize{Mid-30s}} & \mc{1}{c}{\scriptsize{1.661}} & \mc{1}{c}{\scriptsize{-1.176}} & \mc{1}{c}{\scriptsize{5.875}} & \mc{1}{c}{\scriptsize{1.207}} & \mc{1}{c}{\scriptsize{5.649}} & \mc{1}{c}{\scriptsize{-0.661}} & \mc{1}{c}{\scriptsize{-1.935}} & \mc{1}{c}{\scriptsize{-1.259}} \\  

     &  & \mc{1}{c}{\scriptsize{(0.974)}} & \mc{1}{c}{\scriptsize{(1.000)}} & \mc{1}{c}{\scriptsize{(0.884)}} & \mc{1}{c}{\scriptsize{(1.000)}} & \mc{1}{c}{\scriptsize{(0.884)}} & \mc{1}{c}{\scriptsize{(1.000)}} & \mc{1}{c}{\scriptsize{(1.000)}} & \mc{1}{c}{\scriptsize{(1.000)}} \\  

    \mc{1}{l}{\scriptsize{Basophils}} & \mc{1}{c}{\scriptsize{Mid-30s}} & \mc{1}{c}{\scriptsize{-0.044}} & \mc{1}{c}{\scriptsize{-0.035}} & \mc{1}{c}{\scriptsize{0.101}} & \mc{1}{c}{\scriptsize{0.143}} & \mc{1}{c}{\scriptsize{0.091}} & \mc{1}{c}{\scriptsize{-0.063}} & \mc{1}{c}{\scriptsize{-0.059}} & \mc{1}{c}{\scriptsize{-0.065}} \\  

     &  & \mc{1}{c}{\scriptsize{(1.000)}} & \mc{1}{c}{\scriptsize{(1.000)}} & \mc{1}{c}{\scriptsize{(0.725)}} & \mc{1}{c}{\scriptsize{(0.870)}} & \mc{1}{c}{\scriptsize{(0.725)}} & \mc{1}{c}{\scriptsize{(1.000)}} & \mc{1}{c}{\scriptsize{(1.000)}} & \mc{1}{c}{\scriptsize{(1.000)}} \\  

    \mc{1}{l}{\scriptsize{Mean Hemoglobin}} & \mc{1}{c}{\scriptsize{Mid-30s}} & \mc{1}{c}{\scriptsize{-0.328}} & \mc{1}{c}{\scriptsize{-0.548}} & \mc{1}{c}{\scriptsize{-0.324}} & \mc{1}{c}{\scriptsize{-0.771}} & \mc{1}{c}{\scriptsize{0.058}} & \mc{1}{c}{\scriptsize{-0.104}} & \mc{1}{c}{\scriptsize{-0.341}} & \mc{1}{c}{\scriptsize{-0.022}} \\  

     &  & \mc{1}{c}{\scriptsize{(1.000)}} & \mc{1}{c}{\scriptsize{(1.000)}} & \mc{1}{c}{\scriptsize{(1.000)}} & \mc{1}{c}{\scriptsize{(1.000)}} & \mc{1}{c}{\scriptsize{(1.000)}} & \mc{1}{c}{\scriptsize{(1.000)}} & \mc{1}{c}{\scriptsize{(1.000)}} & \mc{1}{c}{\scriptsize{(1.000)}} \\  

    \mc{1}{l}{\scriptsize{White Cells}} & \mc{1}{c}{\scriptsize{Mid-30s}} & \mc{1}{c}{\scriptsize{0.023}} & \mc{1}{c}{\scriptsize{-0.370}} & \mc{1}{c}{\scriptsize{1.001}} & \mc{1}{c}{\scriptsize{1.127}} & \mc{1}{c}{\scriptsize{1.184}} & \mc{1}{c}{\scriptsize{-0.227}} & \mc{1}{c}{\scriptsize{-0.485}} & \mc{1}{c}{\scriptsize{-0.301}} \\  

     &  & \mc{1}{c}{\scriptsize{(1.000)}} & \mc{1}{c}{\scriptsize{(1.000)}} & \mc{1}{c}{\scriptsize{(0.884)}} & \mc{1}{c}{\scriptsize{(0.913)}} & \mc{1}{c}{\scriptsize{(0.783)}} & \mc{1}{c}{\scriptsize{(1.000)}} & \mc{1}{c}{\scriptsize{(1.000)}} & \mc{1}{c}{\scriptsize{(1.000)}} \\  

    \mc{1}{l}{\scriptsize{Hematocrit}} & \mc{1}{c}{\scriptsize{Mid-30s}} & \mc{1}{c}{\scriptsize{-0.055}} & \mc{1}{c}{\scriptsize{-0.568}} & \mc{1}{c}{\scriptsize{-0.074}} & \mc{1}{c}{\scriptsize{0.750}} & \mc{1}{c}{\scriptsize{-0.245}} & \mc{1}{c}{\scriptsize{-0.238}} & \mc{1}{c}{\scriptsize{-0.835}} & \mc{1}{c}{\scriptsize{-0.518}} \\  

     &  & \mc{1}{c}{\scriptsize{(1.000)}} & \mc{1}{c}{\scriptsize{(1.000)}} & \mc{1}{c}{\scriptsize{(1.000)}} & \mc{1}{c}{\scriptsize{(0.971)}} & \mc{1}{c}{\scriptsize{(1.000)}} & \mc{1}{c}{\scriptsize{(1.000)}} & \mc{1}{c}{\scriptsize{(1.000)}} & \mc{1}{c}{\scriptsize{(1.000)}} \\  

    \mc{1}{l}{\scriptsize{Red Cell Width}} & \mc{1}{c}{\scriptsize{Mid-30s}} & \mc{1}{c}{\scriptsize{-0.451}} & \mc{1}{c}{\scriptsize{-0.525}} & \mc{1}{c}{\scriptsize{-0.015}} & \mc{1}{c}{\scriptsize{-0.083}} & \mc{1}{c}{\scriptsize{-0.078}} & \mc{1}{c}{\scriptsize{-0.600}} & \mc{1}{c}{\scriptsize{-0.613}} & \mc{1}{c}{\scriptsize{-0.606}} \\  

     &  & \mc{1}{c}{\scriptsize{(1.000)}} & \mc{1}{c}{\scriptsize{(1.000)}} & \mc{1}{c}{\scriptsize{(1.000)}} & \mc{1}{c}{\scriptsize{(1.000)}} & \mc{1}{c}{\scriptsize{(1.000)}} & \mc{1}{c}{\scriptsize{(1.000)}} & \mc{1}{c}{\scriptsize{(1.000)}} & \mc{1}{c}{\scriptsize{(1.000)}} \\  

    \mc{1}{l}{\scriptsize{Mean Hb Concentration}} & \mc{1}{c}{\scriptsize{Mid-30s}} & \mc{1}{c}{\scriptsize{-0.040}} & \mc{1}{c}{\scriptsize{-0.068}} & \mc{1}{c}{\scriptsize{-0.314}} & \mc{1}{c}{\scriptsize{-0.420}} & \mc{1}{c}{\scriptsize{-0.173}} & \mc{1}{c}{\scriptsize{0.081}} & \mc{1}{c}{\scriptsize{0.012}} & \mc{1}{c}{\scriptsize{0.025}} \\  

     &  & \mc{1}{c}{\scriptsize{(1.000)}} & \mc{1}{c}{\scriptsize{(1.000)}} & \mc{1}{c}{\scriptsize{(1.000)}} & \mc{1}{c}{\scriptsize{(1.000)}} & \mc{1}{c}{\scriptsize{(1.000)}} & \mc{1}{c}{\scriptsize{(0.987)}} & \mc{1}{c}{\scriptsize{(1.000)}} & \mc{1}{c}{\scriptsize{(1.000)}} \\  

  \bottomrule
  \end{tabular}
	\end{table} 

	\begin{table}[H]
     \caption{Treatment Effects on Other Health-Related Information, Male Sample}
     \label{table:abccare_rslt_male_cat37_sd}
	\input{AppResOutput/abccare/rslt_male_cat37_sd}
	\end{table} 

	\begin{table}[H]
     \caption{Treatment Effects on Past Medical History - Diagnosis (Self-Reported), Male Sample}
     \label{table:abccare_rslt_male_cat38_sd}
	  \begin{tabular}{cccccccccc}
  \toprule

    \scriptsize{Variable} & \scriptsize{Age} & \scriptsize{(1)} & \scriptsize{(2)} & \scriptsize{(3)} & \scriptsize{(4)} & \scriptsize{(5)} & \scriptsize{(6)} & \scriptsize{(7)} & \scriptsize{(8)} \\ 
    \midrule  

    \mc{1}{l}{\scriptsize{Ever Told Had: Arthritis/Gout/Lupus/Fibromyalgia}} & \mc{1}{c}{\scriptsize{Mid-30s}} & \mc{1}{c}{\scriptsize{0.034}} & \mc{1}{c}{\scriptsize{0.053}} & \mc{1}{c}{\scriptsize{-0.222}} & \mc{1}{c}{\scriptsize{-0.213}} & \mc{1}{c}{\scriptsize{-0.223}} & \mc{1}{c}{\scriptsize{0.111}} & \mc{1}{c}{\scriptsize{0.086}} & \mc{1}{c}{\scriptsize{0.083}} \\  

     &  & \mc{1}{c}{\scriptsize{(0.592)}} & \mc{1}{c}{\scriptsize{(0.612)}} & \mc{1}{c}{\scriptsize{(0.239)}} & \mc{1}{c}{\scriptsize{(0.226)}} & \mc{1}{c}{\scriptsize{(0.226)}} & \mc{1}{c}{\scriptsize{(0.971)}} & \mc{1}{c}{\scriptsize{(0.862)}} & \mc{1}{c}{\scriptsize{(0.931)}} \\  

    \mc{1}{l}{\scriptsize{Ever Told Had: Prediabetes}} & \mc{1}{c}{\scriptsize{Mid-30s}} &  &  &  &  &  &  &  &  \\  

     &  &  &  &  &  &  &  &  &  \\  

  \bottomrule
  \end{tabular}
	\end{table} 

	\begin{table}[H]
     \caption{Treatment Effects on Past Medical History - Surgery (Self-Reported), Male Sample}
     \label{table:abccare_rslt_male_cat39_sd}
	\input{AppResOutput/abccare/rslt_male_cat39_sd}
	\end{table} 

	\begin{table}[H]
     \caption{Treatment Effects on Physical Activity, Male Sample}
     \label{table:abccare_rslt_male_cat40_sd}
	  \begin{tabular}{cccccccccc}
  \toprule

    \scriptsize{Variable} & \scriptsize{Age} & \scriptsize{(1)} & \scriptsize{(2)} & \scriptsize{(3)} & \scriptsize{(4)} & \scriptsize{(5)} & \scriptsize{(6)} & \scriptsize{(7)} & \scriptsize{(8)} \\ 
    \midrule  

    \mc{1}{l}{\scriptsize{Level of Activity at Work}} & \mc{1}{c}{\scriptsize{Mid-30s}} & \mc{1}{c}{\scriptsize{-0.200}} & \mc{1}{c}{\scriptsize{0.532}} &  &  &  & \mc{1}{c}{\scriptsize{0.467}} & \mc{1}{c}{\scriptsize{0.532}} & \mc{1}{c}{\scriptsize{0.423}} \\  

     &  & \mc{1}{c}{\scriptsize{(0.634)}} & \mc{1}{c}{\scriptsize{(0.171)}} &  &  &  & \mc{1}{c}{\scriptsize{\textbf{(0.016)}}} & \mc{1}{c}{\scriptsize{(0.177)}} & \mc{1}{c}{\scriptsize{\textbf{(0.031)}}} \\  

  \bottomrule
  \end{tabular}
	\end{table} 

	\begin{table}[H]
     \caption{Treatment Effects on Physical Exam - Ear, Male Sample}
     \label{table:abccare_rslt_male_cat41_sd}
	  \begin{tabular}{cccccccccc}
  \toprule

    \scriptsize{Variable} & \scriptsize{Age} & \scriptsize{(1)} & \scriptsize{(2)} & \scriptsize{(3)} & \scriptsize{(4)} & \scriptsize{(5)} & \scriptsize{(6)} & \scriptsize{(7)} & \scriptsize{(8)} \\ 
    \midrule  

    \mc{1}{l}{\scriptsize{Ear: Auditory Canal}} & \mc{1}{c}{\scriptsize{Mid-30s}} & \mc{1}{c}{\scriptsize{0.012}} & \mc{1}{c}{\scriptsize{-0.028}} & \mc{1}{c}{\scriptsize{0.083}} & \mc{1}{c}{\scriptsize{0.124}} & \mc{1}{c}{\scriptsize{0.087}} & \mc{1}{c}{\scriptsize{-0.017}} & \mc{1}{c}{\scriptsize{-0.046}} & \mc{1}{c}{\scriptsize{-0.000}} \\  

     &  & \mc{1}{c}{\scriptsize{(0.776)}} & \mc{1}{c}{\scriptsize{(0.671)}} & \mc{1}{c}{\scriptsize{(0.985)}} & \mc{1}{c}{\scriptsize{(0.912)}} & \mc{1}{c}{\scriptsize{(0.985)}} & \mc{1}{c}{\scriptsize{(0.605)}} & \mc{1}{c}{\scriptsize{(0.566)}} & \mc{1}{c}{\scriptsize{(0.737)}} \\  

    \mc{1}{l}{\scriptsize{Eye: Eyeball}} & \mc{1}{c}{\scriptsize{Mid-30s}} &  &  &  &  &  &  &  &  \\  

     &  &  &  &  &  &  &  &  &  \\  

    \mc{1}{l}{\scriptsize{Eye: Fundi}} & \mc{1}{c}{\scriptsize{Mid-30s}} & \mc{1}{c}{\scriptsize{-0.101}} & \mc{1}{c}{\scriptsize{-0.218}} & \mc{1}{c}{\scriptsize{0.042}} & \mc{1}{c}{\scriptsize{-0.101}} & \mc{1}{c}{\scriptsize{0.046}} & \mc{1}{c}{\scriptsize{-0.158}} & \mc{1}{c}{\scriptsize{-0.248}} & \mc{1}{c}{\scriptsize{-0.265}} \\  

     &  & \mc{1}{c}{\scriptsize{(0.250)}} & \mc{1}{c}{\scriptsize{\textbf{(0.000)}}} & \mc{1}{c}{\scriptsize{(0.912)}} & \mc{1}{c}{\scriptsize{(0.265)}} & \mc{1}{c}{\scriptsize{(0.956)}} & \mc{1}{c}{\scriptsize{(0.184)}} & \mc{1}{c}{\scriptsize{\textbf{(0.053)}}} & \mc{1}{c}{\scriptsize{(0.118)}} \\  

  \bottomrule
  \end{tabular}
	\end{table} 

	\begin{table}[H]
     \caption{Treatment Effects on Physical Exam - General I, Male Sample}
     \label{table:abccare_rslt_male_cat42_sd}
	  \begin{tabular}{cccccccccc}
  \toprule

    \scriptsize{Variable} & \scriptsize{Age} & \scriptsize{(1)} & \scriptsize{(2)} & \scriptsize{(3)} & \scriptsize{(4)} & \scriptsize{(5)} & \scriptsize{(6)} & \scriptsize{(7)} & \scriptsize{(8)} \\ 
    \midrule  

    \mc{1}{l}{\scriptsize{Parental Labor Income}} & \mc{1}{c}{\scriptsize{0}} & \mc{1}{c}{\scriptsize{-2,639}} & \mc{1}{c}{\scriptsize{596}} & \mc{1}{c}{\scriptsize{-4,153}} & \mc{1}{c}{\scriptsize{1,431}} & \mc{1}{c}{\scriptsize{-2,870}} & \mc{1}{c}{\scriptsize{-1,556}} & \mc{1}{c}{\scriptsize{265}} & \mc{1}{c}{\scriptsize{261}} \\  

     &  & \mc{1}{c}{\scriptsize{(0.920)}} & \mc{1}{c}{\scriptsize{(0.427)}} & \mc{1}{c}{\scriptsize{(0.822)}} & \mc{1}{c}{\scriptsize{(0.425)}} & \mc{1}{c}{\scriptsize{(0.740)}} & \mc{1}{c}{\scriptsize{(0.716)}} & \mc{1}{c}{\scriptsize{(0.446)}} & \mc{1}{c}{\scriptsize{(0.459)}} \\  

  \bottomrule
  \end{tabular}
	\end{table} 

	\begin{table}[H]
     \caption{Treatment Effects on Physical Exam - General II, Male Sample}
     \label{table:abccare_rslt_male_cat43_sd}
	\input{AppResOutput/abccare/rslt_male_cat43_sd}
	\end{table} 

	\begin{table}[H]
     \caption{Treatment Effects on Physical Exam (Part II), Male Sample}
     \label{table:abccare_rslt_male_cat44_sd}
	  \begin{tabular}{cccccccccc}
  \toprule

    \scriptsize{Variable} & \scriptsize{Age} & \scriptsize{(1)} & \scriptsize{(2)} & \scriptsize{(3)} & \scriptsize{(4)} & \scriptsize{(5)} & \scriptsize{(6)} & \scriptsize{(7)} & \scriptsize{(8)} \\ 
    \midrule  

    \mc{1}{l}{\scriptsize{Behave Appropriate T Score (Reported by Teacher)}} & \mc{1}{c}{\scriptsize{12}} & \mc{1}{c}{\scriptsize{-7.611}} & \mc{1}{c}{\scriptsize{-5.971}} & \mc{1}{c}{\scriptsize{-14.446}} & \mc{1}{c}{\scriptsize{-18.741}} & \mc{1}{c}{\scriptsize{-12.410}} & \mc{1}{c}{\scriptsize{-4.763}} & \mc{1}{c}{\scriptsize{-2.124}} & \mc{1}{c}{\scriptsize{0.020}} \\  

     &  & \mc{1}{c}{\scriptsize{(1.000)}} & \mc{1}{c}{\scriptsize{(1.000)}} & \mc{1}{c}{\scriptsize{(1.000)}} & \mc{1}{c}{\scriptsize{(1.000)}} & \mc{1}{c}{\scriptsize{(1.000)}} & \mc{1}{c}{\scriptsize{(1.000)}} & \mc{1}{c}{\scriptsize{(1.000)}} & \mc{1}{c}{\scriptsize{(0.961)}} \\  

    \mc{1}{l}{\scriptsize{Activities T Score (Reported by Mother)}} & \mc{1}{c}{\scriptsize{12}} & \mc{1}{c}{\scriptsize{0.874}} & \mc{1}{c}{\scriptsize{-0.249}} & \mc{1}{c}{\scriptsize{-1.571}} & \mc{1}{c}{\scriptsize{-0.919}} & \mc{1}{c}{\scriptsize{-2.181}} & \mc{1}{c}{\scriptsize{1.508}} & \mc{1}{c}{\scriptsize{0.320}} & \mc{1}{c}{\scriptsize{0.747}} \\  

     &  & \mc{1}{c}{\scriptsize{(0.855)}} & \mc{1}{c}{\scriptsize{(0.974)}} & \mc{1}{c}{\scriptsize{(1.000)}} & \mc{1}{c}{\scriptsize{(1.000)}} & \mc{1}{c}{\scriptsize{(1.000)}} & \mc{1}{c}{\scriptsize{(0.724)}} & \mc{1}{c}{\scriptsize{(0.934)}} & \mc{1}{c}{\scriptsize{(0.908)}} \\  

    \mc{1}{l}{\scriptsize{Social T Score (Reported by Mother)}} & \mc{1}{c}{\scriptsize{8}} & \mc{1}{c}{\scriptsize{1.497}} & \mc{1}{c}{\scriptsize{2.152}} & \mc{1}{c}{\scriptsize{4.383}} & \mc{1}{c}{\scriptsize{6.379}} & \mc{1}{c}{\scriptsize{4.659}} & \mc{1}{c}{\scriptsize{0.595}} & \mc{1}{c}{\scriptsize{2.009}} & \mc{1}{c}{\scriptsize{2.274}} \\  

     &  & \mc{1}{c}{\scriptsize{(0.842)}} & \mc{1}{c}{\scriptsize{(0.539)}} & \mc{1}{c}{\scriptsize{(0.560)}} & \mc{1}{c}{\scriptsize{(0.600)}} & \mc{1}{c}{\scriptsize{(0.427)}} & \mc{1}{c}{\scriptsize{(0.908)}} & \mc{1}{c}{\scriptsize{(0.750)}} & \mc{1}{c}{\scriptsize{(0.632)}} \\  

    \mc{1}{l}{\scriptsize{Learning T Score (Reported by Teacher)}} & \mc{1}{c}{\scriptsize{12}} & \mc{1}{c}{\scriptsize{-2.244}} & \mc{1}{c}{\scriptsize{2.586}} & \mc{1}{c}{\scriptsize{-5.138}} & \mc{1}{c}{\scriptsize{-6.042}} & \mc{1}{c}{\scriptsize{-3.137}} & \mc{1}{c}{\scriptsize{-1.038}} & \mc{1}{c}{\scriptsize{4.899}} & \mc{1}{c}{\scriptsize{4.696}} \\  

     &  & \mc{1}{c}{\scriptsize{(1.000)}} & \mc{1}{c}{\scriptsize{(0.868)}} & \mc{1}{c}{\scriptsize{(1.000)}} & \mc{1}{c}{\scriptsize{(1.000)}} & \mc{1}{c}{\scriptsize{(1.000)}} & \mc{1}{c}{\scriptsize{(1.000)}} & \mc{1}{c}{\scriptsize{(0.763)}} & \mc{1}{c}{\scriptsize{(0.487)}} \\  

    \mc{1}{l}{\scriptsize{Social T Score (Reported by Mother)}} & \mc{1}{c}{\scriptsize{12}} & \mc{1}{c}{\scriptsize{0.941}} & \mc{1}{c}{\scriptsize{0.866}} & \mc{1}{c}{\scriptsize{2.319}} & \mc{1}{c}{\scriptsize{4.806}} & \mc{1}{c}{\scriptsize{1.699}} & \mc{1}{c}{\scriptsize{0.584}} & \mc{1}{c}{\scriptsize{0.426}} & \mc{1}{c}{\scriptsize{0.130}} \\  

     &  & \mc{1}{c}{\scriptsize{(0.855)}} & \mc{1}{c}{\scriptsize{(0.868)}} & \mc{1}{c}{\scriptsize{(0.853)}} & \mc{1}{c}{\scriptsize{(0.400)}} & \mc{1}{c}{\scriptsize{(0.893)}} & \mc{1}{c}{\scriptsize{(0.895)}} & \mc{1}{c}{\scriptsize{(0.921)}} & \mc{1}{c}{\scriptsize{(0.947)}} \\  

    \mc{1}{l}{\scriptsize{Activities T Score (Reported by Mother)}} & \mc{1}{c}{\scriptsize{8}} & \mc{1}{c}{\scriptsize{1.725}} & \mc{1}{c}{\scriptsize{1.441}} & \mc{1}{c}{\scriptsize{0.191}} & \mc{1}{c}{\scriptsize{-4.476}} & \mc{1}{c}{\scriptsize{1.495}} & \mc{1}{c}{\scriptsize{2.204}} & \mc{1}{c}{\scriptsize{2.458}} & \mc{1}{c}{\scriptsize{0.869}} \\  

     &  & \mc{1}{c}{\scriptsize{(0.737)}} & \mc{1}{c}{\scriptsize{(0.789)}} & \mc{1}{c}{\scriptsize{(0.987)}} & \mc{1}{c}{\scriptsize{(1.000)}} & \mc{1}{c}{\scriptsize{(0.893)}} & \mc{1}{c}{\scriptsize{(0.632)}} & \mc{1}{c}{\scriptsize{(0.395)}} & \mc{1}{c}{\scriptsize{(0.908)}} \\  

    \mc{1}{l}{\scriptsize{Work Hard T Score (Reported by Teacher)}} & \mc{1}{c}{\scriptsize{12}} & \mc{1}{c}{\scriptsize{-3.063}} & \mc{1}{c}{\scriptsize{2.048}} & \mc{1}{c}{\scriptsize{-8.769}} & \mc{1}{c}{\scriptsize{-9.480}} & \mc{1}{c}{\scriptsize{-6.456}} & \mc{1}{c}{\scriptsize{-0.686}} & \mc{1}{c}{\scriptsize{6.128}} & \mc{1}{c}{\scriptsize{4.633}} \\  

     &  & \mc{1}{c}{\scriptsize{(1.000)}} & \mc{1}{c}{\scriptsize{(0.868)}} & \mc{1}{c}{\scriptsize{(1.000)}} & \mc{1}{c}{\scriptsize{(1.000)}} & \mc{1}{c}{\scriptsize{(1.000)}} & \mc{1}{c}{\scriptsize{(1.000)}} & \mc{1}{c}{\scriptsize{(0.684)}} & \mc{1}{c}{\scriptsize{(0.395)}} \\  

    \mc{1}{l}{\scriptsize{Happiness T Score (Reported by Teacher)}} & \mc{1}{c}{\scriptsize{12}} & \mc{1}{c}{\scriptsize{-8.280}} & \mc{1}{c}{\scriptsize{-2.655}} & \mc{1}{c}{\scriptsize{-9.292}} & \mc{1}{c}{\scriptsize{-11.163}} & \mc{1}{c}{\scriptsize{-7.530}} & \mc{1}{c}{\scriptsize{-7.859}} & \mc{1}{c}{\scriptsize{-0.711}} & \mc{1}{c}{\scriptsize{-6.434}} \\  

     &  & \mc{1}{c}{\scriptsize{(1.000)}} & \mc{1}{c}{\scriptsize{(0.987)}} & \mc{1}{c}{\scriptsize{(1.000)}} & \mc{1}{c}{\scriptsize{(1.000)}} & \mc{1}{c}{\scriptsize{(1.000)}} & \mc{1}{c}{\scriptsize{(1.000)}} & \mc{1}{c}{\scriptsize{(0.987)}} & \mc{1}{c}{\scriptsize{(1.000)}} \\  

  \bottomrule
  \end{tabular}
	\end{table} 

	\begin{table}[H]
     \caption{Treatment Effects on Age 21 Brief Symptom Inventory, Male Sample}
     \label{table:abccare_rslt_male_cat45_sd}
	  \begin{tabular}{cccccccccc}
  \toprule

    \scriptsize{Variable} & \scriptsize{Age} & \scriptsize{(1)} & \scriptsize{(2)} & \scriptsize{(3)} & \scriptsize{(4)} & \scriptsize{(5)} & \scriptsize{(6)} & \scriptsize{(7)} & \scriptsize{(8)} \\ 
    \midrule  

    \mc{1}{l}{\scriptsize{Paranoid Ideation}} & \mc{1}{c}{\scriptsize{21}} & \mc{1}{c}{\scriptsize{1.818}} & \mc{1}{c}{\scriptsize{2.449}} & \mc{1}{c}{\scriptsize{1.788}} & \mc{1}{c}{\scriptsize{3.955}} & \mc{1}{c}{\scriptsize{2.021}} & \mc{1}{c}{\scriptsize{2.188}} & \mc{1}{c}{\scriptsize{2.899}} & \mc{1}{c}{\scriptsize{2.022}} \\  

     &  & \mc{1}{c}{\scriptsize{(1.000)}} & \mc{1}{c}{\scriptsize{(1.000)}} & \mc{1}{c}{\scriptsize{(0.987)}} & \mc{1}{c}{\scriptsize{(1.000)}} & \mc{1}{c}{\scriptsize{(0.987)}} & \mc{1}{c}{\scriptsize{(1.000)}} & \mc{1}{c}{\scriptsize{(1.000)}} & \mc{1}{c}{\scriptsize{(0.987)}} \\  

    \mc{1}{l}{\scriptsize{Obsessive-Compulsive}} & \mc{1}{c}{\scriptsize{21}} & \mc{1}{c}{\scriptsize{-1.400}} & \mc{1}{c}{\scriptsize{-1.428}} & \mc{1}{c}{\scriptsize{1.171}} & \mc{1}{c}{\scriptsize{1.171}} & \mc{1}{c}{\scriptsize{1.434}} & \mc{1}{c}{\scriptsize{-1.983}} & \mc{1}{c}{\scriptsize{-2.333}} & \mc{1}{c}{\scriptsize{-1.789}} \\  

     &  & \mc{1}{c}{\scriptsize{(0.684)}} & \mc{1}{c}{\scriptsize{(0.671)}} & \mc{1}{c}{\scriptsize{(0.987)}} & \mc{1}{c}{\scriptsize{(0.920)}} & \mc{1}{c}{\scriptsize{(0.973)}} & \mc{1}{c}{\scriptsize{(0.513)}} & \mc{1}{c}{\scriptsize{(0.474)}} & \mc{1}{c}{\scriptsize{(0.592)}} \\  

    \mc{1}{l}{\scriptsize{Interpersonal Sense}} & \mc{1}{c}{\scriptsize{21}} & \mc{1}{c}{\scriptsize{0.184}} & \mc{1}{c}{\scriptsize{-1.189}} & \mc{1}{c}{\scriptsize{1.727}} & \mc{1}{c}{\scriptsize{-0.954}} & \mc{1}{c}{\scriptsize{0.897}} & \mc{1}{c}{\scriptsize{-0.236}} & \mc{1}{c}{\scriptsize{-1.605}} & \mc{1}{c}{\scriptsize{-1.324}} \\  

     &  & \mc{1}{c}{\scriptsize{(0.947)}} & \mc{1}{c}{\scriptsize{(0.750)}} & \mc{1}{c}{\scriptsize{(0.987)}} & \mc{1}{c}{\scriptsize{(0.747)}} & \mc{1}{c}{\scriptsize{(0.960)}} & \mc{1}{c}{\scriptsize{(0.895)}} & \mc{1}{c}{\scriptsize{(0.750)}} & \mc{1}{c}{\scriptsize{(0.737)}} \\  

    \mc{1}{l}{\scriptsize{Positive Symptom Distress Index (PSI)}} & \mc{1}{c}{\scriptsize{21}} & \mc{1}{c}{\scriptsize{0.441}} & \mc{1}{c}{\scriptsize{0.511}} & \mc{1}{c}{\scriptsize{4.571}} & \mc{1}{c}{\scriptsize{4.242}} & \mc{1}{c}{\scriptsize{4.193}} & \mc{1}{c}{\scriptsize{-0.154}} & \mc{1}{c}{\scriptsize{0.119}} & \mc{1}{c}{\scriptsize{-0.701}} \\  

     &  & \mc{1}{c}{\scriptsize{(0.961)}} & \mc{1}{c}{\scriptsize{(0.961)}} & \mc{1}{c}{\scriptsize{(1.000)}} & \mc{1}{c}{\scriptsize{(1.000)}} & \mc{1}{c}{\scriptsize{(0.987)}} & \mc{1}{c}{\scriptsize{(0.895)}} & \mc{1}{c}{\scriptsize{(0.934)}} & \mc{1}{c}{\scriptsize{(0.842)}} \\  

    \mc{1}{l}{\scriptsize{Psychoticism}} & \mc{1}{c}{\scriptsize{21}} & \mc{1}{c}{\scriptsize{-3.613}} & \mc{1}{c}{\scriptsize{-2.961}} & \mc{1}{c}{\scriptsize{-3.277}} & \mc{1}{c}{\scriptsize{-0.797}} & \mc{1}{c}{\scriptsize{-3.331}} & \mc{1}{c}{\scriptsize{-3.002}} & \mc{1}{c}{\scriptsize{-2.696}} & \mc{1}{c}{\scriptsize{-3.212}} \\  

     &  & \mc{1}{c}{\scriptsize{(0.368)}} & \mc{1}{c}{\scriptsize{(0.579)}} & \mc{1}{c}{\scriptsize{(0.613)}} & \mc{1}{c}{\scriptsize{(0.773)}} & \mc{1}{c}{\scriptsize{(0.627)}} & \mc{1}{c}{\scriptsize{(0.500)}} & \mc{1}{c}{\scriptsize{(0.671)}} & \mc{1}{c}{\scriptsize{(0.474)}} \\  

    \mc{1}{l}{\scriptsize{Phobic Anxiety}} & \mc{1}{c}{\scriptsize{21}} & \mc{1}{c}{\scriptsize{-1.702}} & \mc{1}{c}{\scriptsize{-4.387}} & \mc{1}{c}{\scriptsize{-0.445}} & \mc{1}{c}{\scriptsize{-7.741}} & \mc{1}{c}{\scriptsize{-1.248}} & \mc{1}{c}{\scriptsize{-1.810}} & \mc{1}{c}{\scriptsize{-3.622}} & \mc{1}{c}{\scriptsize{-3.322}} \\  

     &  & \mc{1}{c}{\scriptsize{(0.645)}} & \mc{1}{c}{\scriptsize{(0.105)}} & \mc{1}{c}{\scriptsize{(0.880)}} & \mc{1}{c}{\scriptsize{\textbf{(0.093)}}} & \mc{1}{c}{\scriptsize{(0.840)}} & \mc{1}{c}{\scriptsize{(0.645)}} & \mc{1}{c}{\scriptsize{(0.329)}} & \mc{1}{c}{\scriptsize{(0.368)}} \\  

    \mc{1}{l}{\scriptsize{Positive Symptom Total (PST)}} & \mc{1}{c}{\scriptsize{21}} & \mc{1}{c}{\scriptsize{0.362}} & \mc{1}{c}{\scriptsize{-0.785}} & \mc{1}{c}{\scriptsize{2.191}} & \mc{1}{c}{\scriptsize{-0.581}} & \mc{1}{c}{\scriptsize{1.528}} & \mc{1}{c}{\scriptsize{0.111}} & \mc{1}{c}{\scriptsize{-1.100}} & \mc{1}{c}{\scriptsize{-0.684}} \\  

     &  & \mc{1}{c}{\scriptsize{(0.961)}} & \mc{1}{c}{\scriptsize{(0.816)}} & \mc{1}{c}{\scriptsize{(0.987)}} & \mc{1}{c}{\scriptsize{(0.787)}} & \mc{1}{c}{\scriptsize{(0.987)}} & \mc{1}{c}{\scriptsize{(0.961)}} & \mc{1}{c}{\scriptsize{(0.829)}} & \mc{1}{c}{\scriptsize{(0.842)}} \\  

  \bottomrule
  \end{tabular}
	\end{table} 

	\begin{table}[H]
     \caption{Treatment Effects on Age 30 Adult Self Report DSM Scale $t$-Score, Male Sample}
     \label{table:abccare_rslt_male_cat46_sd}
	\input{AppResOutput/abccare/rslt_male_cat46_sd}
	\end{table} 

	\begin{table}[H]
     \caption{Treatment Effects on Age 30 Adult Self Report Syndrome Scale $t$-Score, Male Sample}
     \label{table:abccare_rslt_male_cat47_sd}
	  \begin{tabular}{cccccccccc}
  \toprule

    \scriptsize{Variable} & \scriptsize{Age} & \scriptsize{(1)} & \scriptsize{(2)} & \scriptsize{(3)} & \scriptsize{(4)} & \scriptsize{(5)} & \scriptsize{(6)} & \scriptsize{(7)} & \scriptsize{(8)} \\ 
    \midrule  

    \mc{1}{l}{\scriptsize{Somatic Complaints}} & \mc{1}{c}{\scriptsize{30}} & \mc{1}{c}{\scriptsize{1.994}} & \mc{1}{c}{\scriptsize{1.103}} & \mc{1}{c}{\scriptsize{1.686}} & \mc{1}{c}{\scriptsize{-0.368}} & \mc{1}{c}{\scriptsize{1.312}} & \mc{1}{c}{\scriptsize{2.021}} & \mc{1}{c}{\scriptsize{1.141}} & \mc{1}{c}{\scriptsize{1.585}} \\  

     &  & \mc{1}{c}{\scriptsize{(1.000)}} & \mc{1}{c}{\scriptsize{(1.000)}} & \mc{1}{c}{\scriptsize{(1.000)}} & \mc{1}{c}{\scriptsize{(0.880)}} & \mc{1}{c}{\scriptsize{(1.000)}} & \mc{1}{c}{\scriptsize{(1.000)}} & \mc{1}{c}{\scriptsize{(1.000)}} & \mc{1}{c}{\scriptsize{(1.000)}} \\  

    \mc{1}{l}{\scriptsize{Aggressive Behavior}} & \mc{1}{c}{\scriptsize{30}} & \mc{1}{c}{\scriptsize{2.546}} & \mc{1}{c}{\scriptsize{2.333}} & \mc{1}{c}{\scriptsize{3.943}} & \mc{1}{c}{\scriptsize{2.468}} & \mc{1}{c}{\scriptsize{4.431}} & \mc{1}{c}{\scriptsize{2.036}} & \mc{1}{c}{\scriptsize{1.986}} & \mc{1}{c}{\scriptsize{2.409}} \\  

     &  & \mc{1}{c}{\scriptsize{(1.000)}} & \mc{1}{c}{\scriptsize{(1.000)}} & \mc{1}{c}{\scriptsize{(1.000)}} & \mc{1}{c}{\scriptsize{(1.000)}} & \mc{1}{c}{\scriptsize{(1.000)}} & \mc{1}{c}{\scriptsize{(1.000)}} & \mc{1}{c}{\scriptsize{(1.000)}} & \mc{1}{c}{\scriptsize{(1.000)}} \\  

    \mc{1}{l}{\scriptsize{Intrusive}} & \mc{1}{c}{\scriptsize{30}} & \mc{1}{c}{\scriptsize{1.730}} & \mc{1}{c}{\scriptsize{0.833}} & \mc{1}{c}{\scriptsize{3.200}} & \mc{1}{c}{\scriptsize{2.372}} & \mc{1}{c}{\scriptsize{3.464}} & \mc{1}{c}{\scriptsize{1.343}} & \mc{1}{c}{\scriptsize{0.411}} & \mc{1}{c}{\scriptsize{1.565}} \\  

     &  & \mc{1}{c}{\scriptsize{(1.000)}} & \mc{1}{c}{\scriptsize{(1.000)}} & \mc{1}{c}{\scriptsize{(1.000)}} & \mc{1}{c}{\scriptsize{(1.000)}} & \mc{1}{c}{\scriptsize{(1.000)}} & \mc{1}{c}{\scriptsize{(1.000)}} & \mc{1}{c}{\scriptsize{(0.974)}} & \mc{1}{c}{\scriptsize{(1.000)}} \\  

    \mc{1}{l}{\scriptsize{Anxious/Depressed}} & \mc{1}{c}{\scriptsize{30}} & \mc{1}{c}{\scriptsize{1.838}} & \mc{1}{c}{\scriptsize{1.438}} & \mc{1}{c}{\scriptsize{2.400}} & \mc{1}{c}{\scriptsize{1.382}} & \mc{1}{c}{\scriptsize{2.796}} & \mc{1}{c}{\scriptsize{1.605}} & \mc{1}{c}{\scriptsize{1.294}} & \mc{1}{c}{\scriptsize{1.915}} \\  

     &  & \mc{1}{c}{\scriptsize{(1.000)}} & \mc{1}{c}{\scriptsize{(1.000)}} & \mc{1}{c}{\scriptsize{(1.000)}} & \mc{1}{c}{\scriptsize{(0.987)}} & \mc{1}{c}{\scriptsize{(1.000)}} & \mc{1}{c}{\scriptsize{(1.000)}} & \mc{1}{c}{\scriptsize{(1.000)}} & \mc{1}{c}{\scriptsize{(1.000)}} \\  

    \mc{1}{l}{\scriptsize{Externalizing}} & \mc{1}{c}{\scriptsize{30}} & \mc{1}{c}{\scriptsize{3.602}} & \mc{1}{c}{\scriptsize{2.499}} & \mc{1}{c}{\scriptsize{8.514}} & \mc{1}{c}{\scriptsize{5.969}} & \mc{1}{c}{\scriptsize{9.068}} & \mc{1}{c}{\scriptsize{2.347}} & \mc{1}{c}{\scriptsize{1.316}} & \mc{1}{c}{\scriptsize{2.874}} \\  

     &  & \mc{1}{c}{\scriptsize{(1.000)}} & \mc{1}{c}{\scriptsize{(1.000)}} & \mc{1}{c}{\scriptsize{(1.000)}} & \mc{1}{c}{\scriptsize{(1.000)}} & \mc{1}{c}{\scriptsize{(1.000)}} & \mc{1}{c}{\scriptsize{(1.000)}} & \mc{1}{c}{\scriptsize{(1.000)}} & \mc{1}{c}{\scriptsize{(1.000)}} \\  

    \mc{1}{l}{\scriptsize{Thought Problems}} & \mc{1}{c}{\scriptsize{30}} & \mc{1}{c}{\scriptsize{1.311}} & \mc{1}{c}{\scriptsize{-0.129}} & \mc{1}{c}{\scriptsize{3.343}} & \mc{1}{c}{\scriptsize{1.377}} & \mc{1}{c}{\scriptsize{2.902}} & \mc{1}{c}{\scriptsize{0.647}} & \mc{1}{c}{\scriptsize{-0.686}} & \mc{1}{c}{\scriptsize{0.032}} \\  

     &  & \mc{1}{c}{\scriptsize{(1.000)}} & \mc{1}{c}{\scriptsize{(0.961)}} & \mc{1}{c}{\scriptsize{(1.000)}} & \mc{1}{c}{\scriptsize{(0.987)}} & \mc{1}{c}{\scriptsize{(1.000)}} & \mc{1}{c}{\scriptsize{(0.987)}} & \mc{1}{c}{\scriptsize{(0.882)}} & \mc{1}{c}{\scriptsize{(0.961)}} \\  

    \mc{1}{l}{\scriptsize{Total Problems}} & \mc{1}{c}{\scriptsize{30}} & \mc{1}{c}{\scriptsize{4.192}} & \mc{1}{c}{\scriptsize{2.852}} & \mc{1}{c}{\scriptsize{7.800}} & \mc{1}{c}{\scriptsize{5.419}} & \mc{1}{c}{\scriptsize{8.144}} & \mc{1}{c}{\scriptsize{3.210}} & \mc{1}{c}{\scriptsize{1.777}} & \mc{1}{c}{\scriptsize{3.260}} \\  

     &  & \mc{1}{c}{\scriptsize{(1.000)}} & \mc{1}{c}{\scriptsize{(1.000)}} & \mc{1}{c}{\scriptsize{(1.000)}} & \mc{1}{c}{\scriptsize{(1.000)}} & \mc{1}{c}{\scriptsize{(1.000)}} & \mc{1}{c}{\scriptsize{(1.000)}} & \mc{1}{c}{\scriptsize{(1.000)}} & \mc{1}{c}{\scriptsize{(1.000)}} \\  

    \mc{1}{l}{\scriptsize{Withdrawn}} & \mc{1}{c}{\scriptsize{30}} & \mc{1}{c}{\scriptsize{1.775}} & \mc{1}{c}{\scriptsize{2.586}} & \mc{1}{c}{\scriptsize{1.886}} & \mc{1}{c}{\scriptsize{2.961}} & \mc{1}{c}{\scriptsize{2.268}} & \mc{1}{c}{\scriptsize{1.656}} & \mc{1}{c}{\scriptsize{2.164}} & \mc{1}{c}{\scriptsize{1.878}} \\  

     &  & \mc{1}{c}{\scriptsize{(1.000)}} & \mc{1}{c}{\scriptsize{(1.000)}} & \mc{1}{c}{\scriptsize{(1.000)}} & \mc{1}{c}{\scriptsize{(1.000)}} & \mc{1}{c}{\scriptsize{(1.000)}} & \mc{1}{c}{\scriptsize{(1.000)}} & \mc{1}{c}{\scriptsize{(1.000)}} & \mc{1}{c}{\scriptsize{(1.000)}} \\  

    \mc{1}{l}{\scriptsize{Internalizing}} & \mc{1}{c}{\scriptsize{30}} & \mc{1}{c}{\scriptsize{3.128}} & \mc{1}{c}{\scriptsize{2.117}} & \mc{1}{c}{\scriptsize{5.400}} & \mc{1}{c}{\scriptsize{3.103}} & \mc{1}{c}{\scriptsize{5.788}} & \mc{1}{c}{\scriptsize{2.170}} & \mc{1}{c}{\scriptsize{0.973}} & \mc{1}{c}{\scriptsize{2.086}} \\  

     &  & \mc{1}{c}{\scriptsize{(1.000)}} & \mc{1}{c}{\scriptsize{(1.000)}} & \mc{1}{c}{\scriptsize{(1.000)}} & \mc{1}{c}{\scriptsize{(0.987)}} & \mc{1}{c}{\scriptsize{(1.000)}} & \mc{1}{c}{\scriptsize{(0.987)}} & \mc{1}{c}{\scriptsize{(0.987)}} & \mc{1}{c}{\scriptsize{(1.000)}} \\  

    \mc{1}{l}{\scriptsize{Critical Items}} & \mc{1}{c}{\scriptsize{30}} & \mc{1}{c}{\scriptsize{0.531}} & \mc{1}{c}{\scriptsize{-0.124}} & \mc{1}{c}{\scriptsize{2.029}} & \mc{1}{c}{\scriptsize{0.158}} & \mc{1}{c}{\scriptsize{2.432}} & \mc{1}{c}{\scriptsize{0.190}} & \mc{1}{c}{\scriptsize{-0.402}} & \mc{1}{c}{\scriptsize{0.422}} \\  

     &  & \mc{1}{c}{\scriptsize{(1.000)}} & \mc{1}{c}{\scriptsize{(0.961)}} & \mc{1}{c}{\scriptsize{(1.000)}} & \mc{1}{c}{\scriptsize{(0.947)}} & \mc{1}{c}{\scriptsize{(1.000)}} & \mc{1}{c}{\scriptsize{(0.987)}} & \mc{1}{c}{\scriptsize{(0.921)}} & \mc{1}{c}{\scriptsize{(0.974)}} \\  

    \mc{1}{l}{\scriptsize{Rule Breaking}} & \mc{1}{c}{\scriptsize{30}} & \mc{1}{c}{\scriptsize{1.937}} & \mc{1}{c}{\scriptsize{1.528}} & \mc{1}{c}{\scriptsize{4.057}} & \mc{1}{c}{\scriptsize{3.374}} & \mc{1}{c}{\scriptsize{4.399}} & \mc{1}{c}{\scriptsize{1.616}} & \mc{1}{c}{\scriptsize{1.182}} & \mc{1}{c}{\scriptsize{1.967}} \\  

     &  & \mc{1}{c}{\scriptsize{(1.000)}} & \mc{1}{c}{\scriptsize{(1.000)}} & \mc{1}{c}{\scriptsize{(1.000)}} & \mc{1}{c}{\scriptsize{(1.000)}} & \mc{1}{c}{\scriptsize{(1.000)}} & \mc{1}{c}{\scriptsize{(1.000)}} & \mc{1}{c}{\scriptsize{(1.000)}} & \mc{1}{c}{\scriptsize{(1.000)}} \\  

    \mc{1}{l}{\scriptsize{Attention Problems}} & \mc{1}{c}{\scriptsize{30}} & \mc{1}{c}{\scriptsize{1.208}} & \mc{1}{c}{\scriptsize{1.026}} & \mc{1}{c}{\scriptsize{1.600}} & \mc{1}{c}{\scriptsize{1.471}} & \mc{1}{c}{\scriptsize{2.057}} & \mc{1}{c}{\scriptsize{1.234}} & \mc{1}{c}{\scriptsize{0.982}} & \mc{1}{c}{\scriptsize{1.565}} \\  

     &  & \mc{1}{c}{\scriptsize{(1.000)}} & \mc{1}{c}{\scriptsize{(1.000)}} & \mc{1}{c}{\scriptsize{(1.000)}} & \mc{1}{c}{\scriptsize{(0.987)}} & \mc{1}{c}{\scriptsize{(1.000)}} & \mc{1}{c}{\scriptsize{(1.000)}} & \mc{1}{c}{\scriptsize{(1.000)}} & \mc{1}{c}{\scriptsize{(1.000)}} \\  

  \bottomrule
  \end{tabular}
	\end{table} 

	\begin{table}[H]
     \caption{Treatment Effects on BSI 18 $t$-Score, Male Sample}
     \label{table:abccare_rslt_male_cat48_sd}
	  \begin{tabular}{cccccccccc}
  \toprule

    \scriptsize{Variable} & \scriptsize{Age} & \scriptsize{(1)} & \scriptsize{(2)} & \scriptsize{(3)} & \scriptsize{(4)} & \scriptsize{(5)} & \scriptsize{(6)} & \scriptsize{(7)} & \scriptsize{(8)} \\ 
    \midrule  

    \mc{1}{l}{\scriptsize{Global Severity Index}} & \mc{1}{c}{\scriptsize{Mid-30s}} & \mc{1}{c}{\scriptsize{-1.675}} & \mc{1}{c}{\scriptsize{-2.999}} & \mc{1}{c}{\scriptsize{0.111}} & \mc{1}{c}{\scriptsize{1.730}} & \mc{1}{c}{\scriptsize{-0.607}} & \mc{1}{c}{\scriptsize{-2.989}} & \mc{1}{c}{\scriptsize{-4.200}} & \mc{1}{c}{\scriptsize{-2.822}} \\  

     &  & \mc{1}{c}{\scriptsize{(0.395)}} & \mc{1}{c}{\scriptsize{(0.395)}} & \mc{1}{c}{\scriptsize{(0.826)}} & \mc{1}{c}{\scriptsize{(0.841)}} & \mc{1}{c}{\scriptsize{(0.681)}} & \mc{1}{c}{\scriptsize{(0.382)}} & \mc{1}{c}{\scriptsize{(0.329)}} & \mc{1}{c}{\scriptsize{(0.342)}} \\  

    \mc{1}{l}{\scriptsize{Somatization}} & \mc{1}{c}{\scriptsize{Mid-30s}} & \mc{1}{c}{\scriptsize{-2.823}} & \mc{1}{c}{\scriptsize{-3.958}} & \mc{1}{c}{\scriptsize{-1.704}} & \mc{1}{c}{\scriptsize{-3.058}} & \mc{1}{c}{\scriptsize{-2.144}} & \mc{1}{c}{\scriptsize{-3.737}} & \mc{1}{c}{\scriptsize{-4.107}} & \mc{1}{c}{\scriptsize{-3.567}} \\  

     &  & \mc{1}{c}{\scriptsize{(0.289)}} & \mc{1}{c}{\scriptsize{(0.289)}} & \mc{1}{c}{\scriptsize{(0.391)}} & \mc{1}{c}{\scriptsize{(0.406)}} & \mc{1}{c}{\scriptsize{(0.304)}} & \mc{1}{c}{\scriptsize{(0.276)}} & \mc{1}{c}{\scriptsize{(0.276)}} & \mc{1}{c}{\scriptsize{(0.250)}} \\  

    \mc{1}{l}{\scriptsize{Anxiety}} & \mc{1}{c}{\scriptsize{Mid-30s}} & \mc{1}{c}{\scriptsize{-1.508}} & \mc{1}{c}{\scriptsize{-2.915}} & \mc{1}{c}{\scriptsize{1.111}} & \mc{1}{c}{\scriptsize{2.708}} & \mc{1}{c}{\scriptsize{0.439}} & \mc{1}{c}{\scriptsize{-2.822}} & \mc{1}{c}{\scriptsize{-3.931}} & \mc{1}{c}{\scriptsize{-2.694}} \\  

     &  & \mc{1}{c}{\scriptsize{(0.395)}} & \mc{1}{c}{\scriptsize{(0.303)}} & \mc{1}{c}{\scriptsize{(0.913)}} & \mc{1}{c}{\scriptsize{(0.928)}} & \mc{1}{c}{\scriptsize{(0.855)}} & \mc{1}{c}{\scriptsize{(0.342)}} & \mc{1}{c}{\scriptsize{(0.250)}} & \mc{1}{c}{\scriptsize{(0.316)}} \\  

    \mc{1}{l}{\scriptsize{Depression}} & \mc{1}{c}{\scriptsize{Mid-30s}} & \mc{1}{c}{\scriptsize{-1.135}} & \mc{1}{c}{\scriptsize{-2.524}} & \mc{1}{c}{\scriptsize{3.222}} & \mc{1}{c}{\scriptsize{5.281}} & \mc{1}{c}{\scriptsize{2.240}} & \mc{1}{c}{\scriptsize{-2.978}} & \mc{1}{c}{\scriptsize{-4.227}} & \mc{1}{c}{\scriptsize{-2.873}} \\  

     &  & \mc{1}{c}{\scriptsize{(0.408)}} & \mc{1}{c}{\scriptsize{(0.382)}} & \mc{1}{c}{\scriptsize{(1.000)}} & \mc{1}{c}{\scriptsize{(0.986)}} & \mc{1}{c}{\scriptsize{(0.986)}} & \mc{1}{c}{\scriptsize{(0.355)}} & \mc{1}{c}{\scriptsize{(0.224)}} & \mc{1}{c}{\scriptsize{(0.316)}} \\  

  \bottomrule
  \end{tabular}
	\end{table} 

	\begin{table}[H]
     \caption{Treatment Effects on BSI Raw Score, Male Sample}
     \label{table:abccare_rslt_male_cat49_sd}
	\input{AppResOutput/abccare/rslt_male_cat49_sd}
	\end{table} 

	\begin{table}[H]
     \caption{Treatment Effects on BSI $t$-Score, Male Sample}
     \label{table:abccare_rslt_male_cat50_sd}
	  \begin{tabular}{cccccccccc}
  \toprule

    \scriptsize{Variable} & \scriptsize{Age} & \scriptsize{(1)} & \scriptsize{(2)} & \scriptsize{(3)} & \scriptsize{(4)} & \scriptsize{(5)} & \scriptsize{(6)} & \scriptsize{(7)} & \scriptsize{(8)} \\ 
    \midrule  

    \mc{1}{l}{\scriptsize{Impulsitivity - control}} & \mc{1}{c}{\scriptsize{8}} & \mc{1}{c}{\scriptsize{0.938}} & \mc{1}{c}{\scriptsize{1.254}} & \mc{1}{c}{\scriptsize{1.585}} & \mc{1}{c}{\scriptsize{1.912}} & \mc{1}{c}{\scriptsize{1.605}} & \mc{1}{c}{\scriptsize{1.073}} & \mc{1}{c}{\scriptsize{1.311}} & \mc{1}{c}{\scriptsize{1.355}} \\  

     &  & \mc{1}{c}{\scriptsize{(1.000)}} & \mc{1}{c}{\scriptsize{(1.000)}} & \mc{1}{c}{\scriptsize{(1.000)}} & \mc{1}{c}{\scriptsize{(1.000)}} & \mc{1}{c}{\scriptsize{(1.000)}} & \mc{1}{c}{\scriptsize{(1.000)}} & \mc{1}{c}{\scriptsize{(1.000)}} & \mc{1}{c}{\scriptsize{(1.000)}} \\  

    \mc{1}{l}{\scriptsize{Impulsitivity - decisive}} & \mc{1}{c}{\scriptsize{8}} & \mc{1}{c}{\scriptsize{0.344}} & \mc{1}{c}{\scriptsize{-0.517}} & \mc{1}{c}{\scriptsize{2.013}} & \mc{1}{c}{\scriptsize{1.645}} & \mc{1}{c}{\scriptsize{1.679}} & \mc{1}{c}{\scriptsize{0.115}} & \mc{1}{c}{\scriptsize{-0.917}} & \mc{1}{c}{\scriptsize{-0.279}} \\  

     &  & \mc{1}{c}{\scriptsize{(1.000)}} & \mc{1}{c}{\scriptsize{(0.803)}} & \mc{1}{c}{\scriptsize{(1.000)}} & \mc{1}{c}{\scriptsize{(1.000)}} & \mc{1}{c}{\scriptsize{(1.000)}} & \mc{1}{c}{\scriptsize{(1.000)}} & \mc{1}{c}{\scriptsize{(0.592)}} & \mc{1}{c}{\scriptsize{(0.908)}} \\  

    \mc{1}{l}{\scriptsize{Activity - tempo}} & \mc{1}{c}{\scriptsize{8}} & \mc{1}{c}{\scriptsize{0.281}} & \mc{1}{c}{\scriptsize{-0.949}} & \mc{1}{c}{\scriptsize{-1.170}} & \mc{1}{c}{\scriptsize{-2.661}} & \mc{1}{c}{\scriptsize{-1.476}} & \mc{1}{c}{\scriptsize{0.896}} & \mc{1}{c}{\scriptsize{-0.360}} & \mc{1}{c}{\scriptsize{0.395}} \\  

     &  & \mc{1}{c}{\scriptsize{(1.000)}} & \mc{1}{c}{\scriptsize{(0.763)}} & \mc{1}{c}{\scriptsize{(0.840)}} & \mc{1}{c}{\scriptsize{(0.480)}} & \mc{1}{c}{\scriptsize{(0.787)}} & \mc{1}{c}{\scriptsize{(1.000)}} & \mc{1}{c}{\scriptsize{(0.921)}} & \mc{1}{c}{\scriptsize{(1.000)}} \\  

    \mc{1}{l}{\scriptsize{Emotionality - fear}} & \mc{1}{c}{\scriptsize{8}} & \mc{1}{c}{\scriptsize{1.094}} & \mc{1}{c}{\scriptsize{0.906}} & \mc{1}{c}{\scriptsize{0.464}} & \mc{1}{c}{\scriptsize{-0.635}} & \mc{1}{c}{\scriptsize{0.721}} & \mc{1}{c}{\scriptsize{1.500}} & \mc{1}{c}{\scriptsize{1.464}} & \mc{1}{c}{\scriptsize{1.769}} \\  

     &  & \mc{1}{c}{\scriptsize{(1.000)}} & \mc{1}{c}{\scriptsize{(1.000)}} & \mc{1}{c}{\scriptsize{(1.000)}} & \mc{1}{c}{\scriptsize{(0.947)}} & \mc{1}{c}{\scriptsize{(1.000)}} & \mc{1}{c}{\scriptsize{(1.000)}} & \mc{1}{c}{\scriptsize{(1.000)}} & \mc{1}{c}{\scriptsize{(1.000)}} \\  

    \mc{1}{l}{\scriptsize{Emotionality - anger}} & \mc{1}{c}{\scriptsize{8}} & \mc{1}{c}{\scriptsize{0.438}} & \mc{1}{c}{\scriptsize{0.433}} & \mc{1}{c}{\scriptsize{0.120}} & \mc{1}{c}{\scriptsize{-0.522}} & \mc{1}{c}{\scriptsize{0.382}} & \mc{1}{c}{\scriptsize{0.865}} & \mc{1}{c}{\scriptsize{0.837}} & \mc{1}{c}{\scriptsize{0.993}} \\  

     &  & \mc{1}{c}{\scriptsize{(1.000)}} & \mc{1}{c}{\scriptsize{(1.000)}} & \mc{1}{c}{\scriptsize{(1.000)}} & \mc{1}{c}{\scriptsize{(0.960)}} & \mc{1}{c}{\scriptsize{(1.000)}} & \mc{1}{c}{\scriptsize{(1.000)}} & \mc{1}{c}{\scriptsize{(1.000)}} & \mc{1}{c}{\scriptsize{(1.000)}} \\  

    \mc{1}{l}{\scriptsize{Impulsitivity - perservere}} & \mc{1}{c}{\scriptsize{8}} & \mc{1}{c}{\scriptsize{-0.406}} & \mc{1}{c}{\scriptsize{-0.255}} & \mc{1}{c}{\scriptsize{0.746}} & \mc{1}{c}{\scriptsize{1.570}} & \mc{1}{c}{\scriptsize{0.497}} & \mc{1}{c}{\scriptsize{-0.677}} & \mc{1}{c}{\scriptsize{-0.680}} & \mc{1}{c}{\scriptsize{-0.681}} \\  

     &  & \mc{1}{c}{\scriptsize{(0.921)}} & \mc{1}{c}{\scriptsize{(0.974)}} & \mc{1}{c}{\scriptsize{(1.000)}} & \mc{1}{c}{\scriptsize{(1.000)}} & \mc{1}{c}{\scriptsize{(1.000)}} & \mc{1}{c}{\scriptsize{(0.868)}} & \mc{1}{c}{\scriptsize{(0.829)}} & \mc{1}{c}{\scriptsize{(0.882)}} \\  

    \mc{1}{l}{\scriptsize{Sociablity}} & \mc{1}{c}{\scriptsize{8}} & \mc{1}{c}{\scriptsize{0.094}} & \mc{1}{c}{\scriptsize{0.133}} & \mc{1}{c}{\scriptsize{-0.844}} & \mc{1}{c}{\scriptsize{0.198}} & \mc{1}{c}{\scriptsize{-0.738}} & \mc{1}{c}{\scriptsize{0.115}} & \mc{1}{c}{\scriptsize{0.140}} & \mc{1}{c}{\scriptsize{0.244}} \\  

     &  & \mc{1}{c}{\scriptsize{(0.974)}} & \mc{1}{c}{\scriptsize{(0.974)}} & \mc{1}{c}{\scriptsize{(1.000)}} & \mc{1}{c}{\scriptsize{(0.973)}} & \mc{1}{c}{\scriptsize{(1.000)}} & \mc{1}{c}{\scriptsize{(0.974)}} & \mc{1}{c}{\scriptsize{(0.947)}} & \mc{1}{c}{\scriptsize{(0.947)}} \\  

    \mc{1}{l}{\scriptsize{Emotionality - general}} & \mc{1}{c}{\scriptsize{8}} & \mc{1}{c}{\scriptsize{0.969}} & \mc{1}{c}{\scriptsize{0.827}} & \mc{1}{c}{\scriptsize{3.634}} & \mc{1}{c}{\scriptsize{3.741}} & \mc{1}{c}{\scriptsize{3.617}} & \mc{1}{c}{\scriptsize{0.229}} & \mc{1}{c}{\scriptsize{-0.017}} & \mc{1}{c}{\scriptsize{-0.011}} \\  

     &  & \mc{1}{c}{\scriptsize{(1.000)}} & \mc{1}{c}{\scriptsize{(1.000)}} & \mc{1}{c}{\scriptsize{(1.000)}} & \mc{1}{c}{\scriptsize{(1.000)}} & \mc{1}{c}{\scriptsize{(1.000)}} & \mc{1}{c}{\scriptsize{(1.000)}} & \mc{1}{c}{\scriptsize{(1.000)}} & \mc{1}{c}{\scriptsize{(0.974)}} \\  

    \mc{1}{l}{\scriptsize{Activity - vigor}} & \mc{1}{c}{\scriptsize{8}} & \mc{1}{c}{\scriptsize{0.062}} & \mc{1}{c}{\scriptsize{-0.509}} & \mc{1}{c}{\scriptsize{-1.326}} & \mc{1}{c}{\scriptsize{-1.246}} & \mc{1}{c}{\scriptsize{-1.509}} & \mc{1}{c}{\scriptsize{0.448}} & \mc{1}{c}{\scriptsize{-0.182}} & \mc{1}{c}{\scriptsize{0.313}} \\  

     &  & \mc{1}{c}{\scriptsize{(0.987)}} & \mc{1}{c}{\scriptsize{(1.000)}} & \mc{1}{c}{\scriptsize{(1.000)}} & \mc{1}{c}{\scriptsize{(1.000)}} & \mc{1}{c}{\scriptsize{(1.000)}} & \mc{1}{c}{\scriptsize{(0.947)}} & \mc{1}{c}{\scriptsize{(1.000)}} & \mc{1}{c}{\scriptsize{(0.947)}} \\  

    \mc{1}{l}{\scriptsize{Impulsitivity - sensation}} & \mc{1}{c}{\scriptsize{8}} & \mc{1}{c}{\scriptsize{-0.500}} & \mc{1}{c}{\scriptsize{-0.396}} & \mc{1}{c}{\scriptsize{-1.085}} & \mc{1}{c}{\scriptsize{-0.529}} & \mc{1}{c}{\scriptsize{-1.019}} & \mc{1}{c}{\scriptsize{-0.240}} & \mc{1}{c}{\scriptsize{-0.234}} & \mc{1}{c}{\scriptsize{-0.240}} \\  

     &  & \mc{1}{c}{\scriptsize{(0.829)}} & \mc{1}{c}{\scriptsize{(0.908)}} & \mc{1}{c}{\scriptsize{(0.840)}} & \mc{1}{c}{\scriptsize{(0.947)}} & \mc{1}{c}{\scriptsize{(0.853)}} & \mc{1}{c}{\scriptsize{(0.947)}} & \mc{1}{c}{\scriptsize{(0.921)}} & \mc{1}{c}{\scriptsize{(0.947)}} \\  

  \bottomrule
  \end{tabular}
	\end{table} 

	\begin{table}[H]
     \caption{Treatment Effects on Mid-30s Mental Health Conditions, Male Sample}
     \label{table:abccare_rslt_male_cat51_sd}
	  \begin{tabular}{cccccccccc}
  \toprule

    \scriptsize{Variable} & \scriptsize{Age} & \scriptsize{(1)} & \scriptsize{(2)} & \scriptsize{(3)} & \scriptsize{(4)} & \scriptsize{(5)} & \scriptsize{(6)} & \scriptsize{(7)} & \scriptsize{(8)} \\ 
    \midrule  

    \mc{1}{l}{\scriptsize{Current Condition: Any Psychiatric Concern}} & \mc{1}{c}{\scriptsize{Mid-30s}} & \mc{1}{c}{\scriptsize{0.125}} & \mc{1}{c}{\scriptsize{0.132}} & \mc{1}{c}{\scriptsize{0.125}} & \mc{1}{c}{\scriptsize{-0.073}} & \mc{1}{c}{\scriptsize{0.149}} & \mc{1}{c}{\scriptsize{0.125}} & \mc{1}{c}{\scriptsize{0.172}} & \mc{1}{c}{\scriptsize{0.150}} \\  

     &  & \mc{1}{c}{\scriptsize{(1.000)}} & \mc{1}{c}{\scriptsize{(0.987)}} & \mc{1}{c}{\scriptsize{(1.000)}} & \mc{1}{c}{\scriptsize{(0.826)}} & \mc{1}{c}{\scriptsize{(1.000)}} & \mc{1}{c}{\scriptsize{(1.000)}} & \mc{1}{c}{\scriptsize{(0.987)}} & \mc{1}{c}{\scriptsize{(1.000)}} \\  

    \mc{1}{l}{\scriptsize{Current Condition: Sad/Depressed in Past 30 Days}} & \mc{1}{c}{\scriptsize{Mid-30s}} & \mc{1}{c}{\scriptsize{-0.423}} & \mc{1}{c}{\scriptsize{-0.828}} & \mc{1}{c}{\scriptsize{1.815}} & \mc{1}{c}{\scriptsize{3.242}} & \mc{1}{c}{\scriptsize{1.730}} & \mc{1}{c}{\scriptsize{-1.352}} & \mc{1}{c}{\scriptsize{-1.598}} & \mc{1}{c}{\scriptsize{-0.913}} \\  

     &  & \mc{1}{c}{\scriptsize{(0.829)}} & \mc{1}{c}{\scriptsize{(0.750)}} & \mc{1}{c}{\scriptsize{(1.000)}} & \mc{1}{c}{\scriptsize{(1.000)}} & \mc{1}{c}{\scriptsize{(1.000)}} & \mc{1}{c}{\scriptsize{(0.737)}} & \mc{1}{c}{\scriptsize{(0.632)}} & \mc{1}{c}{\scriptsize{(0.816)}} \\  

    \mc{1}{l}{\scriptsize{Current Condition: Mental Problems}} & \mc{1}{c}{\scriptsize{Mid-30s}} & \mc{1}{c}{\scriptsize{0.045}} & \mc{1}{c}{\scriptsize{0.018}} & \mc{1}{c}{\scriptsize{-0.074}} & \mc{1}{c}{\scriptsize{-0.141}} & \mc{1}{c}{\scriptsize{-0.072}} & \mc{1}{c}{\scriptsize{0.059}} & \mc{1}{c}{\scriptsize{0.052}} & \mc{1}{c}{\scriptsize{0.058}} \\  

     &  & \mc{1}{c}{\scriptsize{(1.000)}} & \mc{1}{c}{\scriptsize{(0.908)}} & \mc{1}{c}{\scriptsize{(0.899)}} & \mc{1}{c}{\scriptsize{(0.783)}} & \mc{1}{c}{\scriptsize{(0.855)}} & \mc{1}{c}{\scriptsize{(1.000)}} & \mc{1}{c}{\scriptsize{(0.921)}} & \mc{1}{c}{\scriptsize{(0.974)}} \\  

    \mc{1}{l}{\scriptsize{Current Condition: Worried/Anxious in Past 30 Days}} & \mc{1}{c}{\scriptsize{Mid-30s}} & \mc{1}{c}{\scriptsize{1.341}} & \mc{1}{c}{\scriptsize{1.424}} & \mc{1}{c}{\scriptsize{-1.444}} & \mc{1}{c}{\scriptsize{-1.318}} & \mc{1}{c}{\scriptsize{-0.958}} & \mc{1}{c}{\scriptsize{1.956}} & \mc{1}{c}{\scriptsize{1.787}} & \mc{1}{c}{\scriptsize{2.216}} \\  

     &  & \mc{1}{c}{\scriptsize{(1.000)}} & \mc{1}{c}{\scriptsize{(0.987)}} & \mc{1}{c}{\scriptsize{(0.812)}} & \mc{1}{c}{\scriptsize{(0.797)}} & \mc{1}{c}{\scriptsize{(0.841)}} & \mc{1}{c}{\scriptsize{(1.000)}} & \mc{1}{c}{\scriptsize{(0.987)}} & \mc{1}{c}{\scriptsize{(1.000)}} \\  

    \mc{1}{l}{\scriptsize{Current Condition: Anxiety}} & \mc{1}{c}{\scriptsize{Mid-30s}} & \mc{1}{c}{\scriptsize{0.042}} & \mc{1}{c}{\scriptsize{0.034}} & \mc{1}{c}{\scriptsize{0.042}} & \mc{1}{c}{\scriptsize{-0.043}} & \mc{1}{c}{\scriptsize{0.052}} & \mc{1}{c}{\scriptsize{0.042}} & \mc{1}{c}{\scriptsize{0.051}} & \mc{1}{c}{\scriptsize{0.052}} \\  

     &  & \mc{1}{c}{\scriptsize{(1.000)}} & \mc{1}{c}{\scriptsize{(0.987)}} & \mc{1}{c}{\scriptsize{(1.000)}} & \mc{1}{c}{\scriptsize{(0.783)}} & \mc{1}{c}{\scriptsize{(1.000)}} & \mc{1}{c}{\scriptsize{(1.000)}} & \mc{1}{c}{\scriptsize{(0.987)}} & \mc{1}{c}{\scriptsize{(1.000)}} \\  

    \mc{1}{l}{\scriptsize{Current Condition: Suicidal Ideation}} & \mc{1}{c}{\scriptsize{Mid-30s}} &  &  &  &  &  &  &  &  \\  

     &  &  &  &  &  &  &  &  &  \\  

    \mc{1}{l}{\scriptsize{Current Condition: Insomnia}} & \mc{1}{c}{\scriptsize{Mid-30s}} & \mc{1}{c}{\scriptsize{0.125}} & \mc{1}{c}{\scriptsize{0.132}} & \mc{1}{c}{\scriptsize{0.125}} & \mc{1}{c}{\scriptsize{-0.073}} & \mc{1}{c}{\scriptsize{0.149}} & \mc{1}{c}{\scriptsize{0.125}} & \mc{1}{c}{\scriptsize{0.172}} & \mc{1}{c}{\scriptsize{0.150}} \\  

     &  & \mc{1}{c}{\scriptsize{(1.000)}} & \mc{1}{c}{\scriptsize{(0.987)}} & \mc{1}{c}{\scriptsize{(1.000)}} & \mc{1}{c}{\scriptsize{(0.826)}} & \mc{1}{c}{\scriptsize{(1.000)}} & \mc{1}{c}{\scriptsize{(1.000)}} & \mc{1}{c}{\scriptsize{(0.987)}} & \mc{1}{c}{\scriptsize{(1.000)}} \\  

    \mc{1}{l}{\scriptsize{Current Condition: Depression}} & \mc{1}{c}{\scriptsize{Mid-30s}} &  &  &  &  &  &  &  &  \\  

     &  &  &  &  &  &  &  &  &  \\  

  \bottomrule
  \end{tabular}
	\end{table} 

	\begin{table}[H]
     \caption{Treatment Effects on Smoking and Drinking Behavior, Male Sample}
     \label{table:abccare_rslt_male_cat52_sd}
	  \begin{tabular}{cccccccccc}
  \toprule

    \scriptsize{Variable} & \scriptsize{Age} & \scriptsize{(1)} & \scriptsize{(2)} & \scriptsize{(3)} & \scriptsize{(4)} & \scriptsize{(5)} & \scriptsize{(6)} & \scriptsize{(7)} & \scriptsize{(8)} \\ 
    \midrule  

    \mc{1}{l}{\scriptsize{Temperament cluster - task orientation}} & \mc{1}{c}{\scriptsize{1}} & \mc{1}{c}{\scriptsize{0.896}} & \mc{1}{c}{\scriptsize{1.156}} & \mc{1}{c}{\scriptsize{0.792}} & \mc{1}{c}{\scriptsize{1.097}} & \mc{1}{c}{\scriptsize{1.027}} & \mc{1}{c}{\scriptsize{1.131}} & \mc{1}{c}{\scriptsize{1.488}} & \mc{1}{c}{\scriptsize{1.561}} \\  

     &  & \mc{1}{c}{\scriptsize{(0.750)}} & \mc{1}{c}{\scriptsize{(0.737)}} & \mc{1}{c}{\scriptsize{(0.811)}} & \mc{1}{c}{\scriptsize{(0.757)}} & \mc{1}{c}{\scriptsize{(0.703)}} & \mc{1}{c}{\scriptsize{(0.618)}} & \mc{1}{c}{\scriptsize{(0.461)}} & \mc{1}{c}{\scriptsize{(0.382)}} \\  

    \mc{1}{l}{\scriptsize{Temperament cluster - activity level}} & \mc{1}{c}{\scriptsize{1.5}} & \mc{1}{c}{\scriptsize{-0.917}} & \mc{1}{c}{\scriptsize{-1.168}} & \mc{1}{c}{\scriptsize{0.653}} & \mc{1}{c}{\scriptsize{0.164}} & \mc{1}{c}{\scriptsize{0.971}} & \mc{1}{c}{\scriptsize{-1.365}} & \mc{1}{c}{\scriptsize{-1.434}} & \mc{1}{c}{\scriptsize{-1.100}} \\  

     &  & \mc{1}{c}{\scriptsize{(1.000)}} & \mc{1}{c}{\scriptsize{(1.000)}} & \mc{1}{c}{\scriptsize{(0.973)}} & \mc{1}{c}{\scriptsize{(1.000)}} & \mc{1}{c}{\scriptsize{(0.932)}} & \mc{1}{c}{\scriptsize{(1.000)}} & \mc{1}{c}{\scriptsize{(1.000)}} & \mc{1}{c}{\scriptsize{(1.000)}} \\  

    \mc{1}{l}{\scriptsize{Temperament cluster - sociability}} & \mc{1}{c}{\scriptsize{2}} & \mc{1}{c}{\scriptsize{-0.145}} & \mc{1}{c}{\scriptsize{-0.270}} & \mc{1}{c}{\scriptsize{-1.458}} & \mc{1}{c}{\scriptsize{-0.627}} & \mc{1}{c}{\scriptsize{-1.470}} & \mc{1}{c}{\scriptsize{0.137}} & \mc{1}{c}{\scriptsize{-0.166}} & \mc{1}{c}{\scriptsize{-0.196}} \\  

     &  & \mc{1}{c}{\scriptsize{(1.000)}} & \mc{1}{c}{\scriptsize{(1.000)}} & \mc{1}{c}{\scriptsize{(1.000)}} & \mc{1}{c}{\scriptsize{(1.000)}} & \mc{1}{c}{\scriptsize{(1.000)}} & \mc{1}{c}{\scriptsize{(0.974)}} & \mc{1}{c}{\scriptsize{(1.000)}} & \mc{1}{c}{\scriptsize{(1.000)}} \\  

    \mc{1}{l}{\scriptsize{Temperament cluster - activity level}} & \mc{1}{c}{\scriptsize{0.5}} & \mc{1}{c}{\scriptsize{-0.097}} & \mc{1}{c}{\scriptsize{-0.107}} & \mc{1}{c}{\scriptsize{-0.231}} & \mc{1}{c}{\scriptsize{-1.080}} & \mc{1}{c}{\scriptsize{-0.392}} & \mc{1}{c}{\scriptsize{0.213}} & \mc{1}{c}{\scriptsize{0.191}} & \mc{1}{c}{\scriptsize{-0.062}} \\  

     &  & \mc{1}{c}{\scriptsize{(1.000)}} & \mc{1}{c}{\scriptsize{(1.000)}} & \mc{1}{c}{\scriptsize{(1.000)}} & \mc{1}{c}{\scriptsize{(1.000)}} & \mc{1}{c}{\scriptsize{(1.000)}} & \mc{1}{c}{\scriptsize{(0.974)}} & \mc{1}{c}{\scriptsize{(0.987)}} & \mc{1}{c}{\scriptsize{(1.000)}} \\  

     & \mc{1}{c}{\scriptsize{1}} & \mc{1}{c}{\scriptsize{0.886}} & \mc{1}{c}{\scriptsize{0.710}} & \mc{1}{c}{\scriptsize{1.944}} & \mc{1}{c}{\scriptsize{1.886}} & \mc{1}{c}{\scriptsize{1.728}} & \mc{1}{c}{\scriptsize{0.548}} & \mc{1}{c}{\scriptsize{0.411}} & \mc{1}{c}{\scriptsize{0.244}} \\  

     &  & \mc{1}{c}{\scriptsize{(0.658)}} & \mc{1}{c}{\scriptsize{(0.868)}} & \mc{1}{c}{\scriptsize{(0.284)}} & \mc{1}{c}{\scriptsize{(0.527)}} & \mc{1}{c}{\scriptsize{(0.527)}} & \mc{1}{c}{\scriptsize{(0.921)}} & \mc{1}{c}{\scriptsize{(0.961)}} & \mc{1}{c}{\scriptsize{(0.987)}} \\  

    \mc{1}{l}{\scriptsize{Temperament cluster - sociability}} & \mc{1}{c}{\scriptsize{0.5}} & \mc{1}{c}{\scriptsize{-0.228}} & \mc{1}{c}{\scriptsize{-0.105}} & \mc{1}{c}{\scriptsize{-0.168}} & \mc{1}{c}{\scriptsize{-0.116}} & \mc{1}{c}{\scriptsize{-0.221}} & \mc{1}{c}{\scriptsize{-0.196}} & \mc{1}{c}{\scriptsize{-0.081}} & \mc{1}{c}{\scriptsize{-0.268}} \\  

     &  & \mc{1}{c}{\scriptsize{(1.000)}} & \mc{1}{c}{\scriptsize{(1.000)}} & \mc{1}{c}{\scriptsize{(1.000)}} & \mc{1}{c}{\scriptsize{(1.000)}} & \mc{1}{c}{\scriptsize{(1.000)}} & \mc{1}{c}{\scriptsize{(1.000)}} & \mc{1}{c}{\scriptsize{(1.000)}} & \mc{1}{c}{\scriptsize{(1.000)}} \\  

    \mc{1}{l}{\scriptsize{Temperament cluster - cooperativeness}} & \mc{1}{c}{\scriptsize{2}} & \mc{1}{c}{\scriptsize{3.752}} & \mc{1}{c}{\scriptsize{2.651}} & \mc{1}{c}{\scriptsize{4.944}} & \mc{1}{c}{\scriptsize{3.413}} & \mc{1}{c}{\scriptsize{4.831}} & \mc{1}{c}{\scriptsize{3.535}} & \mc{1}{c}{\scriptsize{0.901}} & \mc{1}{c}{\scriptsize{3.305}} \\  

     &  & \mc{1}{c}{\scriptsize{(0.671)}} & \mc{1}{c}{\scriptsize{(0.974)}} & \mc{1}{c}{\scriptsize{(0.243)}} & \mc{1}{c}{\scriptsize{(0.959)}} & \mc{1}{c}{\scriptsize{(0.689)}} & \mc{1}{c}{\scriptsize{(0.711)}} & \mc{1}{c}{\scriptsize{(1.000)}} & \mc{1}{c}{\scriptsize{(0.921)}} \\  

     & \mc{1}{c}{\scriptsize{1}} & \mc{1}{c}{\scriptsize{-2.581}} & \mc{1}{c}{\scriptsize{-0.159}} & \mc{1}{c}{\scriptsize{-2.389}} & \mc{1}{c}{\scriptsize{2.118}} & \mc{1}{c}{\scriptsize{-0.979}} & \mc{1}{c}{\scriptsize{-2.616}} & \mc{1}{c}{\scriptsize{0.138}} & \mc{1}{c}{\scriptsize{-0.779}} \\  

     &  & \mc{1}{c}{\scriptsize{(1.000)}} & \mc{1}{c}{\scriptsize{(1.000)}} & \mc{1}{c}{\scriptsize{(1.000)}} & \mc{1}{c}{\scriptsize{(1.000)}} & \mc{1}{c}{\scriptsize{(1.000)}} & \mc{1}{c}{\scriptsize{(1.000)}} & \mc{1}{c}{\scriptsize{(1.000)}} & \mc{1}{c}{\scriptsize{(1.000)}} \\  

    \mc{1}{l}{\scriptsize{Temperament cluster - task orientation}} & \mc{1}{c}{\scriptsize{0.5}} & \mc{1}{c}{\scriptsize{-0.300}} & \mc{1}{c}{\scriptsize{0.534}} & \mc{1}{c}{\scriptsize{-0.685}} & \mc{1}{c}{\scriptsize{-0.837}} & \mc{1}{c}{\scriptsize{-0.444}} & \mc{1}{c}{\scriptsize{0.042}} & \mc{1}{c}{\scriptsize{0.889}} & \mc{1}{c}{\scriptsize{0.056}} \\  

     &  & \mc{1}{c}{\scriptsize{(1.000)}} & \mc{1}{c}{\scriptsize{(0.974)}} & \mc{1}{c}{\scriptsize{(1.000)}} & \mc{1}{c}{\scriptsize{(1.000)}} & \mc{1}{c}{\scriptsize{(1.000)}} & \mc{1}{c}{\scriptsize{(0.987)}} & \mc{1}{c}{\scriptsize{(0.737)}} & \mc{1}{c}{\scriptsize{(1.000)}} \\  

    \mc{1}{l}{\scriptsize{Temperament cluster - sociability}} & \mc{1}{c}{\scriptsize{1.5}} & \mc{1}{c}{\scriptsize{-0.710}} & \mc{1}{c}{\scriptsize{-0.341}} & \mc{1}{c}{\scriptsize{-1.571}} & \mc{1}{c}{\scriptsize{-0.972}} & \mc{1}{c}{\scriptsize{-1.200}} & \mc{1}{c}{\scriptsize{-0.464}} & \mc{1}{c}{\scriptsize{-0.188}} & \mc{1}{c}{\scriptsize{-0.094}} \\  

     &  & \mc{1}{c}{\scriptsize{(1.000)}} & \mc{1}{c}{\scriptsize{(1.000)}} & \mc{1}{c}{\scriptsize{(1.000)}} & \mc{1}{c}{\scriptsize{(1.000)}} & \mc{1}{c}{\scriptsize{(1.000)}} & \mc{1}{c}{\scriptsize{(1.000)}} & \mc{1}{c}{\scriptsize{(1.000)}} & \mc{1}{c}{\scriptsize{(1.000)}} \\  

    \mc{1}{l}{\scriptsize{Temperament cluster - cooperativeness}} & \mc{1}{c}{\scriptsize{1.5}} & \mc{1}{c}{\scriptsize{1.829}} & \mc{1}{c}{\scriptsize{5.005}} & \mc{1}{c}{\scriptsize{1.944}} & \mc{1}{c}{\scriptsize{6.344}} & \mc{1}{c}{\scriptsize{5.172}} & \mc{1}{c}{\scriptsize{1.808}} & \mc{1}{c}{\scriptsize{4.281}} & \mc{1}{c}{\scriptsize{5.850}} \\  

     &  & \mc{1}{c}{\scriptsize{(0.974)}} & \mc{1}{c}{\scriptsize{(0.368)}} & \mc{1}{c}{\scriptsize{(0.919)}} & \mc{1}{c}{\scriptsize{(0.946)}} & \mc{1}{c}{\scriptsize{(0.176)}} & \mc{1}{c}{\scriptsize{(0.974)}} & \mc{1}{c}{\scriptsize{(0.776)}} & \mc{1}{c}{\scriptsize{(0.171)}} \\  

    \mc{1}{l}{\scriptsize{Temperament cluster - activity level}} & \mc{1}{c}{\scriptsize{2}} & \mc{1}{c}{\scriptsize{-1.786}} & \mc{1}{c}{\scriptsize{-3.668}} & \mc{1}{c}{\scriptsize{-1.703}} & \mc{1}{c}{\scriptsize{-2.203}} & \mc{1}{c}{\scriptsize{-2.426}} & \mc{1}{c}{\scriptsize{-1.803}} & \mc{1}{c}{\scriptsize{-3.868}} & \mc{1}{c}{\scriptsize{-3.264}} \\  

     &  & \mc{1}{c}{\scriptsize{(1.000)}} & \mc{1}{c}{\scriptsize{(1.000)}} & \mc{1}{c}{\scriptsize{(1.000)}} & \mc{1}{c}{\scriptsize{(1.000)}} & \mc{1}{c}{\scriptsize{(1.000)}} & \mc{1}{c}{\scriptsize{(1.000)}} & \mc{1}{c}{\scriptsize{(1.000)}} & \mc{1}{c}{\scriptsize{(1.000)}} \\  

    \mc{1}{l}{\scriptsize{Temperament cluster - sociability}} & \mc{1}{c}{\scriptsize{1}} & \mc{1}{c}{\scriptsize{0.246}} & \mc{1}{c}{\scriptsize{0.428}} & \mc{1}{c}{\scriptsize{-1.111}} & \mc{1}{c}{\scriptsize{-1.127}} & \mc{1}{c}{\scriptsize{-0.984}} & \mc{1}{c}{\scriptsize{0.702}} & \mc{1}{c}{\scriptsize{0.880}} & \mc{1}{c}{\scriptsize{0.953}} \\  

     &  & \mc{1}{c}{\scriptsize{(0.961)}} & \mc{1}{c}{\scriptsize{(0.789)}} & \mc{1}{c}{\scriptsize{(1.000)}} & \mc{1}{c}{\scriptsize{(1.000)}} & \mc{1}{c}{\scriptsize{(1.000)}} & \mc{1}{c}{\scriptsize{(0.250)}} & \mc{1}{c}{\scriptsize{(0.224)}} & \mc{1}{c}{\scriptsize{\textbf{(0.053)}}} \\  

    \mc{1}{l}{\scriptsize{Temperament cluster - task orientation}} & \mc{1}{c}{\scriptsize{1.5}} & \mc{1}{c}{\scriptsize{1.861}} & \mc{1}{c}{\scriptsize{3.015}} & \mc{1}{c}{\scriptsize{-0.028}} & \mc{1}{c}{\scriptsize{1.599}} & \mc{1}{c}{\scriptsize{0.664}} & \mc{1}{c}{\scriptsize{2.401}} & \mc{1}{c}{\scriptsize{3.406}} & \mc{1}{c}{\scriptsize{3.026}} \\  

     &  & \mc{1}{c}{\scriptsize{(0.395)}} & \mc{1}{c}{\scriptsize{\textbf{(0.079)}}} & \mc{1}{c}{\scriptsize{(1.000)}} & \mc{1}{c}{\scriptsize{(0.865)}} & \mc{1}{c}{\scriptsize{(0.973)}} & \mc{1}{c}{\scriptsize{(0.171)}} & \mc{1}{c}{\scriptsize{\textbf{(0.066)}}} & \mc{1}{c}{\scriptsize{\textbf{(0.066)}}} \\  

    \mc{1}{l}{\scriptsize{Temperament cluster - cooperativeness}} & \mc{1}{c}{\scriptsize{0.5}} & \mc{1}{c}{\scriptsize{0.016}} & \mc{1}{c}{\scriptsize{-0.360}} & \mc{1}{c}{\scriptsize{-1.056}} & \mc{1}{c}{\scriptsize{-4.336}} & \mc{1}{c}{\scriptsize{-2.003}} & \mc{1}{c}{\scriptsize{0.626}} & \mc{1}{c}{\scriptsize{0.673}} & \mc{1}{c}{\scriptsize{0.110}} \\  

     &  & \mc{1}{c}{\scriptsize{(0.987)}} & \mc{1}{c}{\scriptsize{(1.000)}} & \mc{1}{c}{\scriptsize{(1.000)}} & \mc{1}{c}{\scriptsize{(1.000)}} & \mc{1}{c}{\scriptsize{(1.000)}} & \mc{1}{c}{\scriptsize{(0.974)}} & \mc{1}{c}{\scriptsize{(0.987)}} & \mc{1}{c}{\scriptsize{(1.000)}} \\  

    \mc{1}{l}{\scriptsize{Temperament cluster - task orientation}} & \mc{1}{c}{\scriptsize{2}} & \mc{1}{c}{\scriptsize{2.382}} & \mc{1}{c}{\scriptsize{4.061}} & \mc{1}{c}{\scriptsize{2.115}} & \mc{1}{c}{\scriptsize{3.444}} & \mc{1}{c}{\scriptsize{3.148}} & \mc{1}{c}{\scriptsize{2.435}} & \mc{1}{c}{\scriptsize{4.076}} & \mc{1}{c}{\scriptsize{3.926}} \\  

     &  & \mc{1}{c}{\scriptsize{(0.105)}} & \mc{1}{c}{\scriptsize{\textbf{(0.000)}}} & \mc{1}{c}{\scriptsize{(0.149)}} & \mc{1}{c}{\scriptsize{(0.108)}} & \mc{1}{c}{\scriptsize{\textbf{(0.000)}}} & \mc{1}{c}{\scriptsize{(0.118)}} & \mc{1}{c}{\scriptsize{\textbf{(0.026)}}} & \mc{1}{c}{\scriptsize{\textbf{(0.013)}}} \\  

  \bottomrule
  \end{tabular}
	\end{table} 

	\begin{table}[H]
     \caption{Treatment Effects on Tobacco, Drugs, Alcohol, Male Sample}
     \label{table:abccare_rslt_male_cat53_sd}
	  \begin{tabular}{cccccccccc}
  \toprule

    \scriptsize{Variable} & \scriptsize{Age} & \scriptsize{(1)} & \scriptsize{(2)} & \scriptsize{(3)} & \scriptsize{(4)} & \scriptsize{(5)} & \scriptsize{(6)} & \scriptsize{(7)} & \scriptsize{(8)} \\ 
    \midrule  

    \mc{1}{l}{\scriptsize{Days drank alcohol last month}} & \mc{1}{c}{\scriptsize{30}} & \mc{1}{c}{\scriptsize{0.805}} & \mc{1}{c}{\scriptsize{1.083}} & \mc{1}{c}{\scriptsize{-0.186}} & \mc{1}{c}{\scriptsize{-0.648}} & \mc{1}{c}{\scriptsize{0.060}} & \mc{1}{c}{\scriptsize{0.944}} & \mc{1}{c}{\scriptsize{1.348}} & \mc{1}{c}{\scriptsize{1.341}} \\  

     &  & \mc{1}{c}{\scriptsize{(0.829)}} & \mc{1}{c}{\scriptsize{(0.776)}} & \mc{1}{c}{\scriptsize{(0.671)}} & \mc{1}{c}{\scriptsize{(0.513)}} & \mc{1}{c}{\scriptsize{(0.671)}} & \mc{1}{c}{\scriptsize{(0.816)}} & \mc{1}{c}{\scriptsize{(0.842)}} & \mc{1}{c}{\scriptsize{(0.868)}} \\  

    \mc{1}{l}{\scriptsize{Days binge drank alcohol last month}} & \mc{1}{c}{\scriptsize{30}} & \mc{1}{c}{\scriptsize{0.500}} & \mc{1}{c}{\scriptsize{0.474}} & \mc{1}{c}{\scriptsize{0.543}} & \mc{1}{c}{\scriptsize{-0.360}} & \mc{1}{c}{\scriptsize{0.692}} & \mc{1}{c}{\scriptsize{0.491}} & \mc{1}{c}{\scriptsize{0.625}} & \mc{1}{c}{\scriptsize{0.702}} \\  

     &  & \mc{1}{c}{\scriptsize{(0.908)}} & \mc{1}{c}{\scriptsize{(0.895)}} & \mc{1}{c}{\scriptsize{(0.947)}} & \mc{1}{c}{\scriptsize{(0.461)}} & \mc{1}{c}{\scriptsize{(0.974)}} & \mc{1}{c}{\scriptsize{(0.895)}} & \mc{1}{c}{\scriptsize{(0.882)}} & \mc{1}{c}{\scriptsize{(0.961)}} \\  

  \bottomrule
  \end{tabular}
	\end{table} 
\section{Treatment Effects for Female Sample, Step Down}


	\begin{table}[H]
     \caption{Treatment Effects on IQ Scores, Female Sample}
     \label{table:abccare_rslt_female_cat0_sd}
	  \begin{tabular}{cccccccccc}
  \toprule

    \scriptsize{Variable} & \scriptsize{Age} & \scriptsize{(1)} & \scriptsize{(2)} & \scriptsize{(3)} & \scriptsize{(4)} & \scriptsize{(5)} & \scriptsize{(6)} & \scriptsize{(7)} & \scriptsize{(8)} \\ 
    \midrule  

    \mc{1}{l}{\scriptsize{Dyslipidemia}} & \mc{1}{c}{\scriptsize{Mid-30s}} & \mc{1}{c}{\scriptsize{0.051}} & \mc{1}{c}{\scriptsize{0.043}} & \mc{1}{c}{\scriptsize{-0.080}} & \mc{1}{c}{\scriptsize{-0.072}} & \mc{1}{c}{\scriptsize{-0.093}} & \mc{1}{c}{\scriptsize{0.087}} & \mc{1}{c}{\scriptsize{0.101}} & \mc{1}{c}{\scriptsize{0.096}} \\  

     &  & \mc{1}{c}{\scriptsize{(0.961)}} & \mc{1}{c}{\scriptsize{(0.829)}} & \mc{1}{c}{\scriptsize{(0.395)}} & \mc{1}{c}{\scriptsize{(0.566)}} & \mc{1}{c}{\scriptsize{(0.434)}} & \mc{1}{c}{\scriptsize{(1.000)}} & \mc{1}{c}{\scriptsize{(0.984)}} & \mc{1}{c}{\scriptsize{(1.000)}} \\  

    \mc{1}{l}{\scriptsize{High-Density Lipoprotein Chol. (mg/dL)}} & \mc{1}{c}{\scriptsize{Mid-30s}} & \mc{1}{c}{\scriptsize{2.884}} & \mc{1}{c}{\scriptsize{5.555}} & \mc{1}{c}{\scriptsize{10.514}} & \mc{1}{c}{\scriptsize{6.966}} & \mc{1}{c}{\scriptsize{13.319}} & \mc{1}{c}{\scriptsize{0.802}} & \mc{1}{c}{\scriptsize{4.446}} & \mc{1}{c}{\scriptsize{4.243}} \\  

     &  & \mc{1}{c}{\scriptsize{(0.329)}} & \mc{1}{c}{\scriptsize{(0.158)}} & \mc{1}{c}{\scriptsize{\textbf{(0.013)}}} & \mc{1}{c}{\scriptsize{(0.289)}} & \mc{1}{c}{\scriptsize{\textbf{(0.000)}}} & \mc{1}{c}{\scriptsize{(0.641)}} & \mc{1}{c}{\scriptsize{(0.281)}} & \mc{1}{c}{\scriptsize{(0.297)}} \\  

  \bottomrule
  \end{tabular}
	\end{table} 

	\begin{table}[H]
     \caption{Treatment Effects on Achievement Scores, Female Sample}
     \label{table:abccare_rslt_female_cat1_sd}
	  \begin{tabular}{cccccccccc}
  \toprule

    \scriptsize{Variable} & \scriptsize{Age} & \scriptsize{(1)} & \scriptsize{(2)} & \scriptsize{(3)} & \scriptsize{(4)} & \scriptsize{(5)} & \scriptsize{(6)} & \scriptsize{(7)} & \scriptsize{(8)} \\ 
    \midrule  

    \mc{1}{l}{\scriptsize{Std. Achv.  Test}} & \mc{1}{c}{\scriptsize{5.5}} & \mc{1}{c}{\scriptsize{8.029}} & \mc{1}{c}{\scriptsize{6.821}} & \mc{1}{c}{\scriptsize{14.284}} & \mc{1}{c}{\scriptsize{13.907}} & \mc{1}{c}{\scriptsize{18.475}} & \mc{1}{c}{\scriptsize{6.223}} & \mc{1}{c}{\scriptsize{4.725}} & \mc{1}{c}{\scriptsize{11.031}} \\  

     &  & \mc{1}{c}{\scriptsize{\textbf{(0.069)}}} & \mc{1}{c}{\scriptsize{(0.676)}} & \mc{1}{c}{\scriptsize{\textbf{(0.039)}}} & \mc{1}{c}{\scriptsize{\textbf{(0.059)}}} & \mc{1}{c}{\scriptsize{\textbf{(0.010)}}} & \mc{1}{c}{\scriptsize{(0.608)}} & \mc{1}{c}{\scriptsize{(2.853)}} & \mc{1}{c}{\scriptsize{(0.343)}} \\  

     & \mc{1}{c}{\scriptsize{6}} & \mc{1}{c}{\scriptsize{4.543}} & \mc{1}{c}{\scriptsize{5.225}} & \mc{1}{c}{\scriptsize{6.178}} & \mc{1}{c}{\scriptsize{7.895}} & \mc{1}{c}{\scriptsize{10.987}} & \mc{1}{c}{\scriptsize{4.075}} & \mc{1}{c}{\scriptsize{4.556}} & \mc{1}{c}{\scriptsize{5.888}} \\  

     &  & \mc{1}{c}{\scriptsize{(0.157)}} & \mc{1}{c}{\scriptsize{\textbf{(0.020)}}} & \mc{1}{c}{\scriptsize{(1.304)}} & \mc{1}{c}{\scriptsize{(0.441)}} & \mc{1}{c}{\scriptsize{\textbf{(0.049)}}} & \mc{1}{c}{\scriptsize{(0.451)}} & \mc{1}{c}{\scriptsize{(0.118)}} & \mc{1}{c}{\scriptsize{(0.588)}} \\  

     & \mc{1}{c}{\scriptsize{6.5}} & \mc{1}{c}{\scriptsize{2.767}} & \mc{1}{c}{\scriptsize{3.274}} & \mc{1}{c}{\scriptsize{2.049}} & \mc{1}{c}{\scriptsize{4.264}} & \mc{1}{c}{\scriptsize{6.018}} & \mc{1}{c}{\scriptsize{2.931}} & \mc{1}{c}{\scriptsize{3.066}} & \mc{1}{c}{\scriptsize{4.930}} \\  

     &  & \mc{1}{c}{\scriptsize{(0.657)}} & \mc{1}{c}{\scriptsize{(1.245)}} & \mc{1}{c}{\scriptsize{(4.343)}} & \mc{1}{c}{\scriptsize{(3.039)}} & \mc{1}{c}{\scriptsize{(0.794)}} & \mc{1}{c}{\scriptsize{(0.892)}} & \mc{1}{c}{\scriptsize{(2.647)}} & \mc{1}{c}{\scriptsize{(0.824)}} \\  

     & \mc{1}{c}{\scriptsize{7}} & \mc{1}{c}{\scriptsize{3.435}} & \mc{1}{c}{\scriptsize{3.147}} & \mc{1}{c}{\scriptsize{5.227}} & \mc{1}{c}{\scriptsize{6.630}} & \mc{1}{c}{\scriptsize{12.630}} & \mc{1}{c}{\scriptsize{3.025}} & \mc{1}{c}{\scriptsize{2.357}} & \mc{1}{c}{\scriptsize{6.473}} \\  

     &  & \mc{1}{c}{\scriptsize{(0.598)}} & \mc{1}{c}{\scriptsize{(1.490)}} & \mc{1}{c}{\scriptsize{(2.539)}} & \mc{1}{c}{\scriptsize{(1.029)}} & \mc{1}{c}{\scriptsize{\textbf{(0.010)}}} & \mc{1}{c}{\scriptsize{(1.088)}} & \mc{1}{c}{\scriptsize{(3.392)}} & \mc{1}{c}{\scriptsize{(0.510)}} \\  

     & \mc{1}{c}{\scriptsize{7.5}} & \mc{1}{c}{\scriptsize{1.937}} & \mc{1}{c}{\scriptsize{3.101}} & \mc{1}{c}{\scriptsize{0.667}} & \mc{1}{c}{\scriptsize{4.290}} & \mc{1}{c}{\scriptsize{4.569}} & \mc{1}{c}{\scriptsize{2.308}} & \mc{1}{c}{\scriptsize{2.946}} & \mc{1}{c}{\scriptsize{4.332}} \\  

     &  & \mc{1}{c}{\scriptsize{(1.451)}} & \mc{1}{c}{\scriptsize{(0.922)}} & \mc{1}{c}{\scriptsize{(4.441)}} & \mc{1}{c}{\scriptsize{(3.304)}} & \mc{1}{c}{\scriptsize{(2.010)}} & \mc{1}{c}{\scriptsize{(1.196)}} & \mc{1}{c}{\scriptsize{(1.206)}} & \mc{1}{c}{\scriptsize{(0.824)}} \\  

     & \mc{1}{c}{\scriptsize{8}} & \mc{1}{c}{\scriptsize{4.207}} & \mc{1}{c}{\scriptsize{5.146}} & \mc{1}{c}{\scriptsize{1.630}} & \mc{1}{c}{\scriptsize{5.443}} & \mc{1}{c}{\scriptsize{7.731}} & \mc{1}{c}{\scriptsize{4.959}} & \mc{1}{c}{\scriptsize{5.482}} & \mc{1}{c}{\scriptsize{7.042}} \\  

     &  & \mc{1}{c}{\scriptsize{(0.431)}} & \mc{1}{c}{\scriptsize{\textbf{(0.088)}}} & \mc{1}{c}{\scriptsize{(4.343)}} & \mc{1}{c}{\scriptsize{(3.039)}} & \mc{1}{c}{\scriptsize{(0.990)}} & \mc{1}{c}{\scriptsize{(0.422)}} & \mc{1}{c}{\scriptsize{\textbf{(0.069)}}} & \mc{1}{c}{\scriptsize{(0.824)}} \\  

     & \mc{1}{c}{\scriptsize{8.5}} & \mc{1}{c}{\scriptsize{5.938}} & \mc{1}{c}{\scriptsize{6.593}} & \mc{1}{c}{\scriptsize{5.046}} & \mc{1}{c}{\scriptsize{7.976}} & \mc{1}{c}{\scriptsize{12.622}} & \mc{1}{c}{\scriptsize{5.507}} & \mc{1}{c}{\scriptsize{5.907}} & \mc{1}{c}{\scriptsize{7.924}} \\  

     &  & \mc{1}{c}{\scriptsize{(0.157)}} & \mc{1}{c}{\scriptsize{\textbf{(0.029)}}} & \mc{1}{c}{\scriptsize{(2.961)}} & \mc{1}{c}{\scriptsize{(1.451)}} & \mc{1}{c}{\scriptsize{(0.137)}} & \mc{1}{c}{\scriptsize{(0.294)}} & \mc{1}{c}{\scriptsize{(0.108)}} & \mc{1}{c}{\scriptsize{(0.510)}} \\  

     & \mc{1}{c}{\scriptsize{15}} & \mc{1}{c}{\scriptsize{5.163}} & \mc{1}{c}{\scriptsize{3.641}} & \mc{1}{c}{\scriptsize{5.177}} & \mc{1}{c}{\scriptsize{5.378}} & \mc{1}{c}{\scriptsize{8.388}} & \mc{1}{c}{\scriptsize{5.424}} & \mc{1}{c}{\scriptsize{3.200}} & \mc{1}{c}{\scriptsize{7.429}} \\  

     &  & \mc{1}{c}{\scriptsize{(0.255)}} & \mc{1}{c}{\scriptsize{(1.490)}} & \mc{1}{c}{\scriptsize{(2.539)}} & \mc{1}{c}{\scriptsize{(3.039)}} & \mc{1}{c}{\scriptsize{\textbf{(0.078)}}} & \mc{1}{c}{\scriptsize{(0.451)}} & \mc{1}{c}{\scriptsize{(2.853)}} & \mc{1}{c}{\scriptsize{(0.824)}} \\  

     & \mc{1}{c}{\scriptsize{21}} & \mc{1}{c}{\scriptsize{5.217}} & \mc{1}{c}{\scriptsize{2.253}} & \mc{1}{c}{\scriptsize{4.504}} & \mc{1}{c}{\scriptsize{2.087}} & \mc{1}{c}{\scriptsize{6.497}} & \mc{1}{c}{\scriptsize{5.521}} & \mc{1}{c}{\scriptsize{2.027}} & \mc{1}{c}{\scriptsize{7.488}} \\  

     &  & \mc{1}{c}{\scriptsize{(0.431)}} & \mc{1}{c}{\scriptsize{(1.814)}} & \mc{1}{c}{\scriptsize{(2.961)}} & \mc{1}{c}{\scriptsize{(3.304)}} & \mc{1}{c}{\scriptsize{(0.794)}} & \mc{1}{c}{\scriptsize{(0.608)}} & \mc{1}{c}{\scriptsize{(3.392)}} & \mc{1}{c}{\scriptsize{(0.824)}} \\  

    \mc{1}{l}{\scriptsize{Achievement Factor}} & \mc{1}{c}{\scriptsize{5.5 to 12}} & \mc{1}{c}{\scriptsize{0.512}} & \mc{1}{c}{\scriptsize{0.621}} & \mc{1}{c}{\scriptsize{0.634}} & \mc{1}{c}{\scriptsize{0.900}} & \mc{1}{c}{\scriptsize{1.331}} & \mc{1}{c}{\scriptsize{0.474}} & \mc{1}{c}{\scriptsize{0.547}} & \mc{1}{c}{\scriptsize{0.847}} \\  

     &  & \mc{1}{c}{\scriptsize{(0.431)}} & \mc{1}{c}{\scriptsize{(0.255)}} & \mc{1}{c}{\scriptsize{(2.539)}} & \mc{1}{c}{\scriptsize{(1.098)}} & \mc{1}{c}{\scriptsize{(0.196)}} & \mc{1}{c}{\scriptsize{(0.608)}} & \mc{1}{c}{\scriptsize{(0.667)}} & \mc{1}{c}{\scriptsize{(0.824)}} \\  

     & \mc{1}{c}{\scriptsize{15 to 21}} & \mc{1}{c}{\scriptsize{-0.460}} & \mc{1}{c}{\scriptsize{-0.265}} & \mc{1}{c}{\scriptsize{-0.431}} & \mc{1}{c}{\scriptsize{-0.340}} & \mc{1}{c}{\scriptsize{-0.665}} & \mc{1}{c}{\scriptsize{-0.485}} & \mc{1}{c}{\scriptsize{-0.235}} & \mc{1}{c}{\scriptsize{-0.661}} \\  

     &  & \mc{1}{c}{\scriptsize{(0.255)}} & \mc{1}{c}{\scriptsize{(1.618)}} & \mc{1}{c}{\scriptsize{(2.725)}} & \mc{1}{c}{\scriptsize{(3.304)}} & \mc{1}{c}{\scriptsize{(0.196)}} & \mc{1}{c}{\scriptsize{(0.451)}} & \mc{1}{c}{\scriptsize{(3.392)}} & \mc{1}{c}{\scriptsize{(0.824)}} \\  

  \bottomrule
  \end{tabular}
	\end{table} 

	\begin{table}[H]
     \caption{Treatment Effects on Infant Behavior Record, Female Sample}
     \label{table:abccare_rslt_female_cat2_sd}
	  \begin{tabular}{cccccccccc}
  \toprule

    \scriptsize{Variable} & \scriptsize{Age} & \scriptsize{(1)} & \scriptsize{(2)} & \scriptsize{(3)} & \scriptsize{(4)} & \scriptsize{(5)} & \scriptsize{(6)} & \scriptsize{(7)} & \scriptsize{(8)} \\ 
    \midrule  

    \mc{1}{l}{\scriptsize{HOME Score}} & \mc{1}{c}{\scriptsize{0.5}} & \mc{1}{c}{\scriptsize{1.486}} & \mc{1}{c}{\scriptsize{1.038}} & \mc{1}{c}{\scriptsize{1.137}} & \mc{1}{c}{\scriptsize{1.852}} & \mc{1}{c}{\scriptsize{0.527}} & \mc{1}{c}{\scriptsize{1.149}} & \mc{1}{c}{\scriptsize{1.371}} & \mc{1}{c}{\scriptsize{0.603}} \\  

     &  & \mc{1}{c}{\scriptsize{(0.289)}} & \mc{1}{c}{\scriptsize{(0.592)}} & \mc{1}{c}{\scriptsize{(0.434)}} & \mc{1}{c}{\scriptsize{(0.487)}} & \mc{1}{c}{\scriptsize{(0.500)}} & \mc{1}{c}{\scriptsize{(0.487)}} & \mc{1}{c}{\scriptsize{(0.566)}} & \mc{1}{c}{\scriptsize{(0.684)}} \\  

     & \mc{1}{c}{\scriptsize{1.5}} & \mc{1}{c}{\scriptsize{1.765}} & \mc{1}{c}{\scriptsize{1.391}} & \mc{1}{c}{\scriptsize{2.678}} & \mc{1}{c}{\scriptsize{3.183}} & \mc{1}{c}{\scriptsize{2.606}} & \mc{1}{c}{\scriptsize{0.969}} & \mc{1}{c}{\scriptsize{1.268}} & \mc{1}{c}{\scriptsize{1.122}} \\  

     &  & \mc{1}{c}{\scriptsize{(0.303)}} & \mc{1}{c}{\scriptsize{(0.487)}} & \mc{1}{c}{\scriptsize{(0.145)}} & \mc{1}{c}{\scriptsize{(0.342)}} & \mc{1}{c}{\scriptsize{(0.145)}} & \mc{1}{c}{\scriptsize{(0.592)}} & \mc{1}{c}{\scriptsize{(0.579)}} & \mc{1}{c}{\scriptsize{(0.539)}} \\  

     & \mc{1}{c}{\scriptsize{2.5}} & \mc{1}{c}{\scriptsize{0.333}} & \mc{1}{c}{\scriptsize{0.680}} & \mc{1}{c}{\scriptsize{3.315}} & \mc{1}{c}{\scriptsize{4.352}} & \mc{1}{c}{\scriptsize{3.518}} & \mc{1}{c}{\scriptsize{-1.212}} & \mc{1}{c}{\scriptsize{-0.416}} & \mc{1}{c}{\scriptsize{-0.654}} \\  

     &  & \mc{1}{c}{\scriptsize{(0.763)}} & \mc{1}{c}{\scriptsize{(0.671)}} & \mc{1}{c}{\scriptsize{\textbf{(0.079)}}} & \mc{1}{c}{\scriptsize{(0.171)}} & \mc{1}{c}{\scriptsize{\textbf{(0.092)}}} & \mc{1}{c}{\scriptsize{(1.000)}} & \mc{1}{c}{\scriptsize{(0.934)}} & \mc{1}{c}{\scriptsize{(0.974)}} \\  

     & \mc{1}{c}{\scriptsize{3.5}} & \mc{1}{c}{\scriptsize{1.512}} & \mc{1}{c}{\scriptsize{2.062}} & \mc{1}{c}{\scriptsize{7.969}} & \mc{1}{c}{\scriptsize{13.240}} & \mc{1}{c}{\scriptsize{7.683}} & \mc{1}{c}{\scriptsize{-0.834}} & \mc{1}{c}{\scriptsize{0.087}} & \mc{1}{c}{\scriptsize{-0.693}} \\  

     &  & \mc{1}{c}{\scriptsize{(0.566)}} & \mc{1}{c}{\scriptsize{(0.579)}} & \mc{1}{c}{\scriptsize{\textbf{(0.092)}}} & \mc{1}{c}{\scriptsize{(0.105)}} & \mc{1}{c}{\scriptsize{\textbf{(0.053)}}} & \mc{1}{c}{\scriptsize{(0.961)}} & \mc{1}{c}{\scriptsize{(0.829)}} & \mc{1}{c}{\scriptsize{(0.947)}} \\  

     & \mc{1}{c}{\scriptsize{4.5}} & \mc{1}{c}{\scriptsize{1.593}} & \mc{1}{c}{\scriptsize{1.857}} & \mc{1}{c}{\scriptsize{8.624}} & \mc{1}{c}{\scriptsize{16.226}} & \mc{1}{c}{\scriptsize{8.915}} & \mc{1}{c}{\scriptsize{-0.954}} & \mc{1}{c}{\scriptsize{0.666}} & \mc{1}{c}{\scriptsize{-0.301}} \\  

     &  & \mc{1}{c}{\scriptsize{(0.553)}} & \mc{1}{c}{\scriptsize{(0.539)}} & \mc{1}{c}{\scriptsize{\textbf{(0.079)}}} & \mc{1}{c}{\scriptsize{\textbf{(0.039)}}} & \mc{1}{c}{\scriptsize{\textbf{(0.092)}}} & \mc{1}{c}{\scriptsize{(0.974)}} & \mc{1}{c}{\scriptsize{(0.763)}} & \mc{1}{c}{\scriptsize{(0.895)}} \\  

     & \mc{1}{c}{\scriptsize{8}} & \mc{1}{c}{\scriptsize{0.628}} & \mc{1}{c}{\scriptsize{1.186}} & \mc{1}{c}{\scriptsize{6.266}} & \mc{1}{c}{\scriptsize{9.503}} & \mc{1}{c}{\scriptsize{6.827}} & \mc{1}{c}{\scriptsize{-0.951}} & \mc{1}{c}{\scriptsize{-0.008}} & \mc{1}{c}{\scriptsize{0.092}} \\  

     &  & \mc{1}{c}{\scriptsize{(0.737)}} & \mc{1}{c}{\scriptsize{(0.671)}} & \mc{1}{c}{\scriptsize{\textbf{(0.053)}}} & \mc{1}{c}{\scriptsize{(0.118)}} & \mc{1}{c}{\scriptsize{\textbf{(0.053)}}} & \mc{1}{c}{\scriptsize{(0.974)}} & \mc{1}{c}{\scriptsize{(0.829)}} & \mc{1}{c}{\scriptsize{(0.842)}} \\  

    \mc{1}{l}{\scriptsize{HOME Factor}} & \mc{1}{c}{\scriptsize{0.5 to 8}} & \mc{1}{c}{\scriptsize{0.227}} & \mc{1}{c}{\scriptsize{0.328}} & \mc{1}{c}{\scriptsize{1.164}} & \mc{1}{c}{\scriptsize{1.513}} & \mc{1}{c}{\scriptsize{1.220}} & \mc{1}{c}{\scriptsize{-0.058}} & \mc{1}{c}{\scriptsize{0.206}} & \mc{1}{c}{\scriptsize{0.070}} \\  

     &  & \mc{1}{c}{\scriptsize{(0.539)}} & \mc{1}{c}{\scriptsize{(0.395)}} & \mc{1}{c}{\scriptsize{\textbf{(0.000)}}} & \mc{1}{c}{\scriptsize{(0.158)}} & \mc{1}{c}{\scriptsize{\textbf{(0.000)}}} & \mc{1}{c}{\scriptsize{(0.908)}} & \mc{1}{c}{\scriptsize{(0.618)}} & \mc{1}{c}{\scriptsize{(0.829)}} \\  

  \bottomrule
  \end{tabular}
	\end{table} 

	\begin{table}[H]
     \caption{Treatment Effects on Kohn and Rosman: Attentive/Cooperative, Female Sample}
     \label{table:abccare_rslt_female_cat3_sd}
	  \begin{tabular}{cccccccccc}
  \toprule

    \scriptsize{Variable} & \scriptsize{Age} & \scriptsize{(1)} & \scriptsize{(2)} & \scriptsize{(3)} & \scriptsize{(4)} & \scriptsize{(5)} & \scriptsize{(6)} & \scriptsize{(7)} & \scriptsize{(8)} \\ 
    \midrule  

    \mc{1}{l}{\scriptsize{Attentive/Cooperative}} & \mc{1}{c}{\scriptsize{2}} & \mc{1}{c}{\scriptsize{0.815}} & \mc{1}{c}{\scriptsize{1.011}} & \mc{1}{c}{\scriptsize{1.671}} & \mc{1}{c}{\scriptsize{1.851}} & \mc{1}{c}{\scriptsize{1.713}} & \mc{1}{c}{\scriptsize{0.644}} & \mc{1}{c}{\scriptsize{0.512}} & \mc{1}{c}{\scriptsize{0.844}} \\  

     &  & \mc{1}{c}{\scriptsize{\textbf{(0.000)}}} & \mc{1}{c}{\scriptsize{\textbf{(0.013)}}} & \mc{1}{c}{\scriptsize{\textbf{(0.000)}}} & \mc{1}{c}{\scriptsize{(0.118)}} & \mc{1}{c}{\scriptsize{\textbf{(0.000)}}} & \mc{1}{c}{\scriptsize{\textbf{(0.013)}}} & \mc{1}{c}{\scriptsize{(0.263)}} & \mc{1}{c}{\scriptsize{\textbf{(0.000)}}} \\  

     & \mc{1}{c}{\scriptsize{12}} & \mc{1}{c}{\scriptsize{0.361}} & \mc{1}{c}{\scriptsize{0.184}} & \mc{1}{c}{\scriptsize{0.049}} & \mc{1}{c}{\scriptsize{-0.309}} & \mc{1}{c}{\scriptsize{0.071}} & \mc{1}{c}{\scriptsize{0.383}} & \mc{1}{c}{\scriptsize{0.297}} & \mc{1}{c}{\scriptsize{0.409}} \\  

     &  & \mc{1}{c}{\scriptsize{(0.171)}} & \mc{1}{c}{\scriptsize{(0.474)}} & \mc{1}{c}{\scriptsize{(0.684)}} & \mc{1}{c}{\scriptsize{(0.934)}} & \mc{1}{c}{\scriptsize{(0.632)}} & \mc{1}{c}{\scriptsize{(0.211)}} & \mc{1}{c}{\scriptsize{(0.368)}} & \mc{1}{c}{\scriptsize{(0.211)}} \\  

  \bottomrule
  \end{tabular}
	\end{table} 

	\begin{table}[H]
     \caption{Treatment Effects on Classroom Behavior Inventory (Part I), Female Sample}
     \label{table:abccare_rslt_female_cat4_sd}
	  \begin{tabular}{cccccccccc}
  \toprule

    \scriptsize{Variable} & \scriptsize{Age} & \scriptsize{(1)} & \scriptsize{(2)} & \scriptsize{(3)} & \scriptsize{(4)} & \scriptsize{(5)} & \scriptsize{(6)} & \scriptsize{(7)} & \scriptsize{(8)} \\ 
    \midrule  

    \mc{1}{l}{\scriptsize{Mother Works}} & \mc{1}{c}{\scriptsize{2}} & \mc{1}{c}{\scriptsize{0.129}} & \mc{1}{c}{\scriptsize{0.092}} & \mc{1}{c}{\scriptsize{0.186}} & \mc{1}{c}{\scriptsize{0.248}} & \mc{1}{c}{\scriptsize{0.173}} & \mc{1}{c}{\scriptsize{0.091}} & \mc{1}{c}{\scriptsize{0.087}} & \mc{1}{c}{\scriptsize{0.084}} \\  

     &  & \mc{1}{c}{\scriptsize{(0.171)}} & \mc{1}{c}{\scriptsize{(0.408)}} & \mc{1}{c}{\scriptsize{(0.250)}} & \mc{1}{c}{\scriptsize{(0.289)}} & \mc{1}{c}{\scriptsize{(0.303)}} & \mc{1}{c}{\scriptsize{(0.395)}} & \mc{1}{c}{\scriptsize{(0.421)}} & \mc{1}{c}{\scriptsize{(0.408)}} \\  

     & \mc{1}{c}{\scriptsize{3}} & \mc{1}{c}{\scriptsize{0.099}} & \mc{1}{c}{\scriptsize{0.054}} & \mc{1}{c}{\scriptsize{0.305}} & \mc{1}{c}{\scriptsize{0.136}} & \mc{1}{c}{\scriptsize{0.289}} & \mc{1}{c}{\scriptsize{0.026}} & \mc{1}{c}{\scriptsize{0.000}} & \mc{1}{c}{\scriptsize{0.014}} \\  

     &  & \mc{1}{c}{\scriptsize{(0.342)}} & \mc{1}{c}{\scriptsize{(0.539)}} & \mc{1}{c}{\scriptsize{\textbf{(0.079)}}} & \mc{1}{c}{\scriptsize{(0.526)}} & \mc{1}{c}{\scriptsize{(0.132)}} & \mc{1}{c}{\scriptsize{(0.697)}} & \mc{1}{c}{\scriptsize{(0.724)}} & \mc{1}{c}{\scriptsize{(0.750)}} \\  

     & \mc{1}{c}{\scriptsize{4}} & \mc{1}{c}{\scriptsize{0.109}} & \mc{1}{c}{\scriptsize{0.099}} & \mc{1}{c}{\scriptsize{0.337}} & \mc{1}{c}{\scriptsize{0.288}} & \mc{1}{c}{\scriptsize{0.323}} & \mc{1}{c}{\scriptsize{0.026}} & \mc{1}{c}{\scriptsize{0.071}} & \mc{1}{c}{\scriptsize{0.018}} \\  

     &  & \mc{1}{c}{\scriptsize{(0.303)}} & \mc{1}{c}{\scriptsize{(0.408)}} & \mc{1}{c}{\scriptsize{\textbf{(0.053)}}} & \mc{1}{c}{\scriptsize{(0.211)}} & \mc{1}{c}{\scriptsize{\textbf{(0.066)}}} & \mc{1}{c}{\scriptsize{(0.697)}} & \mc{1}{c}{\scriptsize{(0.461)}} & \mc{1}{c}{\scriptsize{(0.737)}} \\  

     & \mc{1}{c}{\scriptsize{5}} & \mc{1}{c}{\scriptsize{0.022}} & \mc{1}{c}{\scriptsize{0.032}} & \mc{1}{c}{\scriptsize{0.200}} & \mc{1}{c}{\scriptsize{0.319}} & \mc{1}{c}{\scriptsize{0.230}} & \mc{1}{c}{\scriptsize{-0.071}} & \mc{1}{c}{\scriptsize{-0.055}} & \mc{1}{c}{\scriptsize{-0.039}} \\  

     &  & \mc{1}{c}{\scriptsize{(0.697)}} & \mc{1}{c}{\scriptsize{(0.553)}} & \mc{1}{c}{\scriptsize{(0.250)}} & \mc{1}{c}{\scriptsize{(0.158)}} & \mc{1}{c}{\scriptsize{(0.250)}} & \mc{1}{c}{\scriptsize{(0.947)}} & \mc{1}{c}{\scriptsize{(0.921)}} & \mc{1}{c}{\scriptsize{(0.908)}} \\  

    \mc{1}{l}{\scriptsize{Mother Works Factor}} & \mc{1}{c}{\scriptsize{2 to 21}} & \mc{1}{c}{\scriptsize{0.173}} & \mc{1}{c}{\scriptsize{-0.055}} & \mc{1}{c}{\scriptsize{0.599}} & \mc{1}{c}{\scriptsize{0.293}} & \mc{1}{c}{\scriptsize{0.590}} & \mc{1}{c}{\scriptsize{0.025}} & \mc{1}{c}{\scriptsize{-0.047}} & \mc{1}{c}{\scriptsize{0.030}} \\  

     &  & \mc{1}{c}{\scriptsize{(0.434)}} & \mc{1}{c}{\scriptsize{(0.829)}} & \mc{1}{c}{\scriptsize{(0.132)}} & \mc{1}{c}{\scriptsize{(0.632)}} & \mc{1}{c}{\scriptsize{(0.158)}} & \mc{1}{c}{\scriptsize{(0.724)}} & \mc{1}{c}{\scriptsize{(0.803)}} & \mc{1}{c}{\scriptsize{(0.750)}} \\  

  \bottomrule
  \end{tabular}
	\end{table} 

	\begin{table}[H]
     \caption{Treatment Effects on Classroom Behavior Inventory (Part II), Female Sample}
     \label{table:abccare_rslt_female_cat5_sd}
	  \begin{tabular}{cccccccccc}
  \toprule

    \scriptsize{Variable} & \scriptsize{Age} & \scriptsize{(1)} & \scriptsize{(2)} & \scriptsize{(3)} & \scriptsize{(4)} & \scriptsize{(5)} & \scriptsize{(6)} & \scriptsize{(7)} & \scriptsize{(8)} \\ 
    \midrule  

    \mc{1}{l}{\scriptsize{Dependence}} & \mc{1}{c}{\scriptsize{6}} & \mc{1}{c}{\scriptsize{-0.326}} & \mc{1}{c}{\scriptsize{-0.382}} & \mc{1}{c}{\scriptsize{0.577}} & \mc{1}{c}{\scriptsize{-0.426}} & \mc{1}{c}{\scriptsize{0.640}} & \mc{1}{c}{\scriptsize{-0.590}} & \mc{1}{c}{\scriptsize{-0.492}} & \mc{1}{c}{\scriptsize{-0.709}} \\  

     &  & \mc{1}{c}{\scriptsize{(0.974)}} & \mc{1}{c}{\scriptsize{(0.947)}} & \mc{1}{c}{\scriptsize{(1.000)}} & \mc{1}{c}{\scriptsize{(0.947)}} & \mc{1}{c}{\scriptsize{(1.000)}} & \mc{1}{c}{\scriptsize{(0.947)}} & \mc{1}{c}{\scriptsize{(0.921)}} & \mc{1}{c}{\scriptsize{(0.829)}} \\  

     & \mc{1}{c}{\scriptsize{7}} & \mc{1}{c}{\scriptsize{-0.617}} & \mc{1}{c}{\scriptsize{-1.389}} & \mc{1}{c}{\scriptsize{-1.291}} & \mc{1}{c}{\scriptsize{-0.335}} & \mc{1}{c}{\scriptsize{-1.286}} & \mc{1}{c}{\scriptsize{-0.402}} & \mc{1}{c}{\scriptsize{-1.381}} & \mc{1}{c}{\scriptsize{-0.512}} \\  

     &  & \mc{1}{c}{\scriptsize{(0.895)}} & \mc{1}{c}{\scriptsize{(0.579)}} & \mc{1}{c}{\scriptsize{(0.592)}} & \mc{1}{c}{\scriptsize{(0.987)}} & \mc{1}{c}{\scriptsize{(0.671)}} & \mc{1}{c}{\scriptsize{(0.987)}} & \mc{1}{c}{\scriptsize{(0.724)}} & \mc{1}{c}{\scriptsize{(0.934)}} \\  

     & \mc{1}{c}{\scriptsize{8}} & \mc{1}{c}{\scriptsize{-1.259}} & \mc{1}{c}{\scriptsize{-1.615}} & \mc{1}{c}{\scriptsize{-1.050}} & \mc{1}{c}{\scriptsize{-1.462}} & \mc{1}{c}{\scriptsize{-1.180}} & \mc{1}{c}{\scriptsize{-1.323}} & \mc{1}{c}{\scriptsize{-1.680}} & \mc{1}{c}{\scriptsize{-1.547}} \\  

     &  & \mc{1}{c}{\scriptsize{(0.553)}} & \mc{1}{c}{\scriptsize{(0.382)}} & \mc{1}{c}{\scriptsize{(0.816)}} & \mc{1}{c}{\scriptsize{(0.750)}} & \mc{1}{c}{\scriptsize{(0.763)}} & \mc{1}{c}{\scriptsize{(0.566)}} & \mc{1}{c}{\scriptsize{(0.474)}} & \mc{1}{c}{\scriptsize{(0.461)}} \\  

     & \mc{1}{c}{\scriptsize{12}} & \mc{1}{c}{\scriptsize{0.444}} & \mc{1}{c}{\scriptsize{0.321}} & \mc{1}{c}{\scriptsize{-2.583}} & \mc{1}{c}{\scriptsize{-2.528}} & \mc{1}{c}{\scriptsize{-2.996}} & \mc{1}{c}{\scriptsize{1.397}} & \mc{1}{c}{\scriptsize{1.631}} & \mc{1}{c}{\scriptsize{1.649}} \\  

     &  & \mc{1}{c}{\scriptsize{(1.000)}} & \mc{1}{c}{\scriptsize{(1.000)}} & \mc{1}{c}{\scriptsize{(0.684)}} & \mc{1}{c}{\scriptsize{(0.816)}} & \mc{1}{c}{\scriptsize{(0.553)}} & \mc{1}{c}{\scriptsize{(1.000)}} & \mc{1}{c}{\scriptsize{(1.000)}} & \mc{1}{c}{\scriptsize{(1.000)}} \\  

    \mc{1}{l}{\scriptsize{Distractibility}} & \mc{1}{c}{\scriptsize{6}} & \mc{1}{c}{\scriptsize{-0.453}} & \mc{1}{c}{\scriptsize{0.210}} & \mc{1}{c}{\scriptsize{0.335}} & \mc{1}{c}{\scriptsize{1.049}} & \mc{1}{c}{\scriptsize{0.384}} & \mc{1}{c}{\scriptsize{-0.683}} & \mc{1}{c}{\scriptsize{-0.039}} & \mc{1}{c}{\scriptsize{-0.889}} \\  

     &  & \mc{1}{c}{\scriptsize{(0.974)}} & \mc{1}{c}{\scriptsize{(1.000)}} & \mc{1}{c}{\scriptsize{(1.000)}} & \mc{1}{c}{\scriptsize{(1.000)}} & \mc{1}{c}{\scriptsize{(1.000)}} & \mc{1}{c}{\scriptsize{(0.934)}} & \mc{1}{c}{\scriptsize{(1.000)}} & \mc{1}{c}{\scriptsize{(0.737)}} \\  

     & \mc{1}{c}{\scriptsize{7}} & \mc{1}{c}{\scriptsize{-0.611}} & \mc{1}{c}{\scriptsize{-0.313}} & \mc{1}{c}{\scriptsize{-1.251}} & \mc{1}{c}{\scriptsize{-1.075}} & \mc{1}{c}{\scriptsize{-0.952}} & \mc{1}{c}{\scriptsize{-0.407}} & \mc{1}{c}{\scriptsize{0.015}} & \mc{1}{c}{\scriptsize{-0.069}} \\  

     &  & \mc{1}{c}{\scriptsize{(0.855)}} & \mc{1}{c}{\scriptsize{(1.000)}} & \mc{1}{c}{\scriptsize{(0.684)}} & \mc{1}{c}{\scriptsize{(0.908)}} & \mc{1}{c}{\scriptsize{(0.868)}} & \mc{1}{c}{\scriptsize{(0.961)}} & \mc{1}{c}{\scriptsize{(1.000)}} & \mc{1}{c}{\scriptsize{(1.000)}} \\  

     & \mc{1}{c}{\scriptsize{8}} & \mc{1}{c}{\scriptsize{0.059}} & \mc{1}{c}{\scriptsize{0.117}} & \mc{1}{c}{\scriptsize{-1.093}} & \mc{1}{c}{\scriptsize{-1.490}} & \mc{1}{c}{\scriptsize{-1.091}} & \mc{1}{c}{\scriptsize{0.410}} & \mc{1}{c}{\scriptsize{0.634}} & \mc{1}{c}{\scriptsize{0.336}} \\  

     &  & \mc{1}{c}{\scriptsize{(1.000)}} & \mc{1}{c}{\scriptsize{(1.000)}} & \mc{1}{c}{\scriptsize{(0.684)}} & \mc{1}{c}{\scriptsize{(0.711)}} & \mc{1}{c}{\scriptsize{(0.684)}} & \mc{1}{c}{\scriptsize{(1.000)}} & \mc{1}{c}{\scriptsize{(1.000)}} & \mc{1}{c}{\scriptsize{(1.000)}} \\  

     & \mc{1}{c}{\scriptsize{12}} & \mc{1}{c}{\scriptsize{0.333}} & \mc{1}{c}{\scriptsize{0.549}} & \mc{1}{c}{\scriptsize{-2.583}} & \mc{1}{c}{\scriptsize{-4.153}} & \mc{1}{c}{\scriptsize{-3.576}} & \mc{1}{c}{\scriptsize{1.321}} & \mc{1}{c}{\scriptsize{1.565}} & \mc{1}{c}{\scriptsize{0.628}} \\  

     &  & \mc{1}{c}{\scriptsize{(1.000)}} & \mc{1}{c}{\scriptsize{(1.000)}} & \mc{1}{c}{\scriptsize{(0.553)}} & \mc{1}{c}{\scriptsize{(0.724)}} & \mc{1}{c}{\scriptsize{(0.316)}} & \mc{1}{c}{\scriptsize{(1.000)}} & \mc{1}{c}{\scriptsize{(1.000)}} & \mc{1}{c}{\scriptsize{(1.000)}} \\  

    \mc{1}{l}{\scriptsize{Hostility}} & \mc{1}{c}{\scriptsize{6}} & \mc{1}{c}{\scriptsize{1.194}} & \mc{1}{c}{\scriptsize{1.227}} & \mc{1}{c}{\scriptsize{2.857}} & \mc{1}{c}{\scriptsize{2.152}} & \mc{1}{c}{\scriptsize{2.731}} & \mc{1}{c}{\scriptsize{0.708}} & \mc{1}{c}{\scriptsize{0.907}} & \mc{1}{c}{\scriptsize{0.358}} \\  

     &  & \mc{1}{c}{\scriptsize{(1.000)}} & \mc{1}{c}{\scriptsize{(1.000)}} & \mc{1}{c}{\scriptsize{(1.000)}} & \mc{1}{c}{\scriptsize{(1.000)}} & \mc{1}{c}{\scriptsize{(1.000)}} & \mc{1}{c}{\scriptsize{(1.000)}} & \mc{1}{c}{\scriptsize{(1.000)}} & \mc{1}{c}{\scriptsize{(1.000)}} \\  

     & \mc{1}{c}{\scriptsize{7}} & \mc{1}{c}{\scriptsize{0.959}} & \mc{1}{c}{\scriptsize{0.301}} & \mc{1}{c}{\scriptsize{2.343}} & \mc{1}{c}{\scriptsize{1.747}} & \mc{1}{c}{\scriptsize{1.736}} & \mc{1}{c}{\scriptsize{0.518}} & \mc{1}{c}{\scriptsize{0.020}} & \mc{1}{c}{\scriptsize{0.349}} \\  

     &  & \mc{1}{c}{\scriptsize{(1.000)}} & \mc{1}{c}{\scriptsize{(1.000)}} & \mc{1}{c}{\scriptsize{(1.000)}} & \mc{1}{c}{\scriptsize{(1.000)}} & \mc{1}{c}{\scriptsize{(1.000)}} & \mc{1}{c}{\scriptsize{(1.000)}} & \mc{1}{c}{\scriptsize{(1.000)}} & \mc{1}{c}{\scriptsize{(1.000)}} \\  

     & \mc{1}{c}{\scriptsize{8}} & \mc{1}{c}{\scriptsize{0.203}} & \mc{1}{c}{\scriptsize{-0.013}} & \mc{1}{c}{\scriptsize{-0.231}} & \mc{1}{c}{\scriptsize{-1.225}} & \mc{1}{c}{\scriptsize{-0.330}} & \mc{1}{c}{\scriptsize{0.334}} & \mc{1}{c}{\scriptsize{0.437}} & \mc{1}{c}{\scriptsize{0.138}} \\  

     &  & \mc{1}{c}{\scriptsize{(1.000)}} & \mc{1}{c}{\scriptsize{(1.000)}} & \mc{1}{c}{\scriptsize{(0.987)}} & \mc{1}{c}{\scriptsize{(0.829)}} & \mc{1}{c}{\scriptsize{(0.974)}} & \mc{1}{c}{\scriptsize{(1.000)}} & \mc{1}{c}{\scriptsize{(1.000)}} & \mc{1}{c}{\scriptsize{(1.000)}} \\  

     & \mc{1}{c}{\scriptsize{12}} & \mc{1}{c}{\scriptsize{0.222}} & \mc{1}{c}{\scriptsize{0.655}} & \mc{1}{c}{\scriptsize{0.278}} & \mc{1}{c}{\scriptsize{0.724}} & \mc{1}{c}{\scriptsize{-0.134}} & \mc{1}{c}{\scriptsize{0.009}} & \mc{1}{c}{\scriptsize{0.665}} & \mc{1}{c}{\scriptsize{-0.554}} \\  

     &  & \mc{1}{c}{\scriptsize{(1.000)}} & \mc{1}{c}{\scriptsize{(1.000)}} & \mc{1}{c}{\scriptsize{(1.000)}} & \mc{1}{c}{\scriptsize{(1.000)}} & \mc{1}{c}{\scriptsize{(1.000)}} & \mc{1}{c}{\scriptsize{(1.000)}} & \mc{1}{c}{\scriptsize{(1.000)}} & \mc{1}{c}{\scriptsize{(0.934)}} \\  

    \mc{1}{l}{\scriptsize{Introversion}} & \mc{1}{c}{\scriptsize{6}} & \mc{1}{c}{\scriptsize{-0.602}} & \mc{1}{c}{\scriptsize{-0.504}} & \mc{1}{c}{\scriptsize{-1.016}} & \mc{1}{c}{\scriptsize{-2.140}} & \mc{1}{c}{\scriptsize{-1.164}} & \mc{1}{c}{\scriptsize{-0.481}} & \mc{1}{c}{\scriptsize{0.049}} & \mc{1}{c}{\scriptsize{-0.701}} \\  

     &  & \mc{1}{c}{\scriptsize{(0.895)}} & \mc{1}{c}{\scriptsize{(0.921)}} & \mc{1}{c}{\scriptsize{(0.737)}} & \mc{1}{c}{\scriptsize{(0.579)}} & \mc{1}{c}{\scriptsize{(0.684)}} & \mc{1}{c}{\scriptsize{(0.947)}} & \mc{1}{c}{\scriptsize{(1.000)}} & \mc{1}{c}{\scriptsize{(0.868)}} \\  

     & \mc{1}{c}{\scriptsize{7}} & \mc{1}{c}{\scriptsize{-0.359}} & \mc{1}{c}{\scriptsize{-0.703}} & \mc{1}{c}{\scriptsize{-1.537}} & \mc{1}{c}{\scriptsize{-0.796}} & \mc{1}{c}{\scriptsize{-1.773}} & \mc{1}{c}{\scriptsize{0.034}} & \mc{1}{c}{\scriptsize{-0.413}} & \mc{1}{c}{\scriptsize{-0.273}} \\  

     &  & \mc{1}{c}{\scriptsize{(0.974)}} & \mc{1}{c}{\scriptsize{(0.829)}} & \mc{1}{c}{\scriptsize{(0.500)}} & \mc{1}{c}{\scriptsize{(0.882)}} & \mc{1}{c}{\scriptsize{(0.395)}} & \mc{1}{c}{\scriptsize{(1.000)}} & \mc{1}{c}{\scriptsize{(0.921)}} & \mc{1}{c}{\scriptsize{(0.947)}} \\  

     & \mc{1}{c}{\scriptsize{8}} & \mc{1}{c}{\scriptsize{-0.703}} & \mc{1}{c}{\scriptsize{-0.736}} & \mc{1}{c}{\scriptsize{-0.055}} & \mc{1}{c}{\scriptsize{-0.214}} & \mc{1}{c}{\scriptsize{-0.329}} & \mc{1}{c}{\scriptsize{-0.900}} & \mc{1}{c}{\scriptsize{-0.989}} & \mc{1}{c}{\scriptsize{-1.132}} \\  

     &  & \mc{1}{c}{\scriptsize{(0.895)}} & \mc{1}{c}{\scriptsize{(0.763)}} & \mc{1}{c}{\scriptsize{(1.000)}} & \mc{1}{c}{\scriptsize{(0.987)}} & \mc{1}{c}{\scriptsize{(0.947)}} & \mc{1}{c}{\scriptsize{(0.763)}} & \mc{1}{c}{\scriptsize{(0.671)}} & \mc{1}{c}{\scriptsize{(0.632)}} \\  

     & \mc{1}{c}{\scriptsize{12}} & \mc{1}{c}{\scriptsize{0.778}} & \mc{1}{c}{\scriptsize{0.615}} & \mc{1}{c}{\scriptsize{0.056}} & \mc{1}{c}{\scriptsize{-0.825}} & \mc{1}{c}{\scriptsize{-0.504}} & \mc{1}{c}{\scriptsize{1.171}} & \mc{1}{c}{\scriptsize{0.870}} & \mc{1}{c}{\scriptsize{0.559}} \\  

     &  & \mc{1}{c}{\scriptsize{(1.000)}} & \mc{1}{c}{\scriptsize{(1.000)}} & \mc{1}{c}{\scriptsize{(1.000)}} & \mc{1}{c}{\scriptsize{(0.987)}} & \mc{1}{c}{\scriptsize{(0.947)}} & \mc{1}{c}{\scriptsize{(1.000)}} & \mc{1}{c}{\scriptsize{(1.000)}} & \mc{1}{c}{\scriptsize{(1.000)}} \\  

  \bottomrule
  \end{tabular}
	\end{table} 

	\begin{table}[H]
     \caption{Treatment Effects on Emotional, Activity, Sociability, Impulsivity Survey, Female Sample}
     \label{table:abccare_rslt_female_cat6_sd}
	  \begin{tabular}{cccccccccc}
  \toprule

    \scriptsize{Variable} & \scriptsize{Age} & \scriptsize{(1)} & \scriptsize{(2)} & \scriptsize{(3)} & \scriptsize{(4)} & \scriptsize{(5)} & \scriptsize{(6)} & \scriptsize{(7)} & \scriptsize{(8)} \\ 
    \midrule  

    \mc{1}{l}{\scriptsize{Graduated High School}} & \mc{1}{c}{\scriptsize{30}} & \mc{1}{c}{\scriptsize{0.213}} & \mc{1}{c}{\scriptsize{0.071}} & \mc{1}{c}{\scriptsize{0.433}} & \mc{1}{c}{\scriptsize{0.459}} & \mc{1}{c}{\scriptsize{0.386}} & \mc{1}{c}{\scriptsize{0.130}} & \mc{1}{c}{\scriptsize{-0.031}} & \mc{1}{c}{\scriptsize{0.064}} \\  

     &  & \mc{1}{c}{\scriptsize{\textbf{(0.079)}}} & \mc{1}{c}{\scriptsize{(0.711)}} & \mc{1}{c}{\scriptsize{\textbf{(0.000)}}} & \mc{1}{c}{\scriptsize{\textbf{(0.039)}}} & \mc{1}{c}{\scriptsize{\textbf{(0.026)}}} & \mc{1}{c}{\scriptsize{(0.368)}} & \mc{1}{c}{\scriptsize{(0.947)}} & \mc{1}{c}{\scriptsize{(0.697)}} \\  

    \mc{1}{l}{\scriptsize{Attended Voc./Tech./Com. College}} & \mc{1}{c}{\scriptsize{30}} & \mc{1}{c}{\scriptsize{-0.045}} & \mc{1}{c}{\scriptsize{-0.100}} & \mc{1}{c}{\scriptsize{0.117}} & \mc{1}{c}{\scriptsize{0.009}} & \mc{1}{c}{\scriptsize{0.097}} & \mc{1}{c}{\scriptsize{-0.088}} & \mc{1}{c}{\scriptsize{-0.140}} & \mc{1}{c}{\scriptsize{-0.101}} \\  

     &  & \mc{1}{c}{\scriptsize{(0.921)}} & \mc{1}{c}{\scriptsize{(1.000)}} & \mc{1}{c}{\scriptsize{(0.645)}} & \mc{1}{c}{\scriptsize{(0.829)}} & \mc{1}{c}{\scriptsize{(0.658)}} & \mc{1}{c}{\scriptsize{(1.000)}} & \mc{1}{c}{\scriptsize{(1.000)}} & \mc{1}{c}{\scriptsize{(0.987)}} \\  

    \mc{1}{l}{\scriptsize{Graduated 4-year College}} & \mc{1}{c}{\scriptsize{30}} & \mc{1}{c}{\scriptsize{0.081}} & \mc{1}{c}{\scriptsize{0.097}} & \mc{1}{c}{\scriptsize{0.036}} & \mc{1}{c}{\scriptsize{0.096}} & \mc{1}{c}{\scriptsize{0.027}} & \mc{1}{c}{\scriptsize{0.092}} & \mc{1}{c}{\scriptsize{0.114}} & \mc{1}{c}{\scriptsize{0.081}} \\  

     &  & \mc{1}{c}{\scriptsize{(0.487)}} & \mc{1}{c}{\scriptsize{(0.474)}} & \mc{1}{c}{\scriptsize{(0.763)}} & \mc{1}{c}{\scriptsize{(0.421)}} & \mc{1}{c}{\scriptsize{(0.789)}} & \mc{1}{c}{\scriptsize{(0.447)}} & \mc{1}{c}{\scriptsize{(0.355)}} & \mc{1}{c}{\scriptsize{(0.566)}} \\  

    \mc{1}{l}{\scriptsize{Years of Edu.}} & \mc{1}{c}{\scriptsize{30}} & \mc{1}{c}{\scriptsize{1.474}} & \mc{1}{c}{\scriptsize{1.266}} & \mc{1}{c}{\scriptsize{1.983}} & \mc{1}{c}{\scriptsize{2.894}} & \mc{1}{c}{\scriptsize{1.871}} & \mc{1}{c}{\scriptsize{1.233}} & \mc{1}{c}{\scriptsize{1.074}} & \mc{1}{c}{\scriptsize{1.077}} \\  

     &  & \mc{1}{c}{\scriptsize{\textbf{(0.053)}}} & \mc{1}{c}{\scriptsize{(0.184)}} & \mc{1}{c}{\scriptsize{(0.237)}} & \mc{1}{c}{\scriptsize{\textbf{(0.039)}}} & \mc{1}{c}{\scriptsize{(0.276)}} & \mc{1}{c}{\scriptsize{(0.105)}} & \mc{1}{c}{\scriptsize{(0.276)}} & \mc{1}{c}{\scriptsize{(0.145)}} \\  

    \mc{1}{l}{\scriptsize{Education Factor}} & \mc{1}{c}{\scriptsize{30}} & \mc{1}{c}{\scriptsize{0.463}} & \mc{1}{c}{\scriptsize{0.380}} & \mc{1}{c}{\scriptsize{0.691}} & \mc{1}{c}{\scriptsize{0.983}} & \mc{1}{c}{\scriptsize{0.625}} & \mc{1}{c}{\scriptsize{0.352}} & \mc{1}{c}{\scriptsize{0.289}} & \mc{1}{c}{\scriptsize{0.269}} \\  

     &  & \mc{1}{c}{\scriptsize{\textbf{(0.092)}}} & \mc{1}{c}{\scriptsize{(0.303)}} & \mc{1}{c}{\scriptsize{(0.316)}} & \mc{1}{c}{\scriptsize{\textbf{(0.053)}}} & \mc{1}{c}{\scriptsize{(0.395)}} & \mc{1}{c}{\scriptsize{(0.237)}} & \mc{1}{c}{\scriptsize{(0.513)}} & \mc{1}{c}{\scriptsize{(0.447)}} \\  

  \bottomrule
  \end{tabular}
	\end{table} 

	\begin{table}[H]
     \caption{Treatment Effects on Harter Importance, Female Sample}
     \label{table:abccare_rslt_female_cat7_sd}
	  \begin{tabular}{cccccccccc}
  \toprule

    \scriptsize{Variable} & \scriptsize{Age} & \scriptsize{(1)} & \scriptsize{(2)} & \scriptsize{(3)} & \scriptsize{(4)} & \scriptsize{(5)} & \scriptsize{(6)} & \scriptsize{(7)} & \scriptsize{(8)} \\ 
    \midrule  

    \mc{1}{l}{\scriptsize{Mean Cell Volum}} & \mc{1}{c}{\scriptsize{Mid-30s}} & \mc{1}{c}{\scriptsize{-1.403}} & \mc{1}{c}{\scriptsize{-0.985}} & \mc{1}{c}{\scriptsize{-0.318}} & \mc{1}{c}{\scriptsize{2.068}} & \mc{1}{c}{\scriptsize{-1.385}} & \mc{1}{c}{\scriptsize{-1.698}} & \mc{1}{c}{\scriptsize{-1.526}} & \mc{1}{c}{\scriptsize{-2.667}} \\  

     &  & \mc{1}{c}{\scriptsize{(1.000)}} & \mc{1}{c}{\scriptsize{(1.000)}} & \mc{1}{c}{\scriptsize{(1.000)}} & \mc{1}{c}{\scriptsize{(0.974)}} & \mc{1}{c}{\scriptsize{(1.000)}} & \mc{1}{c}{\scriptsize{(1.000)}} & \mc{1}{c}{\scriptsize{(1.000)}} & \mc{1}{c}{\scriptsize{(1.000)}} \\  

    \mc{1}{l}{\scriptsize{Platelets}} & \mc{1}{c}{\scriptsize{Mid-30s}} & \mc{1}{c}{\scriptsize{-8.643}} & \mc{1}{c}{\scriptsize{-5.673}} & \mc{1}{c}{\scriptsize{-6.667}} & \mc{1}{c}{\scriptsize{-0.322}} & \mc{1}{c}{\scriptsize{-2.212}} & \mc{1}{c}{\scriptsize{-9.182}} & \mc{1}{c}{\scriptsize{-6.809}} & \mc{1}{c}{\scriptsize{-4.750}} \\  

     &  & \mc{1}{c}{\scriptsize{(1.000)}} & \mc{1}{c}{\scriptsize{(1.000)}} & \mc{1}{c}{\scriptsize{(1.000)}} & \mc{1}{c}{\scriptsize{(1.000)}} & \mc{1}{c}{\scriptsize{(1.000)}} & \mc{1}{c}{\scriptsize{(1.000)}} & \mc{1}{c}{\scriptsize{(1.000)}} & \mc{1}{c}{\scriptsize{(1.000)}} \\  

    \mc{1}{l}{\scriptsize{Eosinophils}} & \mc{1}{c}{\scriptsize{Mid-30s}} & \mc{1}{c}{\scriptsize{0.469}} & \mc{1}{c}{\scriptsize{0.499}} & \mc{1}{c}{\scriptsize{0.625}} & \mc{1}{c}{\scriptsize{1.072}} & \mc{1}{c}{\scriptsize{0.717}} & \mc{1}{c}{\scriptsize{0.425}} & \mc{1}{c}{\scriptsize{0.291}} & \mc{1}{c}{\scriptsize{0.514}} \\  

     &  & \mc{1}{c}{\scriptsize{(0.724)}} & \mc{1}{c}{\scriptsize{(0.553)}} & \mc{1}{c}{\scriptsize{(0.342)}} & \mc{1}{c}{\scriptsize{(0.329)}} & \mc{1}{c}{\scriptsize{(0.250)}} & \mc{1}{c}{\scriptsize{(0.842)}} & \mc{1}{c}{\scriptsize{(0.934)}} & \mc{1}{c}{\scriptsize{(0.737)}} \\  

    \mc{1}{l}{\scriptsize{Hemoglobin}} & \mc{1}{c}{\scriptsize{Mid-30s}} & \mc{1}{c}{\scriptsize{-0.222}} & \mc{1}{c}{\scriptsize{-0.511}} & \mc{1}{c}{\scriptsize{-0.094}} & \mc{1}{c}{\scriptsize{0.130}} & \mc{1}{c}{\scriptsize{-0.486}} & \mc{1}{c}{\scriptsize{-0.256}} & \mc{1}{c}{\scriptsize{-0.623}} & \mc{1}{c}{\scriptsize{-0.667}} \\  

     &  & \mc{1}{c}{\scriptsize{(1.000)}} & \mc{1}{c}{\scriptsize{(1.000)}} & \mc{1}{c}{\scriptsize{(1.000)}} & \mc{1}{c}{\scriptsize{(1.000)}} & \mc{1}{c}{\scriptsize{(1.000)}} & \mc{1}{c}{\scriptsize{(1.000)}} & \mc{1}{c}{\scriptsize{(1.000)}} & \mc{1}{c}{\scriptsize{(1.000)}} \\  

    \mc{1}{l}{\scriptsize{Red Cells}} & \mc{1}{c}{\scriptsize{Mid-30s}} & \mc{1}{c}{\scriptsize{-0.011}} & \mc{1}{c}{\scriptsize{-0.117}} & \mc{1}{c}{\scriptsize{0.013}} & \mc{1}{c}{\scriptsize{-0.062}} & \mc{1}{c}{\scriptsize{-0.037}} & \mc{1}{c}{\scriptsize{-0.018}} & \mc{1}{c}{\scriptsize{-0.123}} & \mc{1}{c}{\scriptsize{-0.068}} \\  

     &  & \mc{1}{c}{\scriptsize{(1.000)}} & \mc{1}{c}{\scriptsize{(1.000)}} & \mc{1}{c}{\scriptsize{(1.000)}} & \mc{1}{c}{\scriptsize{(1.000)}} & \mc{1}{c}{\scriptsize{(1.000)}} & \mc{1}{c}{\scriptsize{(1.000)}} & \mc{1}{c}{\scriptsize{(1.000)}} & \mc{1}{c}{\scriptsize{(1.000)}} \\  

    \mc{1}{l}{\scriptsize{Lymphocytes}} & \mc{1}{c}{\scriptsize{Mid-30s}} & \mc{1}{c}{\scriptsize{-1.522}} & \mc{1}{c}{\scriptsize{-2.676}} & \mc{1}{c}{\scriptsize{1.464}} & \mc{1}{c}{\scriptsize{1.877}} & \mc{1}{c}{\scriptsize{1.702}} & \mc{1}{c}{\scriptsize{-2.376}} & \mc{1}{c}{\scriptsize{-4.076}} & \mc{1}{c}{\scriptsize{-2.784}} \\  

     &  & \mc{1}{c}{\scriptsize{(1.000)}} & \mc{1}{c}{\scriptsize{(1.000)}} & \mc{1}{c}{\scriptsize{(0.987)}} & \mc{1}{c}{\scriptsize{(0.974)}} & \mc{1}{c}{\scriptsize{(0.987)}} & \mc{1}{c}{\scriptsize{(1.000)}} & \mc{1}{c}{\scriptsize{(1.000)}} & \mc{1}{c}{\scriptsize{(1.000)}} \\  

    \mc{1}{l}{\scriptsize{Monocytes}} & \mc{1}{c}{\scriptsize{Mid-30s}} & \mc{1}{c}{\scriptsize{-0.431}} & \mc{1}{c}{\scriptsize{-0.297}} & \mc{1}{c}{\scriptsize{-0.459}} & \mc{1}{c}{\scriptsize{0.323}} & \mc{1}{c}{\scriptsize{-0.530}} & \mc{1}{c}{\scriptsize{-0.423}} & \mc{1}{c}{\scriptsize{-0.470}} & \mc{1}{c}{\scriptsize{-0.582}} \\  

     &  & \mc{1}{c}{\scriptsize{(1.000)}} & \mc{1}{c}{\scriptsize{(1.000)}} & \mc{1}{c}{\scriptsize{(1.000)}} & \mc{1}{c}{\scriptsize{(0.987)}} & \mc{1}{c}{\scriptsize{(1.000)}} & \mc{1}{c}{\scriptsize{(1.000)}} & \mc{1}{c}{\scriptsize{(1.000)}} & \mc{1}{c}{\scriptsize{(1.000)}} \\  

    \mc{1}{l}{\scriptsize{Neutrophils}} & \mc{1}{c}{\scriptsize{Mid-30s}} & \mc{1}{c}{\scriptsize{1.455}} & \mc{1}{c}{\scriptsize{2.443}} & \mc{1}{c}{\scriptsize{-1.697}} & \mc{1}{c}{\scriptsize{-3.341}} & \mc{1}{c}{\scriptsize{-1.978}} & \mc{1}{c}{\scriptsize{2.355}} & \mc{1}{c}{\scriptsize{4.232}} & \mc{1}{c}{\scriptsize{2.811}} \\  

     &  & \mc{1}{c}{\scriptsize{(0.974)}} & \mc{1}{c}{\scriptsize{(0.921)}} & \mc{1}{c}{\scriptsize{(1.000)}} & \mc{1}{c}{\scriptsize{(1.000)}} & \mc{1}{c}{\scriptsize{(1.000)}} & \mc{1}{c}{\scriptsize{(0.974)}} & \mc{1}{c}{\scriptsize{(0.763)}} & \mc{1}{c}{\scriptsize{(0.908)}} \\  

    \mc{1}{l}{\scriptsize{Basophils}} & \mc{1}{c}{\scriptsize{Mid-30s}} & \mc{1}{c}{\scriptsize{0.031}} & \mc{1}{c}{\scriptsize{0.033}} & \mc{1}{c}{\scriptsize{0.066}} & \mc{1}{c}{\scriptsize{0.068}} & \mc{1}{c}{\scriptsize{0.089}} & \mc{1}{c}{\scriptsize{0.021}} & \mc{1}{c}{\scriptsize{0.026}} & \mc{1}{c}{\scriptsize{0.042}} \\  

     &  & \mc{1}{c}{\scriptsize{(0.947)}} & \mc{1}{c}{\scriptsize{(0.961)}} & \mc{1}{c}{\scriptsize{(0.947)}} & \mc{1}{c}{\scriptsize{(0.934)}} & \mc{1}{c}{\scriptsize{(0.829)}} & \mc{1}{c}{\scriptsize{(0.974)}} & \mc{1}{c}{\scriptsize{(0.987)}} & \mc{1}{c}{\scriptsize{(0.895)}} \\  

    \mc{1}{l}{\scriptsize{Mean Hemoglobin}} & \mc{1}{c}{\scriptsize{Mid-30s}} & \mc{1}{c}{\scriptsize{-0.436}} & \mc{1}{c}{\scriptsize{-0.344}} & \mc{1}{c}{\scriptsize{-0.206}} & \mc{1}{c}{\scriptsize{0.797}} & \mc{1}{c}{\scriptsize{-0.712}} & \mc{1}{c}{\scriptsize{-0.498}} & \mc{1}{c}{\scriptsize{-0.578}} & \mc{1}{c}{\scriptsize{-1.034}} \\  

     &  & \mc{1}{c}{\scriptsize{(1.000)}} & \mc{1}{c}{\scriptsize{(1.000)}} & \mc{1}{c}{\scriptsize{(1.000)}} & \mc{1}{c}{\scriptsize{(0.974)}} & \mc{1}{c}{\scriptsize{(1.000)}} & \mc{1}{c}{\scriptsize{(1.000)}} & \mc{1}{c}{\scriptsize{(1.000)}} & \mc{1}{c}{\scriptsize{(1.000)}} \\  

    \mc{1}{l}{\scriptsize{White Cells}} & \mc{1}{c}{\scriptsize{Mid-30s}} & \mc{1}{c}{\scriptsize{0.620}} & \mc{1}{c}{\scriptsize{0.135}} & \mc{1}{c}{\scriptsize{-1.270}} & \mc{1}{c}{\scriptsize{-2.281}} & \mc{1}{c}{\scriptsize{-1.339}} & \mc{1}{c}{\scriptsize{1.136}} & \mc{1}{c}{\scriptsize{0.911}} & \mc{1}{c}{\scriptsize{1.001}} \\  

     &  & \mc{1}{c}{\scriptsize{(0.855)}} & \mc{1}{c}{\scriptsize{(1.000)}} & \mc{1}{c}{\scriptsize{(1.000)}} & \mc{1}{c}{\scriptsize{(1.000)}} & \mc{1}{c}{\scriptsize{(1.000)}} & \mc{1}{c}{\scriptsize{\textbf{(0.079)}}} & \mc{1}{c}{\scriptsize{(0.513)}} & \mc{1}{c}{\scriptsize{(0.329)}} \\  

    \mc{1}{l}{\scriptsize{Hematocrit}} & \mc{1}{c}{\scriptsize{Mid-30s}} & \mc{1}{c}{\scriptsize{-0.710}} & \mc{1}{c}{\scriptsize{-1.467}} & \mc{1}{c}{\scriptsize{-0.123}} & \mc{1}{c}{\scriptsize{0.281}} & \mc{1}{c}{\scriptsize{-1.073}} & \mc{1}{c}{\scriptsize{-0.870}} & \mc{1}{c}{\scriptsize{-1.736}} & \mc{1}{c}{\scriptsize{-1.792}} \\  

     &  & \mc{1}{c}{\scriptsize{(1.000)}} & \mc{1}{c}{\scriptsize{(1.000)}} & \mc{1}{c}{\scriptsize{(1.000)}} & \mc{1}{c}{\scriptsize{(1.000)}} & \mc{1}{c}{\scriptsize{(1.000)}} & \mc{1}{c}{\scriptsize{(1.000)}} & \mc{1}{c}{\scriptsize{(1.000)}} & \mc{1}{c}{\scriptsize{(1.000)}} \\  

    \mc{1}{l}{\scriptsize{Red Cell Width}} & \mc{1}{c}{\scriptsize{Mid-30s}} & \mc{1}{c}{\scriptsize{0.312}} & \mc{1}{c}{\scriptsize{0.346}} & \mc{1}{c}{\scriptsize{-0.680}} & \mc{1}{c}{\scriptsize{-0.887}} & \mc{1}{c}{\scriptsize{-0.234}} & \mc{1}{c}{\scriptsize{0.583}} & \mc{1}{c}{\scriptsize{0.791}} & \mc{1}{c}{\scriptsize{1.002}} \\  

     &  & \mc{1}{c}{\scriptsize{(0.947)}} & \mc{1}{c}{\scriptsize{(0.961)}} & \mc{1}{c}{\scriptsize{(1.000)}} & \mc{1}{c}{\scriptsize{(1.000)}} & \mc{1}{c}{\scriptsize{(1.000)}} & \mc{1}{c}{\scriptsize{(0.842)}} & \mc{1}{c}{\scriptsize{(0.763)}} & \mc{1}{c}{\scriptsize{(0.408)}} \\  

    \mc{1}{l}{\scriptsize{Mean Hb Concentration}} & \mc{1}{c}{\scriptsize{Mid-30s}} & \mc{1}{c}{\scriptsize{0.010}} & \mc{1}{c}{\scriptsize{-0.023}} & \mc{1}{c}{\scriptsize{-0.209}} & \mc{1}{c}{\scriptsize{0.089}} & \mc{1}{c}{\scriptsize{-0.417}} & \mc{1}{c}{\scriptsize{0.069}} & \mc{1}{c}{\scriptsize{-0.062}} & \mc{1}{c}{\scriptsize{-0.209}} \\  

     &  & \mc{1}{c}{\scriptsize{(1.000)}} & \mc{1}{c}{\scriptsize{(1.000)}} & \mc{1}{c}{\scriptsize{(1.000)}} & \mc{1}{c}{\scriptsize{(1.000)}} & \mc{1}{c}{\scriptsize{(1.000)}} & \mc{1}{c}{\scriptsize{(1.000)}} & \mc{1}{c}{\scriptsize{(1.000)}} & \mc{1}{c}{\scriptsize{(1.000)}} \\  

  \bottomrule
  \end{tabular}
	\end{table} 

	\begin{table}[H]
     \caption{Treatment Effects on Achenbach Behavior, Female Sample}
     \label{table:abccare_rslt_female_cat8_sd}
	  \begin{tabular}{cccccccccc}
  \toprule

    \scriptsize{Variable} & \scriptsize{Age} & \scriptsize{(1)} & \scriptsize{(2)} & \scriptsize{(3)} & \scriptsize{(4)} & \scriptsize{(5)} & \scriptsize{(6)} & \scriptsize{(7)} & \scriptsize{(8)} \\ 
    \midrule  

    \mc{1}{l}{\scriptsize{Activities T Score (Reported by Mother)}} & \mc{1}{c}{\scriptsize{8}} & \mc{1}{c}{\scriptsize{3.200}} & \mc{1}{c}{\scriptsize{3.928}} & \mc{1}{c}{\scriptsize{5.367}} & \mc{1}{c}{\scriptsize{5.549}} & \mc{1}{c}{\scriptsize{5.257}} & \mc{1}{c}{\scriptsize{2.858}} & \mc{1}{c}{\scriptsize{3.320}} & \mc{1}{c}{\scriptsize{3.805}} \\  

     &  & \mc{1}{c}{\scriptsize{(0.263)}} & \mc{1}{c}{\scriptsize{(0.171)}} & \mc{1}{c}{\scriptsize{(0.618)}} & \mc{1}{c}{\scriptsize{(0.553)}} & \mc{1}{c}{\scriptsize{(0.697)}} & \mc{1}{c}{\scriptsize{(0.395)}} & \mc{1}{c}{\scriptsize{(0.421)}} & \mc{1}{c}{\scriptsize{(0.197)}} \\  

    \mc{1}{l}{\scriptsize{Social T Score (Reported by Mother)}} & \mc{1}{c}{\scriptsize{8}} & \mc{1}{c}{\scriptsize{2.023}} & \mc{1}{c}{\scriptsize{0.758}} & \mc{1}{c}{\scriptsize{6.325}} & \mc{1}{c}{\scriptsize{4.437}} & \mc{1}{c}{\scriptsize{6.375}} & \mc{1}{c}{\scriptsize{0.520}} & \mc{1}{c}{\scriptsize{-1.180}} & \mc{1}{c}{\scriptsize{0.466}} \\  

     &  & \mc{1}{c}{\scriptsize{(0.618)}} & \mc{1}{c}{\scriptsize{(0.947)}} & \mc{1}{c}{\scriptsize{\textbf{(0.039)}}} & \mc{1}{c}{\scriptsize{(0.303)}} & \mc{1}{c}{\scriptsize{\textbf{(0.026)}}} & \mc{1}{c}{\scriptsize{(0.934)}} & \mc{1}{c}{\scriptsize{(1.000)}} & \mc{1}{c}{\scriptsize{(0.987)}} \\  

    \mc{1}{l}{\scriptsize{Activities T Score (Reported by Mother)}} & \mc{1}{c}{\scriptsize{12}} & \mc{1}{c}{\scriptsize{-1.765}} & \mc{1}{c}{\scriptsize{-1.851}} & \mc{1}{c}{\scriptsize{1.647}} & \mc{1}{c}{\scriptsize{8.028}} & \mc{1}{c}{\scriptsize{6.895}} & \mc{1}{c}{\scriptsize{-2.603}} & \mc{1}{c}{\scriptsize{-3.470}} & \mc{1}{c}{\scriptsize{0.913}} \\  

     &  & \mc{1}{c}{\scriptsize{(1.000)}} & \mc{1}{c}{\scriptsize{(1.000)}} & \mc{1}{c}{\scriptsize{(0.882)}} & \mc{1}{c}{\scriptsize{(0.566)}} & \mc{1}{c}{\scriptsize{(0.513)}} & \mc{1}{c}{\scriptsize{(1.000)}} & \mc{1}{c}{\scriptsize{(1.000)}} & \mc{1}{c}{\scriptsize{(0.987)}} \\  

    \mc{1}{l}{\scriptsize{Behave Appropriate T Score (Reported by Teacher)}} & \mc{1}{c}{\scriptsize{12}} & \mc{1}{c}{\scriptsize{5.059}} & \mc{1}{c}{\scriptsize{4.253}} & \mc{1}{c}{\scriptsize{8.985}} & \mc{1}{c}{\scriptsize{12.243}} & \mc{1}{c}{\scriptsize{10.480}} & \mc{1}{c}{\scriptsize{3.402}} & \mc{1}{c}{\scriptsize{3.562}} & \mc{1}{c}{\scriptsize{4.315}} \\  

     &  & \mc{1}{c}{\scriptsize{(0.171)}} & \mc{1}{c}{\scriptsize{(0.737)}} & \mc{1}{c}{\scriptsize{(0.408)}} & \mc{1}{c}{\scriptsize{(0.184)}} & \mc{1}{c}{\scriptsize{(0.263)}} & \mc{1}{c}{\scriptsize{(0.553)}} & \mc{1}{c}{\scriptsize{(0.816)}} & \mc{1}{c}{\scriptsize{(0.500)}} \\  

    \mc{1}{l}{\scriptsize{Happiness T Score (Reported by Teacher)}} & \mc{1}{c}{\scriptsize{12}} & \mc{1}{c}{\scriptsize{0.941}} & \mc{1}{c}{\scriptsize{1.241}} & \mc{1}{c}{\scriptsize{4.853}} & \mc{1}{c}{\scriptsize{4.741}} & \mc{1}{c}{\scriptsize{7.395}} & \mc{1}{c}{\scriptsize{-1.147}} & \mc{1}{c}{\scriptsize{1.084}} & \mc{1}{c}{\scriptsize{1.139}} \\  

     &  & \mc{1}{c}{\scriptsize{(0.908)}} & \mc{1}{c}{\scriptsize{(0.921)}} & \mc{1}{c}{\scriptsize{(0.132)}} & \mc{1}{c}{\scriptsize{(0.697)}} & \mc{1}{c}{\scriptsize{\textbf{(0.026)}}} & \mc{1}{c}{\scriptsize{(1.000)}} & \mc{1}{c}{\scriptsize{(0.947)}} & \mc{1}{c}{\scriptsize{(0.934)}} \\  

    \mc{1}{l}{\scriptsize{Learning T Score (Reported by Teacher)}} & \mc{1}{c}{\scriptsize{12}} & \mc{1}{c}{\scriptsize{1.677}} & \mc{1}{c}{\scriptsize{2.727}} & \mc{1}{c}{\scriptsize{0.450}} & \mc{1}{c}{\scriptsize{-1.058}} & \mc{1}{c}{\scriptsize{-0.098}} & \mc{1}{c}{\scriptsize{1.871}} & \mc{1}{c}{\scriptsize{3.689}} & \mc{1}{c}{\scriptsize{2.839}} \\  

     &  & \mc{1}{c}{\scriptsize{(0.711)}} & \mc{1}{c}{\scriptsize{(0.750)}} & \mc{1}{c}{\scriptsize{(0.921)}} & \mc{1}{c}{\scriptsize{(0.974)}} & \mc{1}{c}{\scriptsize{(0.987)}} & \mc{1}{c}{\scriptsize{(0.658)}} & \mc{1}{c}{\scriptsize{(0.605)}} & \mc{1}{c}{\scriptsize{(0.421)}} \\  

    \mc{1}{l}{\scriptsize{Social T Score (Reported by Mother)}} & \mc{1}{c}{\scriptsize{12}} & \mc{1}{c}{\scriptsize{1.087}} & \mc{1}{c}{\scriptsize{-0.392}} & \mc{1}{c}{\scriptsize{6.043}} & \mc{1}{c}{\scriptsize{1.135}} & \mc{1}{c}{\scriptsize{5.939}} & \mc{1}{c}{\scriptsize{-0.287}} & \mc{1}{c}{\scriptsize{-0.787}} & \mc{1}{c}{\scriptsize{-0.283}} \\  

     &  & \mc{1}{c}{\scriptsize{(0.882)}} & \mc{1}{c}{\scriptsize{(0.987)}} & \mc{1}{c}{\scriptsize{(0.421)}} & \mc{1}{c}{\scriptsize{(0.803)}} & \mc{1}{c}{\scriptsize{(0.553)}} & \mc{1}{c}{\scriptsize{(0.987)}} & \mc{1}{c}{\scriptsize{(1.000)}} & \mc{1}{c}{\scriptsize{(1.000)}} \\  

    \mc{1}{l}{\scriptsize{Work Hard T Score (Reported by Teacher)}} & \mc{1}{c}{\scriptsize{12}} & \mc{1}{c}{\scriptsize{-1.647}} & \mc{1}{c}{\scriptsize{-1.438}} & \mc{1}{c}{\scriptsize{2.676}} & \mc{1}{c}{\scriptsize{4.409}} & \mc{1}{c}{\scriptsize{5.511}} & \mc{1}{c}{\scriptsize{-3.907}} & \mc{1}{c}{\scriptsize{-2.241}} & \mc{1}{c}{\scriptsize{-1.624}} \\  

     &  & \mc{1}{c}{\scriptsize{(1.000)}} & \mc{1}{c}{\scriptsize{(1.000)}} & \mc{1}{c}{\scriptsize{(0.711)}} & \mc{1}{c}{\scriptsize{(0.711)}} & \mc{1}{c}{\scriptsize{(0.237)}} & \mc{1}{c}{\scriptsize{(1.000)}} & \mc{1}{c}{\scriptsize{(1.000)}} & \mc{1}{c}{\scriptsize{(1.000)}} \\  

  \bottomrule
  \end{tabular}
	\end{table} 

	\begin{table}[H]
     \caption{Treatment Effects on Achenbach Symptom T Score (Reported by Mother), Female Sample}
     \label{table:abccare_rslt_female_cat9_sd}
	  \begin{tabular}{cccccccccc}
  \toprule

    \scriptsize{Variable} & \scriptsize{Age} & \scriptsize{(1)} & \scriptsize{(2)} & \scriptsize{(3)} & \scriptsize{(4)} & \scriptsize{(5)} & \scriptsize{(6)} & \scriptsize{(7)} & \scriptsize{(8)} \\ 
    \midrule  

    \mc{1}{l}{\scriptsize{Aggressive}} & \mc{1}{c}{\scriptsize{8}} & \mc{1}{c}{\scriptsize{3.067}} & \mc{1}{c}{\scriptsize{1.528}} & \mc{1}{c}{\scriptsize{1.214}} & \mc{1}{c}{\scriptsize{-1.777}} & \mc{1}{c}{\scriptsize{-0.027}} & \mc{1}{c}{\scriptsize{3.630}} & \mc{1}{c}{\scriptsize{2.726}} & \mc{1}{c}{\scriptsize{3.007}} \\  

     &  & \mc{1}{c}{\scriptsize{(1.000)}} & \mc{1}{c}{\scriptsize{(1.000)}} & \mc{1}{c}{\scriptsize{(0.974)}} & \mc{1}{c}{\scriptsize{(0.842)}} & \mc{1}{c}{\scriptsize{(0.961)}} & \mc{1}{c}{\scriptsize{(1.000)}} & \mc{1}{c}{\scriptsize{(1.000)}} & \mc{1}{c}{\scriptsize{(1.000)}} \\  

     & \mc{1}{c}{\scriptsize{12}} & \mc{1}{c}{\scriptsize{-3.316}} & \mc{1}{c}{\scriptsize{-2.178}} & \mc{1}{c}{\scriptsize{-8.808}} & \mc{1}{c}{\scriptsize{-5.918}} & \mc{1}{c}{\scriptsize{-8.412}} & \mc{1}{c}{\scriptsize{-1.673}} & \mc{1}{c}{\scriptsize{-0.542}} & \mc{1}{c}{\scriptsize{-1.193}} \\  

     &  & \mc{1}{c}{\scriptsize{(0.132)}} & \mc{1}{c}{\scriptsize{(0.618)}} & \mc{1}{c}{\scriptsize{\textbf{(0.092)}}} & \mc{1}{c}{\scriptsize{(0.211)}} & \mc{1}{c}{\scriptsize{(0.118)}} & \mc{1}{c}{\scriptsize{(0.579)}} & \mc{1}{c}{\scriptsize{(0.934)}} & \mc{1}{c}{\scriptsize{(0.829)}} \\  

    \mc{1}{l}{\scriptsize{Delinquent}} & \mc{1}{c}{\scriptsize{8}} & \mc{1}{c}{\scriptsize{0.883}} & \mc{1}{c}{\scriptsize{1.149}} & \mc{1}{c}{\scriptsize{1.607}} & \mc{1}{c}{\scriptsize{2.751}} & \mc{1}{c}{\scriptsize{1.170}} & \mc{1}{c}{\scriptsize{0.663}} & \mc{1}{c}{\scriptsize{1.097}} & \mc{1}{c}{\scriptsize{0.139}} \\  

     &  & \mc{1}{c}{\scriptsize{(1.000)}} & \mc{1}{c}{\scriptsize{(1.000)}} & \mc{1}{c}{\scriptsize{(0.974)}} & \mc{1}{c}{\scriptsize{(1.000)}} & \mc{1}{c}{\scriptsize{(0.987)}} & \mc{1}{c}{\scriptsize{(0.987)}} & \mc{1}{c}{\scriptsize{(0.987)}} & \mc{1}{c}{\scriptsize{(0.974)}} \\  

     & \mc{1}{c}{\scriptsize{12}} & \mc{1}{c}{\scriptsize{-2.243}} & \mc{1}{c}{\scriptsize{-1.224}} & \mc{1}{c}{\scriptsize{-4.942}} & \mc{1}{c}{\scriptsize{-2.737}} & \mc{1}{c}{\scriptsize{-4.699}} & \mc{1}{c}{\scriptsize{-1.287}} & \mc{1}{c}{\scriptsize{-0.178}} & \mc{1}{c}{\scriptsize{-0.927}} \\  

     &  & \mc{1}{c}{\scriptsize{(0.316)}} & \mc{1}{c}{\scriptsize{(0.829)}} & \mc{1}{c}{\scriptsize{(0.224)}} & \mc{1}{c}{\scriptsize{(0.618)}} & \mc{1}{c}{\scriptsize{(0.289)}} & \mc{1}{c}{\scriptsize{(0.671)}} & \mc{1}{c}{\scriptsize{(0.961)}} & \mc{1}{c}{\scriptsize{(0.842)}} \\  

    \mc{1}{l}{\scriptsize{Depressed}} & \mc{1}{c}{\scriptsize{8}} & \mc{1}{c}{\scriptsize{1.500}} & \mc{1}{c}{\scriptsize{0.265}} & \mc{1}{c}{\scriptsize{1.833}} & \mc{1}{c}{\scriptsize{-2.193}} & \mc{1}{c}{\scriptsize{1.338}} & \mc{1}{c}{\scriptsize{1.399}} & \mc{1}{c}{\scriptsize{0.967}} & \mc{1}{c}{\scriptsize{2.000}} \\  

     &  & \mc{1}{c}{\scriptsize{(1.000)}} & \mc{1}{c}{\scriptsize{(0.987)}} & \mc{1}{c}{\scriptsize{(1.000)}} & \mc{1}{c}{\scriptsize{(0.711)}} & \mc{1}{c}{\scriptsize{(0.987)}} & \mc{1}{c}{\scriptsize{(1.000)}} & \mc{1}{c}{\scriptsize{(0.987)}} & \mc{1}{c}{\scriptsize{(1.000)}} \\  

    \mc{1}{l}{\scriptsize{Externalizing}} & \mc{1}{c}{\scriptsize{8}} & \mc{1}{c}{\scriptsize{0.917}} & \mc{1}{c}{\scriptsize{-0.787}} & \mc{1}{c}{\scriptsize{-1.702}} & \mc{1}{c}{\scriptsize{-5.146}} & \mc{1}{c}{\scriptsize{-3.512}} & \mc{1}{c}{\scriptsize{1.714}} & \mc{1}{c}{\scriptsize{1.395}} & \mc{1}{c}{\scriptsize{1.320}} \\  

     &  & \mc{1}{c}{\scriptsize{(1.000)}} & \mc{1}{c}{\scriptsize{(0.947)}} & \mc{1}{c}{\scriptsize{(0.776)}} & \mc{1}{c}{\scriptsize{(0.579)}} & \mc{1}{c}{\scriptsize{(0.671)}} & \mc{1}{c}{\scriptsize{(1.000)}} & \mc{1}{c}{\scriptsize{(0.987)}} & \mc{1}{c}{\scriptsize{(0.974)}} \\  

     & \mc{1}{c}{\scriptsize{12}} & \mc{1}{c}{\scriptsize{-3.249}} & \mc{1}{c}{\scriptsize{-1.746}} & \mc{1}{c}{\scriptsize{-6.992}} & \mc{1}{c}{\scriptsize{-3.715}} & \mc{1}{c}{\scriptsize{-6.279}} & \mc{1}{c}{\scriptsize{-2.067}} & \mc{1}{c}{\scriptsize{-0.572}} & \mc{1}{c}{\scriptsize{-1.187}} \\  

     &  & \mc{1}{c}{\scriptsize{(0.197)}} & \mc{1}{c}{\scriptsize{(0.750)}} & \mc{1}{c}{\scriptsize{(0.237)}} & \mc{1}{c}{\scriptsize{(0.605)}} & \mc{1}{c}{\scriptsize{(0.303)}} & \mc{1}{c}{\scriptsize{(0.513)}} & \mc{1}{c}{\scriptsize{(0.934)}} & \mc{1}{c}{\scriptsize{(0.842)}} \\  

    \mc{1}{l}{\scriptsize{Hyperactive}} & \mc{1}{c}{\scriptsize{8}} & \mc{1}{c}{\scriptsize{-0.517}} & \mc{1}{c}{\scriptsize{-1.339}} & \mc{1}{c}{\scriptsize{-2.417}} & \mc{1}{c}{\scriptsize{-5.989}} & \mc{1}{c}{\scriptsize{-3.982}} & \mc{1}{c}{\scriptsize{0.062}} & \mc{1}{c}{\scriptsize{0.215}} & \mc{1}{c}{\scriptsize{0.023}} \\  

     &  & \mc{1}{c}{\scriptsize{(0.882)}} & \mc{1}{c}{\scriptsize{(0.842)}} & \mc{1}{c}{\scriptsize{(0.618)}} & \mc{1}{c}{\scriptsize{(0.303)}} & \mc{1}{c}{\scriptsize{(0.526)}} & \mc{1}{c}{\scriptsize{(0.934)}} & \mc{1}{c}{\scriptsize{(0.987)}} & \mc{1}{c}{\scriptsize{(0.974)}} \\  

    \mc{1}{l}{\scriptsize{Internalizing}} & \mc{1}{c}{\scriptsize{8}} & \mc{1}{c}{\scriptsize{0.850}} & \mc{1}{c}{\scriptsize{-0.543}} & \mc{1}{c}{\scriptsize{1.060}} & \mc{1}{c}{\scriptsize{-5.353}} & \mc{1}{c}{\scriptsize{-0.190}} & \mc{1}{c}{\scriptsize{0.786}} & \mc{1}{c}{\scriptsize{0.727}} & \mc{1}{c}{\scriptsize{1.508}} \\  

     &  & \mc{1}{c}{\scriptsize{(1.000)}} & \mc{1}{c}{\scriptsize{(0.961)}} & \mc{1}{c}{\scriptsize{(0.961)}} & \mc{1}{c}{\scriptsize{(0.355)}} & \mc{1}{c}{\scriptsize{(0.947)}} & \mc{1}{c}{\scriptsize{(0.987)}} & \mc{1}{c}{\scriptsize{(0.987)}} & \mc{1}{c}{\scriptsize{(0.974)}} \\  

     & \mc{1}{c}{\scriptsize{12}} & \mc{1}{c}{\scriptsize{-3.276}} & \mc{1}{c}{\scriptsize{-3.265}} & \mc{1}{c}{\scriptsize{-5.975}} & \mc{1}{c}{\scriptsize{-5.469}} & \mc{1}{c}{\scriptsize{-6.086}} & \mc{1}{c}{\scriptsize{-2.540}} & \mc{1}{c}{\scriptsize{-2.605}} & \mc{1}{c}{\scriptsize{-2.183}} \\  

     &  & \mc{1}{c}{\scriptsize{(0.132)}} & \mc{1}{c}{\scriptsize{(0.316)}} & \mc{1}{c}{\scriptsize{(0.158)}} & \mc{1}{c}{\scriptsize{(0.224)}} & \mc{1}{c}{\scriptsize{(0.158)}} & \mc{1}{c}{\scriptsize{(0.395)}} & \mc{1}{c}{\scriptsize{(0.526)}} & \mc{1}{c}{\scriptsize{(0.553)}} \\  

    \mc{1}{l}{\scriptsize{Schizoid}} & \mc{1}{c}{\scriptsize{12}} & \mc{1}{c}{\scriptsize{-1.718}} & \mc{1}{c}{\scriptsize{-0.631}} & \mc{1}{c}{\scriptsize{-3.475}} & \mc{1}{c}{\scriptsize{-2.538}} & \mc{1}{c}{\scriptsize{-3.202}} & \mc{1}{c}{\scriptsize{-1.320}} & \mc{1}{c}{\scriptsize{-0.024}} & \mc{1}{c}{\scriptsize{-0.291}} \\  

     &  & \mc{1}{c}{\scriptsize{(0.487)}} & \mc{1}{c}{\scriptsize{(0.934)}} & \mc{1}{c}{\scriptsize{(0.461)}} & \mc{1}{c}{\scriptsize{(0.632)}} & \mc{1}{c}{\scriptsize{(0.513)}} & \mc{1}{c}{\scriptsize{(0.737)}} & \mc{1}{c}{\scriptsize{(0.987)}} & \mc{1}{c}{\scriptsize{(0.908)}} \\  

    \mc{1}{l}{\scriptsize{Somatic Complaints}} & \mc{1}{c}{\scriptsize{8}} & \mc{1}{c}{\scriptsize{0.150}} & \mc{1}{c}{\scriptsize{-0.241}} & \mc{1}{c}{\scriptsize{0.083}} & \mc{1}{c}{\scriptsize{-1.905}} & \mc{1}{c}{\scriptsize{0.109}} & \mc{1}{c}{\scriptsize{0.170}} & \mc{1}{c}{\scriptsize{-0.169}} & \mc{1}{c}{\scriptsize{0.741}} \\  

     &  & \mc{1}{c}{\scriptsize{(0.974)}} & \mc{1}{c}{\scriptsize{(0.974)}} & \mc{1}{c}{\scriptsize{(0.934)}} & \mc{1}{c}{\scriptsize{(0.474)}} & \mc{1}{c}{\scriptsize{(0.961)}} & \mc{1}{c}{\scriptsize{(0.947)}} & \mc{1}{c}{\scriptsize{(0.961)}} & \mc{1}{c}{\scriptsize{(0.974)}} \\  

     & \mc{1}{c}{\scriptsize{12}} & \mc{1}{c}{\scriptsize{-2.594}} & \mc{1}{c}{\scriptsize{-3.123}} & \mc{1}{c}{\scriptsize{-3.550}} & \mc{1}{c}{\scriptsize{-4.276}} & \mc{1}{c}{\scriptsize{-4.007}} & \mc{1}{c}{\scriptsize{-2.100}} & \mc{1}{c}{\scriptsize{-2.955}} & \mc{1}{c}{\scriptsize{-2.143}} \\  

     &  & \mc{1}{c}{\scriptsize{(0.118)}} & \mc{1}{c}{\scriptsize{(0.184)}} & \mc{1}{c}{\scriptsize{(0.316)}} & \mc{1}{c}{\scriptsize{(0.408)}} & \mc{1}{c}{\scriptsize{(0.263)}} & \mc{1}{c}{\scriptsize{(0.329)}} & \mc{1}{c}{\scriptsize{(0.303)}} & \mc{1}{c}{\scriptsize{(0.263)}} \\  

  \bottomrule
  \end{tabular}
	\end{table} 

	\begin{table}[H]
     \caption{Treatment Effects on Achenbach Symptom T Score (Reported by Teacher), Female Sample}
     \label{table:abccare_rslt_female_cat10_sd}
	  \begin{tabular}{cccccccccc}
  \toprule

    \scriptsize{Variable} & \scriptsize{Age} & \scriptsize{(1)} & \scriptsize{(2)} & \scriptsize{(3)} & \scriptsize{(4)} & \scriptsize{(5)} & \scriptsize{(6)} & \scriptsize{(7)} & \scriptsize{(8)} \\ 
    \midrule  

    \mc{1}{l}{\scriptsize{Self-reported Health}} & \mc{1}{c}{\scriptsize{30}} & \mc{1}{c}{\scriptsize{-0.274}} & \mc{1}{c}{\scriptsize{-0.328}} & \mc{1}{c}{\scriptsize{-0.425}} & \mc{1}{c}{\scriptsize{-3.250}} & \mc{1}{c}{\scriptsize{-0.915}} & \mc{1}{c}{\scriptsize{-0.164}} & \mc{1}{c}{\scriptsize{-0.141}} & \mc{1}{c}{\scriptsize{-0.176}} \\  

     &  & \mc{1}{c}{\scriptsize{(0.408)}} & \mc{1}{c}{\scriptsize{(0.487)}} & \mc{1}{c}{\scriptsize{(0.447)}} & \mc{1}{c}{\scriptsize{(0.125)}} & \mc{1}{c}{\scriptsize{(0.329)}} & \mc{1}{c}{\scriptsize{(0.553)}} & \mc{1}{c}{\scriptsize{(0.908)}} & \mc{1}{c}{\scriptsize{(0.671)}} \\  

     & \mc{1}{c}{\scriptsize{Mid-30s}} & \mc{1}{c}{\scriptsize{0.350}} & \mc{1}{c}{\scriptsize{0.589}} & \mc{1}{c}{\scriptsize{0.029}} & \mc{1}{c}{\scriptsize{0.610}} & \mc{1}{c}{\scriptsize{0.209}} & \mc{1}{c}{\scriptsize{0.600}} & \mc{1}{c}{\scriptsize{0.512}} & \mc{1}{c}{\scriptsize{0.545}} \\  

     &  & \mc{1}{c}{\scriptsize{(1.000)}} & \mc{1}{c}{\scriptsize{(1.000)}} & \mc{1}{c}{\scriptsize{(0.842)}} & \mc{1}{c}{\scriptsize{(0.859)}} & \mc{1}{c}{\scriptsize{(0.921)}} & \mc{1}{c}{\scriptsize{(1.000)}} & \mc{1}{c}{\scriptsize{(0.987)}} & \mc{1}{c}{\scriptsize{(1.000)}} \\  

    \mc{1}{l}{\scriptsize{Self-reported Health Factor}} & \mc{1}{c}{\scriptsize{30 to Mid-30s}} & \mc{1}{c}{\scriptsize{0.157}} & \mc{1}{c}{\scriptsize{0.407}} & \mc{1}{c}{\scriptsize{-0.012}} & \mc{1}{c}{\scriptsize{0.111}} & \mc{1}{c}{\scriptsize{0.256}} & \mc{1}{c}{\scriptsize{0.289}} & \mc{1}{c}{\scriptsize{0.160}} & \mc{1}{c}{\scriptsize{0.283}} \\  

     &  & \mc{1}{c}{\scriptsize{(1.000)}} & \mc{1}{c}{\scriptsize{(1.000)}} & \mc{1}{c}{\scriptsize{(0.763)}} & \mc{1}{c}{\scriptsize{(0.781)}} & \mc{1}{c}{\scriptsize{(0.987)}} & \mc{1}{c}{\scriptsize{(1.000)}} & \mc{1}{c}{\scriptsize{(0.961)}} & \mc{1}{c}{\scriptsize{(1.000)}} \\  

  \bottomrule
  \end{tabular}
	\end{table} 

	\begin{table}[H]
     \caption{Treatment Effects on Child Assessment Schedule (CAS), Female Sample}
     \label{table:abccare_rslt_female_cat11_sd}
	  \begin{tabular}{cccccccccc}
  \toprule

    \scriptsize{Variable} & \scriptsize{Age} & \scriptsize{(1)} & \scriptsize{(2)} & \scriptsize{(3)} & \scriptsize{(4)} & \scriptsize{(5)} & \scriptsize{(6)} & \scriptsize{(7)} & \scriptsize{(8)} \\ 
    \midrule  

    \mc{1}{l}{\scriptsize{Denies Any Worries}} & \mc{1}{c}{\scriptsize{12}} & \mc{1}{c}{\scriptsize{-0.110}} & \mc{1}{c}{\scriptsize{-0.144}} & \mc{1}{c}{\scriptsize{-0.058}} & \mc{1}{c}{\scriptsize{-0.081}} & \mc{1}{c}{\scriptsize{-0.048}} & \mc{1}{c}{\scriptsize{-0.133}} & \mc{1}{c}{\scriptsize{-0.178}} & \mc{1}{c}{\scriptsize{-0.131}} \\  

     &  & \mc{1}{c}{\scriptsize{(0.789)}} & \mc{1}{c}{\scriptsize{(0.750)}} & \mc{1}{c}{\scriptsize{(0.934)}} & \mc{1}{c}{\scriptsize{(0.895)}} & \mc{1}{c}{\scriptsize{(0.974)}} & \mc{1}{c}{\scriptsize{(0.750)}} & \mc{1}{c}{\scriptsize{(0.711)}} & \mc{1}{c}{\scriptsize{(0.737)}} \\  

    \mc{1}{l}{\scriptsize{Family Proud of You}} & \mc{1}{c}{\scriptsize{12}} & \mc{1}{c}{\scriptsize{0.123}} & \mc{1}{c}{\scriptsize{0.093}} & \mc{1}{c}{\scriptsize{0.175}} & \mc{1}{c}{\scriptsize{0.127}} & \mc{1}{c}{\scriptsize{0.183}} & \mc{1}{c}{\scriptsize{0.100}} & \mc{1}{c}{\scriptsize{0.075}} & \mc{1}{c}{\scriptsize{0.099}} \\  

     &  & \mc{1}{c}{\scriptsize{(0.895)}} & \mc{1}{c}{\scriptsize{(0.947)}} & \mc{1}{c}{\scriptsize{(0.697)}} & \mc{1}{c}{\scriptsize{(0.868)}} & \mc{1}{c}{\scriptsize{(0.658)}} & \mc{1}{c}{\scriptsize{(0.987)}} & \mc{1}{c}{\scriptsize{(0.974)}} & \mc{1}{c}{\scriptsize{(0.974)}} \\  

    \mc{1}{l}{\scriptsize{Feels Inadequate, Inferior}} & \mc{1}{c}{\scriptsize{12}} & \mc{1}{c}{\scriptsize{-0.161}} & \mc{1}{c}{\scriptsize{-0.142}} & \mc{1}{c}{\scriptsize{-0.242}} & \mc{1}{c}{\scriptsize{-0.303}} & \mc{1}{c}{\scriptsize{-0.221}} & \mc{1}{c}{\scriptsize{-0.067}} & \mc{1}{c}{\scriptsize{-0.098}} & \mc{1}{c}{\scriptsize{-0.055}} \\  

     &  & \mc{1}{c}{\scriptsize{(0.882)}} & \mc{1}{c}{\scriptsize{(0.882)}} & \mc{1}{c}{\scriptsize{(0.908)}} & \mc{1}{c}{\scriptsize{(0.803)}} & \mc{1}{c}{\scriptsize{(0.882)}} & \mc{1}{c}{\scriptsize{(0.987)}} & \mc{1}{c}{\scriptsize{(0.961)}} & \mc{1}{c}{\scriptsize{(0.987)}} \\  

    \mc{1}{l}{\scriptsize{Good Description of Self}} & \mc{1}{c}{\scriptsize{12}} & \mc{1}{c}{\scriptsize{0.031}} & \mc{1}{c}{\scriptsize{-0.089}} & \mc{1}{c}{\scriptsize{-0.108}} & \mc{1}{c}{\scriptsize{-0.378}} & \mc{1}{c}{\scriptsize{-0.222}} & \mc{1}{c}{\scriptsize{0.107}} & \mc{1}{c}{\scriptsize{0.048}} & \mc{1}{c}{\scriptsize{-0.012}} \\  

     &  & \mc{1}{c}{\scriptsize{(1.000)}} & \mc{1}{c}{\scriptsize{(1.000)}} & \mc{1}{c}{\scriptsize{(1.000)}} & \mc{1}{c}{\scriptsize{(1.000)}} & \mc{1}{c}{\scriptsize{(1.000)}} & \mc{1}{c}{\scriptsize{(0.961)}} & \mc{1}{c}{\scriptsize{(1.000)}} & \mc{1}{c}{\scriptsize{(1.000)}} \\  

    \mc{1}{l}{\scriptsize{Ignores Situation}} & \mc{1}{c}{\scriptsize{12}} & \mc{1}{c}{\scriptsize{-0.184}} & \mc{1}{c}{\scriptsize{-0.217}} & \mc{1}{c}{\scriptsize{0.058}} & \mc{1}{c}{\scriptsize{-0.046}} & \mc{1}{c}{\scriptsize{0.048}} & \mc{1}{c}{\scriptsize{-0.247}} & \mc{1}{c}{\scriptsize{-0.285}} & \mc{1}{c}{\scriptsize{-0.243}} \\  

     &  & \mc{1}{c}{\scriptsize{(0.697)}} & \mc{1}{c}{\scriptsize{(0.658)}} & \mc{1}{c}{\scriptsize{(1.000)}} & \mc{1}{c}{\scriptsize{(0.987)}} & \mc{1}{c}{\scriptsize{(1.000)}} & \mc{1}{c}{\scriptsize{(0.408)}} & \mc{1}{c}{\scriptsize{(0.645)}} & \mc{1}{c}{\scriptsize{(0.553)}} \\  

    \mc{1}{l}{\scriptsize{Impulsivity}} & \mc{1}{c}{\scriptsize{12}} & \mc{1}{c}{\scriptsize{-0.006}} & \mc{1}{c}{\scriptsize{0.069}} & \mc{1}{c}{\scriptsize{-0.175}} & \mc{1}{c}{\scriptsize{0.003}} & \mc{1}{c}{\scriptsize{-0.144}} & \mc{1}{c}{\scriptsize{0.020}} & \mc{1}{c}{\scriptsize{0.102}} & \mc{1}{c}{\scriptsize{0.065}} \\  

     &  & \mc{1}{c}{\scriptsize{(1.000)}} & \mc{1}{c}{\scriptsize{(1.000)}} & \mc{1}{c}{\scriptsize{(0.632)}} & \mc{1}{c}{\scriptsize{(1.000)}} & \mc{1}{c}{\scriptsize{(0.750)}} & \mc{1}{c}{\scriptsize{(1.000)}} & \mc{1}{c}{\scriptsize{(1.000)}} & \mc{1}{c}{\scriptsize{(1.000)}} \\  

    \mc{1}{l}{\scriptsize{Not Cope with Problem}} & \mc{1}{c}{\scriptsize{12}} & \mc{1}{c}{\scriptsize{-0.033}} & \mc{1}{c}{\scriptsize{-0.039}} & \mc{1}{c}{\scriptsize{-0.158}} & \mc{1}{c}{\scriptsize{-0.112}} & \mc{1}{c}{\scriptsize{-0.126}} & \mc{1}{c}{\scriptsize{0.027}} & \mc{1}{c}{\scriptsize{0.012}} & \mc{1}{c}{\scriptsize{0.042}} \\  

     &  & \mc{1}{c}{\scriptsize{(1.000)}} & \mc{1}{c}{\scriptsize{(1.000)}} & \mc{1}{c}{\scriptsize{(0.921)}} & \mc{1}{c}{\scriptsize{(0.908)}} & \mc{1}{c}{\scriptsize{(0.947)}} & \mc{1}{c}{\scriptsize{(1.000)}} & \mc{1}{c}{\scriptsize{(1.000)}} & \mc{1}{c}{\scriptsize{(1.000)}} \\  

    \mc{1}{l}{\scriptsize{Often Mad or Angry}} & \mc{1}{c}{\scriptsize{12}} & \mc{1}{c}{\scriptsize{-0.106}} & \mc{1}{c}{\scriptsize{-0.197}} & \mc{1}{c}{\scriptsize{0.174}} & \mc{1}{c}{\scriptsize{0.034}} & \mc{1}{c}{\scriptsize{0.027}} & \mc{1}{c}{\scriptsize{-0.176}} & \mc{1}{c}{\scriptsize{-0.280}} & \mc{1}{c}{\scriptsize{-0.342}} \\  

     &  & \mc{1}{c}{\scriptsize{(0.987)}} & \mc{1}{c}{\scriptsize{(0.658)}} & \mc{1}{c}{\scriptsize{(1.000)}} & \mc{1}{c}{\scriptsize{(1.000)}} & \mc{1}{c}{\scriptsize{(1.000)}} & \mc{1}{c}{\scriptsize{(0.974)}} & \mc{1}{c}{\scriptsize{(0.671)}} & \mc{1}{c}{\scriptsize{(0.408)}} \\  

    \mc{1}{l}{\scriptsize{Participates in Activity}} & \mc{1}{c}{\scriptsize{12}} & \mc{1}{c}{\scriptsize{0.108}} & \mc{1}{c}{\scriptsize{0.143}} & \mc{1}{c}{\scriptsize{0.292}} & \mc{1}{c}{\scriptsize{0.217}} & \mc{1}{c}{\scriptsize{0.278}} & \mc{1}{c}{\scriptsize{0.067}} & \mc{1}{c}{\scriptsize{0.113}} & \mc{1}{c}{\scriptsize{0.029}} \\  

     &  & \mc{1}{c}{\scriptsize{(0.895)}} & \mc{1}{c}{\scriptsize{(0.658)}} & \mc{1}{c}{\scriptsize{(0.395)}} & \mc{1}{c}{\scriptsize{(0.842)}} & \mc{1}{c}{\scriptsize{(0.474)}} & \mc{1}{c}{\scriptsize{(0.987)}} & \mc{1}{c}{\scriptsize{(0.882)}} & \mc{1}{c}{\scriptsize{(1.000)}} \\  

    \mc{1}{l}{\scriptsize{Proud about Self}} & \mc{1}{c}{\scriptsize{12}} & \mc{1}{c}{\scriptsize{0.027}} & \mc{1}{c}{\scriptsize{-0.003}} & \mc{1}{c}{\scriptsize{0.050}} & \mc{1}{c}{\scriptsize{0.026}} & \mc{1}{c}{\scriptsize{0.058}} & \mc{1}{c}{\scriptsize{0.020}} & \mc{1}{c}{\scriptsize{-0.002}} & \mc{1}{c}{\scriptsize{0.016}} \\  

     &  & \mc{1}{c}{\scriptsize{(1.000)}} & \mc{1}{c}{\scriptsize{(1.000)}} & \mc{1}{c}{\scriptsize{(0.974)}} & \mc{1}{c}{\scriptsize{(1.000)}} & \mc{1}{c}{\scriptsize{(0.974)}} & \mc{1}{c}{\scriptsize{(1.000)}} & \mc{1}{c}{\scriptsize{(1.000)}} & \mc{1}{c}{\scriptsize{(1.000)}} \\  

    \mc{1}{l}{\scriptsize{Significant Fears}} & \mc{1}{c}{\scriptsize{12}} & \mc{1}{c}{\scriptsize{-0.178}} & \mc{1}{c}{\scriptsize{-0.232}} & \mc{1}{c}{\scriptsize{-0.017}} & \mc{1}{c}{\scriptsize{0.160}} & \mc{1}{c}{\scriptsize{-0.057}} & \mc{1}{c}{\scriptsize{-0.267}} & \mc{1}{c}{\scriptsize{-0.356}} & \mc{1}{c}{\scriptsize{-0.308}} \\  

     &  & \mc{1}{c}{\scriptsize{(0.461)}} & \mc{1}{c}{\scriptsize{(0.395)}} & \mc{1}{c}{\scriptsize{(0.987)}} & \mc{1}{c}{\scriptsize{(1.000)}} & \mc{1}{c}{\scriptsize{(0.974)}} & \mc{1}{c}{\scriptsize{\textbf{(0.053)}}} & \mc{1}{c}{\scriptsize{\textbf{(0.053)}}} & \mc{1}{c}{\scriptsize{\textbf{(0.013)}}} \\  

    \mc{1}{l}{\scriptsize{Time spent reading}} & \mc{1}{c}{\scriptsize{12}} & \mc{1}{c}{\scriptsize{4.063}} & \mc{1}{c}{\scriptsize{4.024}} & \mc{1}{c}{\scriptsize{3.333}} & \mc{1}{c}{\scriptsize{2.840}} & \mc{1}{c}{\scriptsize{3.872}} & \mc{1}{c}{\scriptsize{4.227}} & \mc{1}{c}{\scriptsize{4.398}} & \mc{1}{c}{\scriptsize{4.700}} \\  

     &  & \mc{1}{c}{\scriptsize{(0.211)}} & \mc{1}{c}{\scriptsize{(0.447)}} & \mc{1}{c}{\scriptsize{(0.526)}} & \mc{1}{c}{\scriptsize{(0.855)}} & \mc{1}{c}{\scriptsize{(0.500)}} & \mc{1}{c}{\scriptsize{(0.171)}} & \mc{1}{c}{\scriptsize{(0.500)}} & \mc{1}{c}{\scriptsize{(0.237)}} \\  

    \mc{1}{l}{\scriptsize{Views Self as Clumsy}} & \mc{1}{c}{\scriptsize{12}} & \mc{1}{c}{\scriptsize{0.072}} & \mc{1}{c}{\scriptsize{0.023}} & \mc{1}{c}{\scriptsize{0.242}} & \mc{1}{c}{\scriptsize{0.190}} & \mc{1}{c}{\scriptsize{0.221}} & \mc{1}{c}{\scriptsize{0.007}} & \mc{1}{c}{\scriptsize{-0.038}} & \mc{1}{c}{\scriptsize{-0.030}} \\  

     &  & \mc{1}{c}{\scriptsize{(1.000)}} & \mc{1}{c}{\scriptsize{(1.000)}} & \mc{1}{c}{\scriptsize{(1.000)}} & \mc{1}{c}{\scriptsize{(1.000)}} & \mc{1}{c}{\scriptsize{(1.000)}} & \mc{1}{c}{\scriptsize{(1.000)}} & \mc{1}{c}{\scriptsize{(1.000)}} & \mc{1}{c}{\scriptsize{(1.000)}} \\  

    \mc{1}{l}{\scriptsize{Views Self as Dumb}} & \mc{1}{c}{\scriptsize{12}} & \mc{1}{c}{\scriptsize{0.106}} & \mc{1}{c}{\scriptsize{0.058}} & \mc{1}{c}{\scriptsize{0.025}} & \mc{1}{c}{\scriptsize{-0.073}} & \mc{1}{c}{\scriptsize{0.009}} & \mc{1}{c}{\scriptsize{0.160}} & \mc{1}{c}{\scriptsize{0.086}} & \mc{1}{c}{\scriptsize{0.135}} \\  

     &  & \mc{1}{c}{\scriptsize{(1.000)}} & \mc{1}{c}{\scriptsize{(1.000)}} & \mc{1}{c}{\scriptsize{(1.000)}} & \mc{1}{c}{\scriptsize{(0.974)}} & \mc{1}{c}{\scriptsize{(1.000)}} & \mc{1}{c}{\scriptsize{(1.000)}} & \mc{1}{c}{\scriptsize{(1.000)}} & \mc{1}{c}{\scriptsize{(1.000)}} \\  

    \mc{1}{l}{\scriptsize{Views Self as Not Liked}} & \mc{1}{c}{\scriptsize{12}} & \mc{1}{c}{\scriptsize{-0.165}} & \mc{1}{c}{\scriptsize{-0.192}} & \mc{1}{c}{\scriptsize{-0.275}} & \mc{1}{c}{\scriptsize{-0.254}} & \mc{1}{c}{\scriptsize{-0.298}} & \mc{1}{c}{\scriptsize{-0.140}} & \mc{1}{c}{\scriptsize{-0.172}} & \mc{1}{c}{\scriptsize{-0.174}} \\  

     &  & \mc{1}{c}{\scriptsize{(0.618)}} & \mc{1}{c}{\scriptsize{(0.526)}} & \mc{1}{c}{\scriptsize{(0.829)}} & \mc{1}{c}{\scriptsize{(0.763)}} & \mc{1}{c}{\scriptsize{(0.697)}} & \mc{1}{c}{\scriptsize{(0.750)}} & \mc{1}{c}{\scriptsize{(0.776)}} & \mc{1}{c}{\scriptsize{(0.553)}} \\  

    \mc{1}{l}{\scriptsize{Withdraws Excessively}} & \mc{1}{c}{\scriptsize{12}} & \mc{1}{c}{\scriptsize{-0.141}} & \mc{1}{c}{\scriptsize{-0.140}} & \mc{1}{c}{\scriptsize{-0.200}} & \mc{1}{c}{\scriptsize{-0.102}} & \mc{1}{c}{\scriptsize{-0.154}} & \mc{1}{c}{\scriptsize{-0.100}} & \mc{1}{c}{\scriptsize{-0.148}} & \mc{1}{c}{\scriptsize{-0.029}} \\  

     &  & \mc{1}{c}{\scriptsize{(0.921)}} & \mc{1}{c}{\scriptsize{(0.921)}} & \mc{1}{c}{\scriptsize{(0.921)}} & \mc{1}{c}{\scriptsize{(0.974)}} & \mc{1}{c}{\scriptsize{(0.974)}} & \mc{1}{c}{\scriptsize{(0.987)}} & \mc{1}{c}{\scriptsize{(0.882)}} & \mc{1}{c}{\scriptsize{(1.000)}} \\  

  \bottomrule
  \end{tabular}
	\end{table} 

	\begin{table}[H]
     \caption{Treatment Effects on Mother's Income, Female Sample}
     \label{table:abccare_rslt_female_cat12_sd}
	  \begin{tabular}{cccccccccc}
  \toprule

    \scriptsize{Variable} & \scriptsize{Age} & \scriptsize{(1)} & \scriptsize{(2)} & \scriptsize{(3)} & \scriptsize{(4)} & \scriptsize{(5)} & \scriptsize{(6)} & \scriptsize{(7)} & \scriptsize{(8)} \\ 
    \midrule  

    \mc{1}{l}{\scriptsize{Mother's Earned Inc.}} & \mc{1}{c}{\scriptsize{0}} & \mc{1}{c}{\scriptsize{3,474}} & \mc{1}{c}{\scriptsize{-2,270}} & \mc{1}{c}{\scriptsize{3,464}} & \mc{1}{c}{\scriptsize{11,119}} & \mc{1}{c}{\scriptsize{3,426}} & \mc{1}{c}{\scriptsize{3,496}} & \mc{1}{c}{\scriptsize{-20,659}} & \mc{1}{c}{\scriptsize{3,444}} \\  

     &  & \mc{1}{c}{\scriptsize{(0.276)}} & \mc{1}{c}{\scriptsize{(0.724)}} & \mc{1}{c}{\scriptsize{(0.250)}} & \mc{1}{c}{\scriptsize{(0.260)}} & \mc{1}{c}{\scriptsize{(0.421)}} & \mc{1}{c}{\scriptsize{(0.312)}} & \mc{1}{c}{\scriptsize{(0.984)}} & \mc{1}{c}{\scriptsize{(0.391)}} \\  

    \mc{1}{l}{\scriptsize{Mother's Public-Transfer Inc.}} & \mc{1}{c}{\scriptsize{21}} & \mc{1}{c}{\scriptsize{-3,826}} & \mc{1}{c}{\scriptsize{-6,693}} & \mc{1}{c}{\scriptsize{-9,561}} & \mc{1}{c}{\scriptsize{-7,179}} & \mc{1}{c}{\scriptsize{-10,253}} & \mc{1}{c}{\scriptsize{-1,676}} & \mc{1}{c}{\scriptsize{-2,532}} & \mc{1}{c}{\scriptsize{-1,344}} \\  

     &  & \mc{1}{c}{\scriptsize{(0.987)}} & \mc{1}{c}{\scriptsize{(1.000)}} & \mc{1}{c}{\scriptsize{(0.987)}} & \mc{1}{c}{\scriptsize{(0.890)}} & \mc{1}{c}{\scriptsize{(0.987)}} & \mc{1}{c}{\scriptsize{(0.891)}} & \mc{1}{c}{\scriptsize{(0.734)}} & \mc{1}{c}{\scriptsize{(0.844)}} \\  

  \bottomrule
  \end{tabular}
	\end{table} 

	\begin{table}[H]
     \caption{Treatment Effects on Parental Labor Income, Female Sample}
     \label{table:abccare_rslt_female_cat13_sd}
	  \begin{tabular}{cccccccccc}
  \toprule

    \scriptsize{Variable} & \scriptsize{Age} & \scriptsize{(1)} & \scriptsize{(2)} & \scriptsize{(3)} & \scriptsize{(4)} & \scriptsize{(5)} & \scriptsize{(6)} & \scriptsize{(7)} & \scriptsize{(8)} \\ 
    \midrule  

    \mc{1}{l}{\scriptsize{Hemoglobin Level (\%)}} & \mc{1}{c}{\scriptsize{Mid-30s}} & \mc{1}{c}{\scriptsize{-0.277}} & \mc{1}{c}{\scriptsize{-0.101}} & \mc{1}{c}{\scriptsize{-0.176}} & \mc{1}{c}{\scriptsize{-0.088}} & \mc{1}{c}{\scriptsize{-0.190}} & \mc{1}{c}{\scriptsize{-0.304}} & \mc{1}{c}{\scriptsize{-0.037}} & \mc{1}{c}{\scriptsize{-0.355}} \\  

     &  & \mc{1}{c}{\scriptsize{(0.368)}} & \mc{1}{c}{\scriptsize{(0.513)}} & \mc{1}{c}{\scriptsize{(0.184)}} & \mc{1}{c}{\scriptsize{(0.447)}} & \mc{1}{c}{\scriptsize{(0.197)}} & \mc{1}{c}{\scriptsize{(0.421)}} & \mc{1}{c}{\scriptsize{(0.711)}} & \mc{1}{c}{\scriptsize{(0.355)}} \\  

    \mc{1}{l}{\scriptsize{Prediabetes}} & \mc{1}{c}{\scriptsize{Mid-30s}} & \mc{1}{c}{\scriptsize{0.088}} & \mc{1}{c}{\scriptsize{0.163}} & \mc{1}{c}{\scriptsize{0.076}} & \mc{1}{c}{\scriptsize{0.176}} & \mc{1}{c}{\scriptsize{0.035}} & \mc{1}{c}{\scriptsize{0.091}} & \mc{1}{c}{\scriptsize{0.161}} & \mc{1}{c}{\scriptsize{0.029}} \\  

     &  & \mc{1}{c}{\scriptsize{(0.895)}} & \mc{1}{c}{\scriptsize{(0.961)}} & \mc{1}{c}{\scriptsize{(0.724)}} & \mc{1}{c}{\scriptsize{(0.816)}} & \mc{1}{c}{\scriptsize{(0.645)}} & \mc{1}{c}{\scriptsize{(0.895)}} & \mc{1}{c}{\scriptsize{(0.947)}} & \mc{1}{c}{\scriptsize{(0.750)}} \\  

    \mc{1}{l}{\scriptsize{Diabetes}} & \mc{1}{c}{\scriptsize{Mid-30s}} & \mc{1}{c}{\scriptsize{-0.071}} & \mc{1}{c}{\scriptsize{-0.032}} &  &  &  & \mc{1}{c}{\scriptsize{-0.091}} & \mc{1}{c}{\scriptsize{-0.039}} & \mc{1}{c}{\scriptsize{-0.095}} \\  

     &  & \mc{1}{c}{\scriptsize{(0.145)}} & \mc{1}{c}{\scriptsize{(0.342)}} &  &  &  & \mc{1}{c}{\scriptsize{(0.184)}} & \mc{1}{c}{\scriptsize{(0.368)}} & \mc{1}{c}{\scriptsize{(0.118)}} \\  

    \mc{1}{l}{\scriptsize{Diabetes Factor}} & \mc{1}{c}{\scriptsize{Mid-30s}} & \mc{1}{c}{\scriptsize{-0.249}} & \mc{1}{c}{\scriptsize{-0.086}} & \mc{1}{c}{\scriptsize{-0.086}} & \mc{1}{c}{\scriptsize{-0.029}} & \mc{1}{c}{\scriptsize{-0.098}} & \mc{1}{c}{\scriptsize{-0.294}} & \mc{1}{c}{\scriptsize{-0.064}} & \mc{1}{c}{\scriptsize{-0.334}} \\  

     &  & \mc{1}{c}{\scriptsize{(0.237)}} & \mc{1}{c}{\scriptsize{(0.474)}} & \mc{1}{c}{\scriptsize{(0.408)}} & \mc{1}{c}{\scriptsize{(0.513)}} & \mc{1}{c}{\scriptsize{(0.316)}} & \mc{1}{c}{\scriptsize{(0.276)}} & \mc{1}{c}{\scriptsize{(0.618)}} & \mc{1}{c}{\scriptsize{(0.224)}} \\  

  \bottomrule
  \end{tabular}
	\end{table} 

	\begin{table}[H]
     \caption{Treatment Effects on Parental Public Transfer Income, Female Sample}
     \label{table:abccare_rslt_female_cat14_sd}
	  \begin{tabular}{cccccccccc}
  \toprule

    \scriptsize{Variable} & \scriptsize{Age} & \scriptsize{(1)} & \scriptsize{(2)} & \scriptsize{(3)} & \scriptsize{(4)} & \scriptsize{(5)} & \scriptsize{(6)} & \scriptsize{(7)} & \scriptsize{(8)} \\ 
    \midrule  

    \mc{1}{l}{\scriptsize{Vitamin D Deficiency}} & \mc{1}{c}{\scriptsize{Mid-30s}} & \mc{1}{c}{\scriptsize{-0.050}} & \mc{1}{c}{\scriptsize{0.029}} & \mc{1}{c}{\scriptsize{-0.061}} & \mc{1}{c}{\scriptsize{0.112}} & \mc{1}{c}{\scriptsize{-0.026}} & \mc{1}{c}{\scriptsize{-0.047}} & \mc{1}{c}{\scriptsize{0.004}} & \mc{1}{c}{\scriptsize{-0.033}} \\  

     &  & \mc{1}{c}{\scriptsize{(0.329)}} & \mc{1}{c}{\scriptsize{(0.553)}} & \mc{1}{c}{\scriptsize{(0.355)}} & \mc{1}{c}{\scriptsize{(0.645)}} & \mc{1}{c}{\scriptsize{(0.487)}} & \mc{1}{c}{\scriptsize{(0.355)}} & \mc{1}{c}{\scriptsize{(0.526)}} & \mc{1}{c}{\scriptsize{(0.395)}} \\  

  \bottomrule
  \end{tabular}
	\end{table} 

	\begin{table}[H]
     \caption{Treatment Effects on Adoption, Female Sample}
     \label{table:abccare_rslt_female_cat15_sd}
	  \begin{tabular}{cccccccccc}
  \toprule

    \scriptsize{Variable} & \scriptsize{Age} & \scriptsize{(1)} & \scriptsize{(2)} & \scriptsize{(3)} & \scriptsize{(4)} & \scriptsize{(5)} & \scriptsize{(6)} & \scriptsize{(7)} & \scriptsize{(8)} \\ 
    \midrule  

    \mc{1}{l}{\scriptsize{Measured BMI}} & \mc{1}{c}{\scriptsize{Mid-30s}} & \mc{1}{c}{\scriptsize{4.357}} & \mc{1}{c}{\scriptsize{8.665}} & \mc{1}{c}{\scriptsize{3.292}} & \mc{1}{c}{\scriptsize{3.623}} & \mc{1}{c}{\scriptsize{3.982}} & \mc{1}{c}{\scriptsize{4.737}} & \mc{1}{c}{\scriptsize{10.099}} & \mc{1}{c}{\scriptsize{6.770}} \\  

     &  & \mc{1}{c}{\scriptsize{(1.000)}} & \mc{1}{c}{\scriptsize{(1.000)}} & \mc{1}{c}{\scriptsize{(0.961)}} & \mc{1}{c}{\scriptsize{(0.961)}} & \mc{1}{c}{\scriptsize{(1.000)}} & \mc{1}{c}{\scriptsize{(1.000)}} & \mc{1}{c}{\scriptsize{(1.000)}} & \mc{1}{c}{\scriptsize{(1.000)}} \\  

    \mc{1}{l}{\scriptsize{Obesity}} & \mc{1}{c}{\scriptsize{Mid-30s}} & \mc{1}{c}{\scriptsize{0.055}} & \mc{1}{c}{\scriptsize{0.269}} & \mc{1}{c}{\scriptsize{-0.061}} & \mc{1}{c}{\scriptsize{-0.120}} & \mc{1}{c}{\scriptsize{-0.027}} & \mc{1}{c}{\scriptsize{0.096}} & \mc{1}{c}{\scriptsize{0.358}} & \mc{1}{c}{\scriptsize{0.200}} \\  

     &  & \mc{1}{c}{\scriptsize{(0.961)}} & \mc{1}{c}{\scriptsize{(1.000)}} & \mc{1}{c}{\scriptsize{(0.724)}} & \mc{1}{c}{\scriptsize{(0.592)}} & \mc{1}{c}{\scriptsize{(0.803)}} & \mc{1}{c}{\scriptsize{(0.987)}} & \mc{1}{c}{\scriptsize{(1.000)}} & \mc{1}{c}{\scriptsize{(1.000)}} \\  

    \mc{1}{l}{\scriptsize{Severe Obesity}} & \mc{1}{c}{\scriptsize{Mid-30s}} & \mc{1}{c}{\scriptsize{-0.021}} & \mc{1}{c}{\scriptsize{0.213}} & \mc{1}{c}{\scriptsize{-0.052}} & \mc{1}{c}{\scriptsize{0.170}} & \mc{1}{c}{\scriptsize{-0.003}} & \mc{1}{c}{\scriptsize{-0.009}} & \mc{1}{c}{\scriptsize{0.265}} & \mc{1}{c}{\scriptsize{0.110}} \\  

     &  & \mc{1}{c}{\scriptsize{(0.789)}} & \mc{1}{c}{\scriptsize{(1.000)}} & \mc{1}{c}{\scriptsize{(0.724)}} & \mc{1}{c}{\scriptsize{(0.961)}} & \mc{1}{c}{\scriptsize{(0.842)}} & \mc{1}{c}{\scriptsize{(0.908)}} & \mc{1}{c}{\scriptsize{(1.000)}} & \mc{1}{c}{\scriptsize{(1.000)}} \\  

    \mc{1}{l}{\scriptsize{Waist-hip Ratio}} & \mc{1}{c}{\scriptsize{Mid-30s}} & \mc{1}{c}{\scriptsize{-0.017}} & \mc{1}{c}{\scriptsize{0.037}} & \mc{1}{c}{\scriptsize{-0.054}} & \mc{1}{c}{\scriptsize{-0.060}} & \mc{1}{c}{\scriptsize{-0.046}} & \mc{1}{c}{\scriptsize{-0.003}} & \mc{1}{c}{\scriptsize{0.065}} & \mc{1}{c}{\scriptsize{0.026}} \\  

     &  & \mc{1}{c}{\scriptsize{(0.671)}} & \mc{1}{c}{\scriptsize{(0.961)}} & \mc{1}{c}{\scriptsize{(0.447)}} & \mc{1}{c}{\scriptsize{(0.592)}} & \mc{1}{c}{\scriptsize{(0.539)}} & \mc{1}{c}{\scriptsize{(0.908)}} & \mc{1}{c}{\scriptsize{(1.000)}} & \mc{1}{c}{\scriptsize{(1.000)}} \\  

    \mc{1}{l}{\scriptsize{Abdominal Obesity}} & \mc{1}{c}{\scriptsize{Mid-30s}} & \mc{1}{c}{\scriptsize{-0.138}} & \mc{1}{c}{\scriptsize{0.066}} & \mc{1}{c}{\scriptsize{-0.181}} & \mc{1}{c}{\scriptsize{-0.214}} & \mc{1}{c}{\scriptsize{-0.146}} & \mc{1}{c}{\scriptsize{-0.122}} & \mc{1}{c}{\scriptsize{0.099}} & \mc{1}{c}{\scriptsize{0.021}} \\  

     &  & \mc{1}{c}{\scriptsize{(0.500)}} & \mc{1}{c}{\scriptsize{(0.947)}} & \mc{1}{c}{\scriptsize{(0.539)}} & \mc{1}{c}{\scriptsize{(0.447)}} & \mc{1}{c}{\scriptsize{(0.579)}} & \mc{1}{c}{\scriptsize{(0.618)}} & \mc{1}{c}{\scriptsize{(0.987)}} & \mc{1}{c}{\scriptsize{(0.974)}} \\  

    \mc{1}{l}{\scriptsize{Framingham Risk Score}} & \mc{1}{c}{\scriptsize{Mid-30s}} & \mc{1}{c}{\scriptsize{-0.100}} & \mc{1}{c}{\scriptsize{-0.206}} & \mc{1}{c}{\scriptsize{-0.240}} & \mc{1}{c}{\scriptsize{-0.676}} & \mc{1}{c}{\scriptsize{-0.377}} & \mc{1}{c}{\scriptsize{-0.050}} & \mc{1}{c}{\scriptsize{0.005}} & \mc{1}{c}{\scriptsize{-0.050}} \\  

     &  & \mc{1}{c}{\scriptsize{(0.671)}} & \mc{1}{c}{\scriptsize{(0.526)}} & \mc{1}{c}{\scriptsize{(0.566)}} & \mc{1}{c}{\scriptsize{(0.395)}} & \mc{1}{c}{\scriptsize{(0.421)}} & \mc{1}{c}{\scriptsize{(0.803)}} & \mc{1}{c}{\scriptsize{(0.908)}} & \mc{1}{c}{\scriptsize{(0.855)}} \\  

    \mc{1}{l}{\scriptsize{Obesity Factor}} & \mc{1}{c}{\scriptsize{Mid-30s}} & \mc{1}{c}{\scriptsize{0.175}} & \mc{1}{c}{\scriptsize{0.629}} & \mc{1}{c}{\scriptsize{-0.050}} & \mc{1}{c}{\scriptsize{-0.060}} & \mc{1}{c}{\scriptsize{-0.080}} & \mc{1}{c}{\scriptsize{0.259}} & \mc{1}{c}{\scriptsize{0.839}} & \mc{1}{c}{\scriptsize{0.474}} \\  

     &  & \mc{1}{c}{\scriptsize{(0.987)}} & \mc{1}{c}{\scriptsize{(1.000)}} & \mc{1}{c}{\scriptsize{(0.776)}} & \mc{1}{c}{\scriptsize{(0.829)}} & \mc{1}{c}{\scriptsize{(0.776)}} & \mc{1}{c}{\scriptsize{(1.000)}} & \mc{1}{c}{\scriptsize{(1.000)}} & \mc{1}{c}{\scriptsize{(1.000)}} \\  

  \bottomrule
  \end{tabular}
	\end{table} 

	\begin{table}[H]
     \caption{Treatment Effects on Childhood Household Income, Female Sample}
     \label{table:abccare_rslt_female_cat16_sd}
	  \begin{tabular}{cccccccccc}
  \toprule

    \scriptsize{Variable} & \scriptsize{Age} & \scriptsize{(1)} & \scriptsize{(2)} & \scriptsize{(3)} & \scriptsize{(4)} & \scriptsize{(5)} & \scriptsize{(6)} & \scriptsize{(7)} & \scriptsize{(8)} \\ 
    \midrule  

    \mc{1}{l}{\scriptsize{No trouble with spouse family}} & \mc{1}{c}{\scriptsize{30}} & \mc{1}{c}{\scriptsize{0.150}} & \mc{1}{c}{\scriptsize{0.107}} & \mc{1}{c}{\scriptsize{0.233}} & \mc{1}{c}{\scriptsize{0.624}} & \mc{1}{c}{\scriptsize{0.206}} & \mc{1}{c}{\scriptsize{0.133}} & \mc{1}{c}{\scriptsize{0.040}} & \mc{1}{c}{\scriptsize{-0.050}} \\  

     &  & \mc{1}{c}{\scriptsize{(0.447)}} & \mc{1}{c}{\scriptsize{(0.763)}} & \mc{1}{c}{\scriptsize{(0.667)}} & \mc{1}{c}{\scriptsize{\textbf{(0.078)}}} & \mc{1}{c}{\scriptsize{(0.697)}} & \mc{1}{c}{\scriptsize{(0.474)}} & \mc{1}{c}{\scriptsize{(0.789)}} & \mc{1}{c}{\scriptsize{(0.934)}} \\  

    \mc{1}{l}{\scriptsize{Get along well with spouse}} & \mc{1}{c}{\scriptsize{30}} & \mc{1}{c}{\scriptsize{-0.050}} & \mc{1}{c}{\scriptsize{-0.080}} & \mc{1}{c}{\scriptsize{0.033}} & \mc{1}{c}{\scriptsize{0.097}} & \mc{1}{c}{\scriptsize{0.078}} & \mc{1}{c}{\scriptsize{-0.067}} & \mc{1}{c}{\scriptsize{-0.131}} & \mc{1}{c}{\scriptsize{-0.033}} \\  

     &  & \mc{1}{c}{\scriptsize{(0.961)}} & \mc{1}{c}{\scriptsize{(0.908)}} & \mc{1}{c}{\scriptsize{(0.909)}} & \mc{1}{c}{\scriptsize{(0.969)}} & \mc{1}{c}{\scriptsize{(0.864)}} & \mc{1}{c}{\scriptsize{(0.934)}} & \mc{1}{c}{\scriptsize{(0.961)}} & \mc{1}{c}{\scriptsize{(0.868)}} \\  

    \mc{1}{l}{\scriptsize{No disagreement on living arrangement}} & \mc{1}{c}{\scriptsize{30}} & \mc{1}{c}{\scriptsize{0.217}} & \mc{1}{c}{\scriptsize{-0.136}} & \mc{1}{c}{\scriptsize{0.300}} & \mc{1}{c}{\scriptsize{0.238}} & \mc{1}{c}{\scriptsize{0.288}} & \mc{1}{c}{\scriptsize{0.200}} & \mc{1}{c}{\scriptsize{-0.255}} & \mc{1}{c}{\scriptsize{0.004}} \\  

     &  & \mc{1}{c}{\scriptsize{(0.329)}} & \mc{1}{c}{\scriptsize{(0.934)}} & \mc{1}{c}{\scriptsize{(0.621)}} & \mc{1}{c}{\scriptsize{(0.969)}} & \mc{1}{c}{\scriptsize{(0.591)}} & \mc{1}{c}{\scriptsize{(0.382)}} & \mc{1}{c}{\scriptsize{(1.000)}} & \mc{1}{c}{\scriptsize{(0.816)}} \\  

  \bottomrule
  \end{tabular}
	\end{table} 

	\begin{table}[H]
     \caption{Treatment Effects on Father at Home, Female Sample}
     \label{table:abccare_rslt_female_cat17_sd}
	  \begin{tabular}{cccccccccc}
  \toprule

    \scriptsize{Variable} & \scriptsize{Age} & \scriptsize{(1)} & \scriptsize{(2)} & \scriptsize{(3)} & \scriptsize{(4)} & \scriptsize{(5)} & \scriptsize{(6)} & \scriptsize{(7)} & \scriptsize{(8)} \\ 
    \midrule  

    \mc{1}{l}{\scriptsize{Measured BMI}} & \mc{1}{c}{\scriptsize{Mid-30s}} & \mc{1}{c}{\scriptsize{1.785}} & \mc{1}{c}{\scriptsize{5.941}} & \mc{1}{c}{\scriptsize{-0.577}} & \mc{1}{c}{\scriptsize{-8.894}} & \mc{1}{c}{\scriptsize{1.354}} & \mc{1}{c}{\scriptsize{2.158}} & \mc{1}{c}{\scriptsize{9.068}} & \mc{1}{c}{\scriptsize{5.631}} \\  

     &  & \mc{1}{c}{\scriptsize{(0.961)}} & \mc{1}{c}{\scriptsize{(1.000)}} & \mc{1}{c}{\scriptsize{(0.514)}} & \mc{1}{c}{\scriptsize{(0.568)}} & \mc{1}{c}{\scriptsize{(0.986)}} & \mc{1}{c}{\scriptsize{(0.974)}} & \mc{1}{c}{\scriptsize{(1.000)}} & \mc{1}{c}{\scriptsize{(1.000)}} \\  

    \mc{1}{l}{\scriptsize{Obesity}} & \mc{1}{c}{\scriptsize{Mid-30s}} & \mc{1}{c}{\scriptsize{-0.061}} & \mc{1}{c}{\scriptsize{0.158}} & \mc{1}{c}{\scriptsize{-0.333}} & \mc{1}{c}{\scriptsize{-0.155}} & \mc{1}{c}{\scriptsize{-0.253}} & \mc{1}{c}{\scriptsize{-0.018}} & \mc{1}{c}{\scriptsize{0.245}} & \mc{1}{c}{\scriptsize{0.095}} \\  

     &  & \mc{1}{c}{\scriptsize{(0.763)}} & \mc{1}{c}{\scriptsize{(1.000)}} & \mc{1}{c}{\scriptsize{\textbf{(0.014)}}} & \mc{1}{c}{\scriptsize{(0.716)}} & \mc{1}{c}{\scriptsize{(0.189)}} & \mc{1}{c}{\scriptsize{(0.868)}} & \mc{1}{c}{\scriptsize{(1.000)}} & \mc{1}{c}{\scriptsize{(0.974)}} \\  

    \mc{1}{l}{\scriptsize{Severe Obesity}} & \mc{1}{c}{\scriptsize{Mid-30s}} & \mc{1}{c}{\scriptsize{-0.141}} & \mc{1}{c}{\scriptsize{-0.008}} & \mc{1}{c}{\scriptsize{-0.111}} & \mc{1}{c}{\scriptsize{-0.436}} & \mc{1}{c}{\scriptsize{-0.035}} & \mc{1}{c}{\scriptsize{-0.146}} & \mc{1}{c}{\scriptsize{0.089}} & \mc{1}{c}{\scriptsize{-0.051}} \\  

     &  & \mc{1}{c}{\scriptsize{(0.513)}} & \mc{1}{c}{\scriptsize{(0.868)}} & \mc{1}{c}{\scriptsize{(0.432)}} & \mc{1}{c}{\scriptsize{(0.473)}} & \mc{1}{c}{\scriptsize{(0.905)}} & \mc{1}{c}{\scriptsize{(0.553)}} & \mc{1}{c}{\scriptsize{(0.961)}} & \mc{1}{c}{\scriptsize{(0.829)}} \\  

    \mc{1}{l}{\scriptsize{Waist-hip Ratio}} & \mc{1}{c}{\scriptsize{Mid-30s}} & \mc{1}{c}{\scriptsize{-0.057}} & \mc{1}{c}{\scriptsize{-0.053}} & \mc{1}{c}{\scriptsize{-0.162}} & \mc{1}{c}{\scriptsize{-0.179}} & \mc{1}{c}{\scriptsize{-0.162}} & \mc{1}{c}{\scriptsize{-0.039}} & \mc{1}{c}{\scriptsize{-0.034}} & \mc{1}{c}{\scriptsize{-0.034}} \\  

     &  & \mc{1}{c}{\scriptsize{(0.132)}} & \mc{1}{c}{\scriptsize{(0.368)}} & \mc{1}{c}{\scriptsize{\textbf{(0.068)}}} & \mc{1}{c}{\scriptsize{(0.284)}} & \mc{1}{c}{\scriptsize{(0.162)}} & \mc{1}{c}{\scriptsize{(0.474)}} & \mc{1}{c}{\scriptsize{(0.618)}} & \mc{1}{c}{\scriptsize{(0.592)}} \\  

    \mc{1}{l}{\scriptsize{Abdominal Obesity}} & \mc{1}{c}{\scriptsize{Mid-30s}} & \mc{1}{c}{\scriptsize{-0.199}} & \mc{1}{c}{\scriptsize{-0.079}} & \mc{1}{c}{\scriptsize{-0.438}} & \mc{1}{c}{\scriptsize{-0.042}} & \mc{1}{c}{\scriptsize{-0.377}} & \mc{1}{c}{\scriptsize{-0.160}} & \mc{1}{c}{\scriptsize{-0.087}} & \mc{1}{c}{\scriptsize{-0.107}} \\  

     &  & \mc{1}{c}{\scriptsize{(0.487)}} & \mc{1}{c}{\scriptsize{(0.776)}} & \mc{1}{c}{\scriptsize{\textbf{(0.014)}}} & \mc{1}{c}{\scriptsize{(0.838)}} & \mc{1}{c}{\scriptsize{(0.108)}} & \mc{1}{c}{\scriptsize{(0.605)}} & \mc{1}{c}{\scriptsize{(0.803)}} & \mc{1}{c}{\scriptsize{(0.737)}} \\  

    \mc{1}{l}{\scriptsize{Framingham Risk Score}} & \mc{1}{c}{\scriptsize{Mid-30s}} & \mc{1}{c}{\scriptsize{-0.471}} & \mc{1}{c}{\scriptsize{-0.725}} & \mc{1}{c}{\scriptsize{-1.419}} & \mc{1}{c}{\scriptsize{-1.555}} & \mc{1}{c}{\scriptsize{-1.484}} & \mc{1}{c}{\scriptsize{-0.322}} & \mc{1}{c}{\scriptsize{-0.517}} & \mc{1}{c}{\scriptsize{-0.409}} \\  

     &  & \mc{1}{c}{\scriptsize{(0.105)}} & \mc{1}{c}{\scriptsize{(0.184)}} & \mc{1}{c}{\scriptsize{\textbf{(0.000)}}} & \mc{1}{c}{\scriptsize{\textbf{(0.041)}}} & \mc{1}{c}{\scriptsize{\textbf{(0.000)}}} & \mc{1}{c}{\scriptsize{(0.461)}} & \mc{1}{c}{\scriptsize{(0.487)}} & \mc{1}{c}{\scriptsize{(0.395)}} \\  

    \mc{1}{l}{\scriptsize{Obesity Factor}} & \mc{1}{c}{\scriptsize{Mid-30s}} & \mc{1}{c}{\scriptsize{-0.185}} & \mc{1}{c}{\scriptsize{-0.006}} & \mc{1}{c}{\scriptsize{-0.779}} & \mc{1}{c}{\scriptsize{-1.315}} & \mc{1}{c}{\scriptsize{-0.660}} & \mc{1}{c}{\scriptsize{-0.086}} & \mc{1}{c}{\scriptsize{0.267}} & \mc{1}{c}{\scriptsize{0.161}} \\  

     &  & \mc{1}{c}{\scriptsize{(0.737)}} & \mc{1}{c}{\scriptsize{(0.895)}} & \mc{1}{c}{\scriptsize{\textbf{(0.095)}}} & \mc{1}{c}{\scriptsize{(0.405)}} & \mc{1}{c}{\scriptsize{(0.351)}} & \mc{1}{c}{\scriptsize{(0.855)}} & \mc{1}{c}{\scriptsize{(0.987)}} & \mc{1}{c}{\scriptsize{(0.934)}} \\  

  \bottomrule
  \end{tabular}
	\end{table} 

	\begin{table}[H]
     \caption{Treatment Effects on HOME Scores, Female Sample}
     \label{table:abccare_rslt_female_cat18_sd}
	  \begin{tabular}{cccccccccc}
  \toprule

    \scriptsize{Variable} & \scriptsize{Age} & \scriptsize{(1)} & \scriptsize{(2)} & \scriptsize{(3)} & \scriptsize{(4)} & \scriptsize{(5)} & \scriptsize{(6)} & \scriptsize{(7)} & \scriptsize{(8)} \\ 
    \midrule  

    \mc{1}{l}{\scriptsize{Room density (room/people)}} & \mc{1}{c}{\scriptsize{30}} & \mc{1}{c}{\scriptsize{0.130}} & \mc{1}{c}{\scriptsize{0.255}} & \mc{1}{c}{\scriptsize{0.286}} & \mc{1}{c}{\scriptsize{0.476}} & \mc{1}{c}{\scriptsize{0.340}} & \mc{1}{c}{\scriptsize{0.120}} & \mc{1}{c}{\scriptsize{0.150}} & \mc{1}{c}{\scriptsize{0.162}} \\  

     &  & \mc{1}{c}{\scriptsize{(0.724)}} & \mc{1}{c}{\scriptsize{(0.632)}} & \mc{1}{c}{\scriptsize{(0.605)}} & \mc{1}{c}{\scriptsize{(0.566)}} & \mc{1}{c}{\scriptsize{(0.605)}} & \mc{1}{c}{\scriptsize{(0.816)}} & \mc{1}{c}{\scriptsize{(0.776)}} & \mc{1}{c}{\scriptsize{(0.750)}} \\  

    \mc{1}{l}{\scriptsize{Own computers}} & \mc{1}{c}{\scriptsize{30}} & \mc{1}{c}{\scriptsize{0.058}} & \mc{1}{c}{\scriptsize{0.009}} & \mc{1}{c}{\scriptsize{0.108}} & \mc{1}{c}{\scriptsize{0.143}} & \mc{1}{c}{\scriptsize{0.095}} & \mc{1}{c}{\scriptsize{0.030}} & \mc{1}{c}{\scriptsize{-0.044}} & \mc{1}{c}{\scriptsize{-0.012}} \\  

     &  & \mc{1}{c}{\scriptsize{(0.671)}} & \mc{1}{c}{\scriptsize{(0.895)}} & \mc{1}{c}{\scriptsize{(0.618)}} & \mc{1}{c}{\scriptsize{(0.618)}} & \mc{1}{c}{\scriptsize{(0.711)}} & \mc{1}{c}{\scriptsize{(0.868)}} & \mc{1}{c}{\scriptsize{(0.961)}} & \mc{1}{c}{\scriptsize{(0.895)}} \\  

    \mc{1}{l}{\scriptsize{Own cars}} & \mc{1}{c}{\scriptsize{30}} & \mc{1}{c}{\scriptsize{0.158}} & \mc{1}{c}{\scriptsize{0.191}} & \mc{1}{c}{\scriptsize{0.208}} & \mc{1}{c}{\scriptsize{0.199}} & \mc{1}{c}{\scriptsize{0.254}} & \mc{1}{c}{\scriptsize{0.167}} & \mc{1}{c}{\scriptsize{0.170}} & \mc{1}{c}{\scriptsize{0.187}} \\  

     &  & \mc{1}{c}{\scriptsize{(0.224)}} & \mc{1}{c}{\scriptsize{(0.145)}} & \mc{1}{c}{\scriptsize{(0.434)}} & \mc{1}{c}{\scriptsize{(0.553)}} & \mc{1}{c}{\scriptsize{(0.289)}} & \mc{1}{c}{\scriptsize{(0.250)}} & \mc{1}{c}{\scriptsize{(0.276)}} & \mc{1}{c}{\scriptsize{(0.158)}} \\  

    \mc{1}{l}{\scriptsize{Own residences}} & \mc{1}{c}{\scriptsize{30}} & \mc{1}{c}{\scriptsize{0.165}} & \mc{1}{c}{\scriptsize{0.176}} & \mc{1}{c}{\scriptsize{0.300}} & \mc{1}{c}{\scriptsize{0.304}} & \mc{1}{c}{\scriptsize{0.320}} & \mc{1}{c}{\scriptsize{0.115}} & \mc{1}{c}{\scriptsize{0.119}} & \mc{1}{c}{\scriptsize{0.128}} \\  

     &  & \mc{1}{c}{\scriptsize{\textbf{(0.092)}}} & \mc{1}{c}{\scriptsize{(0.158)}} & \mc{1}{c}{\scriptsize{\textbf{(0.000)}}} & \mc{1}{c}{\scriptsize{\textbf{(0.079)}}} & \mc{1}{c}{\scriptsize{\textbf{(0.000)}}} & \mc{1}{c}{\scriptsize{(0.316)}} & \mc{1}{c}{\scriptsize{(0.421)}} & \mc{1}{c}{\scriptsize{(0.237)}} \\  

  \bottomrule
  \end{tabular}
	\end{table} 

	\begin{table}[H]
     \caption{Treatment Effects on Relation with Spouse, Female Sample}
     \label{table:abccare_rslt_female_cat19_sd}
	\input{AppResOutput/abccare/rslt_female_cat19_sd}
	\end{table} 

	\begin{table}[H]
     \caption{Treatment Effects on Spouse Characteristics, Female Sample}
     \label{table:abccare_rslt_female_cat20_sd}
	  \begin{tabular}{cccccccccc}
  \toprule

    \scriptsize{Variable} & \scriptsize{Age} & \scriptsize{(1)} & \scriptsize{(2)} & \scriptsize{(3)} & \scriptsize{(4)} & \scriptsize{(5)} & \scriptsize{(6)} & \scriptsize{(7)} & \scriptsize{(8)} \\ 
    \midrule  

    \mc{1}{l}{\scriptsize{Spouse annual income}} & \mc{1}{c}{\scriptsize{30}} & \mc{1}{c}{\scriptsize{11,936}} & \mc{1}{c}{\scriptsize{4,560}} & \mc{1}{c}{\scriptsize{11,811}} & \mc{1}{c}{\scriptsize{-83,156}} & \mc{1}{c}{\scriptsize{4,308}} & \mc{1}{c}{\scriptsize{11,961}} & \mc{1}{c}{\scriptsize{2,846}} & \mc{1}{c}{\scriptsize{3,213}} \\  

     &  & \mc{1}{c}{\scriptsize{(0.276)}} & \mc{1}{c}{\scriptsize{(0.453)}} & \mc{1}{c}{\scriptsize{(0.116)}} & \mc{1}{c}{\scriptsize{(1.000)}} & \mc{1}{c}{\scriptsize{(0.551)}} & \mc{1}{c}{\scriptsize{(0.250)}} & \mc{1}{c}{\scriptsize{(0.520)}} & \mc{1}{c}{\scriptsize{(0.434)}} \\  

    \mc{1}{l}{\scriptsize{Spouse employment status}} & \mc{1}{c}{\scriptsize{30}} & \mc{1}{c}{\scriptsize{-0.006}} & \mc{1}{c}{\scriptsize{-0.312}} & \mc{1}{c}{\scriptsize{-0.083}} & \mc{1}{c}{\scriptsize{-0.237}} & \mc{1}{c}{\scriptsize{-0.372}} & \mc{1}{c}{\scriptsize{0.008}} & \mc{1}{c}{\scriptsize{-0.346}} & \mc{1}{c}{\scriptsize{-0.289}} \\  

     &  & \mc{1}{c}{\scriptsize{(0.605)}} & \mc{1}{c}{\scriptsize{(0.987)}} & \mc{1}{c}{\scriptsize{(0.957)}} & \mc{1}{c}{\scriptsize{(0.961)}} & \mc{1}{c}{\scriptsize{(1.000)}} & \mc{1}{c}{\scriptsize{(0.566)}} & \mc{1}{c}{\scriptsize{(0.947)}} & \mc{1}{c}{\scriptsize{(0.961)}} \\  

  \bottomrule
  \end{tabular}
	\end{table} 

	\begin{table}[H]
     \caption{Treatment Effects on Subject Home and Property, Female Sample}
     \label{table:abccare_rslt_female_cat21_sd}
	  \begin{tabular}{cccccccccc}
  \toprule

    \scriptsize{Variable} & \scriptsize{Age} & \scriptsize{(1)} & \scriptsize{(2)} & \scriptsize{(3)} & \scriptsize{(4)} & \scriptsize{(5)} & \scriptsize{(6)} & \scriptsize{(7)} & \scriptsize{(8)} \\ 
    \midrule  

    \mc{1}{l}{\scriptsize{Ever Adopted}} & \mc{1}{c}{\scriptsize{nan}} & \mc{1}{c}{\scriptsize{0.044}} & \mc{1}{c}{\scriptsize{0.107}} & \mc{1}{c}{\scriptsize{-0.173}} & \mc{1}{c}{\scriptsize{-0.189}} & \mc{1}{c}{\scriptsize{-0.160}} & \mc{1}{c}{\scriptsize{0.077}} & \mc{1}{c}{\scriptsize{0.175}} & \mc{1}{c}{\scriptsize{0.091}} \\  

     &  & \mc{1}{c}{\scriptsize{(0.197)}} & \mc{1}{c}{\scriptsize{(0.105)}} & \mc{1}{c}{\scriptsize{(0.724)}} & \mc{1}{c}{\scriptsize{(0.776)}} & \mc{1}{c}{\scriptsize{(0.711)}} & \mc{1}{c}{\scriptsize{\textbf{(0.053)}}} & \mc{1}{c}{\scriptsize{\textbf{(0.000)}}} & \mc{1}{c}{\scriptsize{\textbf{(0.079)}}} \\  

  \bottomrule
  \end{tabular}
	\end{table} 

	\begin{table}[H]
     \caption{Treatment Effects on Education, Female Sample}
     \label{table:abccare_rslt_female_cat22_sd}
	  \begin{tabular}{cccccccccc}
  \toprule

    \scriptsize{Variable} & \scriptsize{Age} & \scriptsize{(1)} & \scriptsize{(2)} & \scriptsize{(3)} & \scriptsize{(4)} & \scriptsize{(5)} & \scriptsize{(6)} & \scriptsize{(7)} & \scriptsize{(8)} \\ 
    \midrule  

    \mc{1}{l}{\scriptsize{Household Earned Income}} & \mc{1}{c}{\scriptsize{9}} & \mc{1}{c}{\scriptsize{17,749}} & \mc{1}{c}{\scriptsize{39,936}} & \mc{1}{c}{\scriptsize{17,749}} & \mc{1}{c}{\scriptsize{39,936}} & \mc{1}{c}{\scriptsize{9,005}} &  &  &  \\  

     &  & \mc{1}{c}{\scriptsize{(0.303)}} & \mc{1}{c}{\scriptsize{\textbf{(0.013)}}} & \mc{1}{c}{\scriptsize{(0.237)}} & \mc{1}{c}{\scriptsize{\textbf{(0.013)}}} & \mc{1}{c}{\scriptsize{(0.474)}} &  &  &  \\  

     & \mc{1}{c}{\scriptsize{0}} & \mc{1}{c}{\scriptsize{3,156}} & \mc{1}{c}{\scriptsize{8,599}} & \mc{1}{c}{\scriptsize{5,175}} & \mc{1}{c}{\scriptsize{21,776}} & \mc{1}{c}{\scriptsize{9,760}} & \mc{1}{c}{\scriptsize{1,540}} & \mc{1}{c}{\scriptsize{6,550}} & \mc{1}{c}{\scriptsize{6,107}} \\  

     &  & \mc{1}{c}{\scriptsize{(0.553)}} & \mc{1}{c}{\scriptsize{(0.566)}} & \mc{1}{c}{\scriptsize{(0.395)}} & \mc{1}{c}{\scriptsize{(0.342)}} & \mc{1}{c}{\scriptsize{(0.158)}} & \mc{1}{c}{\scriptsize{(0.618)}} & \mc{1}{c}{\scriptsize{(0.579)}} & \mc{1}{c}{\scriptsize{(0.368)}} \\  

     & \mc{1}{c}{\scriptsize{12}} & \mc{1}{c}{\scriptsize{11,182}} & \mc{1}{c}{\scriptsize{12,209}} & \mc{1}{c}{\scriptsize{24,645}} & \mc{1}{c}{\scriptsize{32,766}} & \mc{1}{c}{\scriptsize{23,251}} & \mc{1}{c}{\scriptsize{8,122}} & \mc{1}{c}{\scriptsize{5,571}} & \mc{1}{c}{\scriptsize{8,482}} \\  

     &  & \mc{1}{c}{\scriptsize{(0.132)}} & \mc{1}{c}{\scriptsize{(0.184)}} & \mc{1}{c}{\scriptsize{\textbf{(0.000)}}} & \mc{1}{c}{\scriptsize{(0.132)}} & \mc{1}{c}{\scriptsize{\textbf{(0.026)}}} & \mc{1}{c}{\scriptsize{(0.276)}} & \mc{1}{c}{\scriptsize{(0.395)}} & \mc{1}{c}{\scriptsize{(0.197)}} \\  

  \bottomrule
  \end{tabular}
	\end{table} 

	\begin{table}[H]
     \caption{Treatment Effects on Subject Employment and Income, Female Sample}
     \label{table:abccare_rslt_female_cat23_sd}
	  \begin{tabular}{cccccccccc}
  \toprule

    \scriptsize{Variable} & \scriptsize{Age} & \scriptsize{(1)} & \scriptsize{(2)} & \scriptsize{(3)} & \scriptsize{(4)} & \scriptsize{(5)} & \scriptsize{(6)} & \scriptsize{(7)} & \scriptsize{(8)} \\ 
    \midrule  

    \mc{1}{l}{\scriptsize{Father at Home}} & \mc{1}{c}{\scriptsize{3}} & \mc{1}{c}{\scriptsize{-0.079}} & \mc{1}{c}{\scriptsize{-0.038}} & \mc{1}{c}{\scriptsize{-0.337}} & \mc{1}{c}{\scriptsize{-0.330}} & \mc{1}{c}{\scriptsize{-0.315}} & \mc{1}{c}{\scriptsize{0.034}} & \mc{1}{c}{\scriptsize{0.095}} & \mc{1}{c}{\scriptsize{0.087}} \\  

     &  & \mc{1}{c}{\scriptsize{(0.934)}} & \mc{1}{c}{\scriptsize{(0.882)}} & \mc{1}{c}{\scriptsize{(1.000)}} & \mc{1}{c}{\scriptsize{(1.000)}} & \mc{1}{c}{\scriptsize{(1.000)}} & \mc{1}{c}{\scriptsize{(0.618)}} & \mc{1}{c}{\scriptsize{(0.382)}} & \mc{1}{c}{\scriptsize{(0.368)}} \\  

     & \mc{1}{c}{\scriptsize{2}} & \mc{1}{c}{\scriptsize{-0.012}} & \mc{1}{c}{\scriptsize{0.015}} & \mc{1}{c}{\scriptsize{-0.115}} & \mc{1}{c}{\scriptsize{-0.166}} & \mc{1}{c}{\scriptsize{-0.094}} & \mc{1}{c}{\scriptsize{0.034}} & \mc{1}{c}{\scriptsize{0.095}} & \mc{1}{c}{\scriptsize{0.087}} \\  

     &  & \mc{1}{c}{\scriptsize{(0.776)}} & \mc{1}{c}{\scriptsize{(0.789)}} & \mc{1}{c}{\scriptsize{(0.961)}} & \mc{1}{c}{\scriptsize{(1.000)}} & \mc{1}{c}{\scriptsize{(0.961)}} & \mc{1}{c}{\scriptsize{(0.618)}} & \mc{1}{c}{\scriptsize{(0.382)}} & \mc{1}{c}{\scriptsize{(0.368)}} \\  

     & \mc{1}{c}{\scriptsize{8}} & \mc{1}{c}{\scriptsize{0.056}} & \mc{1}{c}{\scriptsize{0.078}} & \mc{1}{c}{\scriptsize{-0.064}} & \mc{1}{c}{\scriptsize{0.016}} & \mc{1}{c}{\scriptsize{-0.055}} & \mc{1}{c}{\scriptsize{0.092}} & \mc{1}{c}{\scriptsize{0.124}} & \mc{1}{c}{\scriptsize{0.058}} \\  

     &  & \mc{1}{c}{\scriptsize{(0.579)}} & \mc{1}{c}{\scriptsize{(0.421)}} & \mc{1}{c}{\scriptsize{(0.868)}} & \mc{1}{c}{\scriptsize{(0.855)}} & \mc{1}{c}{\scriptsize{(0.868)}} & \mc{1}{c}{\scriptsize{(0.329)}} & \mc{1}{c}{\scriptsize{(0.289)}} & \mc{1}{c}{\scriptsize{(0.579)}} \\  

     & \mc{1}{c}{\scriptsize{5}} & \mc{1}{c}{\scriptsize{-0.139}} & \mc{1}{c}{\scriptsize{-0.101}} & \mc{1}{c}{\scriptsize{-0.333}} & \mc{1}{c}{\scriptsize{-0.278}} & \mc{1}{c}{\scriptsize{-0.328}} & \mc{1}{c}{\scriptsize{-0.056}} & \mc{1}{c}{\scriptsize{-0.025}} & \mc{1}{c}{\scriptsize{-0.020}} \\  

     &  & \mc{1}{c}{\scriptsize{(0.987)}} & \mc{1}{c}{\scriptsize{(0.974)}} & \mc{1}{c}{\scriptsize{(1.000)}} & \mc{1}{c}{\scriptsize{(1.000)}} & \mc{1}{c}{\scriptsize{(1.000)}} & \mc{1}{c}{\scriptsize{(0.868)}} & \mc{1}{c}{\scriptsize{(0.803)}} & \mc{1}{c}{\scriptsize{(0.763)}} \\  

     & \mc{1}{c}{\scriptsize{4}} & \mc{1}{c}{\scriptsize{-0.071}} & \mc{1}{c}{\scriptsize{-0.042}} & \mc{1}{c}{\scriptsize{-0.330}} & \mc{1}{c}{\scriptsize{-0.350}} & \mc{1}{c}{\scriptsize{-0.306}} & \mc{1}{c}{\scriptsize{0.041}} & \mc{1}{c}{\scriptsize{0.091}} & \mc{1}{c}{\scriptsize{0.096}} \\  

     &  & \mc{1}{c}{\scriptsize{(0.921)}} & \mc{1}{c}{\scriptsize{(0.882)}} & \mc{1}{c}{\scriptsize{(1.000)}} & \mc{1}{c}{\scriptsize{(1.000)}} & \mc{1}{c}{\scriptsize{(1.000)}} & \mc{1}{c}{\scriptsize{(0.605)}} & \mc{1}{c}{\scriptsize{(0.421)}} & \mc{1}{c}{\scriptsize{(0.368)}} \\  

  \bottomrule
  \end{tabular}
	\end{table} 

	\begin{table}[H]
     \caption{Treatment Effects on Job Attitude, Female Sample}
     \label{table:abccare_rslt_female_cat24_sd}
	  \begin{tabular}{cccccccccc}
  \toprule

    \scriptsize{Variable} & \scriptsize{Age} & \scriptsize{(1)} & \scriptsize{(2)} & \scriptsize{(3)} & \scriptsize{(4)} & \scriptsize{(5)} & \scriptsize{(6)} & \scriptsize{(7)} & \scriptsize{(8)} \\ 
    \midrule  

    \mc{1}{l}{\scriptsize{HOME Score}} & \mc{1}{c}{\scriptsize{3.5}} & \mc{1}{c}{\scriptsize{2.858}} & \mc{1}{c}{\scriptsize{2.561}} & \mc{1}{c}{\scriptsize{13.719}} & \mc{1}{c}{\scriptsize{15.918}} & \mc{1}{c}{\scriptsize{13.454}} & \mc{1}{c}{\scriptsize{-0.309}} & \mc{1}{c}{\scriptsize{-1.851}} & \mc{1}{c}{\scriptsize{-0.058}} \\  

     &  & \mc{1}{c}{\scriptsize{(0.342)}} & \mc{1}{c}{\scriptsize{(0.474)}} & \mc{1}{c}{\scriptsize{\textbf{(0.000)}}} & \mc{1}{c}{\scriptsize{\textbf{(0.039)}}} & \mc{1}{c}{\scriptsize{\textbf{(0.000)}}} & \mc{1}{c}{\scriptsize{(0.880)}} & \mc{1}{c}{\scriptsize{(1.000)}} & \mc{1}{c}{\scriptsize{(0.892)}} \\  

     & \mc{1}{c}{\scriptsize{1.5}} & \mc{1}{c}{\scriptsize{2.668}} & \mc{1}{c}{\scriptsize{1.723}} & \mc{1}{c}{\scriptsize{4.729}} & \mc{1}{c}{\scriptsize{3.279}} & \mc{1}{c}{\scriptsize{4.661}} & \mc{1}{c}{\scriptsize{1.544}} & \mc{1}{c}{\scriptsize{0.992}} & \mc{1}{c}{\scriptsize{1.746}} \\  

     &  & \mc{1}{c}{\scriptsize{\textbf{(0.053)}}} & \mc{1}{c}{\scriptsize{(0.342)}} & \mc{1}{c}{\scriptsize{(0.118)}} & \mc{1}{c}{\scriptsize{(0.211)}} & \mc{1}{c}{\scriptsize{(0.105)}} & \mc{1}{c}{\scriptsize{(0.493)}} & \mc{1}{c}{\scriptsize{(0.903)}} & \mc{1}{c}{\scriptsize{(0.541)}} \\  

     & \mc{1}{c}{\scriptsize{4.5}} & \mc{1}{c}{\scriptsize{2.736}} & \mc{1}{c}{\scriptsize{1.288}} & \mc{1}{c}{\scriptsize{12.957}} & \mc{1}{c}{\scriptsize{11.926}} & \mc{1}{c}{\scriptsize{13.245}} & \mc{1}{c}{\scriptsize{-0.273}} & \mc{1}{c}{\scriptsize{-1.462}} & \mc{1}{c}{\scriptsize{0.468}} \\  

     &  & \mc{1}{c}{\scriptsize{(0.382)}} & \mc{1}{c}{\scriptsize{(0.750)}} & \mc{1}{c}{\scriptsize{\textbf{(0.026)}}} & \mc{1}{c}{\scriptsize{(0.105)}} & \mc{1}{c}{\scriptsize{\textbf{(0.026)}}} & \mc{1}{c}{\scriptsize{(0.880)}} & \mc{1}{c}{\scriptsize{(1.000)}} & \mc{1}{c}{\scriptsize{(0.824)}} \\  

     & \mc{1}{c}{\scriptsize{2.5}} & \mc{1}{c}{\scriptsize{0.762}} & \mc{1}{c}{\scriptsize{0.832}} & \mc{1}{c}{\scriptsize{4.434}} & \mc{1}{c}{\scriptsize{5.502}} & \mc{1}{c}{\scriptsize{4.637}} & \mc{1}{c}{\scriptsize{-0.899}} & \mc{1}{c}{\scriptsize{-0.932}} & \mc{1}{c}{\scriptsize{-0.256}} \\  

     &  & \mc{1}{c}{\scriptsize{(0.724)}} & \mc{1}{c}{\scriptsize{(0.737)}} & \mc{1}{c}{\scriptsize{\textbf{(0.013)}}} & \mc{1}{c}{\scriptsize{\textbf{(0.092)}}} & \mc{1}{c}{\scriptsize{\textbf{(0.000)}}} & \mc{1}{c}{\scriptsize{(0.933)}} & \mc{1}{c}{\scriptsize{(0.986)}} & \mc{1}{c}{\scriptsize{(0.919)}} \\  

     & \mc{1}{c}{\scriptsize{8}} & \mc{1}{c}{\scriptsize{0.659}} & \mc{1}{c}{\scriptsize{1.073}} & \mc{1}{c}{\scriptsize{5.909}} & \mc{1}{c}{\scriptsize{6.850}} & \mc{1}{c}{\scriptsize{7.057}} & \mc{1}{c}{\scriptsize{-0.773}} & \mc{1}{c}{\scriptsize{-0.879}} & \mc{1}{c}{\scriptsize{0.448}} \\  

     &  & \mc{1}{c}{\scriptsize{(0.816)}} & \mc{1}{c}{\scriptsize{(0.763)}} & \mc{1}{c}{\scriptsize{(0.145)}} & \mc{1}{c}{\scriptsize{(0.145)}} & \mc{1}{c}{\scriptsize{\textbf{(0.092)}}} & \mc{1}{c}{\scriptsize{(0.933)}} & \mc{1}{c}{\scriptsize{(1.000)}} & \mc{1}{c}{\scriptsize{(0.811)}} \\  

     & \mc{1}{c}{\scriptsize{0.5}} & \mc{1}{c}{\scriptsize{1.581}} & \mc{1}{c}{\scriptsize{0.749}} & \mc{1}{c}{\scriptsize{1.684}} & \mc{1}{c}{\scriptsize{1.131}} & \mc{1}{c}{\scriptsize{1.067}} & \mc{1}{c}{\scriptsize{0.980}} & \mc{1}{c}{\scriptsize{0.498}} & \mc{1}{c}{\scriptsize{0.438}} \\  

     &  & \mc{1}{c}{\scriptsize{(0.211)}} & \mc{1}{c}{\scriptsize{(0.737)}} & \mc{1}{c}{\scriptsize{(0.395)}} & \mc{1}{c}{\scriptsize{(0.553)}} & \mc{1}{c}{\scriptsize{(0.658)}} & \mc{1}{c}{\scriptsize{(0.573)}} & \mc{1}{c}{\scriptsize{(0.639)}} & \mc{1}{c}{\scriptsize{(0.716)}} \\  

  \bottomrule
  \end{tabular}
	\end{table} 

	\begin{table}[H]
     \caption{Treatment Effects on Job Satisfaction Score, Female Sample}
     \label{table:abccare_rslt_female_cat25_sd}
	  \begin{tabular}{cccccccccc}
  \toprule

    \scriptsize{Variable} & \scriptsize{Age} & \scriptsize{(1)} & \scriptsize{(2)} & \scriptsize{(3)} & \scriptsize{(4)} & \scriptsize{(5)} & \scriptsize{(6)} & \scriptsize{(7)} & \scriptsize{(8)} \\ 
    \midrule  

    \mc{1}{l}{\scriptsize{Recognition for good work}} & \mc{1}{c}{\scriptsize{30}} & \mc{1}{c}{\scriptsize{0.161}} & \mc{1}{c}{\scriptsize{0.309}} & \mc{1}{c}{\scriptsize{0.628}} & \mc{1}{c}{\scriptsize{0.765}} & \mc{1}{c}{\scriptsize{0.814}} & \mc{1}{c}{\scriptsize{0.076}} & \mc{1}{c}{\scriptsize{0.195}} & \mc{1}{c}{\scriptsize{0.201}} \\  

     &  & \mc{1}{c}{\scriptsize{(0.750)}} & \mc{1}{c}{\scriptsize{(0.671)}} & \mc{1}{c}{\scriptsize{(0.197)}} & \mc{1}{c}{\scriptsize{(0.395)}} & \mc{1}{c}{\scriptsize{(0.105)}} & \mc{1}{c}{\scriptsize{(0.868)}} & \mc{1}{c}{\scriptsize{(0.776)}} & \mc{1}{c}{\scriptsize{(0.855)}} \\  

    \mc{1}{l}{\scriptsize{Total}} & \mc{1}{c}{\scriptsize{30}} & \mc{1}{c}{\scriptsize{0.158}} & \mc{1}{c}{\scriptsize{0.308}} & \mc{1}{c}{\scriptsize{0.638}} & \mc{1}{c}{\scriptsize{0.551}} & \mc{1}{c}{\scriptsize{0.784}} & \mc{1}{c}{\scriptsize{0.080}} & \mc{1}{c}{\scriptsize{0.240}} & \mc{1}{c}{\scriptsize{0.197}} \\  

     &  & \mc{1}{c}{\scriptsize{(0.737)}} & \mc{1}{c}{\scriptsize{(0.592)}} & \mc{1}{c}{\scriptsize{(0.250)}} & \mc{1}{c}{\scriptsize{(0.566)}} & \mc{1}{c}{\scriptsize{(0.158)}} & \mc{1}{c}{\scriptsize{(0.868)}} & \mc{1}{c}{\scriptsize{(0.697)}} & \mc{1}{c}{\scriptsize{(0.829)}} \\  

    \mc{1}{l}{\scriptsize{Operating policies and procedures}} & \mc{1}{c}{\scriptsize{30}} & \mc{1}{c}{\scriptsize{0.334}} & \mc{1}{c}{\scriptsize{0.402}} & \mc{1}{c}{\scriptsize{0.529}} & \mc{1}{c}{\scriptsize{0.505}} & \mc{1}{c}{\scriptsize{0.637}} & \mc{1}{c}{\scriptsize{0.273}} & \mc{1}{c}{\scriptsize{0.344}} & \mc{1}{c}{\scriptsize{0.377}} \\  

     &  & \mc{1}{c}{\scriptsize{(0.355)}} & \mc{1}{c}{\scriptsize{(0.237)}} & \mc{1}{c}{\scriptsize{(0.500)}} & \mc{1}{c}{\scriptsize{(0.553)}} & \mc{1}{c}{\scriptsize{(0.434)}} & \mc{1}{c}{\scriptsize{(0.539)}} & \mc{1}{c}{\scriptsize{(0.474)}} & \mc{1}{c}{\scriptsize{(0.513)}} \\  

    \mc{1}{l}{\scriptsize{Immediate supervisor}} & \mc{1}{c}{\scriptsize{30}} & \mc{1}{c}{\scriptsize{0.347}} & \mc{1}{c}{\scriptsize{0.400}} & \mc{1}{c}{\scriptsize{0.467}} & \mc{1}{c}{\scriptsize{0.294}} & \mc{1}{c}{\scriptsize{0.649}} & \mc{1}{c}{\scriptsize{0.256}} & \mc{1}{c}{\scriptsize{0.461}} & \mc{1}{c}{\scriptsize{0.419}} \\  

     &  & \mc{1}{c}{\scriptsize{(0.592)}} & \mc{1}{c}{\scriptsize{(0.461)}} & \mc{1}{c}{\scriptsize{(0.421)}} & \mc{1}{c}{\scriptsize{(0.842)}} & \mc{1}{c}{\scriptsize{(0.197)}} & \mc{1}{c}{\scriptsize{(0.658)}} & \mc{1}{c}{\scriptsize{(0.421)}} & \mc{1}{c}{\scriptsize{(0.605)}} \\  

    \mc{1}{l}{\scriptsize{Pay and remuneration}} & \mc{1}{c}{\scriptsize{30}} & \mc{1}{c}{\scriptsize{-0.194}} & \mc{1}{c}{\scriptsize{0.154}} & \mc{1}{c}{\scriptsize{0.249}} & \mc{1}{c}{\scriptsize{0.431}} & \mc{1}{c}{\scriptsize{0.453}} & \mc{1}{c}{\scriptsize{-0.222}} & \mc{1}{c}{\scriptsize{0.058}} & \mc{1}{c}{\scriptsize{-0.051}} \\  

     &  & \mc{1}{c}{\scriptsize{(1.000)}} & \mc{1}{c}{\scriptsize{(0.803)}} & \mc{1}{c}{\scriptsize{(0.513)}} & \mc{1}{c}{\scriptsize{(0.645)}} & \mc{1}{c}{\scriptsize{(0.211)}} & \mc{1}{c}{\scriptsize{(1.000)}} & \mc{1}{c}{\scriptsize{(0.868)}} & \mc{1}{c}{\scriptsize{(0.974)}} \\  

    \mc{1}{l}{\scriptsize{Coworkers}} & \mc{1}{c}{\scriptsize{30}} & \mc{1}{c}{\scriptsize{0.556}} & \mc{1}{c}{\scriptsize{0.869}} & \mc{1}{c}{\scriptsize{0.820}} & \mc{1}{c}{\scriptsize{0.847}} & \mc{1}{c}{\scriptsize{1.042}} & \mc{1}{c}{\scriptsize{0.456}} & \mc{1}{c}{\scriptsize{0.874}} & \mc{1}{c}{\scriptsize{0.706}} \\  

     &  & \mc{1}{c}{\scriptsize{\textbf{(0.079)}}} & \mc{1}{c}{\scriptsize{\textbf{(0.026)}}} & \mc{1}{c}{\scriptsize{(0.145)}} & \mc{1}{c}{\scriptsize{(0.329)}} & \mc{1}{c}{\scriptsize{\textbf{(0.066)}}} & \mc{1}{c}{\scriptsize{(0.263)}} & \mc{1}{c}{\scriptsize{\textbf{(0.039)}}} & \mc{1}{c}{\scriptsize{\textbf{(0.039)}}} \\  

    \mc{1}{l}{\scriptsize{Job tasks}} & \mc{1}{c}{\scriptsize{30}} & \mc{1}{c}{\scriptsize{-0.274}} & \mc{1}{c}{\scriptsize{0.025}} & \mc{1}{c}{\scriptsize{0.167}} & \mc{1}{c}{\scriptsize{0.095}} & \mc{1}{c}{\scriptsize{0.364}} & \mc{1}{c}{\scriptsize{-0.333}} & \mc{1}{c}{\scriptsize{-0.034}} & \mc{1}{c}{\scriptsize{-0.100}} \\  

     &  & \mc{1}{c}{\scriptsize{(1.000)}} & \mc{1}{c}{\scriptsize{(0.921)}} & \mc{1}{c}{\scriptsize{(0.803)}} & \mc{1}{c}{\scriptsize{(0.908)}} & \mc{1}{c}{\scriptsize{(0.724)}} & \mc{1}{c}{\scriptsize{(1.000)}} & \mc{1}{c}{\scriptsize{(0.974)}} & \mc{1}{c}{\scriptsize{(0.987)}} \\  

    \mc{1}{l}{\scriptsize{Fringe benefits}} & \mc{1}{c}{\scriptsize{30}} & \mc{1}{c}{\scriptsize{0.189}} & \mc{1}{c}{\scriptsize{0.075}} & \mc{1}{c}{\scriptsize{0.708}} & \mc{1}{c}{\scriptsize{0.469}} & \mc{1}{c}{\scriptsize{0.606}} & \mc{1}{c}{\scriptsize{0.095}} & \mc{1}{c}{\scriptsize{-0.043}} & \mc{1}{c}{\scriptsize{-0.062}} \\  

     &  & \mc{1}{c}{\scriptsize{(0.658)}} & \mc{1}{c}{\scriptsize{(0.895)}} & \mc{1}{c}{\scriptsize{\textbf{(0.000)}}} & \mc{1}{c}{\scriptsize{(0.645)}} & \mc{1}{c}{\scriptsize{(0.105)}} & \mc{1}{c}{\scriptsize{(0.776)}} & \mc{1}{c}{\scriptsize{(0.974)}} & \mc{1}{c}{\scriptsize{(0.974)}} \\  

    \mc{1}{l}{\scriptsize{Communication with organization}} & \mc{1}{c}{\scriptsize{30}} & \mc{1}{c}{\scriptsize{-0.078}} & \mc{1}{c}{\scriptsize{0.043}} & \mc{1}{c}{\scriptsize{0.256}} & \mc{1}{c}{\scriptsize{0.082}} & \mc{1}{c}{\scriptsize{0.415}} & \mc{1}{c}{\scriptsize{-0.103}} & \mc{1}{c}{\scriptsize{-0.009}} & \mc{1}{c}{\scriptsize{-0.003}} \\  

     &  & \mc{1}{c}{\scriptsize{(0.974)}} & \mc{1}{c}{\scriptsize{(0.908)}} & \mc{1}{c}{\scriptsize{(0.671)}} & \mc{1}{c}{\scriptsize{(0.908)}} & \mc{1}{c}{\scriptsize{(0.408)}} & \mc{1}{c}{\scriptsize{(0.987)}} & \mc{1}{c}{\scriptsize{(0.895)}} & \mc{1}{c}{\scriptsize{(0.961)}} \\  

    \mc{1}{l}{\scriptsize{Promotion opportunities}} & \mc{1}{c}{\scriptsize{30}} & \mc{1}{c}{\scriptsize{-0.140}} & \mc{1}{c}{\scriptsize{0.059}} & \mc{1}{c}{\scriptsize{-0.099}} & \mc{1}{c}{\scriptsize{0.039}} & \mc{1}{c}{\scriptsize{-0.032}} & \mc{1}{c}{\scriptsize{-0.081}} & \mc{1}{c}{\scriptsize{0.045}} & \mc{1}{c}{\scriptsize{-0.073}} \\  

     &  & \mc{1}{c}{\scriptsize{(0.987)}} & \mc{1}{c}{\scriptsize{(0.908)}} & \mc{1}{c}{\scriptsize{(0.934)}} & \mc{1}{c}{\scriptsize{(0.921)}} & \mc{1}{c}{\scriptsize{(0.947)}} & \mc{1}{c}{\scriptsize{(0.987)}} & \mc{1}{c}{\scriptsize{(0.882)}} & \mc{1}{c}{\scriptsize{(0.974)}} \\  

  \bottomrule
  \end{tabular}
	\end{table} 

	\begin{table}[H]
     \caption{Treatment Effects on Crime, Female Sample}
     \label{table:abccare_rslt_female_cat26_sd}
	\input{AppResOutput/abccare/rslt_female_cat26_sd}
	\end{table} 

	\begin{table}[H]
     \caption{Treatment Effects on Childhood and Adolescence Physical Health, Female Sample}
     \label{table:abccare_rslt_female_cat27_sd}
	  \begin{tabular}{cccccccccc}
  \toprule

    \scriptsize{Variable} & \scriptsize{Age} & \scriptsize{(1)} & \scriptsize{(2)} & \scriptsize{(3)} & \scriptsize{(4)} & \scriptsize{(5)} & \scriptsize{(6)} & \scriptsize{(7)} & \scriptsize{(8)} \\ 
    \midrule  

    \mc{1}{l}{\scriptsize{Body Mass Index (BMI)}} & \mc{1}{c}{\scriptsize{0}} & \mc{1}{c}{\scriptsize{-1.503}} & \mc{1}{c}{\scriptsize{-2.382}} & \mc{1}{c}{\scriptsize{-3.870}} & \mc{1}{c}{\scriptsize{-4.872}} & \mc{1}{c}{\scriptsize{-3.384}} & \mc{1}{c}{\scriptsize{-1.235}} & \mc{1}{c}{\scriptsize{-2.165}} & \mc{1}{c}{\scriptsize{-2.513}} \\  

     &  & \mc{1}{c}{\scriptsize{(0.566)}} & \mc{1}{c}{\scriptsize{(0.276)}} & \mc{1}{c}{\scriptsize{(0.737)}} & \mc{1}{c}{\scriptsize{(0.605)}} & \mc{1}{c}{\scriptsize{(0.789)}} & \mc{1}{c}{\scriptsize{(0.566)}} & \mc{1}{c}{\scriptsize{(0.316)}} & \mc{1}{c}{\scriptsize{(0.368)}} \\  

     & \mc{1}{c}{\scriptsize{0.25}} & \mc{1}{c}{\scriptsize{-0.995}} & \mc{1}{c}{\scriptsize{-1.195}} & \mc{1}{c}{\scriptsize{-0.821}} & \mc{1}{c}{\scriptsize{-1.407}} & \mc{1}{c}{\scriptsize{-0.925}} & \mc{1}{c}{\scriptsize{-1.175}} & \mc{1}{c}{\scriptsize{-1.206}} & \mc{1}{c}{\scriptsize{-1.264}} \\  

     &  & \mc{1}{c}{\scriptsize{\textbf{(0.053)}}} & \mc{1}{c}{\scriptsize{(0.171)}} & \mc{1}{c}{\scriptsize{(0.579)}} & \mc{1}{c}{\scriptsize{(0.605)}} & \mc{1}{c}{\scriptsize{(0.579)}} & \mc{1}{c}{\scriptsize{\textbf{(0.092)}}} & \mc{1}{c}{\scriptsize{(0.237)}} & \mc{1}{c}{\scriptsize{(0.118)}} \\  

     & \mc{1}{c}{\scriptsize{0.5}} & \mc{1}{c}{\scriptsize{-0.794}} & \mc{1}{c}{\scriptsize{-0.846}} & \mc{1}{c}{\scriptsize{-0.454}} & \mc{1}{c}{\scriptsize{-0.253}} & \mc{1}{c}{\scriptsize{-0.676}} & \mc{1}{c}{\scriptsize{-0.952}} & \mc{1}{c}{\scriptsize{-0.961}} & \mc{1}{c}{\scriptsize{-1.019}} \\  

     &  & \mc{1}{c}{\scriptsize{(0.224)}} & \mc{1}{c}{\scriptsize{(0.303)}} & \mc{1}{c}{\scriptsize{(0.789)}} & \mc{1}{c}{\scriptsize{(0.921)}} & \mc{1}{c}{\scriptsize{(0.763)}} & \mc{1}{c}{\scriptsize{(0.118)}} & \mc{1}{c}{\scriptsize{(0.276)}} & \mc{1}{c}{\scriptsize{(0.105)}} \\  

     & \mc{1}{c}{\scriptsize{0.75}} & \mc{1}{c}{\scriptsize{-1.541}} & \mc{1}{c}{\scriptsize{-1.150}} & \mc{1}{c}{\scriptsize{-1.353}} & \mc{1}{c}{\scriptsize{-1.053}} & \mc{1}{c}{\scriptsize{-1.369}} & \mc{1}{c}{\scriptsize{-1.624}} & \mc{1}{c}{\scriptsize{-1.163}} & \mc{1}{c}{\scriptsize{-1.559}} \\  

     &  & \mc{1}{c}{\scriptsize{\textbf{(0.000)}}} & \mc{1}{c}{\scriptsize{(0.171)}} & \mc{1}{c}{\scriptsize{(0.592)}} & \mc{1}{c}{\scriptsize{(0.921)}} & \mc{1}{c}{\scriptsize{(0.618)}} & \mc{1}{c}{\scriptsize{\textbf{(0.013)}}} & \mc{1}{c}{\scriptsize{(0.132)}} & \mc{1}{c}{\scriptsize{\textbf{(0.000)}}} \\  

     & \mc{1}{c}{\scriptsize{1}} & \mc{1}{c}{\scriptsize{-0.456}} & \mc{1}{c}{\scriptsize{-0.159}} & \mc{1}{c}{\scriptsize{-0.052}} & \mc{1}{c}{\scriptsize{-0.009}} & \mc{1}{c}{\scriptsize{-0.231}} & \mc{1}{c}{\scriptsize{-0.424}} & \mc{1}{c}{\scriptsize{-0.208}} & \mc{1}{c}{\scriptsize{-0.580}} \\  

     &  & \mc{1}{c}{\scriptsize{(0.592)}} & \mc{1}{c}{\scriptsize{(0.934)}} & \mc{1}{c}{\scriptsize{(0.882)}} & \mc{1}{c}{\scriptsize{(0.947)}} & \mc{1}{c}{\scriptsize{(0.855)}} & \mc{1}{c}{\scriptsize{(0.579)}} & \mc{1}{c}{\scriptsize{(0.934)}} & \mc{1}{c}{\scriptsize{(0.461)}} \\  

     & \mc{1}{c}{\scriptsize{1.5}} & \mc{1}{c}{\scriptsize{-0.746}} & \mc{1}{c}{\scriptsize{-0.638}} & \mc{1}{c}{\scriptsize{-0.566}} & \mc{1}{c}{\scriptsize{-0.679}} & \mc{1}{c}{\scriptsize{-0.705}} & \mc{1}{c}{\scriptsize{-0.919}} & \mc{1}{c}{\scriptsize{-0.617}} & \mc{1}{c}{\scriptsize{-0.919}} \\  

     &  & \mc{1}{c}{\scriptsize{(0.368)}} & \mc{1}{c}{\scriptsize{(0.513)}} & \mc{1}{c}{\scriptsize{(0.671)}} & \mc{1}{c}{\scriptsize{(0.711)}} & \mc{1}{c}{\scriptsize{(0.750)}} & \mc{1}{c}{\scriptsize{(0.184)}} & \mc{1}{c}{\scriptsize{(0.658)}} & \mc{1}{c}{\scriptsize{(0.263)}} \\  

     & \mc{1}{c}{\scriptsize{2}} & \mc{1}{c}{\scriptsize{-0.336}} & \mc{1}{c}{\scriptsize{0.079}} & \mc{1}{c}{\scriptsize{-0.438}} & \mc{1}{c}{\scriptsize{-0.401}} & \mc{1}{c}{\scriptsize{-0.415}} & \mc{1}{c}{\scriptsize{-0.291}} & \mc{1}{c}{\scriptsize{0.430}} & \mc{1}{c}{\scriptsize{0.148}} \\  

     &  & \mc{1}{c}{\scriptsize{(0.632)}} & \mc{1}{c}{\scriptsize{(0.974)}} & \mc{1}{c}{\scriptsize{(0.671)}} & \mc{1}{c}{\scriptsize{(0.921)}} & \mc{1}{c}{\scriptsize{(0.776)}} & \mc{1}{c}{\scriptsize{(0.684)}} & \mc{1}{c}{\scriptsize{(1.000)}} & \mc{1}{c}{\scriptsize{(0.987)}} \\  

     & \mc{1}{c}{\scriptsize{2.5}} & \mc{1}{c}{\scriptsize{-0.061}} & \mc{1}{c}{\scriptsize{-0.188}} & \mc{1}{c}{\scriptsize{-0.052}} & \mc{1}{c}{\scriptsize{0.532}} & \mc{1}{c}{\scriptsize{0.248}} & \mc{1}{c}{\scriptsize{-0.289}} & \mc{1}{c}{\scriptsize{0.186}} & \mc{1}{c}{\scriptsize{-0.235}} \\  

     &  & \mc{1}{c}{\scriptsize{(0.895)}} & \mc{1}{c}{\scriptsize{(0.947)}} & \mc{1}{c}{\scriptsize{(0.882)}} & \mc{1}{c}{\scriptsize{(0.987)}} & \mc{1}{c}{\scriptsize{(0.987)}} & \mc{1}{c}{\scriptsize{(0.737)}} & \mc{1}{c}{\scriptsize{(1.000)}} & \mc{1}{c}{\scriptsize{(0.882)}} \\  

     & \mc{1}{c}{\scriptsize{3}} & \mc{1}{c}{\scriptsize{-0.036}} & \mc{1}{c}{\scriptsize{0.395}} & \mc{1}{c}{\scriptsize{0.252}} & \mc{1}{c}{\scriptsize{0.138}} & \mc{1}{c}{\scriptsize{0.394}} & \mc{1}{c}{\scriptsize{-0.094}} & \mc{1}{c}{\scriptsize{0.418}} & \mc{1}{c}{\scriptsize{0.177}} \\  

     &  & \mc{1}{c}{\scriptsize{(0.921)}} & \mc{1}{c}{\scriptsize{(1.000)}} & \mc{1}{c}{\scriptsize{(0.987)}} & \mc{1}{c}{\scriptsize{(0.961)}} & \mc{1}{c}{\scriptsize{(1.000)}} & \mc{1}{c}{\scriptsize{(0.882)}} & \mc{1}{c}{\scriptsize{(1.000)}} & \mc{1}{c}{\scriptsize{(0.987)}} \\  

     & \mc{1}{c}{\scriptsize{4}} & \mc{1}{c}{\scriptsize{0.608}} & \mc{1}{c}{\scriptsize{0.769}} & \mc{1}{c}{\scriptsize{1.048}} & \mc{1}{c}{\scriptsize{1.048}} & \mc{1}{c}{\scriptsize{0.906}} & \mc{1}{c}{\scriptsize{0.466}} & \mc{1}{c}{\scriptsize{0.774}} & \mc{1}{c}{\scriptsize{0.401}} \\  

     &  & \mc{1}{c}{\scriptsize{(1.000)}} & \mc{1}{c}{\scriptsize{(1.000)}} & \mc{1}{c}{\scriptsize{(1.000)}} & \mc{1}{c}{\scriptsize{(1.000)}} & \mc{1}{c}{\scriptsize{(1.000)}} & \mc{1}{c}{\scriptsize{(1.000)}} & \mc{1}{c}{\scriptsize{(1.000)}} & \mc{1}{c}{\scriptsize{(1.000)}} \\  

     & \mc{1}{c}{\scriptsize{5}} & \mc{1}{c}{\scriptsize{0.437}} & \mc{1}{c}{\scriptsize{0.873}} & \mc{1}{c}{\scriptsize{1.137}} & \mc{1}{c}{\scriptsize{1.010}} & \mc{1}{c}{\scriptsize{1.118}} & \mc{1}{c}{\scriptsize{0.241}} & \mc{1}{c}{\scriptsize{0.792}} & \mc{1}{c}{\scriptsize{0.336}} \\  

     &  & \mc{1}{c}{\scriptsize{(1.000)}} & \mc{1}{c}{\scriptsize{(1.000)}} & \mc{1}{c}{\scriptsize{(1.000)}} & \mc{1}{c}{\scriptsize{(1.000)}} & \mc{1}{c}{\scriptsize{(1.000)}} & \mc{1}{c}{\scriptsize{(0.987)}} & \mc{1}{c}{\scriptsize{(1.000)}} & \mc{1}{c}{\scriptsize{(1.000)}} \\  

     & \mc{1}{c}{\scriptsize{8}} & \mc{1}{c}{\scriptsize{0.959}} & \mc{1}{c}{\scriptsize{2.085}} & \mc{1}{c}{\scriptsize{1.474}} & \mc{1}{c}{\scriptsize{2.361}} & \mc{1}{c}{\scriptsize{1.446}} & \mc{1}{c}{\scriptsize{0.815}} & \mc{1}{c}{\scriptsize{2.086}} & \mc{1}{c}{\scriptsize{0.969}} \\  

     &  & \mc{1}{c}{\scriptsize{(1.000)}} & \mc{1}{c}{\scriptsize{(1.000)}} & \mc{1}{c}{\scriptsize{(1.000)}} & \mc{1}{c}{\scriptsize{(1.000)}} & \mc{1}{c}{\scriptsize{(1.000)}} & \mc{1}{c}{\scriptsize{(1.000)}} & \mc{1}{c}{\scriptsize{(1.000)}} & \mc{1}{c}{\scriptsize{(1.000)}} \\  

  \bottomrule
  \end{tabular}
	\end{table} 

	\begin{table}[H]
     \caption{Treatment Effects on Childhood Health Problems, Female Sample}
     \label{table:abccare_rslt_female_cat28_sd}
	\input{AppResOutput/abccare/rslt_female_cat28_sd}
	\end{table} 

	\begin{table}[H]
     \caption{Treatment Effects on Cholesterol, Female Sample}
     \label{table:abccare_rslt_female_cat29_sd}
	  \begin{tabular}{cccccccccc}
  \toprule

    \scriptsize{Variable} & \scriptsize{Age} & \scriptsize{(1)} & \scriptsize{(2)} & \scriptsize{(3)} & \scriptsize{(4)} & \scriptsize{(5)} & \scriptsize{(6)} & \scriptsize{(7)} & \scriptsize{(8)} \\ 
    \midrule  

    \mc{1}{l}{\scriptsize{Days drank alcohol last month}} & \mc{1}{c}{\scriptsize{30}} & \mc{1}{c}{\scriptsize{-0.742}} & \mc{1}{c}{\scriptsize{0.085}} & \mc{1}{c}{\scriptsize{-0.567}} & \mc{1}{c}{\scriptsize{0.274}} & \mc{1}{c}{\scriptsize{-0.261}} & \mc{1}{c}{\scriptsize{-0.918}} & \mc{1}{c}{\scriptsize{0.043}} & \mc{1}{c}{\scriptsize{-0.464}} \\  

     &  & \mc{1}{c}{\scriptsize{(0.355)}} & \mc{1}{c}{\scriptsize{(0.658)}} & \mc{1}{c}{\scriptsize{(0.421)}} & \mc{1}{c}{\scriptsize{(0.539)}} & \mc{1}{c}{\scriptsize{(0.487)}} & \mc{1}{c}{\scriptsize{(0.382)}} & \mc{1}{c}{\scriptsize{(0.645)}} & \mc{1}{c}{\scriptsize{(0.539)}} \\  

    \mc{1}{l}{\scriptsize{Days binge drank alcohol last month}} & \mc{1}{c}{\scriptsize{30}} & \mc{1}{c}{\scriptsize{-0.358}} & \mc{1}{c}{\scriptsize{0.216}} & \mc{1}{c}{\scriptsize{-1.062}} & \mc{1}{c}{\scriptsize{-0.070}} & \mc{1}{c}{\scriptsize{-0.922}} & \mc{1}{c}{\scriptsize{-0.232}} & \mc{1}{c}{\scriptsize{0.455}} & \mc{1}{c}{\scriptsize{0.034}} \\  

     &  & \mc{1}{c}{\scriptsize{(0.355)}} & \mc{1}{c}{\scriptsize{(0.724)}} & \mc{1}{c}{\scriptsize{(0.355)}} & \mc{1}{c}{\scriptsize{(0.526)}} & \mc{1}{c}{\scriptsize{(0.382)}} & \mc{1}{c}{\scriptsize{(0.500)}} & \mc{1}{c}{\scriptsize{(0.882)}} & \mc{1}{c}{\scriptsize{(0.697)}} \\  

  \bottomrule
  \end{tabular}
	\end{table} 

	\begin{table}[H]
     \caption{Treatment Effects on Current Health Condition (Self-Reported), Female Sample}
     \label{table:abccare_rslt_female_cat30_sd}
	  \begin{tabular}{cccccccccc}
  \toprule

    \scriptsize{Variable} & \scriptsize{Age} & \scriptsize{(1)} & \scriptsize{(2)} & \scriptsize{(3)} & \scriptsize{(4)} & \scriptsize{(5)} & \scriptsize{(6)} & \scriptsize{(7)} & \scriptsize{(8)} \\ 
    \midrule  

    \mc{1}{l}{\scriptsize{Asthma}} & \mc{1}{c}{\scriptsize{Mid-30s}} & \mc{1}{c}{\scriptsize{0.043}} &  & \mc{1}{c}{\scriptsize{0.043}} &  &  & \mc{1}{c}{\scriptsize{0.043}} &  &  \\  

     &  & \mc{1}{c}{\scriptsize{(1.000)}} &  & \mc{1}{c}{\scriptsize{(1.000)}} &  &  & \mc{1}{c}{\scriptsize{(0.987)}} &  &  \\  

    \mc{1}{l}{\scriptsize{High Blood Pressure (Hypertension)}} & \mc{1}{c}{\scriptsize{Mid-30s}} & \mc{1}{c}{\scriptsize{-0.036}} & \mc{1}{c}{\scriptsize{-0.060}} &  &  &  & \mc{1}{c}{\scriptsize{-0.045}} & \mc{1}{c}{\scriptsize{-0.078}} & \mc{1}{c}{\scriptsize{-0.053}} \\  

     &  & \mc{1}{c}{\scriptsize{(0.329)}} & \mc{1}{c}{\scriptsize{(0.461)}} &  &  &  & \mc{1}{c}{\scriptsize{(0.289)}} & \mc{1}{c}{\scriptsize{(0.382)}} & \mc{1}{c}{\scriptsize{(0.171)}} \\  

    \mc{1}{l}{\scriptsize{Arthritis or Generative Disease}} & \mc{1}{c}{\scriptsize{Mid-30s}} & \mc{1}{c}{\scriptsize{0.043}} & \mc{1}{c}{\scriptsize{0.051}} & \mc{1}{c}{\scriptsize{0.043}} & \mc{1}{c}{\scriptsize{0.029}} & \mc{1}{c}{\scriptsize{0.046}} & \mc{1}{c}{\scriptsize{0.043}} & \mc{1}{c}{\scriptsize{0.057}} & \mc{1}{c}{\scriptsize{0.046}} \\  

     &  & \mc{1}{c}{\scriptsize{(1.000)}} & \mc{1}{c}{\scriptsize{(1.000)}} & \mc{1}{c}{\scriptsize{(1.000)}} & \mc{1}{c}{\scriptsize{(0.974)}} & \mc{1}{c}{\scriptsize{(1.000)}} & \mc{1}{c}{\scriptsize{(1.000)}} & \mc{1}{c}{\scriptsize{(1.000)}} & \mc{1}{c}{\scriptsize{(1.000)}} \\  

    \mc{1}{l}{\scriptsize{Diabetes}} & \mc{1}{c}{\scriptsize{Mid-30s}} &  &  &  &  &  &  &  &  \\  

     &  &  &  &  &  &  &  &  &  \\  

  \bottomrule
  \end{tabular}
	\end{table} 

	\begin{table}[H]
     \caption{Treatment Effects on Diabetes, Female Sample}
     \label{table:abccare_rslt_female_cat31_sd}
	  \begin{tabular}{cccccccccc}
  \toprule

    \scriptsize{Variable} & \scriptsize{Age} & \scriptsize{(1)} & \scriptsize{(2)} & \scriptsize{(3)} & \scriptsize{(4)} & \scriptsize{(5)} & \scriptsize{(6)} & \scriptsize{(7)} & \scriptsize{(8)} \\ 
    \midrule  

    \mc{1}{l}{\scriptsize{Prediabetes}} & \mc{1}{c}{\scriptsize{Mid-30s}} & \mc{1}{c}{\scriptsize{0.088}} & \mc{1}{c}{\scriptsize{0.163}} & \mc{1}{c}{\scriptsize{0.076}} & \mc{1}{c}{\scriptsize{0.176}} & \mc{1}{c}{\scriptsize{0.035}} & \mc{1}{c}{\scriptsize{0.091}} & \mc{1}{c}{\scriptsize{0.161}} & \mc{1}{c}{\scriptsize{0.029}} \\  

     &  & \mc{1}{c}{\scriptsize{(0.882)}} & \mc{1}{c}{\scriptsize{(0.961)}} & \mc{1}{c}{\scriptsize{(0.711)}} & \mc{1}{c}{\scriptsize{(0.789)}} & \mc{1}{c}{\scriptsize{(0.618)}} & \mc{1}{c}{\scriptsize{(0.875)}} & \mc{1}{c}{\scriptsize{(0.938)}} & \mc{1}{c}{\scriptsize{(0.750)}} \\  

    \mc{1}{l}{\scriptsize{Hemoglobin Level (\%)}} & \mc{1}{c}{\scriptsize{Mid-30s}} & \mc{1}{c}{\scriptsize{-0.277}} & \mc{1}{c}{\scriptsize{-0.101}} & \mc{1}{c}{\scriptsize{-0.176}} & \mc{1}{c}{\scriptsize{-0.088}} & \mc{1}{c}{\scriptsize{-0.190}} & \mc{1}{c}{\scriptsize{-0.304}} & \mc{1}{c}{\scriptsize{-0.037}} & \mc{1}{c}{\scriptsize{-0.355}} \\  

     &  & \mc{1}{c}{\scriptsize{(0.355)}} & \mc{1}{c}{\scriptsize{(0.513)}} & \mc{1}{c}{\scriptsize{(0.132)}} & \mc{1}{c}{\scriptsize{(0.434)}} & \mc{1}{c}{\scriptsize{(0.158)}} & \mc{1}{c}{\scriptsize{(0.406)}} & \mc{1}{c}{\scriptsize{(0.719)}} & \mc{1}{c}{\scriptsize{(0.359)}} \\  

    \mc{1}{l}{\scriptsize{Diabetes}} & \mc{1}{c}{\scriptsize{Mid-30s}} & \mc{1}{c}{\scriptsize{-0.071}} & \mc{1}{c}{\scriptsize{-0.032}} &  &  &  & \mc{1}{c}{\scriptsize{-0.091}} & \mc{1}{c}{\scriptsize{-0.039}} & \mc{1}{c}{\scriptsize{-0.095}} \\  

     &  & \mc{1}{c}{\scriptsize{(0.145)}} & \mc{1}{c}{\scriptsize{(0.342)}} &  &  &  & \mc{1}{c}{\scriptsize{(0.141)}} & \mc{1}{c}{\scriptsize{(0.344)}} & \mc{1}{c}{\scriptsize{(0.125)}} \\  

  \bottomrule
  \end{tabular}
	\end{table} 

	\begin{table}[H]
     \caption{Treatment Effects on Drug Behavior and ASR Substance Scale, Female Sample}
     \label{table:abccare_rslt_female_cat32_sd}
	  \begin{tabular}{cccccccccc}
  \toprule

    \scriptsize{Variable} & \scriptsize{Age} & \scriptsize{(1)} & \scriptsize{(2)} & \scriptsize{(3)} & \scriptsize{(4)} & \scriptsize{(5)} & \scriptsize{(6)} & \scriptsize{(7)} & \scriptsize{(8)} \\ 
    \midrule  

    \mc{1}{l}{\scriptsize{Cocaine: Smokes Reguarly}} & \mc{1}{c}{\scriptsize{30}} & \mc{1}{c}{\scriptsize{-0.041}} & \mc{1}{c}{\scriptsize{-0.014}} & \mc{1}{c}{\scriptsize{-0.308}} & \mc{1}{c}{\scriptsize{-0.205}} & \mc{1}{c}{\scriptsize{-0.294}} & \mc{1}{c}{\scriptsize{0.030}} & \mc{1}{c}{\scriptsize{0.043}} & \mc{1}{c}{\scriptsize{0.042}} \\  

     &  & \mc{1}{c}{\scriptsize{(0.855)}} & \mc{1}{c}{\scriptsize{(0.974)}} & \mc{1}{c}{\scriptsize{(0.263)}} & \mc{1}{c}{\scriptsize{(0.592)}} & \mc{1}{c}{\scriptsize{(0.329)}} & \mc{1}{c}{\scriptsize{(0.974)}} & \mc{1}{c}{\scriptsize{(0.974)}} & \mc{1}{c}{\scriptsize{(0.974)}} \\  

    \mc{1}{l}{\scriptsize{Marijuana: Times Used}} & \mc{1}{c}{\scriptsize{30}} & \mc{1}{c}{\scriptsize{-0.194}} & \mc{1}{c}{\scriptsize{-0.137}} & \mc{1}{c}{\scriptsize{-1.792}} & \mc{1}{c}{\scriptsize{-0.645}} & \mc{1}{c}{\scriptsize{-1.861}} & \mc{1}{c}{\scriptsize{0.410}} & \mc{1}{c}{\scriptsize{0.005}} & \mc{1}{c}{\scriptsize{0.279}} \\  

     &  & \mc{1}{c}{\scriptsize{(0.895)}} & \mc{1}{c}{\scriptsize{(0.947)}} & \mc{1}{c}{\scriptsize{(0.105)}} & \mc{1}{c}{\scriptsize{(0.842)}} & \mc{1}{c}{\scriptsize{(0.105)}} & \mc{1}{c}{\scriptsize{(0.974)}} & \mc{1}{c}{\scriptsize{(0.921)}} & \mc{1}{c}{\scriptsize{(0.961)}} \\  

    \mc{1}{l}{\scriptsize{Marijuana: Smokes Regularly}} & \mc{1}{c}{\scriptsize{30}} & \mc{1}{c}{\scriptsize{-0.014}} & \mc{1}{c}{\scriptsize{-0.069}} & \mc{1}{c}{\scriptsize{-0.058}} & \mc{1}{c}{\scriptsize{-0.019}} & \mc{1}{c}{\scriptsize{-0.086}} & \mc{1}{c}{\scriptsize{-0.007}} & \mc{1}{c}{\scriptsize{-0.082}} & \mc{1}{c}{\scriptsize{-0.037}} \\  

     &  & \mc{1}{c}{\scriptsize{(0.934)}} & \mc{1}{c}{\scriptsize{(0.816)}} & \mc{1}{c}{\scriptsize{(0.908)}} & \mc{1}{c}{\scriptsize{(0.961)}} & \mc{1}{c}{\scriptsize{(0.842)}} & \mc{1}{c}{\scriptsize{(0.934)}} & \mc{1}{c}{\scriptsize{(0.645)}} & \mc{1}{c}{\scriptsize{(0.829)}} \\  

    \mc{1}{l}{\scriptsize{Cocaine: Times Used}} & \mc{1}{c}{\scriptsize{30}} & \mc{1}{c}{\scriptsize{-0.234}} & \mc{1}{c}{\scriptsize{-0.132}} & \mc{1}{c}{\scriptsize{-1.542}} & \mc{1}{c}{\scriptsize{-1.088}} & \mc{1}{c}{\scriptsize{-1.472}} & \mc{1}{c}{\scriptsize{0.111}} & \mc{1}{c}{\scriptsize{0.150}} & \mc{1}{c}{\scriptsize{0.169}} \\  

     &  & \mc{1}{c}{\scriptsize{(0.737)}} & \mc{1}{c}{\scriptsize{(0.947)}} & \mc{1}{c}{\scriptsize{(0.263)}} & \mc{1}{c}{\scriptsize{(0.566)}} & \mc{1}{c}{\scriptsize{(0.303)}} & \mc{1}{c}{\scriptsize{(0.974)}} & \mc{1}{c}{\scriptsize{(0.961)}} & \mc{1}{c}{\scriptsize{(0.974)}} \\  

    \mc{1}{l}{\scriptsize{Marijuana: Times Used in Past 30 Days}} & \mc{1}{c}{\scriptsize{30}} & \mc{1}{c}{\scriptsize{-0.005}} & \mc{1}{c}{\scriptsize{-0.156}} & \mc{1}{c}{\scriptsize{-0.100}} & \mc{1}{c}{\scriptsize{0.140}} & \mc{1}{c}{\scriptsize{-0.143}} & \mc{1}{c}{\scriptsize{-0.007}} & \mc{1}{c}{\scriptsize{-0.212}} & \mc{1}{c}{\scriptsize{-0.065}} \\  

     &  & \mc{1}{c}{\scriptsize{(0.974)}} & \mc{1}{c}{\scriptsize{(0.908)}} & \mc{1}{c}{\scriptsize{(0.961)}} & \mc{1}{c}{\scriptsize{(1.000)}} & \mc{1}{c}{\scriptsize{(0.921)}} & \mc{1}{c}{\scriptsize{(0.934)}} & \mc{1}{c}{\scriptsize{(0.750)}} & \mc{1}{c}{\scriptsize{(0.895)}} \\  

    \mc{1}{l}{\scriptsize{Times Used Other Illegal Drugs in Past 30 Days}} & \mc{1}{c}{\scriptsize{21}} & \mc{1}{c}{\scriptsize{0.039}} & \mc{1}{c}{\scriptsize{0.064}} & \mc{1}{c}{\scriptsize{-0.058}} & \mc{1}{c}{\scriptsize{0.048}} & \mc{1}{c}{\scriptsize{-0.044}} & \mc{1}{c}{\scriptsize{0.067}} & \mc{1}{c}{\scriptsize{0.065}} & \mc{1}{c}{\scriptsize{0.080}} \\  

     &  & \mc{1}{c}{\scriptsize{(1.000)}} & \mc{1}{c}{\scriptsize{(1.000)}} & \mc{1}{c}{\scriptsize{(0.908)}} & \mc{1}{c}{\scriptsize{(1.000)}} & \mc{1}{c}{\scriptsize{(0.921)}} & \mc{1}{c}{\scriptsize{(1.000)}} & \mc{1}{c}{\scriptsize{(1.000)}} & \mc{1}{c}{\scriptsize{(1.000)}} \\  

    \mc{1}{l}{\scriptsize{ASR Substance Use Scale: Alcohol}} & \mc{1}{c}{\scriptsize{30}} & \mc{1}{c}{\scriptsize{1.354}} & \mc{1}{c}{\scriptsize{2.344}} & \mc{1}{c}{\scriptsize{0.675}} & \mc{1}{c}{\scriptsize{4.520}} & \mc{1}{c}{\scriptsize{0.929}} & \mc{1}{c}{\scriptsize{1.819}} & \mc{1}{c}{\scriptsize{2.205}} & \mc{1}{c}{\scriptsize{2.021}} \\  

     &  & \mc{1}{c}{\scriptsize{(1.000)}} & \mc{1}{c}{\scriptsize{(1.000)}} & \mc{1}{c}{\scriptsize{(1.000)}} & \mc{1}{c}{\scriptsize{(1.000)}} & \mc{1}{c}{\scriptsize{(1.000)}} & \mc{1}{c}{\scriptsize{(0.987)}} & \mc{1}{c}{\scriptsize{(1.000)}} & \mc{1}{c}{\scriptsize{(0.987)}} \\  

    \mc{1}{l}{\scriptsize{Marijuana: Times Used in Past 30 Days}} & \mc{1}{c}{\scriptsize{21}} & \mc{1}{c}{\scriptsize{-0.828}} & \mc{1}{c}{\scriptsize{-0.689}} & \mc{1}{c}{\scriptsize{-0.800}} & \mc{1}{c}{\scriptsize{-0.499}} & \mc{1}{c}{\scriptsize{-0.800}} & \mc{1}{c}{\scriptsize{-0.839}} & \mc{1}{c}{\scriptsize{-0.811}} & \mc{1}{c}{\scriptsize{-0.762}} \\  

     &  & \mc{1}{c}{\scriptsize{\textbf{(0.013)}}} & \mc{1}{c}{\scriptsize{(0.237)}} & \mc{1}{c}{\scriptsize{(0.526)}} & \mc{1}{c}{\scriptsize{(0.776)}} & \mc{1}{c}{\scriptsize{(0.579)}} & \mc{1}{c}{\scriptsize{\textbf{(0.026)}}} & \mc{1}{c}{\scriptsize{(0.211)}} & \mc{1}{c}{\scriptsize{\textbf{(0.066)}}} \\  

    \mc{1}{l}{\scriptsize{ASR Substance Use Scale: Mean Substance Abuse}} & \mc{1}{c}{\scriptsize{30}} & \mc{1}{c}{\scriptsize{1.185}} & \mc{1}{c}{\scriptsize{2.239}} & \mc{1}{c}{\scriptsize{1.271}} & \mc{1}{c}{\scriptsize{6.160}} & \mc{1}{c}{\scriptsize{1.620}} & \mc{1}{c}{\scriptsize{1.081}} & \mc{1}{c}{\scriptsize{1.707}} & \mc{1}{c}{\scriptsize{1.409}} \\  

     &  & \mc{1}{c}{\scriptsize{(1.000)}} & \mc{1}{c}{\scriptsize{(1.000)}} & \mc{1}{c}{\scriptsize{(1.000)}} & \mc{1}{c}{\scriptsize{(1.000)}} & \mc{1}{c}{\scriptsize{(1.000)}} & \mc{1}{c}{\scriptsize{(0.974)}} & \mc{1}{c}{\scriptsize{(1.000)}} & \mc{1}{c}{\scriptsize{(0.987)}} \\  

    \mc{1}{l}{\scriptsize{Cocaine: Number of Times Used Crack Cocaine}} & \mc{1}{c}{\scriptsize{30}} & \mc{1}{c}{\scriptsize{-0.299}} & \mc{1}{c}{\scriptsize{-0.133}} & \mc{1}{c}{\scriptsize{-1.742}} & \mc{1}{c}{\scriptsize{-1.298}} & \mc{1}{c}{\scriptsize{-1.712}} & \mc{1}{c}{\scriptsize{0.096}} & \mc{1}{c}{\scriptsize{0.202}} & \mc{1}{c}{\scriptsize{0.122}} \\  

     &  & \mc{1}{c}{\scriptsize{(0.632)}} & \mc{1}{c}{\scriptsize{(0.947)}} & \mc{1}{c}{\scriptsize{(0.184)}} & \mc{1}{c}{\scriptsize{(0.368)}} & \mc{1}{c}{\scriptsize{(0.197)}} & \mc{1}{c}{\scriptsize{(0.974)}} & \mc{1}{c}{\scriptsize{(1.000)}} & \mc{1}{c}{\scriptsize{(0.987)}} \\  

    \mc{1}{l}{\scriptsize{ASR Substance Use Scale: Tobacco}} & \mc{1}{c}{\scriptsize{30}} & \mc{1}{c}{\scriptsize{-0.778}} & \mc{1}{c}{\scriptsize{-0.382}} & \mc{1}{c}{\scriptsize{-3.125}} & \mc{1}{c}{\scriptsize{-1.206}} & \mc{1}{c}{\scriptsize{-3.033}} & \mc{1}{c}{\scriptsize{-0.204}} & \mc{1}{c}{\scriptsize{0.296}} & \mc{1}{c}{\scriptsize{-0.019}} \\  

     &  & \mc{1}{c}{\scriptsize{(0.737)}} & \mc{1}{c}{\scriptsize{(0.947)}} & \mc{1}{c}{\scriptsize{(0.329)}} & \mc{1}{c}{\scriptsize{(0.908)}} & \mc{1}{c}{\scriptsize{(0.329)}} & \mc{1}{c}{\scriptsize{(0.895)}} & \mc{1}{c}{\scriptsize{(0.961)}} & \mc{1}{c}{\scriptsize{(0.921)}} \\  

    \mc{1}{l}{\scriptsize{Marijuana: Smokes Regularly}} & \mc{1}{c}{\scriptsize{Mid-30s}} & \mc{1}{c}{\scriptsize{0.092}} & \mc{1}{c}{\scriptsize{-0.023}} & \mc{1}{c}{\scriptsize{0.017}} & \mc{1}{c}{\scriptsize{-0.132}} & \mc{1}{c}{\scriptsize{0.022}} & \mc{1}{c}{\scriptsize{0.112}} & \mc{1}{c}{\scriptsize{0.030}} & \mc{1}{c}{\scriptsize{0.113}} \\  

     &  & \mc{1}{c}{\scriptsize{(1.000)}} & \mc{1}{c}{\scriptsize{(0.974)}} & \mc{1}{c}{\scriptsize{(1.000)}} & \mc{1}{c}{\scriptsize{(0.855)}} & \mc{1}{c}{\scriptsize{(1.000)}} & \mc{1}{c}{\scriptsize{(0.987)}} & \mc{1}{c}{\scriptsize{(0.947)}} & \mc{1}{c}{\scriptsize{(0.987)}} \\  

    \mc{1}{l}{\scriptsize{ASR Substance Use Scale: Drugs}} & \mc{1}{c}{\scriptsize{30}} & \mc{1}{c}{\scriptsize{1.463}} & \mc{1}{c}{\scriptsize{2.449}} & \mc{1}{c}{\scriptsize{2.379}} & \mc{1}{c}{\scriptsize{7.915}} & \mc{1}{c}{\scriptsize{2.760}} & \mc{1}{c}{\scriptsize{1.157}} & \mc{1}{c}{\scriptsize{1.482}} & \mc{1}{c}{\scriptsize{1.480}} \\  

     &  & \mc{1}{c}{\scriptsize{(1.000)}} & \mc{1}{c}{\scriptsize{(1.000)}} & \mc{1}{c}{\scriptsize{(1.000)}} & \mc{1}{c}{\scriptsize{(1.000)}} & \mc{1}{c}{\scriptsize{(1.000)}} & \mc{1}{c}{\scriptsize{(0.974)}} & \mc{1}{c}{\scriptsize{(0.987)}} & \mc{1}{c}{\scriptsize{(0.974)}} \\  

  \bottomrule
  \end{tabular}
	\end{table} 

	\begin{table}[H]
     \caption{Treatment Effects on Health Insurance, Female Sample}
     \label{table:abccare_rslt_female_cat33_sd}
	  \begin{tabular}{cccccccccc}
  \toprule

    \scriptsize{Variable} & \scriptsize{Age} & \scriptsize{(1)} & \scriptsize{(2)} & \scriptsize{(3)} & \scriptsize{(4)} & \scriptsize{(5)} & \scriptsize{(6)} & \scriptsize{(7)} & \scriptsize{(8)} \\ 
    \midrule  

    \mc{1}{l}{\scriptsize{Has Health Insurance}} & \mc{1}{c}{\scriptsize{30}} & \mc{1}{c}{\scriptsize{-0.090}} & \mc{1}{c}{\scriptsize{-0.087}} & \mc{1}{c}{\scriptsize{-0.083}} & \mc{1}{c}{\scriptsize{-0.195}} & \mc{1}{c}{\scriptsize{-0.070}} & \mc{1}{c}{\scriptsize{-0.074}} & \mc{1}{c}{\scriptsize{-0.094}} & \mc{1}{c}{\scriptsize{-0.051}} \\  

     &  & \mc{1}{c}{\scriptsize{(0.934)}} & \mc{1}{c}{\scriptsize{(0.895)}} & \mc{1}{c}{\scriptsize{(0.776)}} & \mc{1}{c}{\scriptsize{(0.908)}} & \mc{1}{c}{\scriptsize{(0.816)}} & \mc{1}{c}{\scriptsize{(0.875)}} & \mc{1}{c}{\scriptsize{(0.906)}} & \mc{1}{c}{\scriptsize{(0.812)}} \\  

     & \mc{1}{c}{\scriptsize{21}} & \mc{1}{c}{\scriptsize{0.285}} & \mc{1}{c}{\scriptsize{0.278}} & \mc{1}{c}{\scriptsize{0.276}} & \mc{1}{c}{\scriptsize{0.299}} & \mc{1}{c}{\scriptsize{0.289}} & \mc{1}{c}{\scriptsize{0.231}} & \mc{1}{c}{\scriptsize{0.244}} & \mc{1}{c}{\scriptsize{0.310}} \\  

     &  & \mc{1}{c}{\scriptsize{\textbf{(0.013)}}} & \mc{1}{c}{\scriptsize{\textbf{(0.066)}}} & \mc{1}{c}{\scriptsize{\textbf{(0.092)}}} & \mc{1}{c}{\scriptsize{(0.224)}} & \mc{1}{c}{\scriptsize{(0.132)}} & \mc{1}{c}{\scriptsize{\textbf{(0.047)}}} & \mc{1}{c}{\scriptsize{(0.141)}} & \mc{1}{c}{\scriptsize{\textbf{(0.016)}}} \\  

  \bottomrule
  \end{tabular}
	\end{table} 

	\begin{table}[H]
     \caption{Treatment Effects on Hypertension, Female Sample}
     \label{table:abccare_rslt_female_cat34_sd}
	\input{AppResOutput/abccare/rslt_female_cat34_sd}
	\end{table} 

	\begin{table}[H]
     \caption{Treatment Effects on Laboratory Test  - Metabolic Panel, Female Sample}
     \label{table:abccare_rslt_female_cat35_sd}
	\input{AppResOutput/abccare/rslt_female_cat35_sd}
	\end{table} 

	\begin{table}[H]
     \caption{Treatment Effects on Laboratory Test - Complete Blood Count, Female Sample}
     \label{table:abccare_rslt_female_cat36_sd}
	\input{AppResOutput/abccare/rslt_female_cat36_sd}
	\end{table} 

	\begin{table}[H]
     \caption{Treatment Effects on Other Health-Related Information, Female Sample}
     \label{table:abccare_rslt_female_cat37_sd}
	  \begin{tabular}{cccccccccc}
  \toprule

    \scriptsize{Variable} & \scriptsize{Age} & \scriptsize{(1)} & \scriptsize{(2)} & \scriptsize{(3)} & \scriptsize{(4)} & \scriptsize{(5)} & \scriptsize{(6)} & \scriptsize{(7)} & \scriptsize{(8)} \\ 
    \midrule  

    \mc{1}{l}{\scriptsize{Number of Days Very Healthy in Past 30 Days}} & \mc{1}{c}{\scriptsize{Mid-30s}} & \mc{1}{c}{\scriptsize{-2.298}} & \mc{1}{c}{\scriptsize{-3.546}} & \mc{1}{c}{\scriptsize{-2.536}} & \mc{1}{c}{\scriptsize{-0.042}} & \mc{1}{c}{\scriptsize{-3.447}} & \mc{1}{c}{\scriptsize{-2.233}} & \mc{1}{c}{\scriptsize{-5.391}} & \mc{1}{c}{\scriptsize{-4.818}} \\  

     &  & \mc{1}{c}{\scriptsize{(1.000)}} & \mc{1}{c}{\scriptsize{(0.987)}} & \mc{1}{c}{\scriptsize{(0.987)}} & \mc{1}{c}{\scriptsize{(0.855)}} & \mc{1}{c}{\scriptsize{(0.987)}} & \mc{1}{c}{\scriptsize{(0.961)}} & \mc{1}{c}{\scriptsize{(1.000)}} & \mc{1}{c}{\scriptsize{(1.000)}} \\  

    \mc{1}{l}{\scriptsize{How Subject Thinks of Own Weight}} & \mc{1}{c}{\scriptsize{30}} & \mc{1}{c}{\scriptsize{0.456}} & \mc{1}{c}{\scriptsize{0.618}} & \mc{1}{c}{\scriptsize{0.267}} & \mc{1}{c}{\scriptsize{0.498}} & \mc{1}{c}{\scriptsize{0.361}} & \mc{1}{c}{\scriptsize{0.526}} & \mc{1}{c}{\scriptsize{0.604}} & \mc{1}{c}{\scriptsize{0.591}} \\  

     &  & \mc{1}{c}{\scriptsize{(0.974)}} & \mc{1}{c}{\scriptsize{(1.000)}} & \mc{1}{c}{\scriptsize{(0.776)}} & \mc{1}{c}{\scriptsize{(0.947)}} & \mc{1}{c}{\scriptsize{(0.868)}} & \mc{1}{c}{\scriptsize{(0.829)}} & \mc{1}{c}{\scriptsize{(0.842)}} & \mc{1}{c}{\scriptsize{(0.842)}} \\  

    \mc{1}{l}{\scriptsize{Number of Days in Pain in Past 30 Days}} & \mc{1}{c}{\scriptsize{Mid-30s}} & \mc{1}{c}{\scriptsize{1.882}} & \mc{1}{c}{\scriptsize{1.860}} & \mc{1}{c}{\scriptsize{-0.427}} & \mc{1}{c}{\scriptsize{0.885}} & \mc{1}{c}{\scriptsize{-1.254}} & \mc{1}{c}{\scriptsize{2.512}} & \mc{1}{c}{\scriptsize{2.246}} & \mc{1}{c}{\scriptsize{1.410}} \\  

     &  & \mc{1}{c}{\scriptsize{(1.000)}} & \mc{1}{c}{\scriptsize{(0.987)}} & \mc{1}{c}{\scriptsize{(0.947)}} & \mc{1}{c}{\scriptsize{(0.934)}} & \mc{1}{c}{\scriptsize{(0.658)}} & \mc{1}{c}{\scriptsize{(0.987)}} & \mc{1}{c}{\scriptsize{(1.000)}} & \mc{1}{c}{\scriptsize{(0.974)}} \\  

    \mc{1}{l}{\scriptsize{Physical/Nervous Condition Prevents Work}} & \mc{1}{c}{\scriptsize{30}} & \mc{1}{c}{\scriptsize{-0.048}} & \mc{1}{c}{\scriptsize{-0.013}} & \mc{1}{c}{\scriptsize{-0.217}} & \mc{1}{c}{\scriptsize{-0.108}} & \mc{1}{c}{\scriptsize{-0.209}} & \mc{1}{c}{\scriptsize{-0.004}} & \mc{1}{c}{\scriptsize{0.011}} & \mc{1}{c}{\scriptsize{0.001}} \\  

     &  & \mc{1}{c}{\scriptsize{(0.158)}} & \mc{1}{c}{\scriptsize{(0.461)}} & \mc{1}{c}{\scriptsize{\textbf{(0.092)}}} & \mc{1}{c}{\scriptsize{(0.237)}} & \mc{1}{c}{\scriptsize{(0.118)}} & \mc{1}{c}{\scriptsize{(0.382)}} & \mc{1}{c}{\scriptsize{(0.395)}} & \mc{1}{c}{\scriptsize{(0.382)}} \\  

  \bottomrule
  \end{tabular}
	\end{table} 

	\begin{table}[H]
     \caption{Treatment Effects on Past Medical History - Diagnosis (Self-Reported), Female Sample}
     \label{table:abccare_rslt_female_cat38_sd}
	\input{AppResOutput/abccare/rslt_female_cat38_sd}
	\end{table} 

	\begin{table}[H]
     \caption{Treatment Effects on Past Medical History - Surgery (Self-Reported), Female Sample}
     \label{table:abccare_rslt_female_cat39_sd}
	  \begin{tabular}{cccccccccc}
  \toprule

    \scriptsize{Variable} & \scriptsize{Age} & \scriptsize{(1)} & \scriptsize{(2)} & \scriptsize{(3)} & \scriptsize{(4)} & \scriptsize{(5)} & \scriptsize{(6)} & \scriptsize{(7)} & \scriptsize{(8)} \\ 
    \midrule  

    \mc{1}{l}{\scriptsize{Past Surgery: Cholecystectomy}} & \mc{1}{c}{\scriptsize{Mid-30s}} & \mc{1}{c}{\scriptsize{0.051}} & \mc{1}{c}{\scriptsize{0.014}} & \mc{1}{c}{\scriptsize{0.087}} & \mc{1}{c}{\scriptsize{0.087}} & \mc{1}{c}{\scriptsize{0.096}} & \mc{1}{c}{\scriptsize{0.042}} & \mc{1}{c}{\scriptsize{-0.002}} & \mc{1}{c}{\scriptsize{0.028}} \\  

     &  & \mc{1}{c}{\scriptsize{(1.000)}} & \mc{1}{c}{\scriptsize{(0.921)}} & \mc{1}{c}{\scriptsize{(1.000)}} & \mc{1}{c}{\scriptsize{(0.987)}} & \mc{1}{c}{\scriptsize{(1.000)}} & \mc{1}{c}{\scriptsize{(1.000)}} & \mc{1}{c}{\scriptsize{(0.908)}} & \mc{1}{c}{\scriptsize{(0.974)}} \\  

    \mc{1}{l}{\scriptsize{Past Surgery: Orthopedic Surgery}} & \mc{1}{c}{\scriptsize{Mid-30s}} &  &  &  &  &  &  &  &  \\  

     &  &  &  &  &  &  &  &  &  \\  

    \mc{1}{l}{\scriptsize{Past Surgery: Appendectomy}} & \mc{1}{c}{\scriptsize{Mid-30s}} & \mc{1}{c}{\scriptsize{-0.028}} & \mc{1}{c}{\scriptsize{0.012}} & \mc{1}{c}{\scriptsize{-0.123}} & \mc{1}{c}{\scriptsize{-0.021}} & \mc{1}{c}{\scriptsize{-0.113}} & \mc{1}{c}{\scriptsize{-0.002}} & \mc{1}{c}{\scriptsize{0.054}} & \mc{1}{c}{\scriptsize{0.002}} \\  

     &  & \mc{1}{c}{\scriptsize{(0.724)}} & \mc{1}{c}{\scriptsize{(0.921)}} & \mc{1}{c}{\scriptsize{(0.553)}} & \mc{1}{c}{\scriptsize{(0.868)}} & \mc{1}{c}{\scriptsize{(0.553)}} & \mc{1}{c}{\scriptsize{(0.908)}} & \mc{1}{c}{\scriptsize{(1.000)}} & \mc{1}{c}{\scriptsize{(0.921)}} \\  

    \mc{1}{l}{\scriptsize{Past Surgery: Ectopic Pregnancy}} & \mc{1}{c}{\scriptsize{Mid-30s}} & \mc{1}{c}{\scriptsize{0.008}} & \mc{1}{c}{\scriptsize{0.036}} & \mc{1}{c}{\scriptsize{0.043}} & \mc{1}{c}{\scriptsize{0.065}} & \mc{1}{c}{\scriptsize{0.051}} & \mc{1}{c}{\scriptsize{-0.002}} & \mc{1}{c}{\scriptsize{0.040}} & \mc{1}{c}{\scriptsize{0.002}} \\  

     &  & \mc{1}{c}{\scriptsize{(0.908)}} & \mc{1}{c}{\scriptsize{(0.987)}} & \mc{1}{c}{\scriptsize{(0.961)}} & \mc{1}{c}{\scriptsize{(0.974)}} & \mc{1}{c}{\scriptsize{(0.987)}} & \mc{1}{c}{\scriptsize{(0.908)}} & \mc{1}{c}{\scriptsize{(0.987)}} & \mc{1}{c}{\scriptsize{(0.921)}} \\  

    \mc{1}{l}{\scriptsize{Past Surgery: Hysterectomy}} & \mc{1}{c}{\scriptsize{Mid-30s}} & \mc{1}{c}{\scriptsize{0.008}} & \mc{1}{c}{\scriptsize{0.020}} & \mc{1}{c}{\scriptsize{-0.123}} & \mc{1}{c}{\scriptsize{-0.044}} & \mc{1}{c}{\scriptsize{-0.114}} & \mc{1}{c}{\scriptsize{0.043}} & \mc{1}{c}{\scriptsize{0.069}} & \mc{1}{c}{\scriptsize{0.050}} \\  

     &  & \mc{1}{c}{\scriptsize{(0.908)}} & \mc{1}{c}{\scriptsize{(0.934)}} & \mc{1}{c}{\scriptsize{(0.553)}} & \mc{1}{c}{\scriptsize{(0.829)}} & \mc{1}{c}{\scriptsize{(0.553)}} & \mc{1}{c}{\scriptsize{(1.000)}} & \mc{1}{c}{\scriptsize{(1.000)}} & \mc{1}{c}{\scriptsize{(1.000)}} \\  

  \bottomrule
  \end{tabular}
	\end{table} 

	\begin{table}[H]
     \caption{Treatment Effects on Physical Activity, Female Sample}
     \label{table:abccare_rslt_female_cat40_sd}
	  \begin{tabular}{cccccccccc}
  \toprule

    \scriptsize{Variable} & \scriptsize{Age} & \scriptsize{(1)} & \scriptsize{(2)} & \scriptsize{(3)} & \scriptsize{(4)} & \scriptsize{(5)} & \scriptsize{(6)} & \scriptsize{(7)} & \scriptsize{(8)} \\ 
    \midrule  

    \mc{1}{l}{\scriptsize{Level of Activity at Work}} & \mc{1}{c}{\scriptsize{Mid-30s}} & \mc{1}{c}{\scriptsize{0.300}} & \mc{1}{c}{\scriptsize{0.194}} & \mc{1}{c}{\scriptsize{-0.500}} & \mc{1}{c}{\scriptsize{-1.008}} & \mc{1}{c}{\scriptsize{-0.583}} & \mc{1}{c}{\scriptsize{0.357}} & \mc{1}{c}{\scriptsize{0.292}} & \mc{1}{c}{\scriptsize{0.228}} \\  

     &  & \mc{1}{c}{\scriptsize{(0.133)}} & \mc{1}{c}{\scriptsize{(0.197)}} & \mc{1}{c}{\scriptsize{(1.000)}} & \mc{1}{c}{\scriptsize{(0.826)}} & \mc{1}{c}{\scriptsize{(1.000)}} & \mc{1}{c}{\scriptsize{\textbf{(0.079)}}} & \mc{1}{c}{\scriptsize{(0.132)}} & \mc{1}{c}{\scriptsize{(0.105)}} \\  

  \bottomrule
  \end{tabular}
	\end{table} 

	\begin{table}[H]
     \caption{Treatment Effects on Physical Exam - Ear, Female Sample}
     \label{table:abccare_rslt_female_cat41_sd}
	\input{AppResOutput/abccare/rslt_female_cat41_sd}
	\end{table} 

	\begin{table}[H]
     \caption{Treatment Effects on Physical Exam - General I, Female Sample}
     \label{table:abccare_rslt_female_cat42_sd}
	  \begin{tabular}{cccccccccc}
  \toprule

    \scriptsize{Variable} & \scriptsize{Age} & \scriptsize{(1)} & \scriptsize{(2)} & \scriptsize{(3)} & \scriptsize{(4)} & \scriptsize{(5)} & \scriptsize{(6)} & \scriptsize{(7)} & \scriptsize{(8)} \\ 
    \midrule  

    \mc{1}{l}{\scriptsize{Respirations}} & \mc{1}{c}{\scriptsize{Mid-30s}} & \mc{1}{c}{\scriptsize{-0.555}} & \mc{1}{c}{\scriptsize{-0.341}} & \mc{1}{c}{\scriptsize{-1.286}} & \mc{1}{c}{\scriptsize{-1.240}} & \mc{1}{c}{\scriptsize{-1.379}} & \mc{1}{c}{\scriptsize{-0.336}} & \mc{1}{c}{\scriptsize{0.151}} & \mc{1}{c}{\scriptsize{-0.010}} \\  

     &  & \mc{1}{c}{\scriptsize{(0.500)}} & \mc{1}{c}{\scriptsize{(0.737)}} & \mc{1}{c}{\scriptsize{\textbf{(0.000)}}} & \mc{1}{c}{\scriptsize{(0.211)}} & \mc{1}{c}{\scriptsize{\textbf{(0.000)}}} & \mc{1}{c}{\scriptsize{(0.855)}} & \mc{1}{c}{\scriptsize{(0.961)}} & \mc{1}{c}{\scriptsize{(0.974)}} \\  

    \mc{1}{l}{\scriptsize{Temp (F)}} & \mc{1}{c}{\scriptsize{Mid-30s}} & \mc{1}{c}{\scriptsize{0.185}} & \mc{1}{c}{\scriptsize{0.161}} & \mc{1}{c}{\scriptsize{0.062}} & \mc{1}{c}{\scriptsize{-0.006}} & \mc{1}{c}{\scriptsize{0.045}} & \mc{1}{c}{\scriptsize{0.219}} & \mc{1}{c}{\scriptsize{0.218}} & \mc{1}{c}{\scriptsize{0.187}} \\  

     &  & \mc{1}{c}{\scriptsize{(1.000)}} & \mc{1}{c}{\scriptsize{(1.000)}} & \mc{1}{c}{\scriptsize{(0.987)}} & \mc{1}{c}{\scriptsize{(0.921)}} & \mc{1}{c}{\scriptsize{(0.961)}} & \mc{1}{c}{\scriptsize{(1.000)}} & \mc{1}{c}{\scriptsize{(1.000)}} & \mc{1}{c}{\scriptsize{(1.000)}} \\  

    \mc{1}{l}{\scriptsize{Pulse}} & \mc{1}{c}{\scriptsize{Mid-30s}} & \mc{1}{c}{\scriptsize{-1.890}} & \mc{1}{c}{\scriptsize{-4.263}} & \mc{1}{c}{\scriptsize{-11.318}} & \mc{1}{c}{\scriptsize{-12.480}} & \mc{1}{c}{\scriptsize{-11.960}} & \mc{1}{c}{\scriptsize{0.682}} & \mc{1}{c}{\scriptsize{-1.009}} & \mc{1}{c}{\scriptsize{0.867}} \\  

     &  & \mc{1}{c}{\scriptsize{(0.763)}} & \mc{1}{c}{\scriptsize{(0.421)}} & \mc{1}{c}{\scriptsize{\textbf{(0.039)}}} & \mc{1}{c}{\scriptsize{(0.224)}} & \mc{1}{c}{\scriptsize{\textbf{(0.053)}}} & \mc{1}{c}{\scriptsize{(0.987)}} & \mc{1}{c}{\scriptsize{(0.855)}} & \mc{1}{c}{\scriptsize{(1.000)}} \\  

    \mc{1}{l}{\scriptsize{Voice}} & \mc{1}{c}{\scriptsize{Mid-30s}} &  &  &  &  &  &  &  &  \\  

     &  &  &  &  &  &  &  &  &  \\  

    \mc{1}{l}{\scriptsize{Orientation}} & \mc{1}{c}{\scriptsize{Mid-30s}} &  &  &  &  &  &  &  &  \\  

     &  &  &  &  &  &  &  &  &  \\  

    \mc{1}{l}{\scriptsize{Nutrition}} & \mc{1}{c}{\scriptsize{Mid-30s}} & \mc{1}{c}{\scriptsize{0.062}} & \mc{1}{c}{\scriptsize{0.227}} & \mc{1}{c}{\scriptsize{0.015}} & \mc{1}{c}{\scriptsize{0.136}} & \mc{1}{c}{\scriptsize{0.026}} & \mc{1}{c}{\scriptsize{0.075}} & \mc{1}{c}{\scriptsize{0.240}} & \mc{1}{c}{\scriptsize{0.157}} \\  

     &  & \mc{1}{c}{\scriptsize{(1.000)}} & \mc{1}{c}{\scriptsize{(1.000)}} & \mc{1}{c}{\scriptsize{(0.961)}} & \mc{1}{c}{\scriptsize{(1.000)}} & \mc{1}{c}{\scriptsize{(0.961)}} & \mc{1}{c}{\scriptsize{(1.000)}} & \mc{1}{c}{\scriptsize{(1.000)}} & \mc{1}{c}{\scriptsize{(1.000)}} \\  

    \mc{1}{l}{\scriptsize{Hydration}} & \mc{1}{c}{\scriptsize{Mid-30s}} &  &  &  &  &  &  &  &  \\  

     &  &  &  &  &  &  &  &  &  \\  

    \mc{1}{l}{\scriptsize{Posture}} & \mc{1}{c}{\scriptsize{Mid-30s}} & \mc{1}{c}{\scriptsize{0.087}} & \mc{1}{c}{\scriptsize{0.147}} & \mc{1}{c}{\scriptsize{0.087}} & \mc{1}{c}{\scriptsize{0.026}} & \mc{1}{c}{\scriptsize{0.100}} & \mc{1}{c}{\scriptsize{0.087}} & \mc{1}{c}{\scriptsize{0.173}} & \mc{1}{c}{\scriptsize{0.101}} \\  

     &  & \mc{1}{c}{\scriptsize{(1.000)}} & \mc{1}{c}{\scriptsize{(1.000)}} & \mc{1}{c}{\scriptsize{(1.000)}} & \mc{1}{c}{\scriptsize{(0.961)}} & \mc{1}{c}{\scriptsize{(1.000)}} & \mc{1}{c}{\scriptsize{(1.000)}} & \mc{1}{c}{\scriptsize{(1.000)}} & \mc{1}{c}{\scriptsize{(1.000)}} \\  

  \bottomrule
  \end{tabular}
	\end{table} 

	\begin{table}[H]
     \caption{Treatment Effects on Physical Exam - General II, Female Sample}
     \label{table:abccare_rslt_female_cat43_sd}
	  \begin{tabular}{cccccccccc}
  \toprule

    \scriptsize{Variable} & \scriptsize{Age} & \scriptsize{(1)} & \scriptsize{(2)} & \scriptsize{(3)} & \scriptsize{(4)} & \scriptsize{(5)} & \scriptsize{(6)} & \scriptsize{(7)} & \scriptsize{(8)} \\ 
    \midrule  

    \mc{1}{l}{\scriptsize{Chest and Lung General}} & \mc{1}{c}{\scriptsize{Mid-30s}} & \mc{1}{c}{\scriptsize{0.087}} & \mc{1}{c}{\scriptsize{0.112}} & \mc{1}{c}{\scriptsize{0.087}} & \mc{1}{c}{\scriptsize{0.011}} & \mc{1}{c}{\scriptsize{0.102}} & \mc{1}{c}{\scriptsize{0.087}} & \mc{1}{c}{\scriptsize{0.139}} & \mc{1}{c}{\scriptsize{0.103}} \\  

     &  & \mc{1}{c}{\scriptsize{(0.987)}} & \mc{1}{c}{\scriptsize{(0.974)}} & \mc{1}{c}{\scriptsize{(0.947)}} & \mc{1}{c}{\scriptsize{(0.829)}} & \mc{1}{c}{\scriptsize{(0.934)}} & \mc{1}{c}{\scriptsize{(1.000)}} & \mc{1}{c}{\scriptsize{(0.961)}} & \mc{1}{c}{\scriptsize{(0.961)}} \\  

    \mc{1}{l}{\scriptsize{Cardiovascular General}} & \mc{1}{c}{\scriptsize{Mid-30s}} & \mc{1}{c}{\scriptsize{-0.036}} &  &  &  &  & \mc{1}{c}{\scriptsize{-0.045}} &  &  \\  

     &  & \mc{1}{c}{\scriptsize{(0.237)}} &  &  &  &  & \mc{1}{c}{\scriptsize{(0.184)}} &  &  \\  

    \mc{1}{l}{\scriptsize{Skin General}} & \mc{1}{c}{\scriptsize{Mid-30s}} & \mc{1}{c}{\scriptsize{-0.056}} & \mc{1}{c}{\scriptsize{-0.072}} & \mc{1}{c}{\scriptsize{-0.246}} & \mc{1}{c}{\scriptsize{-0.177}} & \mc{1}{c}{\scriptsize{-0.233}} & \mc{1}{c}{\scriptsize{-0.004}} & \mc{1}{c}{\scriptsize{-0.054}} & \mc{1}{c}{\scriptsize{-0.006}} \\  

     &  & \mc{1}{c}{\scriptsize{(0.579)}} & \mc{1}{c}{\scriptsize{(0.474)}} & \mc{1}{c}{\scriptsize{(0.158)}} & \mc{1}{c}{\scriptsize{(0.303)}} & \mc{1}{c}{\scriptsize{(0.158)}} & \mc{1}{c}{\scriptsize{(0.816)}} & \mc{1}{c}{\scriptsize{(0.592)}} & \mc{1}{c}{\scriptsize{(0.776)}} \\  

    \mc{1}{l}{\scriptsize{Musculoskeletal General}} & \mc{1}{c}{\scriptsize{Mid-30s}} & \mc{1}{c}{\scriptsize{0.043}} & \mc{1}{c}{\scriptsize{0.081}} & \mc{1}{c}{\scriptsize{0.043}} & \mc{1}{c}{\scriptsize{0.115}} & \mc{1}{c}{\scriptsize{0.036}} & \mc{1}{c}{\scriptsize{0.043}} & \mc{1}{c}{\scriptsize{0.076}} & \mc{1}{c}{\scriptsize{0.035}} \\  

     &  & \mc{1}{c}{\scriptsize{(1.000)}} & \mc{1}{c}{\scriptsize{(1.000)}} & \mc{1}{c}{\scriptsize{(0.947)}} & \mc{1}{c}{\scriptsize{(0.987)}} & \mc{1}{c}{\scriptsize{(0.947)}} & \mc{1}{c}{\scriptsize{(1.000)}} & \mc{1}{c}{\scriptsize{(1.000)}} & \mc{1}{c}{\scriptsize{(0.961)}} \\  

    \mc{1}{l}{\scriptsize{Head General}} & \mc{1}{c}{\scriptsize{Mid-30s}} & \mc{1}{c}{\scriptsize{-0.036}} & \mc{1}{c}{\scriptsize{-0.026}} & \mc{1}{c}{\scriptsize{-0.167}} & \mc{1}{c}{\scriptsize{-0.162}} & \mc{1}{c}{\scriptsize{-0.165}} &  &  &  \\  

     &  & \mc{1}{c}{\scriptsize{(0.250)}} & \mc{1}{c}{\scriptsize{(0.566)}} & \mc{1}{c}{\scriptsize{(0.158)}} & \mc{1}{c}{\scriptsize{(0.237)}} & \mc{1}{c}{\scriptsize{(0.145)}} &  &  &  \\  

  \bottomrule
  \end{tabular}
	\end{table} 

	\begin{table}[H]
     \caption{Treatment Effects on Physical Exam (Part II), Female Sample}
     \label{table:abccare_rslt_female_cat44_sd}
	  \begin{tabular}{cccccccccc}
  \toprule

    \scriptsize{Variable} & \scriptsize{Age} & \scriptsize{(1)} & \scriptsize{(2)} & \scriptsize{(3)} & \scriptsize{(4)} & \scriptsize{(5)} & \scriptsize{(6)} & \scriptsize{(7)} & \scriptsize{(8)} \\ 
    \midrule  

    \mc{1}{l}{\scriptsize{Mouth and Throat: Upper Teeth}} & \mc{1}{c}{\scriptsize{Mid-30s}} & \mc{1}{c}{\scriptsize{0.082}} & \mc{1}{c}{\scriptsize{0.106}} & \mc{1}{c}{\scriptsize{0.094}} & \mc{1}{c}{\scriptsize{0.221}} & \mc{1}{c}{\scriptsize{0.110}} & \mc{1}{c}{\scriptsize{0.079}} & \mc{1}{c}{\scriptsize{0.080}} & \mc{1}{c}{\scriptsize{0.128}} \\  

     &  & \mc{1}{c}{\scriptsize{(0.895)}} & \mc{1}{c}{\scriptsize{(0.961)}} & \mc{1}{c}{\scriptsize{(0.750)}} & \mc{1}{c}{\scriptsize{(0.908)}} & \mc{1}{c}{\scriptsize{(0.829)}} & \mc{1}{c}{\scriptsize{(0.907)}} & \mc{1}{c}{\scriptsize{(0.921)}} & \mc{1}{c}{\scriptsize{(0.961)}} \\  

    \mc{1}{l}{\scriptsize{Muscle Strength: Reflexes}} & \mc{1}{c}{\scriptsize{Mid-30s}} & \mc{1}{c}{\scriptsize{0.087}} & \mc{1}{c}{\scriptsize{0.134}} & \mc{1}{c}{\scriptsize{0.087}} & \mc{1}{c}{\scriptsize{0.163}} & \mc{1}{c}{\scriptsize{0.087}} & \mc{1}{c}{\scriptsize{0.087}} & \mc{1}{c}{\scriptsize{0.131}} & \mc{1}{c}{\scriptsize{0.086}} \\  

     &  & \mc{1}{c}{\scriptsize{(0.924)}} & \mc{1}{c}{\scriptsize{(0.985)}} & \mc{1}{c}{\scriptsize{(0.924)}} & \mc{1}{c}{\scriptsize{(0.864)}} & \mc{1}{c}{\scriptsize{(0.924)}} & \mc{1}{c}{\scriptsize{(0.924)}} & \mc{1}{c}{\scriptsize{(0.970)}} & \mc{1}{c}{\scriptsize{(0.924)}} \\  

    \mc{1}{l}{\scriptsize{Mouth and Throat: Lower Teeth}} & \mc{1}{c}{\scriptsize{Mid-30s}} & \mc{1}{c}{\scriptsize{0.031}} & \mc{1}{c}{\scriptsize{0.003}} & \mc{1}{c}{\scriptsize{0.007}} & \mc{1}{c}{\scriptsize{-0.062}} & \mc{1}{c}{\scriptsize{0.029}} & \mc{1}{c}{\scriptsize{0.037}} & \mc{1}{c}{\scriptsize{0.022}} & \mc{1}{c}{\scriptsize{0.095}} \\  

     &  & \mc{1}{c}{\scriptsize{(0.803)}} & \mc{1}{c}{\scriptsize{(0.750)}} & \mc{1}{c}{\scriptsize{(0.539)}} & \mc{1}{c}{\scriptsize{(0.355)}} & \mc{1}{c}{\scriptsize{(0.579)}} & \mc{1}{c}{\scriptsize{(0.800)}} & \mc{1}{c}{\scriptsize{(0.737)}} & \mc{1}{c}{\scriptsize{(0.947)}} \\  

    \mc{1}{l}{\scriptsize{Muscle Strength: Coordination}} & \mc{1}{c}{\scriptsize{Mid-30s}} &  &  &  &  &  &  &  &  \\  

     &  &  &  &  &  &  &  &  &  \\  

  \bottomrule
  \end{tabular}
	\end{table} 

	\begin{table}[H]
     \caption{Treatment Effects on Age 21 Brief Symptom Inventory, Female Sample}
     \label{table:abccare_rslt_female_cat45_sd}
	  \begin{tabular}{cccccccccc}
  \toprule

    \scriptsize{Variable} & \scriptsize{Age} & \scriptsize{(1)} & \scriptsize{(2)} & \scriptsize{(3)} & \scriptsize{(4)} & \scriptsize{(5)} & \scriptsize{(6)} & \scriptsize{(7)} & \scriptsize{(8)} \\ 
    \midrule  

    \mc{1}{l}{\scriptsize{Paranoid Ideation}} & \mc{1}{c}{\scriptsize{21}} & \mc{1}{c}{\scriptsize{-4.343}} & \mc{1}{c}{\scriptsize{-3.146}} & \mc{1}{c}{\scriptsize{-10.181}} & \mc{1}{c}{\scriptsize{-8.111}} & \mc{1}{c}{\scriptsize{-9.127}} & \mc{1}{c}{\scriptsize{-2.848}} & \mc{1}{c}{\scriptsize{-1.575}} & \mc{1}{c}{\scriptsize{-1.249}} \\  

     &  & \mc{1}{c}{\scriptsize{(0.118)}} & \mc{1}{c}{\scriptsize{(0.329)}} & \mc{1}{c}{\scriptsize{\textbf{(0.000)}}} & \mc{1}{c}{\scriptsize{\textbf{(0.000)}}} & \mc{1}{c}{\scriptsize{\textbf{(0.000)}}} & \mc{1}{c}{\scriptsize{(0.408)}} & \mc{1}{c}{\scriptsize{(0.645)}} & \mc{1}{c}{\scriptsize{(0.711)}} \\  

    \mc{1}{l}{\scriptsize{Obsessive-Compulsive}} & \mc{1}{c}{\scriptsize{21}} & \mc{1}{c}{\scriptsize{-4.265}} & \mc{1}{c}{\scriptsize{-5.087}} & \mc{1}{c}{\scriptsize{-4.293}} & \mc{1}{c}{\scriptsize{-3.048}} & \mc{1}{c}{\scriptsize{-4.366}} & \mc{1}{c}{\scriptsize{-4.832}} & \mc{1}{c}{\scriptsize{-5.360}} & \mc{1}{c}{\scriptsize{-4.805}} \\  

     &  & \mc{1}{c}{\scriptsize{(0.250)}} & \mc{1}{c}{\scriptsize{(0.289)}} & \mc{1}{c}{\scriptsize{(0.250)}} & \mc{1}{c}{\scriptsize{(0.697)}} & \mc{1}{c}{\scriptsize{(0.355)}} & \mc{1}{c}{\scriptsize{(0.237)}} & \mc{1}{c}{\scriptsize{(0.303)}} & \mc{1}{c}{\scriptsize{(0.197)}} \\  

    \mc{1}{l}{\scriptsize{Interpersonal Sense}} & \mc{1}{c}{\scriptsize{21}} & \mc{1}{c}{\scriptsize{-4.491}} & \mc{1}{c}{\scriptsize{-4.059}} & \mc{1}{c}{\scriptsize{-9.491}} & \mc{1}{c}{\scriptsize{-8.701}} & \mc{1}{c}{\scriptsize{-8.567}} & \mc{1}{c}{\scriptsize{-2.972}} & \mc{1}{c}{\scriptsize{-1.907}} & \mc{1}{c}{\scriptsize{-1.586}} \\  

     &  & \mc{1}{c}{\scriptsize{(0.118)}} & \mc{1}{c}{\scriptsize{(0.329)}} & \mc{1}{c}{\scriptsize{\textbf{(0.013)}}} & \mc{1}{c}{\scriptsize{\textbf{(0.079)}}} & \mc{1}{c}{\scriptsize{\textbf{(0.039)}}} & \mc{1}{c}{\scriptsize{(0.434)}} & \mc{1}{c}{\scriptsize{(0.645)}} & \mc{1}{c}{\scriptsize{(0.671)}} \\  

    \mc{1}{l}{\scriptsize{Positive Symptom Distress Index (PSI)}} & \mc{1}{c}{\scriptsize{21}} & \mc{1}{c}{\scriptsize{-6.720}} & \mc{1}{c}{\scriptsize{-6.401}} & \mc{1}{c}{\scriptsize{-10.276}} & \mc{1}{c}{\scriptsize{-9.063}} & \mc{1}{c}{\scriptsize{-10.414}} & \mc{1}{c}{\scriptsize{-5.776}} & \mc{1}{c}{\scriptsize{-5.294}} & \mc{1}{c}{\scriptsize{-5.754}} \\  

     &  & \mc{1}{c}{\scriptsize{\textbf{(0.000)}}} & \mc{1}{c}{\scriptsize{\textbf{(0.013)}}} & \mc{1}{c}{\scriptsize{\textbf{(0.000)}}} & \mc{1}{c}{\scriptsize{\textbf{(0.013)}}} & \mc{1}{c}{\scriptsize{\textbf{(0.000)}}} & \mc{1}{c}{\scriptsize{\textbf{(0.053)}}} & \mc{1}{c}{\scriptsize{(0.158)}} & \mc{1}{c}{\scriptsize{\textbf{(0.066)}}} \\  

    \mc{1}{l}{\scriptsize{Psychoticism}} & \mc{1}{c}{\scriptsize{21}} & \mc{1}{c}{\scriptsize{-5.793}} & \mc{1}{c}{\scriptsize{-7.507}} & \mc{1}{c}{\scriptsize{-11.918}} & \mc{1}{c}{\scriptsize{-12.353}} & \mc{1}{c}{\scriptsize{-12.833}} & \mc{1}{c}{\scriptsize{-4.553}} & \mc{1}{c}{\scriptsize{-5.332}} & \mc{1}{c}{\scriptsize{-5.150}} \\  

     &  & \mc{1}{c}{\scriptsize{\textbf{(0.079)}}} & \mc{1}{c}{\scriptsize{\textbf{(0.013)}}} & \mc{1}{c}{\scriptsize{\textbf{(0.000)}}} & \mc{1}{c}{\scriptsize{\textbf{(0.000)}}} & \mc{1}{c}{\scriptsize{\textbf{(0.000)}}} & \mc{1}{c}{\scriptsize{(0.145)}} & \mc{1}{c}{\scriptsize{(0.184)}} & \mc{1}{c}{\scriptsize{(0.145)}} \\  

    \mc{1}{l}{\scriptsize{Phobic Anxiety}} & \mc{1}{c}{\scriptsize{21}} & \mc{1}{c}{\scriptsize{-4.444}} & \mc{1}{c}{\scriptsize{-3.703}} & \mc{1}{c}{\scriptsize{-8.000}} & \mc{1}{c}{\scriptsize{-7.477}} & \mc{1}{c}{\scriptsize{-7.640}} & \mc{1}{c}{\scriptsize{-3.462}} & \mc{1}{c}{\scriptsize{-2.114}} & \mc{1}{c}{\scriptsize{-2.722}} \\  

     &  & \mc{1}{c}{\scriptsize{(0.118)}} & \mc{1}{c}{\scriptsize{(0.329)}} & \mc{1}{c}{\scriptsize{\textbf{(0.000)}}} & \mc{1}{c}{\scriptsize{\textbf{(0.039)}}} & \mc{1}{c}{\scriptsize{\textbf{(0.000)}}} & \mc{1}{c}{\scriptsize{(0.289)}} & \mc{1}{c}{\scriptsize{(0.592)}} & \mc{1}{c}{\scriptsize{(0.421)}} \\  

    \mc{1}{l}{\scriptsize{Positive Symptom Total (PST)}} & \mc{1}{c}{\scriptsize{21}} & \mc{1}{c}{\scriptsize{-5.961}} & \mc{1}{c}{\scriptsize{-6.422}} & \mc{1}{c}{\scriptsize{-10.975}} & \mc{1}{c}{\scriptsize{-9.827}} & \mc{1}{c}{\scriptsize{-10.827}} & \mc{1}{c}{\scriptsize{-4.985}} & \mc{1}{c}{\scriptsize{-5.380}} & \mc{1}{c}{\scriptsize{-4.583}} \\  

     &  & \mc{1}{c}{\scriptsize{\textbf{(0.079)}}} & \mc{1}{c}{\scriptsize{\textbf{(0.039)}}} & \mc{1}{c}{\scriptsize{\textbf{(0.026)}}} & \mc{1}{c}{\scriptsize{\textbf{(0.039)}}} & \mc{1}{c}{\scriptsize{\textbf{(0.013)}}} & \mc{1}{c}{\scriptsize{(0.105)}} & \mc{1}{c}{\scriptsize{(0.158)}} & \mc{1}{c}{\scriptsize{(0.158)}} \\  

  \bottomrule
  \end{tabular}
	\end{table} 

	\begin{table}[H]
     \caption{Treatment Effects on Age 30 Adult Self Report DSM Scale $t$-Score, Female Sample}
     \label{table:abccare_rslt_female_cat46_sd}
	  \begin{tabular}{cccccccccc}
  \toprule

    \scriptsize{Variable} & \scriptsize{Age} & \scriptsize{(1)} & \scriptsize{(2)} & \scriptsize{(3)} & \scriptsize{(4)} & \scriptsize{(5)} & \scriptsize{(6)} & \scriptsize{(7)} & \scriptsize{(8)} \\ 
    \midrule  

    \mc{1}{l}{\scriptsize{Somatic Problems}} & \mc{1}{c}{\scriptsize{30}} & \mc{1}{c}{\scriptsize{0.787}} & \mc{1}{c}{\scriptsize{-0.124}} & \mc{1}{c}{\scriptsize{1.608}} & \mc{1}{c}{\scriptsize{0.067}} & \mc{1}{c}{\scriptsize{0.798}} & \mc{1}{c}{\scriptsize{0.326}} & \mc{1}{c}{\scriptsize{-0.269}} & \mc{1}{c}{\scriptsize{-0.406}} \\  

     &  & \mc{1}{c}{\scriptsize{(0.947)}} & \mc{1}{c}{\scriptsize{(0.882)}} & \mc{1}{c}{\scriptsize{(0.974)}} & \mc{1}{c}{\scriptsize{(0.908)}} & \mc{1}{c}{\scriptsize{(0.961)}} & \mc{1}{c}{\scriptsize{(0.908)}} & \mc{1}{c}{\scriptsize{(0.816)}} & \mc{1}{c}{\scriptsize{(0.803)}} \\  

    \mc{1}{l}{\scriptsize{AD/H Problems}} & \mc{1}{c}{\scriptsize{30}} & \mc{1}{c}{\scriptsize{-1.600}} & \mc{1}{c}{\scriptsize{-2.497}} & \mc{1}{c}{\scriptsize{-1.475}} & \mc{1}{c}{\scriptsize{-1.807}} & \mc{1}{c}{\scriptsize{-2.470}} & \mc{1}{c}{\scriptsize{-1.933}} & \mc{1}{c}{\scriptsize{-2.262}} & \mc{1}{c}{\scriptsize{-3.051}} \\  

     &  & \mc{1}{c}{\scriptsize{(0.434)}} & \mc{1}{c}{\scriptsize{(0.211)}} & \mc{1}{c}{\scriptsize{(0.566)}} & \mc{1}{c}{\scriptsize{(0.526)}} & \mc{1}{c}{\scriptsize{(0.382)}} & \mc{1}{c}{\scriptsize{(0.355)}} & \mc{1}{c}{\scriptsize{(0.303)}} & \mc{1}{c}{\scriptsize{(0.105)}} \\  

    \mc{1}{l}{\scriptsize{Inattention Subscale}} & \mc{1}{c}{\scriptsize{30}} &  &  &  &  &  &  &  &  \\  

     &  &  &  &  &  &  &  &  &  \\  

    \mc{1}{l}{\scriptsize{Depressive Problems}} & \mc{1}{c}{\scriptsize{30}} & \mc{1}{c}{\scriptsize{0.738}} & \mc{1}{c}{\scriptsize{-0.184}} & \mc{1}{c}{\scriptsize{0.400}} & \mc{1}{c}{\scriptsize{-0.524}} & \mc{1}{c}{\scriptsize{-0.220}} & \mc{1}{c}{\scriptsize{0.715}} & \mc{1}{c}{\scriptsize{-0.174}} & \mc{1}{c}{\scriptsize{0.053}} \\  

     &  & \mc{1}{c}{\scriptsize{(0.961)}} & \mc{1}{c}{\scriptsize{(0.842)}} & \mc{1}{c}{\scriptsize{(0.908)}} & \mc{1}{c}{\scriptsize{(0.803)}} & \mc{1}{c}{\scriptsize{(0.855)}} & \mc{1}{c}{\scriptsize{(0.947)}} & \mc{1}{c}{\scriptsize{(0.803)}} & \mc{1}{c}{\scriptsize{(0.895)}} \\  

    \mc{1}{l}{\scriptsize{Avoidant Personality Problems}} & \mc{1}{c}{\scriptsize{30}} & \mc{1}{c}{\scriptsize{-0.337}} & \mc{1}{c}{\scriptsize{-1.549}} & \mc{1}{c}{\scriptsize{1.308}} & \mc{1}{c}{\scriptsize{-0.511}} & \mc{1}{c}{\scriptsize{0.088}} & \mc{1}{c}{\scriptsize{-0.956}} & \mc{1}{c}{\scriptsize{-2.043}} & \mc{1}{c}{\scriptsize{-1.810}} \\  

     &  & \mc{1}{c}{\scriptsize{(0.789)}} & \mc{1}{c}{\scriptsize{(0.316)}} & \mc{1}{c}{\scriptsize{(0.987)}} & \mc{1}{c}{\scriptsize{(0.816)}} & \mc{1}{c}{\scriptsize{(0.921)}} & \mc{1}{c}{\scriptsize{(0.671)}} & \mc{1}{c}{\scriptsize{(0.132)}} & \mc{1}{c}{\scriptsize{(0.184)}} \\  

    \mc{1}{l}{\scriptsize{Anxiety Problems}} & \mc{1}{c}{\scriptsize{30}} & \mc{1}{c}{\scriptsize{-0.568}} & \mc{1}{c}{\scriptsize{-1.378}} & \mc{1}{c}{\scriptsize{-0.125}} & \mc{1}{c}{\scriptsize{-1.515}} & \mc{1}{c}{\scriptsize{-0.932}} & \mc{1}{c}{\scriptsize{-0.815}} & \mc{1}{c}{\scriptsize{-1.308}} & \mc{1}{c}{\scriptsize{-1.641}} \\  

     &  & \mc{1}{c}{\scriptsize{(0.711)}} & \mc{1}{c}{\scriptsize{\textbf{(0.053)}}} & \mc{1}{c}{\scriptsize{(0.816)}} & \mc{1}{c}{\scriptsize{(0.184)}} & \mc{1}{c}{\scriptsize{(0.434)}} & \mc{1}{c}{\scriptsize{(0.697)}} & \mc{1}{c}{\scriptsize{(0.237)}} & \mc{1}{c}{\scriptsize{(0.171)}} \\  

    \mc{1}{l}{\scriptsize{Antisocial Personality Problems}} & \mc{1}{c}{\scriptsize{30}} & \mc{1}{c}{\scriptsize{-3.441}} & \mc{1}{c}{\scriptsize{-3.424}} & \mc{1}{c}{\scriptsize{-5.208}} & \mc{1}{c}{\scriptsize{-5.609}} & \mc{1}{c}{\scriptsize{-5.745}} & \mc{1}{c}{\scriptsize{-2.963}} & \mc{1}{c}{\scriptsize{-2.845}} & \mc{1}{c}{\scriptsize{-3.259}} \\  

     &  & \mc{1}{c}{\scriptsize{\textbf{(0.000)}}} & \mc{1}{c}{\scriptsize{\textbf{(0.026)}}} & \mc{1}{c}{\scriptsize{(0.132)}} & \mc{1}{c}{\scriptsize{\textbf{(0.026)}}} & \mc{1}{c}{\scriptsize{(0.105)}} & \mc{1}{c}{\scriptsize{\textbf{(0.053)}}} & \mc{1}{c}{\scriptsize{\textbf{(0.066)}}} & \mc{1}{c}{\scriptsize{\textbf{(0.026)}}} \\  

    \mc{1}{l}{\scriptsize{Hyperactivity-Impulsivity Subscale}} & \mc{1}{c}{\scriptsize{30}} &  &  &  &  &  &  &  &  \\  

     &  &  &  &  &  &  &  &  &  \\  

  \bottomrule
  \end{tabular}
	\end{table} 

	\begin{table}[H]
     \caption{Treatment Effects on Age 30 Adult Self Report Syndrome Scale $t$-Score, Female Sample}
     \label{table:abccare_rslt_female_cat47_sd}
	  \begin{tabular}{cccccccccc}
  \toprule

    \scriptsize{Variable} & \scriptsize{Age} & \scriptsize{(1)} & \scriptsize{(2)} & \scriptsize{(3)} & \scriptsize{(4)} & \scriptsize{(5)} & \scriptsize{(6)} & \scriptsize{(7)} & \scriptsize{(8)} \\ 
    \midrule  

    \mc{1}{l}{\scriptsize{Somatic Complaints}} & \mc{1}{c}{\scriptsize{30}} & \mc{1}{c}{\scriptsize{0.356}} & \mc{1}{c}{\scriptsize{-0.839}} & \mc{1}{c}{\scriptsize{0.792}} & \mc{1}{c}{\scriptsize{-0.395}} & \mc{1}{c}{\scriptsize{-0.053}} & \mc{1}{c}{\scriptsize{0.018}} & \mc{1}{c}{\scriptsize{-1.074}} & \mc{1}{c}{\scriptsize{-0.897}} \\  

     &  & \mc{1}{c}{\scriptsize{(0.974)}} & \mc{1}{c}{\scriptsize{(0.750)}} & \mc{1}{c}{\scriptsize{(0.974)}} & \mc{1}{c}{\scriptsize{(0.921)}} & \mc{1}{c}{\scriptsize{(0.974)}} & \mc{1}{c}{\scriptsize{(0.882)}} & \mc{1}{c}{\scriptsize{(0.605)}} & \mc{1}{c}{\scriptsize{(0.592)}} \\  

    \mc{1}{l}{\scriptsize{Aggressive Behavior}} & \mc{1}{c}{\scriptsize{30}} & \mc{1}{c}{\scriptsize{-1.709}} & \mc{1}{c}{\scriptsize{-1.505}} & \mc{1}{c}{\scriptsize{-2.408}} & \mc{1}{c}{\scriptsize{-2.547}} & \mc{1}{c}{\scriptsize{-3.164}} & \mc{1}{c}{\scriptsize{-1.218}} & \mc{1}{c}{\scriptsize{-1.319}} & \mc{1}{c}{\scriptsize{-1.711}} \\  

     &  & \mc{1}{c}{\scriptsize{(0.408)}} & \mc{1}{c}{\scriptsize{(0.579)}} & \mc{1}{c}{\scriptsize{(0.566)}} & \mc{1}{c}{\scriptsize{(0.395)}} & \mc{1}{c}{\scriptsize{(0.382)}} & \mc{1}{c}{\scriptsize{(0.579)}} & \mc{1}{c}{\scriptsize{(0.579)}} & \mc{1}{c}{\scriptsize{(0.395)}} \\  

    \mc{1}{l}{\scriptsize{Intrusive}} & \mc{1}{c}{\scriptsize{30}} & \mc{1}{c}{\scriptsize{-1.147}} & \mc{1}{c}{\scriptsize{-1.471}} & \mc{1}{c}{\scriptsize{-3.758}} & \mc{1}{c}{\scriptsize{-4.231}} & \mc{1}{c}{\scriptsize{-4.036}} & \mc{1}{c}{\scriptsize{-0.374}} & \mc{1}{c}{\scriptsize{-0.525}} & \mc{1}{c}{\scriptsize{-0.718}} \\  

     &  & \mc{1}{c}{\scriptsize{(0.500)}} & \mc{1}{c}{\scriptsize{(0.382)}} & \mc{1}{c}{\scriptsize{(0.224)}} & \mc{1}{c}{\scriptsize{(0.224)}} & \mc{1}{c}{\scriptsize{(0.224)}} & \mc{1}{c}{\scriptsize{(0.776)}} & \mc{1}{c}{\scriptsize{(0.776)}} & \mc{1}{c}{\scriptsize{(0.658)}} \\  

    \mc{1}{l}{\scriptsize{Anxious/Depressed}} & \mc{1}{c}{\scriptsize{30}} & \mc{1}{c}{\scriptsize{-0.286}} & \mc{1}{c}{\scriptsize{-0.993}} & \mc{1}{c}{\scriptsize{-0.425}} & \mc{1}{c}{\scriptsize{-1.774}} & \mc{1}{c}{\scriptsize{-1.666}} & \mc{1}{c}{\scriptsize{-0.319}} & \mc{1}{c}{\scriptsize{-0.686}} & \mc{1}{c}{\scriptsize{-1.262}} \\  

     &  & \mc{1}{c}{\scriptsize{(0.842)}} & \mc{1}{c}{\scriptsize{(0.566)}} & \mc{1}{c}{\scriptsize{(0.882)}} & \mc{1}{c}{\scriptsize{(0.592)}} & \mc{1}{c}{\scriptsize{(0.526)}} & \mc{1}{c}{\scriptsize{(0.842)}} & \mc{1}{c}{\scriptsize{(0.605)}} & \mc{1}{c}{\scriptsize{(0.461)}} \\  

    \mc{1}{l}{\scriptsize{Externalizing}} & \mc{1}{c}{\scriptsize{30}} & \mc{1}{c}{\scriptsize{-3.505}} & \mc{1}{c}{\scriptsize{-3.528}} & \mc{1}{c}{\scriptsize{-6.917}} & \mc{1}{c}{\scriptsize{-7.164}} & \mc{1}{c}{\scriptsize{-7.764}} & \mc{1}{c}{\scriptsize{-2.296}} & \mc{1}{c}{\scriptsize{-2.262}} & \mc{1}{c}{\scriptsize{-2.886}} \\  

     &  & \mc{1}{c}{\scriptsize{(0.342)}} & \mc{1}{c}{\scriptsize{(0.487)}} & \mc{1}{c}{\scriptsize{(0.171)}} & \mc{1}{c}{\scriptsize{(0.145)}} & \mc{1}{c}{\scriptsize{(0.171)}} & \mc{1}{c}{\scriptsize{(0.513)}} & \mc{1}{c}{\scriptsize{(0.632)}} & \mc{1}{c}{\scriptsize{(0.474)}} \\  

    \mc{1}{l}{\scriptsize{Thought Problems}} & \mc{1}{c}{\scriptsize{30}} & \mc{1}{c}{\scriptsize{-3.619}} & \mc{1}{c}{\scriptsize{-3.529}} & \mc{1}{c}{\scriptsize{-3.825}} & \mc{1}{c}{\scriptsize{-4.975}} & \mc{1}{c}{\scriptsize{-4.009}} & \mc{1}{c}{\scriptsize{-3.811}} & \mc{1}{c}{\scriptsize{-2.989}} & \mc{1}{c}{\scriptsize{-3.748}} \\  

     &  & \mc{1}{c}{\scriptsize{\textbf{(0.000)}}} & \mc{1}{c}{\scriptsize{\textbf{(0.026)}}} & \mc{1}{c}{\scriptsize{(0.158)}} & \mc{1}{c}{\scriptsize{(0.132)}} & \mc{1}{c}{\scriptsize{(0.171)}} & \mc{1}{c}{\scriptsize{\textbf{(0.053)}}} & \mc{1}{c}{\scriptsize{(0.145)}} & \mc{1}{c}{\scriptsize{(0.132)}} \\  

    \mc{1}{l}{\scriptsize{Total Problems}} & \mc{1}{c}{\scriptsize{30}} & \mc{1}{c}{\scriptsize{-2.561}} & \mc{1}{c}{\scriptsize{-3.569}} & \mc{1}{c}{\scriptsize{-2.842}} & \mc{1}{c}{\scriptsize{-4.303}} & \mc{1}{c}{\scriptsize{-3.997}} & \mc{1}{c}{\scriptsize{-2.559}} & \mc{1}{c}{\scriptsize{-3.239}} & \mc{1}{c}{\scriptsize{-3.370}} \\  

     &  & \mc{1}{c}{\scriptsize{(0.474)}} & \mc{1}{c}{\scriptsize{(0.316)}} & \mc{1}{c}{\scriptsize{(0.658)}} & \mc{1}{c}{\scriptsize{(0.513)}} & \mc{1}{c}{\scriptsize{(0.500)}} & \mc{1}{c}{\scriptsize{(0.487)}} & \mc{1}{c}{\scriptsize{(0.447)}} & \mc{1}{c}{\scriptsize{(0.329)}} \\  

    \mc{1}{l}{\scriptsize{Withdrawn}} & \mc{1}{c}{\scriptsize{30}} & \mc{1}{c}{\scriptsize{-0.488}} & \mc{1}{c}{\scriptsize{-1.125}} & \mc{1}{c}{\scriptsize{0.633}} & \mc{1}{c}{\scriptsize{-1.128}} & \mc{1}{c}{\scriptsize{0.052}} & \mc{1}{c}{\scriptsize{-0.570}} & \mc{1}{c}{\scriptsize{-1.201}} & \mc{1}{c}{\scriptsize{-0.696}} \\  

     &  & \mc{1}{c}{\scriptsize{(0.829)}} & \mc{1}{c}{\scriptsize{(0.605)}} & \mc{1}{c}{\scriptsize{(0.974)}} & \mc{1}{c}{\scriptsize{(0.671)}} & \mc{1}{c}{\scriptsize{(0.974)}} & \mc{1}{c}{\scriptsize{(0.763)}} & \mc{1}{c}{\scriptsize{(0.487)}} & \mc{1}{c}{\scriptsize{(0.658)}} \\  

    \mc{1}{l}{\scriptsize{Internalizing}} & \mc{1}{c}{\scriptsize{30}} & \mc{1}{c}{\scriptsize{-0.433}} & \mc{1}{c}{\scriptsize{-1.542}} & \mc{1}{c}{\scriptsize{2.067}} & \mc{1}{c}{\scriptsize{-0.912}} & \mc{1}{c}{\scriptsize{0.319}} & \mc{1}{c}{\scriptsize{-1.137}} & \mc{1}{c}{\scriptsize{-1.774}} & \mc{1}{c}{\scriptsize{-2.332}} \\  

     &  & \mc{1}{c}{\scriptsize{(0.868)}} & \mc{1}{c}{\scriptsize{(0.763)}} & \mc{1}{c}{\scriptsize{(0.974)}} & \mc{1}{c}{\scriptsize{(0.908)}} & \mc{1}{c}{\scriptsize{(1.000)}} & \mc{1}{c}{\scriptsize{(0.763)}} & \mc{1}{c}{\scriptsize{(0.658)}} & \mc{1}{c}{\scriptsize{(0.539)}} \\  

    \mc{1}{l}{\scriptsize{Critical Items}} & \mc{1}{c}{\scriptsize{30}} & \mc{1}{c}{\scriptsize{-2.651}} & \mc{1}{c}{\scriptsize{-2.890}} & \mc{1}{c}{\scriptsize{-3.367}} & \mc{1}{c}{\scriptsize{-3.775}} & \mc{1}{c}{\scriptsize{-3.939}} & \mc{1}{c}{\scriptsize{-2.719}} & \mc{1}{c}{\scriptsize{-2.630}} & \mc{1}{c}{\scriptsize{-3.235}} \\  

     &  & \mc{1}{c}{\scriptsize{(0.105)}} & \mc{1}{c}{\scriptsize{\textbf{(0.066)}}} & \mc{1}{c}{\scriptsize{(0.224)}} & \mc{1}{c}{\scriptsize{(0.145)}} & \mc{1}{c}{\scriptsize{(0.171)}} & \mc{1}{c}{\scriptsize{(0.145)}} & \mc{1}{c}{\scriptsize{(0.171)}} & \mc{1}{c}{\scriptsize{(0.132)}} \\  

    \mc{1}{l}{\scriptsize{Rule Breaking}} & \mc{1}{c}{\scriptsize{30}} & \mc{1}{c}{\scriptsize{-2.248}} & \mc{1}{c}{\scriptsize{-2.846}} & \mc{1}{c}{\scriptsize{-4.542}} & \mc{1}{c}{\scriptsize{-4.780}} & \mc{1}{c}{\scriptsize{-4.744}} & \mc{1}{c}{\scriptsize{-1.907}} & \mc{1}{c}{\scriptsize{-1.964}} & \mc{1}{c}{\scriptsize{-2.243}} \\  

     &  & \mc{1}{c}{\scriptsize{(0.158)}} & \mc{1}{c}{\scriptsize{\textbf{(0.066)}}} & \mc{1}{c}{\scriptsize{\textbf{(0.079)}}} & \mc{1}{c}{\scriptsize{(0.158)}} & \mc{1}{c}{\scriptsize{\textbf{(0.066)}}} & \mc{1}{c}{\scriptsize{(0.237)}} & \mc{1}{c}{\scriptsize{(0.395)}} & \mc{1}{c}{\scriptsize{(0.197)}} \\  

    \mc{1}{l}{\scriptsize{Attention Problems}} & \mc{1}{c}{\scriptsize{30}} & \mc{1}{c}{\scriptsize{-0.226}} & \mc{1}{c}{\scriptsize{-1.319}} & \mc{1}{c}{\scriptsize{1.108}} & \mc{1}{c}{\scriptsize{-0.191}} & \mc{1}{c}{\scriptsize{0.036}} & \mc{1}{c}{\scriptsize{-0.878}} & \mc{1}{c}{\scriptsize{-1.435}} & \mc{1}{c}{\scriptsize{-1.834}} \\  

     &  & \mc{1}{c}{\scriptsize{(0.855)}} & \mc{1}{c}{\scriptsize{(0.566)}} & \mc{1}{c}{\scriptsize{(0.987)}} & \mc{1}{c}{\scriptsize{(0.947)}} & \mc{1}{c}{\scriptsize{(0.974)}} & \mc{1}{c}{\scriptsize{(0.724)}} & \mc{1}{c}{\scriptsize{(0.447)}} & \mc{1}{c}{\scriptsize{(0.197)}} \\  

  \bottomrule
  \end{tabular}
	\end{table} 

	\begin{table}[H]
     \caption{Treatment Effects on BSI 18 $t$-Score, Female Sample}
     \label{table:abccare_rslt_female_cat48_sd}
	\input{AppResOutput/abccare/rslt_female_cat48_sd}
	\end{table} 

	\begin{table}[H]
     \caption{Treatment Effects on BSI Raw Score, Female Sample}
     \label{table:abccare_rslt_female_cat49_sd}
	\input{AppResOutput/abccare/rslt_female_cat49_sd}
	\end{table} 

	\begin{table}[H]
     \caption{Treatment Effects on BSI $t$-Score, Female Sample}
     \label{table:abccare_rslt_female_cat50_sd}
	  \begin{tabular}{cccccccccc}
  \toprule

    \scriptsize{Variable} & \scriptsize{Age} & \scriptsize{(1)} & \scriptsize{(2)} & \scriptsize{(3)} & \scriptsize{(4)} & \scriptsize{(5)} & \scriptsize{(6)} & \scriptsize{(7)} & \scriptsize{(8)} \\ 
    \midrule  

    \mc{1}{l}{\scriptsize{Depression $t$-Score}} & \mc{1}{c}{\scriptsize{Mid-30s}} & \mc{1}{c}{\scriptsize{-2.466}} & \mc{1}{c}{\scriptsize{-2.693}} & \mc{1}{c}{\scriptsize{-0.109}} & \mc{1}{c}{\scriptsize{0.228}} & \mc{1}{c}{\scriptsize{-0.047}} & \mc{1}{c}{\scriptsize{-3.109}} & \mc{1}{c}{\scriptsize{-4.431}} & \mc{1}{c}{\scriptsize{-3.039}} \\  

     &  & \mc{1}{c}{\scriptsize{(0.618)}} & \mc{1}{c}{\scriptsize{(0.526)}} & \mc{1}{c}{\scriptsize{(0.908)}} & \mc{1}{c}{\scriptsize{(0.921)}} & \mc{1}{c}{\scriptsize{(0.908)}} & \mc{1}{c}{\scriptsize{(0.566)}} & \mc{1}{c}{\scriptsize{(0.224)}} & \mc{1}{c}{\scriptsize{(0.553)}} \\  

     & \mc{1}{c}{\scriptsize{21}} & \mc{1}{c}{\scriptsize{-5.649}} & \mc{1}{c}{\scriptsize{-5.832}} & \mc{1}{c}{\scriptsize{-9.358}} & \mc{1}{c}{\scriptsize{-10.829}} & \mc{1}{c}{\scriptsize{-9.439}} & \mc{1}{c}{\scriptsize{-4.406}} & \mc{1}{c}{\scriptsize{-4.259}} & \mc{1}{c}{\scriptsize{-4.088}} \\  

     &  & \mc{1}{c}{\scriptsize{\textbf{(0.053)}}} & \mc{1}{c}{\scriptsize{(0.184)}} & \mc{1}{c}{\scriptsize{\textbf{(0.039)}}} & \mc{1}{c}{\scriptsize{\textbf{(0.066)}}} & \mc{1}{c}{\scriptsize{\textbf{(0.039)}}} & \mc{1}{c}{\scriptsize{(0.158)}} & \mc{1}{c}{\scriptsize{(0.447)}} & \mc{1}{c}{\scriptsize{(0.250)}} \\  

    \mc{1}{l}{\scriptsize{Somatization $t$-Score}} & \mc{1}{c}{\scriptsize{21}} & \mc{1}{c}{\scriptsize{-2.671}} & \mc{1}{c}{\scriptsize{-2.795}} & \mc{1}{c}{\scriptsize{-4.893}} & \mc{1}{c}{\scriptsize{-4.797}} & \mc{1}{c}{\scriptsize{-4.844}} & \mc{1}{c}{\scriptsize{-2.258}} & \mc{1}{c}{\scriptsize{-2.535}} & \mc{1}{c}{\scriptsize{-2.172}} \\  

     &  & \mc{1}{c}{\scriptsize{(0.553)}} & \mc{1}{c}{\scriptsize{(0.658)}} & \mc{1}{c}{\scriptsize{(0.447)}} & \mc{1}{c}{\scriptsize{(0.500)}} & \mc{1}{c}{\scriptsize{(0.513)}} & \mc{1}{c}{\scriptsize{(0.618)}} & \mc{1}{c}{\scriptsize{(0.697)}} & \mc{1}{c}{\scriptsize{(0.658)}} \\  

    \mc{1}{l}{\scriptsize{Anxiety $t$-Score}} & \mc{1}{c}{\scriptsize{21}} & \mc{1}{c}{\scriptsize{-6.163}} & \mc{1}{c}{\scriptsize{-6.115}} & \mc{1}{c}{\scriptsize{-9.552}} & \mc{1}{c}{\scriptsize{-10.578}} & \mc{1}{c}{\scriptsize{-8.986}} & \mc{1}{c}{\scriptsize{-5.244}} & \mc{1}{c}{\scriptsize{-4.529}} & \mc{1}{c}{\scriptsize{-4.366}} \\  

     &  & \mc{1}{c}{\scriptsize{\textbf{(0.053)}}} & \mc{1}{c}{\scriptsize{(0.105)}} & \mc{1}{c}{\scriptsize{\textbf{(0.039)}}} & \mc{1}{c}{\scriptsize{\textbf{(0.066)}}} & \mc{1}{c}{\scriptsize{\textbf{(0.066)}}} & \mc{1}{c}{\scriptsize{(0.158)}} & \mc{1}{c}{\scriptsize{(0.395)}} & \mc{1}{c}{\scriptsize{(0.329)}} \\  

    \mc{1}{l}{\scriptsize{Somatization $t$-Score}} & \mc{1}{c}{\scriptsize{Mid-30s}} & \mc{1}{c}{\scriptsize{0.724}} & \mc{1}{c}{\scriptsize{2.247}} & \mc{1}{c}{\scriptsize{-0.015}} & \mc{1}{c}{\scriptsize{-3.046}} & \mc{1}{c}{\scriptsize{0.575}} & \mc{1}{c}{\scriptsize{0.925}} & \mc{1}{c}{\scriptsize{3.114}} & \mc{1}{c}{\scriptsize{1.718}} \\  

     &  & \mc{1}{c}{\scriptsize{(0.961)}} & \mc{1}{c}{\scriptsize{(1.000)}} & \mc{1}{c}{\scriptsize{(0.908)}} & \mc{1}{c}{\scriptsize{(0.789)}} & \mc{1}{c}{\scriptsize{(0.934)}} & \mc{1}{c}{\scriptsize{(0.974)}} & \mc{1}{c}{\scriptsize{(0.987)}} & \mc{1}{c}{\scriptsize{(1.000)}} \\  

    \mc{1}{l}{\scriptsize{Hostility $t$-Score}} & \mc{1}{c}{\scriptsize{Mid-30s}} & \mc{1}{c}{\scriptsize{0.512}} & \mc{1}{c}{\scriptsize{-0.492}} & \mc{1}{c}{\scriptsize{-0.797}} & \mc{1}{c}{\scriptsize{-1.659}} & \mc{1}{c}{\scriptsize{-0.689}} & \mc{1}{c}{\scriptsize{0.870}} & \mc{1}{c}{\scriptsize{-0.784}} & \mc{1}{c}{\scriptsize{1.551}} \\  

     &  & \mc{1}{c}{\scriptsize{(0.961)}} & \mc{1}{c}{\scriptsize{(0.882)}} & \mc{1}{c}{\scriptsize{(0.855)}} & \mc{1}{c}{\scriptsize{(0.816)}} & \mc{1}{c}{\scriptsize{(0.895)}} & \mc{1}{c}{\scriptsize{(0.974)}} & \mc{1}{c}{\scriptsize{(0.882)}} & \mc{1}{c}{\scriptsize{(0.987)}} \\  

    \mc{1}{l}{\scriptsize{Global Severity Index $t$-Score}} & \mc{1}{c}{\scriptsize{21}} & \mc{1}{c}{\scriptsize{-6.436}} & \mc{1}{c}{\scriptsize{-6.353}} & \mc{1}{c}{\scriptsize{-11.241}} & \mc{1}{c}{\scriptsize{-9.877}} & \mc{1}{c}{\scriptsize{-11.011}} & \mc{1}{c}{\scriptsize{-5.472}} & \mc{1}{c}{\scriptsize{-4.887}} & \mc{1}{c}{\scriptsize{-4.976}} \\  

     &  & \mc{1}{c}{\scriptsize{\textbf{(0.039)}}} & \mc{1}{c}{\scriptsize{\textbf{(0.066)}}} & \mc{1}{c}{\scriptsize{\textbf{(0.000)}}} & \mc{1}{c}{\scriptsize{\textbf{(0.039)}}} & \mc{1}{c}{\scriptsize{\textbf{(0.013)}}} & \mc{1}{c}{\scriptsize{(0.105)}} & \mc{1}{c}{\scriptsize{(0.303)}} & \mc{1}{c}{\scriptsize{(0.132)}} \\  

    \mc{1}{l}{\scriptsize{Anxiety $t$-Score}} & \mc{1}{c}{\scriptsize{Mid-30s}} & \mc{1}{c}{\scriptsize{-4.564}} & \mc{1}{c}{\scriptsize{-4.437}} & \mc{1}{c}{\scriptsize{-3.457}} & \mc{1}{c}{\scriptsize{-5.686}} & \mc{1}{c}{\scriptsize{-3.752}} & \mc{1}{c}{\scriptsize{-4.866}} & \mc{1}{c}{\scriptsize{-4.670}} & \mc{1}{c}{\scriptsize{-5.629}} \\  

     &  & \mc{1}{c}{\scriptsize{(0.211)}} & \mc{1}{c}{\scriptsize{(0.158)}} & \mc{1}{c}{\scriptsize{(0.763)}} & \mc{1}{c}{\scriptsize{(0.592)}} & \mc{1}{c}{\scriptsize{(0.697)}} & \mc{1}{c}{\scriptsize{(0.158)}} & \mc{1}{c}{\scriptsize{(0.224)}} & \mc{1}{c}{\scriptsize{(0.118)}} \\  

    \mc{1}{l}{\scriptsize{Hostility $t$-Score}} & \mc{1}{c}{\scriptsize{21}} & \mc{1}{c}{\scriptsize{-4.721}} & \mc{1}{c}{\scriptsize{-6.043}} & \mc{1}{c}{\scriptsize{-10.732}} & \mc{1}{c}{\scriptsize{-11.203}} & \mc{1}{c}{\scriptsize{-10.554}} & \mc{1}{c}{\scriptsize{-3.300}} & \mc{1}{c}{\scriptsize{-4.341}} & \mc{1}{c}{\scriptsize{-2.927}} \\  

     &  & \mc{1}{c}{\scriptsize{\textbf{(0.066)}}} & \mc{1}{c}{\scriptsize{\textbf{(0.053)}}} & \mc{1}{c}{\scriptsize{\textbf{(0.000)}}} & \mc{1}{c}{\scriptsize{\textbf{(0.013)}}} & \mc{1}{c}{\scriptsize{\textbf{(0.013)}}} & \mc{1}{c}{\scriptsize{(0.250)}} & \mc{1}{c}{\scriptsize{(0.303)}} & \mc{1}{c}{\scriptsize{(0.461)}} \\  

  \bottomrule
  \end{tabular}
	\end{table} 

	\begin{table}[H]
     \caption{Treatment Effects on Mid-30s Mental Health Conditions, Female Sample}
     \label{table:abccare_rslt_female_cat51_sd}
	  \begin{tabular}{cccccccccc}
  \toprule

    \scriptsize{Variable} & \scriptsize{Age} & \scriptsize{(1)} & \scriptsize{(2)} & \scriptsize{(3)} & \scriptsize{(4)} & \scriptsize{(5)} & \scriptsize{(6)} & \scriptsize{(7)} & \scriptsize{(8)} \\ 
    \midrule  

    \mc{1}{l}{\scriptsize{Social acceptance}} & \mc{1}{c}{\scriptsize{12}} & \mc{1}{c}{\scriptsize{-0.471}} & \mc{1}{c}{\scriptsize{-0.495}} & \mc{1}{c}{\scriptsize{-1.364}} & \mc{1}{c}{\scriptsize{-0.878}} & \mc{1}{c}{\scriptsize{-1.276}} & \mc{1}{c}{\scriptsize{-0.402}} & \mc{1}{c}{\scriptsize{-0.378}} & \mc{1}{c}{\scriptsize{-0.297}} \\  

     &  & \mc{1}{c}{\scriptsize{(1.000)}} & \mc{1}{c}{\scriptsize{(0.961)}} & \mc{1}{c}{\scriptsize{(1.000)}} & \mc{1}{c}{\scriptsize{(1.000)}} & \mc{1}{c}{\scriptsize{(1.000)}} & \mc{1}{c}{\scriptsize{(1.000)}} & \mc{1}{c}{\scriptsize{(0.947)}} & \mc{1}{c}{\scriptsize{(0.987)}} \\  

    \mc{1}{l}{\scriptsize{Behavioral conduct}} & \mc{1}{c}{\scriptsize{12}} & \mc{1}{c}{\scriptsize{-0.156}} & \mc{1}{c}{\scriptsize{-0.706}} & \mc{1}{c}{\scriptsize{-0.727}} & \mc{1}{c}{\scriptsize{-0.503}} & \mc{1}{c}{\scriptsize{-0.742}} & \mc{1}{c}{\scriptsize{-0.112}} & \mc{1}{c}{\scriptsize{-0.630}} & \mc{1}{c}{\scriptsize{-0.131}} \\  

     &  & \mc{1}{c}{\scriptsize{(0.987)}} & \mc{1}{c}{\scriptsize{(0.961)}} & \mc{1}{c}{\scriptsize{(1.000)}} & \mc{1}{c}{\scriptsize{(1.000)}} & \mc{1}{c}{\scriptsize{(1.000)}} & \mc{1}{c}{\scriptsize{(0.908)}} & \mc{1}{c}{\scriptsize{(0.947)}} & \mc{1}{c}{\scriptsize{(0.947)}} \\  

    \mc{1}{l}{\scriptsize{Physical appearance}} & \mc{1}{c}{\scriptsize{12}} & \mc{1}{c}{\scriptsize{-0.646}} & \mc{1}{c}{\scriptsize{-0.492}} & \mc{1}{c}{\scriptsize{-1.682}} & \mc{1}{c}{\scriptsize{-0.050}} & \mc{1}{c}{\scriptsize{-1.661}} & \mc{1}{c}{\scriptsize{-0.566}} & \mc{1}{c}{\scriptsize{-0.450}} & \mc{1}{c}{\scriptsize{-0.557}} \\  

     &  & \mc{1}{c}{\scriptsize{(1.000)}} & \mc{1}{c}{\scriptsize{(0.947)}} & \mc{1}{c}{\scriptsize{(1.000)}} & \mc{1}{c}{\scriptsize{(0.967)}} & \mc{1}{c}{\scriptsize{(1.000)}} & \mc{1}{c}{\scriptsize{(1.000)}} & \mc{1}{c}{\scriptsize{(0.947)}} & \mc{1}{c}{\scriptsize{(1.000)}} \\  

  \bottomrule
  \end{tabular}
	\end{table} 

	\begin{table}[H]
     \caption{Treatment Effects on Smoking and Drinking Behavior, Female Sample}
     \label{table:abccare_rslt_female_cat52_sd}
	  \begin{tabular}{cccccccccc}
  \toprule

    \scriptsize{Variable} & \scriptsize{Age} & \scriptsize{(1)} & \scriptsize{(2)} & \scriptsize{(3)} & \scriptsize{(4)} & \scriptsize{(5)} & \scriptsize{(6)} & \scriptsize{(7)} & \scriptsize{(8)} \\ 
    \midrule  

    \mc{1}{l}{\scriptsize{Temperament cluster - task orientation}} & \mc{1}{c}{\scriptsize{1}} & \mc{1}{c}{\scriptsize{0.940}} & \mc{1}{c}{\scriptsize{0.974}} & \mc{1}{c}{\scriptsize{0.626}} & \mc{1}{c}{\scriptsize{1.285}} & \mc{1}{c}{\scriptsize{0.837}} & \mc{1}{c}{\scriptsize{0.960}} & \mc{1}{c}{\scriptsize{0.934}} & \mc{1}{c}{\scriptsize{1.004}} \\  

     &  & \mc{1}{c}{\scriptsize{(0.750)}} & \mc{1}{c}{\scriptsize{(0.724)}} & \mc{1}{c}{\scriptsize{(0.947)}} & \mc{1}{c}{\scriptsize{(0.789)}} & \mc{1}{c}{\scriptsize{(0.921)}} & \mc{1}{c}{\scriptsize{(0.853)}} & \mc{1}{c}{\scriptsize{(0.597)}} & \mc{1}{c}{\scriptsize{(0.838)}} \\  

    \mc{1}{l}{\scriptsize{Temperament cluster - activity level}} & \mc{1}{c}{\scriptsize{1.5}} & \mc{1}{c}{\scriptsize{-0.045}} & \mc{1}{c}{\scriptsize{0.401}} & \mc{1}{c}{\scriptsize{-0.051}} & \mc{1}{c}{\scriptsize{0.523}} & \mc{1}{c}{\scriptsize{0.399}} & \mc{1}{c}{\scriptsize{-0.384}} & \mc{1}{c}{\scriptsize{0.293}} & \mc{1}{c}{\scriptsize{-0.035}} \\  

     &  & \mc{1}{c}{\scriptsize{(1.000)}} & \mc{1}{c}{\scriptsize{(0.974)}} & \mc{1}{c}{\scriptsize{(1.000)}} & \mc{1}{c}{\scriptsize{(0.987)}} & \mc{1}{c}{\scriptsize{(1.000)}} & \mc{1}{c}{\scriptsize{(1.000)}} & \mc{1}{c}{\scriptsize{(1.000)}} & \mc{1}{c}{\scriptsize{(1.000)}} \\  

    \mc{1}{l}{\scriptsize{Temperament cluster - sociability}} & \mc{1}{c}{\scriptsize{2}} & \mc{1}{c}{\scriptsize{0.232}} & \mc{1}{c}{\scriptsize{0.059}} & \mc{1}{c}{\scriptsize{0.510}} & \mc{1}{c}{\scriptsize{-0.068}} & \mc{1}{c}{\scriptsize{-0.031}} & \mc{1}{c}{\scriptsize{0.176}} & \mc{1}{c}{\scriptsize{0.129}} & \mc{1}{c}{\scriptsize{0.129}} \\  

     &  & \mc{1}{c}{\scriptsize{(0.987)}} & \mc{1}{c}{\scriptsize{(1.000)}} & \mc{1}{c}{\scriptsize{(0.882)}} & \mc{1}{c}{\scriptsize{(1.000)}} & \mc{1}{c}{\scriptsize{(1.000)}} & \mc{1}{c}{\scriptsize{(0.987)}} & \mc{1}{c}{\scriptsize{(0.972)}} & \mc{1}{c}{\scriptsize{(1.000)}} \\  

    \mc{1}{l}{\scriptsize{Temperament cluster - activity level}} & \mc{1}{c}{\scriptsize{0.5}} & \mc{1}{c}{\scriptsize{1.278}} & \mc{1}{c}{\scriptsize{0.489}} & \mc{1}{c}{\scriptsize{2.192}} & \mc{1}{c}{\scriptsize{1.393}} & \mc{1}{c}{\scriptsize{2.129}} & \mc{1}{c}{\scriptsize{0.896}} & \mc{1}{c}{\scriptsize{0.166}} & \mc{1}{c}{\scriptsize{0.707}} \\  

     &  & \mc{1}{c}{\scriptsize{(0.250)}} & \mc{1}{c}{\scriptsize{(0.934)}} & \mc{1}{c}{\scriptsize{(0.250)}} & \mc{1}{c}{\scriptsize{(0.776)}} & \mc{1}{c}{\scriptsize{(0.276)}} & \mc{1}{c}{\scriptsize{(0.880)}} & \mc{1}{c}{\scriptsize{(1.000)}} & \mc{1}{c}{\scriptsize{(0.946)}} \\  

     & \mc{1}{c}{\scriptsize{1}} & \mc{1}{c}{\scriptsize{1.280}} & \mc{1}{c}{\scriptsize{0.842}} & \mc{1}{c}{\scriptsize{0.374}} & \mc{1}{c}{\scriptsize{-0.991}} & \mc{1}{c}{\scriptsize{0.442}} & \mc{1}{c}{\scriptsize{1.344}} & \mc{1}{c}{\scriptsize{1.632}} & \mc{1}{c}{\scriptsize{1.398}} \\  

     &  & \mc{1}{c}{\scriptsize{(0.368)}} & \mc{1}{c}{\scriptsize{(0.895)}} & \mc{1}{c}{\scriptsize{(1.000)}} & \mc{1}{c}{\scriptsize{(1.000)}} & \mc{1}{c}{\scriptsize{(1.000)}} & \mc{1}{c}{\scriptsize{(0.640)}} & \mc{1}{c}{\scriptsize{(1.000)}} & \mc{1}{c}{\scriptsize{(0.608)}} \\  

    \mc{1}{l}{\scriptsize{Temperament cluster - sociability}} & \mc{1}{c}{\scriptsize{0.5}} & \mc{1}{c}{\scriptsize{0.240}} & \mc{1}{c}{\scriptsize{0.146}} & \mc{1}{c}{\scriptsize{0.960}} & \mc{1}{c}{\scriptsize{0.813}} & \mc{1}{c}{\scriptsize{1.016}} & \mc{1}{c}{\scriptsize{-0.077}} & \mc{1}{c}{\scriptsize{-0.045}} & \mc{1}{c}{\scriptsize{-0.082}} \\  

     &  & \mc{1}{c}{\scriptsize{(0.934)}} & \mc{1}{c}{\scriptsize{(0.974)}} & \mc{1}{c}{\scriptsize{(0.342)}} & \mc{1}{c}{\scriptsize{(0.645)}} & \mc{1}{c}{\scriptsize{(0.342)}} & \mc{1}{c}{\scriptsize{(1.000)}} & \mc{1}{c}{\scriptsize{(1.000)}} & \mc{1}{c}{\scriptsize{(1.000)}} \\  

    \mc{1}{l}{\scriptsize{Temperament cluster - cooperativeness}} & \mc{1}{c}{\scriptsize{2}} & \mc{1}{c}{\scriptsize{0.556}} & \mc{1}{c}{\scriptsize{1.456}} & \mc{1}{c}{\scriptsize{1.500}} & \mc{1}{c}{\scriptsize{4.157}} & \mc{1}{c}{\scriptsize{2.111}} & \mc{1}{c}{\scriptsize{-0.200}} & \mc{1}{c}{\scriptsize{3.475}} & \mc{1}{c}{\scriptsize{0.408}} \\  

     &  & \mc{1}{c}{\scriptsize{(0.987)}} & \mc{1}{c}{\scriptsize{(0.974)}} & \mc{1}{c}{\scriptsize{(0.974)}} & \mc{1}{c}{\scriptsize{(0.961)}} & \mc{1}{c}{\scriptsize{(0.934)}} & \mc{1}{c}{\scriptsize{(1.000)}} & \mc{1}{c}{\scriptsize{(1.000)}} & \mc{1}{c}{\scriptsize{(0.986)}} \\  

     & \mc{1}{c}{\scriptsize{1}} & \mc{1}{c}{\scriptsize{4.111}} & \mc{1}{c}{\scriptsize{4.704}} & \mc{1}{c}{\scriptsize{2.583}} & \mc{1}{c}{\scriptsize{8.856}} & \mc{1}{c}{\scriptsize{3.455}} & \mc{1}{c}{\scriptsize{5.333}} & \mc{1}{c}{\scriptsize{1.325}} & \mc{1}{c}{\scriptsize{6.192}} \\  

     &  & \mc{1}{c}{\scriptsize{(0.605)}} & \mc{1}{c}{\scriptsize{(0.842)}} & \mc{1}{c}{\scriptsize{(0.908)}} & \mc{1}{c}{\scriptsize{(0.961)}} & \mc{1}{c}{\scriptsize{(0.816)}} & \mc{1}{c}{\scriptsize{(0.533)}} & \mc{1}{c}{\scriptsize{(1.000)}} & \mc{1}{c}{\scriptsize{(0.419)}} \\  

    \mc{1}{l}{\scriptsize{Temperament cluster - task orientation}} & \mc{1}{c}{\scriptsize{0.5}} & \mc{1}{c}{\scriptsize{0.145}} & \mc{1}{c}{\scriptsize{0.242}} & \mc{1}{c}{\scriptsize{2.212}} & \mc{1}{c}{\scriptsize{2.447}} & \mc{1}{c}{\scriptsize{2.769}} & \mc{1}{c}{\scriptsize{-0.677}} & \mc{1}{c}{\scriptsize{-0.768}} & \mc{1}{c}{\scriptsize{-0.312}} \\  

     &  & \mc{1}{c}{\scriptsize{(1.000)}} & \mc{1}{c}{\scriptsize{(1.000)}} & \mc{1}{c}{\scriptsize{(0.342)}} & \mc{1}{c}{\scriptsize{(0.276)}} & \mc{1}{c}{\scriptsize{(0.237)}} & \mc{1}{c}{\scriptsize{(1.000)}} & \mc{1}{c}{\scriptsize{(1.000)}} & \mc{1}{c}{\scriptsize{(1.000)}} \\  

    \mc{1}{l}{\scriptsize{Temperament cluster - sociability}} & \mc{1}{c}{\scriptsize{1.5}} & \mc{1}{c}{\scriptsize{1.074}} & \mc{1}{c}{\scriptsize{1.339}} & \mc{1}{c}{\scriptsize{1.626}} & \mc{1}{c}{\scriptsize{1.532}} & \mc{1}{c}{\scriptsize{1.812}} & \mc{1}{c}{\scriptsize{0.951}} & \mc{1}{c}{\scriptsize{1.280}} & \mc{1}{c}{\scriptsize{1.168}} \\  

     &  & \mc{1}{c}{\scriptsize{\textbf{(0.013)}}} & \mc{1}{c}{\scriptsize{\textbf{(0.053)}}} & \mc{1}{c}{\scriptsize{\textbf{(0.026)}}} & \mc{1}{c}{\scriptsize{(0.132)}} & \mc{1}{c}{\scriptsize{\textbf{(0.013)}}} & \mc{1}{c}{\scriptsize{(0.320)}} & \mc{1}{c}{\scriptsize{(1.000)}} & \mc{1}{c}{\scriptsize{(0.135)}} \\  

    \mc{1}{l}{\scriptsize{Temperament cluster - cooperativeness}} & \mc{1}{c}{\scriptsize{1.5}} & \mc{1}{c}{\scriptsize{-0.944}} & \mc{1}{c}{\scriptsize{-0.688}} & \mc{1}{c}{\scriptsize{-2.000}} & \mc{1}{c}{\scriptsize{-6.576}} & \mc{1}{c}{\scriptsize{-1.706}} & \mc{1}{c}{\scriptsize{-0.100}} & \mc{1}{c}{\scriptsize{2.432}} & \mc{1}{c}{\scriptsize{0.190}} \\  

     &  & \mc{1}{c}{\scriptsize{(1.000)}} & \mc{1}{c}{\scriptsize{(1.000)}} & \mc{1}{c}{\scriptsize{(1.000)}} & \mc{1}{c}{\scriptsize{(1.000)}} & \mc{1}{c}{\scriptsize{(1.000)}} & \mc{1}{c}{\scriptsize{(1.000)}} & \mc{1}{c}{\scriptsize{(1.000)}} & \mc{1}{c}{\scriptsize{(1.000)}} \\  

    \mc{1}{l}{\scriptsize{Temperament cluster - activity level}} & \mc{1}{c}{\scriptsize{2}} & \mc{1}{c}{\scriptsize{0.383}} & \mc{1}{c}{\scriptsize{0.184}} & \mc{1}{c}{\scriptsize{-0.342}} & \mc{1}{c}{\scriptsize{-1.578}} & \mc{1}{c}{\scriptsize{-1.443}} & \mc{1}{c}{\scriptsize{0.637}} & \mc{1}{c}{\scriptsize{1.295}} & \mc{1}{c}{\scriptsize{0.992}} \\  

     &  & \mc{1}{c}{\scriptsize{(1.000)}} & \mc{1}{c}{\scriptsize{(1.000)}} & \mc{1}{c}{\scriptsize{(1.000)}} & \mc{1}{c}{\scriptsize{(1.000)}} & \mc{1}{c}{\scriptsize{(1.000)}} & \mc{1}{c}{\scriptsize{(0.947)}} & \mc{1}{c}{\scriptsize{(1.000)}} & \mc{1}{c}{\scriptsize{(0.946)}} \\  

    \mc{1}{l}{\scriptsize{Temperament cluster - sociability}} & \mc{1}{c}{\scriptsize{1}} & \mc{1}{c}{\scriptsize{0.527}} & \mc{1}{c}{\scriptsize{0.429}} & \mc{1}{c}{\scriptsize{0.838}} & \mc{1}{c}{\scriptsize{0.836}} & \mc{1}{c}{\scriptsize{0.842}} & \mc{1}{c}{\scriptsize{0.394}} & \mc{1}{c}{\scriptsize{0.409}} & \mc{1}{c}{\scriptsize{0.386}} \\  

     &  & \mc{1}{c}{\scriptsize{(0.237)}} & \mc{1}{c}{\scriptsize{(0.421)}} & \mc{1}{c}{\scriptsize{(0.605)}} & \mc{1}{c}{\scriptsize{(0.539)}} & \mc{1}{c}{\scriptsize{(0.553)}} & \mc{1}{c}{\scriptsize{(0.947)}} & \mc{1}{c}{\scriptsize{(1.000)}} & \mc{1}{c}{\scriptsize{(0.959)}} \\  

    \mc{1}{l}{\scriptsize{Temperament cluster - task orientation}} & \mc{1}{c}{\scriptsize{1.5}} & \mc{1}{c}{\scriptsize{2.939}} & \mc{1}{c}{\scriptsize{2.892}} & \mc{1}{c}{\scriptsize{3.495}} & \mc{1}{c}{\scriptsize{3.346}} & \mc{1}{c}{\scriptsize{3.638}} & \mc{1}{c}{\scriptsize{2.606}} & \mc{1}{c}{\scriptsize{2.511}} & \mc{1}{c}{\scriptsize{2.790}} \\  

     &  & \mc{1}{c}{\scriptsize{\textbf{(0.000)}}} & \mc{1}{c}{\scriptsize{\textbf{(0.053)}}} & \mc{1}{c}{\scriptsize{\textbf{(0.000)}}} & \mc{1}{c}{\scriptsize{(0.118)}} & \mc{1}{c}{\scriptsize{\textbf{(0.013)}}} & \mc{1}{c}{\scriptsize{\textbf{(0.027)}}} & \mc{1}{c}{\scriptsize{(0.125)}} & \mc{1}{c}{\scriptsize{\textbf{(0.041)}}} \\  

    \mc{1}{l}{\scriptsize{Temperament cluster - cooperativeness}} & \mc{1}{c}{\scriptsize{0.5}} & \mc{1}{c}{\scriptsize{1.156}} & \mc{1}{c}{\scriptsize{-2.872}} & \mc{1}{c}{\scriptsize{-1.150}} & \mc{1}{c}{\scriptsize{-2.884}} & \mc{1}{c}{\scriptsize{-1.016}} & \mc{1}{c}{\scriptsize{3.000}} & \mc{1}{c}{\scriptsize{3.079}} & \mc{1}{c}{\scriptsize{3.146}} \\  

     &  & \mc{1}{c}{\scriptsize{(0.987)}} & \mc{1}{c}{\scriptsize{(1.000)}} & \mc{1}{c}{\scriptsize{(1.000)}} & \mc{1}{c}{\scriptsize{(1.000)}} & \mc{1}{c}{\scriptsize{(1.000)}} & \mc{1}{c}{\scriptsize{(0.880)}} & \mc{1}{c}{\scriptsize{(1.000)}} & \mc{1}{c}{\scriptsize{(0.905)}} \\  

    \mc{1}{l}{\scriptsize{Temperament cluster - task orientation}} & \mc{1}{c}{\scriptsize{2}} & \mc{1}{c}{\scriptsize{1.371}} & \mc{1}{c}{\scriptsize{1.230}} & \mc{1}{c}{\scriptsize{1.646}} & \mc{1}{c}{\scriptsize{3.581}} & \mc{1}{c}{\scriptsize{3.291}} & \mc{1}{c}{\scriptsize{1.270}} & \mc{1}{c}{\scriptsize{0.507}} & \mc{1}{c}{\scriptsize{1.306}} \\  

     &  & \mc{1}{c}{\scriptsize{(0.711)}} & \mc{1}{c}{\scriptsize{(0.763)}} & \mc{1}{c}{\scriptsize{(0.803)}} & \mc{1}{c}{\scriptsize{(0.368)}} & \mc{1}{c}{\scriptsize{(0.368)}} & \mc{1}{c}{\scriptsize{(0.853)}} & \mc{1}{c}{\scriptsize{(0.875)}} & \mc{1}{c}{\scriptsize{(0.851)}} \\  

  \bottomrule
  \end{tabular}
	\end{table} 

	\begin{table}[H]
     \caption{Treatment Effects on Tobacco, Drugs, Alcohol, Female Sample}
     \label{table:abccare_rslt_female_cat53_sd}
	\input{AppResOutput/abccare/rslt_female_cat53_sd}
	\end{table} 
\clearpage





%
\input{Preamble} 

\title{ABC Treatment Effects: Preliminary Estimates} 

\date{\today} 

\begin{document} 

\maketitle 

\tableofcontents 

\clearpage 


\def\arraystretch{0.6}

\setlength\tabcolsep{0.3em}

\section{{Combining Functions, Aggregated}}


\begin{center}
	  \begin{tabular}{ccccccccc}
  \toprule

     & \scriptsize{(1)} & \scriptsize{(2)} & \scriptsize{(3)} & \scriptsize{(4)} & \scriptsize{(5)} & \scriptsize{(6)} & \scriptsize{(7)} & \scriptsize{(8)} \\ 
    \midrule  

    \mc{1}{l}{\scriptsize{\% Pos. TE}} & \mc{1}{c}{\scriptsize{73}} & \mc{1}{c}{\scriptsize{75}} & \mc{1}{c}{\scriptsize{80}} & \mc{1}{c}{\scriptsize{79}} & \mc{1}{c}{\scriptsize{80}} & \mc{1}{c}{\scriptsize{75}} & \mc{1}{c}{\scriptsize{76}} & \mc{1}{c}{\scriptsize{74}} \\  

     & \mc{1}{c}{\scriptsize{\textbf{(0.000)}}} & \mc{1}{c}{\scriptsize{\textbf{(0.000)}}} & \mc{1}{c}{\scriptsize{\textbf{(0.000)}}} & \mc{1}{c}{\scriptsize{\textbf{(0.000)}}} & \mc{1}{c}{\scriptsize{\textbf{(0.000)}}} & \mc{1}{c}{\scriptsize{\textbf{(0.000)}}} & \mc{1}{c}{\scriptsize{\textbf{(0.000)}}} & \mc{1}{c}{\scriptsize{\textbf{(0.000)}}} \\  

    \mc{1}{l}{\scriptsize{\% Pos. TE $|$ 10\% Significance}} & \mc{1}{c}{\scriptsize{41}} & \mc{1}{c}{\scriptsize{43}} & \mc{1}{c}{\scriptsize{46}} & \mc{1}{c}{\scriptsize{46}} & \mc{1}{c}{\scriptsize{45}} & \mc{1}{c}{\scriptsize{40}} & \mc{1}{c}{\scriptsize{39}} & \mc{1}{c}{\scriptsize{41}} \\  

     & \mc{1}{c}{\scriptsize{\textbf{(0.000)}}} & \mc{1}{c}{\scriptsize{\textbf{(0.000)}}} & \mc{1}{c}{\scriptsize{\textbf{(0.000)}}} & \mc{1}{c}{\scriptsize{\textbf{(0.000)}}} & \mc{1}{c}{\scriptsize{\textbf{(0.000)}}} & \mc{1}{c}{\scriptsize{\textbf{(0.000)}}} & \mc{1}{c}{\scriptsize{\textbf{(0.000)}}} & \mc{1}{c}{\scriptsize{\textbf{(0.000)}}} \\  

  \bottomrule
  \end{tabular}
\end{center}

\begin{center}
	\begin{table}[H]
\captionsetup{singlelinecheck=false,justification=centering}
\caption{CARE Percentage of Significant Treatment Effects, Males \label{tab:counts_male}}

  \begin{threeparttable}
  \begin{tabular}{ccccccccc}
  \hline\hline

     & \scriptsize{(1)} & \scriptsize{(2)} & \scriptsize{(3)} & \scriptsize{(4)} & \scriptsize{(5)} & \scriptsize{(6)} & \scriptsize{(7)} & \scriptsize{(8)} \\  

     &  &  & \mc{3}{c}{\scriptsize{$P=0$}} & \mc{3}{c}{\scriptsize{$P=1$}} \\ 
    \cmidrule(lr){4-6} \cmidrule(lr){7-9} 

    \scriptsize{Variable} & \scriptsize{ITT} & \scriptsize{ITT$|X,W$} & \scriptsize{ITT} & \scriptsize{ITT$|X,W$} & \scriptsize{KE$|X,W$} & \scriptsize{ITT} & \scriptsize{ITT$|X,W$} & \scriptsize{KE$|X,W$} \\ 
    \hline  

    \\[0.1cm]
    \mc{1}{l}{\scriptsize{\% Sig. TE}} & \mc{1}{c}{\scriptsize{131}} & \mc{1}{c}{\scriptsize{108}} & \mc{1}{c}{\scriptsize{136}} & \mc{1}{c}{\scriptsize{125}} & \mc{1}{c}{\scriptsize{138}} & \mc{1}{c}{\scriptsize{145}} & \mc{1}{c}{\scriptsize{169}} & \mc{1}{c}{\scriptsize{80}} \\  

    \mc{1}{l}{\scriptsize{$H_0$: $\le$ 25\%}} & \mc{1}{c}{\scriptsize{\textbf{(0.059)}}} & \mc{1}{c}{\scriptsize{(0.216)}} & \mc{1}{c}{\scriptsize{\textbf{(0.039)}}} & \mc{1}{c}{\scriptsize{(0.216)}} & \mc{1}{c}{\scriptsize{\textbf{(0.000)}}} & \mc{1}{c}{\scriptsize{\textbf{(0.039)}}} & \mc{1}{c}{\scriptsize{(0.118)}} & \mc{1}{c}{\scriptsize{(0.137)}} \\  

    \mc{1}{l}{\scriptsize{$H_0$: $\le$ 50\%}} & \mc{1}{c}{\scriptsize{\textbf{(0.078)}}} & \mc{1}{c}{\scriptsize{(0.255)}} & \mc{1}{c}{\scriptsize{\textbf{(0.059)}}} & \mc{1}{c}{\scriptsize{(0.216)}} & \mc{1}{c}{\scriptsize{\textbf{(0.020)}}} & \mc{1}{c}{\scriptsize{\textbf{(0.059)}}} & \mc{1}{c}{\scriptsize{(0.118)}} & \mc{1}{c}{\scriptsize{(0.275)}} \\  

    \mc{1}{l}{\scriptsize{$H_0$: $\le$ 75\%}} & \mc{1}{c}{\scriptsize{\textbf{(0.098)}}} & \mc{1}{c}{\scriptsize{(0.275)}} & \mc{1}{c}{\scriptsize{\textbf{(0.098)}}} & \mc{1}{c}{\scriptsize{(0.235)}} & \mc{1}{c}{\scriptsize{(0.118)}} & \mc{1}{c}{\scriptsize{\textbf{(0.098)}}} & \mc{1}{c}{\scriptsize{(0.137)}} & \mc{1}{c}{\scriptsize{(0.412)}} \\ 
    \hline  

    \\[0.1cm]
    \mc{1}{l}{\scriptsize{\% Sig. TE $|$ 10\% Significance}} & \mc{1}{c}{\scriptsize{31}} & \mc{1}{c}{\scriptsize{29}} & \mc{1}{c}{\scriptsize{31}} & \mc{1}{c}{\scriptsize{35}} & \mc{1}{c}{\scriptsize{43}} & \mc{1}{c}{\scriptsize{66}} & \mc{1}{c}{\scriptsize{29}} & \mc{1}{c}{\scriptsize{35}} \\  

    \mc{1}{l}{\scriptsize{$H_0$: $\le$ 25\%}} & \mc{1}{c}{\scriptsize{(0.373)}} & \mc{1}{c}{\scriptsize{(0.294)}} & \mc{1}{c}{\scriptsize{(0.294)}} & \mc{1}{c}{\scriptsize{(0.275)}} & \mc{1}{c}{\scriptsize{(0.294)}} & \mc{1}{c}{\scriptsize{(0.157)}} & \mc{1}{c}{\scriptsize{(0.333)}} & \mc{1}{c}{\scriptsize{(0.275)}} \\  

    \mc{1}{l}{\scriptsize{$H_0$: $\le$ 50\%}} & \mc{1}{c}{\scriptsize{(0.549)}} & \mc{1}{c}{\scriptsize{(0.294)}} & \mc{1}{c}{\scriptsize{(0.471)}} & \mc{1}{c}{\scriptsize{(0.294)}} & \mc{1}{c}{\scriptsize{(0.412)}} & \mc{1}{c}{\scriptsize{(0.294)}} & \mc{1}{c}{\scriptsize{(0.451)}} & \mc{1}{c}{\scriptsize{(0.725)}} \\  

    \mc{1}{l}{\scriptsize{$H_0$: $\le$ 75\%}} & \mc{1}{c}{\scriptsize{(0.784)}} & \mc{1}{c}{\scriptsize{(0.333)}} & \mc{1}{c}{\scriptsize{(0.686)}} & \mc{1}{c}{\scriptsize{(0.314)}} & \mc{1}{c}{\scriptsize{(0.608)}} & \mc{1}{c}{\scriptsize{(0.471)}} & \mc{1}{c}{\scriptsize{(0.529)}} & \mc{1}{c}{\scriptsize{(0.902)}} \\ 
    \hline  

    \\[0.1cm]
    \mc{1}{l}{\scriptsize{\% Sig. TE $|$ 5\% Significance}} & \mc{1}{c}{\scriptsize{21}} & \mc{1}{c}{\scriptsize{17}} & \mc{1}{c}{\scriptsize{28}} & \mc{1}{c}{\scriptsize{9}} & \mc{1}{c}{\scriptsize{40}} & \mc{1}{c}{\scriptsize{33}} & \mc{1}{c}{\scriptsize{10}} & \mc{1}{c}{\scriptsize{31}} \\  

    \mc{1}{l}{\scriptsize{$H_0$: $\le$ 25\%}} & \mc{1}{c}{\scriptsize{(0.373)}} & \mc{1}{c}{\scriptsize{(0.275)}} & \mc{1}{c}{\scriptsize{(0.314)}} & \mc{1}{c}{\scriptsize{(0.255)}} & \mc{1}{c}{\scriptsize{(0.294)}} & \mc{1}{c}{\scriptsize{(0.392)}} & \mc{1}{c}{\scriptsize{(0.373)}} & \mc{1}{c}{\scriptsize{(0.294)}} \\  

    \mc{1}{l}{\scriptsize{$H_0$: $\le$ 50\%}} & \mc{1}{c}{\scriptsize{(0.667)}} & \mc{1}{c}{\scriptsize{(0.294)}} & \mc{1}{c}{\scriptsize{(0.510)}} & \mc{1}{c}{\scriptsize{(0.275)}} & \mc{1}{c}{\scriptsize{(0.451)}} & \mc{1}{c}{\scriptsize{(0.490)}} & \mc{1}{c}{\scriptsize{(0.490)}} & \mc{1}{c}{\scriptsize{(0.804)}} \\  

    \mc{1}{l}{\scriptsize{$H_0$: $\le$ 75\%}} & \mc{1}{c}{\scriptsize{(0.980)}} & \mc{1}{c}{\scriptsize{(0.353)}} & \mc{1}{c}{\scriptsize{(0.667)}} & \mc{1}{c}{\scriptsize{(0.333)}} & \mc{1}{c}{\scriptsize{(0.627)}} & \mc{1}{c}{\scriptsize{(0.824)}} & \mc{1}{c}{\scriptsize{(0.784)}} & \mc{1}{c}{\scriptsize{(0.922)}} \\  

  \hline\hline
  \end{tabular}
    \begin{tablenotes}
    \scriptsize
    \item 
Note: This table displays the percentage of the 78 outcomes for which we estimate significant
treatment effects. For outcomes where a negative treatment effect is beneficial to the subjects
(e.g. prevalence of diabetes), we reverse the signs of treatment effects so that all beneficial 
effects have positive signs.
Column (1) correpsonds to the ITT, without accounting for any controls.
Column (2) correpsonds to the ITT conditioning on vector of controls, $X$, consisting of APGAR scores 1 
minute after birth, an indicator for the subject being born prematurely, and an indicator for the 
father being home at baseline. We also apply IPW weights, $W$, to account for attrition.
Columns (3)--(4) are analogous to columns (1)--(2), but we restrict the control sample to subjects
who did not enroll in any alternative care.
Column (5) correpsonds to the matching estimate, where we use the Mahalanobis metric and Epanechnikov kernel
to match on controls $X$ listed above, and restrict the control sample to subjects who did not enroll
in any alternative care. Additionally, we apply IPW weights, $W$.
Columns (6)--(8) are analogous to Columns (3)--(5), except we restrict the control sample to subejcts
who did enroll in alternative care.
Numbers in parentheses represent the $p$-value from a single hypothesis test, and are obtained from 
the empirical bootstrap distribution generated by 200 resamples of the original data. 
Bold $p$-values indicate significance at the 10\% level. Blank point estimates indicate that
we are unable to obtain estimates due to a lack of support in the data. 

    \end{tablenotes}
  \end{threeparttable}

\end{table}
\end{center}

\begin{center}
	\begin{sidewaystable}[H]
\captionsetup{singlelinecheck=false,justification=centering}
\caption{Treatment Effects of Center-based Childcare on Females, Counts  \label{tab:counts_female}}

  \begin{threeparttable}
  \begin{tabular}{ccccccccc} \toprule

     & \footnotesize{(1)} & \footnotesize{(2)} & \footnotesize{(3)} & \footnotesize{(4)} & \footnotesize{(5)} & \footnotesize{(6)} & \footnotesize{(7)} & \footnotesize{(8)} \\  

     &  &  & \mc{3}{c}{\footnotesize{$P=0$}} & \mc{3}{c}{\footnotesize{$P=1$}} \\ 
    \cmidrule(lr){4-6} \cmidrule(lr){7-9} 

    \footnotesize{Variable} & \footnotesize{ITT} & \footnotesize{ITT$|X,W$} & \footnotesize{ITT} & \footnotesize{ITT$|X,W$} & \footnotesize{KE$|X,W$} & \footnotesize{ITT} & \footnotesize{ITT$|X,W$} & \footnotesize{KE$|X,W$} \\ 
    \midrule 

    \\[0.1cm]
    \mc{1}{l}{\footnotesize{\% Positive Treatment Effect}} & \mc{1}{c}{\footnotesize{79}} & \mc{1}{c}{\footnotesize{78}} & \mc{1}{c}{\footnotesize{81}} & \mc{1}{c}{\footnotesize{80}} & \mc{1}{c}{\footnotesize{81}} & \mc{1}{c}{\footnotesize{73}} & \mc{1}{c}{\footnotesize{72}} & \mc{1}{c}{\footnotesize{70}} \\  

    \mc{1}{l}{\footnotesize{$H_0$: $\le$ 25\%}} & \mc{1}{c}{\footnotesize{\textbf{(0.000)}}} & \mc{1}{c}{\footnotesize{\textbf{(0.000)}}} & \mc{1}{c}{\footnotesize{\textbf{(0.000)}}} & \mc{1}{c}{\footnotesize{\textbf{(0.000)}}} & \mc{1}{c}{\footnotesize{\textbf{(0.000)}}} & \mc{1}{c}{\footnotesize{\textbf{(0.000)}}} & \mc{1}{c}{\footnotesize{\textbf{(0.000)}}} & \mc{1}{c}{\footnotesize{\textbf{(0.000)}}} \\  

    \mc{1}{l}{\footnotesize{$H_0$: $\le$ 50\%}} & \mc{1}{c}{\footnotesize{\textbf{(0.000)}}} & \mc{1}{c}{\footnotesize{\textbf{(0.000)}}} & \mc{1}{c}{\footnotesize{\textbf{(0.000)}}} & \mc{1}{c}{\footnotesize{\textbf{(0.000)}}} & \mc{1}{c}{\footnotesize{\textbf{(0.000)}}} & \mc{1}{c}{\footnotesize{\textbf{(0.020)}}} & \mc{1}{c}{\footnotesize{\textbf{(0.000)}}} & \mc{1}{c}{\footnotesize{\textbf{(0.020)}}} \\  

    \mc{1}{l}{\footnotesize{$H_0$: $\le$ 75\%}} & \mc{1}{c}{\footnotesize{(0.333)}} & \mc{1}{c}{\footnotesize{(0.353)}} & \mc{1}{c}{\footnotesize{(0.216)}} & \mc{1}{c}{\footnotesize{(0.235)}} & \mc{1}{c}{\footnotesize{(0.294)}} & \mc{1}{c}{\footnotesize{(0.627)}} & \mc{1}{c}{\footnotesize{(0.667)}} & \mc{1}{c}{\footnotesize{(0.725)}} \\ 
    \midrule

    \\[0.1cm]
    \mc{1}{l}{\footnotesize{\% Pos. TE $|$ 10\% Significance}} & \mc{1}{c}{\footnotesize{37}} & \mc{1}{c}{\footnotesize{37}} & \mc{1}{c}{\footnotesize{49}} & \mc{1}{c}{\footnotesize{48}} & \mc{1}{c}{\footnotesize{45}} & \mc{1}{c}{\footnotesize{23}} & \mc{1}{c}{\footnotesize{27}} & \mc{1}{c}{\footnotesize{20}} \\  

    \mc{1}{l}{\footnotesize{$H_0$: $\le$ 25\%}} & \mc{1}{c}{\footnotesize{(0.118)}} & \mc{1}{c}{\footnotesize{\textbf{(0.078)}}} & \mc{1}{c}{\footnotesize{\textbf{(0.000)}}} & \mc{1}{c}{\footnotesize{\textbf{(0.000)}}} & \mc{1}{c}{\footnotesize{\textbf{(0.039)}}} & \mc{1}{c}{\footnotesize{(0.588)}} & \mc{1}{c}{\footnotesize{(0.373)}} & \mc{1}{c}{\footnotesize{(0.725)}} \\  

    \mc{1}{l}{\footnotesize{$H_0$: $\le$ 50\%}} & \mc{1}{c}{\footnotesize{(0.882)}} & \mc{1}{c}{\footnotesize{(0.922)}} & \mc{1}{c}{\footnotesize{(0.549)}} & \mc{1}{c}{\footnotesize{(0.588)}} & \mc{1}{c}{\footnotesize{(0.725)}} & \mc{1}{c}{\footnotesize{(1.000)}} & \mc{1}{c}{\footnotesize{(0.980)}} & \mc{1}{c}{\footnotesize{(1.000)}} \\  

    \mc{1}{l}{\footnotesize{$H_0$: $\le$ 75\%}} & \mc{1}{c}{\footnotesize{(1.000)}} & \mc{1}{c}{\footnotesize{(1.000)}} & \mc{1}{c}{\footnotesize{(1.000)}} & \mc{1}{c}{\footnotesize{(1.000)}} & \mc{1}{c}{\footnotesize{(1.000)}} & \mc{1}{c}{\footnotesize{(1.000)}} & \mc{1}{c}{\footnotesize{(1.000)}} & \mc{1}{c}{\footnotesize{(1.000)}} \\ 
    \toprule
  \end{tabular}
    \begin{tablenotes}
    \footnotesize
    \item 
Note: This table displays the percentage of the 95 outcomes for which we estimate positive
treatment effects. For outcomes where a negative treatment effect is beneficial to the subjects
(e.g. prevalence of diabetes), we reverse the signs of treatment effects so that all beneficial 
effects have positive signs.
Column (1) corresponds to the ITT, without accounting for any controls.
Column (2) corresponds to the ITT conditioning on vector of controls, $X$, consisting of Apgar scores 1 minute and 5 minutes after birth, the HRI index, maternal IQ,
an indicator for having a grandmother residing in the same county, and an index for the number
of relatives living in the same household. We also apply IPW weights, $W$, to account for attrition.
Columns (3)--(4) are analogous to columns (1)--(2), but we restrict the control sample to subjects
who did not enroll in any alternative care.
Column (5) corresponds to the matching estimate, where we use the Mahalanobis metric and Epanechnikov kernel
to match on controls $X$ listed above, and restrict the control sample to subjects who did not enroll
in any alternative care. Additionally, we apply IPW weights, $W$.
Columns (6)--(8) are analogous to Columns (3)--(5), except we restrict the control sample to subjects
who did enroll in alternative care. 
Numbers in parentheses represent the $p$-value from a single hypothesis test, and are obtained from 
the empirical bootstrap distribution generated by 200 resamples of the original data. 
Bold $p$-values indicate significance at the 10\% level. Blank point estimates indicate that
we are unable to obtain estimates due to a lack of support in the data. 

    \end{tablenotes}
  \end{threeparttable}
\end{sidewaystable}
\end{center}
\section{{Combining Functions, by Category}}


\begin{center}
	\input{abc/rslt_pooled_counts_n25a5}
\end{center}

\begin{center}
	\input{abc/rslt_pooled_counts_n25a10}
\end{center}

\begin{center}
	\begin{table}[H]
\captionsetup{singlelinecheck=false,justification=centering}
\caption{CARE Percentage of Significant Treatment Effects by Category, Males and Females \\ $H_0$: $\le$ 25\% \label{tab:counts_pooled}}

  \begin{threeparttable}
  \begin{tabular}{cccccccccc}
  \hline\hline

     & \scriptsize{(1)} & \scriptsize{(2)} & \scriptsize{(3)} & \scriptsize{(4)} & \scriptsize{(5)} & \scriptsize{(6)} & \scriptsize{(7)} & \scriptsize{(8)} &  \\  

     &  &  & \mc{3}{c}{\scriptsize{$P=0$}} & \mc{3}{c}{\scriptsize{$P=1$}} &  \\ 
    \cmidrule(lr){4-6} \cmidrule(lr){7-9} 

    \scriptsize{Category} & \scriptsize{ITT} & \scriptsize{ITT$|X,W$} & \scriptsize{ITT} & \scriptsize{ITT$|X,W$} & \scriptsize{KE$|X,W$} & \scriptsize{ITT} & \scriptsize{ITT$|X,W$} & \scriptsize{KE$|X,W$} & \scriptsize{Outcomes} \\ 
    \hline  

    \mc{1}{l}{\scriptsize{IQ Scores}} & \mc{1}{c}{\scriptsize{100}} & \mc{1}{c}{\scriptsize{100}} & \mc{1}{c}{\scriptsize{100}} & \mc{1}{c}{\scriptsize{100}} & \mc{1}{c}{\scriptsize{100}} & \mc{1}{c}{\scriptsize{100}} & \mc{1}{c}{\scriptsize{100}} & \mc{1}{c}{\scriptsize{100}} & \mc{1}{c}{\scriptsize{10}} \\  

     & \mc{1}{c}{\scriptsize{\textbf{(0.000)}}} & \mc{1}{c}{\scriptsize{\textbf{(0.000)}}} & \mc{1}{c}{\scriptsize{\textbf{(0.000)}}} & \mc{1}{c}{\scriptsize{\textbf{(0.000)}}} & \mc{1}{c}{\scriptsize{\textbf{(0.000)}}} & \mc{1}{c}{\scriptsize{\textbf{(0.000)}}} & \mc{1}{c}{\scriptsize{\textbf{(0.000)}}} & \mc{1}{c}{\scriptsize{\textbf{(0.000)}}} &  \\  

    \mc{1}{l}{\scriptsize{Achievement Scores}} & \mc{1}{c}{\scriptsize{100}} & \mc{1}{c}{\scriptsize{100}} & \mc{1}{c}{\scriptsize{100}} & \mc{1}{c}{\scriptsize{100}} & \mc{1}{c}{\scriptsize{100}} & \mc{1}{c}{\scriptsize{100}} & \mc{1}{c}{\scriptsize{100}} & \mc{1}{c}{\scriptsize{100}} & \mc{1}{c}{\scriptsize{6}} \\  

     & \mc{1}{c}{\scriptsize{\textbf{(0.000)}}} & \mc{1}{c}{\scriptsize{\textbf{(0.000)}}} & \mc{1}{c}{\scriptsize{\textbf{(0.000)}}} & \mc{1}{c}{\scriptsize{\textbf{(0.000)}}} & \mc{1}{c}{\scriptsize{\textbf{(0.000)}}} & \mc{1}{c}{\scriptsize{\textbf{(0.000)}}} & \mc{1}{c}{\scriptsize{\textbf{(0.000)}}} & \mc{1}{c}{\scriptsize{\textbf{(0.000)}}} &  \\  

    \mc{1}{l}{\scriptsize{HOME Scores}} & \mc{1}{c}{\scriptsize{100}} & \mc{1}{c}{\scriptsize{100}} & \mc{1}{c}{\scriptsize{100}} & \mc{1}{c}{\scriptsize{100}} & \mc{1}{c}{\scriptsize{100}} & \mc{1}{c}{\scriptsize{100}} & \mc{1}{c}{\scriptsize{100}} & \mc{1}{c}{\scriptsize{100}} & \mc{1}{c}{\scriptsize{6}} \\  

     & \mc{1}{c}{\scriptsize{\textbf{(0.000)}}} & \mc{1}{c}{\scriptsize{\textbf{(0.000)}}} & \mc{1}{c}{\scriptsize{\textbf{(0.000)}}} & \mc{1}{c}{\scriptsize{\textbf{(0.000)}}} & \mc{1}{c}{\scriptsize{\textbf{(0.000)}}} & \mc{1}{c}{\scriptsize{\textbf{(0.000)}}} & \mc{1}{c}{\scriptsize{\textbf{(0.000)}}} & \mc{1}{c}{\scriptsize{\textbf{(0.000)}}} &  \\  

    \mc{1}{l}{\scriptsize{Parent Income}} & \mc{1}{c}{\scriptsize{100}} & \mc{1}{c}{\scriptsize{100}} & \mc{1}{c}{\scriptsize{100}} & \mc{1}{c}{\scriptsize{100}} & \mc{1}{c}{\scriptsize{100}} & \mc{1}{c}{\scriptsize{100}} & \mc{1}{c}{\scriptsize{100}} & \mc{1}{c}{\scriptsize{100}} & \mc{1}{c}{\scriptsize{4}} \\  

     & \mc{1}{c}{\scriptsize{\textbf{(0.000)}}} & \mc{1}{c}{\scriptsize{\textbf{(0.000)}}} & \mc{1}{c}{\scriptsize{\textbf{(0.000)}}} & \mc{1}{c}{\scriptsize{\textbf{(0.000)}}} & \mc{1}{c}{\scriptsize{\textbf{(0.000)}}} & \mc{1}{c}{\scriptsize{\textbf{(0.000)}}} & \mc{1}{c}{\scriptsize{\textbf{(0.000)}}} & \mc{1}{c}{\scriptsize{\textbf{(0.000)}}} &  \\  

    \mc{1}{l}{\scriptsize{Mother's Employment}} & \mc{1}{c}{\scriptsize{100}} & \mc{1}{c}{\scriptsize{100}} & \mc{1}{c}{\scriptsize{100}} & \mc{1}{c}{\scriptsize{100}} & \mc{1}{c}{\scriptsize{100}} & \mc{1}{c}{\scriptsize{100}} & \mc{1}{c}{\scriptsize{100}} & \mc{1}{c}{\scriptsize{100}} & \mc{1}{c}{\scriptsize{4}} \\  

     & \mc{1}{c}{\scriptsize{\textbf{(0.000)}}} & \mc{1}{c}{\scriptsize{\textbf{(0.000)}}} & \mc{1}{c}{\scriptsize{\textbf{(0.000)}}} & \mc{1}{c}{\scriptsize{\textbf{(0.000)}}} & \mc{1}{c}{\scriptsize{\textbf{(0.000)}}} & \mc{1}{c}{\scriptsize{\textbf{(0.000)}}} & \mc{1}{c}{\scriptsize{\textbf{(0.000)}}} & \mc{1}{c}{\scriptsize{\textbf{(0.000)}}} &  \\  

    \mc{1}{l}{\scriptsize{Father at Home}} & \mc{1}{c}{\scriptsize{100}} & \mc{1}{c}{\scriptsize{100}} & \mc{1}{c}{\scriptsize{100}} & \mc{1}{c}{\scriptsize{100}} & \mc{1}{c}{\scriptsize{100}} & \mc{1}{c}{\scriptsize{100}} & \mc{1}{c}{\scriptsize{100}} & \mc{1}{c}{\scriptsize{100}} & \mc{1}{c}{\scriptsize{5}} \\  

     & \mc{1}{c}{\scriptsize{\textbf{(0.000)}}} & \mc{1}{c}{\scriptsize{\textbf{(0.000)}}} & \mc{1}{c}{\scriptsize{\textbf{(0.000)}}} & \mc{1}{c}{\scriptsize{\textbf{(0.000)}}} & \mc{1}{c}{\scriptsize{\textbf{(0.000)}}} & \mc{1}{c}{\scriptsize{\textbf{(0.000)}}} & \mc{1}{c}{\scriptsize{\textbf{(0.000)}}} & \mc{1}{c}{\scriptsize{\textbf{(0.000)}}} &  \\  

    \mc{1}{l}{\scriptsize{Education}} & \mc{1}{c}{\scriptsize{100}} & \mc{1}{c}{\scriptsize{100}} & \mc{1}{c}{\scriptsize{100}} & \mc{1}{c}{\scriptsize{100}} & \mc{1}{c}{\scriptsize{100}} & \mc{1}{c}{\scriptsize{100}} & \mc{1}{c}{\scriptsize{100}} & \mc{1}{c}{\scriptsize{100}} & \mc{1}{c}{\scriptsize{4}} \\  

     & \mc{1}{c}{\scriptsize{\textbf{(0.000)}}} & \mc{1}{c}{\scriptsize{\textbf{(0.000)}}} & \mc{1}{c}{\scriptsize{\textbf{(0.000)}}} & \mc{1}{c}{\scriptsize{\textbf{(0.000)}}} & \mc{1}{c}{\scriptsize{\textbf{(0.000)}}} & \mc{1}{c}{\scriptsize{\textbf{(0.000)}}} & \mc{1}{c}{\scriptsize{\textbf{(0.000)}}} & \mc{1}{c}{\scriptsize{\textbf{(0.000)}}} &  \\  

    \mc{1}{l}{\scriptsize{Employment and Income}} & \mc{1}{c}{\scriptsize{100}} & \mc{1}{c}{\scriptsize{100}} & \mc{1}{c}{\scriptsize{100}} & \mc{1}{c}{\scriptsize{100}} & \mc{1}{c}{\scriptsize{100}} & \mc{1}{c}{\scriptsize{100}} & \mc{1}{c}{\scriptsize{100}} & \mc{1}{c}{\scriptsize{100}} & \mc{1}{c}{\scriptsize{5}} \\  

     & \mc{1}{c}{\scriptsize{\textbf{(0.000)}}} & \mc{1}{c}{\scriptsize{\textbf{(0.000)}}} & \mc{1}{c}{\scriptsize{\textbf{(0.000)}}} & \mc{1}{c}{\scriptsize{\textbf{(0.000)}}} & \mc{1}{c}{\scriptsize{\textbf{(0.000)}}} & \mc{1}{c}{\scriptsize{\textbf{(0.000)}}} & \mc{1}{c}{\scriptsize{\textbf{(0.000)}}} & \mc{1}{c}{\scriptsize{\textbf{(0.000)}}} &  \\  

    \mc{1}{l}{\scriptsize{Crime}} & \mc{1}{c}{\scriptsize{100}} & \mc{1}{c}{\scriptsize{100}} & \mc{1}{c}{\scriptsize{100}} & \mc{1}{c}{\scriptsize{100}} & \mc{1}{c}{\scriptsize{100}} & \mc{1}{c}{\scriptsize{100}} & \mc{1}{c}{\scriptsize{100}} & \mc{1}{c}{\scriptsize{100}} & \mc{1}{c}{\scriptsize{3}} \\  

     & \mc{1}{c}{\scriptsize{\textbf{(0.000)}}} & \mc{1}{c}{\scriptsize{\textbf{(0.000)}}} & \mc{1}{c}{\scriptsize{\textbf{(0.000)}}} & \mc{1}{c}{\scriptsize{\textbf{(0.000)}}} & \mc{1}{c}{\scriptsize{\textbf{(0.000)}}} & \mc{1}{c}{\scriptsize{\textbf{(0.000)}}} & \mc{1}{c}{\scriptsize{\textbf{(0.000)}}} & \mc{1}{c}{\scriptsize{\textbf{(0.000)}}} &  \\  

    \mc{1}{l}{\scriptsize{Tobacco, Drugs, Alcohol}} & \mc{1}{c}{\scriptsize{100}} & \mc{1}{c}{\scriptsize{100}} & \mc{1}{c}{\scriptsize{100}} & \mc{1}{c}{\scriptsize{100}} & \mc{1}{c}{\scriptsize{100}} & \mc{1}{c}{\scriptsize{100}} & \mc{1}{c}{\scriptsize{100}} & \mc{1}{c}{\scriptsize{100}} & \mc{1}{c}{\scriptsize{4}} \\  

     & \mc{1}{c}{\scriptsize{\textbf{(0.000)}}} & \mc{1}{c}{\scriptsize{\textbf{(0.000)}}} & \mc{1}{c}{\scriptsize{\textbf{(0.000)}}} & \mc{1}{c}{\scriptsize{\textbf{(0.000)}}} & \mc{1}{c}{\scriptsize{\textbf{(0.000)}}} & \mc{1}{c}{\scriptsize{\textbf{(0.000)}}} & \mc{1}{c}{\scriptsize{\textbf{(0.000)}}} & \mc{1}{c}{\scriptsize{\textbf{(0.000)}}} &  \\  

    \mc{1}{l}{\scriptsize{Self-Reported Health}} & \mc{1}{c}{\scriptsize{100}} & \mc{1}{c}{\scriptsize{100}} & \mc{1}{c}{\scriptsize{100}} & \mc{1}{c}{\scriptsize{100}} & \mc{1}{c}{\scriptsize{100}} & \mc{1}{c}{\scriptsize{100}} & \mc{1}{c}{\scriptsize{100}} & \mc{1}{c}{\scriptsize{100}} & \mc{1}{c}{\scriptsize{2}} \\  

     & \mc{1}{c}{\scriptsize{\textbf{(0.000)}}} & \mc{1}{c}{\scriptsize{\textbf{(0.000)}}} & \mc{1}{c}{\scriptsize{\textbf{(0.000)}}} & \mc{1}{c}{\scriptsize{\textbf{(0.000)}}} & \mc{1}{c}{\scriptsize{\textbf{(0.000)}}} & \mc{1}{c}{\scriptsize{\textbf{(0.000)}}} & \mc{1}{c}{\scriptsize{\textbf{(0.000)}}} & \mc{1}{c}{\scriptsize{\textbf{(0.000)}}} &  \\  

    \mc{1}{l}{\scriptsize{Hypertension}} & \mc{1}{c}{\scriptsize{100}} & \mc{1}{c}{\scriptsize{100}} & \mc{1}{c}{\scriptsize{100}} & \mc{1}{c}{\scriptsize{100}} &  & \mc{1}{c}{\scriptsize{100}} & \mc{1}{c}{\scriptsize{100}} &  & \mc{1}{c}{\scriptsize{4}} \\  

     & \mc{1}{c}{\scriptsize{\textbf{(0.000)}}} & \mc{1}{c}{\scriptsize{\textbf{(0.000)}}} & \mc{1}{c}{\scriptsize{\textbf{(0.000)}}} & \mc{1}{c}{\scriptsize{\textbf{(0.000)}}} &  & \mc{1}{c}{\scriptsize{\textbf{(0.000)}}} & \mc{1}{c}{\scriptsize{\textbf{(0.000)}}} &  &  \\  

    \mc{1}{l}{\scriptsize{Cholesterol}} & \mc{1}{c}{\scriptsize{100}} & \mc{1}{c}{\scriptsize{100}} & \mc{1}{c}{\scriptsize{100}} & \mc{1}{c}{\scriptsize{100}} &  & \mc{1}{c}{\scriptsize{100}} & \mc{1}{c}{\scriptsize{100}} &  & \mc{1}{c}{\scriptsize{2}} \\  

     & \mc{1}{c}{\scriptsize{\textbf{(0.000)}}} & \mc{1}{c}{\scriptsize{\textbf{(0.000)}}} & \mc{1}{c}{\scriptsize{\textbf{(0.000)}}} & \mc{1}{c}{\scriptsize{\textbf{(0.000)}}} &  & \mc{1}{c}{\scriptsize{\textbf{(0.000)}}} & \mc{1}{c}{\scriptsize{\textbf{(0.000)}}} &  &  \\  

    \mc{1}{l}{\scriptsize{Diabetes}} & \mc{1}{c}{\scriptsize{100}} & \mc{1}{c}{\scriptsize{100}} & \mc{1}{c}{\scriptsize{100}} & \mc{1}{c}{\scriptsize{100}} &  & \mc{1}{c}{\scriptsize{100}} & \mc{1}{c}{\scriptsize{100}} &  & \mc{1}{c}{\scriptsize{2}} \\  

     & \mc{1}{c}{\scriptsize{\textbf{(0.000)}}} & \mc{1}{c}{\scriptsize{\textbf{(0.000)}}} & \mc{1}{c}{\scriptsize{\textbf{(0.000)}}} & \mc{1}{c}{\scriptsize{\textbf{(0.000)}}} &  & \mc{1}{c}{\scriptsize{\textbf{(0.000)}}} & \mc{1}{c}{\scriptsize{\textbf{(0.000)}}} &  &  \\  

    \mc{1}{l}{\scriptsize{Vitamin D Deficiency}} & \mc{1}{c}{\scriptsize{100}} & \mc{1}{c}{\scriptsize{100}} & \mc{1}{c}{\scriptsize{100}} & \mc{1}{c}{\scriptsize{100}} &  &  & \mc{1}{c}{\scriptsize{100}} &  & \mc{1}{c}{\scriptsize{1}} \\  

     & \mc{1}{c}{\scriptsize{\textbf{(0.000)}}} & \mc{1}{c}{\scriptsize{\textbf{(0.000)}}} & \mc{1}{c}{\scriptsize{\textbf{(0.000)}}} & \mc{1}{c}{\scriptsize{\textbf{(0.000)}}} &  &  & \mc{1}{c}{\scriptsize{\textbf{(0.000)}}} &  &  \\  

    \mc{1}{l}{\scriptsize{Obesity}} & \mc{1}{c}{\scriptsize{100}} & \mc{1}{c}{\scriptsize{100}} & \mc{1}{c}{\scriptsize{100}} & \mc{1}{c}{\scriptsize{100}} &  & \mc{1}{c}{\scriptsize{100}} & \mc{1}{c}{\scriptsize{100}} &  & \mc{1}{c}{\scriptsize{6}} \\  

     & \mc{1}{c}{\scriptsize{\textbf{(0.000)}}} & \mc{1}{c}{\scriptsize{\textbf{(0.000)}}} & \mc{1}{c}{\scriptsize{\textbf{(0.000)}}} & \mc{1}{c}{\scriptsize{\textbf{(0.000)}}} &  & \mc{1}{c}{\scriptsize{\textbf{(0.000)}}} & \mc{1}{c}{\scriptsize{\textbf{(0.000)}}} &  &  \\  

    \mc{1}{l}{\scriptsize{Mental Health}} & \mc{1}{c}{\scriptsize{100}} & \mc{1}{c}{\scriptsize{100}} & \mc{1}{c}{\scriptsize{100}} & \mc{1}{c}{\scriptsize{100}} & \mc{1}{c}{\scriptsize{100}} & \mc{1}{c}{\scriptsize{100}} & \mc{1}{c}{\scriptsize{100}} & \mc{1}{c}{\scriptsize{100}} & \mc{1}{c}{\scriptsize{10}} \\  

     & \mc{1}{c}{\scriptsize{\textbf{(0.000)}}} & \mc{1}{c}{\scriptsize{\textbf{(0.000)}}} & \mc{1}{c}{\scriptsize{\textbf{(0.000)}}} & \mc{1}{c}{\scriptsize{\textbf{(0.000)}}} & \mc{1}{c}{\scriptsize{\textbf{(0.000)}}} & \mc{1}{c}{\scriptsize{\textbf{(0.000)}}} & \mc{1}{c}{\scriptsize{\textbf{(0.000)}}} & \mc{1}{c}{\scriptsize{\textbf{(0.000)}}} &  \\  

  \hline\hline
  \end{tabular}
    \begin{tablenotes}
    \scriptsize
    \item 
Note: This table displays the percentage of the 78 outcomes for which we estimate significant
treatment effects. For outcomes where a negative treatment effect is beneficial to the subjects
(e.g. prevalence of diabetes), we reverse the signs of treatment effects so that all beneficial 
effects have positive signs.
Column (1) correpsonds to the ITT, without accounting for any controls.
Column (2) correpsonds to the ITT conditioning on vector of controls, $X$, consisting of APGAR scores 1 
minute after birth, an indicator for the subject being born prematurely, and an indicator for the 
father being home at baseline. We also apply IPW weights, $W$, to account for attrition.
Columns (3)--(4) are analogous to columns (1)--(2), but we restrict the control sample to subjects
who did not enroll in any alternative care.
Column (5) correpsonds to the matching estimate, where we use the Mahalanobis metric and Epanechnikov kernel
to match on controls $X$ listed above, and restrict the control sample to subjects who did not enroll
in any alternative care. Additionally, we apply IPW weights, $W$.
Columns (6)--(8) are analogous to Columns (3)--(5), except we restrict the control sample to subejcts
who did enroll in alternative care.
Numbers in parentheses represent the $p$-value from a single hypothesis test, and are obtained from 
the empirical bootstrap distribution generated by 200 resamples of the original data. 
Bold $p$-values indicate significance at the 10\% level. Blank point estimates indicate that
we are unable to obtain estimates due to a lack of support in the data. 

    \end{tablenotes}
  \end{threeparttable}

\end{table}
\end{center}

\begin{center}
	\input{abc/rslt_pooled_counts_n50a5}
\end{center}

\begin{center}
	\input{abc/rslt_pooled_counts_n50a10}
\end{center}

\begin{center}
	  \begin{tabular}{cccccccccc}
  \toprule

    \scriptsize{Category} & \scriptsize{(1)} & \scriptsize{(2)} & \scriptsize{(3)} & \scriptsize{(4)} & \scriptsize{(5)} & \scriptsize{(6)} & \scriptsize{(7)} & \scriptsize{(8)} & \scriptsize{N} \\ 
    \midrule  

    \mc{1}{l}{\scriptsize{Cognitive Skills}} & \mc{1}{c}{\scriptsize{93}} & \mc{1}{c}{\scriptsize{93}} & \mc{1}{c}{\scriptsize{93}} & \mc{1}{c}{\scriptsize{90}} & \mc{1}{c}{\scriptsize{93}} & \mc{1}{c}{\scriptsize{90}} & \mc{1}{c}{\scriptsize{93}} & \mc{1}{c}{\scriptsize{93}} & \mc{1}{c}{\scriptsize{29}} \\  

     & \mc{1}{c}{\scriptsize{\textbf{(0.000)}}} & \mc{1}{c}{\scriptsize{\textbf{(0.000)}}} & \mc{1}{c}{\scriptsize{\textbf{(0.000)}}} & \mc{1}{c}{\scriptsize{\textbf{(0.000)}}} & \mc{1}{c}{\scriptsize{\textbf{(0.000)}}} & \mc{1}{c}{\scriptsize{\textbf{(0.000)}}} & \mc{1}{c}{\scriptsize{\textbf{(0.000)}}} & \mc{1}{c}{\scriptsize{\textbf{(0.000)}}} &  \\  

    \mc{1}{l}{\scriptsize{Childhood Household Environment}} & \mc{1}{c}{\scriptsize{62}} & \mc{1}{c}{\scriptsize{69}} & \mc{1}{c}{\scriptsize{54}} & \mc{1}{c}{\scriptsize{54}} & \mc{1}{c}{\scriptsize{54}} & \mc{1}{c}{\scriptsize{85}} & \mc{1}{c}{\scriptsize{62}} & \mc{1}{c}{\scriptsize{92}} & \mc{1}{c}{\scriptsize{13}} \\  

     & \mc{1}{c}{\scriptsize{(0.129)}} & \mc{1}{c}{\scriptsize{(0.297)}} & \mc{1}{c}{\scriptsize{(0.218)}} & \mc{1}{c}{\scriptsize{(0.257)}} & \mc{1}{c}{\scriptsize{(0.149)}} & \mc{1}{c}{\scriptsize{\textbf{(0.099)}}} & \mc{1}{c}{\scriptsize{(0.307)}} & \mc{1}{c}{\scriptsize{\textbf{(0.000)}}} &  \\  

    \mc{1}{l}{\scriptsize{Mother's Employment, Education, and Income}} & \mc{1}{c}{\scriptsize{87}} & \mc{1}{c}{\scriptsize{80}} & \mc{1}{c}{\scriptsize{87}} & \mc{1}{c}{\scriptsize{87}} & \mc{1}{c}{\scriptsize{93}} & \mc{1}{c}{\scriptsize{87}} & \mc{1}{c}{\scriptsize{73}} & \mc{1}{c}{\scriptsize{87}} & \mc{1}{c}{\scriptsize{15}} \\  

     & \mc{1}{c}{\scriptsize{\textbf{(0.000)}}} & \mc{1}{c}{\scriptsize{\textbf{(0.000)}}} & \mc{1}{c}{\scriptsize{\textbf{(0.000)}}} & \mc{1}{c}{\scriptsize{\textbf{(0.000)}}} & \mc{1}{c}{\scriptsize{\textbf{(0.000)}}} & \mc{1}{c}{\scriptsize{\textbf{(0.000)}}} & \mc{1}{c}{\scriptsize{\textbf{(0.050)}}} & \mc{1}{c}{\scriptsize{\textbf{(0.000)}}} &  \\  

    \mc{1}{l}{\scriptsize{Education, Employment, Income}} & \mc{1}{c}{\scriptsize{93}} & \mc{1}{c}{\scriptsize{87}} & \mc{1}{c}{\scriptsize{87}} & \mc{1}{c}{\scriptsize{87}} & \mc{1}{c}{\scriptsize{80}} & \mc{1}{c}{\scriptsize{87}} & \mc{1}{c}{\scriptsize{87}} & \mc{1}{c}{\scriptsize{87}} & \mc{1}{c}{\scriptsize{15}} \\  

     & \mc{1}{c}{\scriptsize{\textbf{(0.000)}}} & \mc{1}{c}{\scriptsize{\textbf{(0.000)}}} & \mc{1}{c}{\scriptsize{\textbf{(0.000)}}} & \mc{1}{c}{\scriptsize{\textbf{(0.000)}}} & \mc{1}{c}{\scriptsize{\textbf{(0.000)}}} & \mc{1}{c}{\scriptsize{\textbf{(0.000)}}} & \mc{1}{c}{\scriptsize{\textbf{(0.000)}}} & \mc{1}{c}{\scriptsize{\textbf{(0.000)}}} &  \\  

    \mc{1}{l}{\scriptsize{Crime}} & \mc{1}{c}{\scriptsize{25}} & \mc{1}{c}{\scriptsize{25}} & \mc{1}{c}{\scriptsize{75}} & \mc{1}{c}{\scriptsize{25}} & \mc{1}{c}{\scriptsize{25}} & \mc{1}{c}{\scriptsize{25}} & \mc{1}{c}{\scriptsize{25}} & \mc{1}{c}{\scriptsize{25}} & \mc{1}{c}{\scriptsize{4}} \\  

     & \mc{1}{c}{\scriptsize{(0.950)}} & \mc{1}{c}{\scriptsize{(0.901)}} & \mc{1}{c}{\scriptsize{(0.386)}} & \mc{1}{c}{\scriptsize{(0.861)}} & \mc{1}{c}{\scriptsize{(0.881)}} & \mc{1}{c}{\scriptsize{(0.980)}} & \mc{1}{c}{\scriptsize{(0.772)}} & \mc{1}{c}{\scriptsize{(0.842)}} &  \\  

    \mc{1}{l}{\scriptsize{Drugs and Alcohol}} & \mc{1}{c}{\scriptsize{20}} & \mc{1}{c}{\scriptsize{20}} & \mc{1}{c}{\scriptsize{80}} & \mc{1}{c}{\scriptsize{80}} & \mc{1}{c}{\scriptsize{60}} & \mc{1}{c}{\scriptsize{20}} & \mc{1}{c}{\scriptsize{20}} & \mc{1}{c}{\scriptsize{20}} & \mc{1}{c}{\scriptsize{5}} \\  

     & \mc{1}{c}{\scriptsize{(0.990)}} & \mc{1}{c}{\scriptsize{(0.980)}} & \mc{1}{c}{\scriptsize{(0.139)}} & \mc{1}{c}{\scriptsize{(0.158)}} & \mc{1}{c}{\scriptsize{(0.337)}} & \mc{1}{c}{\scriptsize{(0.950)}} & \mc{1}{c}{\scriptsize{(0.960)}} & \mc{1}{c}{\scriptsize{(0.950)}} &  \\  

    \mc{1}{l}{\scriptsize{Adult Health}} & \mc{1}{c}{\scriptsize{63}} & \mc{1}{c}{\scriptsize{58}} & \mc{1}{c}{\scriptsize{47}} & \mc{1}{c}{\scriptsize{53}} & \mc{1}{c}{\scriptsize{47}} & \mc{1}{c}{\scriptsize{63}} & \mc{1}{c}{\scriptsize{53}} & \mc{1}{c}{\scriptsize{53}} & \mc{1}{c}{\scriptsize{19}} \\  

     & \mc{1}{c}{\scriptsize{(0.238)}} & \mc{1}{c}{\scriptsize{(0.218)}} & \mc{1}{c}{\scriptsize{(0.584)}} & \mc{1}{c}{\scriptsize{(0.505)}} & \mc{1}{c}{\scriptsize{(0.594)}} & \mc{1}{c}{\scriptsize{(0.218)}} & \mc{1}{c}{\scriptsize{(0.426)}} & \mc{1}{c}{\scriptsize{(0.327)}} &  \\  

    \mc{1}{l}{\scriptsize{Mental Health}} & \mc{1}{c}{\scriptsize{100}} & \mc{1}{c}{\scriptsize{100}} & \mc{1}{c}{\scriptsize{91}} & \mc{1}{c}{\scriptsize{100}} & \mc{1}{c}{\scriptsize{91}} & \mc{1}{c}{\scriptsize{100}} & \mc{1}{c}{\scriptsize{90}} & \mc{1}{c}{\scriptsize{100}} & \mc{1}{c}{\scriptsize{11}} \\  

     & \mc{1}{c}{\scriptsize{\textbf{(0.000)}}} & \mc{1}{c}{\scriptsize{\textbf{(0.000)}}} & \mc{1}{c}{\scriptsize{\textbf{(0.000)}}} & \mc{1}{c}{\scriptsize{\textbf{(0.000)}}} & \mc{1}{c}{\scriptsize{\textbf{(0.000)}}} & \mc{1}{c}{\scriptsize{\textbf{(0.000)}}} & \mc{1}{c}{\scriptsize{\textbf{(0.000)}}} & \mc{1}{c}{\scriptsize{\textbf{(0.000)}}} &  \\  

  \bottomrule
  \end{tabular}
\end{center}

\begin{center}
	\begin{table}[H]
\captionsetup{singlelinecheck=false,justification=centering}
\caption{ABC Percentage of Positive Treatment Effects by Category, Males and Females \\ $H_0$: $\le$ 75\% $|$ 5\% Significance \label{tab:counts_pooled}}

  \begin{threeparttable}
  \begin{tabular}{cccccccccc}
  \hline\hline

     & \scriptsize{(1)} & \scriptsize{(2)} & \scriptsize{(3)} & \scriptsize{(4)} & \scriptsize{(5)} & \scriptsize{(6)} & \scriptsize{(7)} & \scriptsize{(8)} &  \\  

     &  &  & \mc{3}{c}{\scriptsize{$P=0$}} & \mc{3}{c}{\scriptsize{$P=1$}} &  \\ 
    \cmidrule(lr){4-6} \cmidrule(lr){7-9} 

    \scriptsize{Category} & \scriptsize{ITT} & \scriptsize{ITT$|X,W$} & \scriptsize{ITT} & \scriptsize{ITT$|X,W$} & \scriptsize{KE$|X,W$} & \scriptsize{ITT} & \scriptsize{ITT$|X,W$} & \scriptsize{KE$|X,W$} & \scriptsize{Outcomes} \\ 
    \hline  

    \mc{1}{l}{\scriptsize{IQ Scores}} & \mc{1}{c}{\scriptsize{93}} & \mc{1}{c}{\scriptsize{87}} & \mc{1}{c}{\scriptsize{47}} & \mc{1}{c}{\scriptsize{53}} & \mc{1}{c}{\scriptsize{47}} & \mc{1}{c}{\scriptsize{93}} & \mc{1}{c}{\scriptsize{100}} & \mc{1}{c}{\scriptsize{87}} & \mc{1}{c}{\scriptsize{15}} \\  

     & \mc{1}{c}{\scriptsize{\textbf{(0.000)}}} & \mc{1}{c}{\scriptsize{(0.353)}} & \mc{1}{c}{\scriptsize{(0.922)}} & \mc{1}{c}{\scriptsize{(0.902)}} & \mc{1}{c}{\scriptsize{(0.941)}} & \mc{1}{c}{\scriptsize{\textbf{(0.000)}}} & \mc{1}{c}{\scriptsize{\textbf{(0.000)}}} & \mc{1}{c}{\scriptsize{(0.471)}} &  \\  

    \mc{1}{l}{\scriptsize{Achievement Scores}} & \mc{1}{c}{\scriptsize{92}} & \mc{1}{c}{\scriptsize{83}} & \mc{1}{c}{\scriptsize{25}} & \mc{1}{c}{\scriptsize{25}} & \mc{1}{c}{\scriptsize{18}} & \mc{1}{c}{\scriptsize{100}} & \mc{1}{c}{\scriptsize{100}} & \mc{1}{c}{\scriptsize{92}} & \mc{1}{c}{\scriptsize{12}} \\  

     & \mc{1}{c}{\scriptsize{(0.451)}} & \mc{1}{c}{\scriptsize{(0.471)}} & \mc{1}{c}{\scriptsize{(1.000)}} & \mc{1}{c}{\scriptsize{(1.000)}} & \mc{1}{c}{\scriptsize{(1.000)}} & \mc{1}{c}{\scriptsize{(0.235)}} & \mc{1}{c}{\scriptsize{\textbf{(0.000)}}} & \mc{1}{c}{\scriptsize{(0.314)}} &  \\  

    \mc{1}{l}{\scriptsize{HOME Scores}} & \mc{1}{c}{\scriptsize{0}} & \mc{1}{c}{\scriptsize{0}} & \mc{1}{c}{\scriptsize{71}} & \mc{1}{c}{\scriptsize{29}} & \mc{1}{c}{\scriptsize{29}} & \mc{1}{c}{\scriptsize{0}} & \mc{1}{c}{\scriptsize{0}} & \mc{1}{c}{\scriptsize{0}} & \mc{1}{c}{\scriptsize{7}} \\  

     & \mc{1}{c}{\scriptsize{(1.000)}} & \mc{1}{c}{\scriptsize{(1.000)}} & \mc{1}{c}{\scriptsize{(0.529)}} & \mc{1}{c}{\scriptsize{(1.000)}} & \mc{1}{c}{\scriptsize{(1.000)}} & \mc{1}{c}{\scriptsize{(1.000)}} & \mc{1}{c}{\scriptsize{(1.000)}} & \mc{1}{c}{\scriptsize{(1.000)}} &  \\  

    \mc{1}{l}{\scriptsize{Parent Income}} & \mc{1}{c}{\scriptsize{25}} & \mc{1}{c}{\scriptsize{25}} & \mc{1}{c}{\scriptsize{12}} & \mc{1}{c}{\scriptsize{0}} & \mc{1}{c}{\scriptsize{33}} & \mc{1}{c}{\scriptsize{25}} & \mc{1}{c}{\scriptsize{25}} & \mc{1}{c}{\scriptsize{50}} & \mc{1}{c}{\scriptsize{8}} \\  

     & \mc{1}{c}{\scriptsize{(1.000)}} & \mc{1}{c}{\scriptsize{(1.000)}} & \mc{1}{c}{\scriptsize{(1.000)}} & \mc{1}{c}{\scriptsize{(1.000)}} & \mc{1}{c}{\scriptsize{(0.961)}} & \mc{1}{c}{\scriptsize{(1.000)}} & \mc{1}{c}{\scriptsize{(1.000)}} & \mc{1}{c}{\scriptsize{(0.863)}} &  \\  

    \mc{1}{l}{\scriptsize{Mother's Employment}} & \mc{1}{c}{\scriptsize{50}} & \mc{1}{c}{\scriptsize{0}} & \mc{1}{c}{\scriptsize{83}} & \mc{1}{c}{\scriptsize{67}} & \mc{1}{c}{\scriptsize{67}} & \mc{1}{c}{\scriptsize{0}} & \mc{1}{c}{\scriptsize{0}} & \mc{1}{c}{\scriptsize{0}} & \mc{1}{c}{\scriptsize{6}} \\  

     & \mc{1}{c}{\scriptsize{(0.569)}} & \mc{1}{c}{\scriptsize{(1.000)}} & \mc{1}{c}{\scriptsize{(0.510)}} & \mc{1}{c}{\scriptsize{(0.549)}} & \mc{1}{c}{\scriptsize{(0.627)}} & \mc{1}{c}{\scriptsize{(1.000)}} & \mc{1}{c}{\scriptsize{(1.000)}} & \mc{1}{c}{\scriptsize{(1.000)}} &  \\  

    \mc{1}{l}{\scriptsize{Mother's Education}} & \mc{1}{c}{\scriptsize{0}} & \mc{1}{c}{\scriptsize{0}} & \mc{1}{c}{\scriptsize{0}} & \mc{1}{c}{\scriptsize{0}} & \mc{1}{c}{\scriptsize{0}} & \mc{1}{c}{\scriptsize{0}} & \mc{1}{c}{\scriptsize{0}} & \mc{1}{c}{\scriptsize{0}} & \mc{1}{c}{\scriptsize{6}} \\  

     & \mc{1}{c}{\scriptsize{(1.000)}} & \mc{1}{c}{\scriptsize{(1.000)}} & \mc{1}{c}{\scriptsize{(1.000)}} & \mc{1}{c}{\scriptsize{(1.000)}} & \mc{1}{c}{\scriptsize{(1.000)}} & \mc{1}{c}{\scriptsize{(1.000)}} & \mc{1}{c}{\scriptsize{(1.000)}} & \mc{1}{c}{\scriptsize{(1.000)}} &  \\  

    \mc{1}{l}{\scriptsize{Father at Home}} & \mc{1}{c}{\scriptsize{0}} & \mc{1}{c}{\scriptsize{0}} & \mc{1}{c}{\scriptsize{0}} & \mc{1}{c}{\scriptsize{0}} & \mc{1}{c}{\scriptsize{0}} & \mc{1}{c}{\scriptsize{0}} & \mc{1}{c}{\scriptsize{0}} & \mc{1}{c}{\scriptsize{0}} & \mc{1}{c}{\scriptsize{6}} \\  

     & \mc{1}{c}{\scriptsize{(1.000)}} & \mc{1}{c}{\scriptsize{(1.000)}} & \mc{1}{c}{\scriptsize{(1.000)}} & \mc{1}{c}{\scriptsize{(1.000)}} & \mc{1}{c}{\scriptsize{(1.000)}} & \mc{1}{c}{\scriptsize{(1.000)}} & \mc{1}{c}{\scriptsize{(1.000)}} & \mc{1}{c}{\scriptsize{(1.000)}} &  \\  

    \mc{1}{l}{\scriptsize{Adoption}} & \mc{1}{c}{\scriptsize{0}} & \mc{1}{c}{\scriptsize{0}} & \mc{1}{c}{\scriptsize{0}} & \mc{1}{c}{\scriptsize{0}} & \mc{1}{c}{\scriptsize{0}} & \mc{1}{c}{\scriptsize{100}} & \mc{1}{c}{\scriptsize{0}} & \mc{1}{c}{\scriptsize{0}} & \mc{1}{c}{\scriptsize{1}} \\  

     & \mc{1}{c}{\scriptsize{(1.000)}} & \mc{1}{c}{\scriptsize{(0.980)}} & \mc{1}{c}{\scriptsize{(1.000)}} & \mc{1}{c}{\scriptsize{(0.980)}} & \mc{1}{c}{\scriptsize{(0.961)}} & \mc{1}{c}{\scriptsize{(0.627)}} & \mc{1}{c}{\scriptsize{(0.902)}} & \mc{1}{c}{\scriptsize{(0.902)}} &  \\  

    \mc{1}{l}{\scriptsize{Education}} & \mc{1}{c}{\scriptsize{80}} & \mc{1}{c}{\scriptsize{60}} & \mc{1}{c}{\scriptsize{40}} & \mc{1}{c}{\scriptsize{0}} & \mc{1}{c}{\scriptsize{20}} & \mc{1}{c}{\scriptsize{80}} & \mc{1}{c}{\scriptsize{60}} & \mc{1}{c}{\scriptsize{0}} & \mc{1}{c}{\scriptsize{5}} \\  

     & \mc{1}{c}{\scriptsize{(0.549)}} & \mc{1}{c}{\scriptsize{(0.569)}} & \mc{1}{c}{\scriptsize{(0.843)}} & \mc{1}{c}{\scriptsize{(1.000)}} & \mc{1}{c}{\scriptsize{(1.000)}} & \mc{1}{c}{\scriptsize{(0.529)}} & \mc{1}{c}{\scriptsize{(0.510)}} & \mc{1}{c}{\scriptsize{(1.000)}} &  \\  

    \mc{1}{l}{\scriptsize{Employment and Income}} & \mc{1}{c}{\scriptsize{33}} & \mc{1}{c}{\scriptsize{17}} & \mc{1}{c}{\scriptsize{0}} & \mc{1}{c}{\scriptsize{0}} & \mc{1}{c}{\scriptsize{0}} & \mc{1}{c}{\scriptsize{50}} & \mc{1}{c}{\scriptsize{17}} & \mc{1}{c}{\scriptsize{33}} & \mc{1}{c}{\scriptsize{6}} \\  

     & \mc{1}{c}{\scriptsize{(1.000)}} & \mc{1}{c}{\scriptsize{(1.000)}} & \mc{1}{c}{\scriptsize{(1.000)}} & \mc{1}{c}{\scriptsize{(1.000)}} & \mc{1}{c}{\scriptsize{(1.000)}} & \mc{1}{c}{\scriptsize{(0.882)}} & \mc{1}{c}{\scriptsize{(1.000)}} & \mc{1}{c}{\scriptsize{(1.000)}} &  \\  

    \mc{1}{l}{\scriptsize{Crime}} & \mc{1}{c}{\scriptsize{0}} & \mc{1}{c}{\scriptsize{25}} & \mc{1}{c}{\scriptsize{0}} & \mc{1}{c}{\scriptsize{25}} & \mc{1}{c}{\scriptsize{0}} & \mc{1}{c}{\scriptsize{0}} & \mc{1}{c}{\scriptsize{0}} & \mc{1}{c}{\scriptsize{0}} & \mc{1}{c}{\scriptsize{4}} \\  

     & \mc{1}{c}{\scriptsize{(1.000)}} & \mc{1}{c}{\scriptsize{(1.000)}} & \mc{1}{c}{\scriptsize{(1.000)}} & \mc{1}{c}{\scriptsize{(1.000)}} & \mc{1}{c}{\scriptsize{(1.000)}} & \mc{1}{c}{\scriptsize{(1.000)}} & \mc{1}{c}{\scriptsize{(1.000)}} & \mc{1}{c}{\scriptsize{(1.000)}} &  \\  

    \mc{1}{l}{\scriptsize{Tobacco, Drugs, Alcohol}} & \mc{1}{c}{\scriptsize{0}} & \mc{1}{c}{\scriptsize{0}} & \mc{1}{c}{\scriptsize{25}} & \mc{1}{c}{\scriptsize{20}} & \mc{1}{c}{\scriptsize{20}} & \mc{1}{c}{\scriptsize{0}} & \mc{1}{c}{\scriptsize{0}} & \mc{1}{c}{\scriptsize{0}} & \mc{1}{c}{\scriptsize{5}} \\  

     & \mc{1}{c}{\scriptsize{(1.000)}} & \mc{1}{c}{\scriptsize{(1.000)}} & \mc{1}{c}{\scriptsize{(1.000)}} & \mc{1}{c}{\scriptsize{(1.000)}} & \mc{1}{c}{\scriptsize{(1.000)}} & \mc{1}{c}{\scriptsize{(1.000)}} & \mc{1}{c}{\scriptsize{(1.000)}} & \mc{1}{c}{\scriptsize{(1.000)}} &  \\  

    \mc{1}{l}{\scriptsize{Self-Reported Health}} & \mc{1}{c}{\scriptsize{0}} & \mc{1}{c}{\scriptsize{0}} & \mc{1}{c}{\scriptsize{0}} & \mc{1}{c}{\scriptsize{0}} & \mc{1}{c}{\scriptsize{0}} & \mc{1}{c}{\scriptsize{0}} & \mc{1}{c}{\scriptsize{0}} & \mc{1}{c}{\scriptsize{0}} & \mc{1}{c}{\scriptsize{3}} \\  

     & \mc{1}{c}{\scriptsize{(1.000)}} & \mc{1}{c}{\scriptsize{(1.000)}} & \mc{1}{c}{\scriptsize{(1.000)}} & \mc{1}{c}{\scriptsize{(1.000)}} & \mc{1}{c}{\scriptsize{(1.000)}} & \mc{1}{c}{\scriptsize{(1.000)}} & \mc{1}{c}{\scriptsize{(1.000)}} & \mc{1}{c}{\scriptsize{(1.000)}} &  \\  

    \mc{1}{l}{\scriptsize{Hypertension}} & \mc{1}{c}{\scriptsize{80}} & \mc{1}{c}{\scriptsize{20}} & \mc{1}{c}{\scriptsize{0}} & \mc{1}{c}{\scriptsize{0}} & \mc{1}{c}{\scriptsize{0}} & \mc{1}{c}{\scriptsize{80}} & \mc{1}{c}{\scriptsize{80}} & \mc{1}{c}{\scriptsize{80}} & \mc{1}{c}{\scriptsize{5}} \\  

     & \mc{1}{c}{\scriptsize{(0.490)}} & \mc{1}{c}{\scriptsize{(1.000)}} & \mc{1}{c}{\scriptsize{(0.941)}} & \mc{1}{c}{\scriptsize{(0.941)}} & \mc{1}{c}{\scriptsize{(0.588)}} & \mc{1}{c}{\scriptsize{(0.431)}} & \mc{1}{c}{\scriptsize{(0.412)}} & \mc{1}{c}{\scriptsize{(0.412)}} &  \\  

    \mc{1}{l}{\scriptsize{Cholesterol}} & \mc{1}{c}{\scriptsize{0}} & \mc{1}{c}{\scriptsize{33}} & \mc{1}{c}{\scriptsize{67}} & \mc{1}{c}{\scriptsize{0}} & \mc{1}{c}{\scriptsize{33}} & \mc{1}{c}{\scriptsize{0}} & \mc{1}{c}{\scriptsize{67}} & \mc{1}{c}{\scriptsize{0}} & \mc{1}{c}{\scriptsize{3}} \\  

     & \mc{1}{c}{\scriptsize{(1.000)}} & \mc{1}{c}{\scriptsize{(0.745)}} & \mc{1}{c}{\scriptsize{(0.529)}} & \mc{1}{c}{\scriptsize{(0.941)}} & \mc{1}{c}{\scriptsize{(0.412)}} & \mc{1}{c}{\scriptsize{(1.000)}} & \mc{1}{c}{\scriptsize{(0.490)}} & \mc{1}{c}{\scriptsize{(0.627)}} &  \\  

    \mc{1}{l}{\scriptsize{Diabetes}} & \mc{1}{c}{\scriptsize{0}} & \mc{1}{c}{\scriptsize{0}} & \mc{1}{c}{\scriptsize{0}} & \mc{1}{c}{\scriptsize{0}} & \mc{1}{c}{\scriptsize{0}} & \mc{1}{c}{\scriptsize{0}} & \mc{1}{c}{\scriptsize{0}} & \mc{1}{c}{\scriptsize{0}} & \mc{1}{c}{\scriptsize{4}} \\  

     & \mc{1}{c}{\scriptsize{(1.000)}} & \mc{1}{c}{\scriptsize{(1.000)}} & \mc{1}{c}{\scriptsize{(0.941)}} & \mc{1}{c}{\scriptsize{(0.941)}} & \mc{1}{c}{\scriptsize{(0.627)}} & \mc{1}{c}{\scriptsize{(1.000)}} & \mc{1}{c}{\scriptsize{(1.000)}} & \mc{1}{c}{\scriptsize{(0.686)}} &  \\  

    \mc{1}{l}{\scriptsize{Vitamin D Deficiency}} & \mc{1}{c}{\scriptsize{100}} & \mc{1}{c}{\scriptsize{0}} & \mc{1}{c}{\scriptsize{100}} & \mc{1}{c}{\scriptsize{0}} & \mc{1}{c}{\scriptsize{100}} & \mc{1}{c}{\scriptsize{0}} & \mc{1}{c}{\scriptsize{0}} & \mc{1}{c}{\scriptsize{0}} & \mc{1}{c}{\scriptsize{1}} \\  

     & \mc{1}{c}{\scriptsize{(0.510)}} & \mc{1}{c}{\scriptsize{(1.000)}} & \mc{1}{c}{\scriptsize{(0.569)}} & \mc{1}{c}{\scriptsize{(0.941)}} & \mc{1}{c}{\scriptsize{(0.314)}} & \mc{1}{c}{\scriptsize{(1.000)}} & \mc{1}{c}{\scriptsize{(1.000)}} & \mc{1}{c}{\scriptsize{(0.627)}} &  \\  

    \mc{1}{l}{\scriptsize{Obesity}} & \mc{1}{c}{\scriptsize{14}} & \mc{1}{c}{\scriptsize{0}} & \mc{1}{c}{\scriptsize{0}} & \mc{1}{c}{\scriptsize{0}} & \mc{1}{c}{\scriptsize{0}} & \mc{1}{c}{\scriptsize{14}} & \mc{1}{c}{\scriptsize{0}} & \mc{1}{c}{\scriptsize{20}} & \mc{1}{c}{\scriptsize{7}} \\  

     & \mc{1}{c}{\scriptsize{(1.000)}} & \mc{1}{c}{\scriptsize{(1.000)}} & \mc{1}{c}{\scriptsize{(0.941)}} & \mc{1}{c}{\scriptsize{(0.941)}} & \mc{1}{c}{\scriptsize{(0.902)}} & \mc{1}{c}{\scriptsize{(1.000)}} & \mc{1}{c}{\scriptsize{(1.000)}} & \mc{1}{c}{\scriptsize{(0.941)}} &  \\  

    \mc{1}{l}{\scriptsize{Mental Health}} & \mc{1}{c}{\scriptsize{45}} & \mc{1}{c}{\scriptsize{36}} & \mc{1}{c}{\scriptsize{18}} & \mc{1}{c}{\scriptsize{45}} & \mc{1}{c}{\scriptsize{0}} & \mc{1}{c}{\scriptsize{18}} & \mc{1}{c}{\scriptsize{0}} & \mc{1}{c}{\scriptsize{18}} & \mc{1}{c}{\scriptsize{11}} \\  

     & \mc{1}{c}{\scriptsize{(0.882)}} & \mc{1}{c}{\scriptsize{(1.000)}} & \mc{1}{c}{\scriptsize{(1.000)}} & \mc{1}{c}{\scriptsize{(0.824)}} & \mc{1}{c}{\scriptsize{(1.000)}} & \mc{1}{c}{\scriptsize{(1.000)}} & \mc{1}{c}{\scriptsize{(1.000)}} & \mc{1}{c}{\scriptsize{(1.000)}} &  \\  

    \mc{1}{l}{\scriptsize{Child Behavior}} & \mc{1}{c}{\scriptsize{12}} & \mc{1}{c}{\scriptsize{19}} & \mc{1}{c}{\scriptsize{0}} & \mc{1}{c}{\scriptsize{6}} & \mc{1}{c}{\scriptsize{0}} & \mc{1}{c}{\scriptsize{12}} & \mc{1}{c}{\scriptsize{19}} & \mc{1}{c}{\scriptsize{19}} & \mc{1}{c}{\scriptsize{16}} \\  

     & \mc{1}{c}{\scriptsize{(1.000)}} & \mc{1}{c}{\scriptsize{(1.000)}} & \mc{1}{c}{\scriptsize{(1.000)}} & \mc{1}{c}{\scriptsize{(1.000)}} & \mc{1}{c}{\scriptsize{(1.000)}} & \mc{1}{c}{\scriptsize{(1.000)}} & \mc{1}{c}{\scriptsize{(1.000)}} & \mc{1}{c}{\scriptsize{(1.000)}} &  \\  

  \hline\hline
  \end{tabular}
    \begin{tablenotes}
    \scriptsize
    \item 
Note: This table displays the percentage of the 131 outcomes for which we estimate positive
treatment effects. For outcomes where a negative treatment effect is beneficial to the subjects
(e.g. prevalence of diabetes), we reverse the signs of treatment effects so that all beneficial 
effects have positive signs.
Column (1) correpsonds to the ITT, without accounting for any controls.
Column (2) correpsonds to the ITT conditioning on vector of controls, $X$, consisting of the Apgar score 1 minute after birth, the HRI index, maternal IQ, an
indicator for teenage pregnancy of the mother, an indicator for the father being at 
home, and an indicator for having a grandmother residing in the same county. We also apply IPW weights, $W$, to account for attrition.
Columns (3)--(4) are analogous to columns (1)--(2), but we restrict the control sample to subjects
who did not enroll in any alternative care.
Column (5) correpsonds to the matching estimate, where we use the Mahalanobis metric and Epanechnikov kernel
to match on controls $X$ listed above, and restrict the control sample to subjects who did not enroll
in any alternative care. Additionally, we apply IPW weights, $W$.
Columns (6)--(8) are analogous to Columns (3)--(5), except we restrict the control sample to subejcts
who did enroll in alternative care. 
Numbers in parentheses represent the $p$-value from a single hypothesis test, and are obtained from 
the empirical bootstrap distribution generated by 200 resamples of the original data. 
Bold $p$-values indicate significance at the 10\% level. Blank point estimates indicate that
we are unable to obtain estimates due to a lack of support in the data. 

    \end{tablenotes}
  \end{threeparttable}

\end{table}
\end{center}

\begin{center}
	\input{abc/rslt_pooled_counts_n75a10}
\end{center}

\begin{center}
	\input{abc/rslt_pooled_counts_n75a100}
\end{center}

\begin{center}
	\input{abc/rslt_male_counts_n25a5}
\end{center}

\begin{center}
	\input{abc/rslt_male_counts_n25a10}
\end{center}

\begin{center}
	\input{abc/rslt_male_counts_n25a100}
\end{center}

\begin{center}
	\input{abc/rslt_male_counts_n50a5}
\end{center}

\begin{center}
	\input{abc/rslt_male_counts_n50a10}
\end{center}

\begin{center}
	  \begin{tabular}{cccccccccc}
  \toprule

    \scriptsize{Category} & \scriptsize{(1)} & \scriptsize{(2)} & \scriptsize{(3)} & \scriptsize{(4)} & \scriptsize{(5)} & \scriptsize{(6)} & \scriptsize{(7)} & \scriptsize{(8)} & \scriptsize{N} \\ 
    \midrule  

    \mc{1}{l}{\scriptsize{Cognitive Skills}} & \mc{1}{c}{\scriptsize{93}} & \mc{1}{c}{\scriptsize{81}} & \mc{1}{c}{\scriptsize{70}} & \mc{1}{c}{\scriptsize{78}} & \mc{1}{c}{\scriptsize{63}} & \mc{1}{c}{\scriptsize{93}} & \mc{1}{c}{\scriptsize{85}} & \mc{1}{c}{\scriptsize{81}} & \mc{1}{c}{\scriptsize{27}} \\  

     & \mc{1}{c}{\scriptsize{\textbf{(0.000)}}} & \mc{1}{c}{\scriptsize{\textbf{(0.000)}}} & \mc{1}{c}{\scriptsize{(0.145)}} & \mc{1}{c}{\scriptsize{\textbf{(0.001)}}} & \mc{1}{c}{\scriptsize{(0.259)}} & \mc{1}{c}{\scriptsize{\textbf{(0.000)}}} & \mc{1}{c}{\scriptsize{\textbf{(0.000)}}} & \mc{1}{c}{\scriptsize{\textbf{(0.000)}}} &  \\  

    \mc{1}{l}{\scriptsize{Childhood Household Environment}} & \mc{1}{c}{\scriptsize{54}} & \mc{1}{c}{\scriptsize{62}} & \mc{1}{c}{\scriptsize{46}} & \mc{1}{c}{\scriptsize{46}} & \mc{1}{c}{\scriptsize{46}} & \mc{1}{c}{\scriptsize{75}} & \mc{1}{c}{\scriptsize{77}} & \mc{1}{c}{\scriptsize{85}} & \mc{1}{c}{\scriptsize{13}} \\  

     & \mc{1}{c}{\scriptsize{(0.384)}} & \mc{1}{c}{\scriptsize{(0.325)}} & \mc{1}{c}{\scriptsize{(0.652)}} & \mc{1}{c}{\scriptsize{(0.531)}} & \mc{1}{c}{\scriptsize{(0.627)}} & \mc{1}{c}{\scriptsize{(0.147)}} & \mc{1}{c}{\scriptsize{(0.110)}} & \mc{1}{c}{\scriptsize{\textbf{(0.000)}}} &  \\  

    \mc{1}{l}{\scriptsize{Mother's Employment, Education, and Income}} & \mc{1}{c}{\scriptsize{80}} & \mc{1}{c}{\scriptsize{60}} & \mc{1}{c}{\scriptsize{73}} & \mc{1}{c}{\scriptsize{60}} & \mc{1}{c}{\scriptsize{60}} & \mc{1}{c}{\scriptsize{60}} & \mc{1}{c}{\scriptsize{60}} & \mc{1}{c}{\scriptsize{67}} & \mc{1}{c}{\scriptsize{15}} \\  

     & \mc{1}{c}{\scriptsize{\textbf{(0.000)}}} & \mc{1}{c}{\scriptsize{(0.270)}} & \mc{1}{c}{\scriptsize{\textbf{(0.032)}}} & \mc{1}{c}{\scriptsize{(0.280)}} & \mc{1}{c}{\scriptsize{(0.311)}} & \mc{1}{c}{\scriptsize{(0.386)}} & \mc{1}{c}{\scriptsize{(0.321)}} & \mc{1}{c}{\scriptsize{(0.249)}} &  \\  

    \mc{1}{l}{\scriptsize{Education, Employment, Income}} & \mc{1}{c}{\scriptsize{80}} & \mc{1}{c}{\scriptsize{80}} & \mc{1}{c}{\scriptsize{53}} & \mc{1}{c}{\scriptsize{67}} & \mc{1}{c}{\scriptsize{60}} & \mc{1}{c}{\scriptsize{87}} & \mc{1}{c}{\scriptsize{87}} & \mc{1}{c}{\scriptsize{80}} & \mc{1}{c}{\scriptsize{15}} \\  

     & \mc{1}{c}{\scriptsize{\textbf{(0.000)}}} & \mc{1}{c}{\scriptsize{\textbf{(0.000)}}} & \mc{1}{c}{\scriptsize{(0.428)}} & \mc{1}{c}{\scriptsize{\textbf{(0.089)}}} & \mc{1}{c}{\scriptsize{(0.304)}} & \mc{1}{c}{\scriptsize{\textbf{(0.000)}}} & \mc{1}{c}{\scriptsize{\textbf{(0.000)}}} & \mc{1}{c}{\scriptsize{\textbf{(0.000)}}} &  \\  

    \mc{1}{l}{\scriptsize{Crime}} & \mc{1}{c}{\scriptsize{25}} & \mc{1}{c}{\scriptsize{25}} & \mc{1}{c}{\scriptsize{25}} & \mc{1}{c}{\scriptsize{25}} & \mc{1}{c}{\scriptsize{25}} & \mc{1}{c}{\scriptsize{25}} & \mc{1}{c}{\scriptsize{25}} & \mc{1}{c}{\scriptsize{25}} & \mc{1}{c}{\scriptsize{4}} \\  

     & \mc{1}{c}{\scriptsize{(0.895)}} & \mc{1}{c}{\scriptsize{(0.730)}} & \mc{1}{c}{\scriptsize{(1.000)}} & \mc{1}{c}{\scriptsize{(1.000)}} & \mc{1}{c}{\scriptsize{(1.000)}} & \mc{1}{c}{\scriptsize{(0.924)}} & \mc{1}{c}{\scriptsize{(0.766)}} & \mc{1}{c}{\scriptsize{(0.879)}} &  \\  

    \mc{1}{l}{\scriptsize{Drugs and Alcohol}} & \mc{1}{c}{\scriptsize{20}} & \mc{1}{c}{\scriptsize{20}} & \mc{1}{c}{\scriptsize{40}} & \mc{1}{c}{\scriptsize{60}} & \mc{1}{c}{\scriptsize{20}} & \mc{1}{c}{\scriptsize{20}} & \mc{1}{c}{\scriptsize{20}} & \mc{1}{c}{\scriptsize{20}} & \mc{1}{c}{\scriptsize{5}} \\  

     & \mc{1}{c}{\scriptsize{(0.986)}} & \mc{1}{c}{\scriptsize{(0.986)}} & \mc{1}{c}{\scriptsize{(0.695)}} & \mc{1}{c}{\scriptsize{(0.305)}} & \mc{1}{c}{\scriptsize{(0.943)}} & \mc{1}{c}{\scriptsize{(0.950)}} & \mc{1}{c}{\scriptsize{(0.986)}} & \mc{1}{c}{\scriptsize{(0.972)}} &  \\  

    \mc{1}{l}{\scriptsize{Adult Health}} & \mc{1}{c}{\scriptsize{58}} & \mc{1}{c}{\scriptsize{74}} & \mc{1}{c}{\scriptsize{37}} & \mc{1}{c}{\scriptsize{37}} & \mc{1}{c}{\scriptsize{32}} & \mc{1}{c}{\scriptsize{68}} & \mc{1}{c}{\scriptsize{74}} & \mc{1}{c}{\scriptsize{74}} & \mc{1}{c}{\scriptsize{19}} \\  

     & \mc{1}{c}{\scriptsize{(0.342)}} & \mc{1}{c}{\scriptsize{\textbf{(0.037)}}} & \mc{1}{c}{\scriptsize{(0.719)}} & \mc{1}{c}{\scriptsize{(0.752)}} & \mc{1}{c}{\scriptsize{(0.739)}} & \mc{1}{c}{\scriptsize{\textbf{(0.086)}}} & \mc{1}{c}{\scriptsize{\textbf{(0.044)}}} & \mc{1}{c}{\scriptsize{\textbf{(0.016)}}} &  \\  

    \mc{1}{l}{\scriptsize{Mental Health}} & \mc{1}{c}{\scriptsize{82}} & \mc{1}{c}{\scriptsize{82}} & \mc{1}{c}{\scriptsize{36}} & \mc{1}{c}{\scriptsize{18}} & \mc{1}{c}{\scriptsize{36}} & \mc{1}{c}{\scriptsize{91}} & \mc{1}{c}{\scriptsize{82}} & \mc{1}{c}{\scriptsize{91}} & \mc{1}{c}{\scriptsize{11}} \\  

     & \mc{1}{c}{\scriptsize{(0.133)}} & \mc{1}{c}{\scriptsize{\textbf{(0.000)}}} & \mc{1}{c}{\scriptsize{(0.749)}} & \mc{1}{c}{\scriptsize{(0.929)}} & \mc{1}{c}{\scriptsize{(0.767)}} & \mc{1}{c}{\scriptsize{\textbf{(0.000)}}} & \mc{1}{c}{\scriptsize{\textbf{(0.000)}}} & \mc{1}{c}{\scriptsize{\textbf{(0.000)}}} &  \\  

  \bottomrule
  \end{tabular}
\end{center}

\begin{center}
	\input{abc/rslt_male_counts_n75a5}
\end{center}

\begin{center}
	\input{abc/rslt_male_counts_n75a10}
\end{center}

\begin{center}
	\input{abc/rslt_male_counts_n75a100}
\end{center}

\begin{center}
	\input{abc/rslt_female_counts_n25a5}
\end{center}

\begin{center}
	\input{abc/rslt_female_counts_n25a10}
\end{center}

\begin{center}
	\input{abc/rslt_female_counts_n25a100}
\end{center}

\begin{center}
	\input{abc/rslt_female_counts_n50a5}
\end{center}

\begin{center}
	\input{abc/rslt_female_counts_n50a10}
\end{center}

\begin{center}
	  \begin{tabular}{cccccccccc}
  \toprule

    \scriptsize{Category} & \scriptsize{(1)} & \scriptsize{(2)} & \scriptsize{(3)} & \scriptsize{(4)} & \scriptsize{(5)} & \scriptsize{(6)} & \scriptsize{(7)} & \scriptsize{(8)} & \scriptsize{N} \\ 
    \midrule  

    \mc{1}{l}{\scriptsize{Cognitive Skills}} & \mc{1}{c}{\scriptsize{93}} & \mc{1}{c}{\scriptsize{93}} & \mc{1}{c}{\scriptsize{93}} & \mc{1}{c}{\scriptsize{93}} & \mc{1}{c}{\scriptsize{93}} & \mc{1}{c}{\scriptsize{93}} & \mc{1}{c}{\scriptsize{86}} & \mc{1}{c}{\scriptsize{93}} & \mc{1}{c}{\scriptsize{29}} \\  

     & \mc{1}{c}{\scriptsize{\textbf{(0.000)}}} & \mc{1}{c}{\scriptsize{\textbf{(0.000)}}} & \mc{1}{c}{\scriptsize{\textbf{(0.000)}}} & \mc{1}{c}{\scriptsize{\textbf{(0.000)}}} & \mc{1}{c}{\scriptsize{\textbf{(0.000)}}} & \mc{1}{c}{\scriptsize{\textbf{(0.000)}}} & \mc{1}{c}{\scriptsize{\textbf{(0.000)}}} & \mc{1}{c}{\scriptsize{\textbf{(0.000)}}} &  \\  

    \mc{1}{l}{\scriptsize{Childhood Household Environment}} & \mc{1}{c}{\scriptsize{62}} & \mc{1}{c}{\scriptsize{77}} & \mc{1}{c}{\scriptsize{54}} & \mc{1}{c}{\scriptsize{69}} & \mc{1}{c}{\scriptsize{54}} & \mc{1}{c}{\scriptsize{62}} & \mc{1}{c}{\scriptsize{54}} & \mc{1}{c}{\scriptsize{77}} & \mc{1}{c}{\scriptsize{13}} \\  

     & \mc{1}{c}{\scriptsize{(0.208)}} & \mc{1}{c}{\scriptsize{\textbf{(0.099)}}} & \mc{1}{c}{\scriptsize{(0.129)}} & \mc{1}{c}{\scriptsize{(0.129)}} & \mc{1}{c}{\scriptsize{(0.297)}} & \mc{1}{c}{\scriptsize{(0.396)}} & \mc{1}{c}{\scriptsize{(0.386)}} & \mc{1}{c}{\scriptsize{\textbf{(0.089)}}} &  \\  

    \mc{1}{l}{\scriptsize{Mother's Employment, Education, and Income}} & \mc{1}{c}{\scriptsize{87}} & \mc{1}{c}{\scriptsize{93}} & \mc{1}{c}{\scriptsize{87}} & \mc{1}{c}{\scriptsize{93}} & \mc{1}{c}{\scriptsize{93}} & \mc{1}{c}{\scriptsize{80}} & \mc{1}{c}{\scriptsize{86}} & \mc{1}{c}{\scriptsize{80}} & \mc{1}{c}{\scriptsize{15}} \\  

     & \mc{1}{c}{\scriptsize{\textbf{(0.000)}}} & \mc{1}{c}{\scriptsize{\textbf{(0.000)}}} & \mc{1}{c}{\scriptsize{\textbf{(0.000)}}} & \mc{1}{c}{\scriptsize{\textbf{(0.000)}}} & \mc{1}{c}{\scriptsize{\textbf{(0.000)}}} & \mc{1}{c}{\scriptsize{\textbf{(0.000)}}} & \mc{1}{c}{\scriptsize{\textbf{(0.000)}}} & \mc{1}{c}{\scriptsize{\textbf{(0.000)}}} &  \\  

    \mc{1}{l}{\scriptsize{Education, Employment, Income}} & \mc{1}{c}{\scriptsize{87}} & \mc{1}{c}{\scriptsize{79}} & \mc{1}{c}{\scriptsize{80}} & \mc{1}{c}{\scriptsize{80}} & \mc{1}{c}{\scriptsize{80}} & \mc{1}{c}{\scriptsize{80}} & \mc{1}{c}{\scriptsize{53}} & \mc{1}{c}{\scriptsize{80}} & \mc{1}{c}{\scriptsize{15}} \\  

     & \mc{1}{c}{\scriptsize{\textbf{(0.000)}}} & \mc{1}{c}{\scriptsize{\textbf{(0.000)}}} & \mc{1}{c}{\scriptsize{\textbf{(0.000)}}} & \mc{1}{c}{\scriptsize{\textbf{(0.000)}}} & \mc{1}{c}{\scriptsize{\textbf{(0.000)}}} & \mc{1}{c}{\scriptsize{\textbf{(0.000)}}} & \mc{1}{c}{\scriptsize{(0.505)}} & \mc{1}{c}{\scriptsize{\textbf{(0.000)}}} &  \\  

    \mc{1}{l}{\scriptsize{Crime}} & \mc{1}{c}{\scriptsize{100}} & \mc{1}{c}{\scriptsize{100}} & \mc{1}{c}{\scriptsize{100}} & \mc{1}{c}{\scriptsize{100}} & \mc{1}{c}{\scriptsize{100}} & \mc{1}{c}{\scriptsize{100}} & \mc{1}{c}{\scriptsize{100}} & \mc{1}{c}{\scriptsize{75}} & \mc{1}{c}{\scriptsize{4}} \\  

     & \mc{1}{c}{\scriptsize{\textbf{(0.000)}}} & \mc{1}{c}{\scriptsize{\textbf{(0.000)}}} & \mc{1}{c}{\scriptsize{\textbf{(0.000)}}} & \mc{1}{c}{\scriptsize{\textbf{(0.000)}}} & \mc{1}{c}{\scriptsize{\textbf{(0.000)}}} & \mc{1}{c}{\scriptsize{\textbf{(0.000)}}} & \mc{1}{c}{\scriptsize{\textbf{(0.000)}}} & \mc{1}{c}{\scriptsize{(0.455)}} &  \\  

    \mc{1}{l}{\scriptsize{Drugs and Alcohol}} & \mc{1}{c}{\scriptsize{80}} & \mc{1}{c}{\scriptsize{20}} & \mc{1}{c}{\scriptsize{80}} & \mc{1}{c}{\scriptsize{60}} & \mc{1}{c}{\scriptsize{80}} & \mc{1}{c}{\scriptsize{100}} & \mc{1}{c}{\scriptsize{0}} & \mc{1}{c}{\scriptsize{60}} & \mc{1}{c}{\scriptsize{5}} \\  

     & \mc{1}{c}{\scriptsize{(0.238)}} & \mc{1}{c}{\scriptsize{(0.842)}} & \mc{1}{c}{\scriptsize{\textbf{(0.020)}}} & \mc{1}{c}{\scriptsize{(0.337)}} & \mc{1}{c}{\scriptsize{\textbf{(0.010)}}} & \mc{1}{c}{\scriptsize{\textbf{(0.000)}}} & \mc{1}{c}{\scriptsize{(1.000)}} & \mc{1}{c}{\scriptsize{(0.475)}} &  \\  

    \mc{1}{l}{\scriptsize{Adult Health}} & \mc{1}{c}{\scriptsize{74}} & \mc{1}{c}{\scriptsize{44}} & \mc{1}{c}{\scriptsize{50}} & \mc{1}{c}{\scriptsize{41}} & \mc{1}{c}{\scriptsize{56}} & \mc{1}{c}{\scriptsize{74}} & \mc{1}{c}{\scriptsize{44}} & \mc{1}{c}{\scriptsize{63}} & \mc{1}{c}{\scriptsize{19}} \\  

     & \mc{1}{c}{\scriptsize{\textbf{(0.040)}}} & \mc{1}{c}{\scriptsize{(0.614)}} & \mc{1}{c}{\scriptsize{(0.455)}} & \mc{1}{c}{\scriptsize{(0.703)}} & \mc{1}{c}{\scriptsize{(0.317)}} & \mc{1}{c}{\scriptsize{\textbf{(0.079)}}} & \mc{1}{c}{\scriptsize{(0.673)}} & \mc{1}{c}{\scriptsize{(0.228)}} &  \\  

    \mc{1}{l}{\scriptsize{Mental Health}} & \mc{1}{c}{\scriptsize{82}} & \mc{1}{c}{\scriptsize{82}} & \mc{1}{c}{\scriptsize{91}} & \mc{1}{c}{\scriptsize{100}} & \mc{1}{c}{\scriptsize{82}} & \mc{1}{c}{\scriptsize{82}} & \mc{1}{c}{\scriptsize{82}} & \mc{1}{c}{\scriptsize{82}} & \mc{1}{c}{\scriptsize{11}} \\  

     & \mc{1}{c}{\scriptsize{\textbf{(0.000)}}} & \mc{1}{c}{\scriptsize{\textbf{(0.000)}}} & \mc{1}{c}{\scriptsize{\textbf{(0.000)}}} & \mc{1}{c}{\scriptsize{\textbf{(0.000)}}} & \mc{1}{c}{\scriptsize{\textbf{(0.000)}}} & \mc{1}{c}{\scriptsize{\textbf{(0.000)}}} & \mc{1}{c}{\scriptsize{\textbf{(0.000)}}} & \mc{1}{c}{\scriptsize{\textbf{(0.000)}}} &  \\  

  \bottomrule
  \end{tabular}
\end{center}

\begin{center}
	\input{abc/rslt_female_counts_n75a5}
\end{center}

\begin{center}
	\input{abc/rslt_female_counts_n75a10}
\end{center}

\begin{center}
	\input{abc/rslt_female_counts_n75a100}
\end{center}
\section{{Main Results}}


\begin{center}
	  \begin{tabular}{cccccccccc}
  \toprule

    \scriptsize{Variable} & \scriptsize{Age} & \scriptsize{(1)} & \scriptsize{(2)} & \scriptsize{(3)} & \scriptsize{(4)} & \scriptsize{(5)} & \scriptsize{(6)} & \scriptsize{(7)} & \scriptsize{(8)} \\ 
    \midrule  

    \mc{1}{l}{\scriptsize{Std. IQ Test}} & \mc{1}{c}{\scriptsize{12}} & \mc{1}{c}{\scriptsize{6.357}} & \mc{1}{c}{\scriptsize{6.148}} & \mc{1}{c}{\scriptsize{9.071}} & \mc{1}{c}{\scriptsize{11.216}} & \mc{1}{c}{\scriptsize{5.635}} & \mc{1}{c}{\scriptsize{5.206}} & \mc{1}{c}{\scriptsize{4.564}} & \mc{1}{c}{\scriptsize{2.684}} \\  

     &  & \mc{1}{c}{\scriptsize{\textbf{(0.026)}}} & \mc{1}{c}{\scriptsize{(0.105)}} & \mc{1}{c}{\scriptsize{\textbf{(0.000)}}} & \mc{1}{c}{\scriptsize{(0.105)}} & \mc{1}{c}{\scriptsize{(0.197)}} & \mc{1}{c}{\scriptsize{\textbf{(0.066)}}} & \mc{1}{c}{\scriptsize{(0.171)}} & \mc{1}{c}{\scriptsize{(0.263)}} \\  

    \mc{1}{l}{\scriptsize{Std. Achv.  Test}} & \mc{1}{c}{\scriptsize{12}} & \mc{1}{c}{\scriptsize{4.163}} & \mc{1}{c}{\scriptsize{7.051}} & \mc{1}{c}{\scriptsize{9.471}} & \mc{1}{c}{\scriptsize{11.423}} & \mc{1}{c}{\scriptsize{9.815}} & \mc{1}{c}{\scriptsize{1.912}} & \mc{1}{c}{\scriptsize{4.256}} & \mc{1}{c}{\scriptsize{4.345}} \\  

     &  & \mc{1}{c}{\scriptsize{\textbf{(0.039)}}} & \mc{1}{c}{\scriptsize{\textbf{(0.000)}}} & \mc{1}{c}{\scriptsize{\textbf{(0.000)}}} & \mc{1}{c}{\scriptsize{\textbf{(0.079)}}} & \mc{1}{c}{\scriptsize{\textbf{(0.026)}}} & \mc{1}{c}{\scriptsize{(0.211)}} & \mc{1}{c}{\scriptsize{\textbf{(0.079)}}} & \mc{1}{c}{\scriptsize{\textbf{(0.053)}}} \\  

    \mc{1}{l}{\scriptsize{Graduated High School}} & \mc{1}{c}{\scriptsize{30}} & \mc{1}{c}{\scriptsize{0.137}} & \mc{1}{c}{\scriptsize{0.126}} & \mc{1}{c}{\scriptsize{0.226}} & \mc{1}{c}{\scriptsize{0.136}} & \mc{1}{c}{\scriptsize{0.161}} & \mc{1}{c}{\scriptsize{0.092}} & \mc{1}{c}{\scriptsize{0.073}} & \mc{1}{c}{\scriptsize{0.053}} \\  

     &  & \mc{1}{c}{\scriptsize{(0.197)}} & \mc{1}{c}{\scriptsize{(0.224)}} & \mc{1}{c}{\scriptsize{(0.132)}} & \mc{1}{c}{\scriptsize{(0.342)}} & \mc{1}{c}{\scriptsize{(0.250)}} & \mc{1}{c}{\scriptsize{(0.303)}} & \mc{1}{c}{\scriptsize{(0.395)}} & \mc{1}{c}{\scriptsize{(0.342)}} \\  

    \mc{1}{l}{\scriptsize{Years of Edu.}} & \mc{1}{c}{\scriptsize{30}} & \mc{1}{c}{\scriptsize{0.400}} & \mc{1}{c}{\scriptsize{1.214}} & \mc{1}{c}{\scriptsize{1.000}} & \mc{1}{c}{\scriptsize{1.549}} & \mc{1}{c}{\scriptsize{1.687}} & \mc{1}{c}{\scriptsize{0.100}} & \mc{1}{c}{\scriptsize{1.015}} & \mc{1}{c}{\scriptsize{1.068}} \\  

     &  & \mc{1}{c}{\scriptsize{(0.263)}} & \mc{1}{c}{\scriptsize{\textbf{(0.079)}}} & \mc{1}{c}{\scriptsize{(0.211)}} & \mc{1}{c}{\scriptsize{(0.158)}} & \mc{1}{c}{\scriptsize{\textbf{(0.092)}}} & \mc{1}{c}{\scriptsize{(0.461)}} & \mc{1}{c}{\scriptsize{(0.184)}} & \mc{1}{c}{\scriptsize{\textbf{(0.039)}}} \\  

    \mc{1}{l}{\scriptsize{Labor Income}} & \mc{1}{c}{\scriptsize{30}} & \mc{1}{c}{\scriptsize{9,156}} & \mc{1}{c}{\scriptsize{22,349}} & \mc{1}{c}{\scriptsize{9,629}} & \mc{1}{c}{\scriptsize{23,107}} & \mc{1}{c}{\scriptsize{19,530}} & \mc{1}{c}{\scriptsize{8,919}} & \mc{1}{c}{\scriptsize{26,671}} & \mc{1}{c}{\scriptsize{30,036}} \\  

     &  & \mc{1}{c}{\scriptsize{(0.145)}} & \mc{1}{c}{\scriptsize{(0.132)}} & \mc{1}{c}{\scriptsize{(0.118)}} & \mc{1}{c}{\scriptsize{(0.184)}} & \mc{1}{c}{\scriptsize{(0.171)}} & \mc{1}{c}{\scriptsize{(0.158)}} & \mc{1}{c}{\scriptsize{(0.132)}} & \mc{1}{c}{\scriptsize{(0.145)}} \\  

    \mc{1}{l}{\scriptsize{Public-Transfer Income}} & \mc{1}{c}{\scriptsize{30}} & \mc{1}{c}{\scriptsize{-836}} & \mc{1}{c}{\scriptsize{-870}} & \mc{1}{c}{\scriptsize{-1,191}} & \mc{1}{c}{\scriptsize{-1,462}} & \mc{1}{c}{\scriptsize{-1,091}} & \mc{1}{c}{\scriptsize{-659}} & \mc{1}{c}{\scriptsize{-213}} & \mc{1}{c}{\scriptsize{-845}} \\  

     &  & \mc{1}{c}{\scriptsize{\textbf{(0.000)}}} & \mc{1}{c}{\scriptsize{\textbf{(0.053)}}} & \mc{1}{c}{\scriptsize{\textbf{(0.026)}}} & \mc{1}{c}{\scriptsize{(0.158)}} & \mc{1}{c}{\scriptsize{\textbf{(0.092)}}} & \mc{1}{c}{\scriptsize{\textbf{(0.053)}}} & \mc{1}{c}{\scriptsize{(0.276)}} & \mc{1}{c}{\scriptsize{\textbf{(0.026)}}} \\  

    \mc{1}{l}{\scriptsize{Employed}} & \mc{1}{c}{\scriptsize{30}} & \mc{1}{c}{\scriptsize{0.026}} & \mc{1}{c}{\scriptsize{0.177}} & \mc{1}{c}{\scriptsize{0.092}} & \mc{1}{c}{\scriptsize{0.151}} & \mc{1}{c}{\scriptsize{0.234}} & \mc{1}{c}{\scriptsize{-0.008}} & \mc{1}{c}{\scriptsize{0.192}} & \mc{1}{c}{\scriptsize{0.211}} \\  

     &  & \mc{1}{c}{\scriptsize{(0.434)}} & \mc{1}{c}{\scriptsize{(0.145)}} & \mc{1}{c}{\scriptsize{(0.316)}} & \mc{1}{c}{\scriptsize{(0.289)}} & \mc{1}{c}{\scriptsize{(0.105)}} & \mc{1}{c}{\scriptsize{(0.592)}} & \mc{1}{c}{\scriptsize{(0.145)}} & \mc{1}{c}{\scriptsize{\textbf{(0.039)}}} \\  

    \mc{1}{l}{\scriptsize{Total Felony Arrests}} & \mc{1}{c}{\scriptsize{Mid-30s}} & \mc{1}{c}{\scriptsize{1.068}} & \mc{1}{c}{\scriptsize{0.574}} & \mc{1}{c}{\scriptsize{1.014}} & \mc{1}{c}{\scriptsize{-0.166}} & \mc{1}{c}{\scriptsize{1.227}} & \mc{1}{c}{\scriptsize{1.096}} & \mc{1}{c}{\scriptsize{0.935}} & \mc{1}{c}{\scriptsize{1.305}} \\  

     &  & \mc{1}{c}{\scriptsize{(0.816)}} & \mc{1}{c}{\scriptsize{(0.645)}} & \mc{1}{c}{\scriptsize{(0.829)}} & \mc{1}{c}{\scriptsize{(0.434)}} & \mc{1}{c}{\scriptsize{(0.803)}} & \mc{1}{c}{\scriptsize{(0.829)}} & \mc{1}{c}{\scriptsize{(0.724)}} & \mc{1}{c}{\scriptsize{(0.737)}} \\  

    \mc{1}{l}{\scriptsize{Total Misdemeanor Arrests}} & \mc{1}{c}{\scriptsize{Mid-30s}} & \mc{1}{c}{\scriptsize{-0.939}} & \mc{1}{c}{\scriptsize{-0.767}} & \mc{1}{c}{\scriptsize{-1.071}} & \mc{1}{c}{\scriptsize{-1.179}} & \mc{1}{c}{\scriptsize{-0.950}} & \mc{1}{c}{\scriptsize{-0.870}} & \mc{1}{c}{\scriptsize{-0.240}} & \mc{1}{c}{\scriptsize{-0.135}} \\  

     &  & \mc{1}{c}{\scriptsize{\textbf{(0.039)}}} & \mc{1}{c}{\scriptsize{(0.224)}} & \mc{1}{c}{\scriptsize{(0.105)}} & \mc{1}{c}{\scriptsize{(0.289)}} & \mc{1}{c}{\scriptsize{(0.184)}} & \mc{1}{c}{\scriptsize{\textbf{(0.053)}}} & \mc{1}{c}{\scriptsize{(0.342)}} & \mc{1}{c}{\scriptsize{(0.421)}} \\  

    \mc{1}{l}{\scriptsize{Total Years Incarcerated}} & \mc{1}{c}{\scriptsize{30}} & \mc{1}{c}{\scriptsize{0.077}} & \mc{1}{c}{\scriptsize{0.314}} & \mc{1}{c}{\scriptsize{0.312}} & \mc{1}{c}{\scriptsize{0.485}} & \mc{1}{c}{\scriptsize{0.323}} & \mc{1}{c}{\scriptsize{-0.047}} & \mc{1}{c}{\scriptsize{0.323}} & \mc{1}{c}{\scriptsize{0.005}} \\  

     &  & \mc{1}{c}{\scriptsize{(0.553)}} & \mc{1}{c}{\scriptsize{(0.789)}} & \mc{1}{c}{\scriptsize{(0.868)}} & \mc{1}{c}{\scriptsize{(0.724)}} & \mc{1}{c}{\scriptsize{(0.711)}} & \mc{1}{c}{\scriptsize{(0.408)}} & \mc{1}{c}{\scriptsize{(0.763)}} & \mc{1}{c}{\scriptsize{(0.474)}} \\  

  \bottomrule
  \end{tabular}
\end{center}

\begin{center}
	\begin{table}[H]
\captionsetup{singlelinecheck=false,justification=centering}
\caption{ABC/CARE Average Treatment Effects, Males and Females \\ Health Biomarkers \label{tab:ate_pooled_main2}}

  \begin{threeparttable}
  \begin{tabular}{cccccccccc}
  \hline\hline

     &  & \scriptsize{(1)} & \scriptsize{(2)} & \scriptsize{(3)} & \scriptsize{(4)} & \scriptsize{(5)} & \scriptsize{(6)} & \scriptsize{(7)} & \scriptsize{(8)} \\  

     &  &  &  & \mc{3}{c}{\scriptsize{$P=0$}} & \mc{3}{c}{\scriptsize{$P=1$}} \\ 
    \cmidrule(lr){5-7} \cmidrule(lr){8-10} 

    \scriptsize{Variable} & \scriptsize{Age} & \scriptsize{ITT} & \scriptsize{ITT$|X,W$} & \scriptsize{ITT} & \scriptsize{ITT$|X,W$} & \scriptsize{KE$|X,W$} & \scriptsize{ITT} & \scriptsize{ITT$|X,W$} & \scriptsize{KE$|X,W$} \\ 
    \hline  

    \mc{1}{l}{\scriptsize{Systolic Blood Pressure (mm Hg)}} & \mc{1}{c}{\scriptsize{Mid-30s}} & \mc{1}{c}{\scriptsize{-3.100}} & \mc{1}{c}{\scriptsize{-4.274}} & \mc{1}{c}{\scriptsize{5.042}} & \mc{1}{c}{\scriptsize{9.030}} & \mc{1}{c}{\scriptsize{3.247}} & \mc{1}{c}{\scriptsize{-6.674}} & \mc{1}{c}{\scriptsize{-7.279}} & \mc{1}{c}{\scriptsize{-6.527}} \\  

     &  & \mc{1}{c}{\scriptsize{(0.176)}} & \mc{1}{c}{\scriptsize{(0.157)}} & \mc{1}{c}{\scriptsize{(0.941)}} & \mc{1}{c}{\scriptsize{(0.980)}} & \mc{1}{c}{\scriptsize{(0.529)}} & \mc{1}{c}{\scriptsize{\textbf{(0.059)}}} & \mc{1}{c}{\scriptsize{(0.118)}} & \mc{1}{c}{\scriptsize{\textbf{(0.059)}}} \\  

    \mc{1}{l}{\scriptsize{Diastolic Blood Pressure (mm Hg)}} & \mc{1}{c}{\scriptsize{Mid-30s}} & \mc{1}{c}{\scriptsize{-3.719}} & \mc{1}{c}{\scriptsize{-3.981}} & \mc{1}{c}{\scriptsize{0.076}} & \mc{1}{c}{\scriptsize{3.737}} & \mc{1}{c}{\scriptsize{-1.542}} & \mc{1}{c}{\scriptsize{-5.386}} & \mc{1}{c}{\scriptsize{-5.575}} & \mc{1}{c}{\scriptsize{-4.309}} \\  

     &  & \mc{1}{c}{\scriptsize{(0.137)}} & \mc{1}{c}{\scriptsize{(0.157)}} & \mc{1}{c}{\scriptsize{(0.529)}} & \mc{1}{c}{\scriptsize{(0.784)}} & \mc{1}{c}{\scriptsize{(0.235)}} & \mc{1}{c}{\scriptsize{(0.118)}} & \mc{1}{c}{\scriptsize{(0.118)}} & \mc{1}{c}{\scriptsize{(0.118)}} \\  

    \mc{1}{l}{\scriptsize{Prehypertension}} & \mc{1}{c}{\scriptsize{Mid-30s}} & \mc{1}{c}{\scriptsize{-0.135}} & \mc{1}{c}{\scriptsize{-0.136}} & \mc{1}{c}{\scriptsize{0.007}} & \mc{1}{c}{\scriptsize{0.151}} & \mc{1}{c}{\scriptsize{-0.146}} & \mc{1}{c}{\scriptsize{-0.198}} & \mc{1}{c}{\scriptsize{-0.210}} & \mc{1}{c}{\scriptsize{-0.242}} \\  

     &  & \mc{1}{c}{\scriptsize{\textbf{(0.000)}}} & \mc{1}{c}{\scriptsize{\textbf{(0.059)}}} & \mc{1}{c}{\scriptsize{(0.549)}} & \mc{1}{c}{\scriptsize{(0.902)}} & \mc{1}{c}{\scriptsize{\textbf{(0.078)}}} & \mc{1}{c}{\scriptsize{\textbf{(0.000)}}} & \mc{1}{c}{\scriptsize{\textbf{(0.020)}}} & \mc{1}{c}{\scriptsize{\textbf{(0.020)}}} \\  

    \mc{1}{l}{\scriptsize{Hypertension}} & \mc{1}{c}{\scriptsize{Mid-30s}} & \mc{1}{c}{\scriptsize{0.027}} & \mc{1}{c}{\scriptsize{0.023}} & \mc{1}{c}{\scriptsize{0.083}} & \mc{1}{c}{\scriptsize{0.232}} & \mc{1}{c}{\scriptsize{0.010}} & \mc{1}{c}{\scriptsize{0.002}} & \mc{1}{c}{\scriptsize{-0.011}} & \mc{1}{c}{\scriptsize{-0.037}} \\  

     &  & \mc{1}{c}{\scriptsize{(0.549)}} & \mc{1}{c}{\scriptsize{(0.490)}} & \mc{1}{c}{\scriptsize{(0.706)}} & \mc{1}{c}{\scriptsize{(0.902)}} & \mc{1}{c}{\scriptsize{(0.392)}} & \mc{1}{c}{\scriptsize{(0.490)}} & \mc{1}{c}{\scriptsize{(0.431)}} & \mc{1}{c}{\scriptsize{(0.235)}} \\  

    \mc{1}{l}{\scriptsize{High-Density Lipoprotein Chol. (mg/dL)}} & \mc{1}{c}{\scriptsize{Mid-30s}} & \mc{1}{c}{\scriptsize{3.504}} & \mc{1}{c}{\scriptsize{3.466}} & \mc{1}{c}{\scriptsize{7.694}} & \mc{1}{c}{\scriptsize{5.468}} & \mc{1}{c}{\scriptsize{8.639}} & \mc{1}{c}{\scriptsize{1.828}} & \mc{1}{c}{\scriptsize{2.000}} & \mc{1}{c}{\scriptsize{2.791}} \\  

     &  & \mc{1}{c}{\scriptsize{\textbf{(0.098)}}} & \mc{1}{c}{\scriptsize{(0.176)}} & \mc{1}{c}{\scriptsize{\textbf{(0.000)}}} & \mc{1}{c}{\scriptsize{\textbf{(0.078)}}} & \mc{1}{c}{\scriptsize{\textbf{(0.000)}}} & \mc{1}{c}{\scriptsize{(0.294)}} & \mc{1}{c}{\scriptsize{(0.275)}} & \mc{1}{c}{\scriptsize{(0.157)}} \\  

    \mc{1}{l}{\scriptsize{Dyslipidemia}} & \mc{1}{c}{\scriptsize{Mid-30s}} & \mc{1}{c}{\scriptsize{-0.013}} & \mc{1}{c}{\scriptsize{0.008}} & \mc{1}{c}{\scriptsize{-0.076}} & \mc{1}{c}{\scriptsize{0.018}} & \mc{1}{c}{\scriptsize{-0.076}} & \mc{1}{c}{\scriptsize{0.013}} & \mc{1}{c}{\scriptsize{0.028}} & \mc{1}{c}{\scriptsize{-0.010}} \\  

     &  & \mc{1}{c}{\scriptsize{(0.314)}} & \mc{1}{c}{\scriptsize{(0.490)}} & \mc{1}{c}{\scriptsize{(0.157)}} & \mc{1}{c}{\scriptsize{(0.490)}} & \mc{1}{c}{\scriptsize{(0.157)}} & \mc{1}{c}{\scriptsize{(0.588)}} & \mc{1}{c}{\scriptsize{(0.608)}} & \mc{1}{c}{\scriptsize{(0.333)}} \\  

    \mc{1}{l}{\scriptsize{Hemoglobin Level (\%)}} & \mc{1}{c}{\scriptsize{Mid-30s}} & \mc{1}{c}{\scriptsize{0.064}} & \mc{1}{c}{\scriptsize{-0.098}} & \mc{1}{c}{\scriptsize{0.093}} & \mc{1}{c}{\scriptsize{-0.044}} & \mc{1}{c}{\scriptsize{0.048}} & \mc{1}{c}{\scriptsize{0.052}} & \mc{1}{c}{\scriptsize{-0.093}} & \mc{1}{c}{\scriptsize{-0.034}} \\  

     &  & \mc{1}{c}{\scriptsize{(0.647)}} & \mc{1}{c}{\scriptsize{(0.294)}} & \mc{1}{c}{\scriptsize{(0.647)}} & \mc{1}{c}{\scriptsize{(0.353)}} & \mc{1}{c}{\scriptsize{(0.392)}} & \mc{1}{c}{\scriptsize{(0.627)}} & \mc{1}{c}{\scriptsize{(0.333)}} & \mc{1}{c}{\scriptsize{(0.333)}} \\  

    \mc{1}{l}{\scriptsize{Prediabetes}} & \mc{1}{c}{\scriptsize{Mid-30s}} & \mc{1}{c}{\scriptsize{0.007}} & \mc{1}{c}{\scriptsize{-0.036}} & \mc{1}{c}{\scriptsize{-0.040}} & \mc{1}{c}{\scriptsize{-0.097}} & \mc{1}{c}{\scriptsize{-0.027}} & \mc{1}{c}{\scriptsize{0.026}} & \mc{1}{c}{\scriptsize{-0.011}} & \mc{1}{c}{\scriptsize{-0.021}} \\  

     &  & \mc{1}{c}{\scriptsize{(0.569)}} & \mc{1}{c}{\scriptsize{(0.392)}} & \mc{1}{c}{\scriptsize{(0.431)}} & \mc{1}{c}{\scriptsize{(0.255)}} & \mc{1}{c}{\scriptsize{(0.275)}} & \mc{1}{c}{\scriptsize{(0.627)}} & \mc{1}{c}{\scriptsize{(0.549)}} & \mc{1}{c}{\scriptsize{(0.333)}} \\  

    \mc{1}{l}{\scriptsize{Diabetes}} & \mc{1}{c}{\scriptsize{Mid-30s}} & \mc{1}{c}{\scriptsize{0.011}} & \mc{1}{c}{\scriptsize{-0.026}} & \mc{1}{c}{\scriptsize{0.043}} & \mc{1}{c}{\scriptsize{0.014}} & \mc{1}{c}{\scriptsize{0.027}} & \mc{1}{c}{\scriptsize{-0.002}} & \mc{1}{c}{\scriptsize{-0.035}} & \mc{1}{c}{\scriptsize{-0.027}} \\  

     &  & \mc{1}{c}{\scriptsize{(0.627)}} & \mc{1}{c}{\scriptsize{(0.235)}} & \mc{1}{c}{\scriptsize{(0.804)}} & \mc{1}{c}{\scriptsize{(0.686)}} & \mc{1}{c}{\scriptsize{(0.412)}} & \mc{1}{c}{\scriptsize{(0.490)}} & \mc{1}{c}{\scriptsize{(0.196)}} & \mc{1}{c}{\scriptsize{(0.137)}} \\  

    \mc{1}{l}{\scriptsize{Vitamin D Deficiency}} & \mc{1}{c}{\scriptsize{Mid-30s}} & \mc{1}{c}{\scriptsize{-0.169}} & \mc{1}{c}{\scriptsize{-0.096}} & \mc{1}{c}{\scriptsize{-0.264}} & \mc{1}{c}{\scriptsize{-0.137}} & \mc{1}{c}{\scriptsize{-0.204}} & \mc{1}{c}{\scriptsize{-0.131}} & \mc{1}{c}{\scriptsize{-0.066}} & \mc{1}{c}{\scriptsize{-0.129}} \\  

     &  & \mc{1}{c}{\scriptsize{\textbf{(0.059)}}} & \mc{1}{c}{\scriptsize{(0.137)}} & \mc{1}{c}{\scriptsize{\textbf{(0.000)}}} & \mc{1}{c}{\scriptsize{\textbf{(0.098)}}} & \mc{1}{c}{\scriptsize{\textbf{(0.039)}}} & \mc{1}{c}{\scriptsize{\textbf{(0.098)}}} & \mc{1}{c}{\scriptsize{(0.235)}} & \mc{1}{c}{\scriptsize{(0.118)}} \\  

  \hline\hline
  \end{tabular}
    \begin{tablenotes}
    \scriptsize
    \item 
Note: This table displays various estimates of the treatment effect of ABC/CARE's center-based care.
Column (1) displays the ITT, without accounting for any controls.
Column (2) displays the ITT conditioning on vector of controls, $X$, consisting of APGAR scores 1 
minute after birth, an indicator for the subject being born prematurely, and an indicator for the 
father being home at baseline. We also apply IPW weights, $W$, to account for attrition.
Columns (3)--(4) are analogous to columns (1)--(2), but we restrict the control sample to subjects
who did not enroll in any alternative care.
Column (5) displys the matching estimate, where we use the Mahalanobis metric and Epanechnikov kernel
to match on controls $X$ listed above, and restrict the control sample to subjects who did not enroll
in any alternative care. Additionally, we apply IPW weights, $W$.
Columns (6)--(8) are analogous to Columns (3)--(5), except we restrict the control sample to subejcts
who did enroll in alternative care. 
Numbers in parentheses represent the $p$-value from a single hypothesis test, and are obtained from 
the empirical bootstrap distribution generated by 200 resamples of the original data. 
Bold $p$-values indicate significance at the 10\% level.
Blank point estimates indicate that we are unable to obtain estimates due to a lack of support in the data. 

    \end{tablenotes}
  \end{threeparttable}

\end{table}
\end{center}

\begin{center}
	  \begin{tabular}{cccccccccc}
  \toprule

    \scriptsize{Variable} & \scriptsize{Age} & \scriptsize{(1)} & \scriptsize{(2)} & \scriptsize{(3)} & \scriptsize{(4)} & \scriptsize{(5)} & \scriptsize{(6)} & \scriptsize{(7)} & \scriptsize{(8)} \\ 
    \midrule  

    \mc{1}{l}{\scriptsize{Cig. Smoked per day last month}} & \mc{1}{c}{\scriptsize{30}} & \mc{1}{c}{\scriptsize{0.873}} & \mc{1}{c}{\scriptsize{-1.074}} & \mc{1}{c}{\scriptsize{-0.638}} & \mc{1}{c}{\scriptsize{-0.902}} & \mc{1}{c}{\scriptsize{-2.316}} & \mc{1}{c}{\scriptsize{1.628}} & \mc{1}{c}{\scriptsize{-0.720}} & \mc{1}{c}{\scriptsize{0.003}} \\  

     &  & \mc{1}{c}{\scriptsize{(0.645)}} & \mc{1}{c}{\scriptsize{(0.355)}} & \mc{1}{c}{\scriptsize{(0.342)}} & \mc{1}{c}{\scriptsize{(0.329)}} & \mc{1}{c}{\scriptsize{(0.289)}} & \mc{1}{c}{\scriptsize{(0.776)}} & \mc{1}{c}{\scriptsize{(0.395)}} & \mc{1}{c}{\scriptsize{(0.461)}} \\  

    \mc{1}{l}{\scriptsize{Days drank alcohol last month}} & \mc{1}{c}{\scriptsize{30}} & \mc{1}{c}{\scriptsize{-1.958}} & \mc{1}{c}{\scriptsize{-3.674}} & \mc{1}{c}{\scriptsize{-1.803}} & \mc{1}{c}{\scriptsize{-2.272}} & \mc{1}{c}{\scriptsize{-3.226}} & \mc{1}{c}{\scriptsize{-2.036}} & \mc{1}{c}{\scriptsize{-4.755}} & \mc{1}{c}{\scriptsize{-4.546}} \\  

     &  & \mc{1}{c}{\scriptsize{(0.158)}} & \mc{1}{c}{\scriptsize{\textbf{(0.026)}}} & \mc{1}{c}{\scriptsize{(0.250)}} & \mc{1}{c}{\scriptsize{(0.118)}} & \mc{1}{c}{\scriptsize{\textbf{(0.092)}}} & \mc{1}{c}{\scriptsize{(0.145)}} & \mc{1}{c}{\scriptsize{\textbf{(0.013)}}} & \mc{1}{c}{\scriptsize{\textbf{(0.013)}}} \\  

    \mc{1}{l}{\scriptsize{Self-reported drug user}} & \mc{1}{c}{\scriptsize{Mid-30s}} & \mc{1}{c}{\scriptsize{-0.140}} & \mc{1}{c}{\scriptsize{-0.389}} & \mc{1}{c}{\scriptsize{-0.235}} & \mc{1}{c}{\scriptsize{-0.460}} & \mc{1}{c}{\scriptsize{-0.558}} & \mc{1}{c}{\scriptsize{-0.068}} & \mc{1}{c}{\scriptsize{-0.143}} & \mc{1}{c}{\scriptsize{-0.242}} \\  

     &  & \mc{1}{c}{\scriptsize{(0.132)}} & \mc{1}{c}{\scriptsize{\textbf{(0.039)}}} & \mc{1}{c}{\scriptsize{(0.105)}} & \mc{1}{c}{\scriptsize{(0.184)}} & \mc{1}{c}{\scriptsize{\textbf{(0.000)}}} & \mc{1}{c}{\scriptsize{(0.329)}} & \mc{1}{c}{\scriptsize{(0.184)}} & \mc{1}{c}{\scriptsize{\textbf{(0.066)}}} \\  

    \mc{1}{l}{\scriptsize{Measured BMI}} & \mc{1}{c}{\scriptsize{Mid-30s}} & \mc{1}{c}{\scriptsize{9.889}} & \mc{1}{c}{\scriptsize{6.380}} & \mc{1}{c}{\scriptsize{7.088}} & \mc{1}{c}{\scriptsize{3.604}} & \mc{1}{c}{\scriptsize{2.105}} & \mc{1}{c}{\scriptsize{11.990}} & \mc{1}{c}{\scriptsize{8.713}} & \mc{1}{c}{\scriptsize{9.842}} \\  

     &  & \mc{1}{c}{\scriptsize{(1.000)}} & \mc{1}{c}{\scriptsize{(0.961)}} & \mc{1}{c}{\scriptsize{(0.961)}} & \mc{1}{c}{\scriptsize{(0.658)}} & \mc{1}{c}{\scriptsize{(0.592)}} & \mc{1}{c}{\scriptsize{(1.000)}} & \mc{1}{c}{\scriptsize{(0.921)}} & \mc{1}{c}{\scriptsize{(0.974)}} \\  

    \mc{1}{l}{\scriptsize{Obesity}} & \mc{1}{c}{\scriptsize{Mid-30s}} & \mc{1}{c}{\scriptsize{0.282}} & \mc{1}{c}{\scriptsize{0.172}} & \mc{1}{c}{\scriptsize{0.068}} & \mc{1}{c}{\scriptsize{-0.156}} & \mc{1}{c}{\scriptsize{-0.141}} & \mc{1}{c}{\scriptsize{0.443}} & \mc{1}{c}{\scriptsize{0.525}} & \mc{1}{c}{\scriptsize{0.530}} \\  

     &  & \mc{1}{c}{\scriptsize{(1.000)}} & \mc{1}{c}{\scriptsize{(0.842)}} & \mc{1}{c}{\scriptsize{(0.618)}} & \mc{1}{c}{\scriptsize{(0.237)}} & \mc{1}{c}{\scriptsize{(0.289)}} & \mc{1}{c}{\scriptsize{(1.000)}} & \mc{1}{c}{\scriptsize{(0.908)}} & \mc{1}{c}{\scriptsize{(1.000)}} \\  

    \mc{1}{l}{\scriptsize{Severe Obesity}} & \mc{1}{c}{\scriptsize{Mid-30s}} & \mc{1}{c}{\scriptsize{0.260}} & \mc{1}{c}{\scriptsize{0.093}} & \mc{1}{c}{\scriptsize{0.129}} & \mc{1}{c}{\scriptsize{-0.108}} & \mc{1}{c}{\scriptsize{-0.291}} & \mc{1}{c}{\scriptsize{0.358}} & \mc{1}{c}{\scriptsize{0.283}} & \mc{1}{c}{\scriptsize{0.207}} \\  

     &  & \mc{1}{c}{\scriptsize{(0.908)}} & \mc{1}{c}{\scriptsize{(0.645)}} & \mc{1}{c}{\scriptsize{(0.724)}} & \mc{1}{c}{\scriptsize{(0.408)}} & \mc{1}{c}{\scriptsize{(0.197)}} & \mc{1}{c}{\scriptsize{(0.961)}} & \mc{1}{c}{\scriptsize{(0.895)}} & \mc{1}{c}{\scriptsize{(0.803)}} \\  

    \mc{1}{l}{\scriptsize{Waist-hip Ratio}} & \mc{1}{c}{\scriptsize{Mid-30s}} & \mc{1}{c}{\scriptsize{0.064}} & \mc{1}{c}{\scriptsize{0.045}} & \mc{1}{c}{\scriptsize{0.025}} & \mc{1}{c}{\scriptsize{0.011}} & \mc{1}{c}{\scriptsize{-0.000}} & \mc{1}{c}{\scriptsize{0.093}} & \mc{1}{c}{\scriptsize{0.105}} & \mc{1}{c}{\scriptsize{0.092}} \\  

     &  & \mc{1}{c}{\scriptsize{(0.987)}} & \mc{1}{c}{\scriptsize{(0.921)}} & \mc{1}{c}{\scriptsize{(0.763)}} & \mc{1}{c}{\scriptsize{(0.553)}} & \mc{1}{c}{\scriptsize{(0.513)}} & \mc{1}{c}{\scriptsize{(1.000)}} & \mc{1}{c}{\scriptsize{(1.000)}} & \mc{1}{c}{\scriptsize{(0.974)}} \\  

    \mc{1}{l}{\scriptsize{Abdominal Obesity}} & \mc{1}{c}{\scriptsize{Mid-30s}} & \mc{1}{c}{\scriptsize{0.175}} & \mc{1}{c}{\scriptsize{0.070}} & \mc{1}{c}{\scriptsize{-0.015}} & \mc{1}{c}{\scriptsize{-0.122}} & \mc{1}{c}{\scriptsize{-0.186}} & \mc{1}{c}{\scriptsize{0.318}} & \mc{1}{c}{\scriptsize{0.425}} & \mc{1}{c}{\scriptsize{0.394}} \\  

     &  & \mc{1}{c}{\scriptsize{(0.947)}} & \mc{1}{c}{\scriptsize{(0.671)}} & \mc{1}{c}{\scriptsize{(0.434)}} & \mc{1}{c}{\scriptsize{(0.263)}} & \mc{1}{c}{\scriptsize{(0.197)}} & \mc{1}{c}{\scriptsize{(1.000)}} & \mc{1}{c}{\scriptsize{(0.961)}} & \mc{1}{c}{\scriptsize{(0.987)}} \\  

    \mc{1}{l}{\scriptsize{Framingham Risk Score}} & \mc{1}{c}{\scriptsize{Mid-30s}} & \mc{1}{c}{\scriptsize{1.013}} & \mc{1}{c}{\scriptsize{0.517}} & \mc{1}{c}{\scriptsize{0.203}} & \mc{1}{c}{\scriptsize{-0.558}} & \mc{1}{c}{\scriptsize{-0.515}} & \mc{1}{c}{\scriptsize{1.620}} & \mc{1}{c}{\scriptsize{1.979}} & \mc{1}{c}{\scriptsize{1.818}} \\  

     &  & \mc{1}{c}{\scriptsize{(0.908)}} & \mc{1}{c}{\scriptsize{(0.737)}} & \mc{1}{c}{\scriptsize{(0.592)}} & \mc{1}{c}{\scriptsize{(0.355)}} & \mc{1}{c}{\scriptsize{(0.303)}} & \mc{1}{c}{\scriptsize{(1.000)}} & \mc{1}{c}{\scriptsize{(0.987)}} & \mc{1}{c}{\scriptsize{(1.000)}} \\  

  \bottomrule
  \end{tabular}
\end{center}

\begin{center}
	  \begin{tabular}{cccccccccc}
  \toprule

    \scriptsize{Variable} & \scriptsize{Age} & \scriptsize{(1)} & \scriptsize{(2)} & \scriptsize{(3)} & \scriptsize{(4)} & \scriptsize{(5)} & \scriptsize{(6)} & \scriptsize{(7)} & \scriptsize{(8)} \\ 
    \midrule  

    \mc{1}{l}{\scriptsize{Std. IQ Test}} & \mc{1}{c}{\scriptsize{21}} & \mc{1}{c}{\scriptsize{1.550}} & \mc{1}{c}{\scriptsize{0.138}} & \mc{1}{c}{\scriptsize{0.471}} & \mc{1}{c}{\scriptsize{-1.535}} & \mc{1}{c}{\scriptsize{-2.063}} & \mc{1}{c}{\scriptsize{2.307}} & \mc{1}{c}{\scriptsize{0.267}} & \mc{1}{c}{\scriptsize{-0.037}} \\  

     &  & \mc{1}{c}{\scriptsize{(0.224)}} & \mc{1}{c}{\scriptsize{(0.474)}} & \mc{1}{c}{\scriptsize{(0.526)}} & \mc{1}{c}{\scriptsize{(0.816)}} & \mc{1}{c}{\scriptsize{(0.829)}} & \mc{1}{c}{\scriptsize{(0.211)}} & \mc{1}{c}{\scriptsize{(0.434)}} & \mc{1}{c}{\scriptsize{(0.592)}} \\  

    \mc{1}{l}{\scriptsize{Std. Achv.  Test}} & \mc{1}{c}{\scriptsize{21}} & \mc{1}{c}{\scriptsize{1.181}} & \mc{1}{c}{\scriptsize{0.376}} & \mc{1}{c}{\scriptsize{1.168}} & \mc{1}{c}{\scriptsize{-1.188}} & \mc{1}{c}{\scriptsize{-1.115}} & \mc{1}{c}{\scriptsize{1.356}} & \mc{1}{c}{\scriptsize{-0.008}} & \mc{1}{c}{\scriptsize{-0.502}} \\  

     &  & \mc{1}{c}{\scriptsize{(0.329)}} & \mc{1}{c}{\scriptsize{(0.408)}} & \mc{1}{c}{\scriptsize{(0.395)}} & \mc{1}{c}{\scriptsize{(0.553)}} & \mc{1}{c}{\scriptsize{(0.487)}} & \mc{1}{c}{\scriptsize{(0.368)}} & \mc{1}{c}{\scriptsize{(0.487)}} & \mc{1}{c}{\scriptsize{(0.553)}} \\  

    \mc{1}{l}{\scriptsize{Graduated High School}} & \mc{1}{c}{\scriptsize{30}} & \mc{1}{c}{\scriptsize{0.180}} & \mc{1}{c}{\scriptsize{0.183}} & \mc{1}{c}{\scriptsize{0.104}} & \mc{1}{c}{\scriptsize{0.136}} & \mc{1}{c}{\scriptsize{0.060}} & \mc{1}{c}{\scriptsize{0.237}} & \mc{1}{c}{\scriptsize{0.230}} & \mc{1}{c}{\scriptsize{0.159}} \\  

     &  & \mc{1}{c}{\scriptsize{\textbf{(0.066)}}} & \mc{1}{c}{\scriptsize{\textbf{(0.013)}}} & \mc{1}{c}{\scriptsize{(0.303)}} & \mc{1}{c}{\scriptsize{(0.237)}} & \mc{1}{c}{\scriptsize{(0.408)}} & \mc{1}{c}{\scriptsize{\textbf{(0.066)}}} & \mc{1}{c}{\scriptsize{\textbf{(0.053)}}} & \mc{1}{c}{\scriptsize{(0.132)}} \\  

    \mc{1}{l}{\scriptsize{Years of Edu.}} & \mc{1}{c}{\scriptsize{30}} & \mc{1}{c}{\scriptsize{0.868}} & \mc{1}{c}{\scriptsize{0.815}} & \mc{1}{c}{\scriptsize{0.630}} & \mc{1}{c}{\scriptsize{1.041}} & \mc{1}{c}{\scriptsize{0.156}} & \mc{1}{c}{\scriptsize{0.896}} & \mc{1}{c}{\scriptsize{0.874}} & \mc{1}{c}{\scriptsize{0.292}} \\  

     &  & \mc{1}{c}{\scriptsize{\textbf{(0.039)}}} & \mc{1}{c}{\scriptsize{(0.118)}} & \mc{1}{c}{\scriptsize{(0.197)}} & \mc{1}{c}{\scriptsize{(0.237)}} & \mc{1}{c}{\scriptsize{(0.434)}} & \mc{1}{c}{\scriptsize{\textbf{(0.066)}}} & \mc{1}{c}{\scriptsize{(0.132)}} & \mc{1}{c}{\scriptsize{(0.303)}} \\  

    \mc{1}{l}{\scriptsize{Labor Income}} & \mc{1}{c}{\scriptsize{30}} & \mc{1}{c}{\scriptsize{5,843}} & \mc{1}{c}{\scriptsize{3,843}} & \mc{1}{c}{\scriptsize{4,251}} & \mc{1}{c}{\scriptsize{-3,184}} & \mc{1}{c}{\scriptsize{2,870}} & \mc{1}{c}{\scriptsize{6,135}} & \mc{1}{c}{\scriptsize{6,582}} & \mc{1}{c}{\scriptsize{5,644}} \\  

     &  & \mc{1}{c}{\scriptsize{(0.145)}} & \mc{1}{c}{\scriptsize{(0.276)}} & \mc{1}{c}{\scriptsize{(0.250)}} & \mc{1}{c}{\scriptsize{(0.632)}} & \mc{1}{c}{\scriptsize{(0.329)}} & \mc{1}{c}{\scriptsize{(0.171)}} & \mc{1}{c}{\scriptsize{(0.145)}} & \mc{1}{c}{\scriptsize{(0.158)}} \\  

    \mc{1}{l}{\scriptsize{Public-Transfer Income}} & \mc{1}{c}{\scriptsize{30}} & \mc{1}{c}{\scriptsize{-343}} & \mc{1}{c}{\scriptsize{-112}} & \mc{1}{c}{\scriptsize{153}} & \mc{1}{c}{\scriptsize{293}} & \mc{1}{c}{\scriptsize{267}} & \mc{1}{c}{\scriptsize{-204}} & \mc{1}{c}{\scriptsize{-200}} & \mc{1}{c}{\scriptsize{-64.341}} \\  

     &  & \mc{1}{c}{\scriptsize{(0.171)}} & \mc{1}{c}{\scriptsize{(0.329)}} & \mc{1}{c}{\scriptsize{(0.618)}} & \mc{1}{c}{\scriptsize{(0.500)}} & \mc{1}{c}{\scriptsize{(0.618)}} & \mc{1}{c}{\scriptsize{(0.250)}} & \mc{1}{c}{\scriptsize{(0.276)}} & \mc{1}{c}{\scriptsize{(0.355)}} \\  

    \mc{1}{l}{\scriptsize{Employed}} & \mc{1}{c}{\scriptsize{30}} & \mc{1}{c}{\scriptsize{0.222}} & \mc{1}{c}{\scriptsize{0.240}} & \mc{1}{c}{\scriptsize{0.089}} & \mc{1}{c}{\scriptsize{0.055}} & \mc{1}{c}{\scriptsize{0.107}} & \mc{1}{c}{\scriptsize{0.289}} & \mc{1}{c}{\scriptsize{0.289}} & \mc{1}{c}{\scriptsize{0.337}} \\  

     &  & \mc{1}{c}{\scriptsize{\textbf{(0.039)}}} & \mc{1}{c}{\scriptsize{\textbf{(0.013)}}} & \mc{1}{c}{\scriptsize{(0.329)}} & \mc{1}{c}{\scriptsize{(0.368)}} & \mc{1}{c}{\scriptsize{(0.276)}} & \mc{1}{c}{\scriptsize{\textbf{(0.066)}}} & \mc{1}{c}{\scriptsize{\textbf{(0.039)}}} & \mc{1}{c}{\scriptsize{\textbf{(0.026)}}} \\  

    \mc{1}{l}{\scriptsize{Total Felony Arrests}} & \mc{1}{c}{\scriptsize{Mid-30s}} & \mc{1}{c}{\scriptsize{-0.501}} & \mc{1}{c}{\scriptsize{-0.381}} & \mc{1}{c}{\scriptsize{0.530}} & \mc{1}{c}{\scriptsize{0.491}} & \mc{1}{c}{\scriptsize{0.858}} & \mc{1}{c}{\scriptsize{-0.793}} & \mc{1}{c}{\scriptsize{-0.752}} & \mc{1}{c}{\scriptsize{-0.649}} \\  

     &  & \mc{1}{c}{\scriptsize{(0.171)}} & \mc{1}{c}{\scriptsize{(0.316)}} & \mc{1}{c}{\scriptsize{(0.816)}} & \mc{1}{c}{\scriptsize{(0.750)}} & \mc{1}{c}{\scriptsize{(0.934)}} & \mc{1}{c}{\scriptsize{(0.158)}} & \mc{1}{c}{\scriptsize{(0.250)}} & \mc{1}{c}{\scriptsize{(0.197)}} \\  

    \mc{1}{l}{\scriptsize{Total Misdemeanor Arrests}} & \mc{1}{c}{\scriptsize{Mid-30s}} & \mc{1}{c}{\scriptsize{-0.334}} & \mc{1}{c}{\scriptsize{-0.471}} & \mc{1}{c}{\scriptsize{-0.713}} & \mc{1}{c}{\scriptsize{-1.290}} & \mc{1}{c}{\scriptsize{-0.247}} & \mc{1}{c}{\scriptsize{-0.298}} & \mc{1}{c}{\scriptsize{-0.332}} & \mc{1}{c}{\scriptsize{-0.300}} \\  

     &  & \mc{1}{c}{\scriptsize{(0.276)}} & \mc{1}{c}{\scriptsize{(0.158)}} & \mc{1}{c}{\scriptsize{(0.303)}} & \mc{1}{c}{\scriptsize{(0.132)}} & \mc{1}{c}{\scriptsize{(0.329)}} & \mc{1}{c}{\scriptsize{(0.303)}} & \mc{1}{c}{\scriptsize{(0.303)}} & \mc{1}{c}{\scriptsize{(0.263)}} \\  

    \mc{1}{l}{\scriptsize{Total Years Incarcerated}} & \mc{1}{c}{\scriptsize{30}} & \mc{1}{c}{\scriptsize{0.163}} & \mc{1}{c}{\scriptsize{0.323}} & \mc{1}{c}{\scriptsize{0.483}} & \mc{1}{c}{\scriptsize{0.555}} & \mc{1}{c}{\scriptsize{0.592}} & \mc{1}{c}{\scriptsize{0.130}} & \mc{1}{c}{\scriptsize{0.295}} & \mc{1}{c}{\scriptsize{0.194}} \\  

     &  & \mc{1}{c}{\scriptsize{(0.632)}} & \mc{1}{c}{\scriptsize{(0.763)}} & \mc{1}{c}{\scriptsize{(0.974)}} & \mc{1}{c}{\scriptsize{(0.816)}} & \mc{1}{c}{\scriptsize{(0.947)}} & \mc{1}{c}{\scriptsize{(0.605)}} & \mc{1}{c}{\scriptsize{(0.711)}} & \mc{1}{c}{\scriptsize{(0.645)}} \\  

  \bottomrule
  \end{tabular}
\end{center}

\begin{center}
	\begin{sidewaystable}[H]
\captionsetup{singlelinecheck=false,justification=centering}
\caption{Summary of Treatment Effects of Center-based Childcare on Males \\ Health Outcomes \label{tab:ate_male_main2}}

  \begin{threeparttable}
  \begin{tabular}{cccccccccc}
  \toprule

     &  & \footnotesize{(1)} & \footnotesize{(2)} & \footnotesize{(3)} & \footnotesize{(4)} & \footnotesize{(5)} & \footnotesize{(6)} & \footnotesize{(7)} & \footnotesize{(8)} \\  

     &  &  &  & \mc{3}{c}{\footnotesize{$P=0$}} & \mc{3}{c}{\footnotesize{$P=1$}} \\ 
    \cmidrule(lr){5-7} \cmidrule(lr){8-10} 

    \footnotesize{Variable} & \footnotesize{Age} & \footnotesize{ITT} & \footnotesize{ITT$|X,W$} & \footnotesize{ITT} & \footnotesize{ITT$|X,W$} & \footnotesize{KE$|X,W$} & \footnotesize{ITT} & \footnotesize{ITT$|X,W$} & \footnotesize{KE$|X,W$} \\ 
    \midrule

    \mc{1}{l}{\footnotesize{Systolic Blood Pressure (mm Hg)}} & \mc{1}{c}{\footnotesize{Mid-30s}} & \mc{1}{c}{\footnotesize{-5.863}} & \mc{1}{c}{\footnotesize{-9.171}} & \mc{1}{c}{\footnotesize{8.280}} & \mc{1}{c}{\footnotesize{12.991}} & \mc{1}{c}{\footnotesize{3.511}} & \mc{1}{c}{\footnotesize{-12.934}} & \mc{1}{c}{\footnotesize{-14.633}} & \mc{1}{c}{\footnotesize{-12.752}} \\  

     &  & \mc{1}{c}{\footnotesize{(0.176)}} & \mc{1}{c}{\footnotesize{(0.137)}} & \mc{1}{c}{\footnotesize{(0.980)}} & \mc{1}{c}{\footnotesize{(1.000)}} & \mc{1}{c}{\footnotesize{(0.510)}} & \mc{1}{c}{\footnotesize{\textbf{(0.039)}}} & \mc{1}{c}{\footnotesize{\textbf{(0.098)}}} & \mc{1}{c}{\footnotesize{\textbf{(0.039)}}} \\  

    \mc{1}{l}{\footnotesize{Diastolic Blood Pressure (mm Hg)}} & \mc{1}{c}{\footnotesize{Mid-30s}} & \mc{1}{c}{\footnotesize{-9.116}} & \mc{1}{c}{\footnotesize{-11.511}} & \mc{1}{c}{\footnotesize{-4.926}} & \mc{1}{c}{\footnotesize{-3.083}} & \mc{1}{c}{\footnotesize{-10.839}} & \mc{1}{c}{\footnotesize{-11.211}} & \mc{1}{c}{\footnotesize{-13.508}} & \mc{1}{c}{\footnotesize{-11.674}} \\  

     &  & \mc{1}{c}{\footnotesize{\textbf{(0.020)}}} & \mc{1}{c}{\footnotesize{\textbf{(0.039)}}} & \mc{1}{c}{\footnotesize{(0.196)}} & \mc{1}{c}{\footnotesize{(0.333)}} & \mc{1}{c}{\footnotesize{\textbf{(0.020)}}} & \mc{1}{c}{\footnotesize{\textbf{(0.020)}}} & \mc{1}{c}{\footnotesize{\textbf{(0.039)}}} & \mc{1}{c}{\footnotesize{\textbf{(0.000)}}} \\  

    \mc{1}{l}{\footnotesize{Prehypertension}} & \mc{1}{c}{\footnotesize{Mid-30s}} & \mc{1}{c}{\footnotesize{-0.089}} & \mc{1}{c}{\footnotesize{-0.090}} & \mc{1}{c}{\footnotesize{0.149}} & \mc{1}{c}{\footnotesize{0.240}} & \mc{1}{c}{\footnotesize{-0.070}} & \mc{1}{c}{\footnotesize{-0.209}} & \mc{1}{c}{\footnotesize{-0.200}} & \mc{1}{c}{\footnotesize{-0.274}} \\  

     &  & \mc{1}{c}{\footnotesize{(0.255)}} & \mc{1}{c}{\footnotesize{(0.314)}} & \mc{1}{c}{\footnotesize{(0.686)}} & \mc{1}{c}{\footnotesize{(0.804)}} & \mc{1}{c}{\footnotesize{(0.294)}} & \mc{1}{c}{\footnotesize{\textbf{(0.020)}}} & \mc{1}{c}{\footnotesize{\textbf{(0.039)}}} & \mc{1}{c}{\footnotesize{\textbf{(0.059)}}} \\  

    \mc{1}{l}{\footnotesize{Hypertension}} & \mc{1}{c}{\footnotesize{Mid-30s}} & \mc{1}{c}{\footnotesize{-0.149}} & \mc{1}{c}{\footnotesize{-0.188}} & \mc{1}{c}{\footnotesize{-0.006}} & \mc{1}{c}{\footnotesize{0.140}} & \mc{1}{c}{\footnotesize{-0.165}} & \mc{1}{c}{\footnotesize{-0.220}} & \mc{1}{c}{\footnotesize{-0.254}} & \mc{1}{c}{\footnotesize{-0.327}} \\  

     &  & \mc{1}{c}{\footnotesize{(0.235)}} & \mc{1}{c}{\footnotesize{(0.176)}} & \mc{1}{c}{\footnotesize{(0.451)}} & \mc{1}{c}{\footnotesize{(0.725)}} & \mc{1}{c}{\footnotesize{(0.196)}} & \mc{1}{c}{\footnotesize{(0.118)}} & \mc{1}{c}{\footnotesize{\textbf{(0.078)}}} & \mc{1}{c}{\footnotesize{\textbf{(0.020)}}} \\  

    \mc{1}{l}{\footnotesize{High-Density Lipoprotein Chol. (mg/dL)}} & \mc{1}{c}{\footnotesize{Mid-30s}} & \mc{1}{c}{\footnotesize{5.080}} & \mc{1}{c}{\footnotesize{7.206}} & \mc{1}{c}{\footnotesize{6.114}} & \mc{1}{c}{\footnotesize{5.760}} & \mc{1}{c}{\footnotesize{6.655}} & \mc{1}{c}{\footnotesize{4.678}} & \mc{1}{c}{\footnotesize{6.229}} & \mc{1}{c}{\footnotesize{3.062}} \\  

     &  & \mc{1}{c}{\footnotesize{\textbf{(0.098)}}} & \mc{1}{c}{\footnotesize{\textbf{(0.098)}}} & \mc{1}{c}{\footnotesize{\textbf{(0.098)}}} & \mc{1}{c}{\footnotesize{(0.235)}} & \mc{1}{c}{\footnotesize{\textbf{(0.078)}}} & \mc{1}{c}{\footnotesize{(0.196)}} & \mc{1}{c}{\footnotesize{(0.176)}} & \mc{1}{c}{\footnotesize{(0.235)}} \\  

    \mc{1}{l}{\footnotesize{Dyslipidemia}} & \mc{1}{c}{\footnotesize{Mid-30s}} & \mc{1}{c}{\footnotesize{-0.120}} & \mc{1}{c}{\footnotesize{-0.122}} & \mc{1}{c}{\footnotesize{-0.229}} & \mc{1}{c}{\footnotesize{-0.108}} & \mc{1}{c}{\footnotesize{-0.231}} & \mc{1}{c}{\footnotesize{-0.078}} & \mc{1}{c}{\footnotesize{-0.073}} & \mc{1}{c}{\footnotesize{-0.089}} \\  

     &  & \mc{1}{c}{\footnotesize{(0.118)}} & \mc{1}{c}{\footnotesize{(0.255)}} & \mc{1}{c}{\footnotesize{(0.176)}} & \mc{1}{c}{\footnotesize{(0.392)}} & \mc{1}{c}{\footnotesize{(0.157)}} & \mc{1}{c}{\footnotesize{(0.333)}} & \mc{1}{c}{\footnotesize{(0.373)}} & \mc{1}{c}{\footnotesize{(0.255)}} \\  

    \mc{1}{l}{\footnotesize{Hemoglobin Level (\%)}} & \mc{1}{c}{\footnotesize{Mid-30s}} & \mc{1}{c}{\footnotesize{0.328}} & \mc{1}{c}{\footnotesize{0.121}} & \mc{1}{c}{\footnotesize{0.383}} & \mc{1}{c}{\footnotesize{0.130}} & \mc{1}{c}{\footnotesize{0.335}} & \mc{1}{c}{\footnotesize{0.307}} & \mc{1}{c}{\footnotesize{0.097}} & \mc{1}{c}{\footnotesize{0.230}} \\  

     &  & \mc{1}{c}{\footnotesize{(0.784)}} & \mc{1}{c}{\footnotesize{(0.706)}} & \mc{1}{c}{\footnotesize{(0.804)}} & \mc{1}{c}{\footnotesize{(0.627)}} & \mc{1}{c}{\footnotesize{(0.647)}} & \mc{1}{c}{\footnotesize{(0.765)}} & \mc{1}{c}{\footnotesize{(0.627)}} & \mc{1}{c}{\footnotesize{(0.608)}} \\  

    \mc{1}{l}{\footnotesize{Prediabetes}} & \mc{1}{c}{\footnotesize{Mid-30s}} & \mc{1}{c}{\footnotesize{-0.120}} & \mc{1}{c}{\footnotesize{-0.182}} & \mc{1}{c}{\footnotesize{-0.171}} & \mc{1}{c}{\footnotesize{-0.236}} & \mc{1}{c}{\footnotesize{-0.192}} & \mc{1}{c}{\footnotesize{-0.100}} & \mc{1}{c}{\footnotesize{-0.174}} & \mc{1}{c}{\footnotesize{-0.103}} \\  

     &  & \mc{1}{c}{\footnotesize{(0.216)}} & \mc{1}{c}{\footnotesize{(0.157)}} & \mc{1}{c}{\footnotesize{(0.255)}} & \mc{1}{c}{\footnotesize{(0.196)}} & \mc{1}{c}{\footnotesize{(0.176)}} & \mc{1}{c}{\footnotesize{(0.255)}} & \mc{1}{c}{\footnotesize{(0.196)}} & \mc{1}{c}{\footnotesize{(0.235)}} \\  

    \mc{1}{l}{\footnotesize{Diabetes}} & \mc{1}{c}{\footnotesize{Mid-30s}} & \mc{1}{c}{\footnotesize{0.080}} & \mc{1}{c}{\footnotesize{0.041}} & \mc{1}{c}{\footnotesize{0.080}} & \mc{1}{c}{\footnotesize{0.020}} & \mc{1}{c}{\footnotesize{0.064}} & \mc{1}{c}{\footnotesize{0.080}} & \mc{1}{c}{\footnotesize{0.042}} & \mc{1}{c}{\footnotesize{0.056}} \\  

     &  & \mc{1}{c}{\footnotesize{(0.804)}} & \mc{1}{c}{\footnotesize{(0.725)}} & \mc{1}{c}{\footnotesize{(0.804)}} & \mc{1}{c}{\footnotesize{(0.549)}} & \mc{1}{c}{\footnotesize{(0.549)}} & \mc{1}{c}{\footnotesize{(0.804)}} & \mc{1}{c}{\footnotesize{(0.745)}} & \mc{1}{c}{\footnotesize{(0.549)}} \\  

    \mc{1}{l}{\footnotesize{Vitamin D Deficiency}} & \mc{1}{c}{\footnotesize{Mid-30s}} & \mc{1}{c}{\footnotesize{-0.280}} & \mc{1}{c}{\footnotesize{-0.164}} & \mc{1}{c}{\footnotesize{-0.480}} & \mc{1}{c}{\footnotesize{-0.206}} & \mc{1}{c}{\footnotesize{-0.513}} & \mc{1}{c}{\footnotesize{-0.202}} & \mc{1}{c}{\footnotesize{-0.112}} & \mc{1}{c}{\footnotesize{-0.255}} \\  

     &  & \mc{1}{c}{\footnotesize{\textbf{(0.000)}}} & \mc{1}{c}{\footnotesize{(0.137)}} & \mc{1}{c}{\footnotesize{\textbf{(0.000)}}} & \mc{1}{c}{\footnotesize{\textbf{(0.078)}}} & \mc{1}{c}{\footnotesize{\textbf{(0.000)}}} & \mc{1}{c}{\footnotesize{\textbf{(0.078)}}} & \mc{1}{c}{\footnotesize{(0.216)}} & \mc{1}{c}{\footnotesize{\textbf{(0.059)}}} \\  
    
        \mc{1}{l}{\footnotesize{Self-reported drug user}} & \mc{1}{c}{\footnotesize{Mid-30s}} & \mc{1}{c}{\footnotesize{-0.357}} & \mc{1}{c}{\footnotesize{-0.298}} & \mc{1}{c}{\footnotesize{-0.691}} & \mc{1}{c}{\footnotesize{-0.595}} & \mc{1}{c}{\footnotesize{-0.837}} & \mc{1}{c}{\footnotesize{-0.191}} & \mc{1}{c}{\footnotesize{-0.164}} & \mc{1}{c}{\footnotesize{-0.295}} \\  

     &  & \mc{1}{c}{\footnotesize{\textbf{(0.000)}}} & \mc{1}{c}{\footnotesize{\textbf{(0.059)}}} & \mc{1}{c}{\footnotesize{\textbf{(0.000)}}} & \mc{1}{c}{\footnotesize{\textbf{(0.000)}}} & \mc{1}{c}{\footnotesize{\textbf{(0.000)}}} & \mc{1}{c}{\footnotesize{\textbf{(0.098)}}} & \mc{1}{c}{\footnotesize{(0.235)}} & \mc{1}{c}{\footnotesize{\textbf{(0.078)}}} \\  
     
         \mc{1}{l}{\footnotesize{Severe Obesity}} & \mc{1}{c}{\footnotesize{Mid-30s}} & \mc{1}{c}{\footnotesize{-0.102}} & \mc{1}{c}{\footnotesize{-0.141}} & \mc{1}{c}{\footnotesize{-0.138}} & \mc{1}{c}{\footnotesize{-0.029}} & \mc{1}{c}{\footnotesize{-0.284}} & \mc{1}{c}{\footnotesize{-0.083}} & \mc{1}{c}{\footnotesize{-0.129}} & \mc{1}{c}{\footnotesize{-0.080}} \\  

     &  & \mc{1}{c}{\footnotesize{\textbf{(0.098)}}} & \mc{1}{c}{\footnotesize{(0.176)}} & \mc{1}{c}{\footnotesize{(0.216)}} & \mc{1}{c}{\footnotesize{(0.373)}} & \mc{1}{c}{\footnotesize{\textbf{(0.059)}}} & \mc{1}{c}{\footnotesize{(0.235)}} & \mc{1}{c}{\footnotesize{(0.235)}} & \mc{1}{c}{\footnotesize{(0.216)}} \\  

  \bottomrule
  \end{tabular}
    \begin{tablenotes}
    \footnotesize
    \item 
Note: This table displays various estimates of the treatment effect of ABC/CARE's center-based care.
Column (1) displays the ITT, without accounting for any controls.
Column (2) displays the ITT conditioning on vector of controls, $X$, consisting of Apgar scores 1 minute and 5 minutes after birth, the HRI index, maternal IQ,
an indicator for having a grandmother residing in the same county, and an index for the number
of relatives living in the same household. We also apply IPW weights, $W$, to account for attrition.
Columns (3)--(4) are analogous to columns (1)--(2), but we restrict the control sample to subjects
who did not enroll in any alternative care.
Column (5) displays the matching estimate, where we use the Mahalanobis metric and Epanechnikov kernel
to match on controls $X$ listed above, and restrict the control sample to subjects who did not enroll
in any alternative care. Additionally, we apply IPW weights, $W$.
Columns (6)--(8) are analogous to Columns (3)--(5), except we restrict the control sample to subjects
who did enroll in alternative care.  
Numbers in parentheses represent the $p$-value from a single hypothesis test, and are obtained from 
the empirical bootstrap distribution generated by 200 resamples of the original data. 
Bold $p$-values indicate significance at the 10\% level.
Blank point estimates indicate that we are unable to obtain estimates due to a lack of support in the data. 

    \end{tablenotes}
  \end{threeparttable}

\end{sidewaystable}
\end{center}

\begin{center}
	  \begin{tabular}{cccccccccc}
  \toprule

    \scriptsize{Variable} & \scriptsize{Age} & \scriptsize{(1)} & \scriptsize{(2)} & \scriptsize{(3)} & \scriptsize{(4)} & \scriptsize{(5)} & \scriptsize{(6)} & \scriptsize{(7)} & \scriptsize{(8)} \\ 
    \midrule  

    \mc{1}{l}{\scriptsize{Cig. Smoked per day last month}} & \mc{1}{c}{\scriptsize{30}} & \mc{1}{c}{\scriptsize{0.851}} & \mc{1}{c}{\scriptsize{-2.180}} & \mc{1}{c}{\scriptsize{-0.902}} & \mc{1}{c}{\scriptsize{-6.885}} & \mc{1}{c}{\scriptsize{-3.513}} & \mc{1}{c}{\scriptsize{1.497}} & \mc{1}{c}{\scriptsize{-0.749}} & \mc{1}{c}{\scriptsize{0.171}} \\  

     &  & \mc{1}{c}{\scriptsize{(0.737)}} & \mc{1}{c}{\scriptsize{\textbf{(0.079)}}} & \mc{1}{c}{\scriptsize{(0.329)}} & \mc{1}{c}{\scriptsize{\textbf{(0.039)}}} & \mc{1}{c}{\scriptsize{(0.118)}} & \mc{1}{c}{\scriptsize{(0.763)}} & \mc{1}{c}{\scriptsize{(0.316)}} & \mc{1}{c}{\scriptsize{(0.500)}} \\  

    \mc{1}{l}{\scriptsize{Days drank alcohol last month}} & \mc{1}{c}{\scriptsize{30}} & \mc{1}{c}{\scriptsize{-1.351}} & \mc{1}{c}{\scriptsize{-5.244}} & \mc{1}{c}{\scriptsize{-0.027}} & \mc{1}{c}{\scriptsize{-4.387}} & \mc{1}{c}{\scriptsize{-3.381}} & \mc{1}{c}{\scriptsize{-1.839}} & \mc{1}{c}{\scriptsize{-5.790}} & \mc{1}{c}{\scriptsize{-5.533}} \\  

     &  & \mc{1}{c}{\scriptsize{(0.303)}} & \mc{1}{c}{\scriptsize{\textbf{(0.039)}}} & \mc{1}{c}{\scriptsize{(0.526)}} & \mc{1}{c}{\scriptsize{\textbf{(0.079)}}} & \mc{1}{c}{\scriptsize{\textbf{(0.000)}}} & \mc{1}{c}{\scriptsize{(0.237)}} & \mc{1}{c}{\scriptsize{\textbf{(0.039)}}} & \mc{1}{c}{\scriptsize{\textbf{(0.013)}}} \\  

    \mc{1}{l}{\scriptsize{Self-reported drug user}} & \mc{1}{c}{\scriptsize{Mid-30s}} & \mc{1}{c}{\scriptsize{-0.167}} & \mc{1}{c}{\scriptsize{-0.647}} & \mc{1}{c}{\scriptsize{-0.467}} & \mc{1}{c}{\scriptsize{-0.834}} & \mc{1}{c}{\scriptsize{-0.692}} & \mc{1}{c}{\scriptsize{0.048}} & \mc{1}{c}{\scriptsize{-0.248}} & \mc{1}{c}{\scriptsize{-0.236}} \\  

     &  & \mc{1}{c}{\scriptsize{(0.197)}} & \mc{1}{c}{\scriptsize{\textbf{(0.079)}}} & \mc{1}{c}{\scriptsize{\textbf{(0.039)}}} & \mc{1}{c}{\scriptsize{\textbf{(0.013)}}} & \mc{1}{c}{\scriptsize{\textbf{(0.000)}}} & \mc{1}{c}{\scriptsize{(0.487)}} & \mc{1}{c}{\scriptsize{(0.211)}} & \mc{1}{c}{\scriptsize{(0.145)}} \\  

    \mc{1}{l}{\scriptsize{Measured BMI}} & \mc{1}{c}{\scriptsize{Mid-30s}} & \mc{1}{c}{\scriptsize{8.418}} & \mc{1}{c}{\scriptsize{2.636}} & \mc{1}{c}{\scriptsize{2.907}} & \mc{1}{c}{\scriptsize{-1.389}} & \mc{1}{c}{\scriptsize{0.493}} & \mc{1}{c}{\scriptsize{12.355}} & \mc{1}{c}{\scriptsize{16.338}} & \mc{1}{c}{\scriptsize{10.936}} \\  

     &  & \mc{1}{c}{\scriptsize{(0.974)}} & \mc{1}{c}{\scriptsize{(0.618)}} & \mc{1}{c}{\scriptsize{(0.671)}} & \mc{1}{c}{\scriptsize{(0.421)}} & \mc{1}{c}{\scriptsize{(0.553)}} & \mc{1}{c}{\scriptsize{(0.987)}} & \mc{1}{c}{\scriptsize{(0.868)}} & \mc{1}{c}{\scriptsize{(0.908)}} \\  

    \mc{1}{l}{\scriptsize{Obesity}} & \mc{1}{c}{\scriptsize{Mid-30s}} & \mc{1}{c}{\scriptsize{0.250}} & \mc{1}{c}{\scriptsize{0.122}} & \mc{1}{c}{\scriptsize{-0.133}} & \mc{1}{c}{\scriptsize{-0.238}} & \mc{1}{c}{\scriptsize{-0.178}} & \mc{1}{c}{\scriptsize{0.524}} & \mc{1}{c}{\scriptsize{1.086}} & \mc{1}{c}{\scriptsize{0.574}} \\  

     &  & \mc{1}{c}{\scriptsize{(0.842)}} & \mc{1}{c}{\scriptsize{(0.461)}} & \mc{1}{c}{\scriptsize{(0.316)}} & \mc{1}{c}{\scriptsize{(0.211)}} & \mc{1}{c}{\scriptsize{(0.224)}} & \mc{1}{c}{\scriptsize{(0.974)}} & \mc{1}{c}{\scriptsize{(0.908)}} & \mc{1}{c}{\scriptsize{(0.934)}} \\  

    \mc{1}{l}{\scriptsize{Severe Obesity}} & \mc{1}{c}{\scriptsize{Mid-30s}} & \mc{1}{c}{\scriptsize{0.167}} & \mc{1}{c}{\scriptsize{-0.318}} & \mc{1}{c}{\scriptsize{-0.067}} & \mc{1}{c}{\scriptsize{-0.576}} & \mc{1}{c}{\scriptsize{-0.437}} & \mc{1}{c}{\scriptsize{0.333}} & \mc{1}{c}{\scriptsize{0.198}} & \mc{1}{c}{\scriptsize{0.097}} \\  

     &  & \mc{1}{c}{\scriptsize{(0.803)}} & \mc{1}{c}{\scriptsize{(0.158)}} & \mc{1}{c}{\scriptsize{(0.421)}} & \mc{1}{c}{\scriptsize{(0.316)}} & \mc{1}{c}{\scriptsize{(0.145)}} & \mc{1}{c}{\scriptsize{(0.855)}} & \mc{1}{c}{\scriptsize{(0.382)}} & \mc{1}{c}{\scriptsize{(0.539)}} \\  

    \mc{1}{l}{\scriptsize{Waist-hip Ratio}} & \mc{1}{c}{\scriptsize{Mid-30s}} & \mc{1}{c}{\scriptsize{0.044}} & \mc{1}{c}{\scriptsize{-0.048}} & \mc{1}{c}{\scriptsize{-0.021}} & \mc{1}{c}{\scriptsize{-0.085}} & \mc{1}{c}{\scriptsize{-0.066}} & \mc{1}{c}{\scriptsize{0.091}} & \mc{1}{c}{\scriptsize{0.147}} & \mc{1}{c}{\scriptsize{0.066}} \\  

     &  & \mc{1}{c}{\scriptsize{(0.908)}} & \mc{1}{c}{\scriptsize{(0.197)}} & \mc{1}{c}{\scriptsize{(0.263)}} & \mc{1}{c}{\scriptsize{(0.263)}} & \mc{1}{c}{\scriptsize{\textbf{(0.079)}}} & \mc{1}{c}{\scriptsize{(1.000)}} & \mc{1}{c}{\scriptsize{(0.921)}} & \mc{1}{c}{\scriptsize{(0.882)}} \\  

    \mc{1}{l}{\scriptsize{Abdominal Obesity}} & \mc{1}{c}{\scriptsize{Mid-30s}} & \mc{1}{c}{\scriptsize{0.333}} & \mc{1}{c}{\scriptsize{0.031}} & \mc{1}{c}{\scriptsize{-0.167}} & \mc{1}{c}{\scriptsize{-0.238}} & \mc{1}{c}{\scriptsize{-0.279}} & \mc{1}{c}{\scriptsize{0.691}} & \mc{1}{c}{\scriptsize{1.086}} & \mc{1}{c}{\scriptsize{0.574}} \\  

     &  & \mc{1}{c}{\scriptsize{(0.934)}} & \mc{1}{c}{\scriptsize{(0.368)}} & \mc{1}{c}{\scriptsize{\textbf{(0.079)}}} & \mc{1}{c}{\scriptsize{(0.118)}} & \mc{1}{c}{\scriptsize{\textbf{(0.066)}}} & \mc{1}{c}{\scriptsize{(0.987)}} & \mc{1}{c}{\scriptsize{(0.908)}} & \mc{1}{c}{\scriptsize{(0.934)}} \\  

    \mc{1}{l}{\scriptsize{Framingham Risk Score}} & \mc{1}{c}{\scriptsize{Mid-30s}} & \mc{1}{c}{\scriptsize{0.901}} & \mc{1}{c}{\scriptsize{-0.194}} & \mc{1}{c}{\scriptsize{-0.948}} & \mc{1}{c}{\scriptsize{-1.916}} & \mc{1}{c}{\scriptsize{-1.629}} & \mc{1}{c}{\scriptsize{2.222}} & \mc{1}{c}{\scriptsize{3.515}} & \mc{1}{c}{\scriptsize{2.303}} \\  

     &  & \mc{1}{c}{\scriptsize{(0.882)}} & \mc{1}{c}{\scriptsize{(0.355)}} & \mc{1}{c}{\scriptsize{(0.145)}} & \mc{1}{c}{\scriptsize{\textbf{(0.026)}}} & \mc{1}{c}{\scriptsize{\textbf{(0.079)}}} & \mc{1}{c}{\scriptsize{(1.000)}} & \mc{1}{c}{\scriptsize{(0.947)}} & \mc{1}{c}{\scriptsize{(0.974)}} \\  

  \bottomrule
  \end{tabular}
\end{center}

\begin{center}
	\begin{table}[H]
\captionsetup{singlelinecheck=false,justification=centering}
\caption{CARE Average Treatment Effects, Females \\ Education, Employment, and Crime \label{tab:ate_female_main1}}

  \begin{threeparttable}
  \begin{tabular}{cccccccccc}
  \hline\hline

     &  & \scriptsize{(1)} & \scriptsize{(2)} & \scriptsize{(3)} & \scriptsize{(4)} & \scriptsize{(5)} & \scriptsize{(6)} & \scriptsize{(7)} & \scriptsize{(8)} \\  

     &  &  &  & \mc{3}{c}{\scriptsize{$P=0$}} & \mc{3}{c}{\scriptsize{$P=1$}} \\ 
    \cmidrule(lr){5-7} \cmidrule(lr){8-10} 

    \scriptsize{Variable} & \scriptsize{Age} & \scriptsize{ITT} & \scriptsize{ITT$|X,W$} & \scriptsize{ITT} & \scriptsize{ITT$|X,W$} & \scriptsize{KE$|X,W$} & \scriptsize{ITT} & \scriptsize{ITT$|X,W$} & \scriptsize{KE$|X,W$} \\ 
    \hline  

    \mc{1}{l}{\scriptsize{Std. IQ Test}} & \mc{1}{c}{\scriptsize{12}} & \mc{1}{c}{\scriptsize{-4.978}} & \mc{1}{c}{\scriptsize{-4.134}} & \mc{1}{c}{\scriptsize{-5.200}} & \mc{1}{c}{\scriptsize{-2.311}} & \mc{1}{c}{\scriptsize{-2.849}} & \mc{1}{c}{\scriptsize{-4.800}} & \mc{1}{c}{\scriptsize{-0.735}} & \mc{1}{c}{\scriptsize{-4.260}} \\  

     &  & \mc{1}{c}{\scriptsize{(0.863)}} & \mc{1}{c}{\scriptsize{(0.765)}} & \mc{1}{c}{\scriptsize{(0.843)}} & \mc{1}{c}{\scriptsize{(0.510)}} & \mc{1}{c}{\scriptsize{(0.765)}} & \mc{1}{c}{\scriptsize{(0.804)}} & \mc{1}{c}{\scriptsize{(0.549)}} & \mc{1}{c}{\scriptsize{(0.784)}} \\  

    \mc{1}{l}{\scriptsize{Std. Achv.  Test}} & \mc{1}{c}{\scriptsize{12}} & \mc{1}{c}{\scriptsize{1.064}} & \mc{1}{c}{\scriptsize{1.443}} & \mc{1}{c}{\scriptsize{8.170}} & \mc{1}{c}{\scriptsize{5.870}} & \mc{1}{c}{\scriptsize{9.096}} & \mc{1}{c}{\scriptsize{-4.620}} & \mc{1}{c}{\scriptsize{-4.080}} & \mc{1}{c}{\scriptsize{-6.396}} \\  

     &  & \mc{1}{c}{\scriptsize{(0.373)}} & \mc{1}{c}{\scriptsize{(0.451)}} & \mc{1}{c}{\scriptsize{\textbf{(0.020)}}} & \mc{1}{c}{\scriptsize{(0.373)}} & \mc{1}{c}{\scriptsize{\textbf{(0.020)}}} & \mc{1}{c}{\scriptsize{(0.745)}} & \mc{1}{c}{\scriptsize{(0.667)}} & \mc{1}{c}{\scriptsize{(0.882)}} \\  

    \mc{1}{l}{\scriptsize{Graduated High School}} & \mc{1}{c}{\scriptsize{30}} & \mc{1}{c}{\scriptsize{0.256}} & \mc{1}{c}{\scriptsize{0.005}} & \mc{1}{c}{\scriptsize{0.450}} & \mc{1}{c}{\scriptsize{0.291}} & \mc{1}{c}{\scriptsize{0.472}} & \mc{1}{c}{\scriptsize{0.100}} & \mc{1}{c}{\scriptsize{-0.519}} & \mc{1}{c}{\scriptsize{-0.089}} \\  

     &  & \mc{1}{c}{\scriptsize{(0.196)}} & \mc{1}{c}{\scriptsize{(0.490)}} & \mc{1}{c}{\scriptsize{\textbf{(0.020)}}} & \mc{1}{c}{\scriptsize{(0.157)}} & \mc{1}{c}{\scriptsize{\textbf{(0.020)}}} & \mc{1}{c}{\scriptsize{(0.333)}} & \mc{1}{c}{\scriptsize{(0.627)}} & \mc{1}{c}{\scriptsize{(0.588)}} \\  

    \mc{1}{l}{\scriptsize{Years of Edu.}} & \mc{1}{c}{\scriptsize{30}} & \mc{1}{c}{\scriptsize{3.344}} & \mc{1}{c}{\scriptsize{2.284}} & \mc{1}{c}{\scriptsize{4.900}} & \mc{1}{c}{\scriptsize{4.932}} & \mc{1}{c}{\scriptsize{5.449}} & \mc{1}{c}{\scriptsize{2.100}} & \mc{1}{c}{\scriptsize{-1.391}} & \mc{1}{c}{\scriptsize{1.155}} \\  

     &  & \mc{1}{c}{\scriptsize{\textbf{(0.020)}}} & \mc{1}{c}{\scriptsize{\textbf{(0.078)}}} & \mc{1}{c}{\scriptsize{\textbf{(0.000)}}} & \mc{1}{c}{\scriptsize{\textbf{(0.039)}}} & \mc{1}{c}{\scriptsize{\textbf{(0.000)}}} & \mc{1}{c}{\scriptsize{(0.176)}} & \mc{1}{c}{\scriptsize{(0.627)}} & \mc{1}{c}{\scriptsize{(0.275)}} \\  

    \mc{1}{l}{\scriptsize{Labor Income}} & \mc{1}{c}{\scriptsize{30}} & \mc{1}{c}{\scriptsize{3,120}} & \mc{1}{c}{\scriptsize{-3,166}} & \mc{1}{c}{\scriptsize{5,162}} & \mc{1}{c}{\scriptsize{3,547}} & \mc{1}{c}{\scriptsize{4,603}} & \mc{1}{c}{\scriptsize{1,487}} & \mc{1}{c}{\scriptsize{-11,424}} & \mc{1}{c}{\scriptsize{-2,314}} \\  

     &  & \mc{1}{c}{\scriptsize{(0.196)}} & \mc{1}{c}{\scriptsize{(0.843)}} & \mc{1}{c}{\scriptsize{\textbf{(0.098)}}} & \mc{1}{c}{\scriptsize{\textbf{(0.098)}}} & \mc{1}{c}{\scriptsize{\textbf{(0.098)}}} & \mc{1}{c}{\scriptsize{(0.314)}} & \mc{1}{c}{\scriptsize{(0.765)}} & \mc{1}{c}{\scriptsize{(0.686)}} \\  

    \mc{1}{l}{\scriptsize{Public-Transfer Income}} & \mc{1}{c}{\scriptsize{30}} & \mc{1}{c}{\scriptsize{-1,374}} & \mc{1}{c}{\scriptsize{-1,212}} & \mc{1}{c}{\scriptsize{-2,161}} & \mc{1}{c}{\scriptsize{-2,312}} & \mc{1}{c}{\scriptsize{-2,190}} & \mc{1}{c}{\scriptsize{-745}} & \mc{1}{c}{\scriptsize{845}} & \mc{1}{c}{\scriptsize{-525}} \\  

     &  & \mc{1}{c}{\scriptsize{\textbf{(0.078)}}} & \mc{1}{c}{\scriptsize{(0.137)}} & \mc{1}{c}{\scriptsize{\textbf{(0.000)}}} & \mc{1}{c}{\scriptsize{\textbf{(0.020)}}} & \mc{1}{c}{\scriptsize{\textbf{(0.000)}}} & \mc{1}{c}{\scriptsize{(0.235)}} & \mc{1}{c}{\scriptsize{(0.647)}} & \mc{1}{c}{\scriptsize{(0.255)}} \\  

    \mc{1}{l}{\scriptsize{Employed}} & \mc{1}{c}{\scriptsize{30}} & \mc{1}{c}{\scriptsize{0.133}} & \mc{1}{c}{\scriptsize{-0.224}} & \mc{1}{c}{\scriptsize{0.300}} & \mc{1}{c}{\scriptsize{0.078}} & \mc{1}{c}{\scriptsize{0.287}} &  & \mc{1}{c}{\scriptsize{-0.462}} & \mc{1}{c}{\scriptsize{-0.184}} \\  

     &  & \mc{1}{c}{\scriptsize{(0.275)}} & \mc{1}{c}{\scriptsize{(0.706)}} & \mc{1}{c}{\scriptsize{(0.137)}} & \mc{1}{c}{\scriptsize{(0.412)}} & \mc{1}{c}{\scriptsize{(0.196)}} &  & \mc{1}{c}{\scriptsize{(0.549)}} & \mc{1}{c}{\scriptsize{(0.667)}} \\  

    \mc{1}{l}{\scriptsize{Total Felony Arrests}} & \mc{1}{c}{\scriptsize{Mid-30s}} & \mc{1}{c}{\scriptsize{-0.014}} & \mc{1}{c}{\scriptsize{0.104}} & \mc{1}{c}{\scriptsize{-0.222}} & \mc{1}{c}{\scriptsize{0.167}} & \mc{1}{c}{\scriptsize{-0.212}} & \mc{1}{c}{\scriptsize{0.111}} & \mc{1}{c}{\scriptsize{0.268}} & \mc{1}{c}{\scriptsize{0.154}} \\  

     &  & \mc{1}{c}{\scriptsize{(0.392)}} & \mc{1}{c}{\scriptsize{(0.451)}} & \mc{1}{c}{\scriptsize{(0.196)}} & \mc{1}{c}{\scriptsize{(0.314)}} & \mc{1}{c}{\scriptsize{(0.216)}} & \mc{1}{c}{\scriptsize{(0.529)}} & \mc{1}{c}{\scriptsize{(0.275)}} & \mc{1}{c}{\scriptsize{(0.529)}} \\  

    \mc{1}{l}{\scriptsize{Total Misdemeanor Arrests}} & \mc{1}{c}{\scriptsize{Mid-30s}} & \mc{1}{c}{\scriptsize{0.625}} & \mc{1}{c}{\scriptsize{2.495}} & \mc{1}{c}{\scriptsize{1.000}} & \mc{1}{c}{\scriptsize{2.889}} & \mc{1}{c}{\scriptsize{1.175}} & \mc{1}{c}{\scriptsize{0.400}} & \mc{1}{c}{\scriptsize{2.588}} & \mc{1}{c}{\scriptsize{0.713}} \\  

     &  & \mc{1}{c}{\scriptsize{(0.667)}} & \mc{1}{c}{\scriptsize{(0.804)}} & \mc{1}{c}{\scriptsize{(0.725)}} & \mc{1}{c}{\scriptsize{(0.529)}} & \mc{1}{c}{\scriptsize{(0.706)}} & \mc{1}{c}{\scriptsize{(0.569)}} & \mc{1}{c}{\scriptsize{(0.647)}} & \mc{1}{c}{\scriptsize{(0.647)}} \\  

    \mc{1}{l}{\scriptsize{Total Years Incarcerated}} & \mc{1}{c}{\scriptsize{30}} & \mc{1}{c}{\scriptsize{-0.111}} & \mc{1}{c}{\scriptsize{0.011}} &  &  &  & \mc{1}{c}{\scriptsize{-0.200}} & \mc{1}{c}{\scriptsize{-0.118}} & \mc{1}{c}{\scriptsize{-0.059}} \\  

     &  & \mc{1}{c}{\scriptsize{\textbf{(0.059)}}} & \mc{1}{c}{\scriptsize{(0.314)}} &  &  &  & \mc{1}{c}{\scriptsize{\textbf{(0.059)}}} & \mc{1}{c}{\scriptsize{\textbf{(0.078)}}} & \mc{1}{c}{\scriptsize{(0.157)}} \\  

  \hline\hline
  \end{tabular}
    \begin{tablenotes}
    \scriptsize
    \item 
Note: This table displays various estimates of the treatment effect of CARE's family education program.
Column (1) displays the ITT, without accounting for any controls.
Column (2) displays the ITT conditioning on vector of controls, $X$, consisting of APGAR scores 1 
minute after birth, an indicator for the subject being born prematurely, and an indicator for the 
father being home at baseline. We also apply IPW weights, $W$, to account for attrition.
Columns (3)--(4) are analogous to columns (1)--(2), but we restrict the control sample to subjects
who did not enroll in any alternative care.
Column (5) displys the matching estimate, where we use the Mahalanobis metric and Epanechnikov kernel
to match on controls $X$ listed above, and restrict the control sample to subjects who did not enroll
in any alternative care. Additionally, we apply IPW weights, $W$.
Columns (6)--(8) are analogous to Columns (3)--(5), except we restrict the control sample to subejcts
who did enroll in alternative care. 
Numbers in parentheses represent the $p$-value from a single hypothesis test, and are obtained from 
the empirical bootstrap distribution generated by 200 resamples of the original data. 
Bold $p$-values indicate significance at the 10\% level.
Blank point estimates indicate that we are unable to obtain estimates due to a lack of support in the data. 

    \end{tablenotes}
  \end{threeparttable}

\end{table}
\end{center}

\begin{center}
	  \begin{tabular}{cccccccccc}
  \toprule

    \scriptsize{Variable} & \scriptsize{Age} & \scriptsize{(1)} & \scriptsize{(2)} & \scriptsize{(3)} & \scriptsize{(4)} & \scriptsize{(5)} & \scriptsize{(6)} & \scriptsize{(7)} & \scriptsize{(8)} \\ 
    \midrule  

    \mc{1}{l}{\scriptsize{Systolic Blood Pressure (mm Hg)}} & \mc{1}{c}{\scriptsize{Mid-30s}} & \mc{1}{c}{\scriptsize{7.862}} & \mc{1}{c}{\scriptsize{13.240}} & \mc{1}{c}{\scriptsize{5.086}} & \mc{1}{c}{\scriptsize{31.362}} & \mc{1}{c}{\scriptsize{14.345}} & \mc{1}{c}{\scriptsize{10.022}} & \mc{1}{c}{\scriptsize{44.099}} & \mc{1}{c}{\scriptsize{12.942}} \\  

     &  & \mc{1}{c}{\scriptsize{(0.684)}} & \mc{1}{c}{\scriptsize{(0.750)}} & \mc{1}{c}{\scriptsize{(0.632)}} & \mc{1}{c}{\scriptsize{(0.592)}} & \mc{1}{c}{\scriptsize{(0.592)}} & \mc{1}{c}{\scriptsize{(0.737)}} & \mc{1}{c}{\scriptsize{(0.816)}} & \mc{1}{c}{\scriptsize{(0.697)}} \\  

    \mc{1}{l}{\scriptsize{Diastolic Blood Pressure (mm Hg)}} & \mc{1}{c}{\scriptsize{Mid-30s}} & \mc{1}{c}{\scriptsize{13.675}} & \mc{1}{c}{\scriptsize{17.523}} & \mc{1}{c}{\scriptsize{12.086}} & \mc{1}{c}{\scriptsize{28.900}} & \mc{1}{c}{\scriptsize{19.209}} & \mc{1}{c}{\scriptsize{14.911}} & \mc{1}{c}{\scriptsize{30.178}} & \mc{1}{c}{\scriptsize{19.906}} \\  

     &  & \mc{1}{c}{\scriptsize{(0.921)}} & \mc{1}{c}{\scriptsize{(0.855)}} & \mc{1}{c}{\scriptsize{(0.908)}} & \mc{1}{c}{\scriptsize{(0.618)}} & \mc{1}{c}{\scriptsize{(0.921)}} & \mc{1}{c}{\scriptsize{(0.947)}} & \mc{1}{c}{\scriptsize{(0.776)}} & \mc{1}{c}{\scriptsize{(0.947)}} \\  

    \mc{1}{l}{\scriptsize{Prehypertension}} & \mc{1}{c}{\scriptsize{Mid-30s}} & \mc{1}{c}{\scriptsize{0.125}} & \mc{1}{c}{\scriptsize{0.037}} &  &  &  & \mc{1}{c}{\scriptsize{0.222}} & \mc{1}{c}{\scriptsize{-0.018}} & \mc{1}{c}{\scriptsize{0.139}} \\  

     &  & \mc{1}{c}{\scriptsize{(0.816)}} & \mc{1}{c}{\scriptsize{(0.434)}} &  &  &  & \mc{1}{c}{\scriptsize{(0.842)}} & \mc{1}{c}{\scriptsize{(0.329)}} & \mc{1}{c}{\scriptsize{(0.658)}} \\  

    \mc{1}{l}{\scriptsize{Hypertension}} & \mc{1}{c}{\scriptsize{Mid-30s}} & \mc{1}{c}{\scriptsize{0.487}} & \mc{1}{c}{\scriptsize{0.225}} & \mc{1}{c}{\scriptsize{0.371}} & \mc{1}{c}{\scriptsize{0.604}} & \mc{1}{c}{\scriptsize{0.118}} & \mc{1}{c}{\scriptsize{0.578}} & \mc{1}{c}{\scriptsize{0.398}} & \mc{1}{c}{\scriptsize{0.681}} \\  

     &  & \mc{1}{c}{\scriptsize{(1.000)}} & \mc{1}{c}{\scriptsize{(0.776)}} & \mc{1}{c}{\scriptsize{(0.908)}} & \mc{1}{c}{\scriptsize{(0.526)}} & \mc{1}{c}{\scriptsize{(0.526)}} & \mc{1}{c}{\scriptsize{(0.987)}} & \mc{1}{c}{\scriptsize{(0.671)}} & \mc{1}{c}{\scriptsize{(1.000)}} \\  

    \mc{1}{l}{\scriptsize{High-Density Lipoprotein Chol. (mg/dL)}} & \mc{1}{c}{\scriptsize{Mid-30s}} & \mc{1}{c}{\scriptsize{-5.175}} & \mc{1}{c}{\scriptsize{-3.186}} & \mc{1}{c}{\scriptsize{-3.657}} & \mc{1}{c}{\scriptsize{-7.780}} & \mc{1}{c}{\scriptsize{-0.646}} & \mc{1}{c}{\scriptsize{-6.356}} & \mc{1}{c}{\scriptsize{-1.267}} & \mc{1}{c}{\scriptsize{-7.347}} \\  

     &  & \mc{1}{c}{\scriptsize{(0.816)}} & \mc{1}{c}{\scriptsize{(0.605)}} & \mc{1}{c}{\scriptsize{(0.697)}} & \mc{1}{c}{\scriptsize{(0.474)}} & \mc{1}{c}{\scriptsize{(0.697)}} & \mc{1}{c}{\scriptsize{(0.842)}} & \mc{1}{c}{\scriptsize{(0.395)}} & \mc{1}{c}{\scriptsize{(0.895)}} \\  

    \mc{1}{l}{\scriptsize{Dyslipidemia}} & \mc{1}{c}{\scriptsize{Mid-30s}} & \mc{1}{c}{\scriptsize{0.138}} & \mc{1}{c}{\scriptsize{0.287}} & \mc{1}{c}{\scriptsize{0.200}} & \mc{1}{c}{\scriptsize{0.323}} & \mc{1}{c}{\scriptsize{0.283}} & \mc{1}{c}{\scriptsize{0.089}} & \mc{1}{c}{\scriptsize{-0.104}} & \mc{1}{c}{\scriptsize{0.297}} \\  

     &  & \mc{1}{c}{\scriptsize{(0.684)}} & \mc{1}{c}{\scriptsize{(0.513)}} & \mc{1}{c}{\scriptsize{(0.605)}} & \mc{1}{c}{\scriptsize{(0.158)}} & \mc{1}{c}{\scriptsize{(0.303)}} & \mc{1}{c}{\scriptsize{(0.513)}} & \mc{1}{c}{\scriptsize{(0.132)}} & \mc{1}{c}{\scriptsize{(0.513)}} \\  

    \mc{1}{l}{\scriptsize{Hemoglobin Level (\%)}} & \mc{1}{c}{\scriptsize{Mid-30s}} & \mc{1}{c}{\scriptsize{0.266}} & \mc{1}{c}{\scriptsize{0.114}} & \mc{1}{c}{\scriptsize{0.103}} & \mc{1}{c}{\scriptsize{0.093}} & \mc{1}{c}{\scriptsize{-0.242}} & \mc{1}{c}{\scriptsize{0.393}} & \mc{1}{c}{\scriptsize{0.182}} & \mc{1}{c}{\scriptsize{0.623}} \\  

     &  & \mc{1}{c}{\scriptsize{(0.842)}} & \mc{1}{c}{\scriptsize{(0.658)}} & \mc{1}{c}{\scriptsize{(0.592)}} & \mc{1}{c}{\scriptsize{(0.421)}} & \mc{1}{c}{\scriptsize{(0.224)}} & \mc{1}{c}{\scriptsize{(0.961)}} & \mc{1}{c}{\scriptsize{(0.566)}} & \mc{1}{c}{\scriptsize{(0.987)}} \\  

    \mc{1}{l}{\scriptsize{Prediabetes}} & \mc{1}{c}{\scriptsize{Mid-30s}} & \mc{1}{c}{\scriptsize{0.350}} & \mc{1}{c}{\scriptsize{-0.057}} & \mc{1}{c}{\scriptsize{0.457}} & \mc{1}{c}{\scriptsize{0.250}} & \mc{1}{c}{\scriptsize{-0.060}} & \mc{1}{c}{\scriptsize{0.267}} & \mc{1}{c}{\scriptsize{-0.455}} & \mc{1}{c}{\scriptsize{0.329}} \\  

     &  & \mc{1}{c}{\scriptsize{(0.908)}} & \mc{1}{c}{\scriptsize{(0.447)}} & \mc{1}{c}{\scriptsize{(0.921)}} & \mc{1}{c}{\scriptsize{(0.316)}} & \mc{1}{c}{\scriptsize{(0.474)}} & \mc{1}{c}{\scriptsize{(0.776)}} & \mc{1}{c}{\scriptsize{(0.211)}} & \mc{1}{c}{\scriptsize{(0.737)}} \\  

    \mc{1}{l}{\scriptsize{Diabetes}} & \mc{1}{c}{\scriptsize{Mid-30s}} &  &  &  &  &  &  &  &  \\  

     &  &  &  &  &  &  &  &  &  \\  

    \mc{1}{l}{\scriptsize{Vitamin D Deficiency}} & \mc{1}{c}{\scriptsize{Mid-30s}} & \mc{1}{c}{\scriptsize{-0.075}} & \mc{1}{c}{\scriptsize{-0.085}} & \mc{1}{c}{\scriptsize{-0.057}} & \mc{1}{c}{\scriptsize{-0.015}} & \mc{1}{c}{\scriptsize{-0.063}} & \mc{1}{c}{\scriptsize{-0.089}} & \mc{1}{c}{\scriptsize{-0.212}} & \mc{1}{c}{\scriptsize{-0.199}} \\  

     &  & \mc{1}{c}{\scriptsize{(0.368)}} & \mc{1}{c}{\scriptsize{(0.368)}} & \mc{1}{c}{\scriptsize{(0.329)}} & \mc{1}{c}{\scriptsize{(0.250)}} & \mc{1}{c}{\scriptsize{(0.276)}} & \mc{1}{c}{\scriptsize{(0.303)}} & \mc{1}{c}{\scriptsize{(0.171)}} & \mc{1}{c}{\scriptsize{(0.118)}} \\  

  \bottomrule
  \end{tabular}
\end{center}

\begin{center}
	  \begin{tabular}{cccccccccc}
  \toprule

    \scriptsize{Variable} & \scriptsize{Age} & \scriptsize{(1)} & \scriptsize{(2)} & \scriptsize{(3)} & \scriptsize{(4)} & \scriptsize{(5)} & \scriptsize{(6)} & \scriptsize{(7)} & \scriptsize{(8)} \\ 
    \midrule  

    \mc{1}{l}{\scriptsize{Cig. Smoked per day last month}} & \mc{1}{c}{\scriptsize{30}} & \mc{1}{c}{\scriptsize{-0.319}} & \mc{1}{c}{\scriptsize{-0.340}} & \mc{1}{c}{\scriptsize{-2.140}} & \mc{1}{c}{\scriptsize{-1.592}} & \mc{1}{c}{\scriptsize{-1.788}} & \mc{1}{c}{\scriptsize{-0.163}} & \mc{1}{c}{\scriptsize{-0.184}} & \mc{1}{c}{\scriptsize{-0.371}} \\  

     &  & \mc{1}{c}{\scriptsize{(0.408)}} & \mc{1}{c}{\scriptsize{(0.355)}} & \mc{1}{c}{\scriptsize{(0.250)}} & \mc{1}{c}{\scriptsize{(0.368)}} & \mc{1}{c}{\scriptsize{(0.276)}} & \mc{1}{c}{\scriptsize{(0.421)}} & \mc{1}{c}{\scriptsize{(0.382)}} & \mc{1}{c}{\scriptsize{(0.329)}} \\  

    \mc{1}{l}{\scriptsize{Days drank alcohol last month}} & \mc{1}{c}{\scriptsize{30}} & \mc{1}{c}{\scriptsize{-0.356}} & \mc{1}{c}{\scriptsize{1.389}} & \mc{1}{c}{\scriptsize{-0.820}} & \mc{1}{c}{\scriptsize{-2.538}} & \mc{1}{c}{\scriptsize{-0.075}} & \mc{1}{c}{\scriptsize{-0.411}} & \mc{1}{c}{\scriptsize{1.869}} & \mc{1}{c}{\scriptsize{0.510}} \\  

     &  & \mc{1}{c}{\scriptsize{(0.382)}} & \mc{1}{c}{\scriptsize{(0.763)}} & \mc{1}{c}{\scriptsize{(0.368)}} & \mc{1}{c}{\scriptsize{(0.250)}} & \mc{1}{c}{\scriptsize{(0.500)}} & \mc{1}{c}{\scriptsize{(0.355)}} & \mc{1}{c}{\scriptsize{(0.803)}} & \mc{1}{c}{\scriptsize{(0.539)}} \\  

    \mc{1}{l}{\scriptsize{Self-reported drug user}} & \mc{1}{c}{\scriptsize{Mid-30s}} & \mc{1}{c}{\scriptsize{0.051}} & \mc{1}{c}{\scriptsize{0.056}} & \mc{1}{c}{\scriptsize{-0.389}} & \mc{1}{c}{\scriptsize{-0.408}} & \mc{1}{c}{\scriptsize{-0.376}} & \mc{1}{c}{\scriptsize{0.120}} & \mc{1}{c}{\scriptsize{0.207}} & \mc{1}{c}{\scriptsize{0.121}} \\  

     &  & \mc{1}{c}{\scriptsize{(0.618)}} & \mc{1}{c}{\scriptsize{(0.618)}} & \mc{1}{c}{\scriptsize{\textbf{(0.092)}}} & \mc{1}{c}{\scriptsize{(0.237)}} & \mc{1}{c}{\scriptsize{(0.158)}} & \mc{1}{c}{\scriptsize{(0.803)}} & \mc{1}{c}{\scriptsize{(0.882)}} & \mc{1}{c}{\scriptsize{(0.842)}} \\  

    \mc{1}{l}{\scriptsize{Measured BMI}} & \mc{1}{c}{\scriptsize{Mid-30s}} & \mc{1}{c}{\scriptsize{1.785}} & \mc{1}{c}{\scriptsize{5.941}} & \mc{1}{c}{\scriptsize{-0.577}} & \mc{1}{c}{\scriptsize{-8.894}} & \mc{1}{c}{\scriptsize{1.354}} & \mc{1}{c}{\scriptsize{2.158}} & \mc{1}{c}{\scriptsize{9.068}} & \mc{1}{c}{\scriptsize{5.631}} \\  

     &  & \mc{1}{c}{\scriptsize{(0.724)}} & \mc{1}{c}{\scriptsize{(0.921)}} & \mc{1}{c}{\scriptsize{(0.395)}} & \mc{1}{c}{\scriptsize{(0.184)}} & \mc{1}{c}{\scriptsize{(0.553)}} & \mc{1}{c}{\scriptsize{(0.776)}} & \mc{1}{c}{\scriptsize{(0.947)}} & \mc{1}{c}{\scriptsize{(0.934)}} \\  

    \mc{1}{l}{\scriptsize{Obesity}} & \mc{1}{c}{\scriptsize{Mid-30s}} & \mc{1}{c}{\scriptsize{-0.061}} & \mc{1}{c}{\scriptsize{0.158}} & \mc{1}{c}{\scriptsize{-0.333}} & \mc{1}{c}{\scriptsize{-0.155}} & \mc{1}{c}{\scriptsize{-0.253}} & \mc{1}{c}{\scriptsize{-0.018}} & \mc{1}{c}{\scriptsize{0.245}} & \mc{1}{c}{\scriptsize{0.095}} \\  

     &  & \mc{1}{c}{\scriptsize{(0.368)}} & \mc{1}{c}{\scriptsize{(0.816)}} & \mc{1}{c}{\scriptsize{\textbf{(0.000)}}} & \mc{1}{c}{\scriptsize{(0.276)}} & \mc{1}{c}{\scriptsize{\textbf{(0.039)}}} & \mc{1}{c}{\scriptsize{(0.461)}} & \mc{1}{c}{\scriptsize{(0.921)}} & \mc{1}{c}{\scriptsize{(0.724)}} \\  

    \mc{1}{l}{\scriptsize{Severe Obesity}} & \mc{1}{c}{\scriptsize{Mid-30s}} & \mc{1}{c}{\scriptsize{-0.141}} & \mc{1}{c}{\scriptsize{-0.008}} & \mc{1}{c}{\scriptsize{-0.111}} & \mc{1}{c}{\scriptsize{-0.436}} & \mc{1}{c}{\scriptsize{-0.035}} & \mc{1}{c}{\scriptsize{-0.146}} & \mc{1}{c}{\scriptsize{0.089}} & \mc{1}{c}{\scriptsize{-0.051}} \\  

     &  & \mc{1}{c}{\scriptsize{(0.171)}} & \mc{1}{c}{\scriptsize{(0.513)}} & \mc{1}{c}{\scriptsize{(0.342)}} & \mc{1}{c}{\scriptsize{(0.145)}} & \mc{1}{c}{\scriptsize{(0.382)}} & \mc{1}{c}{\scriptsize{(0.158)}} & \mc{1}{c}{\scriptsize{(0.711)}} & \mc{1}{c}{\scriptsize{(0.355)}} \\  

    \mc{1}{l}{\scriptsize{Waist-hip Ratio}} & \mc{1}{c}{\scriptsize{Mid-30s}} & \mc{1}{c}{\scriptsize{-0.057}} & \mc{1}{c}{\scriptsize{-0.053}} & \mc{1}{c}{\scriptsize{-0.162}} & \mc{1}{c}{\scriptsize{-0.179}} & \mc{1}{c}{\scriptsize{-0.162}} & \mc{1}{c}{\scriptsize{-0.039}} & \mc{1}{c}{\scriptsize{-0.034}} & \mc{1}{c}{\scriptsize{-0.034}} \\  

     &  & \mc{1}{c}{\scriptsize{\textbf{(0.039)}}} & \mc{1}{c}{\scriptsize{\textbf{(0.092)}}} & \mc{1}{c}{\scriptsize{\textbf{(0.026)}}} & \mc{1}{c}{\scriptsize{\textbf{(0.066)}}} & \mc{1}{c}{\scriptsize{\textbf{(0.026)}}} & \mc{1}{c}{\scriptsize{(0.132)}} & \mc{1}{c}{\scriptsize{(0.184)}} & \mc{1}{c}{\scriptsize{(0.184)}} \\  

    \mc{1}{l}{\scriptsize{Abdominal Obesity}} & \mc{1}{c}{\scriptsize{Mid-30s}} & \mc{1}{c}{\scriptsize{-0.199}} & \mc{1}{c}{\scriptsize{-0.079}} & \mc{1}{c}{\scriptsize{-0.438}} & \mc{1}{c}{\scriptsize{-0.042}} & \mc{1}{c}{\scriptsize{-0.377}} & \mc{1}{c}{\scriptsize{-0.160}} & \mc{1}{c}{\scriptsize{-0.087}} & \mc{1}{c}{\scriptsize{-0.107}} \\  

     &  & \mc{1}{c}{\scriptsize{(0.118)}} & \mc{1}{c}{\scriptsize{(0.303)}} & \mc{1}{c}{\scriptsize{\textbf{(0.000)}}} & \mc{1}{c}{\scriptsize{(0.447)}} & \mc{1}{c}{\scriptsize{\textbf{(0.039)}}} & \mc{1}{c}{\scriptsize{(0.171)}} & \mc{1}{c}{\scriptsize{(0.303)}} & \mc{1}{c}{\scriptsize{(0.289)}} \\  

    \mc{1}{l}{\scriptsize{Framingham Risk Score}} & \mc{1}{c}{\scriptsize{Mid-30s}} & \mc{1}{c}{\scriptsize{-0.471}} & \mc{1}{c}{\scriptsize{-0.725}} & \mc{1}{c}{\scriptsize{-1.419}} & \mc{1}{c}{\scriptsize{-1.555}} & \mc{1}{c}{\scriptsize{-1.484}} & \mc{1}{c}{\scriptsize{-0.322}} & \mc{1}{c}{\scriptsize{-0.517}} & \mc{1}{c}{\scriptsize{-0.409}} \\  

     &  & \mc{1}{c}{\scriptsize{\textbf{(0.026)}}} & \mc{1}{c}{\scriptsize{\textbf{(0.066)}}} & \mc{1}{c}{\scriptsize{\textbf{(0.000)}}} & \mc{1}{c}{\scriptsize{\textbf{(0.013)}}} & \mc{1}{c}{\scriptsize{\textbf{(0.000)}}} & \mc{1}{c}{\scriptsize{(0.171)}} & \mc{1}{c}{\scriptsize{(0.118)}} & \mc{1}{c}{\scriptsize{(0.145)}} \\  

  \bottomrule
  \end{tabular}
\end{center}
\section{Treatment Effects for Pooled Sample}


\begin{center}
	\begin{table}[H]
\captionsetup{singlelinecheck=false,justification=centering}
\caption{CARE Average Treatment Effects, Males and Females \\ IQ Scores \label{tab:ate_pooled_apx0}}

  \begin{threeparttable}
  \begin{tabular}{cccccccccc}
  \hline\hline

     &  & \scriptsize{(1)} & \scriptsize{(2)} & \scriptsize{(3)} & \scriptsize{(4)} & \scriptsize{(5)} & \scriptsize{(6)} & \scriptsize{(7)} & \scriptsize{(8)} \\  

     &  &  &  & \mc{3}{c}{\scriptsize{$P=0$}} & \mc{3}{c}{\scriptsize{$P=1$}} \\ 
    \cmidrule(lr){5-7} \cmidrule(lr){8-10} 

    \scriptsize{Variable} & \scriptsize{Age} & \scriptsize{ITT} & \scriptsize{ITT$|X,W$} & \scriptsize{ITT} & \scriptsize{ITT$|X,W$} & \scriptsize{KE$|X,W$} & \scriptsize{ITT} & \scriptsize{ITT$|X,W$} & \scriptsize{KE$|X,W$} \\ 
    \hline  

    \mc{1}{l}{\scriptsize{Std. IQ Test}} & \mc{1}{c}{\scriptsize{2}} & \mc{1}{c}{\scriptsize{-5.140}} & \mc{1}{c}{\scriptsize{-5.342}} & \mc{1}{c}{\scriptsize{-3.231}} & \mc{1}{c}{\scriptsize{2.708}} & \mc{1}{c}{\scriptsize{-3.477}} & \mc{1}{c}{\scriptsize{-5.856}} & \mc{1}{c}{\scriptsize{-6.901}} & \mc{1}{c}{\scriptsize{-5.374}} \\  

     &  & \mc{1}{c}{\scriptsize{(0.196)}} & \mc{1}{c}{\scriptsize{(0.255)}} & \mc{1}{c}{\scriptsize{(0.451)}} & \mc{1}{c}{\scriptsize{(0.627)}} & \mc{1}{c}{\scriptsize{(0.510)}} & \mc{1}{c}{\scriptsize{(0.196)}} & \mc{1}{c}{\scriptsize{(0.176)}} & \mc{1}{c}{\scriptsize{(0.294)}} \\  

     & \mc{1}{c}{\scriptsize{2.5}} & \mc{1}{c}{\scriptsize{-9.180}} & \mc{1}{c}{\scriptsize{-8.630}} & \mc{1}{c}{\scriptsize{-3.680}} & \mc{1}{c}{\scriptsize{0.582}} & \mc{1}{c}{\scriptsize{-4.088}} & \mc{1}{c}{\scriptsize{-11.242}} & \mc{1}{c}{\scriptsize{-10.317}} & \mc{1}{c}{\scriptsize{-10.535}} \\  

     &  & \mc{1}{c}{\scriptsize{\textbf{(0.020)}}} & \mc{1}{c}{\scriptsize{\textbf{(0.098)}}} & \mc{1}{c}{\scriptsize{(0.431)}} & \mc{1}{c}{\scriptsize{(0.824)}} & \mc{1}{c}{\scriptsize{(0.392)}} & \mc{1}{c}{\scriptsize{\textbf{(0.020)}}} & \mc{1}{c}{\scriptsize{(0.118)}} & \mc{1}{c}{\scriptsize{\textbf{(0.098)}}} \\  

     & \mc{1}{c}{\scriptsize{3}} & \mc{1}{c}{\scriptsize{-5.106}} & \mc{1}{c}{\scriptsize{-1.313}} & \mc{1}{c}{\scriptsize{1.773}} & \mc{1}{c}{\scriptsize{10.509}} & \mc{1}{c}{\scriptsize{1.407}} & \mc{1}{c}{\scriptsize{-7.685}} & \mc{1}{c}{\scriptsize{-3.319}} & \mc{1}{c}{\scriptsize{-6.441}} \\  

     &  & \mc{1}{c}{\scriptsize{(0.157)}} & \mc{1}{c}{\scriptsize{(0.706)}} & \mc{1}{c}{\scriptsize{(0.686)}} & \mc{1}{c}{\scriptsize{(0.216)}} & \mc{1}{c}{\scriptsize{(0.765)}} & \mc{1}{c}{\scriptsize{\textbf{(0.020)}}} & \mc{1}{c}{\scriptsize{(0.490)}} & \mc{1}{c}{\scriptsize{(0.176)}} \\  

     & \mc{1}{c}{\scriptsize{3.5}} & \mc{1}{c}{\scriptsize{-4.356}} & \mc{1}{c}{\scriptsize{-5.196}} & \mc{1}{c}{\scriptsize{0.083}} & \mc{1}{c}{\scriptsize{3.881}} & \mc{1}{c}{\scriptsize{-0.220}} & \mc{1}{c}{\scriptsize{-6.021}} & \mc{1}{c}{\scriptsize{-6.976}} & \mc{1}{c}{\scriptsize{-6.558}} \\  

     &  & \mc{1}{c}{\scriptsize{(0.235)}} & \mc{1}{c}{\scriptsize{(0.137)}} & \mc{1}{c}{\scriptsize{(0.961)}} & \mc{1}{c}{\scriptsize{(0.490)}} & \mc{1}{c}{\scriptsize{(0.980)}} & \mc{1}{c}{\scriptsize{(0.137)}} & \mc{1}{c}{\scriptsize{\textbf{(0.098)}}} & \mc{1}{c}{\scriptsize{(0.137)}} \\  

     & \mc{1}{c}{\scriptsize{4}} & \mc{1}{c}{\scriptsize{-4.955}} & \mc{1}{c}{\scriptsize{-4.877}} & \mc{1}{c}{\scriptsize{-2.500}} & \mc{1}{c}{\scriptsize{-0.412}} & \mc{1}{c}{\scriptsize{-3.370}} & \mc{1}{c}{\scriptsize{-5.875}} & \mc{1}{c}{\scriptsize{-5.415}} & \mc{1}{c}{\scriptsize{-5.090}} \\  

     &  & \mc{1}{c}{\scriptsize{\textbf{(0.098)}}} & \mc{1}{c}{\scriptsize{(0.196)}} & \mc{1}{c}{\scriptsize{(0.627)}} & \mc{1}{c}{\scriptsize{(0.980)}} & \mc{1}{c}{\scriptsize{(0.608)}} & \mc{1}{c}{\scriptsize{\textbf{(0.078)}}} & \mc{1}{c}{\scriptsize{(0.196)}} & \mc{1}{c}{\scriptsize{(0.176)}} \\  

     & \mc{1}{c}{\scriptsize{4.5}} & \mc{1}{c}{\scriptsize{-6.074}} & \mc{1}{c}{\scriptsize{-5.680}} & \mc{1}{c}{\scriptsize{-0.453}} & \mc{1}{c}{\scriptsize{4.356}} & \mc{1}{c}{\scriptsize{-1.318}} & \mc{1}{c}{\scriptsize{-8.182}} & \mc{1}{c}{\scriptsize{-7.862}} & \mc{1}{c}{\scriptsize{-8.751}} \\  

     &  & \mc{1}{c}{\scriptsize{\textbf{(0.039)}}} & \mc{1}{c}{\scriptsize{(0.118)}} & \mc{1}{c}{\scriptsize{(0.922)}} & \mc{1}{c}{\scriptsize{(0.647)}} & \mc{1}{c}{\scriptsize{(0.824)}} & \mc{1}{c}{\scriptsize{\textbf{(0.020)}}} & \mc{1}{c}{\scriptsize{\textbf{(0.059)}}} & \mc{1}{c}{\scriptsize{\textbf{(0.000)}}} \\  

     & \mc{1}{c}{\scriptsize{5}} & \mc{1}{c}{\scriptsize{-5.477}} & \mc{1}{c}{\scriptsize{-7.522}} & \mc{1}{c}{\scriptsize{-7.583}} & \mc{1}{c}{\scriptsize{-5.305}} & \mc{1}{c}{\scriptsize{-8.082}} & \mc{1}{c}{\scriptsize{-4.688}} & \mc{1}{c}{\scriptsize{-6.450}} & \mc{1}{c}{\scriptsize{-5.838}} \\  

     &  & \mc{1}{c}{\scriptsize{(0.118)}} & \mc{1}{c}{\scriptsize{\textbf{(0.059)}}} & \mc{1}{c}{\scriptsize{(0.255)}} & \mc{1}{c}{\scriptsize{(0.471)}} & \mc{1}{c}{\scriptsize{(0.196)}} & \mc{1}{c}{\scriptsize{(0.235)}} & \mc{1}{c}{\scriptsize{\textbf{(0.078)}}} & \mc{1}{c}{\scriptsize{\textbf{(0.039)}}} \\  

     & \mc{1}{c}{\scriptsize{6.6}} & \mc{1}{c}{\scriptsize{-2.174}} & \mc{1}{c}{\scriptsize{-5.356}} & \mc{1}{c}{\scriptsize{0.398}} & \mc{1}{c}{\scriptsize{3.303}} & \mc{1}{c}{\scriptsize{-1.009}} & \mc{1}{c}{\scriptsize{-3.299}} & \mc{1}{c}{\scriptsize{-5.572}} & \mc{1}{c}{\scriptsize{-4.304}} \\  

     &  & \mc{1}{c}{\scriptsize{(0.431)}} & \mc{1}{c}{\scriptsize{(0.275)}} & \mc{1}{c}{\scriptsize{(0.941)}} & \mc{1}{c}{\scriptsize{(0.725)}} & \mc{1}{c}{\scriptsize{(0.882)}} & \mc{1}{c}{\scriptsize{(0.314)}} & \mc{1}{c}{\scriptsize{(0.353)}} & \mc{1}{c}{\scriptsize{(0.235)}} \\  

     & \mc{1}{c}{\scriptsize{7}} & \mc{1}{c}{\scriptsize{-4.567}} & \mc{1}{c}{\scriptsize{-7.888}} & \mc{1}{c}{\scriptsize{2.977}} & \mc{1}{c}{\scriptsize{5.787}} & \mc{1}{c}{\scriptsize{4.019}} & \mc{1}{c}{\scriptsize{-6.888}} & \mc{1}{c}{\scriptsize{-10.403}} & \mc{1}{c}{\scriptsize{-8.646}} \\  

     &  & \mc{1}{c}{\scriptsize{(0.118)}} & \mc{1}{c}{\scriptsize{\textbf{(0.078)}}} & \mc{1}{c}{\scriptsize{(0.490)}} & \mc{1}{c}{\scriptsize{(0.412)}} & \mc{1}{c}{\scriptsize{(0.333)}} & \mc{1}{c}{\scriptsize{\textbf{(0.039)}}} & \mc{1}{c}{\scriptsize{\textbf{(0.059)}}} & \mc{1}{c}{\scriptsize{\textbf{(0.059)}}} \\  

     & \mc{1}{c}{\scriptsize{8}} & \mc{1}{c}{\scriptsize{-3.765}} & \mc{1}{c}{\scriptsize{-5.660}} & \mc{1}{c}{\scriptsize{-1.440}} & \mc{1}{c}{\scriptsize{0.509}} & \mc{1}{c}{\scriptsize{-2.316}} & \mc{1}{c}{\scriptsize{-4.850}} & \mc{1}{c}{\scriptsize{-6.712}} & \mc{1}{c}{\scriptsize{-6.153}} \\  

     &  & \mc{1}{c}{\scriptsize{(0.255)}} & \mc{1}{c}{\scriptsize{(0.157)}} & \mc{1}{c}{\scriptsize{(0.824)}} & \mc{1}{c}{\scriptsize{(0.980)}} & \mc{1}{c}{\scriptsize{(0.745)}} & \mc{1}{c}{\scriptsize{\textbf{(0.039)}}} & \mc{1}{c}{\scriptsize{\textbf{(0.039)}}} & \mc{1}{c}{\scriptsize{\textbf{(0.020)}}} \\  

     & \mc{1}{c}{\scriptsize{12}} & \mc{1}{c}{\scriptsize{-9.144}} & \mc{1}{c}{\scriptsize{-11.832}} & \mc{1}{c}{\scriptsize{-8.280}} & \mc{1}{c}{\scriptsize{-6.865}} & \mc{1}{c}{\scriptsize{-7.850}} & \mc{1}{c}{\scriptsize{-9.467}} & \mc{1}{c}{\scriptsize{-12.177}} & \mc{1}{c}{\scriptsize{-10.602}} \\  

     &  & \mc{1}{c}{\scriptsize{\textbf{(0.000)}}} & \mc{1}{c}{\scriptsize{\textbf{(0.000)}}} & \mc{1}{c}{\scriptsize{(0.118)}} & \mc{1}{c}{\scriptsize{(0.294)}} & \mc{1}{c}{\scriptsize{(0.137)}} & \mc{1}{c}{\scriptsize{\textbf{(0.000)}}} & \mc{1}{c}{\scriptsize{\textbf{(0.000)}}} & \mc{1}{c}{\scriptsize{\textbf{(0.000)}}} \\ 
    \hline  

    \\[0.1cm]
    \mc{2}{l}{\scriptsize{\% of Sig. TE ($H_0$: $\le$ 25\% $|$ 10\% Significance)}} & \mc{1}{c}{\scriptsize{36}} & \mc{1}{c}{\scriptsize{36}} & \mc{1}{c}{\scriptsize{0}} & \mc{1}{c}{\scriptsize{0}} & \mc{1}{c}{\scriptsize{0}} & \mc{1}{c}{\scriptsize{64}} & \mc{1}{c}{\scriptsize{55}} & \mc{1}{c}{\scriptsize{55}} \\  

     &  & \mc{1}{c}{\scriptsize{(0.353)}} & \mc{1}{c}{\scriptsize{(0.176)}} & \mc{1}{c}{\scriptsize{(0.843)}} & \mc{1}{c}{\scriptsize{(0.882)}} & \mc{1}{c}{\scriptsize{(0.824)}} & \mc{1}{c}{\scriptsize{(0.176)}} & \mc{1}{c}{\scriptsize{(0.176)}} & \mc{1}{c}{\scriptsize{(0.137)}} \\  

    \mc{2}{l}{\scriptsize{\% of Sig. TE ($H_0$: $\le$ 50\% $|$ 10\% Significance)}} & \mc{1}{c}{\scriptsize{36}} & \mc{1}{c}{\scriptsize{36}} & \mc{1}{c}{\scriptsize{0}} & \mc{1}{c}{\scriptsize{0}} & \mc{1}{c}{\scriptsize{0}} & \mc{1}{c}{\scriptsize{64}} & \mc{1}{c}{\scriptsize{55}} & \mc{1}{c}{\scriptsize{55}} \\  

     &  & \mc{1}{c}{\scriptsize{(0.588)}} & \mc{1}{c}{\scriptsize{(0.647)}} & \mc{1}{c}{\scriptsize{(0.980)}} & \mc{1}{c}{\scriptsize{(0.980)}} & \mc{1}{c}{\scriptsize{(0.980)}} & \mc{1}{c}{\scriptsize{(0.314)}} & \mc{1}{c}{\scriptsize{(0.392)}} & \mc{1}{c}{\scriptsize{(0.373)}} \\  

    \mc{2}{l}{\scriptsize{\% of Sig. TE ($H_0$: $\le$ 75\% $|$ 10\% Significance)}} & \mc{1}{c}{\scriptsize{36}} & \mc{1}{c}{\scriptsize{36}} & \mc{1}{c}{\scriptsize{0}} & \mc{1}{c}{\scriptsize{0}} & \mc{1}{c}{\scriptsize{0}} & \mc{1}{c}{\scriptsize{64}} & \mc{1}{c}{\scriptsize{55}} & \mc{1}{c}{\scriptsize{55}} \\  

     &  & \mc{1}{c}{\scriptsize{(0.902)}} & \mc{1}{c}{\scriptsize{(1.000)}} & \mc{1}{c}{\scriptsize{(0.980)}} & \mc{1}{c}{\scriptsize{(0.980)}} & \mc{1}{c}{\scriptsize{(0.980)}} & \mc{1}{c}{\scriptsize{(0.647)}} & \mc{1}{c}{\scriptsize{(0.725)}} & \mc{1}{c}{\scriptsize{(0.745)}} \\  

  \hline\hline
  \end{tabular}
    \begin{tablenotes}
    \scriptsize
    \item 
Note: This table displays various estimates of the treatment effect of CARE's family education program.
Column (1) displays the ITT, without accounting for any controls.
Column (2) displays the ITT conditioning on vector of controls, $X$, consisting of Apgar scores 1 minute and 5 minutes after birth, an indicator for the subject 
being born prematurely, an indicator for the mother being married at baseline, an indicator for
teenage pregnancy of the mother, and an indicator for being born in the fall. We also apply IPW weights, $W$, to account for attrition.
Columns (3)--(4) are analogous to columns (1)--(2), but we restrict the control sample to subjects
who did not enroll in any alternative care.
Column (5) displys the matching estimate, where we use the Mahalanobis metric and Epanechnikov kernel
to match on controls $X$ listed above, and restrict the control sample to subjects who did not enroll
in any alternative care. Additionally, we apply IPW weights, $W$.
Columns (6)--(8) are analogous to Columns (3)--(5), except we restrict the control sample to subejcts
who did enroll in alternative care. The final three pairs of rows display the proportion of treatment effects in the table that are 
socially positive. The first row in each pair displays the percentage of treatment effects, and the
second row presents the inference. 
Numbers in parentheses represent the $p$-value from a single hypothesis test, and are obtained from 
the empirical bootstrap distribution generated by 200 resamples of the original data. 
Bold $p$-values indicate significance at the 10\% level.
Blank point estimates indicate that we are unable to obtain estimates due to a lack of support in the data. 

    \end{tablenotes}
  \end{threeparttable}

\end{table}
\end{center}

\begin{center}
	  \begin{tabular}{cccccccccc}
  \toprule

    \scriptsize{Variable} & \scriptsize{Age} & \scriptsize{(1)} & \scriptsize{(2)} & \scriptsize{(3)} & \scriptsize{(4)} & \scriptsize{(5)} & \scriptsize{(6)} & \scriptsize{(7)} & \scriptsize{(8)} \\ 
    \midrule  

    \mc{1}{l}{\scriptsize{Heart Attack}} & \mc{1}{c}{\scriptsize{Mid-30s}} &  &  &  &  &  &  &  &  \\  

     &  &  &  &  &  &  &  &  &  \\  

    \mc{1}{l}{\scriptsize{Sickle Cell Anemia}} & \mc{1}{c}{\scriptsize{Mid-30s}} &  &  &  &  &  &  &  &  \\  

     &  &  &  &  &  &  &  &  &  \\  

    \mc{1}{l}{\scriptsize{Asthma}} & \mc{1}{c}{\scriptsize{Mid-30s}} & \mc{1}{c}{\scriptsize{0.016}} & \mc{1}{c}{\scriptsize{-0.002}} & \mc{1}{c}{\scriptsize{0.040}} & \mc{1}{c}{\scriptsize{0.021}} & \mc{1}{c}{\scriptsize{0.025}} & \mc{1}{c}{\scriptsize{0.009}} & \mc{1}{c}{\scriptsize{-0.016}} & \mc{1}{c}{\scriptsize{-0.006}} \\  

     &  & \mc{1}{c}{\scriptsize{(0.605)}} & \mc{1}{c}{\scriptsize{(0.355)}} & \mc{1}{c}{\scriptsize{(0.789)}} & \mc{1}{c}{\scriptsize{(0.382)}} & \mc{1}{c}{\scriptsize{(0.461)}} & \mc{1}{c}{\scriptsize{(0.447)}} & \mc{1}{c}{\scriptsize{(0.250)}} & \mc{1}{c}{\scriptsize{(0.355)}} \\  

    \mc{1}{l}{\scriptsize{Stroke}} & \mc{1}{c}{\scriptsize{Mid-30s}} &  &  &  &  &  &  &  &  \\  

     &  &  &  &  &  &  &  &  &  \\  

    \mc{1}{l}{\scriptsize{High Blood Pressure (Hypertension)}} & \mc{1}{c}{\scriptsize{Mid-30s}} & \mc{1}{c}{\scriptsize{-0.003}} & \mc{1}{c}{\scriptsize{-0.010}} & \mc{1}{c}{\scriptsize{0.021}} & \mc{1}{c}{\scriptsize{0.008}} & \mc{1}{c}{\scriptsize{0.026}} & \mc{1}{c}{\scriptsize{-0.010}} & \mc{1}{c}{\scriptsize{-0.017}} & \mc{1}{c}{\scriptsize{-0.007}} \\  

     &  & \mc{1}{c}{\scriptsize{(0.408)}} & \mc{1}{c}{\scriptsize{(0.329)}} & \mc{1}{c}{\scriptsize{(0.539)}} & \mc{1}{c}{\scriptsize{(0.487)}} & \mc{1}{c}{\scriptsize{(0.553)}} & \mc{1}{c}{\scriptsize{(0.395)}} & \mc{1}{c}{\scriptsize{(0.342)}} & \mc{1}{c}{\scriptsize{(0.395)}} \\  

    \mc{1}{l}{\scriptsize{Arthritis or Generative Disease}} & \mc{1}{c}{\scriptsize{Mid-30s}} & \mc{1}{c}{\scriptsize{0.021}} & \mc{1}{c}{\scriptsize{0.024}} & \mc{1}{c}{\scriptsize{0.021}} & \mc{1}{c}{\scriptsize{0.016}} & \mc{1}{c}{\scriptsize{0.022}} & \mc{1}{c}{\scriptsize{0.021}} & \mc{1}{c}{\scriptsize{0.028}} & \mc{1}{c}{\scriptsize{0.022}} \\  

     &  & \mc{1}{c}{\scriptsize{(0.487)}} & \mc{1}{c}{\scriptsize{(0.474)}} & \mc{1}{c}{\scriptsize{(0.487)}} & \mc{1}{c}{\scriptsize{(0.382)}} & \mc{1}{c}{\scriptsize{(0.487)}} & \mc{1}{c}{\scriptsize{(0.487)}} & \mc{1}{c}{\scriptsize{(0.474)}} & \mc{1}{c}{\scriptsize{(0.487)}} \\  

    \mc{1}{l}{\scriptsize{Diabetes}} & \mc{1}{c}{\scriptsize{Mid-30s}} & \mc{1}{c}{\scriptsize{0.021}} & \mc{1}{c}{\scriptsize{0.025}} & \mc{1}{c}{\scriptsize{0.021}} & \mc{1}{c}{\scriptsize{0.002}} & \mc{1}{c}{\scriptsize{0.026}} & \mc{1}{c}{\scriptsize{0.021}} & \mc{1}{c}{\scriptsize{0.030}} & \mc{1}{c}{\scriptsize{0.026}} \\  

     &  & \mc{1}{c}{\scriptsize{(0.579)}} & \mc{1}{c}{\scriptsize{(0.579)}} & \mc{1}{c}{\scriptsize{(0.579)}} & \mc{1}{c}{\scriptsize{(0.276)}} & \mc{1}{c}{\scriptsize{(0.566)}} & \mc{1}{c}{\scriptsize{(0.579)}} & \mc{1}{c}{\scriptsize{(0.579)}} & \mc{1}{c}{\scriptsize{(0.566)}} \\  

    \mc{1}{l}{\scriptsize{Cancer}} & \mc{1}{c}{\scriptsize{Mid-30s}} &  &  &  &  &  &  &  &  \\  

     &  &  &  &  &  &  &  &  &  \\  

    \mc{1}{l}{\scriptsize{Heart Attack or Coronary Disease}} & \mc{1}{c}{\scriptsize{Mid-30s}} &  &  &  &  &  &  &  &  \\  

     &  &  &  &  &  &  &  &  &  \\  

    \mc{1}{l}{\scriptsize{High Cholesterol}} & \mc{1}{c}{\scriptsize{Mid-30s}} &  &  &  &  &  &  &  &  \\  

     &  &  &  &  &  &  &  &  &  \\  

    \mc{1}{l}{\scriptsize{Dementia}} & \mc{1}{c}{\scriptsize{Mid-30s}} &  &  &  &  &  &  &  &  \\  

     &  &  &  &  &  &  &  &  &  \\  

  \bottomrule
  \end{tabular}
\end{center}

\begin{center}
	  \begin{tabular}{cccccccccc}
  \toprule

    \scriptsize{Variable} & \scriptsize{Age} & \scriptsize{(1)} & \scriptsize{(2)} & \scriptsize{(3)} & \scriptsize{(4)} & \scriptsize{(5)} & \scriptsize{(6)} & \scriptsize{(7)} & \scriptsize{(8)} \\ 
    \midrule  

    \mc{1}{l}{\scriptsize{Temperament cluster - activity level}} & \mc{1}{c}{\scriptsize{0.5}} & \mc{1}{c}{\scriptsize{0.609}} & \mc{1}{c}{\scriptsize{0.349}} & \mc{1}{c}{\scriptsize{0.981}} & \mc{1}{c}{\scriptsize{0.608}} & \mc{1}{c}{\scriptsize{0.876}} & \mc{1}{c}{\scriptsize{0.555}} & \mc{1}{c}{\scriptsize{0.272}} & \mc{1}{c}{\scriptsize{0.322}} \\  

     &  & \mc{1}{c}{\scriptsize{(0.132)}} & \mc{1}{c}{\scriptsize{(0.263)}} & \mc{1}{c}{\scriptsize{(0.132)}} & \mc{1}{c}{\scriptsize{(0.197)}} & \mc{1}{c}{\scriptsize{(0.145)}} & \mc{1}{c}{\scriptsize{(0.145)}} & \mc{1}{c}{\scriptsize{(0.303)}} & \mc{1}{c}{\scriptsize{(0.276)}} \\  

     & \mc{1}{c}{\scriptsize{1}} & \mc{1}{c}{\scriptsize{1.076}} & \mc{1}{c}{\scriptsize{0.743}} & \mc{1}{c}{\scriptsize{1.152}} & \mc{1}{c}{\scriptsize{0.831}} & \mc{1}{c}{\scriptsize{1.084}} & \mc{1}{c}{\scriptsize{0.930}} & \mc{1}{c}{\scriptsize{0.714}} & \mc{1}{c}{\scriptsize{0.815}} \\  

     &  & \mc{1}{c}{\scriptsize{\textbf{(0.013)}}} & \mc{1}{c}{\scriptsize{(0.105)}} & \mc{1}{c}{\scriptsize{(0.118)}} & \mc{1}{c}{\scriptsize{(0.211)}} & \mc{1}{c}{\scriptsize{(0.171)}} & \mc{1}{c}{\scriptsize{\textbf{(0.053)}}} & \mc{1}{c}{\scriptsize{(0.132)}} & \mc{1}{c}{\scriptsize{(0.132)}} \\  

     & \mc{1}{c}{\scriptsize{1.5}} & \mc{1}{c}{\scriptsize{-0.496}} & \mc{1}{c}{\scriptsize{-0.506}} & \mc{1}{c}{\scriptsize{0.240}} & \mc{1}{c}{\scriptsize{0.444}} & \mc{1}{c}{\scriptsize{0.637}} & \mc{1}{c}{\scriptsize{-0.893}} & \mc{1}{c}{\scriptsize{-0.749}} & \mc{1}{c}{\scriptsize{-0.567}} \\  

     &  & \mc{1}{c}{\scriptsize{(0.816)}} & \mc{1}{c}{\scriptsize{(0.776)}} & \mc{1}{c}{\scriptsize{(0.355)}} & \mc{1}{c}{\scriptsize{(0.342)}} & \mc{1}{c}{\scriptsize{(0.250)}} & \mc{1}{c}{\scriptsize{(0.921)}} & \mc{1}{c}{\scriptsize{(0.882)}} & \mc{1}{c}{\scriptsize{(0.855)}} \\  

     & \mc{1}{c}{\scriptsize{2}} & \mc{1}{c}{\scriptsize{-0.839}} & \mc{1}{c}{\scriptsize{-1.997}} & \mc{1}{c}{\scriptsize{-1.149}} & \mc{1}{c}{\scriptsize{-1.954}} & \mc{1}{c}{\scriptsize{-2.104}} & \mc{1}{c}{\scriptsize{-0.755}} & \mc{1}{c}{\scriptsize{-1.935}} & \mc{1}{c}{\scriptsize{-1.343}} \\  

     &  & \mc{1}{c}{\scriptsize{(0.855)}} & \mc{1}{c}{\scriptsize{(1.000)}} & \mc{1}{c}{\scriptsize{(0.829)}} & \mc{1}{c}{\scriptsize{(0.961)}} & \mc{1}{c}{\scriptsize{(0.974)}} & \mc{1}{c}{\scriptsize{(0.803)}} & \mc{1}{c}{\scriptsize{(1.000)}} & \mc{1}{c}{\scriptsize{(0.947)}} \\  

    \mc{1}{l}{\scriptsize{Temperament cluster - cooperativeness}} & \mc{1}{c}{\scriptsize{0.5}} & \mc{1}{c}{\scriptsize{0.422}} & \mc{1}{c}{\scriptsize{-0.948}} & \mc{1}{c}{\scriptsize{-1.476}} & \mc{1}{c}{\scriptsize{-3.544}} & \mc{1}{c}{\scriptsize{-1.940}} & \mc{1}{c}{\scriptsize{1.420}} & \mc{1}{c}{\scriptsize{1.183}} & \mc{1}{c}{\scriptsize{1.266}} \\  

     &  & \mc{1}{c}{\scriptsize{(0.329)}} & \mc{1}{c}{\scriptsize{(0.671)}} & \mc{1}{c}{\scriptsize{(0.829)}} & \mc{1}{c}{\scriptsize{(0.566)}} & \mc{1}{c}{\scriptsize{(0.855)}} & \mc{1}{c}{\scriptsize{(0.171)}} & \mc{1}{c}{\scriptsize{(0.355)}} & \mc{1}{c}{\scriptsize{(0.237)}} \\  

     & \mc{1}{c}{\scriptsize{1}} & \mc{1}{c}{\scriptsize{0.100}} & \mc{1}{c}{\scriptsize{2.358}} & \mc{1}{c}{\scriptsize{-0.733}} & \mc{1}{c}{\scriptsize{2.184}} & \mc{1}{c}{\scriptsize{0.778}} & \mc{1}{c}{\scriptsize{0.412}} & \mc{1}{c}{\scriptsize{2.462}} & \mc{1}{c}{\scriptsize{2.226}} \\  

     &  & \mc{1}{c}{\scriptsize{(0.461)}} & \mc{1}{c}{\scriptsize{(0.105)}} & \mc{1}{c}{\scriptsize{(0.618)}} & \mc{1}{c}{\scriptsize{\textbf{(0.066)}}} & \mc{1}{c}{\scriptsize{(0.395)}} & \mc{1}{c}{\scriptsize{(0.408)}} & \mc{1}{c}{\scriptsize{(0.118)}} & \mc{1}{c}{\scriptsize{(0.171)}} \\  

     & \mc{1}{c}{\scriptsize{1.5}} & \mc{1}{c}{\scriptsize{0.694}} & \mc{1}{c}{\scriptsize{2.704}} & \mc{1}{c}{\scriptsize{-0.700}} & \mc{1}{c}{\scriptsize{0.985}} & \mc{1}{c}{\scriptsize{1.191}} & \mc{1}{c}{\scriptsize{1.217}} & \mc{1}{c}{\scriptsize{3.787}} & \mc{1}{c}{\scriptsize{3.621}} \\  

     &  & \mc{1}{c}{\scriptsize{(0.329)}} & \mc{1}{c}{\scriptsize{\textbf{(0.053)}}} & \mc{1}{c}{\scriptsize{(0.605)}} & \mc{1}{c}{\scriptsize{(0.118)}} & \mc{1}{c}{\scriptsize{(0.342)}} & \mc{1}{c}{\scriptsize{(0.250)}} & \mc{1}{c}{\scriptsize{\textbf{(0.013)}}} & \mc{1}{c}{\scriptsize{\textbf{(0.026)}}} \\  

     & \mc{1}{c}{\scriptsize{2}} & \mc{1}{c}{\scriptsize{2.439}} & \mc{1}{c}{\scriptsize{2.385}} & \mc{1}{c}{\scriptsize{2.500}} & \mc{1}{c}{\scriptsize{3.612}} & \mc{1}{c}{\scriptsize{2.746}} & \mc{1}{c}{\scriptsize{2.417}} & \mc{1}{c}{\scriptsize{2.289}} & \mc{1}{c}{\scriptsize{2.500}} \\  

     &  & \mc{1}{c}{\scriptsize{(0.118)}} & \mc{1}{c}{\scriptsize{(0.145)}} & \mc{1}{c}{\scriptsize{(0.132)}} & \mc{1}{c}{\scriptsize{\textbf{(0.026)}}} & \mc{1}{c}{\scriptsize{(0.145)}} & \mc{1}{c}{\scriptsize{(0.118)}} & \mc{1}{c}{\scriptsize{(0.145)}} & \mc{1}{c}{\scriptsize{(0.158)}} \\  

    \mc{1}{l}{\scriptsize{Temperament cluster - sociability}} & \mc{1}{c}{\scriptsize{0.5}} & \mc{1}{c}{\scriptsize{-0.007}} & \mc{1}{c}{\scriptsize{-0.053}} & \mc{1}{c}{\scriptsize{0.383}} & \mc{1}{c}{\scriptsize{0.424}} & \mc{1}{c}{\scriptsize{0.389}} & \mc{1}{c}{\scriptsize{-0.147}} & \mc{1}{c}{\scriptsize{-0.160}} & \mc{1}{c}{\scriptsize{-0.185}} \\  

     &  & \mc{1}{c}{\scriptsize{(0.553)}} & \mc{1}{c}{\scriptsize{(0.605)}} & \mc{1}{c}{\scriptsize{(0.105)}} & \mc{1}{c}{\scriptsize{\textbf{(0.092)}}} & \mc{1}{c}{\scriptsize{(0.118)}} & \mc{1}{c}{\scriptsize{(0.737)}} & \mc{1}{c}{\scriptsize{(0.750)}} & \mc{1}{c}{\scriptsize{(0.737)}} \\  

     & \mc{1}{c}{\scriptsize{1}} & \mc{1}{c}{\scriptsize{0.353}} & \mc{1}{c}{\scriptsize{0.395}} & \mc{1}{c}{\scriptsize{-0.159}} & \mc{1}{c}{\scriptsize{-0.083}} & \mc{1}{c}{\scriptsize{-0.068}} & \mc{1}{c}{\scriptsize{0.537}} & \mc{1}{c}{\scriptsize{0.630}} & \mc{1}{c}{\scriptsize{0.670}} \\  

     &  & \mc{1}{c}{\scriptsize{\textbf{(0.053)}}} & \mc{1}{c}{\scriptsize{\textbf{(0.026)}}} & \mc{1}{c}{\scriptsize{(0.671)}} & \mc{1}{c}{\scriptsize{(0.605)}} & \mc{1}{c}{\scriptsize{(0.592)}} & \mc{1}{c}{\scriptsize{\textbf{(0.026)}}} & \mc{1}{c}{\scriptsize{\textbf{(0.000)}}} & \mc{1}{c}{\scriptsize{\textbf{(0.000)}}} \\  

     & \mc{1}{c}{\scriptsize{1.5}} & \mc{1}{c}{\scriptsize{0.156}} & \mc{1}{c}{\scriptsize{0.463}} & \mc{1}{c}{\scriptsize{0.044}} & \mc{1}{c}{\scriptsize{0.458}} & \mc{1}{c}{\scriptsize{0.364}} & \mc{1}{c}{\scriptsize{0.224}} & \mc{1}{c}{\scriptsize{0.519}} & \mc{1}{c}{\scriptsize{0.554}} \\  

     &  & \mc{1}{c}{\scriptsize{(0.250)}} & \mc{1}{c}{\scriptsize{\textbf{(0.039)}}} & \mc{1}{c}{\scriptsize{(0.395)}} & \mc{1}{c}{\scriptsize{(0.171)}} & \mc{1}{c}{\scriptsize{(0.171)}} & \mc{1}{c}{\scriptsize{(0.211)}} & \mc{1}{c}{\scriptsize{\textbf{(0.013)}}} & \mc{1}{c}{\scriptsize{\textbf{(0.039)}}} \\  

     & \mc{1}{c}{\scriptsize{2}} & \mc{1}{c}{\scriptsize{-0.047}} & \mc{1}{c}{\scriptsize{-0.191}} & \mc{1}{c}{\scriptsize{-0.557}} & \mc{1}{c}{\scriptsize{-0.161}} & \mc{1}{c}{\scriptsize{-0.760}} & \mc{1}{c}{\scriptsize{0.058}} & \mc{1}{c}{\scriptsize{-0.142}} & \mc{1}{c}{\scriptsize{-0.099}} \\  

     &  & \mc{1}{c}{\scriptsize{(0.566)}} & \mc{1}{c}{\scriptsize{(0.684)}} & \mc{1}{c}{\scriptsize{(0.868)}} & \mc{1}{c}{\scriptsize{(0.539)}} & \mc{1}{c}{\scriptsize{(0.947)}} & \mc{1}{c}{\scriptsize{(0.434)}} & \mc{1}{c}{\scriptsize{(0.592)}} & \mc{1}{c}{\scriptsize{(0.579)}} \\  

    \mc{1}{l}{\scriptsize{Temperament cluster - task orientation}} & \mc{1}{c}{\scriptsize{0.5}} & \mc{1}{c}{\scriptsize{-0.071}} & \mc{1}{c}{\scriptsize{0.429}} & \mc{1}{c}{\scriptsize{0.767}} & \mc{1}{c}{\scriptsize{1.466}} & \mc{1}{c}{\scriptsize{1.171}} & \mc{1}{c}{\scriptsize{-0.309}} & \mc{1}{c}{\scriptsize{-0.109}} & \mc{1}{c}{\scriptsize{-0.133}} \\  

     &  & \mc{1}{c}{\scriptsize{(0.553)}} & \mc{1}{c}{\scriptsize{(0.276)}} & \mc{1}{c}{\scriptsize{(0.289)}} & \mc{1}{c}{\scriptsize{(0.105)}} & \mc{1}{c}{\scriptsize{(0.197)}} & \mc{1}{c}{\scriptsize{(0.592)}} & \mc{1}{c}{\scriptsize{(0.539)}} & \mc{1}{c}{\scriptsize{(0.539)}} \\  

     & \mc{1}{c}{\scriptsize{1}} & \mc{1}{c}{\scriptsize{0.946}} & \mc{1}{c}{\scriptsize{1.327}} & \mc{1}{c}{\scriptsize{0.737}} & \mc{1}{c}{\scriptsize{1.472}} & \mc{1}{c}{\scriptsize{0.949}} & \mc{1}{c}{\scriptsize{1.055}} & \mc{1}{c}{\scriptsize{1.429}} & \mc{1}{c}{\scriptsize{1.284}} \\  

     &  & \mc{1}{c}{\scriptsize{\textbf{(0.013)}}} & \mc{1}{c}{\scriptsize{\textbf{(0.026)}}} & \mc{1}{c}{\scriptsize{(0.118)}} & \mc{1}{c}{\scriptsize{\textbf{(0.039)}}} & \mc{1}{c}{\scriptsize{(0.105)}} & \mc{1}{c}{\scriptsize{\textbf{(0.026)}}} & \mc{1}{c}{\scriptsize{\textbf{(0.039)}}} & \mc{1}{c}{\scriptsize{\textbf{(0.013)}}} \\  

     & \mc{1}{c}{\scriptsize{1.5}} & \mc{1}{c}{\scriptsize{2.363}} & \mc{1}{c}{\scriptsize{3.176}} & \mc{1}{c}{\scriptsize{1.762}} & \mc{1}{c}{\scriptsize{2.714}} & \mc{1}{c}{\scriptsize{2.219}} & \mc{1}{c}{\scriptsize{2.483}} & \mc{1}{c}{\scriptsize{3.162}} & \mc{1}{c}{\scriptsize{2.910}} \\  

     &  & \mc{1}{c}{\scriptsize{\textbf{(0.000)}}} & \mc{1}{c}{\scriptsize{\textbf{(0.000)}}} & \mc{1}{c}{\scriptsize{\textbf{(0.039)}}} & \mc{1}{c}{\scriptsize{\textbf{(0.000)}}} & \mc{1}{c}{\scriptsize{\textbf{(0.000)}}} & \mc{1}{c}{\scriptsize{\textbf{(0.000)}}} & \mc{1}{c}{\scriptsize{\textbf{(0.000)}}} & \mc{1}{c}{\scriptsize{\textbf{(0.000)}}} \\  

     & \mc{1}{c}{\scriptsize{2}} & \mc{1}{c}{\scriptsize{1.929}} & \mc{1}{c}{\scriptsize{2.982}} & \mc{1}{c}{\scriptsize{1.893}} & \mc{1}{c}{\scriptsize{3.522}} & \mc{1}{c}{\scriptsize{3.408}} & \mc{1}{c}{\scriptsize{1.938}} & \mc{1}{c}{\scriptsize{2.798}} & \mc{1}{c}{\scriptsize{2.781}} \\  

     &  & \mc{1}{c}{\scriptsize{\textbf{(0.000)}}} & \mc{1}{c}{\scriptsize{\textbf{(0.000)}}} & \mc{1}{c}{\scriptsize{\textbf{(0.026)}}} & \mc{1}{c}{\scriptsize{\textbf{(0.000)}}} & \mc{1}{c}{\scriptsize{\textbf{(0.000)}}} & \mc{1}{c}{\scriptsize{\textbf{(0.000)}}} & \mc{1}{c}{\scriptsize{\textbf{(0.000)}}} & \mc{1}{c}{\scriptsize{\textbf{(0.000)}}} \\  

  \bottomrule
  \end{tabular}
\end{center}

\begin{center}
	  \begin{tabular}{cccccccccc}
  \toprule

    \scriptsize{Variable} & \scriptsize{Age} & \scriptsize{(1)} & \scriptsize{(2)} & \scriptsize{(3)} & \scriptsize{(4)} & \scriptsize{(5)} & \scriptsize{(6)} & \scriptsize{(7)} & \scriptsize{(8)} \\ 
    \midrule  

    \mc{1}{l}{\scriptsize{Parental Income}} & \mc{1}{c}{\scriptsize{1.5}} & \mc{1}{c}{\scriptsize{2,248}} & \mc{1}{c}{\scriptsize{3,277}} & \mc{1}{c}{\scriptsize{2,553}} & \mc{1}{c}{\scriptsize{4,721}} & \mc{1}{c}{\scriptsize{5,028}} & \mc{1}{c}{\scriptsize{1,870}} & \mc{1}{c}{\scriptsize{2,901}} & \mc{1}{c}{\scriptsize{3,718}} \\  

     &  & \mc{1}{c}{\scriptsize{(0.132)}} & \mc{1}{c}{\scriptsize{\textbf{(0.066)}}} & \mc{1}{c}{\scriptsize{(0.211)}} & \mc{1}{c}{\scriptsize{(0.118)}} & \mc{1}{c}{\scriptsize{\textbf{(0.066)}}} & \mc{1}{c}{\scriptsize{(0.184)}} & \mc{1}{c}{\scriptsize{(0.105)}} & \mc{1}{c}{\scriptsize{\textbf{(0.026)}}} \\  

     & \mc{1}{c}{\scriptsize{2.5}} & \mc{1}{c}{\scriptsize{516}} & \mc{1}{c}{\scriptsize{366}} & \mc{1}{c}{\scriptsize{-2,455}} & \mc{1}{c}{\scriptsize{-851}} & \mc{1}{c}{\scriptsize{93.152}} & \mc{1}{c}{\scriptsize{988}} & \mc{1}{c}{\scriptsize{469}} & \mc{1}{c}{\scriptsize{1,555}} \\  

     &  & \mc{1}{c}{\scriptsize{(0.355)}} & \mc{1}{c}{\scriptsize{(0.500)}} & \mc{1}{c}{\scriptsize{(0.763)}} & \mc{1}{c}{\scriptsize{(0.500)}} & \mc{1}{c}{\scriptsize{(0.461)}} & \mc{1}{c}{\scriptsize{(0.329)}} & \mc{1}{c}{\scriptsize{(0.447)}} & \mc{1}{c}{\scriptsize{(0.237)}} \\  

     & \mc{1}{c}{\scriptsize{3.5}} & \mc{1}{c}{\scriptsize{1,821}} & \mc{1}{c}{\scriptsize{1,901}} & \mc{1}{c}{\scriptsize{3,984}} & \mc{1}{c}{\scriptsize{4,990}} & \mc{1}{c}{\scriptsize{5,265}} & \mc{1}{c}{\scriptsize{961}} & \mc{1}{c}{\scriptsize{961}} & \mc{1}{c}{\scriptsize{2,108}} \\  

     &  & \mc{1}{c}{\scriptsize{(0.184)}} & \mc{1}{c}{\scriptsize{(0.237)}} & \mc{1}{c}{\scriptsize{\textbf{(0.066)}}} & \mc{1}{c}{\scriptsize{\textbf{(0.092)}}} & \mc{1}{c}{\scriptsize{\textbf{(0.053)}}} & \mc{1}{c}{\scriptsize{(0.289)}} & \mc{1}{c}{\scriptsize{(0.329)}} & \mc{1}{c}{\scriptsize{(0.197)}} \\  

     & \mc{1}{c}{\scriptsize{4.5}} & \mc{1}{c}{\scriptsize{2,336}} & \mc{1}{c}{\scriptsize{3,565}} & \mc{1}{c}{\scriptsize{4,159}} & \mc{1}{c}{\scriptsize{4,738}} & \mc{1}{c}{\scriptsize{2,703}} & \mc{1}{c}{\scriptsize{1,434}} & \mc{1}{c}{\scriptsize{2,926}} & \mc{1}{c}{\scriptsize{2,697}} \\  

     &  & \mc{1}{c}{\scriptsize{(0.105)}} & \mc{1}{c}{\scriptsize{\textbf{(0.092)}}} & \mc{1}{c}{\scriptsize{\textbf{(0.066)}}} & \mc{1}{c}{\scriptsize{\textbf{(0.079)}}} & \mc{1}{c}{\scriptsize{(0.197)}} & \mc{1}{c}{\scriptsize{(0.263)}} & \mc{1}{c}{\scriptsize{(0.132)}} & \mc{1}{c}{\scriptsize{(0.158)}} \\  

    \mc{1}{l}{\scriptsize{Parental Income Factor}} & \mc{1}{c}{\scriptsize{1.5 to 15}} & \mc{1}{c}{\scriptsize{0.156}} & \mc{1}{c}{\scriptsize{0.120}} & \mc{1}{c}{\scriptsize{0.158}} & \mc{1}{c}{\scriptsize{0.303}} & \mc{1}{c}{\scriptsize{0.207}} & \mc{1}{c}{\scriptsize{0.127}} & \mc{1}{c}{\scriptsize{0.063}} & \mc{1}{c}{\scriptsize{0.183}} \\  

     &  & \mc{1}{c}{\scriptsize{(0.184)}} & \mc{1}{c}{\scriptsize{(0.329)}} & \mc{1}{c}{\scriptsize{(0.355)}} & \mc{1}{c}{\scriptsize{(0.184)}} & \mc{1}{c}{\scriptsize{(0.224)}} & \mc{1}{c}{\scriptsize{(0.224)}} & \mc{1}{c}{\scriptsize{(0.434)}} & \mc{1}{c}{\scriptsize{(0.211)}} \\  

  \bottomrule
  \end{tabular}
\end{center}

\begin{center}
	  \begin{tabular}{cccccccccc}
  \toprule

    \scriptsize{Variable} & \scriptsize{Age} & \scriptsize{(1)} & \scriptsize{(2)} & \scriptsize{(3)} & \scriptsize{(4)} & \scriptsize{(5)} & \scriptsize{(6)} & \scriptsize{(7)} & \scriptsize{(8)} \\ 
    \midrule  

    \mc{1}{l}{\scriptsize{Mother Works}} & \mc{1}{c}{\scriptsize{2}} & \mc{1}{c}{\scriptsize{0.146}} & \mc{1}{c}{\scriptsize{0.097}} & \mc{1}{c}{\scriptsize{0.133}} & \mc{1}{c}{\scriptsize{-0.020}} & \mc{1}{c}{\scriptsize{0.101}} & \mc{1}{c}{\scriptsize{0.152}} & \mc{1}{c}{\scriptsize{0.112}} & \mc{1}{c}{\scriptsize{0.113}} \\  

     &  & \mc{1}{c}{\scriptsize{\textbf{(0.039)}}} & \mc{1}{c}{\scriptsize{(0.289)}} & \mc{1}{c}{\scriptsize{(0.118)}} & \mc{1}{c}{\scriptsize{(0.539)}} & \mc{1}{c}{\scriptsize{(0.289)}} & \mc{1}{c}{\scriptsize{\textbf{(0.066)}}} & \mc{1}{c}{\scriptsize{(0.276)}} & \mc{1}{c}{\scriptsize{(0.224)}} \\  

     & \mc{1}{c}{\scriptsize{3}} & \mc{1}{c}{\scriptsize{0.155}} & \mc{1}{c}{\scriptsize{0.031}} & \mc{1}{c}{\scriptsize{0.324}} & \mc{1}{c}{\scriptsize{0.048}} & \mc{1}{c}{\scriptsize{0.135}} & \mc{1}{c}{\scriptsize{0.076}} & \mc{1}{c}{\scriptsize{-0.020}} & \mc{1}{c}{\scriptsize{-0.042}} \\  

     &  & \mc{1}{c}{\scriptsize{\textbf{(0.092)}}} & \mc{1}{c}{\scriptsize{(0.316)}} & \mc{1}{c}{\scriptsize{\textbf{(0.013)}}} & \mc{1}{c}{\scriptsize{(0.408)}} & \mc{1}{c}{\scriptsize{(0.329)}} & \mc{1}{c}{\scriptsize{(0.263)}} & \mc{1}{c}{\scriptsize{(0.513)}} & \mc{1}{c}{\scriptsize{(0.539)}} \\  

     & \mc{1}{c}{\scriptsize{4}} & \mc{1}{c}{\scriptsize{0.217}} & \mc{1}{c}{\scriptsize{0.222}} & \mc{1}{c}{\scriptsize{0.500}} & \mc{1}{c}{\scriptsize{0.446}} & \mc{1}{c}{\scriptsize{0.453}} & \mc{1}{c}{\scriptsize{0.094}} & \mc{1}{c}{\scriptsize{0.093}} & \mc{1}{c}{\scriptsize{0.112}} \\  

     &  & \mc{1}{c}{\scriptsize{\textbf{(0.000)}}} & \mc{1}{c}{\scriptsize{\textbf{(0.013)}}} & \mc{1}{c}{\scriptsize{\textbf{(0.000)}}} & \mc{1}{c}{\scriptsize{\textbf{(0.039)}}} & \mc{1}{c}{\scriptsize{\textbf{(0.000)}}} & \mc{1}{c}{\scriptsize{\textbf{(0.026)}}} & \mc{1}{c}{\scriptsize{(0.118)}} & \mc{1}{c}{\scriptsize{\textbf{(0.039)}}} \\  

     & \mc{1}{c}{\scriptsize{5}} & \mc{1}{c}{\scriptsize{0.037}} & \mc{1}{c}{\scriptsize{0.026}} & \mc{1}{c}{\scriptsize{0.090}} & \mc{1}{c}{\scriptsize{-0.042}} & \mc{1}{c}{\scriptsize{0.129}} & \mc{1}{c}{\scriptsize{0.017}} & \mc{1}{c}{\scriptsize{0.030}} & \mc{1}{c}{\scriptsize{0.058}} \\  

     &  & \mc{1}{c}{\scriptsize{(0.316)}} & \mc{1}{c}{\scriptsize{(0.368)}} & \mc{1}{c}{\scriptsize{(0.263)}} & \mc{1}{c}{\scriptsize{(0.500)}} & \mc{1}{c}{\scriptsize{(0.118)}} & \mc{1}{c}{\scriptsize{(0.382)}} & \mc{1}{c}{\scriptsize{(0.316)}} & \mc{1}{c}{\scriptsize{(0.184)}} \\  

    \mc{1}{l}{\scriptsize{Mother Works Factor}} & \mc{1}{c}{\scriptsize{2 to 21}} & \mc{1}{c}{\scriptsize{0.453}} & \mc{1}{c}{\scriptsize{0.356}} & \mc{1}{c}{\scriptsize{0.925}} & \mc{1}{c}{\scriptsize{0.680}} & \mc{1}{c}{\scriptsize{0.741}} & \mc{1}{c}{\scriptsize{0.271}} & \mc{1}{c}{\scriptsize{0.242}} & \mc{1}{c}{\scriptsize{0.256}} \\  

     &  & \mc{1}{c}{\scriptsize{\textbf{(0.000)}}} & \mc{1}{c}{\scriptsize{(0.133)}} & \mc{1}{c}{\scriptsize{\textbf{(0.000)}}} & \mc{1}{c}{\scriptsize{(0.187)}} & \mc{1}{c}{\scriptsize{\textbf{(0.040)}}} & \mc{1}{c}{\scriptsize{\textbf{(0.080)}}} & \mc{1}{c}{\scriptsize{(0.213)}} & \mc{1}{c}{\scriptsize{(0.173)}} \\ 
    \midrule  

    \mc{2}{l}{\scriptsize{\% of Pos. TE ($H_0$: $\le$ 50\%)}} & \mc{1}{c}{\scriptsize{100}} & \mc{1}{c}{\scriptsize{100}} & \mc{1}{c}{\scriptsize{100}} & \mc{1}{c}{\scriptsize{60}} & \mc{1}{c}{\scriptsize{100}} & \mc{1}{c}{\scriptsize{100}} & \mc{1}{c}{\scriptsize{80}} & \mc{1}{c}{\scriptsize{80}} \\  

     &  & \mc{1}{c}{\scriptsize{\textbf{(0.000)}}} & \mc{1}{c}{\scriptsize{\textbf{(0.000)}}} & \mc{1}{c}{\scriptsize{\textbf{(0.000)}}} & \mc{1}{c}{\scriptsize{(0.421)}} & \mc{1}{c}{\scriptsize{\textbf{(0.000)}}} & \mc{1}{c}{\scriptsize{\textbf{(0.000)}}} & \mc{1}{c}{\scriptsize{\textbf{(0.000)}}} & \mc{1}{c}{\scriptsize{\textbf{(0.000)}}} \\  

    \mc{2}{l}{\scriptsize{\% of Pos. TE ($H_0$: $\le$ 10\% $|$ 10\% Significance)}} & \mc{1}{c}{\scriptsize{80}} & \mc{1}{c}{\scriptsize{40}} & \mc{1}{c}{\scriptsize{60}} & \mc{1}{c}{\scriptsize{20}} & \mc{1}{c}{\scriptsize{40}} & \mc{1}{c}{\scriptsize{60}} & \mc{1}{c}{\scriptsize{20}} & \mc{1}{c}{\scriptsize{20}} \\  

     &  & \mc{1}{c}{\scriptsize{\textbf{(0.000)}}} & \mc{1}{c}{\scriptsize{(0.197)}} & \mc{1}{c}{\scriptsize{\textbf{(0.000)}}} & \mc{1}{c}{\scriptsize{(0.171)}} & \mc{1}{c}{\scriptsize{(0.263)}} & \mc{1}{c}{\scriptsize{\textbf{(0.039)}}} & \mc{1}{c}{\scriptsize{(0.171)}} & \mc{1}{c}{\scriptsize{(0.250)}} \\  

  \bottomrule
  \end{tabular}
\end{center}

\begin{center}
	  \begin{tabular}{cccccccccc}
  \toprule

    \scriptsize{Variable} & \scriptsize{Age} & \scriptsize{(1)} & \scriptsize{(2)} & \scriptsize{(3)} & \scriptsize{(4)} & \scriptsize{(5)} & \scriptsize{(6)} & \scriptsize{(7)} & \scriptsize{(8)} \\ 
    \midrule  

    \mc{1}{l}{\scriptsize{Mother Works}} & \mc{1}{c}{\scriptsize{2}} & \mc{1}{c}{\scriptsize{0.056}} & \mc{1}{c}{\scriptsize{0.033}} & \mc{1}{c}{\scriptsize{0.264}} & \mc{1}{c}{\scriptsize{0.197}} & \mc{1}{c}{\scriptsize{0.289}} & \mc{1}{c}{\scriptsize{-0.004}} & \mc{1}{c}{\scriptsize{-0.019}} & \mc{1}{c}{\scriptsize{0.039}} \\  

     &  & \mc{1}{c}{\scriptsize{(0.241)}} & \mc{1}{c}{\scriptsize{(0.373)}} & \mc{1}{c}{\scriptsize{(0.998)}} & \mc{1}{c}{\scriptsize{\textbf{(0.001)}}} & \mc{1}{c}{\scriptsize{\textbf{(0.012)}}} & \mc{1}{c}{\scriptsize{(0.490)}} & \mc{1}{c}{\scriptsize{(0.425)}} & \mc{1}{c}{\scriptsize{(0.276)}} \\  

     & \mc{1}{c}{\scriptsize{3}} & \mc{1}{c}{\scriptsize{0.150}} & \mc{1}{c}{\scriptsize{0.112}} & \mc{1}{c}{\scriptsize{0.261}} & \mc{1}{c}{\scriptsize{0.197}} & \mc{1}{c}{\scriptsize{0.210}} & \mc{1}{c}{\scriptsize{0.116}} & \mc{1}{c}{\scriptsize{0.068}} & \mc{1}{c}{\scriptsize{0.087}} \\  

     &  & \mc{1}{c}{\scriptsize{\textbf{(0.055)}}} & \mc{1}{c}{\scriptsize{(0.174)}} & \mc{1}{c}{\scriptsize{(0.999)}} & \mc{1}{c}{\scriptsize{\textbf{(0.001)}}} & \mc{1}{c}{\scriptsize{\textbf{(0.046)}}} & \mc{1}{c}{\scriptsize{(0.128)}} & \mc{1}{c}{\scriptsize{(0.263)}} & \mc{1}{c}{\scriptsize{(0.136)}} \\  

     & \mc{1}{c}{\scriptsize{4}} & \mc{1}{c}{\scriptsize{0.134}} & \mc{1}{c}{\scriptsize{0.146}} & \mc{1}{c}{\scriptsize{0.287}} & \mc{1}{c}{\scriptsize{0.268}} & \mc{1}{c}{\scriptsize{0.303}} & \mc{1}{c}{\scriptsize{0.090}} & \mc{1}{c}{\scriptsize{0.104}} & \mc{1}{c}{\scriptsize{0.071}} \\  

     &  & \mc{1}{c}{\scriptsize{\textbf{(0.061)}}} & \mc{1}{c}{\scriptsize{\textbf{(0.074)}}} & \mc{1}{c}{\scriptsize{\textbf{(0.001)}}} & \mc{1}{c}{\scriptsize{\textbf{(0.089)}}} & \mc{1}{c}{\scriptsize{\textbf{(0.007)}}} & \mc{1}{c}{\scriptsize{(0.157)}} & \mc{1}{c}{\scriptsize{(0.153)}} & \mc{1}{c}{\scriptsize{(0.163)}} \\  

     & \mc{1}{c}{\scriptsize{5}} & \mc{1}{c}{\scriptsize{0.111}} & \mc{1}{c}{\scriptsize{0.127}} & \mc{1}{c}{\scriptsize{0.311}} & \mc{1}{c}{\scriptsize{0.310}} & \mc{1}{c}{\scriptsize{0.359}} & \mc{1}{c}{\scriptsize{0.061}} & \mc{1}{c}{\scriptsize{0.090}} & \mc{1}{c}{\scriptsize{0.016}} \\  

     &  & \mc{1}{c}{\scriptsize{\textbf{(0.098)}}} & \mc{1}{c}{\scriptsize{(0.134)}} & \mc{1}{c}{\scriptsize{(0.998)}} & \mc{1}{c}{\scriptsize{\textbf{(0.004)}}} & \mc{1}{c}{\scriptsize{\textbf{(0.004)}}} & \mc{1}{c}{\scriptsize{(0.242)}} & \mc{1}{c}{\scriptsize{(0.230)}} & \mc{1}{c}{\scriptsize{(0.407)}} \\  

     & \mc{1}{c}{\scriptsize{21}} & \mc{1}{c}{\scriptsize{-0.058}} & \mc{1}{c}{\scriptsize{-0.005}} & \mc{1}{c}{\scriptsize{-0.086}} & \mc{1}{c}{\scriptsize{-0.131}} & \mc{1}{c}{\scriptsize{0.153}} & \mc{1}{c}{\scriptsize{-0.036}} & \mc{1}{c}{\scriptsize{0.043}} & \mc{1}{c}{\scriptsize{-0.089}} \\  

     &  & \mc{1}{c}{\scriptsize{(0.325)}} & \mc{1}{c}{\scriptsize{(0.478)}} & \mc{1}{c}{\scriptsize{(0.996)}} & \mc{1}{c}{\scriptsize{\textbf{(0.005)}}} & \mc{1}{c}{\scriptsize{(0.216)}} & \mc{1}{c}{\scriptsize{(0.404)}} & \mc{1}{c}{\scriptsize{(0.403)}} & \mc{1}{c}{\scriptsize{(0.172)}} \\  

    \mc{1}{l}{\scriptsize{Mother Works Factor}} & \mc{1}{c}{\scriptsize{2 to 21}} & \mc{1}{c}{\scriptsize{-0.341}} & \mc{1}{c}{\scriptsize{-0.271}} & \mc{1}{c}{\scriptsize{-0.932}} & \mc{1}{c}{\scriptsize{-0.795}} & \mc{1}{c}{\scriptsize{-0.796}} & \mc{1}{c}{\scriptsize{-0.182}} & \mc{1}{c}{\scriptsize{-0.119}} & \mc{1}{c}{\scriptsize{-0.128}} \\  

     &  & \mc{1}{c}{\scriptsize{\textbf{(0.088)}}} & \mc{1}{c}{\scriptsize{(0.187)}} & \mc{1}{c}{\scriptsize{\textbf{(0.002)}}} & \mc{1}{c}{\scriptsize{(0.994)}} & \mc{1}{c}{\scriptsize{\textbf{(0.037)}}} & \mc{1}{c}{\scriptsize{(0.213)}} & \mc{1}{c}{\scriptsize{(0.341)}} & \mc{1}{c}{\scriptsize{(0.238)}} \\  

  \bottomrule
  \end{tabular}
\end{center}

\begin{center}
	  \begin{tabular}{cccccccccc}
  \toprule

    \scriptsize{Variable} & \scriptsize{Age} & \scriptsize{(1)} & \scriptsize{(2)} & \scriptsize{(3)} & \scriptsize{(4)} & \scriptsize{(5)} & \scriptsize{(6)} & \scriptsize{(7)} & \scriptsize{(8)} \\ 
    \midrule  

    \mc{1}{l}{\scriptsize{Graduated High School}} & \mc{1}{c}{\scriptsize{30}} & \mc{1}{c}{\scriptsize{0.164}} & \mc{1}{c}{\scriptsize{0.094}} & \mc{1}{c}{\scriptsize{0.390}} & \mc{1}{c}{\scriptsize{0.324}} & \mc{1}{c}{\scriptsize{0.352}} & \mc{1}{c}{\scriptsize{0.103}} & \mc{1}{c}{\scriptsize{0.035}} & \mc{1}{c}{\scriptsize{0.059}} \\  

     &  & \mc{1}{c}{\scriptsize{\textbf{(0.022)}}} & \mc{1}{c}{\scriptsize{(0.144)}} & \mc{1}{c}{\scriptsize{\textbf{(0.003)}}} & \mc{1}{c}{\scriptsize{\textbf{(0.014)}}} & \mc{1}{c}{\scriptsize{\textbf{(0.002)}}} & \mc{1}{c}{\scriptsize{(0.112)}} & \mc{1}{c}{\scriptsize{(0.363)}} & \mc{1}{c}{\scriptsize{(0.268)}} \\  

    \mc{1}{l}{\scriptsize{Attended Voc./Tech./Com. College}} & \mc{1}{c}{\scriptsize{30}} & \mc{1}{c}{\scriptsize{-0.091}} & \mc{1}{c}{\scriptsize{-0.099}} & \mc{1}{c}{\scriptsize{0.000}} & \mc{1}{c}{\scriptsize{0.093}} & \mc{1}{c}{\scriptsize{-0.044}} & \mc{1}{c}{\scriptsize{-0.100}} & \mc{1}{c}{\scriptsize{-0.156}} & \mc{1}{c}{\scriptsize{-0.152}} \\  

     &  & \mc{1}{c}{\scriptsize{(0.145)}} & \mc{1}{c}{\scriptsize{(0.154)}} & \mc{1}{c}{\scriptsize{(0.497)}} & \mc{1}{c}{\scriptsize{(0.277)}} & \mc{1}{c}{\scriptsize{(0.378)}} & \mc{1}{c}{\scriptsize{(0.138)}} & \mc{1}{c}{\scriptsize{\textbf{(0.051)}}} & \mc{1}{c}{\scriptsize{\textbf{(0.070)}}} \\  

    \mc{1}{l}{\scriptsize{Graduated 4-year College}} & \mc{1}{c}{\scriptsize{30}} & \mc{1}{c}{\scriptsize{0.161}} & \mc{1}{c}{\scriptsize{0.116}} & \mc{1}{c}{\scriptsize{0.188}} & \mc{1}{c}{\scriptsize{0.139}} & \mc{1}{c}{\scriptsize{0.176}} & \mc{1}{c}{\scriptsize{0.148}} & \mc{1}{c}{\scriptsize{0.109}} & \mc{1}{c}{\scriptsize{0.120}} \\  

     &  & \mc{1}{c}{\scriptsize{\textbf{(0.011)}}} & \mc{1}{c}{\scriptsize{\textbf{(0.071)}}} & \mc{1}{c}{\scriptsize{\textbf{(0.010)}}} & \mc{1}{c}{\scriptsize{\textbf{(0.075)}}} & \mc{1}{c}{\scriptsize{\textbf{(0.020)}}} & \mc{1}{c}{\scriptsize{\textbf{(0.017)}}} & \mc{1}{c}{\scriptsize{(0.117)}} & \mc{1}{c}{\scriptsize{\textbf{(0.073)}}} \\  

    \mc{1}{l}{\scriptsize{Years of Edu.}} & \mc{1}{c}{\scriptsize{30}} & \mc{1}{c}{\scriptsize{1.367}} & \mc{1}{c}{\scriptsize{1.062}} & \mc{1}{c}{\scriptsize{2.513}} & \mc{1}{c}{\scriptsize{2.188}} & \mc{1}{c}{\scriptsize{2.423}} & \mc{1}{c}{\scriptsize{0.986}} & \mc{1}{c}{\scriptsize{0.804}} & \mc{1}{c}{\scriptsize{0.886}} \\  

     &  & \mc{1}{c}{\scriptsize{\textbf{(0.001)}}} & \mc{1}{c}{\scriptsize{\textbf{(0.006)}}} & \mc{1}{c}{\scriptsize{\textbf{(0.000)}}} & \mc{1}{c}{\scriptsize{\textbf{(0.002)}}} & \mc{1}{c}{\scriptsize{\textbf{(0.000)}}} & \mc{1}{c}{\scriptsize{\textbf{(0.009)}}} & \mc{1}{c}{\scriptsize{\textbf{(0.033)}}} & \mc{1}{c}{\scriptsize{\textbf{(0.018)}}} \\  

    \mc{1}{l}{\scriptsize{Ever Had Special Education by Grade 5}} & \mc{1}{c}{\scriptsize{21}} & \mc{1}{c}{\scriptsize{0.001}} & \mc{1}{c}{\scriptsize{0.001}} & \mc{1}{c}{\scriptsize{0.153}} & \mc{1}{c}{\scriptsize{0.075}} & \mc{1}{c}{\scriptsize{0.127}} & \mc{1}{c}{\scriptsize{-0.030}} & \mc{1}{c}{\scriptsize{-0.026}} & \mc{1}{c}{\scriptsize{-0.040}} \\  

     &  & \mc{1}{c}{\scriptsize{(0.508)}} & \mc{1}{c}{\scriptsize{(0.499)}} & \mc{1}{c}{\scriptsize{(0.160)}} & \mc{1}{c}{\scriptsize{(0.322)}} & \mc{1}{c}{\scriptsize{(0.207)}} & \mc{1}{c}{\scriptsize{(0.367)}} & \mc{1}{c}{\scriptsize{(0.392)}} & \mc{1}{c}{\scriptsize{(0.339)}} \\  

    \mc{1}{l}{\scriptsize{Total Number of Special Education by Grade 5}} & \mc{1}{c}{\scriptsize{21}} & \mc{1}{c}{\scriptsize{-0.547}} & \mc{1}{c}{\scriptsize{-0.468}} & \mc{1}{c}{\scriptsize{0.977}} & \mc{1}{c}{\scriptsize{0.549}} & \mc{1}{c}{\scriptsize{0.975}} & \mc{1}{c}{\scriptsize{-0.844}} & \mc{1}{c}{\scriptsize{-0.653}} & \mc{1}{c}{\scriptsize{-0.852}} \\  

     &  & \mc{1}{c}{\scriptsize{(0.222)}} & \mc{1}{c}{\scriptsize{(0.256)}} & \mc{1}{c}{\scriptsize{(0.131)}} & \mc{1}{c}{\scriptsize{(0.286)}} & \mc{1}{c}{\scriptsize{(0.137)}} & \mc{1}{c}{\scriptsize{(0.136)}} & \mc{1}{c}{\scriptsize{(0.209)}} & \mc{1}{c}{\scriptsize{(0.161)}} \\  

    \mc{1}{l}{\scriptsize{Ever Retained by Grade 5}} & \mc{1}{c}{\scriptsize{21}} & \mc{1}{c}{\scriptsize{-0.170}} & \mc{1}{c}{\scriptsize{-0.211}} & \mc{1}{c}{\scriptsize{-0.175}} & \mc{1}{c}{\scriptsize{-0.248}} & \mc{1}{c}{\scriptsize{-0.176}} & \mc{1}{c}{\scriptsize{-0.170}} & \mc{1}{c}{\scriptsize{-0.209}} & \mc{1}{c}{\scriptsize{-0.183}} \\  

     &  & \mc{1}{c}{\scriptsize{\textbf{(0.015)}}} & \mc{1}{c}{\scriptsize{\textbf{(0.005)}}} & \mc{1}{c}{\scriptsize{(0.117)}} & \mc{1}{c}{\scriptsize{\textbf{(0.072)}}} & \mc{1}{c}{\scriptsize{(0.118)}} & \mc{1}{c}{\scriptsize{\textbf{(0.025)}}} & \mc{1}{c}{\scriptsize{\textbf{(0.007)}}} & \mc{1}{c}{\scriptsize{\textbf{(0.027)}}} \\  

    \mc{1}{l}{\scriptsize{Total Number of Retention by Grade 5}} & \mc{1}{c}{\scriptsize{21}} & \mc{1}{c}{\scriptsize{-0.152}} & \mc{1}{c}{\scriptsize{-0.144}} & \mc{1}{c}{\scriptsize{-0.086}} & \mc{1}{c}{\scriptsize{-0.111}} & \mc{1}{c}{\scriptsize{-0.068}} & \mc{1}{c}{\scriptsize{-0.156}} & \mc{1}{c}{\scriptsize{-0.152}} & \mc{1}{c}{\scriptsize{-0.156}} \\  

     &  & \mc{1}{c}{\scriptsize{\textbf{(0.076)}}} & \mc{1}{c}{\scriptsize{\textbf{(0.080)}}} & \mc{1}{c}{\scriptsize{(0.298)}} & \mc{1}{c}{\scriptsize{(0.287)}} & \mc{1}{c}{\scriptsize{(0.341)}} & \mc{1}{c}{\scriptsize{\textbf{(0.095)}}} & \mc{1}{c}{\scriptsize{\textbf{(0.085)}}} & \mc{1}{c}{\scriptsize{(0.126)}} \\  

    \mc{1}{l}{\scriptsize{Education Factor}} & \mc{1}{c}{\scriptsize{21 to 30}} & \mc{1}{c}{\scriptsize{0.449}} & \mc{1}{c}{\scriptsize{0.396}} & \mc{1}{c}{\scriptsize{0.557}} & \mc{1}{c}{\scriptsize{0.582}} & \mc{1}{c}{\scriptsize{0.504}} & \mc{1}{c}{\scriptsize{0.380}} & \mc{1}{c}{\scriptsize{0.346}} & \mc{1}{c}{\scriptsize{0.331}} \\  

     &  & \mc{1}{c}{\scriptsize{\textbf{(0.007)}}} & \mc{1}{c}{\scriptsize{\textbf{(0.014)}}} & \mc{1}{c}{\scriptsize{\textbf{(0.022)}}} & \mc{1}{c}{\scriptsize{\textbf{(0.028)}}} & \mc{1}{c}{\scriptsize{\textbf{(0.035)}}} & \mc{1}{c}{\scriptsize{\textbf{(0.030)}}} & \mc{1}{c}{\scriptsize{\textbf{(0.032)}}} & \mc{1}{c}{\scriptsize{\textbf{(0.081)}}} \\  

  \bottomrule
  \end{tabular}
\end{center}

\begin{center}
	  \begin{tabular}{cccccccccc}
  \toprule

    \scriptsize{Variable} & \scriptsize{Age} & \scriptsize{(1)} & \scriptsize{(2)} & \scriptsize{(3)} & \scriptsize{(4)} & \scriptsize{(5)} & \scriptsize{(6)} & \scriptsize{(7)} & \scriptsize{(8)} \\ 
    \midrule  

    \mc{1}{l}{\scriptsize{Ever Adopted}} &  & \mc{1}{c}{\scriptsize{0.036}} & \mc{1}{c}{\scriptsize{0.030}} & \mc{1}{c}{\scriptsize{-0.148}} & \mc{1}{c}{\scriptsize{-0.179}} & \mc{1}{c}{\scriptsize{-0.177}} & \mc{1}{c}{\scriptsize{0.074}} & \mc{1}{c}{\scriptsize{0.067}} & \mc{1}{c}{\scriptsize{0.052}} \\  

     &  & \mc{1}{c}{\scriptsize{(0.132)}} & \mc{1}{c}{\scriptsize{(0.263)}} & \mc{1}{c}{\scriptsize{(0.842)}} & \mc{1}{c}{\scriptsize{(0.882)}} & \mc{1}{c}{\scriptsize{(0.882)}} & \mc{1}{c}{\scriptsize{\textbf{(0.026)}}} & \mc{1}{c}{\scriptsize{\textbf{(0.039)}}} & \mc{1}{c}{\scriptsize{\textbf{(0.053)}}} \\ 
    \midrule  

    \mc{2}{l}{\scriptsize{\% of Pos. TE ($H_0$: $\le$ 50\%)}} & \mc{1}{c}{\scriptsize{100}} & \mc{1}{c}{\scriptsize{100}} & \mc{1}{c}{\scriptsize{0}} & \mc{1}{c}{\scriptsize{0}} & \mc{1}{c}{\scriptsize{0}} & \mc{1}{c}{\scriptsize{100}} & \mc{1}{c}{\scriptsize{100}} & \mc{1}{c}{\scriptsize{100}} \\  

     &  & \mc{1}{c}{\scriptsize{\textbf{(0.000)}}} & \mc{1}{c}{\scriptsize{\textbf{(0.000)}}} & \mc{1}{c}{\scriptsize{(1.000)}} & \mc{1}{c}{\scriptsize{(1.000)}} & \mc{1}{c}{\scriptsize{(0.974)}} & \mc{1}{c}{\scriptsize{\textbf{(0.000)}}} & \mc{1}{c}{\scriptsize{\textbf{(0.000)}}} & \mc{1}{c}{\scriptsize{\textbf{(0.000)}}} \\  

    \mc{2}{l}{\scriptsize{\% of Pos. TE ($H_0$: $\le$ 10\% $|$ 10\% Significance)}} & \mc{1}{c}{\scriptsize{0}} & \mc{1}{c}{\scriptsize{0}} & \mc{1}{c}{\scriptsize{0}} & \mc{1}{c}{\scriptsize{0}} & \mc{1}{c}{\scriptsize{0}} & \mc{1}{c}{\scriptsize{100}} & \mc{1}{c}{\scriptsize{100}} & \mc{1}{c}{\scriptsize{100}} \\  

     &  & \mc{1}{c}{\scriptsize{(0.145)}} & \mc{1}{c}{\scriptsize{(0.145)}} & \mc{1}{c}{\scriptsize{(1.000)}} & \mc{1}{c}{\scriptsize{(1.000)}} & \mc{1}{c}{\scriptsize{(0.974)}} & \mc{1}{c}{\scriptsize{\textbf{(0.000)}}} & \mc{1}{c}{\scriptsize{\textbf{(0.000)}}} & \mc{1}{c}{\scriptsize{\textbf{(0.000)}}} \\  

  \bottomrule
  \end{tabular}
\end{center}

\begin{center}
	\begin{table}[H]
\captionsetup{singlelinecheck=false,justification=centering}
\caption{ABC/CARE Average Treatment Effects, Males and Females \\ Crime \label{tab:ate_pooled_apx8}}

  \begin{threeparttable}
  \begin{tabular}{cccccccccc}
  \hline\hline

     &  & \scriptsize{(1)} & \scriptsize{(2)} & \scriptsize{(3)} & \scriptsize{(4)} & \scriptsize{(5)} & \scriptsize{(6)} & \scriptsize{(7)} & \scriptsize{(8)} \\  

     &  &  &  & \mc{3}{c}{\scriptsize{$P=0$}} & \mc{3}{c}{\scriptsize{$P=1$}} \\ 
    \cmidrule(lr){5-7} \cmidrule(lr){8-10} 

    \scriptsize{Variable} & \scriptsize{Age} & \scriptsize{ITT} & \scriptsize{ITT$|X,W$} & \scriptsize{ITT} & \scriptsize{ITT$|X,W$} & \scriptsize{KE$|X,W$} & \scriptsize{ITT} & \scriptsize{ITT$|X,W$} & \scriptsize{KE$|X,W$} \\ 
    \hline  

    \mc{1}{l}{\scriptsize{Total Felony Arrests}} & \mc{1}{c}{\scriptsize{Mid-30s}} & \mc{1}{c}{\scriptsize{0.007}} & \mc{1}{c}{\scriptsize{0.047}} & \mc{1}{c}{\scriptsize{-0.169}} & \mc{1}{c}{\scriptsize{-0.001}} & \mc{1}{c}{\scriptsize{-0.074}} & \mc{1}{c}{\scriptsize{0.088}} & \mc{1}{c}{\scriptsize{0.156}} & \mc{1}{c}{\scriptsize{0.295}} \\  

     &  & \mc{1}{c}{\scriptsize{(0.510)}} & \mc{1}{c}{\scriptsize{(0.569)}} & \mc{1}{c}{\scriptsize{(0.431)}} & \mc{1}{c}{\scriptsize{(0.569)}} & \mc{1}{c}{\scriptsize{(0.451)}} & \mc{1}{c}{\scriptsize{(0.627)}} & \mc{1}{c}{\scriptsize{(0.627)}} & \mc{1}{c}{\scriptsize{(0.804)}} \\  

    \mc{1}{l}{\scriptsize{Total Misdemeanor Arrests}} & \mc{1}{c}{\scriptsize{Mid-30s}} & \mc{1}{c}{\scriptsize{-0.767}} & \mc{1}{c}{\scriptsize{-0.797}} & \mc{1}{c}{\scriptsize{-1.339}} & \mc{1}{c}{\scriptsize{-1.526}} & \mc{1}{c}{\scriptsize{-1.512}} & \mc{1}{c}{\scriptsize{-0.505}} & \mc{1}{c}{\scriptsize{-0.497}} & \mc{1}{c}{\scriptsize{-0.203}} \\  

     &  & \mc{1}{c}{\scriptsize{\textbf{(0.000)}}} & \mc{1}{c}{\scriptsize{\textbf{(0.020)}}} & \mc{1}{c}{\scriptsize{\textbf{(0.020)}}} & \mc{1}{c}{\scriptsize{\textbf{(0.039)}}} & \mc{1}{c}{\scriptsize{\textbf{(0.000)}}} & \mc{1}{c}{\scriptsize{\textbf{(0.078)}}} & \mc{1}{c}{\scriptsize{\textbf{(0.078)}}} & \mc{1}{c}{\scriptsize{(0.176)}} \\  

    \mc{1}{l}{\scriptsize{Total Years Incarcerated}} & \mc{1}{c}{\scriptsize{30}} & \mc{1}{c}{\scriptsize{0.033}} & \mc{1}{c}{\scriptsize{0.083}} & \mc{1}{c}{\scriptsize{0.092}} & \mc{1}{c}{\scriptsize{0.136}} & \mc{1}{c}{\scriptsize{0.167}} & \mc{1}{c}{\scriptsize{0.001}} & \mc{1}{c}{\scriptsize{0.052}} & \mc{1}{c}{\scriptsize{0.084}} \\  

     &  & \mc{1}{c}{\scriptsize{(0.647)}} & \mc{1}{c}{\scriptsize{(0.804)}} & \mc{1}{c}{\scriptsize{(0.667)}} & \mc{1}{c}{\scriptsize{(0.804)}} & \mc{1}{c}{\scriptsize{(0.784)}} & \mc{1}{c}{\scriptsize{(0.510)}} & \mc{1}{c}{\scriptsize{(0.706)}} & \mc{1}{c}{\scriptsize{(0.765)}} \\  

    \mc{1}{l}{\scriptsize{Crime Factor}} & \mc{1}{c}{\scriptsize{30 to Mid-30s}} & \mc{1}{c}{\scriptsize{-0.054}} & \mc{1}{c}{\scriptsize{-0.027}} & \mc{1}{c}{\scriptsize{-0.143}} & \mc{1}{c}{\scriptsize{-0.098}} & \mc{1}{c}{\scriptsize{-0.113}} & \mc{1}{c}{\scriptsize{-0.013}} & \mc{1}{c}{\scriptsize{0.025}} & \mc{1}{c}{\scriptsize{0.083}} \\  

     &  & \mc{1}{c}{\scriptsize{(0.353)}} & \mc{1}{c}{\scriptsize{(0.373)}} & \mc{1}{c}{\scriptsize{(0.314)}} & \mc{1}{c}{\scriptsize{(0.333)}} & \mc{1}{c}{\scriptsize{(0.412)}} & \mc{1}{c}{\scriptsize{(0.431)}} & \mc{1}{c}{\scriptsize{(0.569)}} & \mc{1}{c}{\scriptsize{(0.706)}} \\ 
    \hline  

    \\[0.1cm]
    \mc{2}{l}{\scriptsize{\% of Pos. TE ($H_0$: $\le$ 25\% $|$ 10\% Significance)}} & \mc{1}{c}{\scriptsize{25}} & \mc{1}{c}{\scriptsize{25}} & \mc{1}{c}{\scriptsize{25}} & \mc{1}{c}{\scriptsize{25}} & \mc{1}{c}{\scriptsize{25}} & \mc{1}{c}{\scriptsize{25}} & \mc{1}{c}{\scriptsize{25}} & \mc{1}{c}{\scriptsize{0}} \\  

     &  & \mc{1}{c}{\scriptsize{(0.451)}} & \mc{1}{c}{\scriptsize{(0.333)}} & \mc{1}{c}{\scriptsize{(0.471)}} & \mc{1}{c}{\scriptsize{(0.412)}} & \mc{1}{c}{\scriptsize{(0.431)}} & \mc{1}{c}{\scriptsize{(0.255)}} & \mc{1}{c}{\scriptsize{(0.255)}} & \mc{1}{c}{\scriptsize{(1.000)}} \\  

    \mc{2}{l}{\scriptsize{\% of Pos. TE ($H_0$: $\le$ 50\% $|$ 10\% Significance)}} & \mc{1}{c}{\scriptsize{25}} & \mc{1}{c}{\scriptsize{25}} & \mc{1}{c}{\scriptsize{25}} & \mc{1}{c}{\scriptsize{25}} & \mc{1}{c}{\scriptsize{25}} & \mc{1}{c}{\scriptsize{25}} & \mc{1}{c}{\scriptsize{25}} & \mc{1}{c}{\scriptsize{0}} \\  

     &  & \mc{1}{c}{\scriptsize{(0.902)}} & \mc{1}{c}{\scriptsize{(0.824)}} & \mc{1}{c}{\scriptsize{(0.804)}} & \mc{1}{c}{\scriptsize{(0.804)}} & \mc{1}{c}{\scriptsize{(0.824)}} & \mc{1}{c}{\scriptsize{(0.765)}} & \mc{1}{c}{\scriptsize{(0.686)}} & \mc{1}{c}{\scriptsize{(1.000)}} \\  

    \mc{2}{l}{\scriptsize{\% of Pos. TE ($H_0$: $\le$ 75\% $|$ 10\% Significance)}} & \mc{1}{c}{\scriptsize{25}} & \mc{1}{c}{\scriptsize{25}} & \mc{1}{c}{\scriptsize{25}} & \mc{1}{c}{\scriptsize{25}} & \mc{1}{c}{\scriptsize{25}} & \mc{1}{c}{\scriptsize{25}} & \mc{1}{c}{\scriptsize{25}} & \mc{1}{c}{\scriptsize{0}} \\  

     &  & \mc{1}{c}{\scriptsize{(1.000)}} & \mc{1}{c}{\scriptsize{(1.000)}} & \mc{1}{c}{\scriptsize{(1.000)}} & \mc{1}{c}{\scriptsize{(1.000)}} & \mc{1}{c}{\scriptsize{(1.000)}} & \mc{1}{c}{\scriptsize{(1.000)}} & \mc{1}{c}{\scriptsize{(1.000)}} & \mc{1}{c}{\scriptsize{(1.000)}} \\  

  \hline\hline
  \end{tabular}
    \begin{tablenotes}
    \scriptsize
    \item 
Note: This table displays various estimates of the treatment effect of ABC/CARE's center-based care.
Column (1) displays the ITT, without accounting for any controls.
Column (2) displays the ITT conditioning on vector of controls, $X$, consisting of APGAR scores 1 
minute after birth, an indicator for the subject being born prematurely, and an indicator for the 
father being home at baseline. We also apply IPW weights, $W$, to account for attrition.
Columns (3)--(4) are analogous to columns (1)--(2), but we restrict the control sample to subjects
who did not enroll in any alternative care.
Column (5) displys the matching estimate, where we use the Mahalanobis metric and Epanechnikov kernel
to match on controls $X$ listed above, and restrict the control sample to subjects who did not enroll
in any alternative care. Additionally, we apply IPW weights, $W$.
Columns (6)--(8) are analogous to Columns (3)--(5), except we restrict the control sample to subejcts
who did enroll in alternative care. 
The final three pairs of rows display the proportion of treatment effects in the table that are 
socially positive. The first row in each pair displays the percentage of treatment effects, and the
second row presents the inference.

Numbers in parentheses represent the $p$-value from a single hypothesis test, and are obtained from 
the empirical bootstrap distribution generated by 200 resamples of the original data. 
Bold $p$-values indicate significance at the 10\% level.
Blank point estimates indicate that we are unable to obtain estimates due to a lack of support in the data. 

    \end{tablenotes}
  \end{threeparttable}

\end{table}
\end{center}

\begin{center}
	  \begin{tabular}{cccccccccc}
  \toprule

    \scriptsize{Variable} & \scriptsize{Age} & \scriptsize{(1)} & \scriptsize{(2)} & \scriptsize{(3)} & \scriptsize{(4)} & \scriptsize{(5)} & \scriptsize{(6)} & \scriptsize{(7)} & \scriptsize{(8)} \\ 
    \midrule  

    \mc{1}{l}{\scriptsize{Cig. Smoked per day last month}} & \mc{1}{c}{\scriptsize{30}} & \mc{1}{c}{\scriptsize{0.033}} & \mc{1}{c}{\scriptsize{0.261}} & \mc{1}{c}{\scriptsize{-0.826}} & \mc{1}{c}{\scriptsize{-0.586}} & \mc{1}{c}{\scriptsize{-0.796}} & \mc{1}{c}{\scriptsize{0.434}} & \mc{1}{c}{\scriptsize{0.667}} & \mc{1}{c}{\scriptsize{0.434}} \\  

     &  & \mc{1}{c}{\scriptsize{(0.500)}} & \mc{1}{c}{\scriptsize{(0.414)}} & \mc{1}{c}{\scriptsize{(0.299)}} & \mc{1}{c}{\scriptsize{(0.351)}} & \mc{1}{c}{\scriptsize{(0.301)}} & \mc{1}{c}{\scriptsize{(0.388)}} & \mc{1}{c}{\scriptsize{(0.293)}} & \mc{1}{c}{\scriptsize{(0.374)}} \\  

    \mc{1}{l}{\scriptsize{Days drank alcohol last month}} & \mc{1}{c}{\scriptsize{30}} & \mc{1}{c}{\scriptsize{0.244}} & \mc{1}{c}{\scriptsize{0.242}} & \mc{1}{c}{\scriptsize{-0.156}} & \mc{1}{c}{\scriptsize{-0.561}} & \mc{1}{c}{\scriptsize{0.123}} & \mc{1}{c}{\scriptsize{0.208}} & \mc{1}{c}{\scriptsize{0.347}} & \mc{1}{c}{\scriptsize{0.628}} \\  

     &  & \mc{1}{c}{\scriptsize{(0.407)}} & \mc{1}{c}{\scriptsize{(0.410)}} & \mc{1}{c}{\scriptsize{(0.430)}} & \mc{1}{c}{\scriptsize{(0.385)}} & \mc{1}{c}{\scriptsize{(0.493)}} & \mc{1}{c}{\scriptsize{(0.453)}} & \mc{1}{c}{\scriptsize{(0.399)}} & \mc{1}{c}{\scriptsize{(0.337)}} \\  

    \mc{1}{l}{\scriptsize{Days binge drank alcohol last month}} & \mc{1}{c}{\scriptsize{30}} & \mc{1}{c}{\scriptsize{0.085}} & \mc{1}{c}{\scriptsize{0.391}} & \mc{1}{c}{\scriptsize{-0.267}} & \mc{1}{c}{\scriptsize{-0.115}} & \mc{1}{c}{\scriptsize{-0.120}} & \mc{1}{c}{\scriptsize{0.151}} & \mc{1}{c}{\scriptsize{0.580}} & \mc{1}{c}{\scriptsize{0.392}} \\  

     &  & \mc{1}{c}{\scriptsize{(0.425)}} & \mc{1}{c}{\scriptsize{(0.227)}} & \mc{1}{c}{\scriptsize{(0.358)}} & \mc{1}{c}{\scriptsize{(0.448)}} & \mc{1}{c}{\scriptsize{(0.407)}} & \mc{1}{c}{\scriptsize{(0.384)}} & \mc{1}{c}{\scriptsize{(0.131)}} & \mc{1}{c}{\scriptsize{(0.221)}} \\  

    \mc{1}{l}{\scriptsize{Self-reported drug user}} & \mc{1}{c}{\scriptsize{Mid-30s}} & \mc{1}{c}{\scriptsize{-0.142}} & \mc{1}{c}{\scriptsize{-0.148}} & \mc{1}{c}{\scriptsize{-0.253}} & \mc{1}{c}{\scriptsize{-0.365}} & \mc{1}{c}{\scriptsize{-0.275}} & \mc{1}{c}{\scriptsize{-0.090}} & \mc{1}{c}{\scriptsize{-0.076}} & \mc{1}{c}{\scriptsize{-0.114}} \\  

     &  & \mc{1}{c}{\scriptsize{\textbf{(0.059)}}} & \mc{1}{c}{\scriptsize{\textbf{(0.062)}}} & \mc{1}{c}{\scriptsize{\textbf{(0.080)}}} & \mc{1}{c}{\scriptsize{\textbf{(0.017)}}} & \mc{1}{c}{\scriptsize{\textbf{(0.074)}}} & \mc{1}{c}{\scriptsize{(0.187)}} & \mc{1}{c}{\scriptsize{(0.211)}} & \mc{1}{c}{\scriptsize{(0.129)}} \\  

    \mc{1}{l}{\scriptsize{Substance Use Factor}} & \mc{1}{c}{\scriptsize{30 to Mid-30s}} & \mc{1}{c}{\scriptsize{0.169}} & \mc{1}{c}{\scriptsize{0.221}} & \mc{1}{c}{\scriptsize{0.339}} & \mc{1}{c}{\scriptsize{0.286}} & \mc{1}{c}{\scriptsize{0.375}} & \mc{1}{c}{\scriptsize{0.141}} & \mc{1}{c}{\scriptsize{0.267}} & \mc{1}{c}{\scriptsize{0.201}} \\  

     &  & \mc{1}{c}{\scriptsize{(0.236)}} & \mc{1}{c}{\scriptsize{(0.218)}} & \mc{1}{c}{\scriptsize{(0.181)}} & \mc{1}{c}{\scriptsize{(0.221)}} & \mc{1}{c}{\scriptsize{(0.162)}} & \mc{1}{c}{\scriptsize{(0.288)}} & \mc{1}{c}{\scriptsize{(0.196)}} & \mc{1}{c}{\scriptsize{(0.260)}} \\  

  \bottomrule
  \end{tabular}
\end{center}

\begin{center}
	  \begin{tabular}{cccccccccc}
  \toprule

    \scriptsize{Variable} & \scriptsize{Age} & \scriptsize{(1)} & \scriptsize{(2)} & \scriptsize{(3)} & \scriptsize{(4)} & \scriptsize{(5)} & \scriptsize{(6)} & \scriptsize{(7)} & \scriptsize{(8)} \\ 
    \midrule  

    \mc{1}{l}{\scriptsize{Hypertension Factor}} & \mc{1}{c}{\scriptsize{Mid-30s}} & \mc{1}{c}{\scriptsize{-0.180}} & \mc{1}{c}{\scriptsize{-0.313}} & \mc{1}{c}{\scriptsize{0.004}} & \mc{1}{c}{\scriptsize{0.226}} & \mc{1}{c}{\scriptsize{0.004}} & \mc{1}{c}{\scriptsize{-0.284}} & \mc{1}{c}{\scriptsize{-0.470}} & \mc{1}{c}{\scriptsize{-0.405}} \\  

     &  & \mc{1}{c}{\scriptsize{(0.105)}} & \mc{1}{c}{\scriptsize{\textbf{(0.079)}}} & \mc{1}{c}{\scriptsize{(0.539)}} & \mc{1}{c}{\scriptsize{(0.711)}} & \mc{1}{c}{\scriptsize{(0.566)}} & \mc{1}{c}{\scriptsize{\textbf{(0.039)}}} & \mc{1}{c}{\scriptsize{\textbf{(0.013)}}} & \mc{1}{c}{\scriptsize{\textbf{(0.039)}}} \\  

    \mc{1}{l}{\scriptsize{Diastolic Blood Pressure (mm Hg)}} & \mc{1}{c}{\scriptsize{Mid-30s}} & \mc{1}{c}{\scriptsize{-3.719}} & \mc{1}{c}{\scriptsize{-5.986}} & \mc{1}{c}{\scriptsize{-2.250}} & \mc{1}{c}{\scriptsize{-0.551}} & \mc{1}{c}{\scriptsize{-2.493}} & \mc{1}{c}{\scriptsize{-4.860}} & \mc{1}{c}{\scriptsize{-7.229}} & \mc{1}{c}{\scriptsize{-6.471}} \\  

     &  & \mc{1}{c}{\scriptsize{\textbf{(0.079)}}} & \mc{1}{c}{\scriptsize{\textbf{(0.026)}}} & \mc{1}{c}{\scriptsize{(0.224)}} & \mc{1}{c}{\scriptsize{(0.474)}} & \mc{1}{c}{\scriptsize{(0.197)}} & \mc{1}{c}{\scriptsize{\textbf{(0.039)}}} & \mc{1}{c}{\scriptsize{\textbf{(0.039)}}} & \mc{1}{c}{\scriptsize{\textbf{(0.066)}}} \\  

    \mc{1}{l}{\scriptsize{Hypertension}} & \mc{1}{c}{\scriptsize{Mid-30s}} & \mc{1}{c}{\scriptsize{0.027}} & \mc{1}{c}{\scriptsize{-0.021}} & \mc{1}{c}{\scriptsize{0.042}} & \mc{1}{c}{\scriptsize{0.236}} & \mc{1}{c}{\scriptsize{0.020}} & \mc{1}{c}{\scriptsize{0.012}} & \mc{1}{c}{\scriptsize{-0.044}} & \mc{1}{c}{\scriptsize{-0.046}} \\  

     &  & \mc{1}{c}{\scriptsize{(0.592)}} & \mc{1}{c}{\scriptsize{(0.382)}} & \mc{1}{c}{\scriptsize{(0.579)}} & \mc{1}{c}{\scriptsize{(0.882)}} & \mc{1}{c}{\scriptsize{(0.500)}} & \mc{1}{c}{\scriptsize{(0.513)}} & \mc{1}{c}{\scriptsize{(0.382)}} & \mc{1}{c}{\scriptsize{(0.382)}} \\  

    \mc{1}{l}{\scriptsize{Prehypertension}} & \mc{1}{c}{\scriptsize{Mid-30s}} & \mc{1}{c}{\scriptsize{-0.135}} & \mc{1}{c}{\scriptsize{-0.093}} & \mc{1}{c}{\scriptsize{-0.083}} & \mc{1}{c}{\scriptsize{-0.001}} & \mc{1}{c}{\scriptsize{-0.023}} & \mc{1}{c}{\scriptsize{-0.176}} & \mc{1}{c}{\scriptsize{-0.191}} & \mc{1}{c}{\scriptsize{-0.214}} \\  

     &  & \mc{1}{c}{\scriptsize{\textbf{(0.013)}}} & \mc{1}{c}{\scriptsize{(0.132)}} & \mc{1}{c}{\scriptsize{(0.250)}} & \mc{1}{c}{\scriptsize{(0.553)}} & \mc{1}{c}{\scriptsize{(0.461)}} & \mc{1}{c}{\scriptsize{\textbf{(0.000)}}} & \mc{1}{c}{\scriptsize{\textbf{(0.013)}}} & \mc{1}{c}{\scriptsize{\textbf{(0.013)}}} \\  

    \mc{1}{l}{\scriptsize{Systolic Blood Pressure (mm Hg)}} & \mc{1}{c}{\scriptsize{Mid-30s}} & \mc{1}{c}{\scriptsize{-3.100}} & \mc{1}{c}{\scriptsize{-6.448}} & \mc{1}{c}{\scriptsize{3.312}} & \mc{1}{c}{\scriptsize{7.119}} & \mc{1}{c}{\scriptsize{3.303}} & \mc{1}{c}{\scriptsize{-6.030}} & \mc{1}{c}{\scriptsize{-11.018}} & \mc{1}{c}{\scriptsize{-8.312}} \\  

     &  & \mc{1}{c}{\scriptsize{(0.171)}} & \mc{1}{c}{\scriptsize{(0.132)}} & \mc{1}{c}{\scriptsize{(0.789)}} & \mc{1}{c}{\scriptsize{(0.803)}} & \mc{1}{c}{\scriptsize{(0.763)}} & \mc{1}{c}{\scriptsize{\textbf{(0.039)}}} & \mc{1}{c}{\scriptsize{\textbf{(0.013)}}} & \mc{1}{c}{\scriptsize{\textbf{(0.026)}}} \\ 
    \midrule  

    \mc{2}{l}{\scriptsize{\% of Pos. TE ($H_0$: $\le$ 50\%)}} & \mc{1}{c}{\scriptsize{80}} & \mc{1}{c}{\scriptsize{100}} & \mc{1}{c}{\scriptsize{40}} & \mc{1}{c}{\scriptsize{40}} & \mc{1}{c}{\scriptsize{40}} & \mc{1}{c}{\scriptsize{80}} & \mc{1}{c}{\scriptsize{100}} & \mc{1}{c}{\scriptsize{100}} \\  

     &  & \mc{1}{c}{\scriptsize{\textbf{(0.000)}}} & \mc{1}{c}{\scriptsize{\textbf{(0.000)}}} & \mc{1}{c}{\scriptsize{(0.684)}} & \mc{1}{c}{\scriptsize{(0.500)}} & \mc{1}{c}{\scriptsize{(0.539)}} & \mc{1}{c}{\scriptsize{\textbf{(0.000)}}} & \mc{1}{c}{\scriptsize{\textbf{(0.000)}}} & \mc{1}{c}{\scriptsize{\textbf{(0.000)}}} \\  

    \mc{2}{l}{\scriptsize{\% of Pos. TE ($H_0$: $\le$ 10\% $|$ 10\% Significance)}} & \mc{1}{c}{\scriptsize{40}} & \mc{1}{c}{\scriptsize{40}} & \mc{1}{c}{\scriptsize{0}} & \mc{1}{c}{\scriptsize{0}} & \mc{1}{c}{\scriptsize{0}} & \mc{1}{c}{\scriptsize{80}} & \mc{1}{c}{\scriptsize{80}} & \mc{1}{c}{\scriptsize{80}} \\  

     &  & \mc{1}{c}{\scriptsize{(0.237)}} & \mc{1}{c}{\scriptsize{(0.237)}} & \mc{1}{c}{\scriptsize{(1.000)}} & \mc{1}{c}{\scriptsize{(1.000)}} & \mc{1}{c}{\scriptsize{(1.000)}} & \mc{1}{c}{\scriptsize{\textbf{(0.000)}}} & \mc{1}{c}{\scriptsize{\textbf{(0.000)}}} & \mc{1}{c}{\scriptsize{\textbf{(0.000)}}} \\  

  \bottomrule
  \end{tabular}
\end{center}

\begin{center}
	  \begin{tabular}{cccccccccc}
  \toprule

    \scriptsize{Variable} & \scriptsize{Age} & \scriptsize{(1)} & \scriptsize{(2)} & \scriptsize{(3)} & \scriptsize{(4)} & \scriptsize{(5)} & \scriptsize{(6)} & \scriptsize{(7)} & \scriptsize{(8)} \\ 
    \midrule  

    \mc{1}{l}{\scriptsize{Systolic Blood Pressure (mm Hg)}} & \mc{1}{c}{\scriptsize{Mid-30s}} & \mc{1}{c}{\scriptsize{-5.625}} & \mc{1}{c}{\scriptsize{-8.064}} & \mc{1}{c}{\scriptsize{5.375}} & \mc{1}{c}{\scriptsize{6.662}} & \mc{1}{c}{\scriptsize{3.760}} & \mc{1}{c}{\scriptsize{-9.437}} & \mc{1}{c}{\scriptsize{-12.708}} & \mc{1}{c}{\scriptsize{-11.161}} \\  

     &  & \mc{1}{c}{\scriptsize{\textbf{(0.089)}}} & \mc{1}{c}{\scriptsize{\textbf{(0.079)}}} & \mc{1}{c}{\scriptsize{(0.832)}} & \mc{1}{c}{\scriptsize{(0.772)}} & \mc{1}{c}{\scriptsize{(0.752)}} & \mc{1}{c}{\scriptsize{\textbf{(0.010)}}} & \mc{1}{c}{\scriptsize{\textbf{(0.030)}}} & \mc{1}{c}{\scriptsize{\textbf{(0.000)}}} \\  

    \mc{1}{l}{\scriptsize{Diastolic Blood Pressure (mm Hg)}} & \mc{1}{c}{\scriptsize{Mid-30s}} & \mc{1}{c}{\scriptsize{-5.312}} & \mc{1}{c}{\scriptsize{-6.220}} & \mc{1}{c}{\scriptsize{-1.424}} & \mc{1}{c}{\scriptsize{-0.029}} & \mc{1}{c}{\scriptsize{-2.203}} & \mc{1}{c}{\scriptsize{-7.219}} & \mc{1}{c}{\scriptsize{-7.697}} & \mc{1}{c}{\scriptsize{-8.207}} \\  

     &  & \mc{1}{c}{\scriptsize{\textbf{(0.040)}}} & \mc{1}{c}{\scriptsize{\textbf{(0.020)}}} & \mc{1}{c}{\scriptsize{(0.307)}} & \mc{1}{c}{\scriptsize{(0.505)}} & \mc{1}{c}{\scriptsize{(0.277)}} & \mc{1}{c}{\scriptsize{\textbf{(0.010)}}} & \mc{1}{c}{\scriptsize{\textbf{(0.040)}}} & \mc{1}{c}{\scriptsize{\textbf{(0.000)}}} \\  

    \mc{1}{l}{\scriptsize{Prehypertension}} & \mc{1}{c}{\scriptsize{Mid-30s}} & \mc{1}{c}{\scriptsize{-0.176}} & \mc{1}{c}{\scriptsize{-0.189}} & \mc{1}{c}{\scriptsize{-0.049}} & \mc{1}{c}{\scriptsize{-0.008}} & \mc{1}{c}{\scriptsize{-0.063}} & \mc{1}{c}{\scriptsize{-0.240}} & \mc{1}{c}{\scriptsize{-0.274}} & \mc{1}{c}{\scriptsize{-0.252}} \\  

     &  & \mc{1}{c}{\scriptsize{\textbf{(0.000)}}} & \mc{1}{c}{\scriptsize{\textbf{(0.000)}}} & \mc{1}{c}{\scriptsize{(0.376)}} & \mc{1}{c}{\scriptsize{(0.495)}} & \mc{1}{c}{\scriptsize{(0.307)}} & \mc{1}{c}{\scriptsize{\textbf{(0.000)}}} & \mc{1}{c}{\scriptsize{\textbf{(0.000)}}} & \mc{1}{c}{\scriptsize{\textbf{(0.010)}}} \\  

    \mc{1}{l}{\scriptsize{Hypertension}} & \mc{1}{c}{\scriptsize{Mid-30s}} & \mc{1}{c}{\scriptsize{-0.036}} & \mc{1}{c}{\scriptsize{-0.076}} & \mc{1}{c}{\scriptsize{0.083}} & \mc{1}{c}{\scriptsize{0.153}} & \mc{1}{c}{\scriptsize{0.020}} & \mc{1}{c}{\scriptsize{-0.083}} & \mc{1}{c}{\scriptsize{-0.136}} & \mc{1}{c}{\scriptsize{-0.137}} \\  

     &  & \mc{1}{c}{\scriptsize{(0.356)}} & \mc{1}{c}{\scriptsize{(0.238)}} & \mc{1}{c}{\scriptsize{(0.673)}} & \mc{1}{c}{\scriptsize{(0.723)}} & \mc{1}{c}{\scriptsize{(0.525)}} & \mc{1}{c}{\scriptsize{(0.198)}} & \mc{1}{c}{\scriptsize{(0.139)}} & \mc{1}{c}{\scriptsize{\textbf{(0.079)}}} \\  

    \mc{1}{l}{\scriptsize{Hypertension Factor}} & \mc{1}{c}{\scriptsize{Mid-30s}} & \mc{1}{c}{\scriptsize{-0.332}} & \mc{1}{c}{\scriptsize{-0.375}} & \mc{1}{c}{\scriptsize{0.077}} & \mc{1}{c}{\scriptsize{0.245}} & \mc{1}{c}{\scriptsize{0.017}} & \mc{1}{c}{\scriptsize{-0.501}} & \mc{1}{c}{\scriptsize{-0.578}} & \mc{1}{c}{\scriptsize{-0.586}} \\  

     &  & \mc{1}{c}{\scriptsize{\textbf{(0.059)}}} & \mc{1}{c}{\scriptsize{\textbf{(0.050)}}} & \mc{1}{c}{\scriptsize{(0.584)}} & \mc{1}{c}{\scriptsize{(0.673)}} & \mc{1}{c}{\scriptsize{(0.535)}} & \mc{1}{c}{\scriptsize{\textbf{(0.010)}}} & \mc{1}{c}{\scriptsize{\textbf{(0.000)}}} & \mc{1}{c}{\scriptsize{\textbf{(0.000)}}} \\  

  \bottomrule
  \end{tabular}
\end{center}

\begin{center}
	  \begin{tabular}{cccccccccc}
  \toprule

    \scriptsize{Variable} & \scriptsize{Age} & \scriptsize{(1)} & \scriptsize{(2)} & \scriptsize{(3)} & \scriptsize{(4)} & \scriptsize{(5)} & \scriptsize{(6)} & \scriptsize{(7)} & \scriptsize{(8)} \\ 
    \midrule  

    \mc{1}{l}{\scriptsize{High-Density Lipoprotein Chol. (mg/dL)}} & \mc{1}{c}{\scriptsize{Mid-30s}} & \mc{1}{c}{\scriptsize{3.872}} & \mc{1}{c}{\scriptsize{4.962}} & \mc{1}{c}{\scriptsize{5.806}} & \mc{1}{c}{\scriptsize{3.722}} & \mc{1}{c}{\scriptsize{5.780}} & \mc{1}{c}{\scriptsize{2.964}} & \mc{1}{c}{\scriptsize{5.397}} & \mc{1}{c}{\scriptsize{3.303}} \\  

     &  & \mc{1}{c}{\scriptsize{\textbf{(0.050)}}} & \mc{1}{c}{\scriptsize{\textbf{(0.040)}}} & \mc{1}{c}{\scriptsize{\textbf{(0.030)}}} & \mc{1}{c}{\scriptsize{(0.287)}} & \mc{1}{c}{\scriptsize{\textbf{(0.079)}}} & \mc{1}{c}{\scriptsize{(0.168)}} & \mc{1}{c}{\scriptsize{\textbf{(0.030)}}} & \mc{1}{c}{\scriptsize{(0.129)}} \\  

    \mc{1}{l}{\scriptsize{Dyslipidemia}} & \mc{1}{c}{\scriptsize{Mid-30s}} & \mc{1}{c}{\scriptsize{0.013}} & \mc{1}{c}{\scriptsize{-0.025}} & \mc{1}{c}{\scriptsize{0.035}} & \mc{1}{c}{\scriptsize{0.016}} & \mc{1}{c}{\scriptsize{-0.014}} & \mc{1}{c}{\scriptsize{0.032}} & \mc{1}{c}{\scriptsize{-0.013}} & \mc{1}{c}{\scriptsize{0.006}} \\  

     &  & \mc{1}{c}{\scriptsize{(0.604)}} & \mc{1}{c}{\scriptsize{(0.396)}} & \mc{1}{c}{\scriptsize{(0.594)}} & \mc{1}{c}{\scriptsize{(0.574)}} & \mc{1}{c}{\scriptsize{(0.446)}} & \mc{1}{c}{\scriptsize{(0.644)}} & \mc{1}{c}{\scriptsize{(0.446)}} & \mc{1}{c}{\scriptsize{(0.545)}} \\  

    \mc{1}{l}{\scriptsize{Cholesterol Factor}} & \mc{1}{c}{\scriptsize{Mid-30s}} & \mc{1}{c}{\scriptsize{0.139}} & \mc{1}{c}{\scriptsize{0.155}} & \mc{1}{c}{\scriptsize{0.183}} & \mc{1}{c}{\scriptsize{-0.074}} & \mc{1}{c}{\scriptsize{0.162}} & \mc{1}{c}{\scriptsize{0.070}} & \mc{1}{c}{\scriptsize{0.167}} & \mc{1}{c}{\scriptsize{0.065}} \\  

     &  & \mc{1}{c}{\scriptsize{(0.762)}} & \mc{1}{c}{\scriptsize{(0.743)}} & \mc{1}{c}{\scriptsize{(0.743)}} & \mc{1}{c}{\scriptsize{(0.376)}} & \mc{1}{c}{\scriptsize{(0.772)}} & \mc{1}{c}{\scriptsize{(0.634)}} & \mc{1}{c}{\scriptsize{(0.743)}} & \mc{1}{c}{\scriptsize{(0.614)}} \\  

  \bottomrule
  \end{tabular}
\end{center}

\begin{center}
	  \begin{tabular}{cccccccccc}
  \toprule

    \scriptsize{Variable} & \scriptsize{Age} & \scriptsize{(1)} & \scriptsize{(2)} & \scriptsize{(3)} & \scriptsize{(4)} & \scriptsize{(5)} & \scriptsize{(6)} & \scriptsize{(7)} & \scriptsize{(8)} \\ 
    \midrule  

    \mc{1}{l}{\scriptsize{Hemoglobin Level (\%)}} & \mc{1}{c}{\scriptsize{Mid-30s}} & \mc{1}{c}{\scriptsize{0.079}} & \mc{1}{c}{\scriptsize{-0.159}} & \mc{1}{c}{\scriptsize{-0.031}} & \mc{1}{c}{\scriptsize{-0.351}} & \mc{1}{c}{\scriptsize{-0.316}} & \mc{1}{c}{\scriptsize{0.157}} & \mc{1}{c}{\scriptsize{0.095}} & \mc{1}{c}{\scriptsize{0.227}} \\  

     &  & \mc{1}{c}{\scriptsize{(0.737)}} & \mc{1}{c}{\scriptsize{(0.197)}} & \mc{1}{c}{\scriptsize{(0.395)}} & \mc{1}{c}{\scriptsize{(0.118)}} & \mc{1}{c}{\scriptsize{\textbf{(0.066)}}} & \mc{1}{c}{\scriptsize{(0.842)}} & \mc{1}{c}{\scriptsize{(0.737)}} & \mc{1}{c}{\scriptsize{(0.868)}} \\  

    \mc{1}{l}{\scriptsize{Prediabetes}} & \mc{1}{c}{\scriptsize{Mid-30s}} & \mc{1}{c}{\scriptsize{0.019}} & \mc{1}{c}{\scriptsize{-0.347}} & \mc{1}{c}{\scriptsize{0.030}} & \mc{1}{c}{\scriptsize{-0.471}} & \mc{1}{c}{\scriptsize{-0.487}} & \mc{1}{c}{\scriptsize{0.011}} & \mc{1}{c}{\scriptsize{-0.240}} & \mc{1}{c}{\scriptsize{-0.090}} \\  

     &  & \mc{1}{c}{\scriptsize{(0.592)}} & \mc{1}{c}{\scriptsize{\textbf{(0.066)}}} & \mc{1}{c}{\scriptsize{(0.539)}} & \mc{1}{c}{\scriptsize{(0.184)}} & \mc{1}{c}{\scriptsize{\textbf{(0.026)}}} & \mc{1}{c}{\scriptsize{(0.566)}} & \mc{1}{c}{\scriptsize{(0.105)}} & \mc{1}{c}{\scriptsize{(0.355)}} \\  

    \mc{1}{l}{\scriptsize{Diabetes}} & \mc{1}{c}{\scriptsize{Mid-30s}} &  &  &  &  &  &  &  &  \\  

     &  &  &  &  &  &  &  &  &  \\ 
    \midrule  

    \mc{2}{l}{\scriptsize{\% of Pos. TE ($H_0$: $\le$ 50\%)}} & \mc{1}{c}{\scriptsize{0}} & \mc{1}{c}{\scriptsize{100}} & \mc{1}{c}{\scriptsize{50}} & \mc{1}{c}{\scriptsize{100}} & \mc{1}{c}{\scriptsize{100}} & \mc{1}{c}{\scriptsize{0}} & \mc{1}{c}{\scriptsize{50}} & \mc{1}{c}{\scriptsize{50}} \\  

     &  & \mc{1}{c}{\scriptsize{(1.000)}} & \mc{1}{c}{\scriptsize{\textbf{(0.000)}}} & \mc{1}{c}{\scriptsize{(0.645)}} & \mc{1}{c}{\scriptsize{\textbf{(0.000)}}} & \mc{1}{c}{\scriptsize{\textbf{(0.000)}}} & \mc{1}{c}{\scriptsize{(1.000)}} & \mc{1}{c}{\scriptsize{(0.237)}} & \mc{1}{c}{\scriptsize{(0.684)}} \\  

    \mc{2}{l}{\scriptsize{\% of Pos. TE ($H_0$: $\le$ 10\% $|$ 10\% Significance)}} & \mc{1}{c}{\scriptsize{0}} & \mc{1}{c}{\scriptsize{50}} & \mc{1}{c}{\scriptsize{0}} & \mc{1}{c}{\scriptsize{0}} & \mc{1}{c}{\scriptsize{100}} & \mc{1}{c}{\scriptsize{0}} & \mc{1}{c}{\scriptsize{50}} & \mc{1}{c}{\scriptsize{0}} \\  

     &  & \mc{1}{c}{\scriptsize{(1.000)}} & \mc{1}{c}{\scriptsize{\textbf{(0.039)}}} & \mc{1}{c}{\scriptsize{(1.000)}} & \mc{1}{c}{\scriptsize{(0.355)}} & \mc{1}{c}{\scriptsize{\textbf{(0.000)}}} & \mc{1}{c}{\scriptsize{(1.000)}} & \mc{1}{c}{\scriptsize{(0.145)}} & \mc{1}{c}{\scriptsize{(0.158)}} \\  

  \bottomrule
  \end{tabular}
\end{center}

\begin{center}
	  \begin{tabular}{cccccccccc}
  \toprule

    \scriptsize{Variable} & \scriptsize{Age} & \scriptsize{(1)} & \scriptsize{(2)} & \scriptsize{(3)} & \scriptsize{(4)} & \scriptsize{(5)} & \scriptsize{(6)} & \scriptsize{(7)} & \scriptsize{(8)} \\ 
    \midrule  

    \mc{1}{l}{\scriptsize{High-Density Lipoprotein Chol. (mg/dL)}} & \mc{1}{c}{\scriptsize{Mid-30s}} & \mc{1}{c}{\scriptsize{6.112}} & \mc{1}{c}{\scriptsize{8.635}} & \mc{1}{c}{\scriptsize{9.480}} & \mc{1}{c}{\scriptsize{13.589}} & \mc{1}{c}{\scriptsize{7.464}} & \mc{1}{c}{\scriptsize{5.144}} & \mc{1}{c}{\scriptsize{8.596}} & \mc{1}{c}{\scriptsize{7.432}} \\  

     &  & \mc{1}{c}{\scriptsize{\textbf{(0.053)}}} & \mc{1}{c}{\scriptsize{\textbf{(0.013)}}} & \mc{1}{c}{\scriptsize{\textbf{(0.000)}}} & \mc{1}{c}{\scriptsize{\textbf{(0.079)}}} & \mc{1}{c}{\scriptsize{(0.171)}} & \mc{1}{c}{\scriptsize{(0.118)}} & \mc{1}{c}{\scriptsize{\textbf{(0.026)}}} & \mc{1}{c}{\scriptsize{\textbf{(0.026)}}} \\  

    \mc{1}{l}{\scriptsize{Dyslipidemia}} & \mc{1}{c}{\scriptsize{Mid-30s}} & \mc{1}{c}{\scriptsize{-0.095}} & \mc{1}{c}{\scriptsize{-0.116}} & \mc{1}{c}{\scriptsize{-0.169}} & \mc{1}{c}{\scriptsize{-0.113}} & \mc{1}{c}{\scriptsize{-0.124}} & \mc{1}{c}{\scriptsize{-0.057}} & \mc{1}{c}{\scriptsize{-0.112}} & \mc{1}{c}{\scriptsize{-0.084}} \\  

     &  & \mc{1}{c}{\scriptsize{\textbf{(0.092)}}} & \mc{1}{c}{\scriptsize{\textbf{(0.079)}}} & \mc{1}{c}{\scriptsize{(0.237)}} & \mc{1}{c}{\scriptsize{(0.303)}} & \mc{1}{c}{\scriptsize{(0.303)}} & \mc{1}{c}{\scriptsize{(0.197)}} & \mc{1}{c}{\scriptsize{(0.132)}} & \mc{1}{c}{\scriptsize{(0.171)}} \\  

    \mc{1}{l}{\scriptsize{Cholesterol Factor}} & \mc{1}{c}{\scriptsize{Mid-30s}} & \mc{1}{c}{\scriptsize{-0.279}} & \mc{1}{c}{\scriptsize{-0.372}} & \mc{1}{c}{\scriptsize{-0.457}} & \mc{1}{c}{\scriptsize{-0.501}} & \mc{1}{c}{\scriptsize{-0.349}} & \mc{1}{c}{\scriptsize{-0.207}} & \mc{1}{c}{\scriptsize{-0.366}} & \mc{1}{c}{\scriptsize{-0.301}} \\  

     &  & \mc{1}{c}{\scriptsize{\textbf{(0.026)}}} & \mc{1}{c}{\scriptsize{\textbf{(0.013)}}} & \mc{1}{c}{\scriptsize{(0.105)}} & \mc{1}{c}{\scriptsize{(0.197)}} & \mc{1}{c}{\scriptsize{(0.224)}} & \mc{1}{c}{\scriptsize{(0.105)}} & \mc{1}{c}{\scriptsize{\textbf{(0.013)}}} & \mc{1}{c}{\scriptsize{\textbf{(0.026)}}} \\ 
    \midrule  

    \mc{2}{l}{\scriptsize{\% of Pos. TE ($H_0$: $\le$ 50\%)}} & \mc{1}{c}{\scriptsize{100}} & \mc{1}{c}{\scriptsize{100}} & \mc{1}{c}{\scriptsize{100}} & \mc{1}{c}{\scriptsize{100}} & \mc{1}{c}{\scriptsize{100}} & \mc{1}{c}{\scriptsize{100}} & \mc{1}{c}{\scriptsize{100}} & \mc{1}{c}{\scriptsize{100}} \\  

     &  & \mc{1}{c}{\scriptsize{\textbf{(0.000)}}} & \mc{1}{c}{\scriptsize{\textbf{(0.000)}}} & \mc{1}{c}{\scriptsize{\textbf{(0.000)}}} & \mc{1}{c}{\scriptsize{\textbf{(0.000)}}} & \mc{1}{c}{\scriptsize{\textbf{(0.000)}}} & \mc{1}{c}{\scriptsize{\textbf{(0.000)}}} & \mc{1}{c}{\scriptsize{\textbf{(0.000)}}} & \mc{1}{c}{\scriptsize{\textbf{(0.000)}}} \\  

    \mc{2}{l}{\scriptsize{\% of Pos. TE ($H_0$: $\le$ 10\% $|$ 10\% Significance)}} & \mc{1}{c}{\scriptsize{67}} & \mc{1}{c}{\scriptsize{67}} & \mc{1}{c}{\scriptsize{33}} & \mc{1}{c}{\scriptsize{33}} & \mc{1}{c}{\scriptsize{0}} & \mc{1}{c}{\scriptsize{67}} & \mc{1}{c}{\scriptsize{67}} & \mc{1}{c}{\scriptsize{67}} \\  

     &  & \mc{1}{c}{\scriptsize{(0.197)}} & \mc{1}{c}{\scriptsize{\textbf{(0.000)}}} & \mc{1}{c}{\scriptsize{(0.434)}} & \mc{1}{c}{\scriptsize{(0.184)}} & \mc{1}{c}{\scriptsize{(0.342)}} & \mc{1}{c}{\scriptsize{\textbf{(0.092)}}} & \mc{1}{c}{\scriptsize{\textbf{(0.000)}}} & \mc{1}{c}{\scriptsize{(0.132)}} \\  

  \bottomrule
  \end{tabular}
\end{center}

\begin{center}
	\begin{table}[H]
\captionsetup{singlelinecheck=false,justification=centering}
\caption{CARE Average Treatment Effects, Males and Females \\ Obesity \label{tab:ate_pooled_apx15}}

  \begin{threeparttable}
  \begin{tabular}{cccccccccc}
  \hline\hline

     &  & \scriptsize{(1)} & \scriptsize{(2)} & \scriptsize{(3)} & \scriptsize{(4)} & \scriptsize{(5)} & \scriptsize{(6)} & \scriptsize{(7)} & \scriptsize{(8)} \\  

     &  &  &  & \mc{3}{c}{\scriptsize{$P=0$}} & \mc{3}{c}{\scriptsize{$P=1$}} \\ 
    \cmidrule(lr){5-7} \cmidrule(lr){8-10} 

    \scriptsize{Variable} & \scriptsize{Age} & \scriptsize{ITT} & \scriptsize{ITT$|X,W$} & \scriptsize{ITT} & \scriptsize{ITT$|X,W$} & \scriptsize{KE$|X,W$} & \scriptsize{ITT} & \scriptsize{ITT$|X,W$} & \scriptsize{KE$|X,W$} \\ 
    \hline  

    \mc{1}{l}{\scriptsize{Measured BMI}} & \mc{1}{c}{\scriptsize{Mid-30s}} & \mc{1}{c}{\scriptsize{-2.394}} & \mc{1}{c}{\scriptsize{0.726}} & \mc{1}{c}{\scriptsize{-5.732}} & \mc{1}{c}{\scriptsize{-3.810}} &  & \mc{1}{c}{\scriptsize{0.388}} & \mc{1}{c}{\scriptsize{3.243}} &  \\  

     &  & \mc{1}{c}{\scriptsize{(0.216)}} & \mc{1}{c}{\scriptsize{(0.608)}} & \mc{1}{c}{\scriptsize{\textbf{(0.000)}}} & \mc{1}{c}{\scriptsize{(0.333)}} &  & \mc{1}{c}{\scriptsize{(0.431)}} & \mc{1}{c}{\scriptsize{(0.824)}} &  \\  

    \mc{1}{l}{\scriptsize{Obesity}} & \mc{1}{c}{\scriptsize{Mid-30s}} & \mc{1}{c}{\scriptsize{-0.166}} & \mc{1}{c}{\scriptsize{0.000}} & \mc{1}{c}{\scriptsize{-0.529}} & \mc{1}{c}{\scriptsize{-0.351}} &  & \mc{1}{c}{\scriptsize{0.137}} & \mc{1}{c}{\scriptsize{0.249}} &  \\  

     &  & \mc{1}{c}{\scriptsize{(0.176)}} & \mc{1}{c}{\scriptsize{(0.373)}} & \mc{1}{c}{\scriptsize{\textbf{(0.000)}}} & \mc{1}{c}{\scriptsize{(0.275)}} &  & \mc{1}{c}{\scriptsize{(0.627)}} & \mc{1}{c}{\scriptsize{(0.882)}} &  \\  

    \mc{1}{l}{\scriptsize{Severe Obesity}} & \mc{1}{c}{\scriptsize{Mid-30s}} & \mc{1}{c}{\scriptsize{-0.128}} & \mc{1}{c}{\scriptsize{-0.036}} & \mc{1}{c}{\scriptsize{-0.165}} & \mc{1}{c}{\scriptsize{-0.154}} &  & \mc{1}{c}{\scriptsize{-0.098}} & \mc{1}{c}{\scriptsize{-0.024}} &  \\  

     &  & \mc{1}{c}{\scriptsize{(0.235)}} & \mc{1}{c}{\scriptsize{(0.412)}} & \mc{1}{c}{\scriptsize{(0.235)}} & \mc{1}{c}{\scriptsize{(0.431)}} &  & \mc{1}{c}{\scriptsize{(0.333)}} & \mc{1}{c}{\scriptsize{(0.373)}} &  \\  

    \mc{1}{l}{\scriptsize{Waist-hip Ratio}} & \mc{1}{c}{\scriptsize{Mid-30s}} & \mc{1}{c}{\scriptsize{0.013}} & \mc{1}{c}{\scriptsize{0.031}} & \mc{1}{c}{\scriptsize{-0.032}} & \mc{1}{c}{\scriptsize{0.015}} &  & \mc{1}{c}{\scriptsize{0.051}} & \mc{1}{c}{\scriptsize{0.046}} &  \\  

     &  & \mc{1}{c}{\scriptsize{(0.647)}} & \mc{1}{c}{\scriptsize{(0.784)}} & \mc{1}{c}{\scriptsize{(0.255)}} & \mc{1}{c}{\scriptsize{(0.686)}} &  & \mc{1}{c}{\scriptsize{(0.961)}} & \mc{1}{c}{\scriptsize{(0.882)}} &  \\  

    \mc{1}{l}{\scriptsize{Abdominal Obesity}} & \mc{1}{c}{\scriptsize{Mid-30s}} & \mc{1}{c}{\scriptsize{0.011}} & \mc{1}{c}{\scriptsize{0.064}} & \mc{1}{c}{\scriptsize{-0.353}} & \mc{1}{c}{\scriptsize{-0.230}} &  & \mc{1}{c}{\scriptsize{0.314}} & \mc{1}{c}{\scriptsize{0.291}} &  \\  

     &  & \mc{1}{c}{\scriptsize{(0.510)}} & \mc{1}{c}{\scriptsize{(0.608)}} & \mc{1}{c}{\scriptsize{\textbf{(0.000)}}} & \mc{1}{c}{\scriptsize{(0.294)}} &  & \mc{1}{c}{\scriptsize{(0.902)}} & \mc{1}{c}{\scriptsize{(0.902)}} &  \\  

    \mc{1}{l}{\scriptsize{Framingham Risk Score}} & \mc{1}{c}{\scriptsize{Mid-30s}} & \mc{1}{c}{\scriptsize{-0.166}} & \mc{1}{c}{\scriptsize{-0.131}} & \mc{1}{c}{\scriptsize{-0.594}} & \mc{1}{c}{\scriptsize{-0.476}} &  & \mc{1}{c}{\scriptsize{0.190}} & \mc{1}{c}{\scriptsize{-0.054}} &  \\  

     &  & \mc{1}{c}{\scriptsize{(0.412)}} & \mc{1}{c}{\scriptsize{(0.353)}} & \mc{1}{c}{\scriptsize{(0.294)}} & \mc{1}{c}{\scriptsize{(0.294)}} &  & \mc{1}{c}{\scriptsize{(0.588)}} & \mc{1}{c}{\scriptsize{(0.529)}} &  \\ 
    \hline  

    \\[0.1cm]
    \mc{2}{l}{\scriptsize{\% of Pos. TE ($H_0$: $\le$ 25\% $|$ 10\% Significance)}} & \mc{1}{c}{\scriptsize{0}} & \mc{1}{c}{\scriptsize{0}} & \mc{1}{c}{\scriptsize{50}} & \mc{1}{c}{\scriptsize{0}} &  & \mc{1}{c}{\scriptsize{0}} & \mc{1}{c}{\scriptsize{0}} &  \\  

     &  & \mc{1}{c}{\scriptsize{(0.686)}} & \mc{1}{c}{\scriptsize{(1.000)}} & \mc{1}{c}{\scriptsize{\textbf{(0.059)}}} & \mc{1}{c}{\scriptsize{(0.863)}} &  & \mc{1}{c}{\scriptsize{(1.000)}} & \mc{1}{c}{\scriptsize{(1.000)}} &  \\  

    \mc{2}{l}{\scriptsize{\% of Pos. TE ($H_0$: $\le$ 50\% $|$ 10\% Significance)}} & \mc{1}{c}{\scriptsize{0}} & \mc{1}{c}{\scriptsize{0}} & \mc{1}{c}{\scriptsize{50}} & \mc{1}{c}{\scriptsize{0}} &  & \mc{1}{c}{\scriptsize{0}} & \mc{1}{c}{\scriptsize{0}} &  \\  

     &  & \mc{1}{c}{\scriptsize{(1.000)}} & \mc{1}{c}{\scriptsize{(1.000)}} & \mc{1}{c}{\scriptsize{(0.529)}} & \mc{1}{c}{\scriptsize{(0.863)}} &  & \mc{1}{c}{\scriptsize{(1.000)}} & \mc{1}{c}{\scriptsize{(1.000)}} &  \\  

    \mc{2}{l}{\scriptsize{\% of Pos. TE ($H_0$: $\le$ 75\% $|$ 10\% Significance)}} & \mc{1}{c}{\scriptsize{0}} & \mc{1}{c}{\scriptsize{0}} & \mc{1}{c}{\scriptsize{50}} & \mc{1}{c}{\scriptsize{0}} &  & \mc{1}{c}{\scriptsize{0}} & \mc{1}{c}{\scriptsize{0}} &  \\  

     &  & \mc{1}{c}{\scriptsize{(1.000)}} & \mc{1}{c}{\scriptsize{(1.000)}} & \mc{1}{c}{\scriptsize{(0.745)}} & \mc{1}{c}{\scriptsize{(0.863)}} &  & \mc{1}{c}{\scriptsize{(1.000)}} & \mc{1}{c}{\scriptsize{(1.000)}} &  \\  

  \hline\hline
  \end{tabular}
    \begin{tablenotes}
    \scriptsize
    \item 
Note: This table displays various estimates of the treatment effect of CARE's family education program.
Column (1) displays the ITT, without accounting for any controls.
Column (2) displays the ITT conditioning on vector of controls, $X$, consisting of APGAR scores 1 
minute after birth, an indicator for the subject being born prematurely, and an indicator for the 
father being home at baseline. We also apply IPW weights, $W$, to account for attrition.
Columns (3)--(4) are analogous to columns (1)--(2), but we restrict the control sample to subjects
who did not enroll in any alternative care.
Column (5) displys the matching estimate, where we use the Mahalanobis metric and Epanechnikov kernel
to match on controls $X$ listed above, and restrict the control sample to subjects who did not enroll
in any alternative care. Additionally, we apply IPW weights, $W$.
Columns (6)--(8) are analogous to Columns (3)--(5), except we restrict the control sample to subejcts
who did enroll in alternative care. 
The final three pairs of rows display the proportion of treatment effects in the table that are 
socially positive. The first row in each pair displays the percentage of treatment effects, and the
second row presents the inference.

Numbers in parentheses represent the $p$-value from a single hypothesis test, and are obtained from 
the empirical bootstrap distribution generated by 200 resamples of the original data. 
Bold $p$-values indicate significance at the 10\% level.
Blank point estimates indicate that we are unable to obtain estimates due to a lack of support in the data. 

    \end{tablenotes}
  \end{threeparttable}

\end{table}
\end{center}

\begin{center}
	  \begin{tabular}{cccccccccc}
  \toprule

    \scriptsize{Variable} & \scriptsize{Age} & \scriptsize{(1)} & \scriptsize{(2)} & \scriptsize{(3)} & \scriptsize{(4)} & \scriptsize{(5)} & \scriptsize{(6)} & \scriptsize{(7)} & \scriptsize{(8)} \\ 
    \midrule  

    \mc{1}{l}{\scriptsize{Obesity Factor}} & \mc{1}{c}{\scriptsize{Mid-30s}} & \mc{1}{c}{\scriptsize{-0.329}} & \mc{1}{c}{\scriptsize{-0.247}} & \mc{1}{c}{\scriptsize{-0.343}} & \mc{1}{c}{\scriptsize{-0.601}} & \mc{1}{c}{\scriptsize{0.108}} & \mc{1}{c}{\scriptsize{-0.358}} & \mc{1}{c}{\scriptsize{-0.261}} & \mc{1}{c}{\scriptsize{-0.281}} \\  

     &  & \mc{1}{c}{\scriptsize{\textbf{(0.079)}}} & \mc{1}{c}{\scriptsize{(0.197)}} & \mc{1}{c}{\scriptsize{(0.263)}} & \mc{1}{c}{\scriptsize{(0.224)}} & \mc{1}{c}{\scriptsize{(0.513)}} & \mc{1}{c}{\scriptsize{\textbf{(0.066)}}} & \mc{1}{c}{\scriptsize{(0.237)}} & \mc{1}{c}{\scriptsize{(0.158)}} \\  

    \mc{1}{l}{\scriptsize{Measured BMI}} & \mc{1}{c}{\scriptsize{Mid-30s}} & \mc{1}{c}{\scriptsize{-1.329}} & \mc{1}{c}{\scriptsize{0.516}} & \mc{1}{c}{\scriptsize{-0.673}} & \mc{1}{c}{\scriptsize{-5.302}} & \mc{1}{c}{\scriptsize{2.854}} & \mc{1}{c}{\scriptsize{-1.790}} & \mc{1}{c}{\scriptsize{1.084}} & \mc{1}{c}{\scriptsize{-0.583}} \\  

     &  & \mc{1}{c}{\scriptsize{(0.276)}} & \mc{1}{c}{\scriptsize{(0.632)}} & \mc{1}{c}{\scriptsize{(0.408)}} & \mc{1}{c}{\scriptsize{(0.145)}} & \mc{1}{c}{\scriptsize{(0.671)}} & \mc{1}{c}{\scriptsize{(0.197)}} & \mc{1}{c}{\scriptsize{(0.671)}} & \mc{1}{c}{\scriptsize{(0.408)}} \\  

    \mc{1}{l}{\scriptsize{Framingham Risk Score}} & \mc{1}{c}{\scriptsize{Mid-30s}} & \mc{1}{c}{\scriptsize{0.133}} & \mc{1}{c}{\scriptsize{-1.118}} & \mc{1}{c}{\scriptsize{1.178}} & \mc{1}{c}{\scriptsize{0.578}} & \mc{1}{c}{\scriptsize{1.300}} & \mc{1}{c}{\scriptsize{0.171}} & \mc{1}{c}{\scriptsize{-1.411}} & \mc{1}{c}{\scriptsize{-0.134}} \\  

     &  & \mc{1}{c}{\scriptsize{(0.539)}} & \mc{1}{c}{\scriptsize{\textbf{(0.053)}}} & \mc{1}{c}{\scriptsize{(0.974)}} & \mc{1}{c}{\scriptsize{(0.632)}} & \mc{1}{c}{\scriptsize{(0.947)}} & \mc{1}{c}{\scriptsize{(0.539)}} & \mc{1}{c}{\scriptsize{\textbf{(0.039)}}} & \mc{1}{c}{\scriptsize{(0.447)}} \\  

    \mc{1}{l}{\scriptsize{Obesity}} & \mc{1}{c}{\scriptsize{Mid-30s}} & \mc{1}{c}{\scriptsize{-0.121}} & \mc{1}{c}{\scriptsize{0.030}} & \mc{1}{c}{\scriptsize{-0.171}} & \mc{1}{c}{\scriptsize{-0.197}} & \mc{1}{c}{\scriptsize{0.058}} & \mc{1}{c}{\scriptsize{-0.141}} & \mc{1}{c}{\scriptsize{0.024}} & \mc{1}{c}{\scriptsize{-0.106}} \\  

     &  & \mc{1}{c}{\scriptsize{(0.171)}} & \mc{1}{c}{\scriptsize{(0.579)}} & \mc{1}{c}{\scriptsize{(0.237)}} & \mc{1}{c}{\scriptsize{(0.237)}} & \mc{1}{c}{\scriptsize{(0.474)}} & \mc{1}{c}{\scriptsize{(0.132)}} & \mc{1}{c}{\scriptsize{(0.592)}} & \mc{1}{c}{\scriptsize{(0.250)}} \\  

    \mc{1}{l}{\scriptsize{Abdominal Obesity}} & \mc{1}{c}{\scriptsize{Mid-30s}} & \mc{1}{c}{\scriptsize{-0.187}} & \mc{1}{c}{\scriptsize{-0.135}} & \mc{1}{c}{\scriptsize{-0.144}} & \mc{1}{c}{\scriptsize{0.112}} & \mc{1}{c}{\scriptsize{0.036}} & \mc{1}{c}{\scriptsize{-0.186}} & \mc{1}{c}{\scriptsize{-0.199}} & \mc{1}{c}{\scriptsize{-0.151}} \\  

     &  & \mc{1}{c}{\scriptsize{\textbf{(0.039)}}} & \mc{1}{c}{\scriptsize{(0.171)}} & \mc{1}{c}{\scriptsize{(0.316)}} & \mc{1}{c}{\scriptsize{(0.658)}} & \mc{1}{c}{\scriptsize{(0.434)}} & \mc{1}{c}{\scriptsize{\textbf{(0.026)}}} & \mc{1}{c}{\scriptsize{\textbf{(0.079)}}} & \mc{1}{c}{\scriptsize{\textbf{(0.092)}}} \\  

    \mc{1}{l}{\scriptsize{Severe Obesity}} & \mc{1}{c}{\scriptsize{Mid-30s}} & \mc{1}{c}{\scriptsize{-0.213}} & \mc{1}{c}{\scriptsize{-0.149}} & \mc{1}{c}{\scriptsize{-0.096}} & \mc{1}{c}{\scriptsize{-0.203}} & \mc{1}{c}{\scriptsize{0.036}} & \mc{1}{c}{\scriptsize{-0.246}} & \mc{1}{c}{\scriptsize{-0.145}} & \mc{1}{c}{\scriptsize{-0.211}} \\  

     &  & \mc{1}{c}{\scriptsize{\textbf{(0.013)}}} & \mc{1}{c}{\scriptsize{(0.132)}} & \mc{1}{c}{\scriptsize{(0.316)}} & \mc{1}{c}{\scriptsize{(0.197)}} & \mc{1}{c}{\scriptsize{(0.461)}} & \mc{1}{c}{\scriptsize{\textbf{(0.013)}}} & \mc{1}{c}{\scriptsize{(0.145)}} & \mc{1}{c}{\scriptsize{\textbf{(0.066)}}} \\  

    \mc{1}{l}{\scriptsize{Waist-hip Ratio}} & \mc{1}{c}{\scriptsize{Mid-30s}} & \mc{1}{c}{\scriptsize{-0.033}} & \mc{1}{c}{\scriptsize{-0.035}} & \mc{1}{c}{\scriptsize{-0.075}} & \mc{1}{c}{\scriptsize{-0.091}} & \mc{1}{c}{\scriptsize{-0.046}} & \mc{1}{c}{\scriptsize{-0.027}} & \mc{1}{c}{\scriptsize{-0.037}} & \mc{1}{c}{\scriptsize{-0.026}} \\  

     &  & \mc{1}{c}{\scriptsize{\textbf{(0.066)}}} & \mc{1}{c}{\scriptsize{\textbf{(0.079)}}} & \mc{1}{c}{\scriptsize{(0.158)}} & \mc{1}{c}{\scriptsize{(0.250)}} & \mc{1}{c}{\scriptsize{(0.276)}} & \mc{1}{c}{\scriptsize{(0.132)}} & \mc{1}{c}{\scriptsize{(0.105)}} & \mc{1}{c}{\scriptsize{(0.237)}} \\ 
    \midrule  

    \mc{2}{l}{\scriptsize{\% of Pos. TE ($H_0$: $\le$ 50\%)}} & \mc{1}{c}{\scriptsize{86}} & \mc{1}{c}{\scriptsize{71}} & \mc{1}{c}{\scriptsize{86}} & \mc{1}{c}{\scriptsize{71}} & \mc{1}{c}{\scriptsize{14}} & \mc{1}{c}{\scriptsize{86}} & \mc{1}{c}{\scriptsize{71}} & \mc{1}{c}{\scriptsize{100}} \\  

     &  & \mc{1}{c}{\scriptsize{\textbf{(0.000)}}} & \mc{1}{c}{\scriptsize{(0.276)}} & \mc{1}{c}{\scriptsize{\textbf{(0.000)}}} & \mc{1}{c}{\scriptsize{(0.184)}} & \mc{1}{c}{\scriptsize{(0.816)}} & \mc{1}{c}{\scriptsize{\textbf{(0.000)}}} & \mc{1}{c}{\scriptsize{(0.263)}} & \mc{1}{c}{\scriptsize{\textbf{(0.000)}}} \\  

    \mc{2}{l}{\scriptsize{\% of Pos. TE ($H_0$: $\le$ 10\% $|$ 10\% Significance)}} & \mc{1}{c}{\scriptsize{57}} & \mc{1}{c}{\scriptsize{29}} & \mc{1}{c}{\scriptsize{0}} & \mc{1}{c}{\scriptsize{0}} & \mc{1}{c}{\scriptsize{0}} & \mc{1}{c}{\scriptsize{43}} & \mc{1}{c}{\scriptsize{43}} & \mc{1}{c}{\scriptsize{29}} \\  

     &  & \mc{1}{c}{\scriptsize{\textbf{(0.092)}}} & \mc{1}{c}{\scriptsize{(0.197)}} & \mc{1}{c}{\scriptsize{(0.487)}} & \mc{1}{c}{\scriptsize{(0.421)}} & \mc{1}{c}{\scriptsize{(0.342)}} & \mc{1}{c}{\scriptsize{(0.171)}} & \mc{1}{c}{\scriptsize{(0.118)}} & \mc{1}{c}{\scriptsize{(0.263)}} \\  

  \bottomrule
  \end{tabular}
\end{center}

\begin{center}
	\begin{table}[H]
\captionsetup{singlelinecheck=false,justification=centering}
\caption{ABC/CARE Average Treatment Effects, Males and Females \\ Mental Health \label{tab:ate_pooled_apx17}}

  \begin{threeparttable}
  \begin{tabular}{cccccccccc}
  \hline\hline

     &  & \scriptsize{(1)} & \scriptsize{(2)} & \scriptsize{(3)} & \scriptsize{(4)} & \scriptsize{(5)} & \scriptsize{(6)} & \scriptsize{(7)} & \scriptsize{(8)} \\  

     &  &  &  & \mc{3}{c}{\scriptsize{$P=0$}} & \mc{3}{c}{\scriptsize{$P=1$}} \\ 
    \cmidrule(lr){5-7} \cmidrule(lr){8-10} 

    \scriptsize{Variable} & \scriptsize{Age} & \scriptsize{ITT} & \scriptsize{ITT$|X,W$} & \scriptsize{ITT} & \scriptsize{ITT$|X,W$} & \scriptsize{KE$|X,W$} & \scriptsize{ITT} & \scriptsize{ITT$|X,W$} & \scriptsize{KE$|X,W$} \\ 
    \hline  

    \mc{1}{l}{\scriptsize{Somatization}} & \mc{1}{c}{\scriptsize{21}} & \mc{1}{c}{\scriptsize{0.010}} & \mc{1}{c}{\scriptsize{-0.022}} & \mc{1}{c}{\scriptsize{-0.011}} & \mc{1}{c}{\scriptsize{-0.074}} & \mc{1}{c}{\scriptsize{-0.088}} & \mc{1}{c}{\scriptsize{0.019}} & \mc{1}{c}{\scriptsize{0.003}} & \mc{1}{c}{\scriptsize{-0.024}} \\  

     &  & \mc{1}{c}{\scriptsize{(0.569)}} & \mc{1}{c}{\scriptsize{(0.333)}} & \mc{1}{c}{\scriptsize{(0.373)}} & \mc{1}{c}{\scriptsize{(0.333)}} & \mc{1}{c}{\scriptsize{(0.235)}} & \mc{1}{c}{\scriptsize{(0.627)}} & \mc{1}{c}{\scriptsize{(0.510)}} & \mc{1}{c}{\scriptsize{(0.353)}} \\  

     & \mc{1}{c}{\scriptsize{34}} & \mc{1}{c}{\scriptsize{-0.122}} & \mc{1}{c}{\scriptsize{-0.154}} & \mc{1}{c}{\scriptsize{-0.117}} & \mc{1}{c}{\scriptsize{-0.256}} & \mc{1}{c}{\scriptsize{-0.095}} & \mc{1}{c}{\scriptsize{-0.124}} & \mc{1}{c}{\scriptsize{-0.101}} & \mc{1}{c}{\scriptsize{-0.161}} \\  

     &  & \mc{1}{c}{\scriptsize{(0.176)}} & \mc{1}{c}{\scriptsize{(0.196)}} & \mc{1}{c}{\scriptsize{(0.255)}} & \mc{1}{c}{\scriptsize{(0.196)}} & \mc{1}{c}{\scriptsize{(0.216)}} & \mc{1}{c}{\scriptsize{(0.176)}} & \mc{1}{c}{\scriptsize{(0.275)}} & \mc{1}{c}{\scriptsize{\textbf{(0.098)}}} \\  

    \mc{1}{l}{\scriptsize{Depression}} & \mc{1}{c}{\scriptsize{21}} & \mc{1}{c}{\scriptsize{-0.154}} & \mc{1}{c}{\scriptsize{-0.224}} & \mc{1}{c}{\scriptsize{-0.195}} & \mc{1}{c}{\scriptsize{-0.278}} & \mc{1}{c}{\scriptsize{-0.173}} & \mc{1}{c}{\scriptsize{-0.136}} & \mc{1}{c}{\scriptsize{-0.181}} & \mc{1}{c}{\scriptsize{-0.128}} \\  

     &  & \mc{1}{c}{\scriptsize{(0.118)}} & \mc{1}{c}{\scriptsize{\textbf{(0.078)}}} & \mc{1}{c}{\scriptsize{(0.137)}} & \mc{1}{c}{\scriptsize{(0.118)}} & \mc{1}{c}{\scriptsize{(0.176)}} & \mc{1}{c}{\scriptsize{\textbf{(0.098)}}} & \mc{1}{c}{\scriptsize{(0.118)}} & \mc{1}{c}{\scriptsize{(0.157)}} \\  

     & \mc{1}{c}{\scriptsize{34}} & \mc{1}{c}{\scriptsize{-0.164}} & \mc{1}{c}{\scriptsize{-0.134}} & \mc{1}{c}{\scriptsize{0.060}} & \mc{1}{c}{\scriptsize{-0.033}} & \mc{1}{c}{\scriptsize{0.106}} & \mc{1}{c}{\scriptsize{-0.261}} & \mc{1}{c}{\scriptsize{-0.174}} & \mc{1}{c}{\scriptsize{-0.275}} \\  

     &  & \mc{1}{c}{\scriptsize{(0.176)}} & \mc{1}{c}{\scriptsize{(0.235)}} & \mc{1}{c}{\scriptsize{(0.588)}} & \mc{1}{c}{\scriptsize{(0.373)}} & \mc{1}{c}{\scriptsize{(0.451)}} & \mc{1}{c}{\scriptsize{(0.137)}} & \mc{1}{c}{\scriptsize{(0.216)}} & \mc{1}{c}{\scriptsize{\textbf{(0.059)}}} \\  

    \mc{1}{l}{\scriptsize{Anxiety}} & \mc{1}{c}{\scriptsize{21}} & \mc{1}{c}{\scriptsize{-0.046}} & \mc{1}{c}{\scriptsize{-0.102}} & \mc{1}{c}{\scriptsize{-0.112}} & \mc{1}{c}{\scriptsize{-0.165}} & \mc{1}{c}{\scriptsize{-0.091}} & \mc{1}{c}{\scriptsize{-0.017}} & \mc{1}{c}{\scriptsize{-0.068}} & \mc{1}{c}{\scriptsize{-0.002}} \\  

     &  & \mc{1}{c}{\scriptsize{(0.275)}} & \mc{1}{c}{\scriptsize{(0.176)}} & \mc{1}{c}{\scriptsize{(0.275)}} & \mc{1}{c}{\scriptsize{(0.176)}} & \mc{1}{c}{\scriptsize{(0.294)}} & \mc{1}{c}{\scriptsize{(0.392)}} & \mc{1}{c}{\scriptsize{(0.275)}} & \mc{1}{c}{\scriptsize{(0.451)}} \\  

     & \mc{1}{c}{\scriptsize{34}} & \mc{1}{c}{\scriptsize{-0.211}} & \mc{1}{c}{\scriptsize{-0.228}} & \mc{1}{c}{\scriptsize{-0.086}} & \mc{1}{c}{\scriptsize{-0.164}} & \mc{1}{c}{\scriptsize{-0.083}} & \mc{1}{c}{\scriptsize{-0.264}} & \mc{1}{c}{\scriptsize{-0.230}} & \mc{1}{c}{\scriptsize{-0.306}} \\  

     &  & \mc{1}{c}{\scriptsize{(0.118)}} & \mc{1}{c}{\scriptsize{(0.176)}} & \mc{1}{c}{\scriptsize{(0.196)}} & \mc{1}{c}{\scriptsize{(0.235)}} & \mc{1}{c}{\scriptsize{(0.196)}} & \mc{1}{c}{\scriptsize{\textbf{(0.098)}}} & \mc{1}{c}{\scriptsize{(0.176)}} & \mc{1}{c}{\scriptsize{\textbf{(0.039)}}} \\  

    \mc{1}{l}{\scriptsize{Hostility}} & \mc{1}{c}{\scriptsize{21}} & \mc{1}{c}{\scriptsize{-0.218}} & \mc{1}{c}{\scriptsize{-0.302}} & \mc{1}{c}{\scriptsize{-0.335}} & \mc{1}{c}{\scriptsize{-0.441}} & \mc{1}{c}{\scriptsize{-0.313}} & \mc{1}{c}{\scriptsize{-0.166}} & \mc{1}{c}{\scriptsize{-0.236}} & \mc{1}{c}{\scriptsize{-0.178}} \\  

     &  & \mc{1}{c}{\scriptsize{\textbf{(0.059)}}} & \mc{1}{c}{\scriptsize{\textbf{(0.039)}}} & \mc{1}{c}{\scriptsize{\textbf{(0.059)}}} & \mc{1}{c}{\scriptsize{\textbf{(0.039)}}} & \mc{1}{c}{\scriptsize{\textbf{(0.039)}}} & \mc{1}{c}{\scriptsize{\textbf{(0.098)}}} & \mc{1}{c}{\scriptsize{\textbf{(0.059)}}} & \mc{1}{c}{\scriptsize{(0.137)}} \\  

     & \mc{1}{c}{\scriptsize{34}} & \mc{1}{c}{\scriptsize{-0.132}} & \mc{1}{c}{\scriptsize{-0.116}} & \mc{1}{c}{\scriptsize{-0.023}} & \mc{1}{c}{\scriptsize{-0.013}} & \mc{1}{c}{\scriptsize{-0.017}} & \mc{1}{c}{\scriptsize{-0.179}} & \mc{1}{c}{\scriptsize{-0.127}} & \mc{1}{c}{\scriptsize{-0.181}} \\  

     &  & \mc{1}{c}{\scriptsize{(0.216)}} & \mc{1}{c}{\scriptsize{(0.275)}} & \mc{1}{c}{\scriptsize{(0.412)}} & \mc{1}{c}{\scriptsize{(0.392)}} & \mc{1}{c}{\scriptsize{(0.294)}} & \mc{1}{c}{\scriptsize{(0.176)}} & \mc{1}{c}{\scriptsize{(0.294)}} & \mc{1}{c}{\scriptsize{(0.118)}} \\  

    \mc{1}{l}{\scriptsize{Global Severity Index}} & \mc{1}{c}{\scriptsize{21}} & \mc{1}{c}{\scriptsize{-0.078}} & \mc{1}{c}{\scriptsize{-0.126}} & \mc{1}{c}{\scriptsize{-0.091}} & \mc{1}{c}{\scriptsize{-0.138}} & \mc{1}{c}{\scriptsize{-0.086}} & \mc{1}{c}{\scriptsize{-0.072}} & \mc{1}{c}{\scriptsize{-0.108}} & \mc{1}{c}{\scriptsize{-0.076}} \\  

     &  & \mc{1}{c}{\scriptsize{(0.157)}} & \mc{1}{c}{\scriptsize{(0.118)}} & \mc{1}{c}{\scriptsize{(0.176)}} & \mc{1}{c}{\scriptsize{(0.196)}} & \mc{1}{c}{\scriptsize{(0.314)}} & \mc{1}{c}{\scriptsize{(0.137)}} & \mc{1}{c}{\scriptsize{(0.137)}} & \mc{1}{c}{\scriptsize{(0.255)}} \\  

     & \mc{1}{c}{\scriptsize{34}} & \mc{1}{c}{\scriptsize{-2.980}} & \mc{1}{c}{\scriptsize{-3.095}} & \mc{1}{c}{\scriptsize{-0.858}} & \mc{1}{c}{\scriptsize{-2.717}} & \mc{1}{c}{\scriptsize{-0.428}} & \mc{1}{c}{\scriptsize{-3.889}} & \mc{1}{c}{\scriptsize{-3.033}} & \mc{1}{c}{\scriptsize{-4.447}} \\  

     &  & \mc{1}{c}{\scriptsize{(0.157)}} & \mc{1}{c}{\scriptsize{(0.196)}} & \mc{1}{c}{\scriptsize{(0.314)}} & \mc{1}{c}{\scriptsize{(0.255)}} & \mc{1}{c}{\scriptsize{(0.275)}} & \mc{1}{c}{\scriptsize{(0.137)}} & \mc{1}{c}{\scriptsize{(0.176)}} & \mc{1}{c}{\scriptsize{\textbf{(0.059)}}} \\  

    \mc{1}{l}{\scriptsize{BSI Factor}} & \mc{1}{c}{\scriptsize{21 and 34}} & \mc{1}{c}{\scriptsize{-0.488}} & \mc{1}{c}{\scriptsize{-0.525}} & \mc{1}{c}{\scriptsize{-0.454}} & \mc{1}{c}{\scriptsize{-0.760}} & \mc{1}{c}{\scriptsize{-0.207}} & \mc{1}{c}{\scriptsize{-0.503}} & \mc{1}{c}{\scriptsize{-0.445}} & \mc{1}{c}{\scriptsize{-0.335}} \\  

     &  & \mc{1}{c}{\scriptsize{\textbf{(0.000)}}} & \mc{1}{c}{\scriptsize{\textbf{(0.020)}}} & \mc{1}{c}{\scriptsize{\textbf{(0.078)}}} & \mc{1}{c}{\scriptsize{\textbf{(0.020)}}} & \mc{1}{c}{\scriptsize{(0.157)}} & \mc{1}{c}{\scriptsize{\textbf{(0.000)}}} & \mc{1}{c}{\scriptsize{\textbf{(0.020)}}} & \mc{1}{c}{\scriptsize{\textbf{(0.020)}}} \\ 
    \hline  

    \\[0.1cm]
    \mc{2}{l}{\scriptsize{\% of Pos. TE ($H_0$: $\le$ 25\% $|$ 10\% Significance)}} & \mc{1}{c}{\scriptsize{18}} & \mc{1}{c}{\scriptsize{27}} & \mc{1}{c}{\scriptsize{18}} & \mc{1}{c}{\scriptsize{18}} & \mc{1}{c}{\scriptsize{9}} & \mc{1}{c}{\scriptsize{36}} & \mc{1}{c}{\scriptsize{18}} & \mc{1}{c}{\scriptsize{45}} \\  

     &  & \mc{1}{c}{\scriptsize{(0.647)}} & \mc{1}{c}{\scriptsize{(0.373)}} & \mc{1}{c}{\scriptsize{(0.667)}} & \mc{1}{c}{\scriptsize{(0.588)}} & \mc{1}{c}{\scriptsize{(0.647)}} & \mc{1}{c}{\scriptsize{(0.333)}} & \mc{1}{c}{\scriptsize{(0.529)}} & \mc{1}{c}{\scriptsize{(0.216)}} \\  

    \mc{2}{l}{\scriptsize{\% of Pos. TE ($H_0$: $\le$ 50\% $|$ 10\% Significance)}} & \mc{1}{c}{\scriptsize{18}} & \mc{1}{c}{\scriptsize{27}} & \mc{1}{c}{\scriptsize{18}} & \mc{1}{c}{\scriptsize{18}} & \mc{1}{c}{\scriptsize{9}} & \mc{1}{c}{\scriptsize{36}} & \mc{1}{c}{\scriptsize{18}} & \mc{1}{c}{\scriptsize{45}} \\  

     &  & \mc{1}{c}{\scriptsize{(0.882)}} & \mc{1}{c}{\scriptsize{(0.804)}} & \mc{1}{c}{\scriptsize{(0.941)}} & \mc{1}{c}{\scriptsize{(0.941)}} & \mc{1}{c}{\scriptsize{(1.000)}} & \mc{1}{c}{\scriptsize{(0.745)}} & \mc{1}{c}{\scriptsize{(0.843)}} & \mc{1}{c}{\scriptsize{(0.627)}} \\  

    \mc{2}{l}{\scriptsize{\% of Pos. TE ($H_0$: $\le$ 75\% $|$ 10\% Significance)}} & \mc{1}{c}{\scriptsize{18}} & \mc{1}{c}{\scriptsize{27}} & \mc{1}{c}{\scriptsize{18}} & \mc{1}{c}{\scriptsize{18}} & \mc{1}{c}{\scriptsize{9}} & \mc{1}{c}{\scriptsize{36}} & \mc{1}{c}{\scriptsize{18}} & \mc{1}{c}{\scriptsize{45}} \\  

     &  & \mc{1}{c}{\scriptsize{(1.000)}} & \mc{1}{c}{\scriptsize{(0.941)}} & \mc{1}{c}{\scriptsize{(1.000)}} & \mc{1}{c}{\scriptsize{(1.000)}} & \mc{1}{c}{\scriptsize{(1.000)}} & \mc{1}{c}{\scriptsize{(0.922)}} & \mc{1}{c}{\scriptsize{(1.000)}} & \mc{1}{c}{\scriptsize{(0.804)}} \\  

  \hline\hline
  \end{tabular}
    \begin{tablenotes}
    \scriptsize
    \item 
Note: This table displays various estimates of the treatment effect of ABC/CARE's center-based care.
Column (1) displays the ITT, without accounting for any controls.
Column (2) displays the ITT conditioning on vector of controls, $X$, consisting of APGAR scores 1 
minute after birth, an indicator for the subject being born prematurely, and an indicator for the 
father being home at baseline. We also apply IPW weights, $W$, to account for attrition.
Columns (3)--(4) are analogous to columns (1)--(2), but we restrict the control sample to subjects
who did not enroll in any alternative care.
Column (5) displys the matching estimate, where we use the Mahalanobis metric and Epanechnikov kernel
to match on controls $X$ listed above, and restrict the control sample to subjects who did not enroll
in any alternative care. Additionally, we apply IPW weights, $W$.
Columns (6)--(8) are analogous to Columns (3)--(5), except we restrict the control sample to subejcts
who did enroll in alternative care. 
The final three pairs of rows display the proportion of treatment effects in the table that are 
socially positive. The first row in each pair displays the percentage of treatment effects, and the
second row presents the inference.

Numbers in parentheses represent the $p$-value from a single hypothesis test, and are obtained from 
the empirical bootstrap distribution generated by 200 resamples of the original data. 
Bold $p$-values indicate significance at the 10\% level.
Blank point estimates indicate that we are unable to obtain estimates due to a lack of support in the data. 

    \end{tablenotes}
  \end{threeparttable}

\end{table}
\end{center}

\begin{center}
	  \begin{tabular}{cccccccccc}
  \toprule

    \scriptsize{Variable} & \scriptsize{Age} & \scriptsize{(1)} & \scriptsize{(2)} & \scriptsize{(3)} & \scriptsize{(4)} & \scriptsize{(5)} & \scriptsize{(6)} & \scriptsize{(7)} & \scriptsize{(8)} \\ 
    \midrule  

    \mc{1}{l}{\scriptsize{Somatization}} & \mc{1}{c}{\scriptsize{21}} & \mc{1}{c}{\scriptsize{-0.021}} & \mc{1}{c}{\scriptsize{-0.038}} & \mc{1}{c}{\scriptsize{-0.259}} & \mc{1}{c}{\scriptsize{-0.459}} & \mc{1}{c}{\scriptsize{-0.248}} & \mc{1}{c}{\scriptsize{0.028}} & \mc{1}{c}{\scriptsize{0.019}} & \mc{1}{c}{\scriptsize{0.026}} \\  

     &  & \mc{1}{c}{\scriptsize{(0.461)}} & \mc{1}{c}{\scriptsize{(0.408)}} & \mc{1}{c}{\scriptsize{(0.145)}} & \mc{1}{c}{\scriptsize{\textbf{(0.053)}}} & \mc{1}{c}{\scriptsize{(0.158)}} & \mc{1}{c}{\scriptsize{(0.605)}} & \mc{1}{c}{\scriptsize{(0.632)}} & \mc{1}{c}{\scriptsize{(0.592)}} \\  

     & \mc{1}{c}{\scriptsize{34}} & \mc{1}{c}{\scriptsize{-0.244}} & \mc{1}{c}{\scriptsize{-0.253}} & \mc{1}{c}{\scriptsize{-0.619}} & \mc{1}{c}{\scriptsize{-0.530}} & \mc{1}{c}{\scriptsize{-0.313}} & \mc{1}{c}{\scriptsize{-0.205}} & \mc{1}{c}{\scriptsize{-0.156}} & \mc{1}{c}{\scriptsize{-0.194}} \\  

     &  & \mc{1}{c}{\scriptsize{(0.145)}} & \mc{1}{c}{\scriptsize{(0.132)}} & \mc{1}{c}{\scriptsize{(0.250)}} & \mc{1}{c}{\scriptsize{(0.184)}} & \mc{1}{c}{\scriptsize{(0.237)}} & \mc{1}{c}{\scriptsize{(0.184)}} & \mc{1}{c}{\scriptsize{(0.197)}} & \mc{1}{c}{\scriptsize{(0.184)}} \\  

    \mc{1}{l}{\scriptsize{Depression}} & \mc{1}{c}{\scriptsize{21}} & \mc{1}{c}{\scriptsize{-0.317}} & \mc{1}{c}{\scriptsize{-0.216}} & \mc{1}{c}{\scriptsize{-0.427}} & \mc{1}{c}{\scriptsize{-0.449}} & \mc{1}{c}{\scriptsize{-0.285}} & \mc{1}{c}{\scriptsize{-0.229}} & \mc{1}{c}{\scriptsize{-0.158}} & \mc{1}{c}{\scriptsize{-0.168}} \\  

     &  & \mc{1}{c}{\scriptsize{\textbf{(0.026)}}} & \mc{1}{c}{\scriptsize{(0.132)}} & \mc{1}{c}{\scriptsize{\textbf{(0.066)}}} & \mc{1}{c}{\scriptsize{(0.105)}} & \mc{1}{c}{\scriptsize{(0.171)}} & \mc{1}{c}{\scriptsize{\textbf{(0.053)}}} & \mc{1}{c}{\scriptsize{(0.158)}} & \mc{1}{c}{\scriptsize{(0.145)}} \\  

     & \mc{1}{c}{\scriptsize{34}} & \mc{1}{c}{\scriptsize{-0.195}} & \mc{1}{c}{\scriptsize{-0.240}} & \mc{1}{c}{\scriptsize{-0.495}} & \mc{1}{c}{\scriptsize{-0.356}} & \mc{1}{c}{\scriptsize{-0.246}} & \mc{1}{c}{\scriptsize{-0.171}} & \mc{1}{c}{\scriptsize{-0.176}} & \mc{1}{c}{\scriptsize{-0.194}} \\  

     &  & \mc{1}{c}{\scriptsize{(0.237)}} & \mc{1}{c}{\scriptsize{(0.171)}} & \mc{1}{c}{\scriptsize{(0.263)}} & \mc{1}{c}{\scriptsize{(0.276)}} & \mc{1}{c}{\scriptsize{(0.237)}} & \mc{1}{c}{\scriptsize{(0.250)}} & \mc{1}{c}{\scriptsize{(0.224)}} & \mc{1}{c}{\scriptsize{(0.197)}} \\  

    \mc{1}{l}{\scriptsize{Anxiety}} & \mc{1}{c}{\scriptsize{21}} & \mc{1}{c}{\scriptsize{-0.142}} & \mc{1}{c}{\scriptsize{-0.032}} & \mc{1}{c}{\scriptsize{-0.164}} & \mc{1}{c}{\scriptsize{-0.291}} & \mc{1}{c}{\scriptsize{-0.063}} & \mc{1}{c}{\scriptsize{-0.111}} & \mc{1}{c}{\scriptsize{-0.011}} & \mc{1}{c}{\scriptsize{-0.041}} \\  

     &  & \mc{1}{c}{\scriptsize{(0.132)}} & \mc{1}{c}{\scriptsize{(0.421)}} & \mc{1}{c}{\scriptsize{(0.250)}} & \mc{1}{c}{\scriptsize{\textbf{(0.092)}}} & \mc{1}{c}{\scriptsize{(0.395)}} & \mc{1}{c}{\scriptsize{(0.171)}} & \mc{1}{c}{\scriptsize{(0.461)}} & \mc{1}{c}{\scriptsize{(0.329)}} \\  

     & \mc{1}{c}{\scriptsize{34}} & \mc{1}{c}{\scriptsize{-0.343}} & \mc{1}{c}{\scriptsize{-0.363}} & \mc{1}{c}{\scriptsize{-0.732}} & \mc{1}{c}{\scriptsize{-0.653}} & \mc{1}{c}{\scriptsize{-0.436}} & \mc{1}{c}{\scriptsize{-0.306}} & \mc{1}{c}{\scriptsize{-0.266}} & \mc{1}{c}{\scriptsize{-0.334}} \\  

     &  & \mc{1}{c}{\scriptsize{\textbf{(0.066)}}} & \mc{1}{c}{\scriptsize{\textbf{(0.079)}}} & \mc{1}{c}{\scriptsize{(0.171)}} & \mc{1}{c}{\scriptsize{(0.145)}} & \mc{1}{c}{\scriptsize{(0.184)}} & \mc{1}{c}{\scriptsize{(0.132)}} & \mc{1}{c}{\scriptsize{(0.105)}} & \mc{1}{c}{\scriptsize{(0.105)}} \\  

    \mc{1}{l}{\scriptsize{Hostility}} & \mc{1}{c}{\scriptsize{21}} & \mc{1}{c}{\scriptsize{-0.336}} & \mc{1}{c}{\scriptsize{-0.231}} & \mc{1}{c}{\scriptsize{-0.469}} & \mc{1}{c}{\scriptsize{-0.652}} & \mc{1}{c}{\scriptsize{-0.379}} & \mc{1}{c}{\scriptsize{-0.249}} & \mc{1}{c}{\scriptsize{-0.214}} & \mc{1}{c}{\scriptsize{-0.207}} \\  

     &  & \mc{1}{c}{\scriptsize{\textbf{(0.026)}}} & \mc{1}{c}{\scriptsize{(0.118)}} & \mc{1}{c}{\scriptsize{(0.118)}} & \mc{1}{c}{\scriptsize{\textbf{(0.092)}}} & \mc{1}{c}{\scriptsize{(0.171)}} & \mc{1}{c}{\scriptsize{\textbf{(0.053)}}} & \mc{1}{c}{\scriptsize{(0.105)}} & \mc{1}{c}{\scriptsize{(0.118)}} \\  

     & \mc{1}{c}{\scriptsize{34}} & \mc{1}{c}{\scriptsize{-0.255}} & \mc{1}{c}{\scriptsize{-0.195}} & \mc{1}{c}{\scriptsize{-0.439}} & \mc{1}{c}{\scriptsize{-0.429}} & \mc{1}{c}{\scriptsize{-0.287}} & \mc{1}{c}{\scriptsize{-0.246}} & \mc{1}{c}{\scriptsize{-0.145}} & \mc{1}{c}{\scriptsize{-0.189}} \\  

     &  & \mc{1}{c}{\scriptsize{(0.132)}} & \mc{1}{c}{\scriptsize{(0.158)}} & \mc{1}{c}{\scriptsize{(0.105)}} & \mc{1}{c}{\scriptsize{(0.132)}} & \mc{1}{c}{\scriptsize{(0.158)}} & \mc{1}{c}{\scriptsize{(0.171)}} & \mc{1}{c}{\scriptsize{(0.184)}} & \mc{1}{c}{\scriptsize{(0.211)}} \\  

    \mc{1}{l}{\scriptsize{Global Severity Index}} & \mc{1}{c}{\scriptsize{21}} & \mc{1}{c}{\scriptsize{-0.192}} & \mc{1}{c}{\scriptsize{-0.127}} & \mc{1}{c}{\scriptsize{-0.244}} & \mc{1}{c}{\scriptsize{-0.361}} & \mc{1}{c}{\scriptsize{-0.150}} & \mc{1}{c}{\scriptsize{-0.157}} & \mc{1}{c}{\scriptsize{-0.099}} & \mc{1}{c}{\scriptsize{-0.106}} \\  

     &  & \mc{1}{c}{\scriptsize{\textbf{(0.053)}}} & \mc{1}{c}{\scriptsize{(0.145)}} & \mc{1}{c}{\scriptsize{(0.145)}} & \mc{1}{c}{\scriptsize{\textbf{(0.092)}}} & \mc{1}{c}{\scriptsize{(0.250)}} & \mc{1}{c}{\scriptsize{\textbf{(0.092)}}} & \mc{1}{c}{\scriptsize{(0.224)}} & \mc{1}{c}{\scriptsize{(0.145)}} \\  

     & \mc{1}{c}{\scriptsize{34}} & \mc{1}{c}{\scriptsize{-4.691}} & \mc{1}{c}{\scriptsize{-5.134}} & \mc{1}{c}{\scriptsize{-11.070}} & \mc{1}{c}{\scriptsize{-9.238}} & \mc{1}{c}{\scriptsize{-5.971}} & \mc{1}{c}{\scriptsize{-4.090}} & \mc{1}{c}{\scriptsize{-3.590}} & \mc{1}{c}{\scriptsize{-4.332}} \\  

     &  & \mc{1}{c}{\scriptsize{(0.145)}} & \mc{1}{c}{\scriptsize{(0.105)}} & \mc{1}{c}{\scriptsize{(0.250)}} & \mc{1}{c}{\scriptsize{(0.184)}} & \mc{1}{c}{\scriptsize{(0.211)}} & \mc{1}{c}{\scriptsize{(0.197)}} & \mc{1}{c}{\scriptsize{(0.171)}} & \mc{1}{c}{\scriptsize{(0.184)}} \\  

    \mc{1}{l}{\scriptsize{BSI Factor}} & \mc{1}{c}{\scriptsize{21 and 34}} & \mc{1}{c}{\scriptsize{-0.549}} & \mc{1}{c}{\scriptsize{-0.302}} & \mc{1}{c}{\scriptsize{-1.130}} & \mc{1}{c}{\scriptsize{-1.165}} & \mc{1}{c}{\scriptsize{-0.587}} & \mc{1}{c}{\scriptsize{-0.456}} & \mc{1}{c}{\scriptsize{-0.097}} & \mc{1}{c}{\scriptsize{-0.207}} \\  

     &  & \mc{1}{c}{\scriptsize{\textbf{(0.000)}}} & \mc{1}{c}{\scriptsize{(0.132)}} & \mc{1}{c}{\scriptsize{\textbf{(0.079)}}} & \mc{1}{c}{\scriptsize{(0.105)}} & \mc{1}{c}{\scriptsize{(0.171)}} & \mc{1}{c}{\scriptsize{\textbf{(0.000)}}} & \mc{1}{c}{\scriptsize{(0.382)}} & \mc{1}{c}{\scriptsize{(0.250)}} \\ 
    \midrule  

    \mc{2}{l}{\scriptsize{\% of Pos. TE ($H_0$: $\le$ 50\%)}} & \mc{1}{c}{\scriptsize{100}} & \mc{1}{c}{\scriptsize{100}} & \mc{1}{c}{\scriptsize{100}} & \mc{1}{c}{\scriptsize{100}} & \mc{1}{c}{\scriptsize{100}} & \mc{1}{c}{\scriptsize{91}} & \mc{1}{c}{\scriptsize{91}} & \mc{1}{c}{\scriptsize{91}} \\  

     &  & \mc{1}{c}{\scriptsize{\textbf{(0.000)}}} & \mc{1}{c}{\scriptsize{\textbf{(0.000)}}} & \mc{1}{c}{\scriptsize{\textbf{(0.000)}}} & \mc{1}{c}{\scriptsize{\textbf{(0.000)}}} & \mc{1}{c}{\scriptsize{\textbf{(0.000)}}} & \mc{1}{c}{\scriptsize{\textbf{(0.000)}}} & \mc{1}{c}{\scriptsize{\textbf{(0.000)}}} & \mc{1}{c}{\scriptsize{\textbf{(0.000)}}} \\  

    \mc{2}{l}{\scriptsize{\% of Pos. TE ($H_0$: $\le$ 10\% $|$ 10\% Significance)}} & \mc{1}{c}{\scriptsize{64}} & \mc{1}{c}{\scriptsize{27}} & \mc{1}{c}{\scriptsize{9}} & \mc{1}{c}{\scriptsize{36}} & \mc{1}{c}{\scriptsize{0}} & \mc{1}{c}{\scriptsize{55}} & \mc{1}{c}{\scriptsize{9}} & \mc{1}{c}{\scriptsize{18}} \\  

     &  & \mc{1}{c}{\scriptsize{\textbf{(0.079)}}} & \mc{1}{c}{\scriptsize{(0.224)}} & \mc{1}{c}{\scriptsize{(0.513)}} & \mc{1}{c}{\scriptsize{(0.105)}} & \mc{1}{c}{\scriptsize{(0.539)}} & \mc{1}{c}{\scriptsize{\textbf{(0.092)}}} & \mc{1}{c}{\scriptsize{(0.250)}} & \mc{1}{c}{\scriptsize{(0.303)}} \\  

  \bottomrule
  \end{tabular}
\end{center}

\begin{center}
	  \begin{tabular}{cccccccccc}
  \toprule

    \scriptsize{Variable} & \scriptsize{Age} & \scriptsize{(1)} & \scriptsize{(2)} & \scriptsize{(3)} & \scriptsize{(4)} & \scriptsize{(5)} & \scriptsize{(6)} & \scriptsize{(7)} & \scriptsize{(8)} \\ 
    \midrule  

    \mc{1}{l}{\scriptsize{Participates in Activity}} & \mc{1}{c}{\scriptsize{12}} & \mc{1}{c}{\scriptsize{0.114}} & \mc{1}{c}{\scriptsize{0.089}} & \mc{1}{c}{\scriptsize{0.111}} & \mc{1}{c}{\scriptsize{0.087}} & \mc{1}{c}{\scriptsize{0.081}} & \mc{1}{c}{\scriptsize{0.126}} & \mc{1}{c}{\scriptsize{0.112}} & \mc{1}{c}{\scriptsize{0.114}} \\  

     &  & \mc{1}{c}{\scriptsize{(0.145)}} & \mc{1}{c}{\scriptsize{(0.211)}} & \mc{1}{c}{\scriptsize{(0.211)}} & \mc{1}{c}{\scriptsize{(0.329)}} & \mc{1}{c}{\scriptsize{(0.316)}} & \mc{1}{c}{\scriptsize{(0.145)}} & \mc{1}{c}{\scriptsize{(0.211)}} & \mc{1}{c}{\scriptsize{(0.184)}} \\  

    \mc{1}{l}{\scriptsize{Time spent reading}} & \mc{1}{c}{\scriptsize{12}} & \mc{1}{c}{\scriptsize{1.696}} & \mc{1}{c}{\scriptsize{1.438}} & \mc{1}{c}{\scriptsize{0.619}} & \mc{1}{c}{\scriptsize{-0.088}} & \mc{1}{c}{\scriptsize{1.106}} & \mc{1}{c}{\scriptsize{1.862}} & \mc{1}{c}{\scriptsize{1.709}} & \mc{1}{c}{\scriptsize{2.165}} \\  

     &  & \mc{1}{c}{\scriptsize{(0.132)}} & \mc{1}{c}{\scriptsize{(0.171)}} & \mc{1}{c}{\scriptsize{(0.382)}} & \mc{1}{c}{\scriptsize{(0.500)}} & \mc{1}{c}{\scriptsize{(0.342)}} & \mc{1}{c}{\scriptsize{\textbf{(0.092)}}} & \mc{1}{c}{\scriptsize{(0.145)}} & \mc{1}{c}{\scriptsize{\textbf{(0.066)}}} \\  

    \mc{1}{l}{\scriptsize{Good Description of Self}} & \mc{1}{c}{\scriptsize{12}} & \mc{1}{c}{\scriptsize{0.053}} & \mc{1}{c}{\scriptsize{-0.040}} & \mc{1}{c}{\scriptsize{-0.072}} & \mc{1}{c}{\scriptsize{-0.366}} & \mc{1}{c}{\scriptsize{-0.178}} & \mc{1}{c}{\scriptsize{0.102}} & \mc{1}{c}{\scriptsize{0.010}} & \mc{1}{c}{\scriptsize{-0.045}} \\  

     &  & \mc{1}{c}{\scriptsize{(0.355)}} & \mc{1}{c}{\scriptsize{(0.645)}} & \mc{1}{c}{\scriptsize{(0.618)}} & \mc{1}{c}{\scriptsize{(0.947)}} & \mc{1}{c}{\scriptsize{(0.855)}} & \mc{1}{c}{\scriptsize{(0.224)}} & \mc{1}{c}{\scriptsize{(0.447)}} & \mc{1}{c}{\scriptsize{(0.658)}} \\  

    \mc{1}{l}{\scriptsize{Views Self as Dumb}} & \mc{1}{c}{\scriptsize{12}} & \mc{1}{c}{\scriptsize{0.025}} & \mc{1}{c}{\scriptsize{0.030}} & \mc{1}{c}{\scriptsize{-0.105}} & \mc{1}{c}{\scriptsize{-0.286}} & \mc{1}{c}{\scriptsize{-0.127}} & \mc{1}{c}{\scriptsize{0.073}} & \mc{1}{c}{\scriptsize{0.070}} & \mc{1}{c}{\scriptsize{0.068}} \\  

     &  & \mc{1}{c}{\scriptsize{(0.618)}} & \mc{1}{c}{\scriptsize{(0.526)}} & \mc{1}{c}{\scriptsize{(0.303)}} & \mc{1}{c}{\scriptsize{(0.158)}} & \mc{1}{c}{\scriptsize{(0.289)}} & \mc{1}{c}{\scriptsize{(0.724)}} & \mc{1}{c}{\scriptsize{(0.618)}} & \mc{1}{c}{\scriptsize{(0.697)}} \\  

    \mc{1}{l}{\scriptsize{Views Self as Clumsy}} & \mc{1}{c}{\scriptsize{12}} & \mc{1}{c}{\scriptsize{-0.059}} & \mc{1}{c}{\scriptsize{-0.098}} & \mc{1}{c}{\scriptsize{0.085}} & \mc{1}{c}{\scriptsize{0.098}} & \mc{1}{c}{\scriptsize{0.089}} & \mc{1}{c}{\scriptsize{-0.101}} & \mc{1}{c}{\scriptsize{-0.140}} & \mc{1}{c}{\scriptsize{-0.141}} \\  

     &  & \mc{1}{c}{\scriptsize{(0.211)}} & \mc{1}{c}{\scriptsize{(0.211)}} & \mc{1}{c}{\scriptsize{(0.737)}} & \mc{1}{c}{\scriptsize{(0.763)}} & \mc{1}{c}{\scriptsize{(0.776)}} & \mc{1}{c}{\scriptsize{(0.145)}} & \mc{1}{c}{\scriptsize{(0.145)}} & \mc{1}{c}{\scriptsize{(0.132)}} \\  

    \mc{1}{l}{\scriptsize{Views Self as Not Liked}} & \mc{1}{c}{\scriptsize{12}} & \mc{1}{c}{\scriptsize{-0.116}} & \mc{1}{c}{\scriptsize{-0.055}} & \mc{1}{c}{\scriptsize{0.006}} & \mc{1}{c}{\scriptsize{-0.019}} & \mc{1}{c}{\scriptsize{-0.062}} & \mc{1}{c}{\scriptsize{-0.153}} & \mc{1}{c}{\scriptsize{-0.083}} & \mc{1}{c}{\scriptsize{-0.214}} \\  

     &  & \mc{1}{c}{\scriptsize{(0.118)}} & \mc{1}{c}{\scriptsize{(0.303)}} & \mc{1}{c}{\scriptsize{(0.526)}} & \mc{1}{c}{\scriptsize{(0.421)}} & \mc{1}{c}{\scriptsize{(0.342)}} & \mc{1}{c}{\scriptsize{\textbf{(0.092)}}} & \mc{1}{c}{\scriptsize{(0.211)}} & \mc{1}{c}{\scriptsize{\textbf{(0.053)}}} \\  

    \mc{1}{l}{\scriptsize{Proud about Self}} & \mc{1}{c}{\scriptsize{12}} & \mc{1}{c}{\scriptsize{-0.047}} & \mc{1}{c}{\scriptsize{-0.060}} & \mc{1}{c}{\scriptsize{0.013}} & \mc{1}{c}{\scriptsize{0.068}} & \mc{1}{c}{\scriptsize{0.023}} & \mc{1}{c}{\scriptsize{-0.062}} & \mc{1}{c}{\scriptsize{-0.069}} & \mc{1}{c}{\scriptsize{-0.058}} \\  

     &  & \mc{1}{c}{\scriptsize{(0.658)}} & \mc{1}{c}{\scriptsize{(0.697)}} & \mc{1}{c}{\scriptsize{(0.474)}} & \mc{1}{c}{\scriptsize{(0.368)}} & \mc{1}{c}{\scriptsize{(0.421)}} & \mc{1}{c}{\scriptsize{(0.724)}} & \mc{1}{c}{\scriptsize{(0.697)}} & \mc{1}{c}{\scriptsize{(0.711)}} \\  

    \mc{1}{l}{\scriptsize{Family Proud of You}} & \mc{1}{c}{\scriptsize{12}} & \mc{1}{c}{\scriptsize{0.021}} & \mc{1}{c}{\scriptsize{-0.007}} & \mc{1}{c}{\scriptsize{0.033}} & \mc{1}{c}{\scriptsize{-0.069}} & \mc{1}{c}{\scriptsize{0.023}} & \mc{1}{c}{\scriptsize{0.012}} & \mc{1}{c}{\scriptsize{0.008}} & \mc{1}{c}{\scriptsize{0.002}} \\  

     &  & \mc{1}{c}{\scriptsize{(0.408)}} & \mc{1}{c}{\scriptsize{(0.566)}} & \mc{1}{c}{\scriptsize{(0.447)}} & \mc{1}{c}{\scriptsize{(0.658)}} & \mc{1}{c}{\scriptsize{(0.447)}} & \mc{1}{c}{\scriptsize{(0.434)}} & \mc{1}{c}{\scriptsize{(0.500)}} & \mc{1}{c}{\scriptsize{(0.500)}} \\  

    \mc{1}{l}{\scriptsize{Feels Inadequate, Inferior}} & \mc{1}{c}{\scriptsize{12}} & \mc{1}{c}{\scriptsize{-0.025}} & \mc{1}{c}{\scriptsize{-0.017}} & \mc{1}{c}{\scriptsize{0.183}} & \mc{1}{c}{\scriptsize{0.178}} & \mc{1}{c}{\scriptsize{0.220}} & \mc{1}{c}{\scriptsize{-0.030}} & \mc{1}{c}{\scriptsize{-0.063}} & \mc{1}{c}{\scriptsize{-0.051}} \\  

     &  & \mc{1}{c}{\scriptsize{(0.461)}} & \mc{1}{c}{\scriptsize{(0.408)}} & \mc{1}{c}{\scriptsize{(0.908)}} & \mc{1}{c}{\scriptsize{(0.803)}} & \mc{1}{c}{\scriptsize{(0.947)}} & \mc{1}{c}{\scriptsize{(0.434)}} & \mc{1}{c}{\scriptsize{(0.355)}} & \mc{1}{c}{\scriptsize{(0.395)}} \\  

    \mc{1}{l}{\scriptsize{Withdraws Excessively}} & \mc{1}{c}{\scriptsize{12}} & \mc{1}{c}{\scriptsize{-0.040}} & \mc{1}{c}{\scriptsize{-0.029}} & \mc{1}{c}{\scriptsize{0.137}} & \mc{1}{c}{\scriptsize{0.018}} & \mc{1}{c}{\scriptsize{0.131}} & \mc{1}{c}{\scriptsize{-0.070}} & \mc{1}{c}{\scriptsize{-0.064}} & \mc{1}{c}{\scriptsize{-0.068}} \\  

     &  & \mc{1}{c}{\scriptsize{(0.368)}} & \mc{1}{c}{\scriptsize{(0.421)}} & \mc{1}{c}{\scriptsize{(0.684)}} & \mc{1}{c}{\scriptsize{(0.461)}} & \mc{1}{c}{\scriptsize{(0.724)}} & \mc{1}{c}{\scriptsize{(0.263)}} & \mc{1}{c}{\scriptsize{(0.329)}} & \mc{1}{c}{\scriptsize{(0.329)}} \\  

    \mc{1}{l}{\scriptsize{Ignores Situation}} & \mc{1}{c}{\scriptsize{12}} & \mc{1}{c}{\scriptsize{-0.192}} & \mc{1}{c}{\scriptsize{-0.275}} & \mc{1}{c}{\scriptsize{-0.268}} & \mc{1}{c}{\scriptsize{-0.511}} & \mc{1}{c}{\scriptsize{-0.312}} & \mc{1}{c}{\scriptsize{-0.166}} & \mc{1}{c}{\scriptsize{-0.218}} & \mc{1}{c}{\scriptsize{-0.131}} \\  

     &  & \mc{1}{c}{\scriptsize{\textbf{(0.039)}}} & \mc{1}{c}{\scriptsize{\textbf{(0.026)}}} & \mc{1}{c}{\scriptsize{(0.224)}} & \mc{1}{c}{\scriptsize{\textbf{(0.053)}}} & \mc{1}{c}{\scriptsize{(0.158)}} & \mc{1}{c}{\scriptsize{\textbf{(0.092)}}} & \mc{1}{c}{\scriptsize{\textbf{(0.053)}}} & \mc{1}{c}{\scriptsize{(0.211)}} \\  

    \mc{1}{l}{\scriptsize{Not Cope with Prob.}} & \mc{1}{c}{\scriptsize{12}} & \mc{1}{c}{\scriptsize{-0.084}} & \mc{1}{c}{\scriptsize{-0.106}} & \mc{1}{c}{\scriptsize{-0.065}} & \mc{1}{c}{\scriptsize{-0.177}} & \mc{1}{c}{\scriptsize{-0.021}} & \mc{1}{c}{\scriptsize{-0.077}} & \mc{1}{c}{\scriptsize{-0.107}} & \mc{1}{c}{\scriptsize{-0.033}} \\  

     &  & \mc{1}{c}{\scriptsize{(0.197)}} & \mc{1}{c}{\scriptsize{(0.171)}} & \mc{1}{c}{\scriptsize{(0.368)}} & \mc{1}{c}{\scriptsize{(0.145)}} & \mc{1}{c}{\scriptsize{(0.474)}} & \mc{1}{c}{\scriptsize{(0.237)}} & \mc{1}{c}{\scriptsize{(0.197)}} & \mc{1}{c}{\scriptsize{(0.408)}} \\  

    \mc{1}{l}{\scriptsize{Often Mad of Angry}} & \mc{1}{c}{\scriptsize{12}} & \mc{1}{c}{\scriptsize{-0.130}} & \mc{1}{c}{\scriptsize{-0.223}} & \mc{1}{c}{\scriptsize{0.104}} & \mc{1}{c}{\scriptsize{0.004}} & \mc{1}{c}{\scriptsize{0.047}} & \mc{1}{c}{\scriptsize{-0.193}} & \mc{1}{c}{\scriptsize{-0.263}} & \mc{1}{c}{\scriptsize{-0.272}} \\  

     &  & \mc{1}{c}{\scriptsize{(0.105)}} & \mc{1}{c}{\scriptsize{\textbf{(0.053)}}} & \mc{1}{c}{\scriptsize{(0.934)}} & \mc{1}{c}{\scriptsize{(0.421)}} & \mc{1}{c}{\scriptsize{(0.842)}} & \mc{1}{c}{\scriptsize{\textbf{(0.092)}}} & \mc{1}{c}{\scriptsize{\textbf{(0.066)}}} & \mc{1}{c}{\scriptsize{\textbf{(0.092)}}} \\  

    \mc{1}{l}{\scriptsize{Impulsivity}} & \mc{1}{c}{\scriptsize{12}} & \mc{1}{c}{\scriptsize{-0.039}} & \mc{1}{c}{\scriptsize{0.037}} & \mc{1}{c}{\scriptsize{-0.072}} & \mc{1}{c}{\scriptsize{-0.170}} & \mc{1}{c}{\scriptsize{-0.042}} & \mc{1}{c}{\scriptsize{-0.051}} & \mc{1}{c}{\scriptsize{0.048}} & \mc{1}{c}{\scriptsize{-0.009}} \\  

     &  & \mc{1}{c}{\scriptsize{(0.355)}} & \mc{1}{c}{\scriptsize{(0.645)}} & \mc{1}{c}{\scriptsize{(0.355)}} & \mc{1}{c}{\scriptsize{(0.158)}} & \mc{1}{c}{\scriptsize{(0.382)}} & \mc{1}{c}{\scriptsize{(0.316)}} & \mc{1}{c}{\scriptsize{(0.671)}} & \mc{1}{c}{\scriptsize{(0.474)}} \\  

    \mc{1}{l}{\scriptsize{Significant Fears}} & \mc{1}{c}{\scriptsize{12}} & \mc{1}{c}{\scriptsize{-0.160}} & \mc{1}{c}{\scriptsize{-0.192}} & \mc{1}{c}{\scriptsize{0.183}} & \mc{1}{c}{\scriptsize{0.187}} & \mc{1}{c}{\scriptsize{0.200}} & \mc{1}{c}{\scriptsize{-0.264}} & \mc{1}{c}{\scriptsize{-0.268}} & \mc{1}{c}{\scriptsize{-0.296}} \\  

     &  & \mc{1}{c}{\scriptsize{\textbf{(0.000)}}} & \mc{1}{c}{\scriptsize{\textbf{(0.013)}}} & \mc{1}{c}{\scriptsize{(0.842)}} & \mc{1}{c}{\scriptsize{(0.842)}} & \mc{1}{c}{\scriptsize{(0.855)}} & \mc{1}{c}{\scriptsize{\textbf{(0.000)}}} & \mc{1}{c}{\scriptsize{\textbf{(0.013)}}} & \mc{1}{c}{\scriptsize{\textbf{(0.000)}}} \\  

    \mc{1}{l}{\scriptsize{Denies Any Worries}} & \mc{1}{c}{\scriptsize{12}} & \mc{1}{c}{\scriptsize{-0.239}} & \mc{1}{c}{\scriptsize{-0.331}} & \mc{1}{c}{\scriptsize{-0.163}} & \mc{1}{c}{\scriptsize{-0.164}} & \mc{1}{c}{\scriptsize{-0.156}} & \mc{1}{c}{\scriptsize{-0.266}} & \mc{1}{c}{\scriptsize{-0.357}} & \mc{1}{c}{\scriptsize{-0.281}} \\  

     &  & \mc{1}{c}{\scriptsize{\textbf{(0.000)}}} & \mc{1}{c}{\scriptsize{\textbf{(0.000)}}} & \mc{1}{c}{\scriptsize{(0.224)}} & \mc{1}{c}{\scriptsize{(0.158)}} & \mc{1}{c}{\scriptsize{(0.224)}} & \mc{1}{c}{\scriptsize{\textbf{(0.000)}}} & \mc{1}{c}{\scriptsize{\textbf{(0.000)}}} & \mc{1}{c}{\scriptsize{\textbf{(0.000)}}} \\ 
    \midrule  

    \mc{2}{l}{\scriptsize{\% of Pos. TE ($H_0$: $\le$ 50\%)}} & \mc{1}{c}{\scriptsize{88}} & \mc{1}{c}{\scriptsize{69}} & \mc{1}{c}{\scriptsize{56}} & \mc{1}{c}{\scriptsize{50}} & \mc{1}{c}{\scriptsize{62}} & \mc{1}{c}{\scriptsize{88}} & \mc{1}{c}{\scriptsize{81}} & \mc{1}{c}{\scriptsize{81}} \\  

     &  & \mc{1}{c}{\scriptsize{\textbf{(0.000)}}} & \mc{1}{c}{\scriptsize{\textbf{(0.039)}}} & \mc{1}{c}{\scriptsize{(0.303)}} & \mc{1}{c}{\scriptsize{(0.513)}} & \mc{1}{c}{\scriptsize{(0.118)}} & \mc{1}{c}{\scriptsize{\textbf{(0.000)}}} & \mc{1}{c}{\scriptsize{\textbf{(0.000)}}} & \mc{1}{c}{\scriptsize{\textbf{(0.000)}}} \\  

    \mc{2}{l}{\scriptsize{\% of Pos. TE ($H_0$: $\le$ 10\% $|$ 10\% Significance)}} & \mc{1}{c}{\scriptsize{31}} & \mc{1}{c}{\scriptsize{25}} & \mc{1}{c}{\scriptsize{0}} & \mc{1}{c}{\scriptsize{6}} & \mc{1}{c}{\scriptsize{0}} & \mc{1}{c}{\scriptsize{38}} & \mc{1}{c}{\scriptsize{31}} & \mc{1}{c}{\scriptsize{31}} \\  

     &  & \mc{1}{c}{\scriptsize{\textbf{(0.039)}}} & \mc{1}{c}{\scriptsize{\textbf{(0.039)}}} & \mc{1}{c}{\scriptsize{(1.000)}} & \mc{1}{c}{\scriptsize{(0.461)}} & \mc{1}{c}{\scriptsize{(0.789)}} & \mc{1}{c}{\scriptsize{\textbf{(0.000)}}} & \mc{1}{c}{\scriptsize{\textbf{(0.000)}}} & \mc{1}{c}{\scriptsize{\textbf{(0.000)}}} \\  

  \bottomrule
  \end{tabular}
\end{center}
\section{Treatment Effects for Male Sample}


\begin{center}
	  \begin{tabular}{cccccccccc}
  \toprule

    \scriptsize{Variable} & \scriptsize{Age} & \scriptsize{(1)} & \scriptsize{(2)} & \scriptsize{(3)} & \scriptsize{(4)} & \scriptsize{(5)} & \scriptsize{(6)} & \scriptsize{(7)} & \scriptsize{(8)} \\ 
    \midrule  

    \mc{1}{l}{\scriptsize{Std. IQ Test}} & \mc{1}{c}{\scriptsize{2}} & \mc{1}{c}{\scriptsize{9.528}} & \mc{1}{c}{\scriptsize{10.360}} & \mc{1}{c}{\scriptsize{6.875}} & \mc{1}{c}{\scriptsize{8.336}} & \mc{1}{c}{\scriptsize{7.950}} & \mc{1}{c}{\scriptsize{10.286}} & \mc{1}{c}{\scriptsize{10.890}} & \mc{1}{c}{\scriptsize{11.078}} \\  

     &  & \mc{1}{c}{\scriptsize{\textbf{(0.000)}}} & \mc{1}{c}{\scriptsize{\textbf{(0.000)}}} & \mc{1}{c}{\scriptsize{(0.999)}} & \mc{1}{c}{\scriptsize{\textbf{(0.001)}}} & \mc{1}{c}{\scriptsize{\textbf{(0.024)}}} & \mc{1}{c}{\scriptsize{\textbf{(0.000)}}} & \mc{1}{c}{\scriptsize{\textbf{(0.000)}}} & \mc{1}{c}{\scriptsize{\textbf{(0.000)}}} \\  

     & \mc{1}{c}{\scriptsize{3}} & \mc{1}{c}{\scriptsize{13.410}} & \mc{1}{c}{\scriptsize{14.748}} & \mc{1}{c}{\scriptsize{13.896}} & \mc{1}{c}{\scriptsize{16.532}} & \mc{1}{c}{\scriptsize{15.487}} & \mc{1}{c}{\scriptsize{13.271}} & \mc{1}{c}{\scriptsize{14.145}} & \mc{1}{c}{\scriptsize{14.301}} \\  

     &  & \mc{1}{c}{\scriptsize{\textbf{(0.000)}}} & \mc{1}{c}{\scriptsize{\textbf{(0.000)}}} & \mc{1}{c}{\scriptsize{(0.999)}} & \mc{1}{c}{\scriptsize{\textbf{(0.001)}}} & \mc{1}{c}{\scriptsize{\textbf{(0.000)}}} & \mc{1}{c}{\scriptsize{\textbf{(0.000)}}} & \mc{1}{c}{\scriptsize{\textbf{(0.000)}}} & \mc{1}{c}{\scriptsize{\textbf{(0.000)}}} \\  

     & \mc{1}{c}{\scriptsize{3.5}} & \mc{1}{c}{\scriptsize{8.756}} & \mc{1}{c}{\scriptsize{8.415}} & \mc{1}{c}{\scriptsize{6.354}} & \mc{1}{c}{\scriptsize{6.916}} & \mc{1}{c}{\scriptsize{6.812}} & \mc{1}{c}{\scriptsize{9.443}} & \mc{1}{c}{\scriptsize{8.821}} & \mc{1}{c}{\scriptsize{9.040}} \\  

     &  & \mc{1}{c}{\scriptsize{\textbf{(0.002)}}} & \mc{1}{c}{\scriptsize{\textbf{(0.001)}}} & \mc{1}{c}{\scriptsize{(0.999)}} & \mc{1}{c}{\scriptsize{\textbf{(0.001)}}} & \mc{1}{c}{\scriptsize{\textbf{(0.053)}}} & \mc{1}{c}{\scriptsize{\textbf{(0.003)}}} & \mc{1}{c}{\scriptsize{\textbf{(0.002)}}} & \mc{1}{c}{\scriptsize{\textbf{(0.002)}}} \\  

     & \mc{1}{c}{\scriptsize{4}} & \mc{1}{c}{\scriptsize{12.089}} & \mc{1}{c}{\scriptsize{12.124}} & \mc{1}{c}{\scriptsize{8.950}} & \mc{1}{c}{\scriptsize{9.742}} & \mc{1}{c}{\scriptsize{9.725}} & \mc{1}{c}{\scriptsize{12.986}} & \mc{1}{c}{\scriptsize{12.743}} & \mc{1}{c}{\scriptsize{13.489}} \\  

     &  & \mc{1}{c}{\scriptsize{\textbf{(0.000)}}} & \mc{1}{c}{\scriptsize{\textbf{(0.000)}}} & \mc{1}{c}{\scriptsize{(0.999)}} & \mc{1}{c}{\scriptsize{\textbf{(0.001)}}} & \mc{1}{c}{\scriptsize{\textbf{(0.025)}}} & \mc{1}{c}{\scriptsize{\textbf{(0.000)}}} & \mc{1}{c}{\scriptsize{\textbf{(0.000)}}} & \mc{1}{c}{\scriptsize{\textbf{(0.000)}}} \\  

     & \mc{1}{c}{\scriptsize{4.5}} & \mc{1}{c}{\scriptsize{8.508}} & \mc{1}{c}{\scriptsize{8.583}} & \mc{1}{c}{\scriptsize{10.411}} & \mc{1}{c}{\scriptsize{11.182}} & \mc{1}{c}{\scriptsize{10.668}} & \mc{1}{c}{\scriptsize{7.964}} & \mc{1}{c}{\scriptsize{7.748}} & \mc{1}{c}{\scriptsize{7.795}} \\  

     &  & \mc{1}{c}{\scriptsize{\textbf{(0.001)}}} & \mc{1}{c}{\scriptsize{\textbf{(0.000)}}} & \mc{1}{c}{\scriptsize{(0.999)}} & \mc{1}{c}{\scriptsize{\textbf{(0.001)}}} & \mc{1}{c}{\scriptsize{\textbf{(0.008)}}} & \mc{1}{c}{\scriptsize{\textbf{(0.004)}}} & \mc{1}{c}{\scriptsize{\textbf{(0.003)}}} & \mc{1}{c}{\scriptsize{\textbf{(0.006)}}} \\  

     & \mc{1}{c}{\scriptsize{5}} & \mc{1}{c}{\scriptsize{7.697}} & \mc{1}{c}{\scriptsize{7.067}} & \mc{1}{c}{\scriptsize{4.643}} & \mc{1}{c}{\scriptsize{5.116}} & \mc{1}{c}{\scriptsize{5.034}} & \mc{1}{c}{\scriptsize{8.679}} & \mc{1}{c}{\scriptsize{7.716}} & \mc{1}{c}{\scriptsize{8.174}} \\  

     &  & \mc{1}{c}{\scriptsize{\textbf{(0.000)}}} & \mc{1}{c}{\scriptsize{\textbf{(0.005)}}} & \mc{1}{c}{\scriptsize{\textbf{(0.001)}}} & \mc{1}{c}{\scriptsize{(0.999)}} & \mc{1}{c}{\scriptsize{(0.182)}} & \mc{1}{c}{\scriptsize{\textbf{(0.000)}}} & \mc{1}{c}{\scriptsize{\textbf{(0.002)}}} & \mc{1}{c}{\scriptsize{\textbf{(0.005)}}} \\  

     & \mc{1}{c}{\scriptsize{6.6}} & \mc{1}{c}{\scriptsize{5.803}} & \mc{1}{c}{\scriptsize{7.865}} & \mc{1}{c}{\scriptsize{0.831}} & \mc{1}{c}{\scriptsize{5.791}} & \mc{1}{c}{\scriptsize{3.506}} & \mc{1}{c}{\scriptsize{5.916}} & \mc{1}{c}{\scriptsize{7.543}} & \mc{1}{c}{\scriptsize{7.496}} \\  

     &  & \mc{1}{c}{\scriptsize{\textbf{(0.024)}}} & \mc{1}{c}{\scriptsize{\textbf{(0.007)}}} & \mc{1}{c}{\scriptsize{(0.998)}} & \mc{1}{c}{\scriptsize{(0.175)}} & \mc{1}{c}{\scriptsize{(0.300)}} & \mc{1}{c}{\scriptsize{\textbf{(0.020)}}} & \mc{1}{c}{\scriptsize{\textbf{(0.009)}}} & \mc{1}{c}{\scriptsize{\textbf{(0.012)}}} \\  

     & \mc{1}{c}{\scriptsize{7}} & \mc{1}{c}{\scriptsize{4.390}} & \mc{1}{c}{\scriptsize{7.015}} & \mc{1}{c}{\scriptsize{5.323}} & \mc{1}{c}{\scriptsize{9.798}} & \mc{1}{c}{\scriptsize{4.834}} & \mc{1}{c}{\scriptsize{4.156}} & \mc{1}{c}{\scriptsize{6.457}} & \mc{1}{c}{\scriptsize{6.525}} \\  

     &  & \mc{1}{c}{\scriptsize{\textbf{(0.073)}}} & \mc{1}{c}{\scriptsize{\textbf{(0.008)}}} & \mc{1}{c}{\scriptsize{\textbf{(0.002)}}} & \mc{1}{c}{\scriptsize{\textbf{(0.033)}}} & \mc{1}{c}{\scriptsize{(0.219)}} & \mc{1}{c}{\scriptsize{(0.103)}} & \mc{1}{c}{\scriptsize{\textbf{(0.012)}}} & \mc{1}{c}{\scriptsize{\textbf{(0.021)}}} \\  

     & \mc{1}{c}{\scriptsize{8}} & \mc{1}{c}{\scriptsize{4.160}} & \mc{1}{c}{\scriptsize{5.055}} & \mc{1}{c}{\scriptsize{-2.514}} & \mc{1}{c}{\scriptsize{2.223}} & \mc{1}{c}{\scriptsize{-0.470}} & \mc{1}{c}{\scriptsize{4.754}} & \mc{1}{c}{\scriptsize{4.986}} & \mc{1}{c}{\scriptsize{5.012}} \\  

     &  & \mc{1}{c}{\scriptsize{\textbf{(0.094)}}} & \mc{1}{c}{\scriptsize{\textbf{(0.053)}}} & \mc{1}{c}{\scriptsize{\textbf{(0.002)}}} & \mc{1}{c}{\scriptsize{(0.369)}} & \mc{1}{c}{\scriptsize{(0.471)}} & \mc{1}{c}{\scriptsize{\textbf{(0.043)}}} & \mc{1}{c}{\scriptsize{\textbf{(0.047)}}} & \mc{1}{c}{\scriptsize{\textbf{(0.075)}}} \\  

     & \mc{1}{c}{\scriptsize{12}} & \mc{1}{c}{\scriptsize{0.686}} & \mc{1}{c}{\scriptsize{-1.041}} & \mc{1}{c}{\scriptsize{-0.343}} & \mc{1}{c}{\scriptsize{0.210}} & \mc{1}{c}{\scriptsize{-0.945}} & \mc{1}{c}{\scriptsize{0.943}} & \mc{1}{c}{\scriptsize{-1.477}} & \mc{1}{c}{\scriptsize{-0.802}} \\  

     &  & \mc{1}{c}{\scriptsize{(0.403)}} & \mc{1}{c}{\scriptsize{(0.344)}} & \mc{1}{c}{\scriptsize{(0.999)}} & \mc{1}{c}{\scriptsize{\textbf{(0.002)}}} & \mc{1}{c}{\scriptsize{(0.430)}} & \mc{1}{c}{\scriptsize{(0.359)}} & \mc{1}{c}{\scriptsize{(0.278)}} & \mc{1}{c}{\scriptsize{(0.395)}} \\  

     & \mc{1}{c}{\scriptsize{15}} & \mc{1}{c}{\scriptsize{4.447}} & \mc{1}{c}{\scriptsize{3.635}} & \mc{1}{c}{\scriptsize{-2.057}} & \mc{1}{c}{\scriptsize{-1.598}} & \mc{1}{c}{\scriptsize{-2.949}} & \mc{1}{c}{\scriptsize{6.202}} & \mc{1}{c}{\scriptsize{4.701}} & \mc{1}{c}{\scriptsize{4.512}} \\  

     &  & \mc{1}{c}{\scriptsize{\textbf{(0.066)}}} & \mc{1}{c}{\scriptsize{(0.105)}} & \mc{1}{c}{\scriptsize{\textbf{(0.003)}}} & \mc{1}{c}{\scriptsize{(0.994)}} & \mc{1}{c}{\scriptsize{(0.224)}} & \mc{1}{c}{\scriptsize{\textbf{(0.022)}}} & \mc{1}{c}{\scriptsize{\textbf{(0.081)}}} & \mc{1}{c}{\scriptsize{(0.101)}} \\  

     & \mc{1}{c}{\scriptsize{21}} & \mc{1}{c}{\scriptsize{1.550}} & \mc{1}{c}{\scriptsize{-0.561}} & \mc{1}{c}{\scriptsize{0.471}} & \mc{1}{c}{\scriptsize{-0.373}} & \mc{1}{c}{\scriptsize{-1.522}} & \mc{1}{c}{\scriptsize{2.307}} & \mc{1}{c}{\scriptsize{-0.512}} & \mc{1}{c}{\scriptsize{-0.479}} \\  

     &  & \mc{1}{c}{\scriptsize{(0.269)}} & \mc{1}{c}{\scriptsize{(0.394)}} & \mc{1}{c}{\scriptsize{(0.995)}} & \mc{1}{c}{\scriptsize{\textbf{(0.001)}}} & \mc{1}{c}{\scriptsize{(0.254)}} & \mc{1}{c}{\scriptsize{(0.210)}} & \mc{1}{c}{\scriptsize{(0.415)}} & \mc{1}{c}{\scriptsize{(0.425)}} \\  

    \mc{1}{l}{\scriptsize{IQ Factor}} & \mc{1}{c}{\scriptsize{2 to 5}} & \mc{1}{c}{\scriptsize{0.865}} & \mc{1}{c}{\scriptsize{0.875}} & \mc{1}{c}{\scriptsize{0.735}} & \mc{1}{c}{\scriptsize{0.823}} & \mc{1}{c}{\scriptsize{0.793}} & \mc{1}{c}{\scriptsize{0.903}} & \mc{1}{c}{\scriptsize{0.886}} & \mc{1}{c}{\scriptsize{0.913}} \\  

     &  & \mc{1}{c}{\scriptsize{\textbf{(0.000)}}} & \mc{1}{c}{\scriptsize{\textbf{(0.000)}}} & \mc{1}{c}{\scriptsize{(0.999)}} & \mc{1}{c}{\scriptsize{\textbf{(0.001)}}} & \mc{1}{c}{\scriptsize{\textbf{(0.016)}}} & \mc{1}{c}{\scriptsize{\textbf{(0.000)}}} & \mc{1}{c}{\scriptsize{\textbf{(0.000)}}} & \mc{1}{c}{\scriptsize{\textbf{(0.000)}}} \\  

     & \mc{1}{c}{\scriptsize{6 to 12}} & \mc{1}{c}{\scriptsize{0.329}} & \mc{1}{c}{\scriptsize{0.333}} & \mc{1}{c}{\scriptsize{0.349}} & \mc{1}{c}{\scriptsize{0.584}} & \mc{1}{c}{\scriptsize{0.348}} & \mc{1}{c}{\scriptsize{0.323}} & \mc{1}{c}{\scriptsize{0.250}} & \mc{1}{c}{\scriptsize{0.291}} \\  

     &  & \mc{1}{c}{\scriptsize{(0.120)}} & \mc{1}{c}{\scriptsize{(0.128)}} & \mc{1}{c}{\scriptsize{(0.998)}} & \mc{1}{c}{\scriptsize{\textbf{(0.001)}}} & \mc{1}{c}{\scriptsize{(0.249)}} & \mc{1}{c}{\scriptsize{(0.149)}} & \mc{1}{c}{\scriptsize{(0.181)}} & \mc{1}{c}{\scriptsize{(0.174)}} \\  

     & \mc{1}{c}{\scriptsize{15 to 21}} & \mc{1}{c}{\scriptsize{-0.276}} & \mc{1}{c}{\scriptsize{-0.126}} & \mc{1}{c}{\scriptsize{0.063}} & \mc{1}{c}{\scriptsize{0.089}} & \mc{1}{c}{\scriptsize{0.210}} & \mc{1}{c}{\scriptsize{-0.392}} & \mc{1}{c}{\scriptsize{-0.175}} & \mc{1}{c}{\scriptsize{-0.168}} \\  

     &  & \mc{1}{c}{\scriptsize{(0.141)}} & \mc{1}{c}{\scriptsize{(0.300)}} & \mc{1}{c}{\scriptsize{\textbf{(0.003)}}} & \mc{1}{c}{\scriptsize{\textbf{(0.001)}}} & \mc{1}{c}{\scriptsize{(0.227)}} & \mc{1}{c}{\scriptsize{\textbf{(0.082)}}} & \mc{1}{c}{\scriptsize{(0.278)}} & \mc{1}{c}{\scriptsize{(0.280)}} \\  

  \bottomrule
  \end{tabular}
\end{center}

\begin{center}
	\begin{table}[H]
\captionsetup{singlelinecheck=false,justification=centering}
\caption{ABC/CARE Average Treatment Effects, Males \\ Achievement Scores \label{tab:ate_male_apx1}}

  \begin{threeparttable}
  \begin{tabular}{cccccccccc}
  \hline\hline

     &  & \scriptsize{(1)} & \scriptsize{(2)} & \scriptsize{(3)} & \scriptsize{(4)} & \scriptsize{(5)} & \scriptsize{(6)} & \scriptsize{(7)} & \scriptsize{(8)} \\  

     &  &  &  & \mc{3}{c}{\scriptsize{$P=0$}} & \mc{3}{c}{\scriptsize{$P=1$}} \\ 
    \cmidrule(lr){5-7} \cmidrule(lr){8-10} 

    \scriptsize{Variable} & \scriptsize{Age} & \scriptsize{ITT} & \scriptsize{ITT$|X,W$} & \scriptsize{ITT} & \scriptsize{ITT$|X,W$} & \scriptsize{KE$|X,W$} & \scriptsize{ITT} & \scriptsize{ITT$|X,W$} & \scriptsize{KE$|X,W$} \\ 
    \hline  

    \mc{1}{l}{\scriptsize{Std. Achv.  Test}} & \mc{1}{c}{\scriptsize{5.5}} & \mc{1}{c}{\scriptsize{6.126}} & \mc{1}{c}{\scriptsize{6.727}} & \mc{1}{c}{\scriptsize{10.521}} & \mc{1}{c}{\scriptsize{10.744}} & \mc{1}{c}{\scriptsize{10.096}} & \mc{1}{c}{\scriptsize{4.661}} & \mc{1}{c}{\scriptsize{5.208}} & \mc{1}{c}{\scriptsize{4.942}} \\  

     &  & \mc{1}{c}{\scriptsize{\textbf{(0.000)}}} & \mc{1}{c}{\scriptsize{\textbf{(0.000)}}} & \mc{1}{c}{\scriptsize{\textbf{(0.000)}}} & \mc{1}{c}{\scriptsize{\textbf{(0.000)}}} & \mc{1}{c}{\scriptsize{\textbf{(0.000)}}} & \mc{1}{c}{\scriptsize{\textbf{(0.020)}}} & \mc{1}{c}{\scriptsize{\textbf{(0.020)}}} & \mc{1}{c}{\scriptsize{\textbf{(0.039)}}} \\  

     & \mc{1}{c}{\scriptsize{6}} & \mc{1}{c}{\scriptsize{4.281}} & \mc{1}{c}{\scriptsize{4.784}} & \mc{1}{c}{\scriptsize{5.042}} & \mc{1}{c}{\scriptsize{5.201}} & \mc{1}{c}{\scriptsize{5.653}} & \mc{1}{c}{\scriptsize{4.020}} & \mc{1}{c}{\scriptsize{4.576}} & \mc{1}{c}{\scriptsize{4.325}} \\  

     &  & \mc{1}{c}{\scriptsize{\textbf{(0.000)}}} & \mc{1}{c}{\scriptsize{\textbf{(0.000)}}} & \mc{1}{c}{\scriptsize{\textbf{(0.059)}}} & \mc{1}{c}{\scriptsize{\textbf{(0.059)}}} & \mc{1}{c}{\scriptsize{\textbf{(0.078)}}} & \mc{1}{c}{\scriptsize{\textbf{(0.039)}}} & \mc{1}{c}{\scriptsize{\textbf{(0.000)}}} & \mc{1}{c}{\scriptsize{\textbf{(0.020)}}} \\  

     & \mc{1}{c}{\scriptsize{6.5}} & \mc{1}{c}{\scriptsize{1.708}} & \mc{1}{c}{\scriptsize{1.708}} & \mc{1}{c}{\scriptsize{-5.292}} & \mc{1}{c}{\scriptsize{-5.292}} & \mc{1}{c}{\scriptsize{-1.558}} & \mc{1}{c}{\scriptsize{4.508}} & \mc{1}{c}{\scriptsize{4.508}} & \mc{1}{c}{\scriptsize{4.890}} \\  

     &  & \mc{1}{c}{\scriptsize{(0.196)}} & \mc{1}{c}{\scriptsize{(0.196)}} & \mc{1}{c}{\scriptsize{(0.824)}} & \mc{1}{c}{\scriptsize{(0.824)}} & \mc{1}{c}{\scriptsize{(0.627)}} & \mc{1}{c}{\scriptsize{\textbf{(0.000)}}} & \mc{1}{c}{\scriptsize{\textbf{(0.000)}}} & \mc{1}{c}{\scriptsize{\textbf{(0.000)}}} \\  

     & \mc{1}{c}{\scriptsize{7}} & \mc{1}{c}{\scriptsize{0.622}} & \mc{1}{c}{\scriptsize{0.622}} & \mc{1}{c}{\scriptsize{-5.434}} & \mc{1}{c}{\scriptsize{-5.434}} & \mc{1}{c}{\scriptsize{-2.050}} & \mc{1}{c}{\scriptsize{3.044}} & \mc{1}{c}{\scriptsize{3.044}} & \mc{1}{c}{\scriptsize{2.582}} \\  

     &  & \mc{1}{c}{\scriptsize{(0.392)}} & \mc{1}{c}{\scriptsize{(0.392)}} & \mc{1}{c}{\scriptsize{(0.765)}} & \mc{1}{c}{\scriptsize{(0.765)}} & \mc{1}{c}{\scriptsize{(0.627)}} & \mc{1}{c}{\scriptsize{\textbf{(0.078)}}} & \mc{1}{c}{\scriptsize{\textbf{(0.078)}}} & \mc{1}{c}{\scriptsize{\textbf{(0.098)}}} \\  

     & \mc{1}{c}{\scriptsize{7.5}} & \mc{1}{c}{\scriptsize{-0.302}} & \mc{1}{c}{\scriptsize{2.422}} & \mc{1}{c}{\scriptsize{-0.753}} & \mc{1}{c}{\scriptsize{1.215}} & \mc{1}{c}{\scriptsize{-1.901}} & \mc{1}{c}{\scriptsize{-0.147}} & \mc{1}{c}{\scriptsize{2.956}} & \mc{1}{c}{\scriptsize{-0.941}} \\  

     &  & \mc{1}{c}{\scriptsize{(0.549)}} & \mc{1}{c}{\scriptsize{\textbf{(0.059)}}} & \mc{1}{c}{\scriptsize{(0.627)}} & \mc{1}{c}{\scriptsize{(0.314)}} & \mc{1}{c}{\scriptsize{(0.686)}} & \mc{1}{c}{\scriptsize{(0.588)}} & \mc{1}{c}{\scriptsize{\textbf{(0.059)}}} & \mc{1}{c}{\scriptsize{(0.686)}} \\  

     & \mc{1}{c}{\scriptsize{8}} & \mc{1}{c}{\scriptsize{2.948}} & \mc{1}{c}{\scriptsize{4.501}} & \mc{1}{c}{\scriptsize{0.672}} & \mc{1}{c}{\scriptsize{1.538}} & \mc{1}{c}{\scriptsize{1.135}} & \mc{1}{c}{\scriptsize{3.728}} & \mc{1}{c}{\scriptsize{5.981}} & \mc{1}{c}{\scriptsize{3.971}} \\  

     &  & \mc{1}{c}{\scriptsize{(0.118)}} & \mc{1}{c}{\scriptsize{\textbf{(0.039)}}} & \mc{1}{c}{\scriptsize{(0.490)}} & \mc{1}{c}{\scriptsize{(0.333)}} & \mc{1}{c}{\scriptsize{(0.431)}} & \mc{1}{c}{\scriptsize{\textbf{(0.059)}}} & \mc{1}{c}{\scriptsize{\textbf{(0.000)}}} & \mc{1}{c}{\scriptsize{\textbf{(0.059)}}} \\  

     & \mc{1}{c}{\scriptsize{8.5}} & \mc{1}{c}{\scriptsize{4.521}} & \mc{1}{c}{\scriptsize{5.831}} & \mc{1}{c}{\scriptsize{5.190}} & \mc{1}{c}{\scriptsize{5.207}} & \mc{1}{c}{\scriptsize{5.237}} & \mc{1}{c}{\scriptsize{4.265}} & \mc{1}{c}{\scriptsize{6.313}} & \mc{1}{c}{\scriptsize{3.648}} \\  

     &  & \mc{1}{c}{\scriptsize{\textbf{(0.020)}}} & \mc{1}{c}{\scriptsize{\textbf{(0.000)}}} & \mc{1}{c}{\scriptsize{\textbf{(0.078)}}} & \mc{1}{c}{\scriptsize{\textbf{(0.078)}}} & \mc{1}{c}{\scriptsize{(0.196)}} & \mc{1}{c}{\scriptsize{\textbf{(0.020)}}} & \mc{1}{c}{\scriptsize{\textbf{(0.000)}}} & \mc{1}{c}{\scriptsize{\textbf{(0.059)}}} \\  

     & \mc{1}{c}{\scriptsize{15}} & \mc{1}{c}{\scriptsize{2.231}} & \mc{1}{c}{\scriptsize{2.231}} & \mc{1}{c}{\scriptsize{-3.479}} & \mc{1}{c}{\scriptsize{-3.479}} & \mc{1}{c}{\scriptsize{-2.153}} & \mc{1}{c}{\scriptsize{4.729}} & \mc{1}{c}{\scriptsize{4.729}} & \mc{1}{c}{\scriptsize{2.391}} \\  

     &  & \mc{1}{c}{\scriptsize{(0.176)}} & \mc{1}{c}{\scriptsize{(0.176)}} & \mc{1}{c}{\scriptsize{(0.765)}} & \mc{1}{c}{\scriptsize{(0.765)}} & \mc{1}{c}{\scriptsize{(0.686)}} & \mc{1}{c}{\scriptsize{\textbf{(0.020)}}} & \mc{1}{c}{\scriptsize{\textbf{(0.020)}}} & \mc{1}{c}{\scriptsize{(0.137)}} \\  

     & \mc{1}{c}{\scriptsize{21}} & \mc{1}{c}{\scriptsize{1.181}} & \mc{1}{c}{\scriptsize{1.181}} & \mc{1}{c}{\scriptsize{-4.232}} & \mc{1}{c}{\scriptsize{-4.232}} & \mc{1}{c}{\scriptsize{-3.611}} & \mc{1}{c}{\scriptsize{3.549}} & \mc{1}{c}{\scriptsize{3.549}} & \mc{1}{c}{\scriptsize{0.012}} \\  

     &  & \mc{1}{c}{\scriptsize{(0.294)}} & \mc{1}{c}{\scriptsize{(0.294)}} & \mc{1}{c}{\scriptsize{(0.706)}} & \mc{1}{c}{\scriptsize{(0.706)}} & \mc{1}{c}{\scriptsize{(0.706)}} & \mc{1}{c}{\scriptsize{\textbf{(0.059)}}} & \mc{1}{c}{\scriptsize{\textbf{(0.059)}}} & \mc{1}{c}{\scriptsize{(0.490)}} \\  

    \mc{1}{l}{\scriptsize{Achievement Factor}} & \mc{1}{c}{\scriptsize{5.5 to 12}} & \mc{1}{c}{\scriptsize{0.340}} & \mc{1}{c}{\scriptsize{0.461}} & \mc{1}{c}{\scriptsize{0.341}} & \mc{1}{c}{\scriptsize{0.422}} & \mc{1}{c}{\scriptsize{0.267}} & \mc{1}{c}{\scriptsize{0.340}} & \mc{1}{c}{\scriptsize{0.477}} & \mc{1}{c}{\scriptsize{0.286}} \\  

     &  & \mc{1}{c}{\scriptsize{\textbf{(0.039)}}} & \mc{1}{c}{\scriptsize{\textbf{(0.000)}}} & \mc{1}{c}{\scriptsize{(0.235)}} & \mc{1}{c}{\scriptsize{(0.118)}} & \mc{1}{c}{\scriptsize{(0.373)}} & \mc{1}{c}{\scriptsize{\textbf{(0.020)}}} & \mc{1}{c}{\scriptsize{\textbf{(0.020)}}} & \mc{1}{c}{\scriptsize{\textbf{(0.059)}}} \\ 
    \hline  

    \\[0.1cm]
    \mc{2}{l}{\scriptsize{\% of Pos. TE ($H_0$: $\le$ 25\% $|$ 10\% Significance)}} & \mc{1}{c}{\scriptsize{40}} & \mc{1}{c}{\scriptsize{60}} & \mc{1}{c}{\scriptsize{30}} & \mc{1}{c}{\scriptsize{30}} & \mc{1}{c}{\scriptsize{20}} & \mc{1}{c}{\scriptsize{90}} & \mc{1}{c}{\scriptsize{100}} & \mc{1}{c}{\scriptsize{70}} \\  

     &  & \mc{1}{c}{\scriptsize{(0.235)}} & \mc{1}{c}{\scriptsize{\textbf{(0.098)}}} & \mc{1}{c}{\scriptsize{(0.373)}} & \mc{1}{c}{\scriptsize{(0.431)}} & \mc{1}{c}{\scriptsize{(0.529)}} & \mc{1}{c}{\scriptsize{\textbf{(0.000)}}} & \mc{1}{c}{\scriptsize{\textbf{(0.000)}}} & \mc{1}{c}{\scriptsize{\textbf{(0.020)}}} \\  

    \mc{2}{l}{\scriptsize{\% of Pos. TE ($H_0$: $\le$ 50\% $|$ 10\% Significance)}} & \mc{1}{c}{\scriptsize{40}} & \mc{1}{c}{\scriptsize{60}} & \mc{1}{c}{\scriptsize{30}} & \mc{1}{c}{\scriptsize{30}} & \mc{1}{c}{\scriptsize{20}} & \mc{1}{c}{\scriptsize{90}} & \mc{1}{c}{\scriptsize{100}} & \mc{1}{c}{\scriptsize{70}} \\  

     &  & \mc{1}{c}{\scriptsize{(0.706)}} & \mc{1}{c}{\scriptsize{(0.373)}} & \mc{1}{c}{\scriptsize{(0.627)}} & \mc{1}{c}{\scriptsize{(0.608)}} & \mc{1}{c}{\scriptsize{(0.882)}} & \mc{1}{c}{\scriptsize{\textbf{(0.039)}}} & \mc{1}{c}{\scriptsize{\textbf{(0.000)}}} & \mc{1}{c}{\scriptsize{(0.294)}} \\  

    \mc{2}{l}{\scriptsize{\% of Pos. TE ($H_0$: $\le$ 75\% $|$ 10\% Significance)}} & \mc{1}{c}{\scriptsize{40}} & \mc{1}{c}{\scriptsize{60}} & \mc{1}{c}{\scriptsize{30}} & \mc{1}{c}{\scriptsize{30}} & \mc{1}{c}{\scriptsize{20}} & \mc{1}{c}{\scriptsize{90}} & \mc{1}{c}{\scriptsize{100}} & \mc{1}{c}{\scriptsize{70}} \\  

     &  & \mc{1}{c}{\scriptsize{(0.843)}} & \mc{1}{c}{\scriptsize{(0.745)}} & \mc{1}{c}{\scriptsize{(1.000)}} & \mc{1}{c}{\scriptsize{(1.000)}} & \mc{1}{c}{\scriptsize{(1.000)}} & \mc{1}{c}{\scriptsize{(0.275)}} & \mc{1}{c}{\scriptsize{(0.157)}} & \mc{1}{c}{\scriptsize{(0.529)}} \\  

  \hline\hline
  \end{tabular}
    \begin{tablenotes}
    \scriptsize
    \item 
Note: This table displays various estimates of the treatment effect of ABC/CARE's center-based care.
Column (1) displays the ITT, without accounting for any controls.
Column (2) displays the ITT conditioning on vector of controls, $X$, consisting of APGAR scores 1 
minute after birth, an indicator for the subject being born prematurely, and an indicator for the 
father being home at baseline. We also apply IPW weights, $W$, to account for attrition.
Columns (3)--(4) are analogous to columns (1)--(2), but we restrict the control sample to subjects
who did not enroll in any alternative care.
Column (5) displys the matching estimate, where we use the Mahalanobis metric and Epanechnikov kernel
to match on controls $X$ listed above, and restrict the control sample to subjects who did not enroll
in any alternative care. Additionally, we apply IPW weights, $W$.
Columns (6)--(8) are analogous to Columns (3)--(5), except we restrict the control sample to subejcts
who did enroll in alternative care. 
The final three pairs of rows display the proportion of treatment effects in the table that are 
socially positive. The first row in each pair displays the percentage of treatment effects, and the
second row presents the inference.

Numbers in parentheses represent the $p$-value from a single hypothesis test, and are obtained from 
the empirical bootstrap distribution generated by 200 resamples of the original data. 
Bold $p$-values indicate significance at the 10\% level.
Blank point estimates indicate that we are unable to obtain estimates due to a lack of support in the data. 

    \end{tablenotes}
  \end{threeparttable}

\end{table}
\end{center}

\begin{center}
	\begin{table}[H]
\captionsetup{singlelinecheck=false,justification=centering}
\caption{ABC/CARE Average Treatment Effects, Males \\ HOME Scores \label{tab:ate_male_apx2}}

  \begin{threeparttable}
  \begin{tabular}{cccccccccc}
  \hline\hline

     &  & \scriptsize{(1)} & \scriptsize{(2)} & \scriptsize{(3)} & \scriptsize{(4)} & \scriptsize{(5)} & \scriptsize{(6)} & \scriptsize{(7)} & \scriptsize{(8)} \\  

     &  &  &  & \mc{3}{c}{\scriptsize{$P=0$}} & \mc{3}{c}{\scriptsize{$P=1$}} \\ 
    \cmidrule(lr){5-7} \cmidrule(lr){8-10} 

    \scriptsize{Variable} & \scriptsize{Age} & \scriptsize{ITT} & \scriptsize{ITT$|X,W$} & \scriptsize{ITT} & \scriptsize{ITT$|X,W$} & \scriptsize{KE$|X,W$} & \scriptsize{ITT} & \scriptsize{ITT$|X,W$} & \scriptsize{KE$|X,W$} \\ 
    \hline  

    \mc{1}{l}{\scriptsize{HOME Score}} & \mc{1}{c}{\scriptsize{0.5}} & \mc{1}{c}{\scriptsize{0.827}} & \mc{1}{c}{\scriptsize{1.252}} & \mc{1}{c}{\scriptsize{1.667}} & \mc{1}{c}{\scriptsize{1.822}} & \mc{1}{c}{\scriptsize{0.916}} & \mc{1}{c}{\scriptsize{0.419}} & \mc{1}{c}{\scriptsize{1.057}} & \mc{1}{c}{\scriptsize{-0.073}} \\  

     &  & \mc{1}{c}{\scriptsize{(0.235)}} & \mc{1}{c}{\scriptsize{(0.157)}} & \mc{1}{c}{\scriptsize{(0.216)}} & \mc{1}{c}{\scriptsize{(0.196)}} & \mc{1}{c}{\scriptsize{(0.353)}} & \mc{1}{c}{\scriptsize{(0.373)}} & \mc{1}{c}{\scriptsize{(0.216)}} & \mc{1}{c}{\scriptsize{(0.529)}} \\  

     & \mc{1}{c}{\scriptsize{1.5}} & \mc{1}{c}{\scriptsize{0.173}} & \mc{1}{c}{\scriptsize{0.710}} & \mc{1}{c}{\scriptsize{1.589}} & \mc{1}{c}{\scriptsize{1.988}} & \mc{1}{c}{\scriptsize{0.596}} & \mc{1}{c}{\scriptsize{-0.417}} & \mc{1}{c}{\scriptsize{0.249}} & \mc{1}{c}{\scriptsize{-0.716}} \\  

     &  & \mc{1}{c}{\scriptsize{(0.392)}} & \mc{1}{c}{\scriptsize{(0.294)}} & \mc{1}{c}{\scriptsize{(0.255)}} & \mc{1}{c}{\scriptsize{(0.176)}} & \mc{1}{c}{\scriptsize{(0.392)}} & \mc{1}{c}{\scriptsize{(0.647)}} & \mc{1}{c}{\scriptsize{(0.412)}} & \mc{1}{c}{\scriptsize{(0.745)}} \\  

     & \mc{1}{c}{\scriptsize{2.5}} & \mc{1}{c}{\scriptsize{-0.031}} & \mc{1}{c}{\scriptsize{0.477}} & \mc{1}{c}{\scriptsize{1.263}} & \mc{1}{c}{\scriptsize{1.655}} & \mc{1}{c}{\scriptsize{1.593}} & \mc{1}{c}{\scriptsize{-0.585}} & \mc{1}{c}{\scriptsize{-0.057}} & \mc{1}{c}{\scriptsize{-0.206}} \\  

     &  & \mc{1}{c}{\scriptsize{(0.431)}} & \mc{1}{c}{\scriptsize{(0.373)}} & \mc{1}{c}{\scriptsize{(0.235)}} & \mc{1}{c}{\scriptsize{(0.216)}} & \mc{1}{c}{\scriptsize{(0.196)}} & \mc{1}{c}{\scriptsize{(0.667)}} & \mc{1}{c}{\scriptsize{(0.510)}} & \mc{1}{c}{\scriptsize{(0.549)}} \\  

     & \mc{1}{c}{\scriptsize{3.5}} & \mc{1}{c}{\scriptsize{1.616}} & \mc{1}{c}{\scriptsize{2.660}} & \mc{1}{c}{\scriptsize{4.576}} & \mc{1}{c}{\scriptsize{6.126}} & \mc{1}{c}{\scriptsize{3.336}} & \mc{1}{c}{\scriptsize{0.433}} & \mc{1}{c}{\scriptsize{1.471}} & \mc{1}{c}{\scriptsize{-0.245}} \\  

     &  & \mc{1}{c}{\scriptsize{(0.157)}} & \mc{1}{c}{\scriptsize{\textbf{(0.098)}}} & \mc{1}{c}{\scriptsize{\textbf{(0.078)}}} & \mc{1}{c}{\scriptsize{\textbf{(0.020)}}} & \mc{1}{c}{\scriptsize{(0.176)}} & \mc{1}{c}{\scriptsize{(0.412)}} & \mc{1}{c}{\scriptsize{(0.196)}} & \mc{1}{c}{\scriptsize{(0.569)}} \\  

     & \mc{1}{c}{\scriptsize{4.5}} & \mc{1}{c}{\scriptsize{2.143}} & \mc{1}{c}{\scriptsize{3.288}} & \mc{1}{c}{\scriptsize{4.205}} & \mc{1}{c}{\scriptsize{5.115}} & \mc{1}{c}{\scriptsize{1.826}} & \mc{1}{c}{\scriptsize{1.340}} & \mc{1}{c}{\scriptsize{2.662}} & \mc{1}{c}{\scriptsize{0.012}} \\  

     &  & \mc{1}{c}{\scriptsize{(0.118)}} & \mc{1}{c}{\scriptsize{\textbf{(0.039)}}} & \mc{1}{c}{\scriptsize{\textbf{(0.098)}}} & \mc{1}{c}{\scriptsize{\textbf{(0.020)}}} & \mc{1}{c}{\scriptsize{(0.294)}} & \mc{1}{c}{\scriptsize{(0.157)}} & \mc{1}{c}{\scriptsize{\textbf{(0.078)}}} & \mc{1}{c}{\scriptsize{(0.431)}} \\  

     & \mc{1}{c}{\scriptsize{8}} & \mc{1}{c}{\scriptsize{2.085}} & \mc{1}{c}{\scriptsize{1.907}} & \mc{1}{c}{\scriptsize{1.566}} & \mc{1}{c}{\scriptsize{1.295}} & \mc{1}{c}{\scriptsize{2.711}} & \mc{1}{c}{\scriptsize{2.243}} & \mc{1}{c}{\scriptsize{2.346}} & \mc{1}{c}{\scriptsize{3.023}} \\  

     &  & \mc{1}{c}{\scriptsize{(0.137)}} & \mc{1}{c}{\scriptsize{(0.196)}} & \mc{1}{c}{\scriptsize{(0.353)}} & \mc{1}{c}{\scriptsize{(0.353)}} & \mc{1}{c}{\scriptsize{(0.196)}} & \mc{1}{c}{\scriptsize{(0.118)}} & \mc{1}{c}{\scriptsize{(0.137)}} & \mc{1}{c}{\scriptsize{\textbf{(0.059)}}} \\  

    \mc{1}{l}{\scriptsize{HOME Factor}} & \mc{1}{c}{\scriptsize{0.5 to 8}} & \mc{1}{c}{\scriptsize{0.262}} & \mc{1}{c}{\scriptsize{0.288}} & \mc{1}{c}{\scriptsize{0.255}} & \mc{1}{c}{\scriptsize{0.281}} & \mc{1}{c}{\scriptsize{0.284}} & \mc{1}{c}{\scriptsize{0.265}} & \mc{1}{c}{\scriptsize{0.299}} & \mc{1}{c}{\scriptsize{0.217}} \\  

     &  & \mc{1}{c}{\scriptsize{\textbf{(0.098)}}} & \mc{1}{c}{\scriptsize{\textbf{(0.098)}}} & \mc{1}{c}{\scriptsize{(0.196)}} & \mc{1}{c}{\scriptsize{(0.176)}} & \mc{1}{c}{\scriptsize{(0.176)}} & \mc{1}{c}{\scriptsize{\textbf{(0.098)}}} & \mc{1}{c}{\scriptsize{(0.118)}} & \mc{1}{c}{\scriptsize{(0.137)}} \\ 
    \hline  

    \\[0.1cm]
    \mc{2}{l}{\scriptsize{\% of Pos. TE ($H_0$: $\le$ 25\% $|$ 10\% Significance)}} & \mc{1}{c}{\scriptsize{14}} & \mc{1}{c}{\scriptsize{43}} & \mc{1}{c}{\scriptsize{29}} & \mc{1}{c}{\scriptsize{29}} & \mc{1}{c}{\scriptsize{0}} & \mc{1}{c}{\scriptsize{14}} & \mc{1}{c}{\scriptsize{14}} & \mc{1}{c}{\scriptsize{14}} \\  

     &  & \mc{1}{c}{\scriptsize{(0.529)}} & \mc{1}{c}{\scriptsize{(0.314)}} & \mc{1}{c}{\scriptsize{(0.412)}} & \mc{1}{c}{\scriptsize{(0.431)}} & \mc{1}{c}{\scriptsize{(0.804)}} & \mc{1}{c}{\scriptsize{(0.627)}} & \mc{1}{c}{\scriptsize{(0.588)}} & \mc{1}{c}{\scriptsize{(0.647)}} \\  

    \mc{2}{l}{\scriptsize{\% of Pos. TE ($H_0$: $\le$ 50\% $|$ 10\% Significance)}} & \mc{1}{c}{\scriptsize{14}} & \mc{1}{c}{\scriptsize{43}} & \mc{1}{c}{\scriptsize{29}} & \mc{1}{c}{\scriptsize{29}} & \mc{1}{c}{\scriptsize{0}} & \mc{1}{c}{\scriptsize{14}} & \mc{1}{c}{\scriptsize{14}} & \mc{1}{c}{\scriptsize{14}} \\  

     &  & \mc{1}{c}{\scriptsize{(1.000)}} & \mc{1}{c}{\scriptsize{(0.549)}} & \mc{1}{c}{\scriptsize{(0.725)}} & \mc{1}{c}{\scriptsize{(0.765)}} & \mc{1}{c}{\scriptsize{(1.000)}} & \mc{1}{c}{\scriptsize{(1.000)}} & \mc{1}{c}{\scriptsize{(0.725)}} & \mc{1}{c}{\scriptsize{(1.000)}} \\  

    \mc{2}{l}{\scriptsize{\% of Pos. TE ($H_0$: $\le$ 75\% $|$ 10\% Significance)}} & \mc{1}{c}{\scriptsize{14}} & \mc{1}{c}{\scriptsize{43}} & \mc{1}{c}{\scriptsize{29}} & \mc{1}{c}{\scriptsize{29}} & \mc{1}{c}{\scriptsize{0}} & \mc{1}{c}{\scriptsize{14}} & \mc{1}{c}{\scriptsize{14}} & \mc{1}{c}{\scriptsize{14}} \\  

     &  & \mc{1}{c}{\scriptsize{(1.000)}} & \mc{1}{c}{\scriptsize{(0.784)}} & \mc{1}{c}{\scriptsize{(1.000)}} & \mc{1}{c}{\scriptsize{(0.882)}} & \mc{1}{c}{\scriptsize{(1.000)}} & \mc{1}{c}{\scriptsize{(1.000)}} & \mc{1}{c}{\scriptsize{(1.000)}} & \mc{1}{c}{\scriptsize{(1.000)}} \\  

  \hline\hline
  \end{tabular}
    \begin{tablenotes}
    \scriptsize
    \item 
Note: This table displays various estimates of the treatment effect of ABC/CARE's center-based care.
Column (1) displays the ITT, without accounting for any controls.
Column (2) displays the ITT conditioning on vector of controls, $X$, consisting of Apgar scores 1 minute and 5 minutes after birth, the HRI index, maternal IQ,
an indicator for having a grandmother residing in the same county, and an index for the number
of relatives living in the same household. We also apply IPW weights, $W$, to account for attrition.
Columns (3)--(4) are analogous to columns (1)--(2), but we restrict the control sample to subjects
who did not enroll in any alternative care.
Column (5) displys the matching estimate, where we use the Mahalanobis metric and Epanechnikov kernel
to match on controls $X$ listed above, and restrict the control sample to subjects who did not enroll
in any alternative care. Additionally, we apply IPW weights, $W$.
Columns (6)--(8) are analogous to Columns (3)--(5), except we restrict the control sample to subejcts
who did enroll in alternative care. The final three pairs of rows display the proportion of treatment effects in the table that are 
socially positive. The first row in each pair displays the percentage of treatment effects, and the
second row presents the inference. 
Numbers in parentheses represent the $p$-value from a single hypothesis test, and are obtained from 
the empirical bootstrap distribution generated by 200 resamples of the original data. 
Bold $p$-values indicate significance at the 10\% level.
Blank point estimates indicate that we are unable to obtain estimates due to a lack of support in the data. 

    \end{tablenotes}
  \end{threeparttable}

\end{table}
\end{center}

\begin{center}
	  \begin{tabular}{cccccccccc}
  \toprule

    \scriptsize{Variable} & \scriptsize{Age} & \scriptsize{(1)} & \scriptsize{(2)} & \scriptsize{(3)} & \scriptsize{(4)} & \scriptsize{(5)} & \scriptsize{(6)} & \scriptsize{(7)} & \scriptsize{(8)} \\ 
    \midrule  

    \mc{1}{l}{\scriptsize{Parental Income}} & \mc{1}{c}{\scriptsize{1.5}} & \mc{1}{c}{\scriptsize{-983}} & \mc{1}{c}{\scriptsize{-3,738}} & \mc{1}{c}{\scriptsize{-1,769}} & \mc{1}{c}{\scriptsize{-11,004}} & \mc{1}{c}{\scriptsize{-4,543}} & \mc{1}{c}{\scriptsize{-826}} & \mc{1}{c}{\scriptsize{-3,028}} & \mc{1}{c}{\scriptsize{-659}} \\  

     &  & \mc{1}{c}{\scriptsize{(0.566)}} & \mc{1}{c}{\scriptsize{(0.855)}} & \mc{1}{c}{\scriptsize{(0.632)}} & \mc{1}{c}{\scriptsize{(0.829)}} & \mc{1}{c}{\scriptsize{(0.711)}} & \mc{1}{c}{\scriptsize{(0.618)}} & \mc{1}{c}{\scriptsize{(0.803)}} & \mc{1}{c}{\scriptsize{(0.618)}} \\  

     & \mc{1}{c}{\scriptsize{2.5}} & \mc{1}{c}{\scriptsize{-888}} & \mc{1}{c}{\scriptsize{-3,403}} & \mc{1}{c}{\scriptsize{-1,161}} & \mc{1}{c}{\scriptsize{-9,387}} & \mc{1}{c}{\scriptsize{-3,599}} & \mc{1}{c}{\scriptsize{-834}} & \mc{1}{c}{\scriptsize{-2,814}} & \mc{1}{c}{\scriptsize{-597}} \\  

     &  & \mc{1}{c}{\scriptsize{(0.579)}} & \mc{1}{c}{\scriptsize{(0.842)}} & \mc{1}{c}{\scriptsize{(0.592)}} & \mc{1}{c}{\scriptsize{(0.829)}} & \mc{1}{c}{\scriptsize{(0.684)}} & \mc{1}{c}{\scriptsize{(0.618)}} & \mc{1}{c}{\scriptsize{(0.803)}} & \mc{1}{c}{\scriptsize{(0.632)}} \\  

     & \mc{1}{c}{\scriptsize{3.5}} & \mc{1}{c}{\scriptsize{141}} & \mc{1}{c}{\scriptsize{-980}} & \mc{1}{c}{\scriptsize{-17.176}} & \mc{1}{c}{\scriptsize{-7,060}} & \mc{1}{c}{\scriptsize{-2,164}} & \mc{1}{c}{\scriptsize{173}} & \mc{1}{c}{\scriptsize{-442}} & \mc{1}{c}{\scriptsize{2,046}} \\  

     &  & \mc{1}{c}{\scriptsize{(0.447)}} & \mc{1}{c}{\scriptsize{(0.684)}} & \mc{1}{c}{\scriptsize{(0.461)}} & \mc{1}{c}{\scriptsize{(0.789)}} & \mc{1}{c}{\scriptsize{(0.553)}} & \mc{1}{c}{\scriptsize{(0.461)}} & \mc{1}{c}{\scriptsize{(0.632)}} & \mc{1}{c}{\scriptsize{(0.329)}} \\  

     & \mc{1}{c}{\scriptsize{4.5}} & \mc{1}{c}{\scriptsize{-1,078}} & \mc{1}{c}{\scriptsize{767}} & \mc{1}{c}{\scriptsize{-2,548}} & \mc{1}{c}{\scriptsize{-2,763}} & \mc{1}{c}{\scriptsize{-3,927}} & \mc{1}{c}{\scriptsize{-619}} & \mc{1}{c}{\scriptsize{1,038}} & \mc{1}{c}{\scriptsize{1,959}} \\  

     &  & \mc{1}{c}{\scriptsize{(0.605)}} & \mc{1}{c}{\scriptsize{(0.408)}} & \mc{1}{c}{\scriptsize{(0.789)}} & \mc{1}{c}{\scriptsize{(0.816)}} & \mc{1}{c}{\scriptsize{(0.934)}} & \mc{1}{c}{\scriptsize{(0.539)}} & \mc{1}{c}{\scriptsize{(0.368)}} & \mc{1}{c}{\scriptsize{(0.303)}} \\  

     & \mc{1}{c}{\scriptsize{8}} & \mc{1}{c}{\scriptsize{9,304}} & \mc{1}{c}{\scriptsize{9,266}} & \mc{1}{c}{\scriptsize{5,393}} & \mc{1}{c}{\scriptsize{2,053}} & \mc{1}{c}{\scriptsize{3,658}} & \mc{1}{c}{\scriptsize{10,608}} & \mc{1}{c}{\scriptsize{10,983}} & \mc{1}{c}{\scriptsize{11,729}} \\  

     &  & \mc{1}{c}{\scriptsize{\textbf{(0.026)}}} & \mc{1}{c}{\scriptsize{\textbf{(0.039)}}} & \mc{1}{c}{\scriptsize{(0.158)}} & \mc{1}{c}{\scriptsize{(0.461)}} & \mc{1}{c}{\scriptsize{(0.250)}} & \mc{1}{c}{\scriptsize{\textbf{(0.013)}}} & \mc{1}{c}{\scriptsize{\textbf{(0.092)}}} & \mc{1}{c}{\scriptsize{\textbf{(0.013)}}} \\  

     & \mc{1}{c}{\scriptsize{12}} & \mc{1}{c}{\scriptsize{7,800}} & \mc{1}{c}{\scriptsize{7,596}} & \mc{1}{c}{\scriptsize{9,178}} & \mc{1}{c}{\scriptsize{10,017}} & \mc{1}{c}{\scriptsize{6,917}} & \mc{1}{c}{\scriptsize{7,341}} & \mc{1}{c}{\scriptsize{6,844}} & \mc{1}{c}{\scriptsize{7,868}} \\  

     &  & \mc{1}{c}{\scriptsize{\textbf{(0.000)}}} & \mc{1}{c}{\scriptsize{\textbf{(0.053)}}} & \mc{1}{c}{\scriptsize{\textbf{(0.092)}}} & \mc{1}{c}{\scriptsize{(0.145)}} & \mc{1}{c}{\scriptsize{(0.171)}} & \mc{1}{c}{\scriptsize{\textbf{(0.026)}}} & \mc{1}{c}{\scriptsize{\textbf{(0.053)}}} & \mc{1}{c}{\scriptsize{\textbf{(0.013)}}} \\  

     & \mc{1}{c}{\scriptsize{15}} & \mc{1}{c}{\scriptsize{5,448}} & \mc{1}{c}{\scriptsize{4,367}} & \mc{1}{c}{\scriptsize{4,299}} & \mc{1}{c}{\scriptsize{4,809}} & \mc{1}{c}{\scriptsize{3,937}} & \mc{1}{c}{\scriptsize{4,880}} & \mc{1}{c}{\scriptsize{3,502}} & \mc{1}{c}{\scriptsize{2,901}} \\  

     &  & \mc{1}{c}{\scriptsize{\textbf{(0.039)}}} & \mc{1}{c}{\scriptsize{(0.145)}} & \mc{1}{c}{\scriptsize{(0.263)}} & \mc{1}{c}{\scriptsize{(0.303)}} & \mc{1}{c}{\scriptsize{(0.250)}} & \mc{1}{c}{\scriptsize{(0.145)}} & \mc{1}{c}{\scriptsize{(0.184)}} & \mc{1}{c}{\scriptsize{(0.224)}} \\  

    \mc{1}{l}{\scriptsize{Parental Income Factor}} & \mc{1}{c}{\scriptsize{1.5 to 15}} & \mc{1}{c}{\scriptsize{-0.085}} & \mc{1}{c}{\scriptsize{-0.311}} & \mc{1}{c}{\scriptsize{-0.794}} & \mc{1}{c}{\scriptsize{-0.450}} & \mc{1}{c}{\scriptsize{-0.823}} & \mc{1}{c}{\scriptsize{0.003}} & \mc{1}{c}{\scriptsize{-0.244}} & \mc{1}{c}{\scriptsize{-0.014}} \\  

     &  & \mc{1}{c}{\scriptsize{(0.605)}} & \mc{1}{c}{\scriptsize{(0.737)}} & \mc{1}{c}{\scriptsize{(0.618)}} & \mc{1}{c}{\scriptsize{(0.474)}} & \mc{1}{c}{\scriptsize{(0.618)}} & \mc{1}{c}{\scriptsize{(0.513)}} & \mc{1}{c}{\scriptsize{(0.645)}} & \mc{1}{c}{\scriptsize{(0.513)}} \\ 
    \midrule  

    \mc{2}{l}{\scriptsize{\% of Pos. TE ($H_0$: $\le$ 50\%)}} & \mc{1}{c}{\scriptsize{50}} & \mc{1}{c}{\scriptsize{50}} & \mc{1}{c}{\scriptsize{38}} & \mc{1}{c}{\scriptsize{38}} & \mc{1}{c}{\scriptsize{38}} & \mc{1}{c}{\scriptsize{62}} & \mc{1}{c}{\scriptsize{50}} & \mc{1}{c}{\scriptsize{62}} \\  

     &  & \mc{1}{c}{\scriptsize{(0.421)}} & \mc{1}{c}{\scriptsize{(0.408)}} & \mc{1}{c}{\scriptsize{(0.474)}} & \mc{1}{c}{\scriptsize{(0.855)}} & \mc{1}{c}{\scriptsize{(0.697)}} & \mc{1}{c}{\scriptsize{(0.329)}} & \mc{1}{c}{\scriptsize{(0.500)}} & \mc{1}{c}{\scriptsize{(0.395)}} \\  

    \mc{2}{l}{\scriptsize{\% of Pos. TE ($H_0$: $\le$ 10\% $|$ 10\% Significance)}} & \mc{1}{c}{\scriptsize{38}} & \mc{1}{c}{\scriptsize{25}} & \mc{1}{c}{\scriptsize{12}} & \mc{1}{c}{\scriptsize{12}} & \mc{1}{c}{\scriptsize{0}} & \mc{1}{c}{\scriptsize{25}} & \mc{1}{c}{\scriptsize{25}} & \mc{1}{c}{\scriptsize{25}} \\  

     &  & \mc{1}{c}{\scriptsize{\textbf{(0.053)}}} & \mc{1}{c}{\scriptsize{(0.105)}} & \mc{1}{c}{\scriptsize{(0.395)}} & \mc{1}{c}{\scriptsize{(0.434)}} & \mc{1}{c}{\scriptsize{(0.632)}} & \mc{1}{c}{\scriptsize{\textbf{(0.079)}}} & \mc{1}{c}{\scriptsize{\textbf{(0.053)}}} & \mc{1}{c}{\scriptsize{(0.118)}} \\  

  \bottomrule
  \end{tabular}
\end{center}

\begin{center}
	\begin{table}[H]
\captionsetup{singlelinecheck=false,justification=centering}
\caption{CARE Average Treatment Effects, Males \\ Mother's Employment \label{tab:ate_male_apx4}}

  \begin{threeparttable}
  \begin{tabular}{cccccccccc}
  \hline\hline

     &  & \scriptsize{(1)} & \scriptsize{(2)} & \scriptsize{(3)} & \scriptsize{(4)} & \scriptsize{(5)} & \scriptsize{(6)} & \scriptsize{(7)} & \scriptsize{(8)} \\  

     &  &  &  & \mc{3}{c}{\scriptsize{$P=0$}} & \mc{3}{c}{\scriptsize{$P=1$}} \\ 
    \cmidrule(lr){5-7} \cmidrule(lr){8-10} 

    \scriptsize{Variable} & \scriptsize{Age} & \scriptsize{ITT} & \scriptsize{ITT$|X,W$} & \scriptsize{ITT} & \scriptsize{ITT$|X,W$} & \scriptsize{KE$|X,W$} & \scriptsize{ITT} & \scriptsize{ITT$|X,W$} & \scriptsize{KE$|X,W$} \\ 
    \hline  

    \mc{1}{l}{\scriptsize{Mother Works}} & \mc{1}{c}{\scriptsize{2}} & \mc{1}{c}{\scriptsize{-0.400}} & \mc{1}{c}{\scriptsize{-0.410}} & \mc{1}{c}{\scriptsize{-0.400}} & \mc{1}{c}{\scriptsize{-0.635}} & \mc{1}{c}{\scriptsize{-0.402}} & \mc{1}{c}{\scriptsize{-0.400}} & \mc{1}{c}{\scriptsize{-0.399}} & \mc{1}{c}{\scriptsize{-0.399}} \\  

     &  & \mc{1}{c}{\scriptsize{(1.000)}} & \mc{1}{c}{\scriptsize{(1.000)}} & \mc{1}{c}{\scriptsize{(0.922)}} & \mc{1}{c}{\scriptsize{(0.784)}} & \mc{1}{c}{\scriptsize{(0.922)}} & \mc{1}{c}{\scriptsize{(1.000)}} & \mc{1}{c}{\scriptsize{(1.000)}} & \mc{1}{c}{\scriptsize{(1.000)}} \\  

     & \mc{1}{c}{\scriptsize{3}} & \mc{1}{c}{\scriptsize{-0.169}} & \mc{1}{c}{\scriptsize{-0.187}} & \mc{1}{c}{\scriptsize{-0.400}} & \mc{1}{c}{\scriptsize{-0.667}} & \mc{1}{c}{\scriptsize{-0.401}} & \mc{1}{c}{\scriptsize{-0.127}} & \mc{1}{c}{\scriptsize{-0.111}} & \mc{1}{c}{\scriptsize{-0.059}} \\  

     &  & \mc{1}{c}{\scriptsize{(0.804)}} & \mc{1}{c}{\scriptsize{(0.765)}} & \mc{1}{c}{\scriptsize{(0.922)}} & \mc{1}{c}{\scriptsize{(0.863)}} & \mc{1}{c}{\scriptsize{(0.922)}} & \mc{1}{c}{\scriptsize{(0.745)}} & \mc{1}{c}{\scriptsize{(0.706)}} & \mc{1}{c}{\scriptsize{(0.529)}} \\  

     & \mc{1}{c}{\scriptsize{4}} & \mc{1}{c}{\scriptsize{-0.280}} & \mc{1}{c}{\scriptsize{-0.329}} & \mc{1}{c}{\scriptsize{-0.357}} & \mc{1}{c}{\scriptsize{-0.551}} & \mc{1}{c}{\scriptsize{-0.368}} & \mc{1}{c}{\scriptsize{-0.266}} & \mc{1}{c}{\scriptsize{-0.291}} & \mc{1}{c}{\scriptsize{-0.256}} \\  

     &  & \mc{1}{c}{\scriptsize{(1.000)}} & \mc{1}{c}{\scriptsize{(0.902)}} & \mc{1}{c}{\scriptsize{(0.922)}} & \mc{1}{c}{\scriptsize{(0.647)}} & \mc{1}{c}{\scriptsize{(0.902)}} & \mc{1}{c}{\scriptsize{(0.922)}} & \mc{1}{c}{\scriptsize{(0.882)}} & \mc{1}{c}{\scriptsize{(0.902)}} \\  

     & \mc{1}{c}{\scriptsize{5}} & \mc{1}{c}{\scriptsize{-0.147}} & \mc{1}{c}{\scriptsize{-0.211}} & \mc{1}{c}{\scriptsize{-0.231}} & \mc{1}{c}{\scriptsize{-0.646}} & \mc{1}{c}{\scriptsize{-0.235}} & \mc{1}{c}{\scriptsize{-0.140}} & \mc{1}{c}{\scriptsize{-0.194}} & \mc{1}{c}{\scriptsize{-0.123}} \\  

     &  & \mc{1}{c}{\scriptsize{(0.824)}} & \mc{1}{c}{\scriptsize{(0.824)}} & \mc{1}{c}{\scriptsize{(0.608)}} & \mc{1}{c}{\scriptsize{(0.510)}} & \mc{1}{c}{\scriptsize{(0.588)}} & \mc{1}{c}{\scriptsize{(0.843)}} & \mc{1}{c}{\scriptsize{(0.824)}} & \mc{1}{c}{\scriptsize{(0.784)}} \\ 
    \hline  

    \\[0.1cm]
    \mc{2}{l}{\scriptsize{\% of Pos. TE ($H_0$: $\le$ 25\% $|$ 10\% Significance)}} & \mc{1}{c}{\scriptsize{0}} & \mc{1}{c}{\scriptsize{0}} & \mc{1}{c}{\scriptsize{0}} & \mc{1}{c}{\scriptsize{0}} & \mc{1}{c}{\scriptsize{0}} & \mc{1}{c}{\scriptsize{0}} & \mc{1}{c}{\scriptsize{0}} & \mc{1}{c}{\scriptsize{0}} \\  

     &  & \mc{1}{c}{\scriptsize{(1.000)}} & \mc{1}{c}{\scriptsize{(1.000)}} & \mc{1}{c}{\scriptsize{(0.686)}} & \mc{1}{c}{\scriptsize{(0.627)}} & \mc{1}{c}{\scriptsize{(0.667)}} & \mc{1}{c}{\scriptsize{(1.000)}} & \mc{1}{c}{\scriptsize{(1.000)}} & \mc{1}{c}{\scriptsize{(1.000)}} \\  

    \mc{2}{l}{\scriptsize{\% of Pos. TE ($H_0$: $\le$ 50\% $|$ 10\% Significance)}} & \mc{1}{c}{\scriptsize{0}} & \mc{1}{c}{\scriptsize{0}} & \mc{1}{c}{\scriptsize{0}} & \mc{1}{c}{\scriptsize{0}} & \mc{1}{c}{\scriptsize{0}} & \mc{1}{c}{\scriptsize{0}} & \mc{1}{c}{\scriptsize{0}} & \mc{1}{c}{\scriptsize{0}} \\  

     &  & \mc{1}{c}{\scriptsize{(1.000)}} & \mc{1}{c}{\scriptsize{(1.000)}} & \mc{1}{c}{\scriptsize{(0.686)}} & \mc{1}{c}{\scriptsize{(0.627)}} & \mc{1}{c}{\scriptsize{(0.667)}} & \mc{1}{c}{\scriptsize{(1.000)}} & \mc{1}{c}{\scriptsize{(1.000)}} & \mc{1}{c}{\scriptsize{(1.000)}} \\  

    \mc{2}{l}{\scriptsize{\% of Pos. TE ($H_0$: $\le$ 75\% $|$ 10\% Significance)}} & \mc{1}{c}{\scriptsize{0}} & \mc{1}{c}{\scriptsize{0}} & \mc{1}{c}{\scriptsize{0}} & \mc{1}{c}{\scriptsize{0}} & \mc{1}{c}{\scriptsize{0}} & \mc{1}{c}{\scriptsize{0}} & \mc{1}{c}{\scriptsize{0}} & \mc{1}{c}{\scriptsize{0}} \\  

     &  & \mc{1}{c}{\scriptsize{(1.000)}} & \mc{1}{c}{\scriptsize{(1.000)}} & \mc{1}{c}{\scriptsize{(0.686)}} & \mc{1}{c}{\scriptsize{(0.627)}} & \mc{1}{c}{\scriptsize{(0.667)}} & \mc{1}{c}{\scriptsize{(1.000)}} & \mc{1}{c}{\scriptsize{(1.000)}} & \mc{1}{c}{\scriptsize{(1.000)}} \\  

  \hline\hline
  \end{tabular}
    \begin{tablenotes}
    \scriptsize
    \item 
Note: This table displays various estimates of the treatment effect of CARE's family education program.
Column (1) displays the ITT, without accounting for any controls.
Column (2) displays the ITT conditioning on vector of controls, $X$, consisting of APGAR scores 1 
minute after birth, an indicator for the subject being born prematurely, and an indicator for the 
father being home at baseline. We also apply IPW weights, $W$, to account for attrition.
Columns (3)--(4) are analogous to columns (1)--(2), but we restrict the control sample to subjects
who did not enroll in any alternative care.
Column (5) displys the matching estimate, where we use the Mahalanobis metric and Epanechnikov kernel
to match on controls $X$ listed above, and restrict the control sample to subjects who did not enroll
in any alternative care. Additionally, we apply IPW weights, $W$.
Columns (6)--(8) are analogous to Columns (3)--(5), except we restrict the control sample to subejcts
who did enroll in alternative care. 
The final three pairs of rows display the proportion of treatment effects in the table that are 
socially positive. The first row in each pair displays the percentage of treatment effects, and the
second row presents the inference.

Numbers in parentheses represent the $p$-value from a single hypothesis test, and are obtained from 
the empirical bootstrap distribution generated by 200 resamples of the original data. 
Bold $p$-values indicate significance at the 10\% level.
Blank point estimates indicate that we are unable to obtain estimates due to a lack of support in the data. 

    \end{tablenotes}
  \end{threeparttable}

\end{table}
\end{center}

\begin{center}
	  \begin{tabular}{cccccccccc}
  \toprule

    \scriptsize{Variable} & \scriptsize{Age} & \scriptsize{(1)} & \scriptsize{(2)} & \scriptsize{(3)} & \scriptsize{(4)} & \scriptsize{(5)} & \scriptsize{(6)} & \scriptsize{(7)} & \scriptsize{(8)} \\ 
    \midrule  

    \mc{1}{l}{\scriptsize{Mother Works}} & \mc{1}{c}{\scriptsize{2}} & \mc{1}{c}{\scriptsize{0.168}} & \mc{1}{c}{\scriptsize{0.119}} & \mc{1}{c}{\scriptsize{0.323}} & \mc{1}{c}{\scriptsize{0.348}} & \mc{1}{c}{\scriptsize{0.241}} & \mc{1}{c}{\scriptsize{0.101}} & \mc{1}{c}{\scriptsize{0.078}} & \mc{1}{c}{\scriptsize{-0.018}} \\  

     &  & \mc{1}{c}{\scriptsize{\textbf{(0.033)}}} & \mc{1}{c}{\scriptsize{(0.117)}} & \mc{1}{c}{\scriptsize{\textbf{(0.043)}}} & \mc{1}{c}{\scriptsize{\textbf{(0.054)}}} & \mc{1}{c}{\scriptsize{(0.108)}} & \mc{1}{c}{\scriptsize{(0.134)}} & \mc{1}{c}{\scriptsize{(0.233)}} & \mc{1}{c}{\scriptsize{(0.421)}} \\  

     & \mc{1}{c}{\scriptsize{3}} & \mc{1}{c}{\scriptsize{0.087}} & \mc{1}{c}{\scriptsize{0.027}} & \mc{1}{c}{\scriptsize{0.177}} & \mc{1}{c}{\scriptsize{0.151}} & \mc{1}{c}{\scriptsize{0.241}} & \mc{1}{c}{\scriptsize{0.066}} & \mc{1}{c}{\scriptsize{0.006}} & \mc{1}{c}{\scriptsize{0.117}} \\  

     &  & \mc{1}{c}{\scriptsize{(0.170)}} & \mc{1}{c}{\scriptsize{(0.410)}} & \mc{1}{c}{\scriptsize{(0.154)}} & \mc{1}{c}{\scriptsize{(0.257)}} & \mc{1}{c}{\scriptsize{(0.108)}} & \mc{1}{c}{\scriptsize{(0.251)}} & \mc{1}{c}{\scriptsize{(0.483)}} & \mc{1}{c}{\scriptsize{(0.150)}} \\  

     & \mc{1}{c}{\scriptsize{4}} & \mc{1}{c}{\scriptsize{0.118}} & \mc{1}{c}{\scriptsize{0.082}} & \mc{1}{c}{\scriptsize{0.319}} & \mc{1}{c}{\scriptsize{0.337}} & \mc{1}{c}{\scriptsize{0.271}} & \mc{1}{c}{\scriptsize{0.060}} & \mc{1}{c}{\scriptsize{0.049}} & \mc{1}{c}{\scriptsize{0.089}} \\  

     &  & \mc{1}{c}{\scriptsize{\textbf{(0.088)}}} & \mc{1}{c}{\scriptsize{(0.203)}} & \mc{1}{c}{\scriptsize{\textbf{(0.047)}}} & \mc{1}{c}{\scriptsize{\textbf{(0.065)}}} & \mc{1}{c}{\scriptsize{\textbf{(0.083)}}} & \mc{1}{c}{\scriptsize{(0.243)}} & \mc{1}{c}{\scriptsize{(0.330)}} & \mc{1}{c}{\scriptsize{(0.182)}} \\  

     & \mc{1}{c}{\scriptsize{5}} & \mc{1}{c}{\scriptsize{0.067}} & \mc{1}{c}{\scriptsize{0.008}} & \mc{1}{c}{\scriptsize{0.367}} & \mc{1}{c}{\scriptsize{0.307}} & \mc{1}{c}{\scriptsize{0.291}} & \mc{1}{c}{\scriptsize{-0.056}} & \mc{1}{c}{\scriptsize{-0.062}} & \mc{1}{c}{\scriptsize{0.055}} \\  

     &  & \mc{1}{c}{\scriptsize{(0.227)}} & \mc{1}{c}{\scriptsize{(0.465)}} & \mc{1}{c}{\scriptsize{(0.999)}} & \mc{1}{c}{\scriptsize{\textbf{(0.001)}}} & \mc{1}{c}{\scriptsize{\textbf{(0.074)}}} & \mc{1}{c}{\scriptsize{(0.253)}} & \mc{1}{c}{\scriptsize{(0.196)}} & \mc{1}{c}{\scriptsize{(0.289)}} \\  

     & \mc{1}{c}{\scriptsize{21}} & \mc{1}{c}{\scriptsize{-0.018}} & \mc{1}{c}{\scriptsize{-0.012}} & \mc{1}{c}{\scriptsize{0.510}} & \mc{1}{c}{\scriptsize{0.562}} & \mc{1}{c}{\scriptsize{-0.139}} & \mc{1}{c}{\scriptsize{-0.097}} & \mc{1}{c}{\scriptsize{-0.144}} & \mc{1}{c}{\scriptsize{-0.067}} \\  

     &  & \mc{1}{c}{\scriptsize{(0.422)}} & \mc{1}{c}{\scriptsize{(0.458)}} & \mc{1}{c}{\scriptsize{\textbf{(0.008)}}} & \mc{1}{c}{\scriptsize{(0.979)}} & \mc{1}{c}{\scriptsize{(0.295)}} & \mc{1}{c}{\scriptsize{(0.205)}} & \mc{1}{c}{\scriptsize{(0.166)}} & \mc{1}{c}{\scriptsize{(0.341)}} \\  

    \mc{1}{l}{\scriptsize{Mother Works Factor}} & \mc{1}{c}{\scriptsize{2 to 21}} & \mc{1}{c}{\scriptsize{-0.207}} & \mc{1}{c}{\scriptsize{0.085}} & \mc{1}{c}{\scriptsize{-0.662}} & \mc{1}{c}{\scriptsize{-0.578}} & \mc{1}{c}{\scriptsize{-0.872}} & \mc{1}{c}{\scriptsize{-0.071}} & \mc{1}{c}{\scriptsize{0.103}} & \mc{1}{c}{\scriptsize{-0.166}} \\  

     &  & \mc{1}{c}{\scriptsize{(0.208)}} & \mc{1}{c}{\scriptsize{(0.365)}} & \mc{1}{c}{\scriptsize{\textbf{(0.001)}}} & \mc{1}{c}{\scriptsize{(0.998)}} & \mc{1}{c}{\scriptsize{\textbf{(0.099)}}} & \mc{1}{c}{\scriptsize{(0.383)}} & \mc{1}{c}{\scriptsize{(0.360)}} & \mc{1}{c}{\scriptsize{(0.280)}} \\  

  \bottomrule
  \end{tabular}
\end{center}

\begin{center}
	  \begin{tabular}{cccccccccc}
  \toprule

    \scriptsize{Variable} & \scriptsize{Age} & \scriptsize{(1)} & \scriptsize{(2)} & \scriptsize{(3)} & \scriptsize{(4)} & \scriptsize{(5)} & \scriptsize{(6)} & \scriptsize{(7)} & \scriptsize{(8)} \\ 
    \midrule  

    \mc{1}{l}{\scriptsize{Activity - tempo}} & \mc{1}{c}{\scriptsize{8}} & \mc{1}{c}{\scriptsize{0.281}} & \mc{1}{c}{\scriptsize{-0.949}} & \mc{1}{c}{\scriptsize{-1.170}} & \mc{1}{c}{\scriptsize{-2.661}} & \mc{1}{c}{\scriptsize{-1.476}} & \mc{1}{c}{\scriptsize{0.896}} & \mc{1}{c}{\scriptsize{-0.360}} & \mc{1}{c}{\scriptsize{0.395}} \\  

     &  & \mc{1}{c}{\scriptsize{(0.645)}} & \mc{1}{c}{\scriptsize{(0.250)}} & \mc{1}{c}{\scriptsize{(0.250)}} & \mc{1}{c}{\scriptsize{(0.118)}} & \mc{1}{c}{\scriptsize{(0.237)}} & \mc{1}{c}{\scriptsize{(0.789)}} & \mc{1}{c}{\scriptsize{(0.368)}} & \mc{1}{c}{\scriptsize{(0.632)}} \\  

    \mc{1}{l}{\scriptsize{Activity - vigor}} & \mc{1}{c}{\scriptsize{8}} & \mc{1}{c}{\scriptsize{0.062}} & \mc{1}{c}{\scriptsize{-0.509}} & \mc{1}{c}{\scriptsize{-1.326}} & \mc{1}{c}{\scriptsize{-1.246}} & \mc{1}{c}{\scriptsize{-1.509}} & \mc{1}{c}{\scriptsize{0.448}} & \mc{1}{c}{\scriptsize{-0.182}} & \mc{1}{c}{\scriptsize{0.313}} \\  

     &  & \mc{1}{c}{\scriptsize{(0.461)}} & \mc{1}{c}{\scriptsize{(0.697)}} & \mc{1}{c}{\scriptsize{(0.934)}} & \mc{1}{c}{\scriptsize{(0.789)}} & \mc{1}{c}{\scriptsize{(0.961)}} & \mc{1}{c}{\scriptsize{(0.329)}} & \mc{1}{c}{\scriptsize{(0.579)}} & \mc{1}{c}{\scriptsize{(0.342)}} \\  

    \mc{1}{l}{\scriptsize{Emotionality - anger}} & \mc{1}{c}{\scriptsize{8}} & \mc{1}{c}{\scriptsize{0.438}} & \mc{1}{c}{\scriptsize{0.433}} & \mc{1}{c}{\scriptsize{0.120}} & \mc{1}{c}{\scriptsize{-0.522}} & \mc{1}{c}{\scriptsize{0.382}} & \mc{1}{c}{\scriptsize{0.865}} & \mc{1}{c}{\scriptsize{0.837}} & \mc{1}{c}{\scriptsize{0.993}} \\  

     &  & \mc{1}{c}{\scriptsize{(0.671)}} & \mc{1}{c}{\scriptsize{(0.632)}} & \mc{1}{c}{\scriptsize{(0.500)}} & \mc{1}{c}{\scriptsize{(0.421)}} & \mc{1}{c}{\scriptsize{(0.618)}} & \mc{1}{c}{\scriptsize{(0.776)}} & \mc{1}{c}{\scriptsize{(0.724)}} & \mc{1}{c}{\scriptsize{(0.789)}} \\  

    \mc{1}{l}{\scriptsize{Emotionality - fear}} & \mc{1}{c}{\scriptsize{8}} & \mc{1}{c}{\scriptsize{1.094}} & \mc{1}{c}{\scriptsize{0.906}} & \mc{1}{c}{\scriptsize{0.464}} & \mc{1}{c}{\scriptsize{-0.635}} & \mc{1}{c}{\scriptsize{0.721}} & \mc{1}{c}{\scriptsize{1.500}} & \mc{1}{c}{\scriptsize{1.464}} & \mc{1}{c}{\scriptsize{1.769}} \\  

     &  & \mc{1}{c}{\scriptsize{(0.829)}} & \mc{1}{c}{\scriptsize{(0.737)}} & \mc{1}{c}{\scriptsize{(0.605)}} & \mc{1}{c}{\scriptsize{(0.342)}} & \mc{1}{c}{\scriptsize{(0.684)}} & \mc{1}{c}{\scriptsize{(0.921)}} & \mc{1}{c}{\scriptsize{(0.842)}} & \mc{1}{c}{\scriptsize{(0.947)}} \\  

    \mc{1}{l}{\scriptsize{Emotionality - general}} & \mc{1}{c}{\scriptsize{8}} & \mc{1}{c}{\scriptsize{0.969}} & \mc{1}{c}{\scriptsize{0.827}} & \mc{1}{c}{\scriptsize{3.634}} & \mc{1}{c}{\scriptsize{3.741}} & \mc{1}{c}{\scriptsize{3.617}} & \mc{1}{c}{\scriptsize{0.229}} & \mc{1}{c}{\scriptsize{-0.017}} & \mc{1}{c}{\scriptsize{-0.011}} \\  

     &  & \mc{1}{c}{\scriptsize{(0.855)}} & \mc{1}{c}{\scriptsize{(0.763)}} & \mc{1}{c}{\scriptsize{(0.974)}} & \mc{1}{c}{\scriptsize{(0.934)}} & \mc{1}{c}{\scriptsize{(0.974)}} & \mc{1}{c}{\scriptsize{(0.566)}} & \mc{1}{c}{\scriptsize{(0.539)}} & \mc{1}{c}{\scriptsize{(0.526)}} \\  

    \mc{1}{l}{\scriptsize{Impulsitivity - control}} & \mc{1}{c}{\scriptsize{8}} & \mc{1}{c}{\scriptsize{0.938}} & \mc{1}{c}{\scriptsize{1.254}} & \mc{1}{c}{\scriptsize{1.585}} & \mc{1}{c}{\scriptsize{1.912}} & \mc{1}{c}{\scriptsize{1.605}} & \mc{1}{c}{\scriptsize{1.073}} & \mc{1}{c}{\scriptsize{1.311}} & \mc{1}{c}{\scriptsize{1.355}} \\  

     &  & \mc{1}{c}{\scriptsize{(0.921)}} & \mc{1}{c}{\scriptsize{(0.961)}} & \mc{1}{c}{\scriptsize{(0.921)}} & \mc{1}{c}{\scriptsize{(0.895)}} & \mc{1}{c}{\scriptsize{(0.921)}} & \mc{1}{c}{\scriptsize{(0.947)}} & \mc{1}{c}{\scriptsize{(0.961)}} & \mc{1}{c}{\scriptsize{(0.974)}} \\  

    \mc{1}{l}{\scriptsize{Impulsitivity - decisive}} & \mc{1}{c}{\scriptsize{8}} & \mc{1}{c}{\scriptsize{0.344}} & \mc{1}{c}{\scriptsize{-0.517}} & \mc{1}{c}{\scriptsize{2.013}} & \mc{1}{c}{\scriptsize{1.645}} & \mc{1}{c}{\scriptsize{1.679}} & \mc{1}{c}{\scriptsize{0.115}} & \mc{1}{c}{\scriptsize{-0.917}} & \mc{1}{c}{\scriptsize{-0.279}} \\  

     &  & \mc{1}{c}{\scriptsize{(0.658)}} & \mc{1}{c}{\scriptsize{(0.276)}} & \mc{1}{c}{\scriptsize{(0.987)}} & \mc{1}{c}{\scriptsize{(0.934)}} & \mc{1}{c}{\scriptsize{(0.974)}} & \mc{1}{c}{\scriptsize{(0.566)}} & \mc{1}{c}{\scriptsize{(0.132)}} & \mc{1}{c}{\scriptsize{(0.421)}} \\  

    \mc{1}{l}{\scriptsize{Impulsitivity - perservere}} & \mc{1}{c}{\scriptsize{8}} & \mc{1}{c}{\scriptsize{-0.406}} & \mc{1}{c}{\scriptsize{-0.255}} & \mc{1}{c}{\scriptsize{0.746}} & \mc{1}{c}{\scriptsize{1.570}} & \mc{1}{c}{\scriptsize{0.497}} & \mc{1}{c}{\scriptsize{-0.677}} & \mc{1}{c}{\scriptsize{-0.680}} & \mc{1}{c}{\scriptsize{-0.681}} \\  

     &  & \mc{1}{c}{\scriptsize{(0.329)}} & \mc{1}{c}{\scriptsize{(0.408)}} & \mc{1}{c}{\scriptsize{(0.711)}} & \mc{1}{c}{\scriptsize{(0.750)}} & \mc{1}{c}{\scriptsize{(0.684)}} & \mc{1}{c}{\scriptsize{(0.211)}} & \mc{1}{c}{\scriptsize{(0.237)}} & \mc{1}{c}{\scriptsize{(0.224)}} \\  

    \mc{1}{l}{\scriptsize{Impulsitivity - sensation}} & \mc{1}{c}{\scriptsize{8}} & \mc{1}{c}{\scriptsize{-0.500}} & \mc{1}{c}{\scriptsize{-0.396}} & \mc{1}{c}{\scriptsize{-1.085}} & \mc{1}{c}{\scriptsize{-0.529}} & \mc{1}{c}{\scriptsize{-1.019}} & \mc{1}{c}{\scriptsize{-0.240}} & \mc{1}{c}{\scriptsize{-0.234}} & \mc{1}{c}{\scriptsize{-0.240}} \\  

     &  & \mc{1}{c}{\scriptsize{(0.237)}} & \mc{1}{c}{\scriptsize{(0.342)}} & \mc{1}{c}{\scriptsize{(0.316)}} & \mc{1}{c}{\scriptsize{(0.250)}} & \mc{1}{c}{\scriptsize{(0.329)}} & \mc{1}{c}{\scriptsize{(0.382)}} & \mc{1}{c}{\scriptsize{(0.395)}} & \mc{1}{c}{\scriptsize{(0.408)}} \\  

    \mc{1}{l}{\scriptsize{Sociablity}} & \mc{1}{c}{\scriptsize{8}} & \mc{1}{c}{\scriptsize{0.094}} & \mc{1}{c}{\scriptsize{0.133}} & \mc{1}{c}{\scriptsize{-0.844}} & \mc{1}{c}{\scriptsize{0.198}} & \mc{1}{c}{\scriptsize{-0.738}} & \mc{1}{c}{\scriptsize{0.115}} & \mc{1}{c}{\scriptsize{0.140}} & \mc{1}{c}{\scriptsize{0.244}} \\  

     &  & \mc{1}{c}{\scriptsize{(0.408)}} & \mc{1}{c}{\scriptsize{(0.434)}} & \mc{1}{c}{\scriptsize{(0.737)}} & \mc{1}{c}{\scriptsize{(0.434)}} & \mc{1}{c}{\scriptsize{(0.737)}} & \mc{1}{c}{\scriptsize{(0.434)}} & \mc{1}{c}{\scriptsize{(0.447)}} & \mc{1}{c}{\scriptsize{(0.368)}} \\  

  \bottomrule
  \end{tabular}
\end{center}

\begin{center}
	\begin{table}[H]
\captionsetup{singlelinecheck=false,justification=centering}
\caption{ABC Average Treatment Effects, Males \\ Adoption \label{tab:ate_male_apx7}}

  \begin{threeparttable}
  \begin{tabular}{cccccccccc}
  \hline\hline

     &  & \scriptsize{(1)} & \scriptsize{(2)} & \scriptsize{(3)} & \scriptsize{(4)} & \scriptsize{(5)} & \scriptsize{(6)} & \scriptsize{(7)} & \scriptsize{(8)} \\  

     &  &  &  & \mc{3}{c}{\scriptsize{$P=0$}} & \mc{3}{c}{\scriptsize{$P=1$}} \\ 
    \cmidrule(lr){5-7} \cmidrule(lr){8-10} 

    \scriptsize{Variable} & \scriptsize{Age} & \scriptsize{ITT} & \scriptsize{ITT$|X,W$} & \scriptsize{ITT} & \scriptsize{ITT$|X,W$} & \scriptsize{KE$|X,W$} & \scriptsize{ITT} & \scriptsize{ITT$|X,W$} & \scriptsize{KE$|X,W$} \\ 
    \hline  

    \mc{1}{l}{\scriptsize{Ever Adopted}} &  & \mc{1}{c}{\scriptsize{0.007}} & \mc{1}{c}{\scriptsize{0.032}} & \mc{1}{c}{\scriptsize{-0.013}} & \mc{1}{c}{\scriptsize{-0.010}} & \mc{1}{c}{\scriptsize{0.004}} & \mc{1}{c}{\scriptsize{0.045}} & \mc{1}{c}{\scriptsize{0.022}} & \mc{1}{c}{\scriptsize{0.052}} \\  

     &  & \mc{1}{c}{\scriptsize{(0.471)}} & \mc{1}{c}{\scriptsize{(0.216)}} & \mc{1}{c}{\scriptsize{(0.510)}} & \mc{1}{c}{\scriptsize{(0.549)}} & \mc{1}{c}{\scriptsize{(0.490)}} & \mc{1}{c}{\scriptsize{(0.118)}} & \mc{1}{c}{\scriptsize{(0.275)}} & \mc{1}{c}{\scriptsize{\textbf{(0.078)}}} \\ 
    \hline  

    \\[0.1cm]
    \mc{2}{l}{\scriptsize{\% of Pos. TE ($H_0$: $\le$ 25\% $|$ 10\% Significance)}} & \mc{1}{c}{\scriptsize{0}} & \mc{1}{c}{\scriptsize{0}} & \mc{1}{c}{\scriptsize{0}} & \mc{1}{c}{\scriptsize{0}} & \mc{1}{c}{\scriptsize{0}} & \mc{1}{c}{\scriptsize{0}} & \mc{1}{c}{\scriptsize{0}} & \mc{1}{c}{\scriptsize{100}} \\  

     &  & \mc{1}{c}{\scriptsize{(0.235)}} & \mc{1}{c}{\scriptsize{(0.255)}} & \mc{1}{c}{\scriptsize{(0.882)}} & \mc{1}{c}{\scriptsize{(0.882)}} & \mc{1}{c}{\scriptsize{(0.235)}} & \mc{1}{c}{\scriptsize{(0.490)}} & \mc{1}{c}{\scriptsize{(0.196)}} & \mc{1}{c}{\scriptsize{\textbf{(0.000)}}} \\  

    \mc{2}{l}{\scriptsize{\% of Pos. TE ($H_0$: $\le$ 50\% $|$ 10\% Significance)}} & \mc{1}{c}{\scriptsize{0}} & \mc{1}{c}{\scriptsize{0}} & \mc{1}{c}{\scriptsize{0}} & \mc{1}{c}{\scriptsize{0}} & \mc{1}{c}{\scriptsize{0}} & \mc{1}{c}{\scriptsize{0}} & \mc{1}{c}{\scriptsize{0}} & \mc{1}{c}{\scriptsize{100}} \\  

     &  & \mc{1}{c}{\scriptsize{(0.882)}} & \mc{1}{c}{\scriptsize{(0.882)}} & \mc{1}{c}{\scriptsize{(0.882)}} & \mc{1}{c}{\scriptsize{(0.882)}} & \mc{1}{c}{\scriptsize{(0.843)}} & \mc{1}{c}{\scriptsize{(0.490)}} & \mc{1}{c}{\scriptsize{(0.667)}} & \mc{1}{c}{\scriptsize{\textbf{(0.000)}}} \\  

    \mc{2}{l}{\scriptsize{\% of Pos. TE ($H_0$: $\le$ 75\% $|$ 10\% Significance)}} & \mc{1}{c}{\scriptsize{0}} & \mc{1}{c}{\scriptsize{0}} & \mc{1}{c}{\scriptsize{0}} & \mc{1}{c}{\scriptsize{0}} & \mc{1}{c}{\scriptsize{0}} & \mc{1}{c}{\scriptsize{0}} & \mc{1}{c}{\scriptsize{0}} & \mc{1}{c}{\scriptsize{100}} \\  

     &  & \mc{1}{c}{\scriptsize{(0.882)}} & \mc{1}{c}{\scriptsize{(0.882)}} & \mc{1}{c}{\scriptsize{(0.882)}} & \mc{1}{c}{\scriptsize{(0.882)}} & \mc{1}{c}{\scriptsize{(0.843)}} & \mc{1}{c}{\scriptsize{(0.667)}} & \mc{1}{c}{\scriptsize{(0.667)}} & \mc{1}{c}{\scriptsize{(0.490)}} \\  

  \hline\hline
  \end{tabular}
    \begin{tablenotes}
    \scriptsize
    \item 
Note: This table displays various estimates of the treatment effect of ABC's school age program.
Column (1) displays the ITT, without accounting for any controls.
Column (2) displays the ITT conditioning on vector of controls, $X$, consisting of the Apgar score 1 minute after birth, the HRI index, maternal IQ, an
indicator for teenage pregnancy of the mother, an indicator for the father being at 
home, and an indicator for having a grandmother residing in the same county. We also apply IPW weights, $W$, to account for attrition.
Columns (3)--(4) are analogous to columns (1)--(2), but we restrict the control sample to subjects
who did not enroll in any alternative care.
Column (5) displys the matching estimate, where we use the Mahalanobis metric and Epanechnikov kernel
to match on controls $X$ listed above, and restrict the control sample to subjects who did not enroll
in any alternative care. Additionally, we apply IPW weights, $W$.
Columns (6)--(8) are analogous to Columns (3)--(5), except we restrict the control sample to subejcts
who did enroll in alternative care. The final three pairs of rows display the proportion of treatment effects in the table that are 
socially positive. The first row in each pair displays the percentage of treatment effects, and the
second row presents the inference. 
Numbers in parentheses represent the $p$-value from a single hypothesis test, and are obtained from 
the empirical bootstrap distribution generated by 200 resamples of the original data. 
Bold $p$-values indicate significance at the 10\% level.
Blank point estimates indicate that we are unable to obtain estimates due to a lack of support in the data. 

    \end{tablenotes}
  \end{threeparttable}

\end{table}
\end{center}

\begin{center}
	\begin{table}[H]
\captionsetup{singlelinecheck=false,justification=centering}
\caption{ABC/CARE Average Treatment Effects, Males \\ Crime \label{tab:ate_male_apx8}}

  \begin{threeparttable}
  \begin{tabular}{cccccccccc}
  \hline\hline

     &  & \scriptsize{(1)} & \scriptsize{(2)} & \scriptsize{(3)} & \scriptsize{(4)} & \scriptsize{(5)} & \scriptsize{(6)} & \scriptsize{(7)} & \scriptsize{(8)} \\  

     &  &  &  & \mc{3}{c}{\scriptsize{$P=0$}} & \mc{3}{c}{\scriptsize{$P=1$}} \\ 
    \cmidrule(lr){5-7} \cmidrule(lr){8-10} 

    \scriptsize{Variable} & \scriptsize{Age} & \scriptsize{ITT} & \scriptsize{ITT$|X,W$} & \scriptsize{ITT} & \scriptsize{ITT$|X,W$} & \scriptsize{KE$|X,W$} & \scriptsize{ITT} & \scriptsize{ITT$|X,W$} & \scriptsize{KE$|X,W$} \\ 
    \hline  

    \mc{1}{l}{\scriptsize{Total Felony Arrests}} & \mc{1}{c}{\scriptsize{Mid-30s}} & \mc{1}{c}{\scriptsize{0.145}} & \mc{1}{c}{\scriptsize{0.236}} & \mc{1}{c}{\scriptsize{0.303}} & \mc{1}{c}{\scriptsize{0.756}} & \mc{1}{c}{\scriptsize{0.800}} & \mc{1}{c}{\scriptsize{0.069}} & \mc{1}{c}{\scriptsize{0.148}} & \mc{1}{c}{\scriptsize{0.411}} \\  

     &  & \mc{1}{c}{\scriptsize{(0.608)}} & \mc{1}{c}{\scriptsize{(0.608)}} & \mc{1}{c}{\scriptsize{(0.647)}} & \mc{1}{c}{\scriptsize{(0.784)}} & \mc{1}{c}{\scriptsize{(0.804)}} & \mc{1}{c}{\scriptsize{(0.529)}} & \mc{1}{c}{\scriptsize{(0.569)}} & \mc{1}{c}{\scriptsize{(0.686)}} \\  

    \mc{1}{l}{\scriptsize{Total Misdemeanor Arrests}} & \mc{1}{c}{\scriptsize{Mid-30s}} & \mc{1}{c}{\scriptsize{-0.686}} & \mc{1}{c}{\scriptsize{-0.584}} & \mc{1}{c}{\scriptsize{-0.823}} & \mc{1}{c}{\scriptsize{-0.781}} & \mc{1}{c}{\scriptsize{-0.821}} & \mc{1}{c}{\scriptsize{-0.621}} & \mc{1}{c}{\scriptsize{-0.554}} & \mc{1}{c}{\scriptsize{-0.319}} \\  

     &  & \mc{1}{c}{\scriptsize{\textbf{(0.039)}}} & \mc{1}{c}{\scriptsize{\textbf{(0.098)}}} & \mc{1}{c}{\scriptsize{(0.118)}} & \mc{1}{c}{\scriptsize{(0.137)}} & \mc{1}{c}{\scriptsize{(0.157)}} & \mc{1}{c}{\scriptsize{\textbf{(0.078)}}} & \mc{1}{c}{\scriptsize{\textbf{(0.098)}}} & \mc{1}{c}{\scriptsize{(0.216)}} \\  

    \mc{1}{l}{\scriptsize{Total Years Incarcerated}} & \mc{1}{c}{\scriptsize{30}} & \mc{1}{c}{\scriptsize{0.106}} & \mc{1}{c}{\scriptsize{0.185}} & \mc{1}{c}{\scriptsize{0.184}} & \mc{1}{c}{\scriptsize{0.268}} & \mc{1}{c}{\scriptsize{0.367}} & \mc{1}{c}{\scriptsize{0.066}} & \mc{1}{c}{\scriptsize{0.145}} & \mc{1}{c}{\scriptsize{0.221}} \\  

     &  & \mc{1}{c}{\scriptsize{(0.667)}} & \mc{1}{c}{\scriptsize{(0.824)}} & \mc{1}{c}{\scriptsize{(0.667)}} & \mc{1}{c}{\scriptsize{(0.804)}} & \mc{1}{c}{\scriptsize{(0.843)}} & \mc{1}{c}{\scriptsize{(0.627)}} & \mc{1}{c}{\scriptsize{(0.647)}} & \mc{1}{c}{\scriptsize{(0.843)}} \\  

    \mc{1}{l}{\scriptsize{Crime Factor}} & \mc{1}{c}{\scriptsize{30 to Mid-30s}} & \mc{1}{c}{\scriptsize{-0.001}} & \mc{1}{c}{\scriptsize{0.063}} & \mc{1}{c}{\scriptsize{0.043}} & \mc{1}{c}{\scriptsize{0.194}} & \mc{1}{c}{\scriptsize{0.220}} & \mc{1}{c}{\scriptsize{-0.022}} & \mc{1}{c}{\scriptsize{0.037}} & \mc{1}{c}{\scriptsize{0.114}} \\  

     &  & \mc{1}{c}{\scriptsize{(0.451)}} & \mc{1}{c}{\scriptsize{(0.569)}} & \mc{1}{c}{\scriptsize{(0.569)}} & \mc{1}{c}{\scriptsize{(0.725)}} & \mc{1}{c}{\scriptsize{(0.725)}} & \mc{1}{c}{\scriptsize{(0.412)}} & \mc{1}{c}{\scriptsize{(0.490)}} & \mc{1}{c}{\scriptsize{(0.627)}} \\ 
    \hline  

    \\[0.1cm]
    \mc{2}{l}{\scriptsize{\% of Pos. TE ($H_0$: $\le$ 25\% $|$ 10\% Significance)}} & \mc{1}{c}{\scriptsize{25}} & \mc{1}{c}{\scriptsize{25}} & \mc{1}{c}{\scriptsize{0}} & \mc{1}{c}{\scriptsize{0}} & \mc{1}{c}{\scriptsize{0}} & \mc{1}{c}{\scriptsize{25}} & \mc{1}{c}{\scriptsize{25}} & \mc{1}{c}{\scriptsize{0}} \\  

     &  & \mc{1}{c}{\scriptsize{(0.373)}} & \mc{1}{c}{\scriptsize{(0.275)}} & \mc{1}{c}{\scriptsize{(0.549)}} & \mc{1}{c}{\scriptsize{(1.000)}} & \mc{1}{c}{\scriptsize{(1.000)}} & \mc{1}{c}{\scriptsize{(0.314)}} & \mc{1}{c}{\scriptsize{(0.255)}} & \mc{1}{c}{\scriptsize{(1.000)}} \\  

    \mc{2}{l}{\scriptsize{\% of Pos. TE ($H_0$: $\le$ 50\% $|$ 10\% Significance)}} & \mc{1}{c}{\scriptsize{25}} & \mc{1}{c}{\scriptsize{25}} & \mc{1}{c}{\scriptsize{0}} & \mc{1}{c}{\scriptsize{0}} & \mc{1}{c}{\scriptsize{0}} & \mc{1}{c}{\scriptsize{25}} & \mc{1}{c}{\scriptsize{25}} & \mc{1}{c}{\scriptsize{0}} \\  

     &  & \mc{1}{c}{\scriptsize{(0.745)}} & \mc{1}{c}{\scriptsize{(0.647)}} & \mc{1}{c}{\scriptsize{(1.000)}} & \mc{1}{c}{\scriptsize{(1.000)}} & \mc{1}{c}{\scriptsize{(1.000)}} & \mc{1}{c}{\scriptsize{(0.647)}} & \mc{1}{c}{\scriptsize{(0.588)}} & \mc{1}{c}{\scriptsize{(1.000)}} \\  

    \mc{2}{l}{\scriptsize{\% of Pos. TE ($H_0$: $\le$ 75\% $|$ 10\% Significance)}} & \mc{1}{c}{\scriptsize{25}} & \mc{1}{c}{\scriptsize{25}} & \mc{1}{c}{\scriptsize{0}} & \mc{1}{c}{\scriptsize{0}} & \mc{1}{c}{\scriptsize{0}} & \mc{1}{c}{\scriptsize{25}} & \mc{1}{c}{\scriptsize{25}} & \mc{1}{c}{\scriptsize{0}} \\  

     &  & \mc{1}{c}{\scriptsize{(1.000)}} & \mc{1}{c}{\scriptsize{(1.000)}} & \mc{1}{c}{\scriptsize{(1.000)}} & \mc{1}{c}{\scriptsize{(1.000)}} & \mc{1}{c}{\scriptsize{(1.000)}} & \mc{1}{c}{\scriptsize{(1.000)}} & \mc{1}{c}{\scriptsize{(1.000)}} & \mc{1}{c}{\scriptsize{(1.000)}} \\  

  \hline\hline
  \end{tabular}
    \begin{tablenotes}
    \scriptsize
    \item 
Note: This table displays various estimates of the treatment effect of ABC/CARE's center-based care.
Column (1) displays the ITT, without accounting for any controls.
Column (2) displays the ITT conditioning on vector of controls, $X$, consisting of Apgar scores 1 minute and 5 minutes after birth, the HRI index, maternal IQ,
an indicator for having a grandmother residing in the same county, and an index for the number
of relatives living in the same household. We also apply IPW weights, $W$, to account for attrition.
Columns (3)--(4) are analogous to columns (1)--(2), but we restrict the control sample to subjects
who did not enroll in any alternative care.
Column (5) displys the matching estimate, where we use the Mahalanobis metric and Epanechnikov kernel
to match on controls $X$ listed above, and restrict the control sample to subjects who did not enroll
in any alternative care. Additionally, we apply IPW weights, $W$.
Columns (6)--(8) are analogous to Columns (3)--(5), except we restrict the control sample to subejcts
who did enroll in alternative care. The final three pairs of rows display the proportion of treatment effects in the table that are 
socially positive. The first row in each pair displays the percentage of treatment effects, and the
second row presents the inference. 
Numbers in parentheses represent the $p$-value from a single hypothesis test, and are obtained from 
the empirical bootstrap distribution generated by 200 resamples of the original data. 
Bold $p$-values indicate significance at the 10\% level.
Blank point estimates indicate that we are unable to obtain estimates due to a lack of support in the data. 

    \end{tablenotes}
  \end{threeparttable}

\end{table}
\end{center}

\begin{center}
	  \begin{tabular}{cccccccccc}
  \toprule

    \scriptsize{Variable} & \scriptsize{Age} & \scriptsize{(1)} & \scriptsize{(2)} & \scriptsize{(3)} & \scriptsize{(4)} & \scriptsize{(5)} & \scriptsize{(6)} & \scriptsize{(7)} & \scriptsize{(8)} \\ 
    \midrule  

    \mc{1}{l}{\scriptsize{Total Felony Arrests}} & \mc{1}{c}{\scriptsize{Mid-30s}} & \mc{1}{c}{\scriptsize{0.196}} & \mc{1}{c}{\scriptsize{0.685}} & \mc{1}{c}{\scriptsize{0.946}} & \mc{1}{c}{\scriptsize{1.523}} & \mc{1}{c}{\scriptsize{1.340}} & \mc{1}{c}{\scriptsize{0.017}} & \mc{1}{c}{\scriptsize{0.481}} & \mc{1}{c}{\scriptsize{0.188}} \\  

     &  & \mc{1}{c}{\scriptsize{(0.368)}} & \mc{1}{c}{\scriptsize{(0.183)}} & \mc{1}{c}{\scriptsize{\textbf{(0.002)}}} & \mc{1}{c}{\scriptsize{\textbf{(0.064)}}} & \mc{1}{c}{\scriptsize{\textbf{(0.026)}}} & \mc{1}{c}{\scriptsize{(0.489)}} & \mc{1}{c}{\scriptsize{(0.284)}} & \mc{1}{c}{\scriptsize{(0.410)}} \\  

    \mc{1}{l}{\scriptsize{Total Misdemeanor Arrests}} & \mc{1}{c}{\scriptsize{Mid-30s}} & \mc{1}{c}{\scriptsize{-0.501}} & \mc{1}{c}{\scriptsize{-0.244}} & \mc{1}{c}{\scriptsize{-0.251}} & \mc{1}{c}{\scriptsize{-0.298}} & \mc{1}{c}{\scriptsize{-0.034}} & \mc{1}{c}{\scriptsize{-0.666}} & \mc{1}{c}{\scriptsize{-0.246}} & \mc{1}{c}{\scriptsize{-0.507}} \\  

     &  & \mc{1}{c}{\scriptsize{(0.171)}} & \mc{1}{c}{\scriptsize{(0.289)}} & \mc{1}{c}{\scriptsize{\textbf{(0.001)}}} & \mc{1}{c}{\scriptsize{(0.314)}} & \mc{1}{c}{\scriptsize{(0.422)}} & \mc{1}{c}{\scriptsize{(0.147)}} & \mc{1}{c}{\scriptsize{(0.329)}} & \mc{1}{c}{\scriptsize{(0.168)}} \\  

    \mc{1}{l}{\scriptsize{Total Years Incarcerated}} & \mc{1}{c}{\scriptsize{30}} & \mc{1}{c}{\scriptsize{0.348}} & \mc{1}{c}{\scriptsize{0.548}} & \mc{1}{c}{\scriptsize{0.553}} & \mc{1}{c}{\scriptsize{0.772}} & \mc{1}{c}{\scriptsize{0.701}} & \mc{1}{c}{\scriptsize{0.338}} & \mc{1}{c}{\scriptsize{0.538}} & \mc{1}{c}{\scriptsize{0.471}} \\  

     &  & \mc{1}{c}{\scriptsize{\textbf{(0.088)}}} & \mc{1}{c}{\scriptsize{\textbf{(0.058)}}} & \mc{1}{c}{\scriptsize{\textbf{(0.013)}}} & \mc{1}{c}{\scriptsize{\textbf{(0.014)}}} & \mc{1}{c}{\scriptsize{\textbf{(0.009)}}} & \mc{1}{c}{\scriptsize{(0.103)}} & \mc{1}{c}{\scriptsize{\textbf{(0.070)}}} & \mc{1}{c}{\scriptsize{\textbf{(0.066)}}} \\  

    \mc{1}{l}{\scriptsize{Crime Factor}} & \mc{1}{c}{\scriptsize{30 to Mid-30s}} & \mc{1}{c}{\scriptsize{0.192}} & \mc{1}{c}{\scriptsize{0.397}} & \mc{1}{c}{\scriptsize{0.560}} & \mc{1}{c}{\scriptsize{0.690}} & \mc{1}{c}{\scriptsize{0.649}} & \mc{1}{c}{\scriptsize{0.116}} & \mc{1}{c}{\scriptsize{0.371}} & \mc{1}{c}{\scriptsize{0.226}} \\  

     &  & \mc{1}{c}{\scriptsize{(0.304)}} & \mc{1}{c}{\scriptsize{(0.212)}} & \mc{1}{c}{\scriptsize{\textbf{(0.002)}}} & \mc{1}{c}{\scriptsize{(0.998)}} & \mc{1}{c}{\scriptsize{\textbf{(0.051)}}} & \mc{1}{c}{\scriptsize{(0.402)}} & \mc{1}{c}{\scriptsize{(0.252)}} & \mc{1}{c}{\scriptsize{(0.313)}} \\  

  \bottomrule
  \end{tabular}
\end{center}

\begin{center}
	\begin{table}[H]
\captionsetup{singlelinecheck=false,justification=centering}
\caption{CARE Average Treatment Effects, Males \\ Self-Reported Health \label{tab:ate_male_apx10}}

  \begin{threeparttable}
  \begin{tabular}{cccccccccc}
  \hline\hline

     &  & \scriptsize{(1)} & \scriptsize{(2)} & \scriptsize{(3)} & \scriptsize{(4)} & \scriptsize{(5)} & \scriptsize{(6)} & \scriptsize{(7)} & \scriptsize{(8)} \\  

     &  &  &  & \mc{3}{c}{\scriptsize{$P=0$}} & \mc{3}{c}{\scriptsize{$P=1$}} \\ 
    \cmidrule(lr){5-7} \cmidrule(lr){8-10} 

    \scriptsize{Variable} & \scriptsize{Age} & \scriptsize{ITT} & \scriptsize{ITT$|X,W$} & \scriptsize{ITT} & \scriptsize{ITT$|X,W$} & \scriptsize{KE$|X,W$} & \scriptsize{ITT} & \scriptsize{ITT$|X,W$} & \scriptsize{KE$|X,W$} \\ 
    \hline  

    \mc{1}{l}{\scriptsize{Self-reported Health}} & \mc{1}{c}{\scriptsize{30}} & \mc{1}{c}{\scriptsize{0.350}} & \mc{1}{c}{\scriptsize{0.158}} & \mc{1}{c}{\scriptsize{-0.750}} & \mc{1}{c}{\scriptsize{-1.315}} & \mc{1}{c}{\scriptsize{-0.757}} & \mc{1}{c}{\scriptsize{0.625}} & \mc{1}{c}{\scriptsize{0.554}} & \mc{1}{c}{\scriptsize{0.498}} \\  

     &  & \mc{1}{c}{\scriptsize{(0.706)}} & \mc{1}{c}{\scriptsize{(0.216)}} & \mc{1}{c}{\scriptsize{(0.922)}} & \mc{1}{c}{\scriptsize{(0.745)}} & \mc{1}{c}{\scriptsize{(0.902)}} & \mc{1}{c}{\scriptsize{(0.980)}} & \mc{1}{c}{\scriptsize{(0.745)}} & \mc{1}{c}{\scriptsize{(0.824)}} \\  

     & \mc{1}{c}{\scriptsize{Mid-30s}} & \mc{1}{c}{\scriptsize{-0.725}} & \mc{1}{c}{\scriptsize{-0.435}} & \mc{1}{c}{\scriptsize{-1.625}} & \mc{1}{c}{\scriptsize{-1.421}} &  & \mc{1}{c}{\scriptsize{-0.125}} & \mc{1}{c}{\scriptsize{0.305}} &  \\  

     &  & \mc{1}{c}{\scriptsize{(0.765)}} & \mc{1}{c}{\scriptsize{(0.176)}} & \mc{1}{c}{\scriptsize{(0.922)}} & \mc{1}{c}{\scriptsize{(0.314)}} &  & \mc{1}{c}{\scriptsize{(0.176)}} & \mc{1}{c}{\scriptsize{\textbf{(0.078)}}} &  \\ 
    \hline  

    \\[0.1cm]
    \mc{2}{l}{\scriptsize{\% of Sig. TE ($H_0$: $\le$ 25\% $|$ 10\% Significance)}} & \mc{1}{c}{\scriptsize{0}} & \mc{1}{c}{\scriptsize{0}} & \mc{1}{c}{\scriptsize{0}} & \mc{1}{c}{\scriptsize{0}} & \mc{1}{c}{\scriptsize{0}} & \mc{1}{c}{\scriptsize{0}} & \mc{1}{c}{\scriptsize{50}} & \mc{1}{c}{\scriptsize{0}} \\  

     &  & \mc{1}{c}{\scriptsize{(1.000)}} & \mc{1}{c}{\scriptsize{(1.000)}} & \mc{1}{c}{\scriptsize{(0.686)}} & \mc{1}{c}{\scriptsize{(0.627)}} & \mc{1}{c}{\scriptsize{(0.647)}} & \mc{1}{c}{\scriptsize{(1.000)}} & \mc{1}{c}{\scriptsize{(0.118)}} & \mc{1}{c}{\scriptsize{(1.000)}} \\  

    \mc{2}{l}{\scriptsize{\% of Sig. TE ($H_0$: $\le$ 50\% $|$ 10\% Significance)}} & \mc{1}{c}{\scriptsize{0}} & \mc{1}{c}{\scriptsize{0}} & \mc{1}{c}{\scriptsize{0}} & \mc{1}{c}{\scriptsize{0}} & \mc{1}{c}{\scriptsize{0}} & \mc{1}{c}{\scriptsize{0}} & \mc{1}{c}{\scriptsize{50}} & \mc{1}{c}{\scriptsize{0}} \\  

     &  & \mc{1}{c}{\scriptsize{(1.000)}} & \mc{1}{c}{\scriptsize{(1.000)}} & \mc{1}{c}{\scriptsize{(0.686)}} & \mc{1}{c}{\scriptsize{(0.627)}} & \mc{1}{c}{\scriptsize{(0.647)}} & \mc{1}{c}{\scriptsize{(1.000)}} & \mc{1}{c}{\scriptsize{(0.118)}} & \mc{1}{c}{\scriptsize{(1.000)}} \\  

    \mc{2}{l}{\scriptsize{\% of Sig. TE ($H_0$: $\le$ 75\% $|$ 10\% Significance)}} & \mc{1}{c}{\scriptsize{0}} & \mc{1}{c}{\scriptsize{0}} & \mc{1}{c}{\scriptsize{0}} & \mc{1}{c}{\scriptsize{0}} & \mc{1}{c}{\scriptsize{0}} & \mc{1}{c}{\scriptsize{0}} & \mc{1}{c}{\scriptsize{50}} & \mc{1}{c}{\scriptsize{0}} \\  

     &  & \mc{1}{c}{\scriptsize{(1.000)}} & \mc{1}{c}{\scriptsize{(1.000)}} & \mc{1}{c}{\scriptsize{(0.686)}} & \mc{1}{c}{\scriptsize{(0.627)}} & \mc{1}{c}{\scriptsize{(0.647)}} & \mc{1}{c}{\scriptsize{(1.000)}} & \mc{1}{c}{\scriptsize{(1.000)}} & \mc{1}{c}{\scriptsize{(1.000)}} \\  

  \hline\hline
  \end{tabular}
    \begin{tablenotes}
    \scriptsize
    \item 
Note: This table displays various estimates of the treatment effect of CARE's family education program.
Column (1) displays the ITT, without accounting for any controls.
Column (2) displays the ITT conditioning on vector of controls, $X$, consisting of APGAR scores 1 
minute after birth, an indicator for the subject being born prematurely, and an indicator for the 
father being home at baseline. We also apply IPW weights, $W$, to account for attrition.
Columns (3)--(4) are analogous to columns (1)--(2), but we restrict the control sample to subjects
who did not enroll in any alternative care.
Column (5) displys the matching estimate, where we use the Mahalanobis metric and Epanechnikov kernel
to match on controls $X$ listed above, and restrict the control sample to subjects who did not enroll
in any alternative care. Additionally, we apply IPW weights, $W$.
Columns (6)--(8) are analogous to Columns (3)--(5), except we restrict the control sample to subejcts
who did enroll in alternative care. 
The final three pairs of rows display the proportion of treatment effects in the table that are 
socially positive. The first row in each pair displays the percentage of treatment effects, and the
second row presents the inference.

Numbers in parentheses represent the $p$-value from a single hypothesis test, and are obtained from 
the empirical bootstrap distribution generated by 200 resamples of the original data. 
Bold $p$-values indicate significance at the 10\% level.
Blank point estimates indicate that we are unable to obtain estimates due to a lack of support in the data. 

    \end{tablenotes}
  \end{threeparttable}

\end{table}
\end{center}

\begin{center}
	  \begin{tabular}{cccccccccc}
  \toprule

    \scriptsize{Variable} & \scriptsize{Age} & \scriptsize{(1)} & \scriptsize{(2)} & \scriptsize{(3)} & \scriptsize{(4)} & \scriptsize{(5)} & \scriptsize{(6)} & \scriptsize{(7)} & \scriptsize{(8)} \\ 
    \midrule  

    \mc{1}{l}{\scriptsize{Systolic Blood Pressure (mm Hg)}} & \mc{1}{c}{\scriptsize{Mid-30s}} & \mc{1}{c}{\scriptsize{-9.791}} & \mc{1}{c}{\scriptsize{-13.511}} & \mc{1}{c}{\scriptsize{15.280}} & \mc{1}{c}{\scriptsize{19.304}} & \mc{1}{c}{\scriptsize{14.979}} & \mc{1}{c}{\scriptsize{-19.920}} & \mc{1}{c}{\scriptsize{-23.674}} & \mc{1}{c}{\scriptsize{-18.537}} \\  

     &  & \mc{1}{c}{\scriptsize{(0.112)}} & \mc{1}{c}{\scriptsize{\textbf{(0.074)}}} & \mc{1}{c}{\scriptsize{\textbf{(0.021)}}} & \mc{1}{c}{\scriptsize{\textbf{(0.034)}}} & \mc{1}{c}{\scriptsize{\textbf{(0.000)}}} & \mc{1}{c}{\scriptsize{\textbf{(0.029)}}} & \mc{1}{c}{\scriptsize{\textbf{(0.003)}}} & \mc{1}{c}{\scriptsize{\textbf{(0.019)}}} \\  

    \mc{1}{l}{\scriptsize{Diastolic Blood Pressure (mm Hg)}} & \mc{1}{c}{\scriptsize{Mid-30s}} & \mc{1}{c}{\scriptsize{-10.854}} & \mc{1}{c}{\scriptsize{-16.689}} & \mc{1}{c}{\scriptsize{-8.640}} & \mc{1}{c}{\scriptsize{-11.320}} & \mc{1}{c}{\scriptsize{-8.741}} & \mc{1}{c}{\scriptsize{-14.240}} & \mc{1}{c}{\scriptsize{-19.311}} & \mc{1}{c}{\scriptsize{-13.988}} \\  

     &  & \mc{1}{c}{\scriptsize{\textbf{(0.046)}}} & \mc{1}{c}{\scriptsize{\textbf{(0.000)}}} & \mc{1}{c}{\scriptsize{(0.968)}} & \mc{1}{c}{\scriptsize{\textbf{(0.084)}}} & \mc{1}{c}{\scriptsize{\textbf{(0.020)}}} & \mc{1}{c}{\scriptsize{\textbf{(0.038)}}} & \mc{1}{c}{\scriptsize{\textbf{(0.000)}}} & \mc{1}{c}{\scriptsize{\textbf{(0.018)}}} \\  

    \mc{1}{l}{\scriptsize{Prehypertension}} & \mc{1}{c}{\scriptsize{Mid-30s}} & \mc{1}{c}{\scriptsize{-0.137}} & \mc{1}{c}{\scriptsize{-0.156}} & \mc{1}{c}{\scriptsize{0.053}} & \mc{1}{c}{\scriptsize{0.176}} & \mc{1}{c}{\scriptsize{0.077}} & \mc{1}{c}{\scriptsize{-0.280}} & \mc{1}{c}{\scriptsize{-0.293}} & \mc{1}{c}{\scriptsize{-0.283}} \\  

     &  & \mc{1}{c}{\scriptsize{(0.133)}} & \mc{1}{c}{\scriptsize{(0.146)}} & \mc{1}{c}{\scriptsize{\textbf{(0.022)}}} & \mc{1}{c}{\scriptsize{(0.234)}} & \mc{1}{c}{\scriptsize{(0.371)}} & \mc{1}{c}{\scriptsize{\textbf{(0.004)}}} & \mc{1}{c}{\scriptsize{\textbf{(0.041)}}} & \mc{1}{c}{\scriptsize{\textbf{(0.002)}}} \\  

    \mc{1}{l}{\scriptsize{Hypertension}} & \mc{1}{c}{\scriptsize{Mid-30s}} & \mc{1}{c}{\scriptsize{-0.291}} & \mc{1}{c}{\scriptsize{-0.352}} & \mc{1}{c}{\scriptsize{-0.053}} & \mc{1}{c}{\scriptsize{0.020}} & \mc{1}{c}{\scriptsize{-0.075}} & \mc{1}{c}{\scriptsize{-0.420}} & \mc{1}{c}{\scriptsize{-0.470}} & \mc{1}{c}{\scriptsize{-0.435}} \\  

     &  & \mc{1}{c}{\scriptsize{\textbf{(0.039)}}} & \mc{1}{c}{\scriptsize{\textbf{(0.050)}}} & \mc{1}{c}{\scriptsize{(0.967)}} & \mc{1}{c}{\scriptsize{(0.412)}} & \mc{1}{c}{\scriptsize{(0.371)}} & \mc{1}{c}{\scriptsize{\textbf{(0.003)}}} & \mc{1}{c}{\scriptsize{\textbf{(0.012)}}} & \mc{1}{c}{\scriptsize{\textbf{(0.004)}}} \\  

    \mc{1}{l}{\scriptsize{Hypertension Factor}} & \mc{1}{c}{\scriptsize{Mid-30s}} & \mc{1}{c}{\scriptsize{-0.643}} & \mc{1}{c}{\scriptsize{-0.920}} & \mc{1}{c}{\scriptsize{0.070}} & \mc{1}{c}{\scriptsize{0.146}} & \mc{1}{c}{\scriptsize{-0.025}} & \mc{1}{c}{\scriptsize{-1.044}} & \mc{1}{c}{\scriptsize{-1.315}} & \mc{1}{c}{\scriptsize{-1.140}} \\  

     &  & \mc{1}{c}{\scriptsize{\textbf{(0.031)}}} & \mc{1}{c}{\scriptsize{\textbf{(0.013)}}} & \mc{1}{c}{\scriptsize{\textbf{(0.020)}}} & \mc{1}{c}{\scriptsize{\textbf{(0.020)}}} & \mc{1}{c}{\scriptsize{(0.446)}} & \mc{1}{c}{\scriptsize{\textbf{(0.003)}}} & \mc{1}{c}{\scriptsize{\textbf{(0.001)}}} & \mc{1}{c}{\scriptsize{\textbf{(0.002)}}} \\  

  \bottomrule
  \end{tabular}
\end{center}

\begin{center}
	  \begin{tabular}{cccccccccc}
  \toprule

    \scriptsize{Variable} & \scriptsize{Age} & \scriptsize{(1)} & \scriptsize{(2)} & \scriptsize{(3)} & \scriptsize{(4)} & \scriptsize{(5)} & \scriptsize{(6)} & \scriptsize{(7)} & \scriptsize{(8)} \\ 
    \midrule  

    \mc{1}{l}{\scriptsize{Mother Works Factor}} & \mc{1}{c}{\scriptsize{2 to 21}} & \mc{1}{c}{\scriptsize{0.457}} & \mc{1}{c}{\scriptsize{0.517}} & \mc{1}{c}{\scriptsize{1.014}} & \mc{1}{c}{\scriptsize{0.987}} & \mc{1}{c}{\scriptsize{0.963}} & \mc{1}{c}{\scriptsize{0.297}} & \mc{1}{c}{\scriptsize{0.352}} & \mc{1}{c}{\scriptsize{0.280}} \\  

     &  & \mc{1}{c}{\scriptsize{\textbf{(0.000)}}} & \mc{1}{c}{\scriptsize{\textbf{(0.026)}}} & \mc{1}{c}{\scriptsize{\textbf{(0.000)}}} & \mc{1}{c}{\scriptsize{\textbf{(0.053)}}} & \mc{1}{c}{\scriptsize{\textbf{(0.000)}}} & \mc{1}{c}{\scriptsize{\textbf{(0.066)}}} & \mc{1}{c}{\scriptsize{(0.105)}} & \mc{1}{c}{\scriptsize{(0.145)}} \\  

    \mc{1}{l}{\scriptsize{Mother Works}} & \mc{1}{c}{\scriptsize{2}} & \mc{1}{c}{\scriptsize{0.124}} & \mc{1}{c}{\scriptsize{0.130}} & \mc{1}{c}{\scriptsize{0.197}} & \mc{1}{c}{\scriptsize{0.178}} & \mc{1}{c}{\scriptsize{0.174}} & \mc{1}{c}{\scriptsize{0.099}} & \mc{1}{c}{\scriptsize{0.103}} & \mc{1}{c}{\scriptsize{0.088}} \\  

     &  & \mc{1}{c}{\scriptsize{(0.118)}} & \mc{1}{c}{\scriptsize{(0.145)}} & \mc{1}{c}{\scriptsize{(0.132)}} & \mc{1}{c}{\scriptsize{(0.171)}} & \mc{1}{c}{\scriptsize{(0.132)}} & \mc{1}{c}{\scriptsize{(0.145)}} & \mc{1}{c}{\scriptsize{(0.197)}} & \mc{1}{c}{\scriptsize{(0.171)}} \\  

     & \mc{1}{c}{\scriptsize{3}} & \mc{1}{c}{\scriptsize{0.192}} & \mc{1}{c}{\scriptsize{0.196}} & \mc{1}{c}{\scriptsize{0.347}} & \mc{1}{c}{\scriptsize{0.329}} & \mc{1}{c}{\scriptsize{0.328}} & \mc{1}{c}{\scriptsize{0.136}} & \mc{1}{c}{\scriptsize{0.134}} & \mc{1}{c}{\scriptsize{0.131}} \\  

     &  & \mc{1}{c}{\scriptsize{\textbf{(0.013)}}} & \mc{1}{c}{\scriptsize{\textbf{(0.079)}}} & \mc{1}{c}{\scriptsize{\textbf{(0.013)}}} & \mc{1}{c}{\scriptsize{\textbf{(0.039)}}} & \mc{1}{c}{\scriptsize{\textbf{(0.026)}}} & \mc{1}{c}{\scriptsize{\textbf{(0.066)}}} & \mc{1}{c}{\scriptsize{(0.145)}} & \mc{1}{c}{\scriptsize{(0.105)}} \\  

     & \mc{1}{c}{\scriptsize{4}} & \mc{1}{c}{\scriptsize{0.172}} & \mc{1}{c}{\scriptsize{0.235}} & \mc{1}{c}{\scriptsize{0.412}} & \mc{1}{c}{\scriptsize{0.431}} & \mc{1}{c}{\scriptsize{0.397}} & \mc{1}{c}{\scriptsize{0.096}} & \mc{1}{c}{\scriptsize{0.155}} & \mc{1}{c}{\scriptsize{0.091}} \\  

     &  & \mc{1}{c}{\scriptsize{\textbf{(0.053)}}} & \mc{1}{c}{\scriptsize{\textbf{(0.026)}}} & \mc{1}{c}{\scriptsize{\textbf{(0.000)}}} & \mc{1}{c}{\scriptsize{\textbf{(0.026)}}} & \mc{1}{c}{\scriptsize{\textbf{(0.000)}}} & \mc{1}{c}{\scriptsize{(0.132)}} & \mc{1}{c}{\scriptsize{\textbf{(0.079)}}} & \mc{1}{c}{\scriptsize{(0.171)}} \\  

     & \mc{1}{c}{\scriptsize{5}} & \mc{1}{c}{\scriptsize{0.112}} & \mc{1}{c}{\scriptsize{0.160}} & \mc{1}{c}{\scriptsize{0.282}} & \mc{1}{c}{\scriptsize{0.278}} & \mc{1}{c}{\scriptsize{0.262}} & \mc{1}{c}{\scriptsize{0.067}} & \mc{1}{c}{\scriptsize{0.126}} & \mc{1}{c}{\scriptsize{0.057}} \\  

     &  & \mc{1}{c}{\scriptsize{(0.118)}} & \mc{1}{c}{\scriptsize{\textbf{(0.079)}}} & \mc{1}{c}{\scriptsize{\textbf{(0.066)}}} & \mc{1}{c}{\scriptsize{(0.105)}} & \mc{1}{c}{\scriptsize{\textbf{(0.092)}}} & \mc{1}{c}{\scriptsize{(0.211)}} & \mc{1}{c}{\scriptsize{(0.132)}} & \mc{1}{c}{\scriptsize{(0.289)}} \\ 
    \midrule  

    \mc{2}{l}{\scriptsize{\% of Pos. TE ($H_0$: $\le$ 50\%)}} & \mc{1}{c}{\scriptsize{100}} & \mc{1}{c}{\scriptsize{100}} & \mc{1}{c}{\scriptsize{100}} & \mc{1}{c}{\scriptsize{100}} & \mc{1}{c}{\scriptsize{100}} & \mc{1}{c}{\scriptsize{100}} & \mc{1}{c}{\scriptsize{100}} & \mc{1}{c}{\scriptsize{100}} \\  

     &  & \mc{1}{c}{\scriptsize{\textbf{(0.000)}}} & \mc{1}{c}{\scriptsize{\textbf{(0.000)}}} & \mc{1}{c}{\scriptsize{\textbf{(0.000)}}} & \mc{1}{c}{\scriptsize{\textbf{(0.000)}}} & \mc{1}{c}{\scriptsize{\textbf{(0.000)}}} & \mc{1}{c}{\scriptsize{\textbf{(0.000)}}} & \mc{1}{c}{\scriptsize{\textbf{(0.000)}}} & \mc{1}{c}{\scriptsize{\textbf{(0.000)}}} \\  

    \mc{2}{l}{\scriptsize{\% of Pos. TE ($H_0$: $\le$ 10\% $|$ 10\% Significance)}} & \mc{1}{c}{\scriptsize{100}} & \mc{1}{c}{\scriptsize{80}} & \mc{1}{c}{\scriptsize{60}} & \mc{1}{c}{\scriptsize{60}} & \mc{1}{c}{\scriptsize{60}} & \mc{1}{c}{\scriptsize{40}} & \mc{1}{c}{\scriptsize{20}} & \mc{1}{c}{\scriptsize{20}} \\  

     &  & \mc{1}{c}{\scriptsize{\textbf{(0.000)}}} & \mc{1}{c}{\scriptsize{\textbf{(0.000)}}} & \mc{1}{c}{\scriptsize{\textbf{(0.000)}}} & \mc{1}{c}{\scriptsize{\textbf{(0.092)}}} & \mc{1}{c}{\scriptsize{\textbf{(0.000)}}} & \mc{1}{c}{\scriptsize{(0.289)}} & \mc{1}{c}{\scriptsize{(0.250)}} & \mc{1}{c}{\scriptsize{(0.329)}} \\  

  \bottomrule
  \end{tabular}
\end{center}

\begin{center}
	  \begin{tabular}{cccccccccc}
  \toprule

    \scriptsize{Variable} & \scriptsize{Age} & \scriptsize{(1)} & \scriptsize{(2)} & \scriptsize{(3)} & \scriptsize{(4)} & \scriptsize{(5)} & \scriptsize{(6)} & \scriptsize{(7)} & \scriptsize{(8)} \\ 
    \midrule  

    \mc{1}{l}{\scriptsize{Hemoglobin Level (\%)}} & \mc{1}{c}{\scriptsize{Mid-30s}} & \mc{1}{c}{\scriptsize{0.328}} & \mc{1}{c}{\scriptsize{0.173}} & \mc{1}{c}{\scriptsize{0.273}} & \mc{1}{c}{\scriptsize{0.704}} & \mc{1}{c}{\scriptsize{0.609}} & \mc{1}{c}{\scriptsize{0.307}} & \mc{1}{c}{\scriptsize{0.088}} & \mc{1}{c}{\scriptsize{0.637}} \\  

     &  & \mc{1}{c}{\scriptsize{(0.816)}} & \mc{1}{c}{\scriptsize{(0.697)}} & \mc{1}{c}{\scriptsize{(0.737)}} & \mc{1}{c}{\scriptsize{(0.737)}} & \mc{1}{c}{\scriptsize{(0.842)}} & \mc{1}{c}{\scriptsize{(0.803)}} & \mc{1}{c}{\scriptsize{(0.658)}} & \mc{1}{c}{\scriptsize{(0.842)}} \\  

    \mc{1}{l}{\scriptsize{Prediabetes}} & \mc{1}{c}{\scriptsize{Mid-30s}} & \mc{1}{c}{\scriptsize{-0.120}} & \mc{1}{c}{\scriptsize{-0.290}} & \mc{1}{c}{\scriptsize{-0.267}} & \mc{1}{c}{\scriptsize{-0.651}} & \mc{1}{c}{\scriptsize{-0.349}} & \mc{1}{c}{\scriptsize{-0.100}} & \mc{1}{c}{\scriptsize{-0.188}} & \mc{1}{c}{\scriptsize{-0.098}} \\  

     &  & \mc{1}{c}{\scriptsize{(0.158)}} & \mc{1}{c}{\scriptsize{\textbf{(0.039)}}} & \mc{1}{c}{\scriptsize{(0.105)}} & \mc{1}{c}{\scriptsize{\textbf{(0.026)}}} & \mc{1}{c}{\scriptsize{\textbf{(0.053)}}} & \mc{1}{c}{\scriptsize{(0.289)}} & \mc{1}{c}{\scriptsize{(0.158)}} & \mc{1}{c}{\scriptsize{(0.263)}} \\  

    \mc{1}{l}{\scriptsize{Diabetes}} & \mc{1}{c}{\scriptsize{Mid-30s}} & \mc{1}{c}{\scriptsize{0.080}} & \mc{1}{c}{\scriptsize{0.038}} & \mc{1}{c}{\scriptsize{0.080}} & \mc{1}{c}{\scriptsize{0.027}} & \mc{1}{c}{\scriptsize{0.140}} & \mc{1}{c}{\scriptsize{0.080}} & \mc{1}{c}{\scriptsize{0.033}} & \mc{1}{c}{\scriptsize{0.140}} \\  

     &  & \mc{1}{c}{\scriptsize{(0.776)}} & \mc{1}{c}{\scriptsize{(0.592)}} & \mc{1}{c}{\scriptsize{(0.776)}} & \mc{1}{c}{\scriptsize{(0.474)}} & \mc{1}{c}{\scriptsize{(0.803)}} & \mc{1}{c}{\scriptsize{(0.776)}} & \mc{1}{c}{\scriptsize{(0.592)}} & \mc{1}{c}{\scriptsize{(0.803)}} \\  

    \mc{1}{l}{\scriptsize{Diabetes Factor}} & \mc{1}{c}{\scriptsize{Mid-30s}} & \mc{1}{c}{\scriptsize{0.309}} & \mc{1}{c}{\scriptsize{0.133}} & \mc{1}{c}{\scriptsize{0.261}} & \mc{1}{c}{\scriptsize{0.393}} & \mc{1}{c}{\scriptsize{0.546}} & \mc{1}{c}{\scriptsize{0.298}} & \mc{1}{c}{\scriptsize{0.085}} & \mc{1}{c}{\scriptsize{0.589}} \\  

     &  & \mc{1}{c}{\scriptsize{(0.905)}} & \mc{1}{c}{\scriptsize{(0.608)}} & \mc{1}{c}{\scriptsize{(0.811)}} & \mc{1}{c}{\scriptsize{(0.676)}} & \mc{1}{c}{\scriptsize{(0.878)}} & \mc{1}{c}{\scriptsize{(0.865)}} & \mc{1}{c}{\scriptsize{(0.595)}} & \mc{1}{c}{\scriptsize{(0.905)}} \\ 
    \midrule  

    \mc{2}{l}{\scriptsize{\% of Pos. TE ($H_0$: $\le$ 50\%)}} & \mc{1}{c}{\scriptsize{25}} & \mc{1}{c}{\scriptsize{25}} & \mc{1}{c}{\scriptsize{25}} & \mc{1}{c}{\scriptsize{25}} & \mc{1}{c}{\scriptsize{25}} & \mc{1}{c}{\scriptsize{25}} & \mc{1}{c}{\scriptsize{25}} & \mc{1}{c}{\scriptsize{25}} \\  

     &  & \mc{1}{c}{\scriptsize{(0.829)}} & \mc{1}{c}{\scriptsize{(0.461)}} & \mc{1}{c}{\scriptsize{(0.895)}} & \mc{1}{c}{\scriptsize{(0.500)}} & \mc{1}{c}{\scriptsize{(0.947)}} & \mc{1}{c}{\scriptsize{(0.750)}} & \mc{1}{c}{\scriptsize{(0.592)}} & \mc{1}{c}{\scriptsize{(0.803)}} \\  

    \mc{2}{l}{\scriptsize{\% of Pos. TE ($H_0$: $\le$ 10\% $|$ 10\% Significance)}} & \mc{1}{c}{\scriptsize{0}} & \mc{1}{c}{\scriptsize{25}} & \mc{1}{c}{\scriptsize{0}} & \mc{1}{c}{\scriptsize{0}} & \mc{1}{c}{\scriptsize{25}} & \mc{1}{c}{\scriptsize{0}} & \mc{1}{c}{\scriptsize{0}} & \mc{1}{c}{\scriptsize{0}} \\  

     &  & \mc{1}{c}{\scriptsize{(1.000)}} & \mc{1}{c}{\scriptsize{\textbf{(0.092)}}} & \mc{1}{c}{\scriptsize{(0.447)}} & \mc{1}{c}{\scriptsize{(1.000)}} & \mc{1}{c}{\scriptsize{\textbf{(0.092)}}} & \mc{1}{c}{\scriptsize{(1.000)}} & \mc{1}{c}{\scriptsize{(1.000)}} & \mc{1}{c}{\scriptsize{(1.000)}} \\  

  \bottomrule
  \end{tabular}
\end{center}

\begin{center}
	  \begin{tabular}{cccccccccc}
  \toprule

    \scriptsize{Variable} & \scriptsize{Age} & \scriptsize{(1)} & \scriptsize{(2)} & \scriptsize{(3)} & \scriptsize{(4)} & \scriptsize{(5)} & \scriptsize{(6)} & \scriptsize{(7)} & \scriptsize{(8)} \\ 
    \midrule  

    \mc{1}{l}{\scriptsize{Measured BMI}} & \mc{1}{c}{\scriptsize{Mid-30s}} & \mc{1}{c}{\scriptsize{3.545}} & \mc{1}{c}{\scriptsize{6.421}} & \mc{1}{c}{\scriptsize{1.937}} & \mc{1}{c}{\scriptsize{2.932}} & \mc{1}{c}{\scriptsize{0.910}} & \mc{1}{c}{\scriptsize{3.983}} & \mc{1}{c}{\scriptsize{7.424}} & \mc{1}{c}{\scriptsize{-0.497}} \\  

     &  & \mc{1}{c}{\scriptsize{(0.118)}} & \mc{1}{c}{\scriptsize{\textbf{(0.026)}}} & \mc{1}{c}{\scriptsize{(0.998)}} & \mc{1}{c}{\scriptsize{(0.263)}} & \mc{1}{c}{\scriptsize{(0.431)}} & \mc{1}{c}{\scriptsize{\textbf{(0.096)}}} & \mc{1}{c}{\scriptsize{\textbf{(0.031)}}} & \mc{1}{c}{\scriptsize{(0.452)}} \\  

    \mc{1}{l}{\scriptsize{Obesity}} & \mc{1}{c}{\scriptsize{Mid-30s}} & \mc{1}{c}{\scriptsize{-0.011}} & \mc{1}{c}{\scriptsize{0.143}} & \mc{1}{c}{\scriptsize{-0.261}} &  & \mc{1}{c}{\scriptsize{0.034}} & \mc{1}{c}{\scriptsize{0.057}} & \mc{1}{c}{\scriptsize{0.209}} & \mc{1}{c}{\scriptsize{-0.061}} \\  

     &  & \mc{1}{c}{\scriptsize{(0.470)}} & \mc{1}{c}{\scriptsize{(0.140)}} & \mc{1}{c}{\scriptsize{\textbf{(0.001)}}} &  & \mc{1}{c}{\scriptsize{(0.464)}} & \mc{1}{c}{\scriptsize{(0.331)}} & \mc{1}{c}{\scriptsize{(0.107)}} & \mc{1}{c}{\scriptsize{(0.387)}} \\  

    \mc{1}{l}{\scriptsize{Severe Obesity}} & \mc{1}{c}{\scriptsize{Mid-30s}} & \mc{1}{c}{\scriptsize{-0.045}} & \mc{1}{c}{\scriptsize{0.087}} & \mc{1}{c}{\scriptsize{0.014}} & \mc{1}{c}{\scriptsize{0.134}} & \mc{1}{c}{\scriptsize{-0.125}} & \mc{1}{c}{\scriptsize{-0.061}} & \mc{1}{c}{\scriptsize{0.076}} & \mc{1}{c}{\scriptsize{-0.131}} \\  

     &  & \mc{1}{c}{\scriptsize{(0.365)}} & \mc{1}{c}{\scriptsize{(0.292)}} & \mc{1}{c}{\scriptsize{(0.997)}} & \mc{1}{c}{\scriptsize{(0.302)}} & \mc{1}{c}{\scriptsize{(0.280)}} & \mc{1}{c}{\scriptsize{(0.338)}} & \mc{1}{c}{\scriptsize{(0.313)}} & \mc{1}{c}{\scriptsize{(0.215)}} \\  

    \mc{1}{l}{\scriptsize{Waist-hip Ratio}} & \mc{1}{c}{\scriptsize{Mid-30s}} & \mc{1}{c}{\scriptsize{-0.022}} &  & \mc{1}{c}{\scriptsize{-0.076}} & \mc{1}{c}{\scriptsize{-0.052}} & \mc{1}{c}{\scriptsize{0.022}} & \mc{1}{c}{\scriptsize{-0.007}} & \mc{1}{c}{\scriptsize{0.042}} & \mc{1}{c}{\scriptsize{-0.006}} \\  

     &  & \mc{1}{c}{\scriptsize{(0.225)}} &  & \mc{1}{c}{\scriptsize{\textbf{(0.002)}}} & \mc{1}{c}{\scriptsize{(0.306)}} & \mc{1}{c}{\scriptsize{(0.347)}} & \mc{1}{c}{\scriptsize{(0.416)}} & \mc{1}{c}{\scriptsize{(0.143)}} & \mc{1}{c}{\scriptsize{(0.436)}} \\  

    \mc{1}{l}{\scriptsize{Abdominal Obesity}} & \mc{1}{c}{\scriptsize{Mid-30s}} & \mc{1}{c}{\scriptsize{-0.159}} & \mc{1}{c}{\scriptsize{0.030}} & \mc{1}{c}{\scriptsize{-0.381}} & \mc{1}{c}{\scriptsize{-0.101}} & \mc{1}{c}{\scriptsize{0.047}} & \mc{1}{c}{\scriptsize{-0.095}} & \mc{1}{c}{\scriptsize{0.076}} & \mc{1}{c}{\scriptsize{-0.022}} \\  

     &  & \mc{1}{c}{\scriptsize{(0.120)}} & \mc{1}{c}{\scriptsize{(0.412)}} & \mc{1}{c}{\scriptsize{\textbf{(0.002)}}} & \mc{1}{c}{\scriptsize{(0.222)}} & \mc{1}{c}{\scriptsize{(0.409)}} & \mc{1}{c}{\scriptsize{(0.263)}} & \mc{1}{c}{\scriptsize{(0.330)}} & \mc{1}{c}{\scriptsize{(0.462)}} \\  

    \mc{1}{l}{\scriptsize{Framingham Risk Score}} & \mc{1}{c}{\scriptsize{Mid-30s}} & \mc{1}{c}{\scriptsize{-0.259}} & \mc{1}{c}{\scriptsize{-0.260}} & \mc{1}{c}{\scriptsize{-0.488}} & \mc{1}{c}{\scriptsize{-0.545}} & \mc{1}{c}{\scriptsize{1.813}} & \mc{1}{c}{\scriptsize{-0.197}} & \mc{1}{c}{\scriptsize{-0.105}} & \mc{1}{c}{\scriptsize{-0.704}} \\  

     &  & \mc{1}{c}{\scriptsize{(0.118)}} & \mc{1}{c}{\scriptsize{(0.132)}} & \mc{1}{c}{\scriptsize{\textbf{(0.001)}}} & \mc{1}{c}{\scriptsize{(0.120)}} & \mc{1}{c}{\scriptsize{(0.124)}} & \mc{1}{c}{\scriptsize{(0.198)}} & \mc{1}{c}{\scriptsize{(0.308)}} & \mc{1}{c}{\scriptsize{(0.308)}} \\  

    \mc{1}{l}{\scriptsize{Obesity Factor}} & \mc{1}{c}{\scriptsize{Mid-30s}} & \mc{1}{c}{\scriptsize{-0.006}} & \mc{1}{c}{\scriptsize{-0.411}} & \mc{1}{c}{\scriptsize{0.433}} & \mc{1}{c}{\scriptsize{0.132}} & \mc{1}{c}{\scriptsize{0.085}} & \mc{1}{c}{\scriptsize{-0.132}} & \mc{1}{c}{\scriptsize{-0.611}} & \mc{1}{c}{\scriptsize{0.145}} \\  

     &  & \mc{1}{c}{\scriptsize{(0.492)}} & \mc{1}{c}{\scriptsize{(0.221)}} & \mc{1}{c}{\scriptsize{(0.998)}} & \mc{1}{c}{\scriptsize{(0.997)}} & \mc{1}{c}{\scriptsize{(0.426)}} & \mc{1}{c}{\scriptsize{(0.348)}} & \mc{1}{c}{\scriptsize{(0.214)}} & \mc{1}{c}{\scriptsize{(0.384)}} \\  

  \bottomrule
  \end{tabular}
\end{center}

\begin{center}
	  \begin{tabular}{cccccccccc}
  \toprule

    \scriptsize{Variable} & \scriptsize{Age} & \scriptsize{(1)} & \scriptsize{(2)} & \scriptsize{(3)} & \scriptsize{(4)} & \scriptsize{(5)} & \scriptsize{(6)} & \scriptsize{(7)} & \scriptsize{(8)} \\ 
    \midrule  

    \mc{1}{l}{\scriptsize{Measured BMI}} & \mc{1}{c}{\scriptsize{Mid-30s}} & \mc{1}{c}{\scriptsize{-0.125}} & \mc{1}{c}{\scriptsize{-0.617}} & \mc{1}{c}{\scriptsize{-0.684}} & \mc{1}{c}{\scriptsize{-3.505}} & \mc{1}{c}{\scriptsize{0.867}} & \mc{1}{c}{\scriptsize{-0.627}} & \mc{1}{c}{\scriptsize{0.025}} & \mc{1}{c}{\scriptsize{-0.487}} \\  

     &  & \mc{1}{c}{\scriptsize{(0.500)}} & \mc{1}{c}{\scriptsize{(0.408)}} & \mc{1}{c}{\scriptsize{(0.382)}} & \mc{1}{c}{\scriptsize{(0.303)}} & \mc{1}{c}{\scriptsize{(0.447)}} & \mc{1}{c}{\scriptsize{(0.447)}} & \mc{1}{c}{\scriptsize{(0.539)}} & \mc{1}{c}{\scriptsize{(0.421)}} \\  

    \mc{1}{l}{\scriptsize{Obesity}} & \mc{1}{c}{\scriptsize{Mid-30s}} &  & \mc{1}{c}{\scriptsize{-0.103}} & \mc{1}{c}{\scriptsize{-0.128}} & \mc{1}{c}{\scriptsize{-0.331}} & \mc{1}{c}{\scriptsize{0.031}} & \mc{1}{c}{\scriptsize{-0.017}} & \mc{1}{c}{\scriptsize{-0.045}} & \mc{1}{c}{\scriptsize{-0.060}} \\  

     &  &  & \mc{1}{c}{\scriptsize{(0.303)}} & \mc{1}{c}{\scriptsize{(0.329)}} & \mc{1}{c}{\scriptsize{\textbf{(0.079)}}} & \mc{1}{c}{\scriptsize{(0.487)}} & \mc{1}{c}{\scriptsize{(0.487)}} & \mc{1}{c}{\scriptsize{(0.395)}} & \mc{1}{c}{\scriptsize{(0.395)}} \\  

    \mc{1}{l}{\scriptsize{Severe Obesity}} & \mc{1}{c}{\scriptsize{Mid-30s}} & \mc{1}{c}{\scriptsize{-0.160}} & \mc{1}{c}{\scriptsize{-0.132}} & \mc{1}{c}{\scriptsize{-0.185}} & \mc{1}{c}{\scriptsize{-0.319}} & \mc{1}{c}{\scriptsize{-0.126}} & \mc{1}{c}{\scriptsize{-0.185}} & \mc{1}{c}{\scriptsize{-0.108}} & \mc{1}{c}{\scriptsize{-0.131}} \\  

     &  & \mc{1}{c}{\scriptsize{(0.132)}} & \mc{1}{c}{\scriptsize{(0.211)}} & \mc{1}{c}{\scriptsize{(0.224)}} & \mc{1}{c}{\scriptsize{(0.145)}} & \mc{1}{c}{\scriptsize{(0.276)}} & \mc{1}{c}{\scriptsize{(0.171)}} & \mc{1}{c}{\scriptsize{(0.289)}} & \mc{1}{c}{\scriptsize{(0.250)}} \\  

    \mc{1}{l}{\scriptsize{Waist-hip Ratio}} & \mc{1}{c}{\scriptsize{Mid-30s}} & \mc{1}{c}{\scriptsize{0.004}} & \mc{1}{c}{\scriptsize{-0.014}} & \mc{1}{c}{\scriptsize{0.018}} & \mc{1}{c}{\scriptsize{-0.043}} & \mc{1}{c}{\scriptsize{0.030}} & \mc{1}{c}{\scriptsize{-0.002}} & \mc{1}{c}{\scriptsize{-0.008}} & \mc{1}{c}{\scriptsize{-0.004}} \\  

     &  & \mc{1}{c}{\scriptsize{(0.553)}} & \mc{1}{c}{\scriptsize{(0.237)}} & \mc{1}{c}{\scriptsize{(0.526)}} & \mc{1}{c}{\scriptsize{(0.171)}} & \mc{1}{c}{\scriptsize{(0.553)}} & \mc{1}{c}{\scriptsize{(0.461)}} & \mc{1}{c}{\scriptsize{(0.329)}} & \mc{1}{c}{\scriptsize{(0.421)}} \\  

    \mc{1}{l}{\scriptsize{Abdominal Obesity}} & \mc{1}{c}{\scriptsize{Mid-30s}} & \mc{1}{c}{\scriptsize{0.003}} & \mc{1}{c}{\scriptsize{-0.137}} & \mc{1}{c}{\scriptsize{0.029}} & \mc{1}{c}{\scriptsize{-0.377}} & \mc{1}{c}{\scriptsize{0.100}} & \mc{1}{c}{\scriptsize{0.029}} & \mc{1}{c}{\scriptsize{-0.062}} & \mc{1}{c}{\scriptsize{-0.031}} \\  

     &  & \mc{1}{c}{\scriptsize{(0.461)}} & \mc{1}{c}{\scriptsize{(0.211)}} & \mc{1}{c}{\scriptsize{(0.487)}} & \mc{1}{c}{\scriptsize{(0.171)}} & \mc{1}{c}{\scriptsize{(0.566)}} & \mc{1}{c}{\scriptsize{(0.605)}} & \mc{1}{c}{\scriptsize{(0.382)}} & \mc{1}{c}{\scriptsize{(0.421)}} \\  

    \mc{1}{l}{\scriptsize{Framingham Risk Score}} & \mc{1}{c}{\scriptsize{Mid-30s}} & \mc{1}{c}{\scriptsize{-0.766}} & \mc{1}{c}{\scriptsize{-0.472}} & \mc{1}{c}{\scriptsize{1.491}} & \mc{1}{c}{\scriptsize{2.437}} & \mc{1}{c}{\scriptsize{1.802}} & \mc{1}{c}{\scriptsize{-1.202}} & \mc{1}{c}{\scriptsize{-1.036}} & \mc{1}{c}{\scriptsize{-0.705}} \\  

     &  & \mc{1}{c}{\scriptsize{(0.224)}} & \mc{1}{c}{\scriptsize{(0.368)}} & \mc{1}{c}{\scriptsize{(0.737)}} & \mc{1}{c}{\scriptsize{(0.684)}} & \mc{1}{c}{\scriptsize{(0.816)}} & \mc{1}{c}{\scriptsize{(0.184)}} & \mc{1}{c}{\scriptsize{(0.250)}} & \mc{1}{c}{\scriptsize{(0.316)}} \\  

    \mc{1}{l}{\scriptsize{Obesity Factor}} & \mc{1}{c}{\scriptsize{Mid-30s}} & \mc{1}{c}{\scriptsize{-0.040}} & \mc{1}{c}{\scriptsize{-0.197}} & \mc{1}{c}{\scriptsize{-0.036}} & \mc{1}{c}{\scriptsize{-0.524}} & \mc{1}{c}{\scriptsize{0.211}} & \mc{1}{c}{\scriptsize{-0.101}} & \mc{1}{c}{\scriptsize{-0.091}} & \mc{1}{c}{\scriptsize{-0.116}} \\  

     &  & \mc{1}{c}{\scriptsize{(0.461)}} & \mc{1}{c}{\scriptsize{(0.263)}} & \mc{1}{c}{\scriptsize{(0.408)}} & \mc{1}{c}{\scriptsize{(0.211)}} & \mc{1}{c}{\scriptsize{(0.474)}} & \mc{1}{c}{\scriptsize{(0.395)}} & \mc{1}{c}{\scriptsize{(0.434)}} & \mc{1}{c}{\scriptsize{(0.355)}} \\ 
    \midrule  

    \mc{2}{l}{\scriptsize{\% of Pos. TE ($H_0$: $\le$ 50\%)}} & \mc{1}{c}{\scriptsize{67}} & \mc{1}{c}{\scriptsize{100}} & \mc{1}{c}{\scriptsize{57}} & \mc{1}{c}{\scriptsize{86}} & \mc{1}{c}{\scriptsize{14}} & \mc{1}{c}{\scriptsize{86}} & \mc{1}{c}{\scriptsize{86}} & \mc{1}{c}{\scriptsize{100}} \\  

     &  & \mc{1}{c}{\scriptsize{(0.395)}} & \mc{1}{c}{\scriptsize{\textbf{(0.000)}}} & \mc{1}{c}{\scriptsize{(0.421)}} & \mc{1}{c}{\scriptsize{\textbf{(0.000)}}} & \mc{1}{c}{\scriptsize{(0.737)}} & \mc{1}{c}{\scriptsize{\textbf{(0.000)}}} & \mc{1}{c}{\scriptsize{\textbf{(0.000)}}} & \mc{1}{c}{\scriptsize{\textbf{(0.000)}}} \\  

    \mc{2}{l}{\scriptsize{\% of Pos. TE ($H_0$: $\le$ 10\% $|$ 10\% Significance)}} & \mc{1}{c}{\scriptsize{17}} & \mc{1}{c}{\scriptsize{0}} & \mc{1}{c}{\scriptsize{0}} & \mc{1}{c}{\scriptsize{0}} & \mc{1}{c}{\scriptsize{0}} & \mc{1}{c}{\scriptsize{14}} & \mc{1}{c}{\scriptsize{0}} & \mc{1}{c}{\scriptsize{0}} \\  

     &  & \mc{1}{c}{\scriptsize{(0.145)}} & \mc{1}{c}{\scriptsize{(0.395)}} & \mc{1}{c}{\scriptsize{(0.329)}} & \mc{1}{c}{\scriptsize{(0.500)}} & \mc{1}{c}{\scriptsize{(0.342)}} & \mc{1}{c}{\scriptsize{(0.184)}} & \mc{1}{c}{\scriptsize{(1.000)}} & \mc{1}{c}{\scriptsize{(0.368)}} \\  

  \bottomrule
  \end{tabular}
\end{center}

\begin{center}
	  \begin{tabular}{cccccccccc}
  \toprule

    \scriptsize{Variable} & \scriptsize{Age} & \scriptsize{(1)} & \scriptsize{(2)} & \scriptsize{(3)} & \scriptsize{(4)} & \scriptsize{(5)} & \scriptsize{(6)} & \scriptsize{(7)} & \scriptsize{(8)} \\ 
    \midrule  

    \mc{1}{l}{\scriptsize{Vitamin D Deficiency}} & \mc{1}{c}{\scriptsize{Mid-30s}} & \mc{1}{c}{\scriptsize{-0.382}} & \mc{1}{c}{\scriptsize{-0.255}} & \mc{1}{c}{\scriptsize{-0.632}} & \mc{1}{c}{\scriptsize{-1.113}} & \mc{1}{c}{\scriptsize{-0.679}} & \mc{1}{c}{\scriptsize{-0.332}} & \mc{1}{c}{\scriptsize{-0.139}} & \mc{1}{c}{\scriptsize{-0.307}} \\  

     &  & \mc{1}{c}{\scriptsize{\textbf{(0.000)}}} & \mc{1}{c}{\scriptsize{(0.105)}} & \mc{1}{c}{\scriptsize{\textbf{(0.000)}}} & \mc{1}{c}{\scriptsize{\textbf{(0.000)}}} & \mc{1}{c}{\scriptsize{\textbf{(0.000)}}} & \mc{1}{c}{\scriptsize{\textbf{(0.066)}}} & \mc{1}{c}{\scriptsize{(0.263)}} & \mc{1}{c}{\scriptsize{\textbf{(0.079)}}} \\ 
    \midrule  

    \mc{2}{l}{\scriptsize{\% of Pos. TE ($H_0$: $\le$ 50\%)}} & \mc{1}{c}{\scriptsize{100}} & \mc{1}{c}{\scriptsize{100}} & \mc{1}{c}{\scriptsize{100}} & \mc{1}{c}{\scriptsize{100}} & \mc{1}{c}{\scriptsize{100}} & \mc{1}{c}{\scriptsize{100}} & \mc{1}{c}{\scriptsize{100}} & \mc{1}{c}{\scriptsize{100}} \\  

     &  & \mc{1}{c}{\scriptsize{\textbf{(0.000)}}} & \mc{1}{c}{\scriptsize{\textbf{(0.000)}}} & \mc{1}{c}{\scriptsize{\textbf{(0.000)}}} & \mc{1}{c}{\scriptsize{\textbf{(0.000)}}} & \mc{1}{c}{\scriptsize{\textbf{(0.000)}}} & \mc{1}{c}{\scriptsize{\textbf{(0.000)}}} & \mc{1}{c}{\scriptsize{\textbf{(0.000)}}} & \mc{1}{c}{\scriptsize{\textbf{(0.000)}}} \\  

    \mc{2}{l}{\scriptsize{\% of Pos. TE ($H_0$: $\le$ 10\% $|$ 10\% Significance)}} & \mc{1}{c}{\scriptsize{100}} & \mc{1}{c}{\scriptsize{0}} & \mc{1}{c}{\scriptsize{100}} & \mc{1}{c}{\scriptsize{100}} & \mc{1}{c}{\scriptsize{100}} & \mc{1}{c}{\scriptsize{100}} & \mc{1}{c}{\scriptsize{0}} & \mc{1}{c}{\scriptsize{100}} \\  

     &  & \mc{1}{c}{\scriptsize{\textbf{(0.000)}}} & \mc{1}{c}{\scriptsize{(0.184)}} & \mc{1}{c}{\scriptsize{\textbf{(0.000)}}} & \mc{1}{c}{\scriptsize{\textbf{(0.000)}}} & \mc{1}{c}{\scriptsize{\textbf{(0.000)}}} & \mc{1}{c}{\scriptsize{\textbf{(0.000)}}} & \mc{1}{c}{\scriptsize{(0.132)}} & \mc{1}{c}{\scriptsize{\textbf{(0.000)}}} \\  

  \bottomrule
  \end{tabular}
\end{center}

\begin{center}
	\begin{table}[H]
\captionsetup{singlelinecheck=false,justification=centering}
\caption{ABC Average Treatment Effects, Males \\ Obesity \label{tab:ate_male_apx17}}

  \begin{threeparttable}
  \begin{tabular}{cccccccccc}
  \hline\hline

     &  & \scriptsize{(1)} & \scriptsize{(2)} & \scriptsize{(3)} & \scriptsize{(4)} & \scriptsize{(5)} & \scriptsize{(6)} & \scriptsize{(7)} & \scriptsize{(8)} \\  

     &  &  &  & \mc{3}{c}{\scriptsize{$P=0$}} & \mc{3}{c}{\scriptsize{$P=1$}} \\ 
    \cmidrule(lr){5-7} \cmidrule(lr){8-10} 

    \scriptsize{Variable} & \scriptsize{Age} & \scriptsize{ITT} & \scriptsize{ITT$|X,W$} & \scriptsize{ITT} & \scriptsize{ITT$|X,W$} & \scriptsize{KE$|X,W$} & \scriptsize{ITT} & \scriptsize{ITT$|X,W$} & \scriptsize{KE$|X,W$} \\ 
    \hline  

    \mc{1}{l}{\scriptsize{Measured BMI}} & \mc{1}{c}{\scriptsize{Mid-30s}} & \mc{1}{c}{\scriptsize{-0.041}} & \mc{1}{c}{\scriptsize{-2.966}} & \mc{1}{c}{\scriptsize{4.219}} & \mc{1}{c}{\scriptsize{2.774}} &  & \mc{1}{c}{\scriptsize{-6.856}} & \mc{1}{c}{\scriptsize{-9.309}} & \mc{1}{c}{\scriptsize{-7.135}} \\  

     &  & \mc{1}{c}{\scriptsize{(0.529)}} & \mc{1}{c}{\scriptsize{(0.216)}} & \mc{1}{c}{\scriptsize{(0.941)}} & \mc{1}{c}{\scriptsize{(0.765)}} &  & \mc{1}{c}{\scriptsize{\textbf{(0.059)}}} & \mc{1}{c}{\scriptsize{\textbf{(0.078)}}} & \mc{1}{c}{\scriptsize{\textbf{(0.059)}}} \\  

    \mc{1}{l}{\scriptsize{Obesity}} & \mc{1}{c}{\scriptsize{Mid-30s}} & \mc{1}{c}{\scriptsize{0.238}} & \mc{1}{c}{\scriptsize{-0.019}} & \mc{1}{c}{\scriptsize{0.444}} & \mc{1}{c}{\scriptsize{0.361}} &  & \mc{1}{c}{\scriptsize{-0.133}} & \mc{1}{c}{\scriptsize{-0.347}} & \mc{1}{c}{\scriptsize{-0.225}} \\  

     &  & \mc{1}{c}{\scriptsize{(0.863)}} & \mc{1}{c}{\scriptsize{(0.490)}} & \mc{1}{c}{\scriptsize{(0.980)}} & \mc{1}{c}{\scriptsize{(0.824)}} &  & \mc{1}{c}{\scriptsize{(0.294)}} & \mc{1}{c}{\scriptsize{(0.118)}} & \mc{1}{c}{\scriptsize{(0.157)}} \\  

    \mc{1}{l}{\scriptsize{Severe Obesity}} & \mc{1}{c}{\scriptsize{Mid-30s}} & \mc{1}{c}{\scriptsize{-0.033}} & \mc{1}{c}{\scriptsize{-0.097}} & \mc{1}{c}{\scriptsize{0.167}} & \mc{1}{c}{\scriptsize{0.182}} & \mc{1}{c}{\scriptsize{0.171}} & \mc{1}{c}{\scriptsize{-0.433}} & \mc{1}{c}{\scriptsize{-0.450}} & \mc{1}{c}{\scriptsize{-0.410}} \\  

     &  & \mc{1}{c}{\scriptsize{(0.373)}} & \mc{1}{c}{\scriptsize{(0.314)}} & \mc{1}{c}{\scriptsize{(0.706)}} & \mc{1}{c}{\scriptsize{(0.588)}} & \mc{1}{c}{\scriptsize{(0.647)}} & \mc{1}{c}{\scriptsize{\textbf{(0.059)}}} & \mc{1}{c}{\scriptsize{(0.216)}} & \mc{1}{c}{\scriptsize{\textbf{(0.078)}}} \\  

    \mc{1}{l}{\scriptsize{Waist-hip Ratio}} & \mc{1}{c}{\scriptsize{Mid-30s}} & \mc{1}{c}{\scriptsize{0.015}} & \mc{1}{c}{\scriptsize{-0.005}} & \mc{1}{c}{\scriptsize{0.049}} & \mc{1}{c}{\scriptsize{0.044}} & \mc{1}{c}{\scriptsize{0.045}} & \mc{1}{c}{\scriptsize{-0.052}} & \mc{1}{c}{\scriptsize{-0.059}} & \mc{1}{c}{\scriptsize{-0.049}} \\  

     &  & \mc{1}{c}{\scriptsize{(0.667)}} & \mc{1}{c}{\scriptsize{(0.412)}} & \mc{1}{c}{\scriptsize{(0.863)}} & \mc{1}{c}{\scriptsize{(0.745)}} & \mc{1}{c}{\scriptsize{(0.647)}} & \mc{1}{c}{\scriptsize{(0.118)}} & \mc{1}{c}{\scriptsize{(0.294)}} & \mc{1}{c}{\scriptsize{(0.235)}} \\  

    \mc{1}{l}{\scriptsize{Abdominal Obesity}} & \mc{1}{c}{\scriptsize{Mid-30s}} & \mc{1}{c}{\scriptsize{-0.023}} & \mc{1}{c}{\scriptsize{-0.159}} & \mc{1}{c}{\scriptsize{0.102}} & \mc{1}{c}{\scriptsize{-0.030}} & \mc{1}{c}{\scriptsize{0.105}} & \mc{1}{c}{\scriptsize{-0.273}} & \mc{1}{c}{\scriptsize{-0.298}} & \mc{1}{c}{\scriptsize{-0.296}} \\  

     &  & \mc{1}{c}{\scriptsize{(0.510)}} & \mc{1}{c}{\scriptsize{(0.255)}} & \mc{1}{c}{\scriptsize{(0.647)}} & \mc{1}{c}{\scriptsize{(0.412)}} & \mc{1}{c}{\scriptsize{(0.451)}} & \mc{1}{c}{\scriptsize{\textbf{(0.059)}}} & \mc{1}{c}{\scriptsize{(0.255)}} & \mc{1}{c}{\scriptsize{\textbf{(0.059)}}} \\  

    \mc{1}{l}{\scriptsize{Framingham Risk Score}} & \mc{1}{c}{\scriptsize{Mid-30s}} & \mc{1}{c}{\scriptsize{-0.469}} & \mc{1}{c}{\scriptsize{0.565}} & \mc{1}{c}{\scriptsize{1.022}} & \mc{1}{c}{\scriptsize{2.145}} &  & \mc{1}{c}{\scriptsize{-2.854}} & \mc{1}{c}{\scriptsize{-2.133}} & \mc{1}{c}{\scriptsize{-2.114}} \\  

     &  & \mc{1}{c}{\scriptsize{(0.294)}} & \mc{1}{c}{\scriptsize{(0.608)}} & \mc{1}{c}{\scriptsize{(0.725)}} & \mc{1}{c}{\scriptsize{(0.824)}} &  & \mc{1}{c}{\scriptsize{\textbf{(0.039)}}} & \mc{1}{c}{\scriptsize{(0.196)}} & \mc{1}{c}{\scriptsize{(0.118)}} \\  

    \mc{1}{l}{\scriptsize{Obesity Factor}} & \mc{1}{c}{\scriptsize{Mid-30s}} & \mc{1}{c}{\scriptsize{0.076}} & \mc{1}{c}{\scriptsize{-0.263}} & \mc{1}{c}{\scriptsize{0.647}} & \mc{1}{c}{\scriptsize{0.442}} & \mc{1}{c}{\scriptsize{0.591}} & \mc{1}{c}{\scriptsize{-1.065}} & \mc{1}{c}{\scriptsize{-1.169}} & \mc{1}{c}{\scriptsize{-1.038}} \\  

     &  & \mc{1}{c}{\scriptsize{(0.608)}} & \mc{1}{c}{\scriptsize{(0.314)}} & \mc{1}{c}{\scriptsize{(0.961)}} & \mc{1}{c}{\scriptsize{(0.784)}} & \mc{1}{c}{\scriptsize{(0.647)}} & \mc{1}{c}{\scriptsize{\textbf{(0.000)}}} & \mc{1}{c}{\scriptsize{(0.235)}} & \mc{1}{c}{\scriptsize{\textbf{(0.000)}}} \\ 
    \hline  

    \\[0.1cm]
    \mc{2}{l}{\scriptsize{\% of Pos. TE ($H_0$: $\le$ 25\% $|$ 10\% Significance)}} & \mc{1}{c}{\scriptsize{0}} & \mc{1}{c}{\scriptsize{0}} & \mc{1}{c}{\scriptsize{0}} & \mc{1}{c}{\scriptsize{0}} & \mc{1}{c}{\scriptsize{0}} & \mc{1}{c}{\scriptsize{71}} & \mc{1}{c}{\scriptsize{14}} & \mc{1}{c}{\scriptsize{57}} \\  

     &  & \mc{1}{c}{\scriptsize{(1.000)}} & \mc{1}{c}{\scriptsize{(1.000)}} & \mc{1}{c}{\scriptsize{(1.000)}} & \mc{1}{c}{\scriptsize{(1.000)}} & \mc{1}{c}{\scriptsize{(0.980)}} & \mc{1}{c}{\scriptsize{\textbf{(0.000)}}} & \mc{1}{c}{\scriptsize{(0.490)}} & \mc{1}{c}{\scriptsize{(0.176)}} \\  

    \mc{2}{l}{\scriptsize{\% of Pos. TE ($H_0$: $\le$ 50\% $|$ 10\% Significance)}} & \mc{1}{c}{\scriptsize{0}} & \mc{1}{c}{\scriptsize{0}} & \mc{1}{c}{\scriptsize{0}} & \mc{1}{c}{\scriptsize{0}} & \mc{1}{c}{\scriptsize{0}} & \mc{1}{c}{\scriptsize{71}} & \mc{1}{c}{\scriptsize{14}} & \mc{1}{c}{\scriptsize{57}} \\  

     &  & \mc{1}{c}{\scriptsize{(1.000)}} & \mc{1}{c}{\scriptsize{(1.000)}} & \mc{1}{c}{\scriptsize{(1.000)}} & \mc{1}{c}{\scriptsize{(1.000)}} & \mc{1}{c}{\scriptsize{(0.980)}} & \mc{1}{c}{\scriptsize{(0.157)}} & \mc{1}{c}{\scriptsize{(0.863)}} & \mc{1}{c}{\scriptsize{(0.490)}} \\  

    \mc{2}{l}{\scriptsize{\% of Pos. TE ($H_0$: $\le$ 75\% $|$ 10\% Significance)}} & \mc{1}{c}{\scriptsize{0}} & \mc{1}{c}{\scriptsize{0}} & \mc{1}{c}{\scriptsize{0}} & \mc{1}{c}{\scriptsize{0}} & \mc{1}{c}{\scriptsize{0}} & \mc{1}{c}{\scriptsize{71}} & \mc{1}{c}{\scriptsize{14}} & \mc{1}{c}{\scriptsize{57}} \\  

     &  & \mc{1}{c}{\scriptsize{(1.000)}} & \mc{1}{c}{\scriptsize{(1.000)}} & \mc{1}{c}{\scriptsize{(1.000)}} & \mc{1}{c}{\scriptsize{(1.000)}} & \mc{1}{c}{\scriptsize{(0.980)}} & \mc{1}{c}{\scriptsize{(0.608)}} & \mc{1}{c}{\scriptsize{(0.863)}} & \mc{1}{c}{\scriptsize{(0.706)}} \\  

  \hline\hline
  \end{tabular}
    \begin{tablenotes}
    \scriptsize
    \item 
Note: This table displays various estimates of the treatment effect of ABC's school age program.
Column (1) displays the ITT, without accounting for any controls.
Column (2) displays the ITT conditioning on vector of controls, $X$, consisting of the Apgar score 1 minute after birth, the HRI index, maternal IQ, an
indicator for teenage pregnancy of the mother, an indicator for the father being at 
home, and an indicator for having a grandmother residing in the same county. We also apply IPW weights, $W$, to account for attrition.
Columns (3)--(4) are analogous to columns (1)--(2), but we restrict the control sample to subjects
who did not enroll in any alternative care.
Column (5) displys the matching estimate, where we use the Mahalanobis metric and Epanechnikov kernel
to match on controls $X$ listed above, and restrict the control sample to subjects who did not enroll
in any alternative care. Additionally, we apply IPW weights, $W$.
Columns (6)--(8) are analogous to Columns (3)--(5), except we restrict the control sample to subejcts
who did enroll in alternative care. The final three pairs of rows display the proportion of treatment effects in the table that are 
socially positive. The first row in each pair displays the percentage of treatment effects, and the
second row presents the inference. 
Numbers in parentheses represent the $p$-value from a single hypothesis test, and are obtained from 
the empirical bootstrap distribution generated by 200 resamples of the original data. 
Bold $p$-values indicate significance at the 10\% level.
Blank point estimates indicate that we are unable to obtain estimates due to a lack of support in the data. 

    \end{tablenotes}
  \end{threeparttable}

\end{table}
\end{center}

\begin{center}
	  \begin{tabular}{cccccccccc}
  \toprule

    \scriptsize{Variable} & \scriptsize{Age} & \scriptsize{(1)} & \scriptsize{(2)} & \scriptsize{(3)} & \scriptsize{(4)} & \scriptsize{(5)} & \scriptsize{(6)} & \scriptsize{(7)} & \scriptsize{(8)} \\ 
    \midrule  

    \mc{1}{l}{\scriptsize{Somatization}} & \mc{1}{c}{\scriptsize{21}} & \mc{1}{c}{\scriptsize{0.035}} & \mc{1}{c}{\scriptsize{-0.051}} & \mc{1}{c}{\scriptsize{-0.021}} & \mc{1}{c}{\scriptsize{-0.259}} & \mc{1}{c}{\scriptsize{-0.019}} & \mc{1}{c}{\scriptsize{0.064}} & \mc{1}{c}{\scriptsize{-0.010}} & \mc{1}{c}{\scriptsize{0.056}} \\  

     &  & \mc{1}{c}{\scriptsize{(0.592)}} & \mc{1}{c}{\scriptsize{(0.342)}} & \mc{1}{c}{\scriptsize{(0.434)}} & \mc{1}{c}{\scriptsize{(0.197)}} & \mc{1}{c}{\scriptsize{(0.487)}} & \mc{1}{c}{\scriptsize{(0.618)}} & \mc{1}{c}{\scriptsize{(0.408)}} & \mc{1}{c}{\scriptsize{(0.618)}} \\  

     & \mc{1}{c}{\scriptsize{34}} & \mc{1}{c}{\scriptsize{-0.466}} & \mc{1}{c}{\scriptsize{-0.502}} & \mc{1}{c}{\scriptsize{0.071}} & \mc{1}{c}{\scriptsize{-0.062}} & \mc{1}{c}{\scriptsize{0.048}} & \mc{1}{c}{\scriptsize{-0.619}} & \mc{1}{c}{\scriptsize{-0.566}} & \mc{1}{c}{\scriptsize{-0.649}} \\  

     &  & \mc{1}{c}{\scriptsize{(0.118)}} & \mc{1}{c}{\scriptsize{(0.118)}} & \mc{1}{c}{\scriptsize{(0.632)}} & \mc{1}{c}{\scriptsize{\textbf{(0.092)}}} & \mc{1}{c}{\scriptsize{(0.579)}} & \mc{1}{c}{\scriptsize{\textbf{(0.079)}}} & \mc{1}{c}{\scriptsize{(0.132)}} & \mc{1}{c}{\scriptsize{\textbf{(0.079)}}} \\  

    \mc{1}{l}{\scriptsize{Depression}} & \mc{1}{c}{\scriptsize{21}} & \mc{1}{c}{\scriptsize{-0.116}} & \mc{1}{c}{\scriptsize{-0.180}} & \mc{1}{c}{\scriptsize{0.105}} & \mc{1}{c}{\scriptsize{-0.142}} & \mc{1}{c}{\scriptsize{0.119}} & \mc{1}{c}{\scriptsize{-0.078}} & \mc{1}{c}{\scriptsize{-0.180}} & \mc{1}{c}{\scriptsize{-0.117}} \\  

     &  & \mc{1}{c}{\scriptsize{(0.263)}} & \mc{1}{c}{\scriptsize{(0.224)}} & \mc{1}{c}{\scriptsize{(0.553)}} & \mc{1}{c}{\scriptsize{(0.329)}} & \mc{1}{c}{\scriptsize{(0.592)}} & \mc{1}{c}{\scriptsize{(0.368)}} & \mc{1}{c}{\scriptsize{(0.237)}} & \mc{1}{c}{\scriptsize{(0.289)}} \\  

     & \mc{1}{c}{\scriptsize{34}} & \mc{1}{c}{\scriptsize{-0.320}} & \mc{1}{c}{\scriptsize{-0.265}} & \mc{1}{c}{\scriptsize{0.254}} & \mc{1}{c}{\scriptsize{0.257}} & \mc{1}{c}{\scriptsize{0.262}} & \mc{1}{c}{\scriptsize{-0.484}} & \mc{1}{c}{\scriptsize{-0.343}} & \mc{1}{c}{\scriptsize{-0.458}} \\  

     &  & \mc{1}{c}{\scriptsize{(0.224)}} & \mc{1}{c}{\scriptsize{(0.263)}} & \mc{1}{c}{\scriptsize{(0.645)}} & \mc{1}{c}{\scriptsize{(0.592)}} & \mc{1}{c}{\scriptsize{(0.632)}} & \mc{1}{c}{\scriptsize{(0.171)}} & \mc{1}{c}{\scriptsize{(0.263)}} & \mc{1}{c}{\scriptsize{(0.171)}} \\  

    \mc{1}{l}{\scriptsize{Anxiety}} & \mc{1}{c}{\scriptsize{21}} & \mc{1}{c}{\scriptsize{0.119}} & \mc{1}{c}{\scriptsize{0.006}} & \mc{1}{c}{\scriptsize{0.286}} & \mc{1}{c}{\scriptsize{-0.023}} & \mc{1}{c}{\scriptsize{0.278}} & \mc{1}{c}{\scriptsize{0.070}} & \mc{1}{c}{\scriptsize{-0.025}} & \mc{1}{c}{\scriptsize{0.019}} \\  

     &  & \mc{1}{c}{\scriptsize{(0.868)}} & \mc{1}{c}{\scriptsize{(0.526)}} & \mc{1}{c}{\scriptsize{(0.895)}} & \mc{1}{c}{\scriptsize{(0.434)}} & \mc{1}{c}{\scriptsize{(0.868)}} & \mc{1}{c}{\scriptsize{(0.737)}} & \mc{1}{c}{\scriptsize{(0.434)}} & \mc{1}{c}{\scriptsize{(0.579)}} \\  

     & \mc{1}{c}{\scriptsize{34}} & \mc{1}{c}{\scriptsize{-0.415}} & \mc{1}{c}{\scriptsize{-0.399}} & \mc{1}{c}{\scriptsize{0.103}} & \mc{1}{c}{\scriptsize{0.102}} & \mc{1}{c}{\scriptsize{0.117}} & \mc{1}{c}{\scriptsize{-0.564}} & \mc{1}{c}{\scriptsize{-0.471}} & \mc{1}{c}{\scriptsize{-0.571}} \\  

     &  & \mc{1}{c}{\scriptsize{(0.105)}} & \mc{1}{c}{\scriptsize{(0.158)}} & \mc{1}{c}{\scriptsize{(0.658)}} & \mc{1}{c}{\scriptsize{(0.553)}} & \mc{1}{c}{\scriptsize{(0.658)}} & \mc{1}{c}{\scriptsize{(0.118)}} & \mc{1}{c}{\scriptsize{(0.158)}} & \mc{1}{c}{\scriptsize{\textbf{(0.092)}}} \\  

    \mc{1}{l}{\scriptsize{Hostility}} & \mc{1}{c}{\scriptsize{21}} & \mc{1}{c}{\scriptsize{-0.180}} & \mc{1}{c}{\scriptsize{-0.165}} & \mc{1}{c}{\scriptsize{0.118}} & \mc{1}{c}{\scriptsize{-0.073}} & \mc{1}{c}{\scriptsize{0.109}} & \mc{1}{c}{\scriptsize{-0.238}} & \mc{1}{c}{\scriptsize{-0.195}} & \mc{1}{c}{\scriptsize{-0.257}} \\  

     &  & \mc{1}{c}{\scriptsize{(0.184)}} & \mc{1}{c}{\scriptsize{(0.250)}} & \mc{1}{c}{\scriptsize{(0.632)}} & \mc{1}{c}{\scriptsize{(0.382)}} & \mc{1}{c}{\scriptsize{(0.632)}} & \mc{1}{c}{\scriptsize{(0.158)}} & \mc{1}{c}{\scriptsize{(0.237)}} & \mc{1}{c}{\scriptsize{(0.145)}} \\  

     & \mc{1}{c}{\scriptsize{34}} & \mc{1}{c}{\scriptsize{-0.368}} & \mc{1}{c}{\scriptsize{-0.335}} & \mc{1}{c}{\scriptsize{0.143}} & \mc{1}{c}{\scriptsize{0.385}} & \mc{1}{c}{\scriptsize{0.133}} & \mc{1}{c}{\scriptsize{-0.514}} & \mc{1}{c}{\scriptsize{-0.439}} & \mc{1}{c}{\scriptsize{-0.522}} \\  

     &  & \mc{1}{c}{\scriptsize{(0.105)}} & \mc{1}{c}{\scriptsize{(0.171)}} & \mc{1}{c}{\scriptsize{(0.645)}} & \mc{1}{c}{\scriptsize{(0.618)}} & \mc{1}{c}{\scriptsize{(0.645)}} & \mc{1}{c}{\scriptsize{(0.132)}} & \mc{1}{c}{\scriptsize{(0.158)}} & \mc{1}{c}{\scriptsize{\textbf{(0.092)}}} \\  

    \mc{1}{l}{\scriptsize{Global Severity Index}} & \mc{1}{c}{\scriptsize{21}} & \mc{1}{c}{\scriptsize{-0.014}} & \mc{1}{c}{\scriptsize{-0.098}} & \mc{1}{c}{\scriptsize{0.188}} & \mc{1}{c}{\scriptsize{-0.102}} & \mc{1}{c}{\scriptsize{0.189}} & \mc{1}{c}{\scriptsize{-0.046}} & \mc{1}{c}{\scriptsize{-0.107}} & \mc{1}{c}{\scriptsize{-0.075}} \\  

     &  & \mc{1}{c}{\scriptsize{(0.474)}} & \mc{1}{c}{\scriptsize{(0.276)}} & \mc{1}{c}{\scriptsize{(0.829)}} & \mc{1}{c}{\scriptsize{(0.289)}} & \mc{1}{c}{\scriptsize{(0.842)}} & \mc{1}{c}{\scriptsize{(0.368)}} & \mc{1}{c}{\scriptsize{(0.289)}} & \mc{1}{c}{\scriptsize{(0.289)}} \\  

     & \mc{1}{c}{\scriptsize{34}} & \mc{1}{c}{\scriptsize{-7.206}} & \mc{1}{c}{\scriptsize{-6.999}} & \mc{1}{c}{\scriptsize{2.571}} & \mc{1}{c}{\scriptsize{1.777}} & \mc{1}{c}{\scriptsize{2.559}} & \mc{1}{c}{\scriptsize{-10.000}} & \mc{1}{c}{\scriptsize{-8.283}} & \mc{1}{c}{\scriptsize{-10.071}} \\  

     &  & \mc{1}{c}{\scriptsize{(0.118)}} & \mc{1}{c}{\scriptsize{(0.171)}} & \mc{1}{c}{\scriptsize{(0.658)}} & \mc{1}{c}{\scriptsize{(0.566)}} & \mc{1}{c}{\scriptsize{(0.645)}} & \mc{1}{c}{\scriptsize{(0.132)}} & \mc{1}{c}{\scriptsize{(0.158)}} & \mc{1}{c}{\scriptsize{(0.105)}} \\  

    \mc{1}{l}{\scriptsize{BSI Factor}} & \mc{1}{c}{\scriptsize{21 and 34}} & \mc{1}{c}{\scriptsize{-0.260}} & \mc{1}{c}{\scriptsize{-0.083}} & \mc{1}{c}{\scriptsize{0.240}} & \mc{1}{c}{\scriptsize{-0.012}} & \mc{1}{c}{\scriptsize{0.279}} & \mc{1}{c}{\scriptsize{-0.343}} & \mc{1}{c}{\scriptsize{-0.150}} & \mc{1}{c}{\scriptsize{-0.186}} \\  

     &  & \mc{1}{c}{\scriptsize{(0.211)}} & \mc{1}{c}{\scriptsize{(0.355)}} & \mc{1}{c}{\scriptsize{(0.618)}} & \mc{1}{c}{\scriptsize{(0.316)}} & \mc{1}{c}{\scriptsize{(0.645)}} & \mc{1}{c}{\scriptsize{(0.237)}} & \mc{1}{c}{\scriptsize{(0.355)}} & \mc{1}{c}{\scriptsize{(0.263)}} \\ 
    \midrule  

    \mc{2}{l}{\scriptsize{\% of Pos. TE ($H_0$: $\le$ 50\%)}} & \mc{1}{c}{\scriptsize{82}} & \mc{1}{c}{\scriptsize{91}} & \mc{1}{c}{\scriptsize{9}} & \mc{1}{c}{\scriptsize{64}} & \mc{1}{c}{\scriptsize{9}} & \mc{1}{c}{\scriptsize{82}} & \mc{1}{c}{\scriptsize{100}} & \mc{1}{c}{\scriptsize{82}} \\  

     &  & \mc{1}{c}{\scriptsize{(0.132)}} & \mc{1}{c}{\scriptsize{\textbf{(0.000)}}} & \mc{1}{c}{\scriptsize{(1.000)}} & \mc{1}{c}{\scriptsize{(0.329)}} & \mc{1}{c}{\scriptsize{(1.000)}} & \mc{1}{c}{\scriptsize{\textbf{(0.000)}}} & \mc{1}{c}{\scriptsize{\textbf{(0.000)}}} & \mc{1}{c}{\scriptsize{\textbf{(0.000)}}} \\  

    \mc{2}{l}{\scriptsize{\% of Pos. TE ($H_0$: $\le$ 10\% $|$ 10\% Significance)}} & \mc{1}{c}{\scriptsize{0}} & \mc{1}{c}{\scriptsize{0}} & \mc{1}{c}{\scriptsize{0}} & \mc{1}{c}{\scriptsize{0}} & \mc{1}{c}{\scriptsize{0}} & \mc{1}{c}{\scriptsize{0}} & \mc{1}{c}{\scriptsize{0}} & \mc{1}{c}{\scriptsize{9}} \\  

     &  & \mc{1}{c}{\scriptsize{(0.487)}} & \mc{1}{c}{\scriptsize{(1.000)}} & \mc{1}{c}{\scriptsize{(1.000)}} & \mc{1}{c}{\scriptsize{(1.000)}} & \mc{1}{c}{\scriptsize{(1.000)}} & \mc{1}{c}{\scriptsize{(0.579)}} & \mc{1}{c}{\scriptsize{(1.000)}} & \mc{1}{c}{\scriptsize{(0.276)}} \\  

  \bottomrule
  \end{tabular}
\end{center}

\begin{center}
	  \begin{tabular}{cccccccccc}
  \toprule

    \scriptsize{Variable} & \scriptsize{Age} & \scriptsize{(1)} & \scriptsize{(2)} & \scriptsize{(3)} & \scriptsize{(4)} & \scriptsize{(5)} & \scriptsize{(6)} & \scriptsize{(7)} & \scriptsize{(8)} \\ 
    \midrule  

    \mc{1}{l}{\scriptsize{No trouble with spouse family}} & \mc{1}{c}{\scriptsize{30}} & \mc{1}{c}{\scriptsize{-0.382}} & \mc{1}{c}{\scriptsize{-0.357}} & \mc{1}{c}{\scriptsize{-0.500}} & \mc{1}{c}{\scriptsize{-0.512}} & \mc{1}{c}{\scriptsize{-0.593}} & \mc{1}{c}{\scriptsize{-0.346}} & \mc{1}{c}{\scriptsize{-0.365}} & \mc{1}{c}{\scriptsize{-0.427}} \\  

     &  & \mc{1}{c}{\scriptsize{(1.000)}} & \mc{1}{c}{\scriptsize{(1.000)}} & \mc{1}{c}{\scriptsize{(0.947)}} & \mc{1}{c}{\scriptsize{(0.145)}} & \mc{1}{c}{\scriptsize{(0.947)}} & \mc{1}{c}{\scriptsize{(1.000)}} & \mc{1}{c}{\scriptsize{(0.974)}} & \mc{1}{c}{\scriptsize{(1.000)}} \\  

    \mc{1}{l}{\scriptsize{Get along well with spouse}} & \mc{1}{c}{\scriptsize{30}} & \mc{1}{c}{\scriptsize{0.099}} & \mc{1}{c}{\scriptsize{0.159}} & \mc{1}{c}{\scriptsize{0.021}} & \mc{1}{c}{\scriptsize{0.353}} & \mc{1}{c}{\scriptsize{-0.231}} & \mc{1}{c}{\scriptsize{0.149}} & \mc{1}{c}{\scriptsize{0.160}} & \mc{1}{c}{\scriptsize{-0.186}} \\  

     &  & \mc{1}{c}{\scriptsize{(0.237)}} & \mc{1}{c}{\scriptsize{(0.184)}} & \mc{1}{c}{\scriptsize{(0.382)}} & \mc{1}{c}{\scriptsize{(0.145)}} & \mc{1}{c}{\scriptsize{(0.697)}} & \mc{1}{c}{\scriptsize{(0.184)}} & \mc{1}{c}{\scriptsize{(0.184)}} & \mc{1}{c}{\scriptsize{(0.855)}} \\  

    \mc{1}{l}{\scriptsize{No disagreement on living arrangement}} & \mc{1}{c}{\scriptsize{30}} & \mc{1}{c}{\scriptsize{0.217}} & \mc{1}{c}{\scriptsize{0.263}} & \mc{1}{c}{\scriptsize{0.688}} & \mc{1}{c}{\scriptsize{0.524}} & \mc{1}{c}{\scriptsize{0.716}} & \mc{1}{c}{\scriptsize{0.149}} & \mc{1}{c}{\scriptsize{0.254}} & \mc{1}{c}{\scriptsize{0.171}} \\  

     &  & \mc{1}{c}{\scriptsize{(0.118)}} & \mc{1}{c}{\scriptsize{(0.105)}} & \mc{1}{c}{\scriptsize{\textbf{(0.000)}}} & \mc{1}{c}{\scriptsize{(0.105)}} & \mc{1}{c}{\scriptsize{\textbf{(0.000)}}} & \mc{1}{c}{\scriptsize{(0.197)}} & \mc{1}{c}{\scriptsize{(0.158)}} & \mc{1}{c}{\scriptsize{(0.197)}} \\  

  \bottomrule
  \end{tabular}
\end{center}
\section{Treatment Effects for Female Sample}


\begin{center}
	\begin{table}[H]
\captionsetup{singlelinecheck=false,justification=centering}
\caption{ABC/CARE Average Treatment Effects, Females \\ IQ Scores \label{tab:ate_female_apx0}}

  \begin{threeparttable}
  \begin{tabular}{cccccccccc}
  \hline\hline

     &  & \scriptsize{(1)} & \scriptsize{(2)} & \scriptsize{(3)} & \scriptsize{(4)} & \scriptsize{(5)} & \scriptsize{(6)} & \scriptsize{(7)} & \scriptsize{(8)} \\  

     &  &  &  & \mc{3}{c}{\scriptsize{$P=0$}} & \mc{3}{c}{\scriptsize{$P=1$}} \\ 
    \cmidrule(lr){5-7} \cmidrule(lr){8-10} 

    \scriptsize{Variable} & \scriptsize{Age} & \scriptsize{ITT} & \scriptsize{ITT$|X,W$} & \scriptsize{ITT} & \scriptsize{ITT$|X,W$} & \scriptsize{KE$|X,W$} & \scriptsize{ITT} & \scriptsize{ITT$|X,W$} & \scriptsize{KE$|X,W$} \\ 
    \hline  

    \mc{1}{l}{\scriptsize{Std. IQ Test}} & \mc{1}{c}{\scriptsize{2}} & \mc{1}{c}{\scriptsize{9.461}} & \mc{1}{c}{\scriptsize{10.218}} & \mc{1}{c}{\scriptsize{13.855}} & \mc{1}{c}{\scriptsize{14.032}} & \mc{1}{c}{\scriptsize{15.444}} & \mc{1}{c}{\scriptsize{7.401}} & \mc{1}{c}{\scriptsize{8.273}} & \mc{1}{c}{\scriptsize{6.729}} \\  

     &  & \mc{1}{c}{\scriptsize{\textbf{(0.000)}}} & \mc{1}{c}{\scriptsize{\textbf{(0.000)}}} & \mc{1}{c}{\scriptsize{\textbf{(0.000)}}} & \mc{1}{c}{\scriptsize{\textbf{(0.000)}}} & \mc{1}{c}{\scriptsize{\textbf{(0.000)}}} & \mc{1}{c}{\scriptsize{\textbf{(0.000)}}} & \mc{1}{c}{\scriptsize{\textbf{(0.000)}}} & \mc{1}{c}{\scriptsize{\textbf{(0.000)}}} \\  

     & \mc{1}{c}{\scriptsize{3}} & \mc{1}{c}{\scriptsize{12.782}} & \mc{1}{c}{\scriptsize{14.474}} & \mc{1}{c}{\scriptsize{23.497}} & \mc{1}{c}{\scriptsize{26.800}} & \mc{1}{c}{\scriptsize{24.314}} & \mc{1}{c}{\scriptsize{7.760}} & \mc{1}{c}{\scriptsize{9.502}} & \mc{1}{c}{\scriptsize{7.570}} \\  

     &  & \mc{1}{c}{\scriptsize{\textbf{(0.000)}}} & \mc{1}{c}{\scriptsize{\textbf{(0.000)}}} & \mc{1}{c}{\scriptsize{\textbf{(0.000)}}} & \mc{1}{c}{\scriptsize{\textbf{(0.000)}}} & \mc{1}{c}{\scriptsize{\textbf{(0.000)}}} & \mc{1}{c}{\scriptsize{\textbf{(0.020)}}} & \mc{1}{c}{\scriptsize{\textbf{(0.020)}}} & \mc{1}{c}{\scriptsize{\textbf{(0.039)}}} \\  

     & \mc{1}{c}{\scriptsize{3.5}} & \mc{1}{c}{\scriptsize{7.532}} & \mc{1}{c}{\scriptsize{8.622}} & \mc{1}{c}{\scriptsize{16.361}} & \mc{1}{c}{\scriptsize{18.924}} & \mc{1}{c}{\scriptsize{16.869}} & \mc{1}{c}{\scriptsize{3.670}} & \mc{1}{c}{\scriptsize{4.772}} & \mc{1}{c}{\scriptsize{3.182}} \\  

     &  & \mc{1}{c}{\scriptsize{\textbf{(0.000)}}} & \mc{1}{c}{\scriptsize{\textbf{(0.000)}}} & \mc{1}{c}{\scriptsize{\textbf{(0.000)}}} & \mc{1}{c}{\scriptsize{\textbf{(0.000)}}} & \mc{1}{c}{\scriptsize{\textbf{(0.000)}}} & \mc{1}{c}{\scriptsize{\textbf{(0.098)}}} & \mc{1}{c}{\scriptsize{\textbf{(0.039)}}} & \mc{1}{c}{\scriptsize{(0.118)}} \\  

     & \mc{1}{c}{\scriptsize{4}} & \mc{1}{c}{\scriptsize{5.649}} & \mc{1}{c}{\scriptsize{6.389}} & \mc{1}{c}{\scriptsize{13.758}} & \mc{1}{c}{\scriptsize{16.022}} & \mc{1}{c}{\scriptsize{15.061}} & \mc{1}{c}{\scriptsize{2.101}} & \mc{1}{c}{\scriptsize{2.719}} & \mc{1}{c}{\scriptsize{2.462}} \\  

     &  & \mc{1}{c}{\scriptsize{\textbf{(0.000)}}} & \mc{1}{c}{\scriptsize{\textbf{(0.000)}}} & \mc{1}{c}{\scriptsize{\textbf{(0.000)}}} & \mc{1}{c}{\scriptsize{\textbf{(0.000)}}} & \mc{1}{c}{\scriptsize{\textbf{(0.000)}}} & \mc{1}{c}{\scriptsize{(0.216)}} & \mc{1}{c}{\scriptsize{(0.157)}} & \mc{1}{c}{\scriptsize{(0.235)}} \\  

     & \mc{1}{c}{\scriptsize{4.5}} & \mc{1}{c}{\scriptsize{8.276}} & \mc{1}{c}{\scriptsize{9.169}} & \mc{1}{c}{\scriptsize{17.618}} & \mc{1}{c}{\scriptsize{20.548}} & \mc{1}{c}{\scriptsize{17.683}} & \mc{1}{c}{\scriptsize{4.481}} & \mc{1}{c}{\scriptsize{5.697}} & \mc{1}{c}{\scriptsize{4.510}} \\  

     &  & \mc{1}{c}{\scriptsize{\textbf{(0.000)}}} & \mc{1}{c}{\scriptsize{\textbf{(0.000)}}} & \mc{1}{c}{\scriptsize{\textbf{(0.000)}}} & \mc{1}{c}{\scriptsize{\textbf{(0.000)}}} & \mc{1}{c}{\scriptsize{\textbf{(0.000)}}} & \mc{1}{c}{\scriptsize{\textbf{(0.098)}}} & \mc{1}{c}{\scriptsize{\textbf{(0.020)}}} & \mc{1}{c}{\scriptsize{\textbf{(0.078)}}} \\  

     & \mc{1}{c}{\scriptsize{5}} & \mc{1}{c}{\scriptsize{5.359}} & \mc{1}{c}{\scriptsize{5.800}} & \mc{1}{c}{\scriptsize{13.211}} & \mc{1}{c}{\scriptsize{16.455}} & \mc{1}{c}{\scriptsize{14.058}} & \mc{1}{c}{\scriptsize{2.169}} & \mc{1}{c}{\scriptsize{2.679}} & \mc{1}{c}{\scriptsize{1.887}} \\  

     &  & \mc{1}{c}{\scriptsize{\textbf{(0.020)}}} & \mc{1}{c}{\scriptsize{\textbf{(0.020)}}} & \mc{1}{c}{\scriptsize{\textbf{(0.000)}}} & \mc{1}{c}{\scriptsize{\textbf{(0.000)}}} & \mc{1}{c}{\scriptsize{\textbf{(0.000)}}} & \mc{1}{c}{\scriptsize{(0.255)}} & \mc{1}{c}{\scriptsize{(0.196)}} & \mc{1}{c}{\scriptsize{(0.216)}} \\  

     & \mc{1}{c}{\scriptsize{6.6}} & \mc{1}{c}{\scriptsize{5.728}} & \mc{1}{c}{\scriptsize{7.788}} & \mc{1}{c}{\scriptsize{10.234}} & \mc{1}{c}{\scriptsize{16.517}} & \mc{1}{c}{\scriptsize{12.054}} & \mc{1}{c}{\scriptsize{4.076}} & \mc{1}{c}{\scriptsize{5.920}} & \mc{1}{c}{\scriptsize{4.829}} \\  

     &  & \mc{1}{c}{\scriptsize{\textbf{(0.059)}}} & \mc{1}{c}{\scriptsize{\textbf{(0.020)}}} & \mc{1}{c}{\scriptsize{\textbf{(0.000)}}} & \mc{1}{c}{\scriptsize{\textbf{(0.020)}}} & \mc{1}{c}{\scriptsize{\textbf{(0.000)}}} & \mc{1}{c}{\scriptsize{(0.137)}} & \mc{1}{c}{\scriptsize{\textbf{(0.059)}}} & \mc{1}{c}{\scriptsize{\textbf{(0.059)}}} \\  

     & \mc{1}{c}{\scriptsize{7}} & \mc{1}{c}{\scriptsize{4.546}} & \mc{1}{c}{\scriptsize{6.759}} & \mc{1}{c}{\scriptsize{8.157}} & \mc{1}{c}{\scriptsize{13.518}} & \mc{1}{c}{\scriptsize{8.340}} & \mc{1}{c}{\scriptsize{3.209}} & \mc{1}{c}{\scriptsize{4.721}} & \mc{1}{c}{\scriptsize{3.256}} \\  

     &  & \mc{1}{c}{\scriptsize{\textbf{(0.098)}}} & \mc{1}{c}{\scriptsize{\textbf{(0.000)}}} & \mc{1}{c}{\scriptsize{\textbf{(0.000)}}} & \mc{1}{c}{\scriptsize{\textbf{(0.000)}}} & \mc{1}{c}{\scriptsize{\textbf{(0.000)}}} & \mc{1}{c}{\scriptsize{(0.137)}} & \mc{1}{c}{\scriptsize{\textbf{(0.098)}}} & \mc{1}{c}{\scriptsize{(0.137)}} \\  

     & \mc{1}{c}{\scriptsize{8}} & \mc{1}{c}{\scriptsize{6.236}} & \mc{1}{c}{\scriptsize{7.285}} & \mc{1}{c}{\scriptsize{9.333}} & \mc{1}{c}{\scriptsize{14.908}} & \mc{1}{c}{\scriptsize{10.473}} & \mc{1}{c}{\scriptsize{5.100}} & \mc{1}{c}{\scriptsize{5.859}} & \mc{1}{c}{\scriptsize{5.166}} \\  

     &  & \mc{1}{c}{\scriptsize{\textbf{(0.000)}}} & \mc{1}{c}{\scriptsize{\textbf{(0.000)}}} & \mc{1}{c}{\scriptsize{\textbf{(0.000)}}} & \mc{1}{c}{\scriptsize{\textbf{(0.000)}}} & \mc{1}{c}{\scriptsize{\textbf{(0.000)}}} & \mc{1}{c}{\scriptsize{\textbf{(0.020)}}} & \mc{1}{c}{\scriptsize{\textbf{(0.000)}}} & \mc{1}{c}{\scriptsize{\textbf{(0.020)}}} \\  

     & \mc{1}{c}{\scriptsize{12}} & \mc{1}{c}{\scriptsize{8.259}} & \mc{1}{c}{\scriptsize{10.271}} & \mc{1}{c}{\scriptsize{8.457}} & \mc{1}{c}{\scriptsize{11.647}} & \mc{1}{c}{\scriptsize{7.881}} & \mc{1}{c}{\scriptsize{8.167}} & \mc{1}{c}{\scriptsize{9.877}} & \mc{1}{c}{\scriptsize{7.186}} \\  

     &  & \mc{1}{c}{\scriptsize{\textbf{(0.000)}}} & \mc{1}{c}{\scriptsize{\textbf{(0.000)}}} & \mc{1}{c}{\scriptsize{\textbf{(0.000)}}} & \mc{1}{c}{\scriptsize{\textbf{(0.020)}}} & \mc{1}{c}{\scriptsize{\textbf{(0.000)}}} & \mc{1}{c}{\scriptsize{\textbf{(0.000)}}} & \mc{1}{c}{\scriptsize{\textbf{(0.000)}}} & \mc{1}{c}{\scriptsize{\textbf{(0.020)}}} \\  

     & \mc{1}{c}{\scriptsize{15}} & \mc{1}{c}{\scriptsize{6.467}} & \mc{1}{c}{\scriptsize{6.467}} & \mc{1}{c}{\scriptsize{7.074}} & \mc{1}{c}{\scriptsize{7.074}} & \mc{1}{c}{\scriptsize{3.489}} & \mc{1}{c}{\scriptsize{6.265}} & \mc{1}{c}{\scriptsize{6.265}} & \mc{1}{c}{\scriptsize{4.768}} \\  

     &  & \mc{1}{c}{\scriptsize{\textbf{(0.020)}}} & \mc{1}{c}{\scriptsize{\textbf{(0.020)}}} & \mc{1}{c}{\scriptsize{\textbf{(0.020)}}} & \mc{1}{c}{\scriptsize{\textbf{(0.020)}}} & \mc{1}{c}{\scriptsize{(0.196)}} & \mc{1}{c}{\scriptsize{\textbf{(0.020)}}} & \mc{1}{c}{\scriptsize{\textbf{(0.020)}}} & \mc{1}{c}{\scriptsize{(0.137)}} \\  

     & \mc{1}{c}{\scriptsize{21}} & \mc{1}{c}{\scriptsize{7.261}} & \mc{1}{c}{\scriptsize{7.261}} & \mc{1}{c}{\scriptsize{9.440}} & \mc{1}{c}{\scriptsize{9.440}} & \mc{1}{c}{\scriptsize{7.800}} & \mc{1}{c}{\scriptsize{6.535}} & \mc{1}{c}{\scriptsize{6.535}} & \mc{1}{c}{\scriptsize{5.305}} \\  

     &  & \mc{1}{c}{\scriptsize{\textbf{(0.020)}}} & \mc{1}{c}{\scriptsize{\textbf{(0.020)}}} & \mc{1}{c}{\scriptsize{\textbf{(0.000)}}} & \mc{1}{c}{\scriptsize{\textbf{(0.000)}}} & \mc{1}{c}{\scriptsize{\textbf{(0.000)}}} & \mc{1}{c}{\scriptsize{\textbf{(0.020)}}} & \mc{1}{c}{\scriptsize{\textbf{(0.020)}}} & \mc{1}{c}{\scriptsize{\textbf{(0.020)}}} \\  

    \mc{1}{l}{\scriptsize{IQ Factor}} & \mc{1}{c}{\scriptsize{2 to 5}} & \mc{1}{c}{\scriptsize{0.614}} & \mc{1}{c}{\scriptsize{0.686}} & \mc{1}{c}{\scriptsize{1.347}} & \mc{1}{c}{\scriptsize{1.612}} & \mc{1}{c}{\scriptsize{1.397}} & \mc{1}{c}{\scriptsize{0.340}} & \mc{1}{c}{\scriptsize{0.422}} & \mc{1}{c}{\scriptsize{0.328}} \\  

     &  & \mc{1}{c}{\scriptsize{\textbf{(0.000)}}} & \mc{1}{c}{\scriptsize{\textbf{(0.000)}}} & \mc{1}{c}{\scriptsize{\textbf{(0.000)}}} & \mc{1}{c}{\scriptsize{\textbf{(0.000)}}} & \mc{1}{c}{\scriptsize{\textbf{(0.000)}}} & \mc{1}{c}{\scriptsize{\textbf{(0.078)}}} & \mc{1}{c}{\scriptsize{\textbf{(0.020)}}} & \mc{1}{c}{\scriptsize{\textbf{(0.059)}}} \\  

     & \mc{1}{c}{\scriptsize{6 to 12}} & \mc{1}{c}{\scriptsize{0.491}} & \mc{1}{c}{\scriptsize{0.679}} & \mc{1}{c}{\scriptsize{0.598}} & \mc{1}{c}{\scriptsize{1.130}} & \mc{1}{c}{\scriptsize{0.706}} & \mc{1}{c}{\scriptsize{0.451}} & \mc{1}{c}{\scriptsize{0.596}} & \mc{1}{c}{\scriptsize{0.489}} \\  

     &  & \mc{1}{c}{\scriptsize{\textbf{(0.000)}}} & \mc{1}{c}{\scriptsize{\textbf{(0.000)}}} & \mc{1}{c}{\scriptsize{\textbf{(0.000)}}} & \mc{1}{c}{\scriptsize{\textbf{(0.039)}}} & \mc{1}{c}{\scriptsize{\textbf{(0.000)}}} & \mc{1}{c}{\scriptsize{\textbf{(0.039)}}} & \mc{1}{c}{\scriptsize{\textbf{(0.000)}}} & \mc{1}{c}{\scriptsize{\textbf{(0.020)}}} \\ 
    \hline  

    \\[0.1cm]
    \mc{2}{l}{\scriptsize{\% of Pos. TE ($H_0$: $\le$ 25\% $|$ 10\% Significance)}} & \mc{1}{c}{\scriptsize{100}} & \mc{1}{c}{\scriptsize{100}} & \mc{1}{c}{\scriptsize{100}} & \mc{1}{c}{\scriptsize{100}} & \mc{1}{c}{\scriptsize{93}} & \mc{1}{c}{\scriptsize{71}} & \mc{1}{c}{\scriptsize{86}} & \mc{1}{c}{\scriptsize{64}} \\  

     &  & \mc{1}{c}{\scriptsize{\textbf{(0.000)}}} & \mc{1}{c}{\scriptsize{\textbf{(0.000)}}} & \mc{1}{c}{\scriptsize{\textbf{(0.000)}}} & \mc{1}{c}{\scriptsize{\textbf{(0.000)}}} & \mc{1}{c}{\scriptsize{\textbf{(0.000)}}} & \mc{1}{c}{\scriptsize{\textbf{(0.000)}}} & \mc{1}{c}{\scriptsize{\textbf{(0.000)}}} & \mc{1}{c}{\scriptsize{(0.137)}} \\  

    \mc{2}{l}{\scriptsize{\% of Pos. TE ($H_0$: $\le$ 50\% $|$ 10\% Significance)}} & \mc{1}{c}{\scriptsize{100}} & \mc{1}{c}{\scriptsize{100}} & \mc{1}{c}{\scriptsize{100}} & \mc{1}{c}{\scriptsize{100}} & \mc{1}{c}{\scriptsize{93}} & \mc{1}{c}{\scriptsize{71}} & \mc{1}{c}{\scriptsize{86}} & \mc{1}{c}{\scriptsize{64}} \\  

     &  & \mc{1}{c}{\scriptsize{\textbf{(0.000)}}} & \mc{1}{c}{\scriptsize{\textbf{(0.000)}}} & \mc{1}{c}{\scriptsize{\textbf{(0.000)}}} & \mc{1}{c}{\scriptsize{\textbf{(0.000)}}} & \mc{1}{c}{\scriptsize{\textbf{(0.000)}}} & \mc{1}{c}{\scriptsize{(0.353)}} & \mc{1}{c}{\scriptsize{\textbf{(0.000)}}} & \mc{1}{c}{\scriptsize{(0.412)}} \\  

    \mc{2}{l}{\scriptsize{\% of Pos. TE ($H_0$: $\le$ 75\% $|$ 10\% Significance)}} & \mc{1}{c}{\scriptsize{100}} & \mc{1}{c}{\scriptsize{100}} & \mc{1}{c}{\scriptsize{100}} & \mc{1}{c}{\scriptsize{100}} & \mc{1}{c}{\scriptsize{93}} & \mc{1}{c}{\scriptsize{71}} & \mc{1}{c}{\scriptsize{86}} & \mc{1}{c}{\scriptsize{64}} \\  

     &  & \mc{1}{c}{\scriptsize{\textbf{(0.000)}}} & \mc{1}{c}{\scriptsize{\textbf{(0.000)}}} & \mc{1}{c}{\scriptsize{\textbf{(0.000)}}} & \mc{1}{c}{\scriptsize{\textbf{(0.000)}}} & \mc{1}{c}{\scriptsize{\textbf{(0.000)}}} & \mc{1}{c}{\scriptsize{(0.647)}} & \mc{1}{c}{\scriptsize{(0.490)}} & \mc{1}{c}{\scriptsize{(0.627)}} \\  

  \hline\hline
  \end{tabular}
    \begin{tablenotes}
    \scriptsize
    \item 
Note: This table displays various estimates of the treatment effect of ABC/CARE's center-based care.
Column (1) displays the ITT, without accounting for any controls.
Column (2) displays the ITT conditioning on vector of controls, $X$, consisting of APGAR scores 1 
minute after birth, an indicator for the subject being born prematurely, and an indicator for the 
father being home at baseline. We also apply IPW weights, $W$, to account for attrition.
Columns (3)--(4) are analogous to columns (1)--(2), but we restrict the control sample to subjects
who did not enroll in any alternative care.
Column (5) displys the matching estimate, where we use the Mahalanobis metric and Epanechnikov kernel
to match on controls $X$ listed above, and restrict the control sample to subjects who did not enroll
in any alternative care. Additionally, we apply IPW weights, $W$.
Columns (6)--(8) are analogous to Columns (3)--(5), except we restrict the control sample to subejcts
who did enroll in alternative care. 
The final three pairs of rows display the proportion of treatment effects in the table that are 
socially positive. The first row in each pair displays the percentage of treatment effects, and the
second row presents the inference.

Numbers in parentheses represent the $p$-value from a single hypothesis test, and are obtained from 
the empirical bootstrap distribution generated by 200 resamples of the original data. 
Bold $p$-values indicate significance at the 10\% level.
Blank point estimates indicate that we are unable to obtain estimates due to a lack of support in the data. 

    \end{tablenotes}
  \end{threeparttable}

\end{table}
\end{center}

\begin{center}
	  \begin{tabular}{cccccccccc}
  \toprule

    \scriptsize{Variable} & \scriptsize{Age} & \scriptsize{(1)} & \scriptsize{(2)} & \scriptsize{(3)} & \scriptsize{(4)} & \scriptsize{(5)} & \scriptsize{(6)} & \scriptsize{(7)} & \scriptsize{(8)} \\ 
    \midrule  

    \mc{1}{l}{\scriptsize{Std. Achv.  Test}} & \mc{1}{c}{\scriptsize{15}} & \mc{1}{c}{\scriptsize{8.275}} & \mc{1}{c}{\scriptsize{7.462}} & \mc{1}{c}{\scriptsize{9.618}} & \mc{1}{c}{\scriptsize{6.801}} & \mc{1}{c}{\scriptsize{8.583}} & \mc{1}{c}{\scriptsize{8.477}} & \mc{1}{c}{\scriptsize{7.601}} & \mc{1}{c}{\scriptsize{8.018}} \\  

     &  & \mc{1}{c}{\scriptsize{\textbf{(0.026)}}} & \mc{1}{c}{\scriptsize{\textbf{(0.026)}}} & \mc{1}{c}{\scriptsize{\textbf{(0.000)}}} & \mc{1}{c}{\scriptsize{(0.118)}} & \mc{1}{c}{\scriptsize{\textbf{(0.000)}}} & \mc{1}{c}{\scriptsize{\textbf{(0.026)}}} & \mc{1}{c}{\scriptsize{\textbf{(0.066)}}} & \mc{1}{c}{\scriptsize{\textbf{(0.013)}}} \\  

     & \mc{1}{c}{\scriptsize{21}} & \mc{1}{c}{\scriptsize{9.116}} & \mc{1}{c}{\scriptsize{7.303}} & \mc{1}{c}{\scriptsize{8.420}} & \mc{1}{c}{\scriptsize{5.586}} & \mc{1}{c}{\scriptsize{5.601}} & \mc{1}{c}{\scriptsize{9.420}} & \mc{1}{c}{\scriptsize{7.783}} & \mc{1}{c}{\scriptsize{7.862}} \\  

     &  & \mc{1}{c}{\scriptsize{\textbf{(0.026)}}} & \mc{1}{c}{\scriptsize{\textbf{(0.053)}}} & \mc{1}{c}{\scriptsize{\textbf{(0.013)}}} & \mc{1}{c}{\scriptsize{(0.132)}} & \mc{1}{c}{\scriptsize{\textbf{(0.066)}}} & \mc{1}{c}{\scriptsize{\textbf{(0.026)}}} & \mc{1}{c}{\scriptsize{\textbf{(0.079)}}} & \mc{1}{c}{\scriptsize{\textbf{(0.026)}}} \\  

     & \mc{1}{c}{\scriptsize{5.5}} & \mc{1}{c}{\scriptsize{9.750}} & \mc{1}{c}{\scriptsize{6.913}} & \mc{1}{c}{\scriptsize{20.733}} & \mc{1}{c}{\scriptsize{7.580}} & \mc{1}{c}{\scriptsize{20.882}} & \mc{1}{c}{\scriptsize{7.812}} & \mc{1}{c}{\scriptsize{6.118}} & \mc{1}{c}{\scriptsize{9.080}} \\  

     &  & \mc{1}{c}{\scriptsize{\textbf{(0.039)}}} & \mc{1}{c}{\scriptsize{\textbf{(0.026)}}} & \mc{1}{c}{\scriptsize{\textbf{(0.000)}}} & \mc{1}{c}{\scriptsize{(0.171)}} & \mc{1}{c}{\scriptsize{\textbf{(0.000)}}} & \mc{1}{c}{\scriptsize{\textbf{(0.066)}}} & \mc{1}{c}{\scriptsize{\textbf{(0.092)}}} & \mc{1}{c}{\scriptsize{\textbf{(0.013)}}} \\  

     & \mc{1}{c}{\scriptsize{6}} & \mc{1}{c}{\scriptsize{5.285}} & \mc{1}{c}{\scriptsize{6.542}} & \mc{1}{c}{\scriptsize{10.406}} & \mc{1}{c}{\scriptsize{4.484}} & \mc{1}{c}{\scriptsize{10.347}} & \mc{1}{c}{\scriptsize{4.476}} & \mc{1}{c}{\scriptsize{5.663}} & \mc{1}{c}{\scriptsize{5.088}} \\  

     &  & \mc{1}{c}{\scriptsize{\textbf{(0.000)}}} & \mc{1}{c}{\scriptsize{\textbf{(0.000)}}} & \mc{1}{c}{\scriptsize{\textbf{(0.039)}}} & \mc{1}{c}{\scriptsize{(0.211)}} & \mc{1}{c}{\scriptsize{\textbf{(0.039)}}} & \mc{1}{c}{\scriptsize{\textbf{(0.013)}}} & \mc{1}{c}{\scriptsize{\textbf{(0.026)}}} & \mc{1}{c}{\scriptsize{\textbf{(0.013)}}} \\  

     & \mc{1}{c}{\scriptsize{6.5}} & \mc{1}{c}{\scriptsize{3.909}} & \mc{1}{c}{\scriptsize{3.850}} & \mc{1}{c}{\scriptsize{6.394}} & \mc{1}{c}{\scriptsize{1.048}} & \mc{1}{c}{\scriptsize{6.264}} & \mc{1}{c}{\scriptsize{3.517}} & \mc{1}{c}{\scriptsize{3.845}} & \mc{1}{c}{\scriptsize{4.218}} \\  

     &  & \mc{1}{c}{\scriptsize{\textbf{(0.026)}}} & \mc{1}{c}{\scriptsize{\textbf{(0.013)}}} & \mc{1}{c}{\scriptsize{\textbf{(0.000)}}} & \mc{1}{c}{\scriptsize{(0.421)}} & \mc{1}{c}{\scriptsize{\textbf{(0.026)}}} & \mc{1}{c}{\scriptsize{\textbf{(0.092)}}} & \mc{1}{c}{\scriptsize{\textbf{(0.039)}}} & \mc{1}{c}{\scriptsize{\textbf{(0.013)}}} \\  

     & \mc{1}{c}{\scriptsize{7}} & \mc{1}{c}{\scriptsize{6.411}} & \mc{1}{c}{\scriptsize{6.373}} & \mc{1}{c}{\scriptsize{12.724}} & \mc{1}{c}{\scriptsize{5.785}} & \mc{1}{c}{\scriptsize{13.215}} & \mc{1}{c}{\scriptsize{5.415}} & \mc{1}{c}{\scriptsize{6.065}} & \mc{1}{c}{\scriptsize{6.425}} \\  

     &  & \mc{1}{c}{\scriptsize{\textbf{(0.000)}}} & \mc{1}{c}{\scriptsize{\textbf{(0.000)}}} & \mc{1}{c}{\scriptsize{\textbf{(0.000)}}} & \mc{1}{c}{\scriptsize{\textbf{(0.079)}}} & \mc{1}{c}{\scriptsize{\textbf{(0.000)}}} & \mc{1}{c}{\scriptsize{\textbf{(0.026)}}} & \mc{1}{c}{\scriptsize{\textbf{(0.000)}}} & \mc{1}{c}{\scriptsize{\textbf{(0.000)}}} \\  

     & \mc{1}{c}{\scriptsize{7.5}} & \mc{1}{c}{\scriptsize{5.213}} & \mc{1}{c}{\scriptsize{5.180}} & \mc{1}{c}{\scriptsize{10.086}} & \mc{1}{c}{\scriptsize{7.457}} & \mc{1}{c}{\scriptsize{10.191}} & \mc{1}{c}{\scriptsize{4.401}} & \mc{1}{c}{\scriptsize{4.464}} & \mc{1}{c}{\scriptsize{5.045}} \\  

     &  & \mc{1}{c}{\scriptsize{\textbf{(0.000)}}} & \mc{1}{c}{\scriptsize{\textbf{(0.013)}}} & \mc{1}{c}{\scriptsize{\textbf{(0.000)}}} & \mc{1}{c}{\scriptsize{\textbf{(0.079)}}} & \mc{1}{c}{\scriptsize{\textbf{(0.000)}}} & \mc{1}{c}{\scriptsize{\textbf{(0.026)}}} & \mc{1}{c}{\scriptsize{\textbf{(0.053)}}} & \mc{1}{c}{\scriptsize{\textbf{(0.013)}}} \\  

     & \mc{1}{c}{\scriptsize{8}} & \mc{1}{c}{\scriptsize{6.651}} & \mc{1}{c}{\scriptsize{6.033}} & \mc{1}{c}{\scriptsize{11.626}} & \mc{1}{c}{\scriptsize{7.532}} & \mc{1}{c}{\scriptsize{12.673}} & \mc{1}{c}{\scriptsize{5.822}} & \mc{1}{c}{\scriptsize{5.346}} & \mc{1}{c}{\scriptsize{6.853}} \\  

     &  & \mc{1}{c}{\scriptsize{\textbf{(0.013)}}} & \mc{1}{c}{\scriptsize{\textbf{(0.066)}}} & \mc{1}{c}{\scriptsize{\textbf{(0.000)}}} & \mc{1}{c}{\scriptsize{(0.105)}} & \mc{1}{c}{\scriptsize{\textbf{(0.000)}}} & \mc{1}{c}{\scriptsize{\textbf{(0.053)}}} & \mc{1}{c}{\scriptsize{\textbf{(0.079)}}} & \mc{1}{c}{\scriptsize{\textbf{(0.000)}}} \\  

     & \mc{1}{c}{\scriptsize{8.5}} & \mc{1}{c}{\scriptsize{8.656}} & \mc{1}{c}{\scriptsize{9.006}} & \mc{1}{c}{\scriptsize{17.116}} & \mc{1}{c}{\scriptsize{9.592}} & \mc{1}{c}{\scriptsize{17.199}} & \mc{1}{c}{\scriptsize{7.246}} & \mc{1}{c}{\scriptsize{8.136}} & \mc{1}{c}{\scriptsize{8.625}} \\  

     &  & \mc{1}{c}{\scriptsize{\textbf{(0.000)}}} & \mc{1}{c}{\scriptsize{\textbf{(0.000)}}} & \mc{1}{c}{\scriptsize{\textbf{(0.000)}}} & \mc{1}{c}{\scriptsize{\textbf{(0.079)}}} & \mc{1}{c}{\scriptsize{\textbf{(0.000)}}} & \mc{1}{c}{\scriptsize{\textbf{(0.026)}}} & \mc{1}{c}{\scriptsize{\textbf{(0.039)}}} & \mc{1}{c}{\scriptsize{\textbf{(0.000)}}} \\  

    \mc{1}{l}{\scriptsize{Achievement Factor}} & \mc{1}{c}{\scriptsize{5.5 to 12}} & \mc{1}{c}{\scriptsize{0.809}} & \mc{1}{c}{\scriptsize{0.857}} & \mc{1}{c}{\scriptsize{1.689}} & \mc{1}{c}{\scriptsize{0.941}} & \mc{1}{c}{\scriptsize{1.747}} & \mc{1}{c}{\scriptsize{0.621}} & \mc{1}{c}{\scriptsize{0.745}} & \mc{1}{c}{\scriptsize{0.807}} \\  

     &  & \mc{1}{c}{\scriptsize{\textbf{(0.000)}}} & \mc{1}{c}{\scriptsize{\textbf{(0.000)}}} & \mc{1}{c}{\scriptsize{\textbf{(0.000)}}} & \mc{1}{c}{\scriptsize{(0.158)}} & \mc{1}{c}{\scriptsize{\textbf{(0.000)}}} & \mc{1}{c}{\scriptsize{\textbf{(0.013)}}} & \mc{1}{c}{\scriptsize{\textbf{(0.026)}}} & \mc{1}{c}{\scriptsize{\textbf{(0.000)}}} \\  

     & \mc{1}{c}{\scriptsize{15 to 21}} & \mc{1}{c}{\scriptsize{0.724}} & \mc{1}{c}{\scriptsize{0.617}} & \mc{1}{c}{\scriptsize{0.757}} & \mc{1}{c}{\scriptsize{0.521}} & \mc{1}{c}{\scriptsize{0.600}} & \mc{1}{c}{\scriptsize{0.745}} & \mc{1}{c}{\scriptsize{0.642}} & \mc{1}{c}{\scriptsize{0.664}} \\  

     &  & \mc{1}{c}{\scriptsize{\textbf{(0.000)}}} & \mc{1}{c}{\scriptsize{\textbf{(0.039)}}} & \mc{1}{c}{\scriptsize{\textbf{(0.000)}}} & \mc{1}{c}{\scriptsize{(0.132)}} & \mc{1}{c}{\scriptsize{\textbf{(0.000)}}} & \mc{1}{c}{\scriptsize{\textbf{(0.013)}}} & \mc{1}{c}{\scriptsize{\textbf{(0.053)}}} & \mc{1}{c}{\scriptsize{\textbf{(0.013)}}} \\  

    \mc{1}{l}{\scriptsize{PIAT Math Std. Score}} & \mc{1}{c}{\scriptsize{7}} & \mc{1}{c}{\scriptsize{2.182}} & \mc{1}{c}{\scriptsize{3.442}} & \mc{1}{c}{\scriptsize{8.773}} & \mc{1}{c}{\scriptsize{-3.145}} & \mc{1}{c}{\scriptsize{7.689}} & \mc{1}{c}{\scriptsize{1.141}} & \mc{1}{c}{\scriptsize{3.616}} & \mc{1}{c}{\scriptsize{2.106}} \\  

     &  & \mc{1}{c}{\scriptsize{(0.237)}} & \mc{1}{c}{\scriptsize{(0.237)}} & \mc{1}{c}{\scriptsize{\textbf{(0.000)}}} & \mc{1}{c}{\scriptsize{(0.724)}} & \mc{1}{c}{\scriptsize{\textbf{(0.000)}}} & \mc{1}{c}{\scriptsize{(0.355)}} & \mc{1}{c}{\scriptsize{(0.250)}} & \mc{1}{c}{\scriptsize{(0.237)}} \\ 
    \midrule  

    \mc{2}{l}{\scriptsize{\% of Pos. TE ($H_0$: $\le$ 50\%)}} & \mc{1}{c}{\scriptsize{100}} & \mc{1}{c}{\scriptsize{100}} & \mc{1}{c}{\scriptsize{100}} & \mc{1}{c}{\scriptsize{92}} & \mc{1}{c}{\scriptsize{100}} & \mc{1}{c}{\scriptsize{100}} & \mc{1}{c}{\scriptsize{100}} & \mc{1}{c}{\scriptsize{100}} \\  

     &  & \mc{1}{c}{\scriptsize{\textbf{(0.000)}}} & \mc{1}{c}{\scriptsize{\textbf{(0.000)}}} & \mc{1}{c}{\scriptsize{\textbf{(0.000)}}} & \mc{1}{c}{\scriptsize{\textbf{(0.000)}}} & \mc{1}{c}{\scriptsize{\textbf{(0.000)}}} & \mc{1}{c}{\scriptsize{\textbf{(0.000)}}} & \mc{1}{c}{\scriptsize{\textbf{(0.000)}}} & \mc{1}{c}{\scriptsize{\textbf{(0.000)}}} \\  

    \mc{2}{l}{\scriptsize{\% of Pos. TE ($H_0$: $\le$ 10\% $|$ 10\% Significance)}} & \mc{1}{c}{\scriptsize{92}} & \mc{1}{c}{\scriptsize{92}} & \mc{1}{c}{\scriptsize{100}} & \mc{1}{c}{\scriptsize{0}} & \mc{1}{c}{\scriptsize{92}} & \mc{1}{c}{\scriptsize{92}} & \mc{1}{c}{\scriptsize{75}} & \mc{1}{c}{\scriptsize{92}} \\  

     &  & \mc{1}{c}{\scriptsize{\textbf{(0.000)}}} & \mc{1}{c}{\scriptsize{\textbf{(0.000)}}} & \mc{1}{c}{\scriptsize{\textbf{(0.000)}}} & \mc{1}{c}{\scriptsize{(0.697)}} & \mc{1}{c}{\scriptsize{\textbf{(0.000)}}} & \mc{1}{c}{\scriptsize{\textbf{(0.000)}}} & \mc{1}{c}{\scriptsize{\textbf{(0.000)}}} & \mc{1}{c}{\scriptsize{\textbf{(0.000)}}} \\  

  \bottomrule
  \end{tabular}
\end{center}

\begin{center}
	  \begin{tabular}{cccccccccc}
  \toprule

    \scriptsize{Variable} & \scriptsize{Age} & \scriptsize{(1)} & \scriptsize{(2)} & \scriptsize{(3)} & \scriptsize{(4)} & \scriptsize{(5)} & \scriptsize{(6)} & \scriptsize{(7)} & \scriptsize{(8)} \\ 
    \midrule  

    \mc{1}{l}{\scriptsize{HOME Score}} & \mc{1}{c}{\scriptsize{0.5}} & \mc{1}{c}{\scriptsize{1.581}} & \mc{1}{c}{\scriptsize{0.380}} & \mc{1}{c}{\scriptsize{1.684}} & \mc{1}{c}{\scriptsize{0.946}} & \mc{1}{c}{\scriptsize{1.264}} & \mc{1}{c}{\scriptsize{0.980}} & \mc{1}{c}{\scriptsize{-0.045}} & \mc{1}{c}{\scriptsize{0.440}} \\  

     &  & \mc{1}{c}{\scriptsize{\textbf{(0.088)}}} & \mc{1}{c}{\scriptsize{(0.396)}} & \mc{1}{c}{\scriptsize{(0.168)}} & \mc{1}{c}{\scriptsize{(0.307)}} & \mc{1}{c}{\scriptsize{(0.235)}} & \mc{1}{c}{\scriptsize{(0.220)}} & \mc{1}{c}{\scriptsize{(0.480)}} & \mc{1}{c}{\scriptsize{(0.377)}} \\  

     & \mc{1}{c}{\scriptsize{1.5}} & \mc{1}{c}{\scriptsize{2.668}} & \mc{1}{c}{\scriptsize{2.107}} & \mc{1}{c}{\scriptsize{4.729}} & \mc{1}{c}{\scriptsize{3.783}} & \mc{1}{c}{\scriptsize{5.472}} & \mc{1}{c}{\scriptsize{1.544}} & \mc{1}{c}{\scriptsize{1.237}} & \mc{1}{c}{\scriptsize{1.756}} \\  

     &  & \mc{1}{c}{\scriptsize{\textbf{(0.026)}}} & \mc{1}{c}{\scriptsize{\textbf{(0.092)}}} & \mc{1}{c}{\scriptsize{\textbf{(0.023)}}} & \mc{1}{c}{\scriptsize{\textbf{(0.069)}}} & \mc{1}{c}{\scriptsize{\textbf{(0.014)}}} & \mc{1}{c}{\scriptsize{(0.167)}} & \mc{1}{c}{\scriptsize{(0.239)}} & \mc{1}{c}{\scriptsize{(0.140)}} \\  

     & \mc{1}{c}{\scriptsize{2.5}} & \mc{1}{c}{\scriptsize{0.762}} & \mc{1}{c}{\scriptsize{0.760}} & \mc{1}{c}{\scriptsize{4.434}} & \mc{1}{c}{\scriptsize{5.322}} & \mc{1}{c}{\scriptsize{5.173}} & \mc{1}{c}{\scriptsize{-0.899}} & \mc{1}{c}{\scriptsize{-1.068}} & \mc{1}{c}{\scriptsize{-0.252}} \\  

     &  & \mc{1}{c}{\scriptsize{(0.285)}} & \mc{1}{c}{\scriptsize{(0.300)}} & \mc{1}{c}{\scriptsize{\textbf{(0.004)}}} & \mc{1}{c}{\scriptsize{\textbf{(0.007)}}} & \mc{1}{c}{\scriptsize{\textbf{(0.001)}}} & \mc{1}{c}{\scriptsize{(0.277)}} & \mc{1}{c}{\scriptsize{(0.228)}} & \mc{1}{c}{\scriptsize{(0.435)}} \\  

     & \mc{1}{c}{\scriptsize{3.5}} & \mc{1}{c}{\scriptsize{2.858}} & \mc{1}{c}{\scriptsize{2.354}} & \mc{1}{c}{\scriptsize{13.719}} & \mc{1}{c}{\scriptsize{14.981}} & \mc{1}{c}{\scriptsize{14.927}} & \mc{1}{c}{\scriptsize{-0.309}} & \mc{1}{c}{\scriptsize{-1.804}} & \mc{1}{c}{\scriptsize{-0.048}} \\  

     &  & \mc{1}{c}{\scriptsize{\textbf{(0.096)}}} & \mc{1}{c}{\scriptsize{(0.188)}} & \mc{1}{c}{\scriptsize{\textbf{(0.000)}}} & \mc{1}{c}{\scriptsize{\textbf{(0.002)}}} & \mc{1}{c}{\scriptsize{\textbf{(0.001)}}} & \mc{1}{c}{\scriptsize{(0.441)}} & \mc{1}{c}{\scriptsize{(0.237)}} & \mc{1}{c}{\scriptsize{(0.500)}} \\  

     & \mc{1}{c}{\scriptsize{4.5}} & \mc{1}{c}{\scriptsize{2.736}} & \mc{1}{c}{\scriptsize{1.505}} & \mc{1}{c}{\scriptsize{12.957}} & \mc{1}{c}{\scriptsize{13.445}} & \mc{1}{c}{\scriptsize{13.953}} & \mc{1}{c}{\scriptsize{-0.273}} & \mc{1}{c}{\scriptsize{-1.703}} & \mc{1}{c}{\scriptsize{0.470}} \\  

     &  & \mc{1}{c}{\scriptsize{(0.140)}} & \mc{1}{c}{\scriptsize{(0.297)}} & \mc{1}{c}{\scriptsize{\textbf{(0.002)}}} & \mc{1}{c}{\scriptsize{\textbf{(0.014)}}} & \mc{1}{c}{\scriptsize{\textbf{(0.002)}}} & \mc{1}{c}{\scriptsize{(0.437)}} & \mc{1}{c}{\scriptsize{(0.275)}} & \mc{1}{c}{\scriptsize{(0.422)}} \\  

     & \mc{1}{c}{\scriptsize{8}} & \mc{1}{c}{\scriptsize{0.659}} & \mc{1}{c}{\scriptsize{1.112}} & \mc{1}{c}{\scriptsize{5.909}} & \mc{1}{c}{\scriptsize{8.035}} & \mc{1}{c}{\scriptsize{7.078}} & \mc{1}{c}{\scriptsize{-0.773}} & \mc{1}{c}{\scriptsize{-1.326}} & \mc{1}{c}{\scriptsize{0.447}} \\  

     &  & \mc{1}{c}{\scriptsize{(0.383)}} & \mc{1}{c}{\scriptsize{(0.304)}} & \mc{1}{c}{\scriptsize{(0.998)}} & \mc{1}{c}{\scriptsize{\textbf{(0.016)}}} & \mc{1}{c}{\scriptsize{\textbf{(0.031)}}} & \mc{1}{c}{\scriptsize{(0.359)}} & \mc{1}{c}{\scriptsize{(0.265)}} & \mc{1}{c}{\scriptsize{(0.428)}} \\  

    \mc{1}{l}{\scriptsize{HOME Factor}} & \mc{1}{c}{\scriptsize{0.5 to 8}} & \mc{1}{c}{\scriptsize{0.266}} & \mc{1}{c}{\scriptsize{0.179}} & \mc{1}{c}{\scriptsize{1.162}} & \mc{1}{c}{\scriptsize{1.281}} & \mc{1}{c}{\scriptsize{1.218}} & \mc{1}{c}{\scriptsize{0.010}} & \mc{1}{c}{\scriptsize{-0.169}} & \mc{1}{c}{\scriptsize{0.142}} \\  

     &  & \mc{1}{c}{\scriptsize{(0.196)}} & \mc{1}{c}{\scriptsize{(0.312)}} & \mc{1}{c}{\scriptsize{\textbf{(0.004)}}} & \mc{1}{c}{\scriptsize{\textbf{(0.021)}}} & \mc{1}{c}{\scriptsize{\textbf{(0.005)}}} & \mc{1}{c}{\scriptsize{(0.478)}} & \mc{1}{c}{\scriptsize{(0.336)}} & \mc{1}{c}{\scriptsize{(0.313)}} \\  

  \bottomrule
  \end{tabular}
\end{center}

\begin{center}
	  \begin{tabular}{cccccccccc}
  \toprule

    \scriptsize{Variable} & \scriptsize{Age} & \scriptsize{(1)} & \scriptsize{(2)} & \scriptsize{(3)} & \scriptsize{(4)} & \scriptsize{(5)} & \scriptsize{(6)} & \scriptsize{(7)} & \scriptsize{(8)} \\ 
    \midrule  

    \mc{1}{l}{\scriptsize{Ever Adopted}} &  & \mc{1}{c}{\scriptsize{0.044}} & \mc{1}{c}{\scriptsize{0.072}} & \mc{1}{c}{\scriptsize{-0.173}} & \mc{1}{c}{\scriptsize{-0.204}} & \mc{1}{c}{\scriptsize{-0.179}} & \mc{1}{c}{\scriptsize{0.077}} & \mc{1}{c}{\scriptsize{0.104}} & \mc{1}{c}{\scriptsize{0.065}} \\  

     &  & \mc{1}{c}{\scriptsize{(0.224)}} & \mc{1}{c}{\scriptsize{(0.250)}} & \mc{1}{c}{\scriptsize{(0.776)}} & \mc{1}{c}{\scriptsize{(0.789)}} & \mc{1}{c}{\scriptsize{(0.789)}} & \mc{1}{c}{\scriptsize{\textbf{(0.066)}}} & \mc{1}{c}{\scriptsize{\textbf{(0.039)}}} & \mc{1}{c}{\scriptsize{\textbf{(0.066)}}} \\ 
    \midrule  

    \mc{2}{l}{\scriptsize{\% of Pos. TE ($H_0$: $\le$ 50\%)}} & \mc{1}{c}{\scriptsize{100}} & \mc{1}{c}{\scriptsize{100}} & \mc{1}{c}{\scriptsize{0}} & \mc{1}{c}{\scriptsize{0}} & \mc{1}{c}{\scriptsize{0}} & \mc{1}{c}{\scriptsize{100}} & \mc{1}{c}{\scriptsize{100}} & \mc{1}{c}{\scriptsize{100}} \\  

     &  & \mc{1}{c}{\scriptsize{\textbf{(0.000)}}} & \mc{1}{c}{\scriptsize{\textbf{(0.000)}}} & \mc{1}{c}{\scriptsize{(0.974)}} & \mc{1}{c}{\scriptsize{(0.974)}} & \mc{1}{c}{\scriptsize{(0.934)}} & \mc{1}{c}{\scriptsize{\textbf{(0.000)}}} & \mc{1}{c}{\scriptsize{\textbf{(0.000)}}} & \mc{1}{c}{\scriptsize{\textbf{(0.000)}}} \\  

    \mc{2}{l}{\scriptsize{\% of Pos. TE ($H_0$: $\le$ 10\% $|$ 10\% Significance)}} & \mc{1}{c}{\scriptsize{0}} & \mc{1}{c}{\scriptsize{0}} & \mc{1}{c}{\scriptsize{0}} & \mc{1}{c}{\scriptsize{0}} & \mc{1}{c}{\scriptsize{0}} & \mc{1}{c}{\scriptsize{100}} & \mc{1}{c}{\scriptsize{100}} & \mc{1}{c}{\scriptsize{0}} \\  

     &  & \mc{1}{c}{\scriptsize{(0.197)}} & \mc{1}{c}{\scriptsize{(0.263)}} & \mc{1}{c}{\scriptsize{(0.171)}} & \mc{1}{c}{\scriptsize{(0.974)}} & \mc{1}{c}{\scriptsize{(0.118)}} & \mc{1}{c}{\scriptsize{\textbf{(0.000)}}} & \mc{1}{c}{\scriptsize{\textbf{(0.000)}}} & \mc{1}{c}{\scriptsize{(0.592)}} \\  

  \bottomrule
  \end{tabular}
\end{center}

\begin{center}
	  \begin{tabular}{cccccccccc}
  \toprule

    \scriptsize{Variable} & \scriptsize{Age} & \scriptsize{(1)} & \scriptsize{(2)} & \scriptsize{(3)} & \scriptsize{(4)} & \scriptsize{(5)} & \scriptsize{(6)} & \scriptsize{(7)} & \scriptsize{(8)} \\ 
    \midrule  

    \mc{1}{l}{\scriptsize{Mother Works}} & \mc{1}{c}{\scriptsize{2}} & \mc{1}{c}{\scriptsize{0.168}} & \mc{1}{c}{\scriptsize{0.112}} & \mc{1}{c}{\scriptsize{0.323}} & \mc{1}{c}{\scriptsize{0.297}} & \mc{1}{c}{\scriptsize{0.333}} & \mc{1}{c}{\scriptsize{0.101}} & \mc{1}{c}{\scriptsize{0.066}} & \mc{1}{c}{\scriptsize{0.097}} \\  

     &  & \mc{1}{c}{\scriptsize{\textbf{(0.035)}}} & \mc{1}{c}{\scriptsize{(0.137)}} & \mc{1}{c}{\scriptsize{\textbf{(0.050)}}} & \mc{1}{c}{\scriptsize{\textbf{(0.084)}}} & \mc{1}{c}{\scriptsize{\textbf{(0.051)}}} & \mc{1}{c}{\scriptsize{(0.158)}} & \mc{1}{c}{\scriptsize{(0.245)}} & \mc{1}{c}{\scriptsize{(0.174)}} \\  

     & \mc{1}{c}{\scriptsize{3}} & \mc{1}{c}{\scriptsize{0.087}} & \mc{1}{c}{\scriptsize{0.027}} & \mc{1}{c}{\scriptsize{0.177}} & \mc{1}{c}{\scriptsize{0.139}} & \mc{1}{c}{\scriptsize{0.179}} & \mc{1}{c}{\scriptsize{0.066}} & \mc{1}{c}{\scriptsize{-0.001}} & \mc{1}{c}{\scriptsize{0.058}} \\  

     &  & \mc{1}{c}{\scriptsize{(0.194)}} & \mc{1}{c}{\scriptsize{(0.399)}} & \mc{1}{c}{\scriptsize{(0.174)}} & \mc{1}{c}{\scriptsize{(0.237)}} & \mc{1}{c}{\scriptsize{(0.176)}} & \mc{1}{c}{\scriptsize{(0.263)}} & \mc{1}{c}{\scriptsize{(0.512)}} & \mc{1}{c}{\scriptsize{(0.306)}} \\  

     & \mc{1}{c}{\scriptsize{4}} & \mc{1}{c}{\scriptsize{0.118}} & \mc{1}{c}{\scriptsize{0.071}} & \mc{1}{c}{\scriptsize{0.319}} & \mc{1}{c}{\scriptsize{0.287}} & \mc{1}{c}{\scriptsize{0.328}} & \mc{1}{c}{\scriptsize{0.060}} & \mc{1}{c}{\scriptsize{0.025}} & \mc{1}{c}{\scriptsize{0.054}} \\  

     &  & \mc{1}{c}{\scriptsize{\textbf{(0.097)}}} & \mc{1}{c}{\scriptsize{(0.245)}} & \mc{1}{c}{\scriptsize{\textbf{(0.052)}}} & \mc{1}{c}{\scriptsize{\textbf{(0.087)}}} & \mc{1}{c}{\scriptsize{\textbf{(0.052)}}} & \mc{1}{c}{\scriptsize{(0.267)}} & \mc{1}{c}{\scriptsize{(0.390)}} & \mc{1}{c}{\scriptsize{(0.282)}} \\  

     & \mc{1}{c}{\scriptsize{5}} & \mc{1}{c}{\scriptsize{0.067}} & \mc{1}{c}{\scriptsize{0.038}} & \mc{1}{c}{\scriptsize{0.367}} & \mc{1}{c}{\scriptsize{0.276}} & \mc{1}{c}{\scriptsize{0.422}} & \mc{1}{c}{\scriptsize{-0.056}} & \mc{1}{c}{\scriptsize{-0.076}} & \mc{1}{c}{\scriptsize{-0.024}} \\  

     &  & \mc{1}{c}{\scriptsize{(0.243)}} & \mc{1}{c}{\scriptsize{(0.350)}} & \mc{1}{c}{\scriptsize{\textbf{(0.028)}}} & \mc{1}{c}{\scriptsize{\textbf{(0.082)}}} & \mc{1}{c}{\scriptsize{\textbf{(0.018)}}} & \mc{1}{c}{\scriptsize{(0.232)}} & \mc{1}{c}{\scriptsize{(0.162)}} & \mc{1}{c}{\scriptsize{(0.382)}} \\  

     & \mc{1}{c}{\scriptsize{21}} & \mc{1}{c}{\scriptsize{-0.018}} & \mc{1}{c}{\scriptsize{-0.005}} & \mc{1}{c}{\scriptsize{0.510}} & \mc{1}{c}{\scriptsize{0.497}} & \mc{1}{c}{\scriptsize{0.512}} & \mc{1}{c}{\scriptsize{-0.097}} & \mc{1}{c}{\scriptsize{-0.107}} & \mc{1}{c}{\scriptsize{-0.088}} \\  

     &  & \mc{1}{c}{\scriptsize{(0.441)}} & \mc{1}{c}{\scriptsize{(0.478)}} & \mc{1}{c}{\scriptsize{(0.985)}} & \mc{1}{c}{\scriptsize{(0.985)}} & \mc{1}{c}{\scriptsize{\textbf{(0.000)}}} & \mc{1}{c}{\scriptsize{(0.207)}} & \mc{1}{c}{\scriptsize{(0.214)}} & \mc{1}{c}{\scriptsize{(0.239)}} \\  

    \mc{1}{l}{\scriptsize{Mother Works Factor}} & \mc{1}{c}{\scriptsize{2 to 21}} & \mc{1}{c}{\scriptsize{-0.207}} & \mc{1}{c}{\scriptsize{-0.069}} & \mc{1}{c}{\scriptsize{-0.662}} & \mc{1}{c}{\scriptsize{-0.527}} & \mc{1}{c}{\scriptsize{-0.731}} & \mc{1}{c}{\scriptsize{-0.071}} & \mc{1}{c}{\scriptsize{0.081}} & \mc{1}{c}{\scriptsize{-0.092}} \\  

     &  & \mc{1}{c}{\scriptsize{(0.208)}} & \mc{1}{c}{\scriptsize{(0.381)}} & \mc{1}{c}{\scriptsize{\textbf{(0.098)}}} & \mc{1}{c}{\scriptsize{(0.156)}} & \mc{1}{c}{\scriptsize{\textbf{(0.088)}}} & \mc{1}{c}{\scriptsize{(0.385)}} & \mc{1}{c}{\scriptsize{(0.375)}} & \mc{1}{c}{\scriptsize{(0.361)}} \\  

  \bottomrule
  \end{tabular}
\end{center}

\begin{center}
	  \begin{tabular}{cccccccccc}
  \toprule

    \scriptsize{Variable} & \scriptsize{Age} & \scriptsize{(1)} & \scriptsize{(2)} & \scriptsize{(3)} & \scriptsize{(4)} & \scriptsize{(5)} & \scriptsize{(6)} & \scriptsize{(7)} & \scriptsize{(8)} \\ 
    \midrule  

    \mc{1}{l}{\scriptsize{Father at Home}} & \mc{1}{c}{\scriptsize{2}} & \mc{1}{c}{\scriptsize{0.018}} & \mc{1}{c}{\scriptsize{-0.166}} & \mc{1}{c}{\scriptsize{0.083}} &  & \mc{1}{c}{\scriptsize{-0.271}} & \mc{1}{c}{\scriptsize{-0.030}} & \mc{1}{c}{\scriptsize{-0.153}} & \mc{1}{c}{\scriptsize{0.067}} \\  

     &  & \mc{1}{c}{\scriptsize{(0.487)}} & \mc{1}{c}{\scriptsize{(0.816)}} & \mc{1}{c}{\scriptsize{(0.395)}} &  & \mc{1}{c}{\scriptsize{(0.750)}} & \mc{1}{c}{\scriptsize{(0.500)}} & \mc{1}{c}{\scriptsize{(0.605)}} & \mc{1}{c}{\scriptsize{(0.355)}} \\  

     & \mc{1}{c}{\scriptsize{3}} & \mc{1}{c}{\scriptsize{-0.088}} & \mc{1}{c}{\scriptsize{-0.205}} & \mc{1}{c}{\scriptsize{-0.292}} & \mc{1}{c}{\scriptsize{-0.260}} & \mc{1}{c}{\scriptsize{-0.579}} & \mc{1}{c}{\scriptsize{0.061}} & \mc{1}{c}{\scriptsize{-0.191}} & \mc{1}{c}{\scriptsize{0.089}} \\  

     &  & \mc{1}{c}{\scriptsize{(0.645)}} & \mc{1}{c}{\scriptsize{(0.842)}} & \mc{1}{c}{\scriptsize{(0.816)}} & \mc{1}{c}{\scriptsize{(0.395)}} & \mc{1}{c}{\scriptsize{(0.974)}} & \mc{1}{c}{\scriptsize{(0.342)}} & \mc{1}{c}{\scriptsize{(0.632)}} & \mc{1}{c}{\scriptsize{(0.368)}} \\  

     & \mc{1}{c}{\scriptsize{4}} & \mc{1}{c}{\scriptsize{-0.126}} & \mc{1}{c}{\scriptsize{-0.182}} & \mc{1}{c}{\scriptsize{-0.475}} & \mc{1}{c}{\scriptsize{-1.000}} & \mc{1}{c}{\scriptsize{-0.622}} & \mc{1}{c}{\scriptsize{0.127}} & \mc{1}{c}{\scriptsize{0.041}} & \mc{1}{c}{\scriptsize{0.120}} \\  

     &  & \mc{1}{c}{\scriptsize{(0.658)}} & \mc{1}{c}{\scriptsize{(0.789)}} & \mc{1}{c}{\scriptsize{(0.921)}} & \mc{1}{c}{\scriptsize{(0.697)}} & \mc{1}{c}{\scriptsize{(0.934)}} & \mc{1}{c}{\scriptsize{(0.289)}} & \mc{1}{c}{\scriptsize{(0.434)}} & \mc{1}{c}{\scriptsize{(0.329)}} \\  

     & \mc{1}{c}{\scriptsize{5}} & \mc{1}{c}{\scriptsize{-0.276}} & \mc{1}{c}{\scriptsize{-0.387}} & \mc{1}{c}{\scriptsize{-0.625}} & \mc{1}{c}{\scriptsize{-1.000}} & \mc{1}{c}{\scriptsize{-0.803}} & \mc{1}{c}{\scriptsize{-0.023}} & \mc{1}{c}{\scriptsize{0.022}} & \mc{1}{c}{\scriptsize{-0.096}} \\  

     &  & \mc{1}{c}{\scriptsize{(0.803)}} & \mc{1}{c}{\scriptsize{(0.895)}} & \mc{1}{c}{\scriptsize{(0.987)}} & \mc{1}{c}{\scriptsize{(0.763)}} & \mc{1}{c}{\scriptsize{(0.987)}} & \mc{1}{c}{\scriptsize{(0.434)}} & \mc{1}{c}{\scriptsize{(0.487)}} & \mc{1}{c}{\scriptsize{(0.526)}} \\  

     & \mc{1}{c}{\scriptsize{8}} & \mc{1}{c}{\scriptsize{-0.042}} & \mc{1}{c}{\scriptsize{-0.250}} & \mc{1}{c}{\scriptsize{-0.167}} &  & \mc{1}{c}{\scriptsize{-0.452}} & \mc{1}{c}{\scriptsize{0.333}} &  & \mc{1}{c}{\scriptsize{0.043}} \\  

     &  & \mc{1}{c}{\scriptsize{(0.447)}} & \mc{1}{c}{\scriptsize{(0.329)}} & \mc{1}{c}{\scriptsize{(0.605)}} &  & \mc{1}{c}{\scriptsize{(0.789)}} & \mc{1}{c}{\scriptsize{\textbf{(0.079)}}} &  & \mc{1}{c}{\scriptsize{\textbf{(0.066)}}} \\  

    \mc{1}{l}{\scriptsize{Father at Home Factor}} & \mc{1}{c}{\scriptsize{2 to 8}} & \mc{1}{c}{\scriptsize{-0.692}} & \mc{1}{c}{\scriptsize{-2.154}} & \mc{1}{c}{\scriptsize{-1.160}} & \mc{1}{c}{\scriptsize{-1.834}} & \mc{1}{c}{\scriptsize{-1.603}} & \mc{1}{c}{\scriptsize{0.712}} &  & \mc{1}{c}{\scriptsize{0.092}} \\  

     &  & \mc{1}{c}{\scriptsize{(0.750)}} & \mc{1}{c}{\scriptsize{(0.711)}} & \mc{1}{c}{\scriptsize{(1.000)}} & \mc{1}{c}{\scriptsize{(0.737)}} & \mc{1}{c}{\scriptsize{(0.961)}} & \mc{1}{c}{\scriptsize{\textbf{(0.079)}}} &  & \mc{1}{c}{\scriptsize{\textbf{(0.053)}}} \\ 
    \midrule  

    \mc{2}{l}{\scriptsize{\% of Pos. TE ($H_0$: $\le$ 50\%)}} & \mc{1}{c}{\scriptsize{17}} & \mc{1}{c}{\scriptsize{0}} & \mc{1}{c}{\scriptsize{17}} & \mc{1}{c}{\scriptsize{0}} & \mc{1}{c}{\scriptsize{0}} & \mc{1}{c}{\scriptsize{67}} & \mc{1}{c}{\scriptsize{50}} & \mc{1}{c}{\scriptsize{83}} \\  

     &  & \mc{1}{c}{\scriptsize{(0.697)}} & \mc{1}{c}{\scriptsize{(1.000)}} & \mc{1}{c}{\scriptsize{(1.000)}} & \mc{1}{c}{\scriptsize{(0.908)}} & \mc{1}{c}{\scriptsize{(1.000)}} & \mc{1}{c}{\scriptsize{(0.539)}} & \mc{1}{c}{\scriptsize{(0.329)}} & \mc{1}{c}{\scriptsize{(0.368)}} \\  

    \mc{2}{l}{\scriptsize{\% of Pos. TE ($H_0$: $\le$ 10\% $|$ 10\% Significance)}} & \mc{1}{c}{\scriptsize{0}} & \mc{1}{c}{\scriptsize{0}} & \mc{1}{c}{\scriptsize{0}} & \mc{1}{c}{\scriptsize{0}} & \mc{1}{c}{\scriptsize{0}} & \mc{1}{c}{\scriptsize{33}} & \mc{1}{c}{\scriptsize{0}} & \mc{1}{c}{\scriptsize{33}} \\  

     &  & \mc{1}{c}{\scriptsize{(1.000)}} & \mc{1}{c}{\scriptsize{(1.000)}} & \mc{1}{c}{\scriptsize{(1.000)}} & \mc{1}{c}{\scriptsize{(0.237)}} & \mc{1}{c}{\scriptsize{(1.000)}} & \mc{1}{c}{\scriptsize{(0.211)}} & \mc{1}{c}{\scriptsize{(0.211)}} & \mc{1}{c}{\scriptsize{(0.145)}} \\  

  \bottomrule
  \end{tabular}
\end{center}

\begin{center}
	  \begin{tabular}{cccccccccc}
  \toprule

    \scriptsize{Variable} & \scriptsize{Age} & \scriptsize{(1)} & \scriptsize{(2)} & \scriptsize{(3)} & \scriptsize{(4)} & \scriptsize{(5)} & \scriptsize{(6)} & \scriptsize{(7)} & \scriptsize{(8)} \\ 
    \midrule  

    \mc{1}{l}{\scriptsize{Graduated High School}} & \mc{1}{c}{\scriptsize{30}} & \mc{1}{c}{\scriptsize{0.253}} & \mc{1}{c}{\scriptsize{0.148}} & \mc{1}{c}{\scriptsize{0.642}} & \mc{1}{c}{\scriptsize{0.519}} & \mc{1}{c}{\scriptsize{0.595}} & \mc{1}{c}{\scriptsize{0.137}} & \mc{1}{c}{\scriptsize{0.004}} & \mc{1}{c}{\scriptsize{0.066}} \\  

     &  & \mc{1}{c}{\scriptsize{\textbf{(0.000)}}} & \mc{1}{c}{\scriptsize{\textbf{(0.066)}}} & \mc{1}{c}{\scriptsize{\textbf{(0.000)}}} & \mc{1}{c}{\scriptsize{\textbf{(0.013)}}} & \mc{1}{c}{\scriptsize{\textbf{(0.000)}}} & \mc{1}{c}{\scriptsize{(0.158)}} & \mc{1}{c}{\scriptsize{(0.461)}} & \mc{1}{c}{\scriptsize{(0.263)}} \\  

    \mc{1}{l}{\scriptsize{Attended Voc./Tech./Com. College}} & \mc{1}{c}{\scriptsize{30}} & \mc{1}{c}{\scriptsize{-0.057}} & \mc{1}{c}{\scriptsize{-0.115}} & \mc{1}{c}{\scriptsize{-0.050}} & \mc{1}{c}{\scriptsize{-0.098}} & \mc{1}{c}{\scriptsize{-0.071}} & \mc{1}{c}{\scriptsize{-0.041}} & \mc{1}{c}{\scriptsize{-0.145}} & \mc{1}{c}{\scriptsize{-0.051}} \\  

     &  & \mc{1}{c}{\scriptsize{(0.684)}} & \mc{1}{c}{\scriptsize{(0.803)}} & \mc{1}{c}{\scriptsize{(0.618)}} & \mc{1}{c}{\scriptsize{(0.671)}} & \mc{1}{c}{\scriptsize{(0.697)}} & \mc{1}{c}{\scriptsize{(0.618)}} & \mc{1}{c}{\scriptsize{(0.895)}} & \mc{1}{c}{\scriptsize{(0.724)}} \\  

    \mc{1}{l}{\scriptsize{Graduated 4-year College}} & \mc{1}{c}{\scriptsize{30}} & \mc{1}{c}{\scriptsize{0.134}} & \mc{1}{c}{\scriptsize{0.102}} & \mc{1}{c}{\scriptsize{0.217}} & \mc{1}{c}{\scriptsize{0.097}} & \mc{1}{c}{\scriptsize{0.210}} & \mc{1}{c}{\scriptsize{0.106}} & \mc{1}{c}{\scriptsize{0.073}} & \mc{1}{c}{\scriptsize{0.095}} \\  

     &  & \mc{1}{c}{\scriptsize{\textbf{(0.066)}}} & \mc{1}{c}{\scriptsize{(0.158)}} & \mc{1}{c}{\scriptsize{\textbf{(0.013)}}} & \mc{1}{c}{\scriptsize{(0.184)}} & \mc{1}{c}{\scriptsize{\textbf{(0.000)}}} & \mc{1}{c}{\scriptsize{(0.145)}} & \mc{1}{c}{\scriptsize{(0.316)}} & \mc{1}{c}{\scriptsize{(0.197)}} \\  

    \mc{1}{l}{\scriptsize{Years of Edu.}} & \mc{1}{c}{\scriptsize{30}} & \mc{1}{c}{\scriptsize{2.143}} & \mc{1}{c}{\scriptsize{1.695}} & \mc{1}{c}{\scriptsize{4.025}} & \mc{1}{c}{\scriptsize{2.984}} & \mc{1}{c}{\scriptsize{3.918}} & \mc{1}{c}{\scriptsize{1.567}} & \mc{1}{c}{\scriptsize{1.155}} & \mc{1}{c}{\scriptsize{1.409}} \\  

     &  & \mc{1}{c}{\scriptsize{\textbf{(0.000)}}} & \mc{1}{c}{\scriptsize{\textbf{(0.000)}}} & \mc{1}{c}{\scriptsize{\textbf{(0.000)}}} & \mc{1}{c}{\scriptsize{\textbf{(0.000)}}} & \mc{1}{c}{\scriptsize{\textbf{(0.000)}}} & \mc{1}{c}{\scriptsize{\textbf{(0.013)}}} & \mc{1}{c}{\scriptsize{\textbf{(0.066)}}} & \mc{1}{c}{\scriptsize{\textbf{(0.026)}}} \\  

    \mc{1}{l}{\scriptsize{Education Factor}} & \mc{1}{c}{\scriptsize{30}} & \mc{1}{c}{\scriptsize{0.661}} & \mc{1}{c}{\scriptsize{0.461}} & \mc{1}{c}{\scriptsize{1.277}} & \mc{1}{c}{\scriptsize{0.799}} & \mc{1}{c}{\scriptsize{1.212}} & \mc{1}{c}{\scriptsize{0.447}} & \mc{1}{c}{\scriptsize{0.261}} & \mc{1}{c}{\scriptsize{0.361}} \\  

     &  & \mc{1}{c}{\scriptsize{\textbf{(0.000)}}} & \mc{1}{c}{\scriptsize{\textbf{(0.039)}}} & \mc{1}{c}{\scriptsize{\textbf{(0.000)}}} & \mc{1}{c}{\scriptsize{\textbf{(0.039)}}} & \mc{1}{c}{\scriptsize{\textbf{(0.000)}}} & \mc{1}{c}{\scriptsize{\textbf{(0.039)}}} & \mc{1}{c}{\scriptsize{(0.250)}} & \mc{1}{c}{\scriptsize{(0.105)}} \\ 
    \midrule  

    \mc{2}{l}{\scriptsize{\% of Pos. TE ($H_0$: $\le$ 50\%)}} & \mc{1}{c}{\scriptsize{80}} & \mc{1}{c}{\scriptsize{80}} & \mc{1}{c}{\scriptsize{80}} & \mc{1}{c}{\scriptsize{80}} & \mc{1}{c}{\scriptsize{80}} & \mc{1}{c}{\scriptsize{80}} & \mc{1}{c}{\scriptsize{80}} & \mc{1}{c}{\scriptsize{80}} \\  

     &  & \mc{1}{c}{\scriptsize{\textbf{(0.000)}}} & \mc{1}{c}{\scriptsize{\textbf{(0.000)}}} & \mc{1}{c}{\scriptsize{\textbf{(0.000)}}} & \mc{1}{c}{\scriptsize{\textbf{(0.000)}}} & \mc{1}{c}{\scriptsize{\textbf{(0.000)}}} & \mc{1}{c}{\scriptsize{\textbf{(0.000)}}} & \mc{1}{c}{\scriptsize{\textbf{(0.026)}}} & \mc{1}{c}{\scriptsize{\textbf{(0.000)}}} \\  

    \mc{2}{l}{\scriptsize{\% of Pos. TE ($H_0$: $\le$ 10\% $|$ 10\% Significance)}} & \mc{1}{c}{\scriptsize{80}} & \mc{1}{c}{\scriptsize{40}} & \mc{1}{c}{\scriptsize{80}} & \mc{1}{c}{\scriptsize{60}} & \mc{1}{c}{\scriptsize{80}} & \mc{1}{c}{\scriptsize{40}} & \mc{1}{c}{\scriptsize{20}} & \mc{1}{c}{\scriptsize{40}} \\  

     &  & \mc{1}{c}{\scriptsize{\textbf{(0.000)}}} & \mc{1}{c}{\scriptsize{(0.118)}} & \mc{1}{c}{\scriptsize{\textbf{(0.000)}}} & \mc{1}{c}{\scriptsize{\textbf{(0.000)}}} & \mc{1}{c}{\scriptsize{\textbf{(0.000)}}} & \mc{1}{c}{\scriptsize{(0.132)}} & \mc{1}{c}{\scriptsize{(0.250)}} & \mc{1}{c}{\scriptsize{(0.211)}} \\  

  \bottomrule
  \end{tabular}
\end{center}

\begin{center}
	  \begin{tabular}{cccccccccc}
  \toprule

    \scriptsize{Variable} & \scriptsize{Age} & \scriptsize{(1)} & \scriptsize{(2)} & \scriptsize{(3)} & \scriptsize{(4)} & \scriptsize{(5)} & \scriptsize{(6)} & \scriptsize{(7)} & \scriptsize{(8)} \\ 
    \midrule  

    \mc{1}{l}{\scriptsize{Behavioral conduct}} & \mc{1}{c}{\scriptsize{12}} & \mc{1}{c}{\scriptsize{-0.156}} & \mc{1}{c}{\scriptsize{-0.706}} & \mc{1}{c}{\scriptsize{-0.727}} & \mc{1}{c}{\scriptsize{-0.503}} & \mc{1}{c}{\scriptsize{-0.742}} & \mc{1}{c}{\scriptsize{-0.112}} & \mc{1}{c}{\scriptsize{-0.630}} & \mc{1}{c}{\scriptsize{-0.131}} \\  

     &  & \mc{1}{c}{\scriptsize{(0.711)}} & \mc{1}{c}{\scriptsize{(0.763)}} & \mc{1}{c}{\scriptsize{(0.671)}} & \mc{1}{c}{\scriptsize{(0.276)}} & \mc{1}{c}{\scriptsize{(0.658)}} & \mc{1}{c}{\scriptsize{(0.632)}} & \mc{1}{c}{\scriptsize{(0.697)}} & \mc{1}{c}{\scriptsize{(0.671)}} \\  

    \mc{1}{l}{\scriptsize{Physical appearance}} & \mc{1}{c}{\scriptsize{12}} & \mc{1}{c}{\scriptsize{-0.646}} & \mc{1}{c}{\scriptsize{-0.492}} & \mc{1}{c}{\scriptsize{-1.682}} & \mc{1}{c}{\scriptsize{-0.050}} & \mc{1}{c}{\scriptsize{-1.661}} & \mc{1}{c}{\scriptsize{-0.566}} & \mc{1}{c}{\scriptsize{-0.450}} & \mc{1}{c}{\scriptsize{-0.557}} \\  

     &  & \mc{1}{c}{\scriptsize{(0.974)}} & \mc{1}{c}{\scriptsize{(0.711)}} & \mc{1}{c}{\scriptsize{(0.671)}} & \mc{1}{c}{\scriptsize{(0.184)}} & \mc{1}{c}{\scriptsize{(0.671)}} & \mc{1}{c}{\scriptsize{(0.961)}} & \mc{1}{c}{\scriptsize{(0.724)}} & \mc{1}{c}{\scriptsize{(0.947)}} \\  

    \mc{1}{l}{\scriptsize{Social acceptance}} & \mc{1}{c}{\scriptsize{12}} & \mc{1}{c}{\scriptsize{-0.471}} & \mc{1}{c}{\scriptsize{-0.495}} & \mc{1}{c}{\scriptsize{-1.364}} & \mc{1}{c}{\scriptsize{-0.878}} & \mc{1}{c}{\scriptsize{-1.276}} & \mc{1}{c}{\scriptsize{-0.402}} & \mc{1}{c}{\scriptsize{-0.378}} & \mc{1}{c}{\scriptsize{-0.297}} \\  

     &  & \mc{1}{c}{\scriptsize{(0.934)}} & \mc{1}{c}{\scriptsize{(0.711)}} & \mc{1}{c}{\scriptsize{(0.671)}} & \mc{1}{c}{\scriptsize{(0.368)}} & \mc{1}{c}{\scriptsize{(0.671)}} & \mc{1}{c}{\scriptsize{(0.921)}} & \mc{1}{c}{\scriptsize{(0.711)}} & \mc{1}{c}{\scriptsize{(0.750)}} \\  

  \bottomrule
  \end{tabular}
\end{center}

\begin{center}
	  \begin{tabular}{cccccccccc}
  \toprule

    \scriptsize{Variable} & \scriptsize{Age} & \scriptsize{(1)} & \scriptsize{(2)} & \scriptsize{(3)} & \scriptsize{(4)} & \scriptsize{(5)} & \scriptsize{(6)} & \scriptsize{(7)} & \scriptsize{(8)} \\ 
    \midrule  

    \mc{1}{l}{\scriptsize{Total Felony Arrests}} & \mc{1}{c}{\scriptsize{Mid-30s}} & \mc{1}{c}{\scriptsize{-0.328}} & \mc{1}{c}{\scriptsize{-0.394}} & \mc{1}{c}{\scriptsize{-1.345}} & \mc{1}{c}{\scriptsize{-1.006}} & \mc{1}{c}{\scriptsize{-0.965}} & \mc{1}{c}{\scriptsize{-0.077}} & \mc{1}{c}{\scriptsize{-0.083}} & \mc{1}{c}{\scriptsize{0.005}} \\  

     &  & \mc{1}{c}{\scriptsize{(0.109)}} & \mc{1}{c}{\scriptsize{\textbf{(0.099)}}} & \mc{1}{c}{\scriptsize{\textbf{(0.099)}}} & \mc{1}{c}{\scriptsize{(0.158)}} & \mc{1}{c}{\scriptsize{\textbf{(0.079)}}} & \mc{1}{c}{\scriptsize{(0.218)}} & \mc{1}{c}{\scriptsize{(0.277)}} & \mc{1}{c}{\scriptsize{(0.515)}} \\  

    \mc{1}{l}{\scriptsize{Total Misdemeanor Arrests}} & \mc{1}{c}{\scriptsize{Mid-30s}} & \mc{1}{c}{\scriptsize{-0.973}} & \mc{1}{c}{\scriptsize{-1.212}} & \mc{1}{c}{\scriptsize{-2.708}} & \mc{1}{c}{\scriptsize{-2.303}} & \mc{1}{c}{\scriptsize{-2.448}} & \mc{1}{c}{\scriptsize{-0.588}} & \mc{1}{c}{\scriptsize{-0.466}} & \mc{1}{c}{\scriptsize{-0.201}} \\  

     &  & \mc{1}{c}{\scriptsize{\textbf{(0.069)}}} & \mc{1}{c}{\scriptsize{(0.109)}} & \mc{1}{c}{\scriptsize{\textbf{(0.099)}}} & \mc{1}{c}{\scriptsize{(0.139)}} & \mc{1}{c}{\scriptsize{(0.109)}} & \mc{1}{c}{\scriptsize{(0.109)}} & \mc{1}{c}{\scriptsize{(0.208)}} & \mc{1}{c}{\scriptsize{(0.277)}} \\  

    \mc{1}{l}{\scriptsize{Total Years Incarcerated}} & \mc{1}{c}{\scriptsize{30}} & \mc{1}{c}{\scriptsize{-0.024}} & \mc{1}{c}{\scriptsize{-0.010}} &  &  &  & \mc{1}{c}{\scriptsize{-0.037}} & \mc{1}{c}{\scriptsize{-0.020}} & \mc{1}{c}{\scriptsize{-0.038}} \\  

     &  & \mc{1}{c}{\scriptsize{\textbf{(0.089)}}} & \mc{1}{c}{\scriptsize{(0.178)}} &  &  &  & \mc{1}{c}{\scriptsize{\textbf{(0.079)}}} & \mc{1}{c}{\scriptsize{\textbf{(0.099)}}} & \mc{1}{c}{\scriptsize{\textbf{(0.089)}}} \\  

    \mc{1}{l}{\scriptsize{Crime Factor}} & \mc{1}{c}{\scriptsize{30 to Mid-30s}} & \mc{1}{c}{\scriptsize{-0.239}} & \mc{1}{c}{\scriptsize{-0.304}} & \mc{1}{c}{\scriptsize{-0.735}} & \mc{1}{c}{\scriptsize{-0.764}} & \mc{1}{c}{\scriptsize{-0.725}} & \mc{1}{c}{\scriptsize{-0.124}} & \mc{1}{c}{\scriptsize{-0.108}} & \mc{1}{c}{\scriptsize{-0.070}} \\  

     &  & \mc{1}{c}{\scriptsize{\textbf{(0.079)}}} & \mc{1}{c}{\scriptsize{(0.129)}} & \mc{1}{c}{\scriptsize{(0.109)}} & \mc{1}{c}{\scriptsize{(0.129)}} & \mc{1}{c}{\scriptsize{(0.119)}} & \mc{1}{c}{\scriptsize{(0.139)}} & \mc{1}{c}{\scriptsize{(0.158)}} & \mc{1}{c}{\scriptsize{(0.218)}} \\  

  \bottomrule
  \end{tabular}
\end{center}

\begin{center}
	  \begin{tabular}{cccccccccc}
  \toprule

    \scriptsize{Variable} & \scriptsize{Age} & \scriptsize{(1)} & \scriptsize{(2)} & \scriptsize{(3)} & \scriptsize{(4)} & \scriptsize{(5)} & \scriptsize{(6)} & \scriptsize{(7)} & \scriptsize{(8)} \\ 
    \midrule  

    \mc{1}{l}{\scriptsize{Cig. Smoked per day last month}} & \mc{1}{c}{\scriptsize{30}} & \mc{1}{c}{\scriptsize{-0.772}} & \mc{1}{c}{\scriptsize{1.073}} & \mc{1}{c}{\scriptsize{-2.650}} & \mc{1}{c}{\scriptsize{-0.052}} & \mc{1}{c}{\scriptsize{-2.447}} & \mc{1}{c}{\scriptsize{-0.279}} & \mc{1}{c}{\scriptsize{1.318}} & \mc{1}{c}{\scriptsize{-0.045}} \\  

     &  & \mc{1}{c}{\scriptsize{(0.303)}} & \mc{1}{c}{\scriptsize{(0.750)}} & \mc{1}{c}{\scriptsize{(0.132)}} & \mc{1}{c}{\scriptsize{(0.526)}} & \mc{1}{c}{\scriptsize{(0.132)}} & \mc{1}{c}{\scriptsize{(0.421)}} & \mc{1}{c}{\scriptsize{(0.776)}} & \mc{1}{c}{\scriptsize{(0.461)}} \\  

    \mc{1}{l}{\scriptsize{Days drank alcohol last month}} & \mc{1}{c}{\scriptsize{30}} & \mc{1}{c}{\scriptsize{-1.077}} & \mc{1}{c}{\scriptsize{0.377}} & \mc{1}{c}{\scriptsize{-2.233}} & \mc{1}{c}{\scriptsize{-1.652}} & \mc{1}{c}{\scriptsize{-1.934}} & \mc{1}{c}{\scriptsize{-0.779}} & \mc{1}{c}{\scriptsize{0.509}} & \mc{1}{c}{\scriptsize{-0.349}} \\  

     &  & \mc{1}{c}{\scriptsize{(0.211)}} & \mc{1}{c}{\scriptsize{(0.618)}} & \mc{1}{c}{\scriptsize{(0.197)}} & \mc{1}{c}{\scriptsize{(0.237)}} & \mc{1}{c}{\scriptsize{(0.224)}} & \mc{1}{c}{\scriptsize{(0.303)}} & \mc{1}{c}{\scriptsize{(0.658)}} & \mc{1}{c}{\scriptsize{(0.382)}} \\  

    \mc{1}{l}{\scriptsize{Days binge drank alcohol last month}} & \mc{1}{c}{\scriptsize{30}} & \mc{1}{c}{\scriptsize{-0.144}} & \mc{1}{c}{\scriptsize{0.360}} & \mc{1}{c}{\scriptsize{-0.542}} & \mc{1}{c}{\scriptsize{-0.470}} & \mc{1}{c}{\scriptsize{-0.402}} & \mc{1}{c}{\scriptsize{-0.053}} & \mc{1}{c}{\scriptsize{0.647}} & \mc{1}{c}{\scriptsize{0.194}} \\  

     &  & \mc{1}{c}{\scriptsize{(0.461)}} & \mc{1}{c}{\scriptsize{(0.684)}} & \mc{1}{c}{\scriptsize{(0.329)}} & \mc{1}{c}{\scriptsize{(0.408)}} & \mc{1}{c}{\scriptsize{(0.355)}} & \mc{1}{c}{\scriptsize{(0.513)}} & \mc{1}{c}{\scriptsize{(0.776)}} & \mc{1}{c}{\scriptsize{(0.645)}} \\  

    \mc{1}{l}{\scriptsize{Self-reported drug user}} & \mc{1}{c}{\scriptsize{Mid-30s}} & \mc{1}{c}{\scriptsize{0.007}} & \mc{1}{c}{\scriptsize{-0.060}} & \mc{1}{c}{\scriptsize{-0.083}} & \mc{1}{c}{\scriptsize{-0.214}} & \mc{1}{c}{\scriptsize{-0.167}} & \mc{1}{c}{\scriptsize{0.039}} & \mc{1}{c}{\scriptsize{0.028}} & \mc{1}{c}{\scriptsize{-0.024}} \\  

     &  & \mc{1}{c}{\scriptsize{(0.566)}} & \mc{1}{c}{\scriptsize{(0.289)}} & \mc{1}{c}{\scriptsize{(0.263)}} & \mc{1}{c}{\scriptsize{(0.224)}} & \mc{1}{c}{\scriptsize{(0.171)}} & \mc{1}{c}{\scriptsize{(0.605)}} & \mc{1}{c}{\scriptsize{(0.592)}} & \mc{1}{c}{\scriptsize{(0.447)}} \\  

    \mc{1}{l}{\scriptsize{Substance Use Factor}} & \mc{1}{c}{\scriptsize{30 to Mid-30s}} & \mc{1}{c}{\scriptsize{0.017}} & \mc{1}{c}{\scriptsize{0.405}} & \mc{1}{c}{\scriptsize{0.090}} & \mc{1}{c}{\scriptsize{0.474}} & \mc{1}{c}{\scriptsize{0.065}} & \mc{1}{c}{\scriptsize{-0.009}} & \mc{1}{c}{\scriptsize{0.407}} & \mc{1}{c}{\scriptsize{0.005}} \\  

     &  & \mc{1}{c}{\scriptsize{(0.579)}} & \mc{1}{c}{\scriptsize{(0.934)}} & \mc{1}{c}{\scriptsize{(0.579)}} & \mc{1}{c}{\scriptsize{(0.921)}} & \mc{1}{c}{\scriptsize{(0.539)}} & \mc{1}{c}{\scriptsize{(0.526)}} & \mc{1}{c}{\scriptsize{(0.934)}} & \mc{1}{c}{\scriptsize{(0.513)}} \\ 
    \midrule  

    \mc{2}{l}{\scriptsize{\% of Pos. TE ($H_0$: $\le$ 50\%)}} & \mc{1}{c}{\scriptsize{60}} & \mc{1}{c}{\scriptsize{20}} & \mc{1}{c}{\scriptsize{80}} & \mc{1}{c}{\scriptsize{80}} & \mc{1}{c}{\scriptsize{80}} & \mc{1}{c}{\scriptsize{80}} & \mc{1}{c}{\scriptsize{0}} & \mc{1}{c}{\scriptsize{60}} \\  

     &  & \mc{1}{c}{\scriptsize{(0.474)}} & \mc{1}{c}{\scriptsize{(0.763)}} & \mc{1}{c}{\scriptsize{(0.171)}} & \mc{1}{c}{\scriptsize{(0.118)}} & \mc{1}{c}{\scriptsize{(0.171)}} & \mc{1}{c}{\scriptsize{(0.145)}} & \mc{1}{c}{\scriptsize{(1.000)}} & \mc{1}{c}{\scriptsize{(0.513)}} \\  

    \mc{2}{l}{\scriptsize{\% of Pos. TE ($H_0$: $\le$ 10\% $|$ 10\% Significance)}} & \mc{1}{c}{\scriptsize{0}} & \mc{1}{c}{\scriptsize{0}} & \mc{1}{c}{\scriptsize{0}} & \mc{1}{c}{\scriptsize{0}} & \mc{1}{c}{\scriptsize{0}} & \mc{1}{c}{\scriptsize{0}} & \mc{1}{c}{\scriptsize{0}} & \mc{1}{c}{\scriptsize{0}} \\  

     &  & \mc{1}{c}{\scriptsize{(1.000)}} & \mc{1}{c}{\scriptsize{(1.000)}} & \mc{1}{c}{\scriptsize{(0.316)}} & \mc{1}{c}{\scriptsize{(1.000)}} & \mc{1}{c}{\scriptsize{(0.382)}} & \mc{1}{c}{\scriptsize{(1.000)}} & \mc{1}{c}{\scriptsize{(1.000)}} & \mc{1}{c}{\scriptsize{(1.000)}} \\  

  \bottomrule
  \end{tabular}
\end{center}

\begin{center}
	  \begin{tabular}{cccccccccc}
  \toprule

    \scriptsize{Variable} & \scriptsize{Age} & \scriptsize{(1)} & \scriptsize{(2)} & \scriptsize{(3)} & \scriptsize{(4)} & \scriptsize{(5)} & \scriptsize{(6)} & \scriptsize{(7)} & \scriptsize{(8)} \\ 
    \midrule  

    \mc{1}{l}{\scriptsize{Self-reported Health}} & \mc{1}{c}{\scriptsize{30}} & \mc{1}{c}{\scriptsize{-0.274}} & \mc{1}{c}{\scriptsize{-0.328}} & \mc{1}{c}{\scriptsize{-0.425}} & \mc{1}{c}{\scriptsize{-3.250}} & \mc{1}{c}{\scriptsize{-0.915}} & \mc{1}{c}{\scriptsize{-0.164}} & \mc{1}{c}{\scriptsize{-0.141}} & \mc{1}{c}{\scriptsize{-0.176}} \\  

     &  & \mc{1}{c}{\scriptsize{(0.184)}} & \mc{1}{c}{\scriptsize{(0.171)}} & \mc{1}{c}{\scriptsize{(0.197)}} & \mc{1}{c}{\scriptsize{\textbf{(0.039)}}} & \mc{1}{c}{\scriptsize{(0.158)}} & \mc{1}{c}{\scriptsize{(0.263)}} & \mc{1}{c}{\scriptsize{(0.566)}} & \mc{1}{c}{\scriptsize{(0.342)}} \\  

     & \mc{1}{c}{\scriptsize{Mid-30s}} & \mc{1}{c}{\scriptsize{0.350}} & \mc{1}{c}{\scriptsize{0.589}} & \mc{1}{c}{\scriptsize{0.029}} & \mc{1}{c}{\scriptsize{0.610}} & \mc{1}{c}{\scriptsize{0.209}} & \mc{1}{c}{\scriptsize{0.600}} & \mc{1}{c}{\scriptsize{0.512}} & \mc{1}{c}{\scriptsize{0.545}} \\  

     &  & \mc{1}{c}{\scriptsize{(0.776)}} & \mc{1}{c}{\scriptsize{(0.842)}} & \mc{1}{c}{\scriptsize{(0.447)}} & \mc{1}{c}{\scriptsize{(0.461)}} & \mc{1}{c}{\scriptsize{(0.605)}} & \mc{1}{c}{\scriptsize{(0.961)}} & \mc{1}{c}{\scriptsize{(0.592)}} & \mc{1}{c}{\scriptsize{(0.855)}} \\  

    \mc{1}{l}{\scriptsize{Self-reported Health Factor}} & \mc{1}{c}{\scriptsize{30 to Mid-30s}} & \mc{1}{c}{\scriptsize{0.157}} & \mc{1}{c}{\scriptsize{0.407}} & \mc{1}{c}{\scriptsize{-0.012}} & \mc{1}{c}{\scriptsize{0.111}} & \mc{1}{c}{\scriptsize{0.256}} & \mc{1}{c}{\scriptsize{0.289}} & \mc{1}{c}{\scriptsize{0.160}} & \mc{1}{c}{\scriptsize{0.283}} \\  

     &  & \mc{1}{c}{\scriptsize{(0.671)}} & \mc{1}{c}{\scriptsize{(0.789)}} & \mc{1}{c}{\scriptsize{(0.408)}} & \mc{1}{c}{\scriptsize{(0.303)}} & \mc{1}{c}{\scriptsize{(0.803)}} & \mc{1}{c}{\scriptsize{(0.776)}} & \mc{1}{c}{\scriptsize{(0.500)}} & \mc{1}{c}{\scriptsize{(0.750)}} \\ 
    \midrule  

    \mc{2}{l}{\scriptsize{\% of Pos. TE ($H_0$: $\le$ 50\%)}} & \mc{1}{c}{\scriptsize{33}} & \mc{1}{c}{\scriptsize{33}} & \mc{1}{c}{\scriptsize{67}} & \mc{1}{c}{\scriptsize{33}} & \mc{1}{c}{\scriptsize{33}} & \mc{1}{c}{\scriptsize{33}} & \mc{1}{c}{\scriptsize{33}} & \mc{1}{c}{\scriptsize{33}} \\  

     &  & \mc{1}{c}{\scriptsize{(0.974)}} & \mc{1}{c}{\scriptsize{(0.908)}} & \mc{1}{c}{\scriptsize{(0.184)}} & \mc{1}{c}{\scriptsize{(0.605)}} & \mc{1}{c}{\scriptsize{(0.855)}} & \mc{1}{c}{\scriptsize{(0.895)}} & \mc{1}{c}{\scriptsize{(0.855)}} & \mc{1}{c}{\scriptsize{(0.842)}} \\  

    \mc{2}{l}{\scriptsize{\% of Pos. TE ($H_0$: $\le$ 10\% $|$ 10\% Significance)}} & \mc{1}{c}{\scriptsize{0}} & \mc{1}{c}{\scriptsize{0}} & \mc{1}{c}{\scriptsize{0}} & \mc{1}{c}{\scriptsize{33}} & \mc{1}{c}{\scriptsize{0}} & \mc{1}{c}{\scriptsize{0}} & \mc{1}{c}{\scriptsize{0}} & \mc{1}{c}{\scriptsize{0}} \\  

     &  & \mc{1}{c}{\scriptsize{(0.303)}} & \mc{1}{c}{\scriptsize{(1.000)}} & \mc{1}{c}{\scriptsize{(0.250)}} & \mc{1}{c}{\scriptsize{(0.184)}} & \mc{1}{c}{\scriptsize{(0.329)}} & \mc{1}{c}{\scriptsize{(1.000)}} & \mc{1}{c}{\scriptsize{(0.289)}} & \mc{1}{c}{\scriptsize{(0.289)}} \\  

  \bottomrule
  \end{tabular}
\end{center}

\begin{center}
	  \begin{tabular}{cccccccccc}
  \toprule

    \scriptsize{Variable} & \scriptsize{Age} & \scriptsize{(1)} & \scriptsize{(2)} & \scriptsize{(3)} & \scriptsize{(4)} & \scriptsize{(5)} & \scriptsize{(6)} & \scriptsize{(7)} & \scriptsize{(8)} \\ 
    \midrule  

    \mc{1}{l}{\scriptsize{Cig. Smoked per day last month}} & \mc{1}{c}{\scriptsize{30}} & \mc{1}{c}{\scriptsize{-0.319}} & \mc{1}{c}{\scriptsize{-0.340}} & \mc{1}{c}{\scriptsize{-2.140}} & \mc{1}{c}{\scriptsize{-1.592}} & \mc{1}{c}{\scriptsize{-1.788}} & \mc{1}{c}{\scriptsize{-0.163}} & \mc{1}{c}{\scriptsize{-0.184}} & \mc{1}{c}{\scriptsize{-0.371}} \\  

     &  & \mc{1}{c}{\scriptsize{(0.408)}} & \mc{1}{c}{\scriptsize{(0.355)}} & \mc{1}{c}{\scriptsize{(0.250)}} & \mc{1}{c}{\scriptsize{(0.368)}} & \mc{1}{c}{\scriptsize{(0.276)}} & \mc{1}{c}{\scriptsize{(0.421)}} & \mc{1}{c}{\scriptsize{(0.382)}} & \mc{1}{c}{\scriptsize{(0.329)}} \\  

    \mc{1}{l}{\scriptsize{Days drank alcohol last month}} & \mc{1}{c}{\scriptsize{30}} & \mc{1}{c}{\scriptsize{-0.356}} & \mc{1}{c}{\scriptsize{1.389}} & \mc{1}{c}{\scriptsize{-0.820}} & \mc{1}{c}{\scriptsize{-2.538}} & \mc{1}{c}{\scriptsize{-0.075}} & \mc{1}{c}{\scriptsize{-0.411}} & \mc{1}{c}{\scriptsize{1.869}} & \mc{1}{c}{\scriptsize{0.510}} \\  

     &  & \mc{1}{c}{\scriptsize{(0.382)}} & \mc{1}{c}{\scriptsize{(0.763)}} & \mc{1}{c}{\scriptsize{(0.368)}} & \mc{1}{c}{\scriptsize{(0.250)}} & \mc{1}{c}{\scriptsize{(0.500)}} & \mc{1}{c}{\scriptsize{(0.355)}} & \mc{1}{c}{\scriptsize{(0.803)}} & \mc{1}{c}{\scriptsize{(0.539)}} \\  

    \mc{1}{l}{\scriptsize{Days binge drank alcohol last month}} & \mc{1}{c}{\scriptsize{30}} & \mc{1}{c}{\scriptsize{-0.573}} & \mc{1}{c}{\scriptsize{0.578}} & \mc{1}{c}{\scriptsize{-2.805}} & \mc{1}{c}{\scriptsize{-1.778}} & \mc{1}{c}{\scriptsize{-2.331}} & \mc{1}{c}{\scriptsize{-0.294}} & \mc{1}{c}{\scriptsize{0.863}} & \mc{1}{c}{\scriptsize{0.060}} \\  

     &  & \mc{1}{c}{\scriptsize{(0.250)}} & \mc{1}{c}{\scriptsize{(0.671)}} & \mc{1}{c}{\scriptsize{(0.158)}} & \mc{1}{c}{\scriptsize{(0.316)}} & \mc{1}{c}{\scriptsize{(0.237)}} & \mc{1}{c}{\scriptsize{(0.382)}} & \mc{1}{c}{\scriptsize{(0.711)}} & \mc{1}{c}{\scriptsize{(0.579)}} \\  

    \mc{1}{l}{\scriptsize{Self-reported drug user}} & \mc{1}{c}{\scriptsize{Mid-30s}} & \mc{1}{c}{\scriptsize{0.051}} & \mc{1}{c}{\scriptsize{0.056}} & \mc{1}{c}{\scriptsize{-0.389}} & \mc{1}{c}{\scriptsize{-0.408}} & \mc{1}{c}{\scriptsize{-0.376}} & \mc{1}{c}{\scriptsize{0.120}} & \mc{1}{c}{\scriptsize{0.207}} & \mc{1}{c}{\scriptsize{0.121}} \\  

     &  & \mc{1}{c}{\scriptsize{(0.618)}} & \mc{1}{c}{\scriptsize{(0.618)}} & \mc{1}{c}{\scriptsize{\textbf{(0.092)}}} & \mc{1}{c}{\scriptsize{(0.237)}} & \mc{1}{c}{\scriptsize{(0.158)}} & \mc{1}{c}{\scriptsize{(0.803)}} & \mc{1}{c}{\scriptsize{(0.882)}} & \mc{1}{c}{\scriptsize{(0.842)}} \\  

    \mc{1}{l}{\scriptsize{Substance Use Factor}} & \mc{1}{c}{\scriptsize{30 to Mid-30s}} & \mc{1}{c}{\scriptsize{0.037}} & \mc{1}{c}{\scriptsize{0.524}} & \mc{1}{c}{\scriptsize{0.331}} & \mc{1}{c}{\scriptsize{0.310}} & \mc{1}{c}{\scriptsize{0.346}} & \mc{1}{c}{\scriptsize{-0.009}} & \mc{1}{c}{\scriptsize{0.553}} & \mc{1}{c}{\scriptsize{0.162}} \\  

     &  & \mc{1}{c}{\scriptsize{(0.539)}} & \mc{1}{c}{\scriptsize{(0.987)}} & \mc{1}{c}{\scriptsize{(0.934)}} & \mc{1}{c}{\scriptsize{(0.645)}} & \mc{1}{c}{\scriptsize{(0.934)}} & \mc{1}{c}{\scriptsize{(0.461)}} & \mc{1}{c}{\scriptsize{(0.974)}} & \mc{1}{c}{\scriptsize{(0.711)}} \\ 
    \midrule  

    \mc{2}{l}{\scriptsize{\% of Pos. TE ($H_0$: $\le$ 50\%)}} & \mc{1}{c}{\scriptsize{60}} & \mc{1}{c}{\scriptsize{20}} & \mc{1}{c}{\scriptsize{80}} & \mc{1}{c}{\scriptsize{80}} & \mc{1}{c}{\scriptsize{80}} & \mc{1}{c}{\scriptsize{80}} & \mc{1}{c}{\scriptsize{20}} & \mc{1}{c}{\scriptsize{20}} \\  

     &  & \mc{1}{c}{\scriptsize{(0.434)}} & \mc{1}{c}{\scriptsize{(0.776)}} & \mc{1}{c}{\scriptsize{\textbf{(0.039)}}} & \mc{1}{c}{\scriptsize{(0.303)}} & \mc{1}{c}{\scriptsize{\textbf{(0.053)}}} & \mc{1}{c}{\scriptsize{(0.171)}} & \mc{1}{c}{\scriptsize{(1.000)}} & \mc{1}{c}{\scriptsize{(0.829)}} \\  

    \mc{2}{l}{\scriptsize{\% of Pos. TE ($H_0$: $\le$ 10\% $|$ 10\% Significance)}} & \mc{1}{c}{\scriptsize{0}} & \mc{1}{c}{\scriptsize{0}} & \mc{1}{c}{\scriptsize{20}} & \mc{1}{c}{\scriptsize{0}} & \mc{1}{c}{\scriptsize{0}} & \mc{1}{c}{\scriptsize{0}} & \mc{1}{c}{\scriptsize{0}} & \mc{1}{c}{\scriptsize{0}} \\  

     &  & \mc{1}{c}{\scriptsize{(0.329)}} & \mc{1}{c}{\scriptsize{(1.000)}} & \mc{1}{c}{\scriptsize{\textbf{(0.066)}}} & \mc{1}{c}{\scriptsize{(0.342)}} & \mc{1}{c}{\scriptsize{(0.539)}} & \mc{1}{c}{\scriptsize{(1.000)}} & \mc{1}{c}{\scriptsize{(1.000)}} & \mc{1}{c}{\scriptsize{(1.000)}} \\  

  \bottomrule
  \end{tabular}
\end{center}

\begin{center}
	  \begin{tabular}{cccccccccc}
  \toprule

    \scriptsize{Variable} & \scriptsize{Age} & \scriptsize{(1)} & \scriptsize{(2)} & \scriptsize{(3)} & \scriptsize{(4)} & \scriptsize{(5)} & \scriptsize{(6)} & \scriptsize{(7)} & \scriptsize{(8)} \\ 
    \midrule  

    \mc{1}{l}{\scriptsize{High-Density Lipoprotein Chol. (mg/dL)}} & \mc{1}{c}{\scriptsize{Mid-30s}} & \mc{1}{c}{\scriptsize{2.884}} & \mc{1}{c}{\scriptsize{6.643}} & \mc{1}{c}{\scriptsize{10.514}} & \mc{1}{c}{\scriptsize{8.119}} & \mc{1}{c}{\scriptsize{13.513}} & \mc{1}{c}{\scriptsize{0.802}} &  & \mc{1}{c}{\scriptsize{3.246}} \\  

     &  & \mc{1}{c}{\scriptsize{(0.228)}} & \mc{1}{c}{\scriptsize{\textbf{(0.059)}}} & \mc{1}{c}{\scriptsize{\textbf{(0.010)}}} & \mc{1}{c}{\scriptsize{(0.129)}} & \mc{1}{c}{\scriptsize{\textbf{(0.010)}}} & \mc{1}{c}{\scriptsize{(0.436)}} &  & \mc{1}{c}{\scriptsize{(0.248)}} \\  

    \mc{1}{l}{\scriptsize{Dyslipidemia}} & \mc{1}{c}{\scriptsize{Mid-30s}} & \mc{1}{c}{\scriptsize{0.051}} & \mc{1}{c}{\scriptsize{0.023}} & \mc{1}{c}{\scriptsize{-0.080}} & \mc{1}{c}{\scriptsize{-0.088}} & \mc{1}{c}{\scriptsize{-0.146}} & \mc{1}{c}{\scriptsize{0.087}} & \mc{1}{c}{\scriptsize{0.078}} & \mc{1}{c}{\scriptsize{0.089}} \\  

     &  & \mc{1}{c}{\scriptsize{(0.752)}} & \mc{1}{c}{\scriptsize{(0.505)}} & \mc{1}{c}{\scriptsize{(0.307)}} & \mc{1}{c}{\scriptsize{(0.426)}} & \mc{1}{c}{\scriptsize{(0.267)}} & \mc{1}{c}{\scriptsize{(0.861)}} & \mc{1}{c}{\scriptsize{(0.802)}} & \mc{1}{c}{\scriptsize{(0.772)}} \\  

    \mc{1}{l}{\scriptsize{Cholesterol Factor}} & \mc{1}{c}{\scriptsize{Mid-30s}} & \mc{1}{c}{\scriptsize{0.034}} & \mc{1}{c}{\scriptsize{0.120}} & \mc{1}{c}{\scriptsize{0.568}} & \mc{1}{c}{\scriptsize{0.309}} & \mc{1}{c}{\scriptsize{0.600}} & \mc{1}{c}{\scriptsize{-0.111}} & \mc{1}{c}{\scriptsize{0.030}} & \mc{1}{c}{\scriptsize{-0.077}} \\  

     &  & \mc{1}{c}{\scriptsize{(0.515)}} & \mc{1}{c}{\scriptsize{(0.644)}} & \mc{1}{c}{\scriptsize{(0.941)}} & \mc{1}{c}{\scriptsize{(0.663)}} & \mc{1}{c}{\scriptsize{(0.891)}} & \mc{1}{c}{\scriptsize{(0.287)}} & \mc{1}{c}{\scriptsize{(0.525)}} & \mc{1}{c}{\scriptsize{(0.356)}} \\  

  \bottomrule
  \end{tabular}
\end{center}

\begin{center}
	  \begin{tabular}{cccccccccc}
  \toprule

    \scriptsize{Variable} & \scriptsize{Age} & \scriptsize{(1)} & \scriptsize{(2)} & \scriptsize{(3)} & \scriptsize{(4)} & \scriptsize{(5)} & \scriptsize{(6)} & \scriptsize{(7)} & \scriptsize{(8)} \\ 
    \midrule  

    \mc{1}{l}{\scriptsize{Hemoglobin Level (\%)}} & \mc{1}{c}{\scriptsize{Mid-30s}} & \mc{1}{c}{\scriptsize{-0.277}} & \mc{1}{c}{\scriptsize{-0.101}} & \mc{1}{c}{\scriptsize{-0.176}} & \mc{1}{c}{\scriptsize{-0.088}} & \mc{1}{c}{\scriptsize{-0.190}} & \mc{1}{c}{\scriptsize{-0.304}} & \mc{1}{c}{\scriptsize{-0.037}} & \mc{1}{c}{\scriptsize{-0.355}} \\  

     &  & \mc{1}{c}{\scriptsize{(0.211)}} & \mc{1}{c}{\scriptsize{(0.316)}} & \mc{1}{c}{\scriptsize{\textbf{(0.092)}}} & \mc{1}{c}{\scriptsize{(0.355)}} & \mc{1}{c}{\scriptsize{(0.105)}} & \mc{1}{c}{\scriptsize{(0.211)}} & \mc{1}{c}{\scriptsize{(0.474)}} & \mc{1}{c}{\scriptsize{(0.197)}} \\  

    \mc{1}{l}{\scriptsize{Prediabetes}} & \mc{1}{c}{\scriptsize{Mid-30s}} & \mc{1}{c}{\scriptsize{0.088}} & \mc{1}{c}{\scriptsize{0.163}} & \mc{1}{c}{\scriptsize{0.076}} & \mc{1}{c}{\scriptsize{0.176}} & \mc{1}{c}{\scriptsize{0.035}} & \mc{1}{c}{\scriptsize{0.091}} & \mc{1}{c}{\scriptsize{0.161}} & \mc{1}{c}{\scriptsize{0.029}} \\  

     &  & \mc{1}{c}{\scriptsize{(0.737)}} & \mc{1}{c}{\scriptsize{(0.789)}} & \mc{1}{c}{\scriptsize{(0.632)}} & \mc{1}{c}{\scriptsize{(0.763)}} & \mc{1}{c}{\scriptsize{(0.526)}} & \mc{1}{c}{\scriptsize{(0.697)}} & \mc{1}{c}{\scriptsize{(0.750)}} & \mc{1}{c}{\scriptsize{(0.579)}} \\  

    \mc{1}{l}{\scriptsize{Diabetes}} & \mc{1}{c}{\scriptsize{Mid-30s}} & \mc{1}{c}{\scriptsize{-0.071}} & \mc{1}{c}{\scriptsize{-0.032}} &  &  &  & \mc{1}{c}{\scriptsize{-0.091}} & \mc{1}{c}{\scriptsize{-0.039}} & \mc{1}{c}{\scriptsize{-0.095}} \\  

     &  & \mc{1}{c}{\scriptsize{\textbf{(0.066)}}} & \mc{1}{c}{\scriptsize{(0.171)}} &  &  &  & \mc{1}{c}{\scriptsize{\textbf{(0.066)}}} & \mc{1}{c}{\scriptsize{(0.171)}} & \mc{1}{c}{\scriptsize{\textbf{(0.039)}}} \\  

    \mc{1}{l}{\scriptsize{Diabetes Factor}} & \mc{1}{c}{\scriptsize{Mid-30s}} & \mc{1}{c}{\scriptsize{-0.249}} & \mc{1}{c}{\scriptsize{-0.086}} & \mc{1}{c}{\scriptsize{-0.086}} & \mc{1}{c}{\scriptsize{-0.029}} & \mc{1}{c}{\scriptsize{-0.098}} & \mc{1}{c}{\scriptsize{-0.294}} & \mc{1}{c}{\scriptsize{-0.064}} & \mc{1}{c}{\scriptsize{-0.334}} \\  

     &  & \mc{1}{c}{\scriptsize{(0.145)}} & \mc{1}{c}{\scriptsize{(0.250)}} & \mc{1}{c}{\scriptsize{(0.211)}} & \mc{1}{c}{\scriptsize{(0.382)}} & \mc{1}{c}{\scriptsize{(0.158)}} & \mc{1}{c}{\scriptsize{(0.145)}} & \mc{1}{c}{\scriptsize{(0.408)}} & \mc{1}{c}{\scriptsize{(0.145)}} \\ 
    \midrule  

    \mc{2}{l}{\scriptsize{\% of Pos. TE ($H_0$: $\le$ 50\%)}} & \mc{1}{c}{\scriptsize{75}} & \mc{1}{c}{\scriptsize{75}} & \mc{1}{c}{\scriptsize{67}} & \mc{1}{c}{\scriptsize{67}} & \mc{1}{c}{\scriptsize{67}} & \mc{1}{c}{\scriptsize{75}} & \mc{1}{c}{\scriptsize{75}} & \mc{1}{c}{\scriptsize{75}} \\  

     &  & \mc{1}{c}{\scriptsize{\textbf{(0.000)}}} & \mc{1}{c}{\scriptsize{(0.158)}} & \mc{1}{c}{\scriptsize{(0.395)}} & \mc{1}{c}{\scriptsize{(0.592)}} & \mc{1}{c}{\scriptsize{(0.395)}} & \mc{1}{c}{\scriptsize{(0.276)}} & \mc{1}{c}{\scriptsize{(0.158)}} & \mc{1}{c}{\scriptsize{\textbf{(0.000)}}} \\  

    \mc{2}{l}{\scriptsize{\% of Pos. TE ($H_0$: $\le$ 10\% $|$ 10\% Significance)}} & \mc{1}{c}{\scriptsize{50}} & \mc{1}{c}{\scriptsize{0}} & \mc{1}{c}{\scriptsize{0}} & \mc{1}{c}{\scriptsize{0}} & \mc{1}{c}{\scriptsize{0}} & \mc{1}{c}{\scriptsize{25}} & \mc{1}{c}{\scriptsize{0}} & \mc{1}{c}{\scriptsize{50}} \\  

     &  & \mc{1}{c}{\scriptsize{\textbf{(0.066)}}} & \mc{1}{c}{\scriptsize{(1.000)}} & \mc{1}{c}{\scriptsize{(0.368)}} & \mc{1}{c}{\scriptsize{(1.000)}} & \mc{1}{c}{\scriptsize{(0.342)}} & \mc{1}{c}{\scriptsize{(0.237)}} & \mc{1}{c}{\scriptsize{(1.000)}} & \mc{1}{c}{\scriptsize{\textbf{(0.066)}}} \\  

  \bottomrule
  \end{tabular}
\end{center}

\begin{center}
	  \begin{tabular}{cccccccccc}
  \toprule

    \scriptsize{Variable} & \scriptsize{Age} & \scriptsize{(1)} & \scriptsize{(2)} & \scriptsize{(3)} & \scriptsize{(4)} & \scriptsize{(5)} & \scriptsize{(6)} & \scriptsize{(7)} & \scriptsize{(8)} \\ 
    \midrule  

    \mc{1}{l}{\scriptsize{Measured BMI}} & \mc{1}{c}{\scriptsize{Mid-30s}} & \mc{1}{c}{\scriptsize{3.545}} & \mc{1}{c}{\scriptsize{6.421}} & \mc{1}{c}{\scriptsize{1.937}} & \mc{1}{c}{\scriptsize{2.932}} & \mc{1}{c}{\scriptsize{1.978}} & \mc{1}{c}{\scriptsize{3.983}} & \mc{1}{c}{\scriptsize{7.424}} & \mc{1}{c}{\scriptsize{4.711}} \\  

     &  & \mc{1}{c}{\scriptsize{(0.901)}} & \mc{1}{c}{\scriptsize{(0.911)}} & \mc{1}{c}{\scriptsize{\textbf{(0.010)}}} & \mc{1}{c}{\scriptsize{(0.703)}} & \mc{1}{c}{\scriptsize{(0.723)}} & \mc{1}{c}{\scriptsize{(0.931)}} & \mc{1}{c}{\scriptsize{(0.921)}} & \mc{1}{c}{\scriptsize{(0.941)}} \\  

    \mc{1}{l}{\scriptsize{Obesity}} & \mc{1}{c}{\scriptsize{Mid-30s}} & \mc{1}{c}{\scriptsize{-0.011}} & \mc{1}{c}{\scriptsize{0.143}} & \mc{1}{c}{\scriptsize{-0.261}} &  & \mc{1}{c}{\scriptsize{-0.199}} & \mc{1}{c}{\scriptsize{0.057}} & \mc{1}{c}{\scriptsize{0.209}} & \mc{1}{c}{\scriptsize{0.109}} \\  

     &  & \mc{1}{c}{\scriptsize{(0.465)}} & \mc{1}{c}{\scriptsize{(0.812)}} & \mc{1}{c}{\scriptsize{\textbf{(0.010)}}} &  & \mc{1}{c}{\scriptsize{\textbf{(0.030)}}} & \mc{1}{c}{\scriptsize{(0.693)}} & \mc{1}{c}{\scriptsize{(0.842)}} & \mc{1}{c}{\scriptsize{(0.762)}} \\  

    \mc{1}{l}{\scriptsize{Severe Obesity}} & \mc{1}{c}{\scriptsize{Mid-30s}} & \mc{1}{c}{\scriptsize{-0.045}} & \mc{1}{c}{\scriptsize{0.087}} & \mc{1}{c}{\scriptsize{0.014}} & \mc{1}{c}{\scriptsize{0.134}} & \mc{1}{c}{\scriptsize{0.020}} & \mc{1}{c}{\scriptsize{-0.061}} & \mc{1}{c}{\scriptsize{0.076}} & \mc{1}{c}{\scriptsize{-0.040}} \\  

     &  & \mc{1}{c}{\scriptsize{(0.386)}} & \mc{1}{c}{\scriptsize{(0.594)}} & \mc{1}{c}{\scriptsize{\textbf{(0.010)}}} & \mc{1}{c}{\scriptsize{(0.604)}} & \mc{1}{c}{\scriptsize{(0.545)}} & \mc{1}{c}{\scriptsize{(0.366)}} & \mc{1}{c}{\scriptsize{(0.604)}} & \mc{1}{c}{\scriptsize{(0.426)}} \\  

    \mc{1}{l}{\scriptsize{Waist-hip Ratio}} & \mc{1}{c}{\scriptsize{Mid-30s}} & \mc{1}{c}{\scriptsize{-0.022}} &  & \mc{1}{c}{\scriptsize{-0.076}} & \mc{1}{c}{\scriptsize{-0.052}} & \mc{1}{c}{\scriptsize{-0.072}} & \mc{1}{c}{\scriptsize{-0.007}} & \mc{1}{c}{\scriptsize{0.042}} & \mc{1}{c}{\scriptsize{0.014}} \\  

     &  & \mc{1}{c}{\scriptsize{(0.228)}} &  & \mc{1}{c}{\scriptsize{(0.990)}} & \mc{1}{c}{\scriptsize{(0.327)}} & \mc{1}{c}{\scriptsize{(0.139)}} & \mc{1}{c}{\scriptsize{(0.465)}} & \mc{1}{c}{\scriptsize{(0.812)}} & \mc{1}{c}{\scriptsize{(0.693)}} \\  

    \mc{1}{l}{\scriptsize{Abdominal Obesity}} & \mc{1}{c}{\scriptsize{Mid-30s}} & \mc{1}{c}{\scriptsize{-0.159}} & \mc{1}{c}{\scriptsize{0.030}} & \mc{1}{c}{\scriptsize{-0.381}} & \mc{1}{c}{\scriptsize{-0.101}} & \mc{1}{c}{\scriptsize{-0.284}} & \mc{1}{c}{\scriptsize{-0.095}} & \mc{1}{c}{\scriptsize{0.076}} & \mc{1}{c}{\scriptsize{0.022}} \\  

     &  & \mc{1}{c}{\scriptsize{(0.119)}} & \mc{1}{c}{\scriptsize{(0.574)}} & \mc{1}{c}{\scriptsize{(0.990)}} & \mc{1}{c}{\scriptsize{(0.228)}} & \mc{1}{c}{\scriptsize{\textbf{(0.010)}}} & \mc{1}{c}{\scriptsize{(0.238)}} & \mc{1}{c}{\scriptsize{(0.693)}} & \mc{1}{c}{\scriptsize{(0.564)}} \\  

    \mc{1}{l}{\scriptsize{Framingham Risk Score}} & \mc{1}{c}{\scriptsize{Mid-30s}} & \mc{1}{c}{\scriptsize{-0.259}} & \mc{1}{c}{\scriptsize{-0.260}} & \mc{1}{c}{\scriptsize{-0.488}} & \mc{1}{c}{\scriptsize{-0.545}} & \mc{1}{c}{\scriptsize{-0.526}} & \mc{1}{c}{\scriptsize{-0.197}} & \mc{1}{c}{\scriptsize{-0.105}} & \mc{1}{c}{\scriptsize{-0.221}} \\  

     &  & \mc{1}{c}{\scriptsize{\textbf{(0.079)}}} & \mc{1}{c}{\scriptsize{(0.139)}} & \mc{1}{c}{\scriptsize{(0.990)}} & \mc{1}{c}{\scriptsize{\textbf{(0.089)}}} & \mc{1}{c}{\scriptsize{\textbf{(0.099)}}} & \mc{1}{c}{\scriptsize{(0.228)}} & \mc{1}{c}{\scriptsize{(0.317)}} & \mc{1}{c}{\scriptsize{(0.198)}} \\  

    \mc{1}{l}{\scriptsize{Obesity Factor}} & \mc{1}{c}{\scriptsize{Mid-30s}} & \mc{1}{c}{\scriptsize{-0.006}} & \mc{1}{c}{\scriptsize{-0.411}} & \mc{1}{c}{\scriptsize{0.433}} & \mc{1}{c}{\scriptsize{0.132}} & \mc{1}{c}{\scriptsize{0.364}} & \mc{1}{c}{\scriptsize{-0.132}} & \mc{1}{c}{\scriptsize{-0.611}} & \mc{1}{c}{\scriptsize{-0.256}} \\  

     &  & \mc{1}{c}{\scriptsize{(0.495)}} & \mc{1}{c}{\scriptsize{(0.228)}} & \mc{1}{c}{\scriptsize{\textbf{(0.010)}}} & \mc{1}{c}{\scriptsize{(0.644)}} & \mc{1}{c}{\scriptsize{(0.723)}} & \mc{1}{c}{\scriptsize{(0.337)}} & \mc{1}{c}{\scriptsize{(0.248)}} & \mc{1}{c}{\scriptsize{(0.257)}} \\  

  \bottomrule
  \end{tabular}
\end{center}

\begin{center}
	  \begin{tabular}{cccccccccc}
  \toprule

    \scriptsize{Variable} & \scriptsize{Age} & \scriptsize{(1)} & \scriptsize{(2)} & \scriptsize{(3)} & \scriptsize{(4)} & \scriptsize{(5)} & \scriptsize{(6)} & \scriptsize{(7)} & \scriptsize{(8)} \\ 
    \midrule  

    \mc{1}{l}{\scriptsize{Somatization $t$-Score}} & \mc{1}{c}{\scriptsize{21}} & \mc{1}{c}{\scriptsize{-2.671}} & \mc{1}{c}{\scriptsize{-2.668}} & \mc{1}{c}{\scriptsize{-4.893}} & \mc{1}{c}{\scriptsize{-4.641}} & \mc{1}{c}{\scriptsize{-4.852}} & \mc{1}{c}{\scriptsize{-2.258}} & \mc{1}{c}{\scriptsize{-2.403}} & \mc{1}{c}{\scriptsize{-2.170}} \\  

     &  & \mc{1}{c}{\scriptsize{(0.137)}} & \mc{1}{c}{\scriptsize{(0.183)}} & \mc{1}{c}{\scriptsize{(0.999)}} & \mc{1}{c}{\scriptsize{(0.999)}} & \mc{1}{c}{\scriptsize{(0.158)}} & \mc{1}{c}{\scriptsize{(0.180)}} & \mc{1}{c}{\scriptsize{(0.224)}} & \mc{1}{c}{\scriptsize{(0.225)}} \\  

     & \mc{1}{c}{\scriptsize{Mid-30s}} & \mc{1}{c}{\scriptsize{0.724}} & \mc{1}{c}{\scriptsize{2.202}} & \mc{1}{c}{\scriptsize{-0.014}} &  & \mc{1}{c}{\scriptsize{0.571}} & \mc{1}{c}{\scriptsize{0.925}} & \mc{1}{c}{\scriptsize{3.638}} & \mc{1}{c}{\scriptsize{1.724}} \\  

     &  & \mc{1}{c}{\scriptsize{(0.410)}} & \mc{1}{c}{\scriptsize{(0.213)}} & \mc{1}{c}{\scriptsize{\textbf{(0.001)}}} &  & \mc{1}{c}{\scriptsize{(0.501)}} & \mc{1}{c}{\scriptsize{(0.399)}} & \mc{1}{c}{\scriptsize{(0.123)}} & \mc{1}{c}{\scriptsize{(0.306)}} \\  

    \mc{1}{l}{\scriptsize{Depression $t$-Score}} & \mc{1}{c}{\scriptsize{21}} & \mc{1}{c}{\scriptsize{-5.649}} & \mc{1}{c}{\scriptsize{-6.052}} & \mc{1}{c}{\scriptsize{-9.358}} & \mc{1}{c}{\scriptsize{-11.051}} & \mc{1}{c}{\scriptsize{-9.416}} & \mc{1}{c}{\scriptsize{-4.406}} & \mc{1}{c}{\scriptsize{-4.460}} & \mc{1}{c}{\scriptsize{-4.096}} \\  

     &  & \mc{1}{c}{\scriptsize{\textbf{(0.009)}}} & \mc{1}{c}{\scriptsize{\textbf{(0.023)}}} & \mc{1}{c}{\scriptsize{\textbf{(0.001)}}} & \mc{1}{c}{\scriptsize{(0.998)}} & \mc{1}{c}{\scriptsize{\textbf{(0.005)}}} & \mc{1}{c}{\scriptsize{\textbf{(0.052)}}} & \mc{1}{c}{\scriptsize{(0.103)}} & \mc{1}{c}{\scriptsize{\textbf{(0.062)}}} \\  

     & \mc{1}{c}{\scriptsize{Mid-30s}} & \mc{1}{c}{\scriptsize{-2.466}} & \mc{1}{c}{\scriptsize{-2.366}} & \mc{1}{c}{\scriptsize{-0.109}} &  & \mc{1}{c}{\scriptsize{-0.058}} & \mc{1}{c}{\scriptsize{-3.109}} & \mc{1}{c}{\scriptsize{-3.364}} & \mc{1}{c}{\scriptsize{-3.034}} \\  

     &  & \mc{1}{c}{\scriptsize{(0.209)}} & \mc{1}{c}{\scriptsize{(0.187)}} & \mc{1}{c}{\scriptsize{(0.998)}} &  & \mc{1}{c}{\scriptsize{(0.470)}} & \mc{1}{c}{\scriptsize{(0.155)}} & \mc{1}{c}{\scriptsize{(0.115)}} & \mc{1}{c}{\scriptsize{(0.149)}} \\  

    \mc{1}{l}{\scriptsize{Anxiety $t$-Score}} & \mc{1}{c}{\scriptsize{21}} & \mc{1}{c}{\scriptsize{-6.163}} & \mc{1}{c}{\scriptsize{-5.939}} & \mc{1}{c}{\scriptsize{-9.552}} & \mc{1}{c}{\scriptsize{-9.018}} & \mc{1}{c}{\scriptsize{-8.949}} & \mc{1}{c}{\scriptsize{-5.244}} & \mc{1}{c}{\scriptsize{-4.560}} & \mc{1}{c}{\scriptsize{-4.376}} \\  

     &  & \mc{1}{c}{\scriptsize{\textbf{(0.012)}}} & \mc{1}{c}{\scriptsize{\textbf{(0.035)}}} & \mc{1}{c}{\scriptsize{\textbf{(0.001)}}} & \mc{1}{c}{\scriptsize{(0.999)}} & \mc{1}{c}{\scriptsize{\textbf{(0.016)}}} & \mc{1}{c}{\scriptsize{\textbf{(0.029)}}} & \mc{1}{c}{\scriptsize{(0.102)}} & \mc{1}{c}{\scriptsize{\textbf{(0.081)}}} \\  

     & \mc{1}{c}{\scriptsize{Mid-30s}} & \mc{1}{c}{\scriptsize{-4.564}} & \mc{1}{c}{\scriptsize{-4.137}} & \mc{1}{c}{\scriptsize{-3.457}} & \mc{1}{c}{\scriptsize{-7.905}} & \mc{1}{c}{\scriptsize{-3.765}} & \mc{1}{c}{\scriptsize{-4.866}} & \mc{1}{c}{\scriptsize{-3.416}} & \mc{1}{c}{\scriptsize{-5.625}} \\  

     &  & \mc{1}{c}{\scriptsize{\textbf{(0.063)}}} & \mc{1}{c}{\scriptsize{\textbf{(0.065)}}} & \mc{1}{c}{\scriptsize{\textbf{(0.001)}}} & \mc{1}{c}{\scriptsize{(0.105)}} & \mc{1}{c}{\scriptsize{(0.258)}} & \mc{1}{c}{\scriptsize{\textbf{(0.064)}}} & \mc{1}{c}{\scriptsize{(0.115)}} & \mc{1}{c}{\scriptsize{\textbf{(0.043)}}} \\  

    \mc{1}{l}{\scriptsize{Hostility $t$-Score}} & \mc{1}{c}{\scriptsize{21}} & \mc{1}{c}{\scriptsize{-4.721}} & \mc{1}{c}{\scriptsize{-5.735}} & \mc{1}{c}{\scriptsize{-10.732}} & \mc{1}{c}{\scriptsize{-10.913}} & \mc{1}{c}{\scriptsize{-10.540}} & \mc{1}{c}{\scriptsize{-3.299}} & \mc{1}{c}{\scriptsize{-4.233}} & \mc{1}{c}{\scriptsize{-2.927}} \\  

     &  & \mc{1}{c}{\scriptsize{\textbf{(0.013)}}} & \mc{1}{c}{\scriptsize{\textbf{(0.005)}}} & \mc{1}{c}{\scriptsize{\textbf{(0.001)}}} & \mc{1}{c}{\scriptsize{(0.999)}} & \mc{1}{c}{\scriptsize{\textbf{(0.000)}}} & \mc{1}{c}{\scriptsize{\textbf{(0.079)}}} & \mc{1}{c}{\scriptsize{\textbf{(0.055)}}} & \mc{1}{c}{\scriptsize{\textbf{(0.090)}}} \\  

     & \mc{1}{c}{\scriptsize{Mid-30s}} & \mc{1}{c}{\scriptsize{0.512}} & \mc{1}{c}{\scriptsize{0.164}} & \mc{1}{c}{\scriptsize{-0.797}} & \mc{1}{c}{\scriptsize{-4.159}} & \mc{1}{c}{\scriptsize{-0.702}} & \mc{1}{c}{\scriptsize{0.870}} & \mc{1}{c}{\scriptsize{0.701}} & \mc{1}{c}{\scriptsize{1.563}} \\  

     &  & \mc{1}{c}{\scriptsize{(0.435)}} & \mc{1}{c}{\scriptsize{(0.453)}} & \mc{1}{c}{\scriptsize{\textbf{(0.001)}}} & \mc{1}{c}{\scriptsize{(0.259)}} & \mc{1}{c}{\scriptsize{(0.456)}} & \mc{1}{c}{\scriptsize{(0.412)}} & \mc{1}{c}{\scriptsize{(0.400)}} & \mc{1}{c}{\scriptsize{(0.331)}} \\  

    \mc{1}{l}{\scriptsize{Global Severity Index $t$-Score}} & \mc{1}{c}{\scriptsize{21}} & \mc{1}{c}{\scriptsize{-6.436}} & \mc{1}{c}{\scriptsize{-6.271}} & \mc{1}{c}{\scriptsize{-11.241}} & \mc{1}{c}{\scriptsize{-9.194}} & \mc{1}{c}{\scriptsize{-10.996}} & \mc{1}{c}{\scriptsize{-5.472}} & \mc{1}{c}{\scriptsize{-4.928}} & \mc{1}{c}{\scriptsize{-4.987}} \\  

     &  & \mc{1}{c}{\scriptsize{\textbf{(0.007)}}} & \mc{1}{c}{\scriptsize{\textbf{(0.017)}}} & \mc{1}{c}{\scriptsize{\textbf{(0.001)}}} & \mc{1}{c}{\scriptsize{(0.999)}} & \mc{1}{c}{\scriptsize{\textbf{(0.000)}}} & \mc{1}{c}{\scriptsize{\textbf{(0.017)}}} & \mc{1}{c}{\scriptsize{\textbf{(0.071)}}} & \mc{1}{c}{\scriptsize{\textbf{(0.036)}}} \\  

    \mc{1}{l}{\scriptsize{Global Severity Index $t$-Score (BSI 18)}} & \mc{1}{c}{\scriptsize{Mid-30s}} & \mc{1}{c}{\scriptsize{-2.365}} & \mc{1}{c}{\scriptsize{-1.290}} & \mc{1}{c}{\scriptsize{0.290}} & \mc{1}{c}{\scriptsize{-4.084}} & \mc{1}{c}{\scriptsize{0.330}} & \mc{1}{c}{\scriptsize{-3.089}} & \mc{1}{c}{\scriptsize{-1.441}} & \mc{1}{c}{\scriptsize{-3.108}} \\  

     &  & \mc{1}{c}{\scriptsize{(0.272)}} & \mc{1}{c}{\scriptsize{(0.334)}} & \mc{1}{c}{\scriptsize{(0.997)}} & \mc{1}{c}{\scriptsize{(0.284)}} & \mc{1}{c}{\scriptsize{(0.513)}} & \mc{1}{c}{\scriptsize{(0.202)}} & \mc{1}{c}{\scriptsize{(0.317)}} & \mc{1}{c}{\scriptsize{(0.191)}} \\  

    \mc{1}{l}{\scriptsize{BSI Factor}} & \mc{1}{c}{\scriptsize{21 to Mid-30s}} & \mc{1}{c}{\scriptsize{-0.624}} & \mc{1}{c}{\scriptsize{-0.429}} & \mc{1}{c}{\scriptsize{-0.747}} & \mc{1}{c}{\scriptsize{-0.923}} & \mc{1}{c}{\scriptsize{-0.678}} & \mc{1}{c}{\scriptsize{-0.589}} & \mc{1}{c}{\scriptsize{-0.302}} & \mc{1}{c}{\scriptsize{-0.552}} \\  

     &  & \mc{1}{c}{\scriptsize{\textbf{(0.015)}}} & \mc{1}{c}{\scriptsize{(0.130)}} & \mc{1}{c}{\scriptsize{(0.997)}} & \mc{1}{c}{\scriptsize{\textbf{(0.001)}}} & \mc{1}{c}{\scriptsize{(0.136)}} & \mc{1}{c}{\scriptsize{\textbf{(0.033)}}} & \mc{1}{c}{\scriptsize{(0.237)}} & \mc{1}{c}{\scriptsize{\textbf{(0.041)}}} \\  

  \bottomrule
  \end{tabular}
\end{center}

\begin{center}
	  \begin{tabular}{cccccccccc}
  \toprule

    \scriptsize{Variable} & \scriptsize{Age} & \scriptsize{(1)} & \scriptsize{(2)} & \scriptsize{(3)} & \scriptsize{(4)} & \scriptsize{(5)} & \scriptsize{(6)} & \scriptsize{(7)} & \scriptsize{(8)} \\ 
    \midrule  

    \mc{1}{l}{\scriptsize{Somatization}} & \mc{1}{c}{\scriptsize{21}} & \mc{1}{c}{\scriptsize{-0.113}} & \mc{1}{c}{\scriptsize{-0.132}} & \mc{1}{c}{\scriptsize{-0.182}} & \mc{1}{c}{\scriptsize{-0.036}} & \mc{1}{c}{\scriptsize{-0.068}} & \mc{1}{c}{\scriptsize{-0.057}} & \mc{1}{c}{\scriptsize{-0.575}} & \mc{1}{c}{\scriptsize{0.022}} \\  

     &  & \mc{1}{c}{\scriptsize{(0.171)}} & \mc{1}{c}{\scriptsize{(0.263)}} & \mc{1}{c}{\scriptsize{(0.105)}} & \mc{1}{c}{\scriptsize{(0.342)}} & \mc{1}{c}{\scriptsize{(0.368)}} & \mc{1}{c}{\scriptsize{(0.342)}} & \mc{1}{c}{\scriptsize{(0.171)}} & \mc{1}{c}{\scriptsize{(0.461)}} \\  

     & \mc{1}{c}{\scriptsize{34}} & \mc{1}{c}{\scriptsize{-0.131}} & \mc{1}{c}{\scriptsize{0.004}} & \mc{1}{c}{\scriptsize{-0.124}} & \mc{1}{c}{\scriptsize{0.028}} & \mc{1}{c}{\scriptsize{-0.069}} & \mc{1}{c}{\scriptsize{-0.137}} & \mc{1}{c}{\scriptsize{-0.553}} & \mc{1}{c}{\scriptsize{-0.081}} \\  

     &  & \mc{1}{c}{\scriptsize{(0.118)}} & \mc{1}{c}{\scriptsize{(0.421)}} & \mc{1}{c}{\scriptsize{(0.184)}} & \mc{1}{c}{\scriptsize{(0.197)}} & \mc{1}{c}{\scriptsize{(0.329)}} & \mc{1}{c}{\scriptsize{(0.171)}} & \mc{1}{c}{\scriptsize{(0.276)}} & \mc{1}{c}{\scriptsize{(0.355)}} \\  

    \mc{1}{l}{\scriptsize{Depression}} & \mc{1}{c}{\scriptsize{21}} & \mc{1}{c}{\scriptsize{-0.330}} & \mc{1}{c}{\scriptsize{-0.713}} & \mc{1}{c}{\scriptsize{-0.471}} & \mc{1}{c}{\scriptsize{-1.333}} & \mc{1}{c}{\scriptsize{-0.630}} & \mc{1}{c}{\scriptsize{-0.217}} & \mc{1}{c}{\scriptsize{-0.393}} & \mc{1}{c}{\scriptsize{-0.356}} \\  

     &  & \mc{1}{c}{\scriptsize{\textbf{(0.039)}}} & \mc{1}{c}{\scriptsize{(0.105)}} & \mc{1}{c}{\scriptsize{\textbf{(0.053)}}} & \mc{1}{c}{\scriptsize{(0.224)}} & \mc{1}{c}{\scriptsize{\textbf{(0.079)}}} & \mc{1}{c}{\scriptsize{(0.184)}} & \mc{1}{c}{\scriptsize{(0.237)}} & \mc{1}{c}{\scriptsize{\textbf{(0.079)}}} \\  

     & \mc{1}{c}{\scriptsize{34}} & \mc{1}{c}{\scriptsize{-0.348}} & \mc{1}{c}{\scriptsize{-0.402}} & \mc{1}{c}{\scriptsize{-0.043}} & \mc{1}{c}{\scriptsize{0.328}} & \mc{1}{c}{\scriptsize{0.013}} & \mc{1}{c}{\scriptsize{-0.585}} & \mc{1}{c}{\scriptsize{-1.081}} & \mc{1}{c}{\scriptsize{-0.681}} \\  

     &  & \mc{1}{c}{\scriptsize{\textbf{(0.079)}}} & \mc{1}{c}{\scriptsize{(0.368)}} & \mc{1}{c}{\scriptsize{(0.303)}} & \mc{1}{c}{\scriptsize{(0.421)}} & \mc{1}{c}{\scriptsize{(0.526)}} & \mc{1}{c}{\scriptsize{\textbf{(0.079)}}} & \mc{1}{c}{\scriptsize{(0.329)}} & \mc{1}{c}{\scriptsize{(0.118)}} \\  

    \mc{1}{l}{\scriptsize{Anxiety}} & \mc{1}{c}{\scriptsize{21}} & \mc{1}{c}{\scriptsize{-0.333}} & \mc{1}{c}{\scriptsize{-0.816}} & \mc{1}{c}{\scriptsize{-0.516}} & \mc{1}{c}{\scriptsize{-1.375}} & \mc{1}{c}{\scriptsize{-0.771}} & \mc{1}{c}{\scriptsize{-0.187}} & \mc{1}{c}{\scriptsize{-0.977}} & \mc{1}{c}{\scriptsize{-0.090}} \\  

     &  & \mc{1}{c}{\scriptsize{\textbf{(0.079)}}} & \mc{1}{c}{\scriptsize{\textbf{(0.079)}}} & \mc{1}{c}{\scriptsize{\textbf{(0.079)}}} & \mc{1}{c}{\scriptsize{(0.237)}} & \mc{1}{c}{\scriptsize{\textbf{(0.079)}}} & \mc{1}{c}{\scriptsize{(0.197)}} & \mc{1}{c}{\scriptsize{(0.184)}} & \mc{1}{c}{\scriptsize{(0.355)}} \\  

     & \mc{1}{c}{\scriptsize{34}} & \mc{1}{c}{\scriptsize{-0.217}} & \mc{1}{c}{\scriptsize{-0.468}} & \mc{1}{c}{\scriptsize{-0.110}} & \mc{1}{c}{\scriptsize{0.012}} & \mc{1}{c}{\scriptsize{-0.206}} & \mc{1}{c}{\scriptsize{-0.300}} & \mc{1}{c}{\scriptsize{-0.990}} & \mc{1}{c}{\scriptsize{-0.407}} \\  

     &  & \mc{1}{c}{\scriptsize{\textbf{(0.079)}}} & \mc{1}{c}{\scriptsize{(0.145)}} & \mc{1}{c}{\scriptsize{(0.132)}} & \mc{1}{c}{\scriptsize{(0.224)}} & \mc{1}{c}{\scriptsize{\textbf{(0.066)}}} & \mc{1}{c}{\scriptsize{\textbf{(0.079)}}} & \mc{1}{c}{\scriptsize{(0.105)}} & \mc{1}{c}{\scriptsize{\textbf{(0.079)}}} \\  

    \mc{1}{l}{\scriptsize{Hostility}} & \mc{1}{c}{\scriptsize{21}} & \mc{1}{c}{\scriptsize{-0.399}} & \mc{1}{c}{\scriptsize{-0.454}} & \mc{1}{c}{\scriptsize{-0.772}} & \mc{1}{c}{\scriptsize{-1.200}} & \mc{1}{c}{\scriptsize{-0.672}} & \mc{1}{c}{\scriptsize{-0.100}} & \mc{1}{c}{\scriptsize{-0.705}} & \mc{1}{c}{\scriptsize{-0.127}} \\  

     &  & \mc{1}{c}{\scriptsize{\textbf{(0.000)}}} & \mc{1}{c}{\scriptsize{\textbf{(0.079)}}} & \mc{1}{c}{\scriptsize{\textbf{(0.013)}}} & \mc{1}{c}{\scriptsize{\textbf{(0.013)}}} & \mc{1}{c}{\scriptsize{\textbf{(0.026)}}} & \mc{1}{c}{\scriptsize{(0.276)}} & \mc{1}{c}{\scriptsize{(0.171)}} & \mc{1}{c}{\scriptsize{(0.303)}} \\  

     & \mc{1}{c}{\scriptsize{34}} & \mc{1}{c}{\scriptsize{0.085}} & \mc{1}{c}{\scriptsize{-0.006}} & \mc{1}{c}{\scriptsize{0.103}} & \mc{1}{c}{\scriptsize{0.584}} & \mc{1}{c}{\scriptsize{-0.102}} & \mc{1}{c}{\scriptsize{0.071}} & \mc{1}{c}{\scriptsize{-0.157}} & \mc{1}{c}{\scriptsize{0.010}} \\  

     &  & \mc{1}{c}{\scriptsize{(0.618)}} & \mc{1}{c}{\scriptsize{(0.553)}} & \mc{1}{c}{\scriptsize{(0.697)}} & \mc{1}{c}{\scriptsize{(0.500)}} & \mc{1}{c}{\scriptsize{(0.382)}} & \mc{1}{c}{\scriptsize{(0.618)}} & \mc{1}{c}{\scriptsize{(0.355)}} & \mc{1}{c}{\scriptsize{(0.539)}} \\  

    \mc{1}{l}{\scriptsize{Global Severity Index}} & \mc{1}{c}{\scriptsize{21}} & \mc{1}{c}{\scriptsize{-0.196}} & \mc{1}{c}{\scriptsize{-0.440}} & \mc{1}{c}{\scriptsize{-0.372}} & \mc{1}{c}{\scriptsize{-0.703}} & \mc{1}{c}{\scriptsize{-0.444}} & \mc{1}{c}{\scriptsize{-0.055}} & \mc{1}{c}{\scriptsize{-0.651}} & \mc{1}{c}{\scriptsize{-0.014}} \\  

     &  & \mc{1}{c}{\scriptsize{\textbf{(0.092)}}} & \mc{1}{c}{\scriptsize{\textbf{(0.092)}}} & \mc{1}{c}{\scriptsize{\textbf{(0.000)}}} & \mc{1}{c}{\scriptsize{(0.224)}} & \mc{1}{c}{\scriptsize{\textbf{(0.039)}}} & \mc{1}{c}{\scriptsize{(0.342)}} & \mc{1}{c}{\scriptsize{(0.184)}} & \mc{1}{c}{\scriptsize{(0.421)}} \\  

     & \mc{1}{c}{\scriptsize{34}} & \mc{1}{c}{\scriptsize{-4.175}} & \mc{1}{c}{\scriptsize{-5.192}} & \mc{1}{c}{\scriptsize{-1.657}} & \mc{1}{c}{\scriptsize{2.212}} & \mc{1}{c}{\scriptsize{-1.574}} & \mc{1}{c}{\scriptsize{-6.133}} & \mc{1}{c}{\scriptsize{-15.745}} & \mc{1}{c}{\scriptsize{-7.019}} \\  

     &  & \mc{1}{c}{\scriptsize{\textbf{(0.066)}}} & \mc{1}{c}{\scriptsize{(0.303)}} & \mc{1}{c}{\scriptsize{(0.224)}} & \mc{1}{c}{\scriptsize{(0.395)}} & \mc{1}{c}{\scriptsize{(0.316)}} & \mc{1}{c}{\scriptsize{\textbf{(0.066)}}} & \mc{1}{c}{\scriptsize{\textbf{(0.092)}}} & \mc{1}{c}{\scriptsize{\textbf{(0.079)}}} \\  

    \mc{1}{l}{\scriptsize{BSI Factor}} & \mc{1}{c}{\scriptsize{21 and 34}} & \mc{1}{c}{\scriptsize{-0.367}} & \mc{1}{c}{\scriptsize{-0.540}} & \mc{1}{c}{\scriptsize{-0.271}} & \mc{1}{c}{\scriptsize{0.438}} & \mc{1}{c}{\scriptsize{-0.402}} & \mc{1}{c}{\scriptsize{-0.452}} & \mc{1}{c}{\scriptsize{-2.040}} & \mc{1}{c}{\scriptsize{-0.545}} \\  

     &  & \mc{1}{c}{\scriptsize{(0.211)}} & \mc{1}{c}{\scriptsize{(0.276)}} & \mc{1}{c}{\scriptsize{(0.316)}} & \mc{1}{c}{\scriptsize{(0.434)}} & \mc{1}{c}{\scriptsize{(0.263)}} & \mc{1}{c}{\scriptsize{(0.145)}} & \mc{1}{c}{\scriptsize{(0.105)}} & \mc{1}{c}{\scriptsize{(0.184)}} \\ 
    \midrule  

    \mc{2}{l}{\scriptsize{\% of Pos. TE ($H_0$: $\le$ 50\%)}} & \mc{1}{c}{\scriptsize{91}} & \mc{1}{c}{\scriptsize{91}} & \mc{1}{c}{\scriptsize{91}} & \mc{1}{c}{\scriptsize{45}} & \mc{1}{c}{\scriptsize{91}} & \mc{1}{c}{\scriptsize{91}} & \mc{1}{c}{\scriptsize{100}} & \mc{1}{c}{\scriptsize{82}} \\  

     &  & \mc{1}{c}{\scriptsize{\textbf{(0.000)}}} & \mc{1}{c}{\scriptsize{\textbf{(0.000)}}} & \mc{1}{c}{\scriptsize{\textbf{(0.000)}}} & \mc{1}{c}{\scriptsize{(0.421)}} & \mc{1}{c}{\scriptsize{\textbf{(0.000)}}} & \mc{1}{c}{\scriptsize{\textbf{(0.000)}}} & \mc{1}{c}{\scriptsize{\textbf{(0.000)}}} & \mc{1}{c}{\scriptsize{\textbf{(0.079)}}} \\  

    \mc{2}{l}{\scriptsize{\% of Pos. TE ($H_0$: $\le$ 10\% $|$ 10\% Significance)}} & \mc{1}{c}{\scriptsize{36}} & \mc{1}{c}{\scriptsize{36}} & \mc{1}{c}{\scriptsize{45}} & \mc{1}{c}{\scriptsize{9}} & \mc{1}{c}{\scriptsize{36}} & \mc{1}{c}{\scriptsize{18}} & \mc{1}{c}{\scriptsize{9}} & \mc{1}{c}{\scriptsize{27}} \\  

     &  & \mc{1}{c}{\scriptsize{(0.158)}} & \mc{1}{c}{\scriptsize{\textbf{(0.092)}}} & \mc{1}{c}{\scriptsize{\textbf{(0.053)}}} & \mc{1}{c}{\scriptsize{(0.461)}} & \mc{1}{c}{\scriptsize{(0.105)}} & \mc{1}{c}{\scriptsize{(0.368)}} & \mc{1}{c}{\scriptsize{(0.382)}} & \mc{1}{c}{\scriptsize{(0.145)}} \\  

  \bottomrule
  \end{tabular}
\end{center}

\begin{center}
	\begin{table}[H]
\captionsetup{singlelinecheck=false,justification=centering}
\caption{ABC Average Treatment Effects, Females \\ Obesity \label{tab:ate_female_apx17}}

  \begin{threeparttable}
  \begin{tabular}{cccccccccc}
  \hline\hline

     &  & \scriptsize{(1)} & \scriptsize{(2)} & \scriptsize{(3)} & \scriptsize{(4)} & \scriptsize{(5)} & \scriptsize{(6)} & \scriptsize{(7)} & \scriptsize{(8)} \\  

     &  &  &  & \mc{3}{c}{\scriptsize{$P=0$}} & \mc{3}{c}{\scriptsize{$P=1$}} \\ 
    \cmidrule(lr){5-7} \cmidrule(lr){8-10} 

    \scriptsize{Variable} & \scriptsize{Age} & \scriptsize{ITT} & \scriptsize{ITT$|X,W$} & \scriptsize{ITT} & \scriptsize{ITT$|X,W$} & \scriptsize{KE$|X,W$} & \scriptsize{ITT} & \scriptsize{ITT$|X,W$} & \scriptsize{KE$|X,W$} \\ 
    \hline  

    \mc{1}{l}{\scriptsize{Measured BMI}} & \mc{1}{c}{\scriptsize{Mid-30s}} & \mc{1}{c}{\scriptsize{1.785}} & \mc{1}{c}{\scriptsize{5.519}} & \mc{1}{c}{\scriptsize{1.479}} & \mc{1}{c}{\scriptsize{-6.567}} & \mc{1}{c}{\scriptsize{0.725}} & \mc{1}{c}{\scriptsize{1.853}} & \mc{1}{c}{\scriptsize{9.134}} & \mc{1}{c}{\scriptsize{3.874}} \\  

     &  & \mc{1}{c}{\scriptsize{(0.745)}} & \mc{1}{c}{\scriptsize{(0.902)}} & \mc{1}{c}{\scriptsize{(0.608)}} & \mc{1}{c}{\scriptsize{(0.235)}} & \mc{1}{c}{\scriptsize{(0.471)}} & \mc{1}{c}{\scriptsize{(0.706)}} & \mc{1}{c}{\scriptsize{(1.000)}} & \mc{1}{c}{\scriptsize{(0.784)}} \\  

    \mc{1}{l}{\scriptsize{Obesity}} & \mc{1}{c}{\scriptsize{Mid-30s}} & \mc{1}{c}{\scriptsize{-0.061}} & \mc{1}{c}{\scriptsize{0.183}} & \mc{1}{c}{\scriptsize{-0.083}} & \mc{1}{c}{\scriptsize{-0.068}} & \mc{1}{c}{\scriptsize{-0.141}} & \mc{1}{c}{\scriptsize{-0.056}} & \mc{1}{c}{\scriptsize{0.293}} & \mc{1}{c}{\scriptsize{0.065}} \\  

     &  & \mc{1}{c}{\scriptsize{(0.235)}} & \mc{1}{c}{\scriptsize{(0.863)}} & \mc{1}{c}{\scriptsize{(0.294)}} & \mc{1}{c}{\scriptsize{(0.373)}} & \mc{1}{c}{\scriptsize{(0.118)}} & \mc{1}{c}{\scriptsize{(0.275)}} & \mc{1}{c}{\scriptsize{(0.863)}} & \mc{1}{c}{\scriptsize{(0.647)}} \\  

    \mc{1}{l}{\scriptsize{Severe Obesity}} & \mc{1}{c}{\scriptsize{Mid-30s}} & \mc{1}{c}{\scriptsize{-0.141}} & \mc{1}{c}{\scriptsize{0.011}} & \mc{1}{c}{\scriptsize{-0.028}} & \mc{1}{c}{\scriptsize{-0.411}} & \mc{1}{c}{\scriptsize{-0.054}} & \mc{1}{c}{\scriptsize{-0.167}} & \mc{1}{c}{\scriptsize{0.138}} & \mc{1}{c}{\scriptsize{-0.035}} \\  

     &  & \mc{1}{c}{\scriptsize{(0.157)}} & \mc{1}{c}{\scriptsize{(0.431)}} & \mc{1}{c}{\scriptsize{(0.373)}} & \mc{1}{c}{\scriptsize{(0.176)}} & \mc{1}{c}{\scriptsize{(0.353)}} & \mc{1}{c}{\scriptsize{\textbf{(0.059)}}} & \mc{1}{c}{\scriptsize{(0.725)}} & \mc{1}{c}{\scriptsize{(0.412)}} \\  

    \mc{1}{l}{\scriptsize{Waist-hip Ratio}} & \mc{1}{c}{\scriptsize{Mid-30s}} & \mc{1}{c}{\scriptsize{-0.057}} & \mc{1}{c}{\scriptsize{-0.059}} & \mc{1}{c}{\scriptsize{-0.137}} & \mc{1}{c}{\scriptsize{-0.193}} & \mc{1}{c}{\scriptsize{-0.137}} & \mc{1}{c}{\scriptsize{-0.037}} & \mc{1}{c}{\scriptsize{-0.029}} & \mc{1}{c}{\scriptsize{-0.021}} \\  

     &  & \mc{1}{c}{\scriptsize{\textbf{(0.039)}}} & \mc{1}{c}{\scriptsize{(0.137)}} & \mc{1}{c}{\scriptsize{\textbf{(0.000)}}} & \mc{1}{c}{\scriptsize{\textbf{(0.020)}}} & \mc{1}{c}{\scriptsize{\textbf{(0.039)}}} & \mc{1}{c}{\scriptsize{(0.118)}} & \mc{1}{c}{\scriptsize{(0.353)}} & \mc{1}{c}{\scriptsize{(0.255)}} \\  

    \mc{1}{l}{\scriptsize{Abdominal Obesity}} & \mc{1}{c}{\scriptsize{Mid-30s}} & \mc{1}{c}{\scriptsize{-0.199}} & \mc{1}{c}{\scriptsize{-0.150}} & \mc{1}{c}{\scriptsize{-0.438}} & \mc{1}{c}{\scriptsize{-0.198}} & \mc{1}{c}{\scriptsize{-0.386}} & \mc{1}{c}{\scriptsize{-0.143}} & \mc{1}{c}{\scriptsize{-0.124}} & \mc{1}{c}{\scriptsize{-0.086}} \\  

     &  & \mc{1}{c}{\scriptsize{(0.118)}} & \mc{1}{c}{\scriptsize{(0.294)}} & \mc{1}{c}{\scriptsize{\textbf{(0.000)}}} & \mc{1}{c}{\scriptsize{(0.333)}} & \mc{1}{c}{\scriptsize{\textbf{(0.000)}}} & \mc{1}{c}{\scriptsize{(0.157)}} & \mc{1}{c}{\scriptsize{(0.353)}} & \mc{1}{c}{\scriptsize{(0.314)}} \\  

    \mc{1}{l}{\scriptsize{Framingham Risk Score}} & \mc{1}{c}{\scriptsize{Mid-30s}} & \mc{1}{c}{\scriptsize{-0.471}} & \mc{1}{c}{\scriptsize{-0.729}} & \mc{1}{c}{\scriptsize{-1.005}} & \mc{1}{c}{\scriptsize{-1.527}} & \mc{1}{c}{\scriptsize{-1.263}} & \mc{1}{c}{\scriptsize{-0.353}} & \mc{1}{c}{\scriptsize{-0.481}} & \mc{1}{c}{\scriptsize{-0.387}} \\  

     &  & \mc{1}{c}{\scriptsize{\textbf{(0.059)}}} & \mc{1}{c}{\scriptsize{(0.137)}} & \mc{1}{c}{\scriptsize{\textbf{(0.000)}}} & \mc{1}{c}{\scriptsize{\textbf{(0.020)}}} & \mc{1}{c}{\scriptsize{\textbf{(0.000)}}} & \mc{1}{c}{\scriptsize{(0.137)}} & \mc{1}{c}{\scriptsize{(0.216)}} & \mc{1}{c}{\scriptsize{(0.118)}} \\  

    \mc{1}{l}{\scriptsize{Obesity Factor}} & \mc{1}{c}{\scriptsize{Mid-30s}} & \mc{1}{c}{\scriptsize{-0.185}} & \mc{1}{c}{\scriptsize{-0.022}} & \mc{1}{c}{\scriptsize{-0.495}} & \mc{1}{c}{\scriptsize{-1.355}} & \mc{1}{c}{\scriptsize{-0.567}} & \mc{1}{c}{\scriptsize{-0.112}} & \mc{1}{c}{\scriptsize{0.364}} & \mc{1}{c}{\scriptsize{0.119}} \\  

     &  & \mc{1}{c}{\scriptsize{(0.294)}} & \mc{1}{c}{\scriptsize{(0.373)}} & \mc{1}{c}{\scriptsize{(0.157)}} & \mc{1}{c}{\scriptsize{\textbf{(0.098)}}} & \mc{1}{c}{\scriptsize{\textbf{(0.078)}}} & \mc{1}{c}{\scriptsize{(0.392)}} & \mc{1}{c}{\scriptsize{(0.784)}} & \mc{1}{c}{\scriptsize{(0.569)}} \\ 
    \hline  

    \\[0.1cm]
    \mc{2}{l}{\scriptsize{\% of Pos. TE ($H_0$: $\le$ 25\% $|$ 10\% Significance)}} & \mc{1}{c}{\scriptsize{29}} & \mc{1}{c}{\scriptsize{0}} & \mc{1}{c}{\scriptsize{43}} & \mc{1}{c}{\scriptsize{43}} & \mc{1}{c}{\scriptsize{57}} & \mc{1}{c}{\scriptsize{14}} & \mc{1}{c}{\scriptsize{0}} & \mc{1}{c}{\scriptsize{0}} \\  

     &  & \mc{1}{c}{\scriptsize{(0.373)}} & \mc{1}{c}{\scriptsize{(0.765)}} & \mc{1}{c}{\scriptsize{(0.157)}} & \mc{1}{c}{\scriptsize{(0.216)}} & \mc{1}{c}{\scriptsize{\textbf{(0.098)}}} & \mc{1}{c}{\scriptsize{(0.569)}} & \mc{1}{c}{\scriptsize{(1.000)}} & \mc{1}{c}{\scriptsize{(0.980)}} \\  

    \mc{2}{l}{\scriptsize{\% of Pos. TE ($H_0$: $\le$ 50\% $|$ 10\% Significance)}} & \mc{1}{c}{\scriptsize{29}} & \mc{1}{c}{\scriptsize{0}} & \mc{1}{c}{\scriptsize{43}} & \mc{1}{c}{\scriptsize{43}} & \mc{1}{c}{\scriptsize{57}} & \mc{1}{c}{\scriptsize{14}} & \mc{1}{c}{\scriptsize{0}} & \mc{1}{c}{\scriptsize{0}} \\  

     &  & \mc{1}{c}{\scriptsize{(0.745)}} & \mc{1}{c}{\scriptsize{(1.000)}} & \mc{1}{c}{\scriptsize{(0.510)}} & \mc{1}{c}{\scriptsize{(0.412)}} & \mc{1}{c}{\scriptsize{(0.294)}} & \mc{1}{c}{\scriptsize{(1.000)}} & \mc{1}{c}{\scriptsize{(1.000)}} & \mc{1}{c}{\scriptsize{(0.980)}} \\  

    \mc{2}{l}{\scriptsize{\% of Pos. TE ($H_0$: $\le$ 75\% $|$ 10\% Significance)}} & \mc{1}{c}{\scriptsize{29}} & \mc{1}{c}{\scriptsize{0}} & \mc{1}{c}{\scriptsize{43}} & \mc{1}{c}{\scriptsize{43}} & \mc{1}{c}{\scriptsize{57}} & \mc{1}{c}{\scriptsize{14}} & \mc{1}{c}{\scriptsize{0}} & \mc{1}{c}{\scriptsize{0}} \\  

     &  & \mc{1}{c}{\scriptsize{(1.000)}} & \mc{1}{c}{\scriptsize{(1.000)}} & \mc{1}{c}{\scriptsize{(0.882)}} & \mc{1}{c}{\scriptsize{(0.725)}} & \mc{1}{c}{\scriptsize{(0.647)}} & \mc{1}{c}{\scriptsize{(1.000)}} & \mc{1}{c}{\scriptsize{(1.000)}} & \mc{1}{c}{\scriptsize{(0.980)}} \\  

  \hline\hline
  \end{tabular}
    \begin{tablenotes}
    \scriptsize
    \item 
Note: This table displays various estimates of the treatment effect of ABC's center-based care.
Column (1) displays the ITT, without accounting for any controls.
Column (2) displays the ITT conditioning on vector of controls, $X$, consisting of the Apgar score 1 minute after birth, the HRI index, maternal IQ, an
indicator for teenage pregnancy of the mother, an indicator for the father being at 
home, and an indicator for having a grandmother residing in the same county. We also apply IPW weights, $W$, to account for attrition.
Columns (3)--(4) are analogous to columns (1)--(2), but we restrict the control sample to subjects
who did not enroll in any alternative care.
Column (5) displys the matching estimate, where we use the Mahalanobis metric and Epanechnikov kernel
to match on controls $X$ listed above, and restrict the control sample to subjects who did not enroll
in any alternative care. Additionally, we apply IPW weights, $W$.
Columns (6)--(8) are analogous to Columns (3)--(5), except we restrict the control sample to subejcts
who did enroll in alternative care. The final three pairs of rows display the proportion of treatment effects in the table that are 
socially positive. The first row in each pair displays the percentage of treatment effects, and the
second row presents the inference. 
Numbers in parentheses represent the $p$-value from a single hypothesis test, and are obtained from 
the empirical bootstrap distribution generated by 200 resamples of the original data. 
Bold $p$-values indicate significance at the 10\% level.
Blank point estimates indicate that we are unable to obtain estimates due to a lack of support in the data. 

    \end{tablenotes}
  \end{threeparttable}

\end{table}
\end{center}

\begin{center}
	\begin{table}[H]
\captionsetup{singlelinecheck=false,justification=centering}
\caption{ABC Average Treatment Effects, Females \\ Mental Health \label{tab:ate_female_apx18}}

  \begin{threeparttable}
  \begin{tabular}{cccccccccc}
  \hline\hline

     &  & \scriptsize{(1)} & \scriptsize{(2)} & \scriptsize{(3)} & \scriptsize{(4)} & \scriptsize{(5)} & \scriptsize{(6)} & \scriptsize{(7)} & \scriptsize{(8)} \\  

     &  &  &  & \mc{3}{c}{\scriptsize{$P=0$}} & \mc{3}{c}{\scriptsize{$P=1$}} \\ 
    \cmidrule(lr){5-7} \cmidrule(lr){8-10} 

    \scriptsize{Variable} & \scriptsize{Age} & \scriptsize{ITT} & \scriptsize{ITT$|X,W$} & \scriptsize{ITT} & \scriptsize{ITT$|X,W$} & \scriptsize{KE$|X,W$} & \scriptsize{ITT} & \scriptsize{ITT$|X,W$} & \scriptsize{KE$|X,W$} \\ 
    \hline  

    \mc{1}{l}{\scriptsize{Somatization}} & \mc{1}{c}{\scriptsize{21}} & \mc{1}{c}{\scriptsize{-0.054}} & \mc{1}{c}{\scriptsize{-0.211}} & \mc{1}{c}{\scriptsize{-0.320}} & \mc{1}{c}{\scriptsize{-0.925}} & \mc{1}{c}{\scriptsize{-0.624}} & \mc{1}{c}{\scriptsize{0.034}} & \mc{1}{c}{\scriptsize{-0.054}} & \mc{1}{c}{\scriptsize{-0.046}} \\  

     &  & \mc{1}{c}{\scriptsize{(0.353)}} & \mc{1}{c}{\scriptsize{(0.196)}} & \mc{1}{c}{\scriptsize{(0.157)}} & \mc{1}{c}{\scriptsize{\textbf{(0.078)}}} & \mc{1}{c}{\scriptsize{(0.118)}} & \mc{1}{c}{\scriptsize{(0.549)}} & \mc{1}{c}{\scriptsize{(0.431)}} & \mc{1}{c}{\scriptsize{(0.471)}} \\  

     & \mc{1}{c}{\scriptsize{34}} & \mc{1}{c}{\scriptsize{-0.013}} & \mc{1}{c}{\scriptsize{-0.159}} & \mc{1}{c}{\scriptsize{-0.736}} & \mc{1}{c}{\scriptsize{-1.110}} & \mc{1}{c}{\scriptsize{-0.840}} & \mc{1}{c}{\scriptsize{0.148}} & \mc{1}{c}{\scriptsize{0.221}} & \mc{1}{c}{\scriptsize{-0.103}} \\  

     &  & \mc{1}{c}{\scriptsize{(0.510)}} & \mc{1}{c}{\scriptsize{(0.275)}} & \mc{1}{c}{\scriptsize{\textbf{(0.098)}}} & \mc{1}{c}{\scriptsize{(0.196)}} & \mc{1}{c}{\scriptsize{\textbf{(0.078)}}} & \mc{1}{c}{\scriptsize{(0.686)}} & \mc{1}{c}{\scriptsize{(0.784)}} & \mc{1}{c}{\scriptsize{(0.294)}} \\  

    \mc{1}{l}{\scriptsize{Depression}} & \mc{1}{c}{\scriptsize{21}} & \mc{1}{c}{\scriptsize{-0.460}} & \mc{1}{c}{\scriptsize{-0.459}} & \mc{1}{c}{\scriptsize{-0.841}} & \mc{1}{c}{\scriptsize{-0.840}} & \mc{1}{c}{\scriptsize{-0.922}} & \mc{1}{c}{\scriptsize{-0.333}} & \mc{1}{c}{\scriptsize{-0.337}} & \mc{1}{c}{\scriptsize{-0.261}} \\  

     &  & \mc{1}{c}{\scriptsize{\textbf{(0.000)}}} & \mc{1}{c}{\scriptsize{\textbf{(0.078)}}} & \mc{1}{c}{\scriptsize{\textbf{(0.020)}}} & \mc{1}{c}{\scriptsize{\textbf{(0.059)}}} & \mc{1}{c}{\scriptsize{\textbf{(0.000)}}} & \mc{1}{c}{\scriptsize{\textbf{(0.098)}}} & \mc{1}{c}{\scriptsize{(0.137)}} & \mc{1}{c}{\scriptsize{(0.176)}} \\  

     & \mc{1}{c}{\scriptsize{34}} & \mc{1}{c}{\scriptsize{-0.048}} & \mc{1}{c}{\scriptsize{-0.191}} & \mc{1}{c}{\scriptsize{-0.431}} & \mc{1}{c}{\scriptsize{-0.960}} & \mc{1}{c}{\scriptsize{-0.623}} & \mc{1}{c}{\scriptsize{0.037}} & \mc{1}{c}{\scriptsize{0.076}} & \mc{1}{c}{\scriptsize{-0.257}} \\  

     &  & \mc{1}{c}{\scriptsize{(0.471)}} & \mc{1}{c}{\scriptsize{(0.294)}} & \mc{1}{c}{\scriptsize{(0.255)}} & \mc{1}{c}{\scriptsize{(0.157)}} & \mc{1}{c}{\scriptsize{(0.196)}} & \mc{1}{c}{\scriptsize{(0.608)}} & \mc{1}{c}{\scriptsize{(0.627)}} & \mc{1}{c}{\scriptsize{(0.176)}} \\  

    \mc{1}{l}{\scriptsize{Anxiety}} & \mc{1}{c}{\scriptsize{21}} & \mc{1}{c}{\scriptsize{-0.371}} & \mc{1}{c}{\scriptsize{-0.241}} & \mc{1}{c}{\scriptsize{-0.734}} & \mc{1}{c}{\scriptsize{-0.791}} & \mc{1}{c}{\scriptsize{-0.642}} & \mc{1}{c}{\scriptsize{-0.250}} & \mc{1}{c}{\scriptsize{-0.038}} & \mc{1}{c}{\scriptsize{-0.131}} \\  

     &  & \mc{1}{c}{\scriptsize{\textbf{(0.059)}}} & \mc{1}{c}{\scriptsize{(0.157)}} & \mc{1}{c}{\scriptsize{\textbf{(0.020)}}} & \mc{1}{c}{\scriptsize{\textbf{(0.059)}}} & \mc{1}{c}{\scriptsize{\textbf{(0.000)}}} & \mc{1}{c}{\scriptsize{\textbf{(0.098)}}} & \mc{1}{c}{\scriptsize{(0.392)}} & \mc{1}{c}{\scriptsize{(0.216)}} \\  

     & \mc{1}{c}{\scriptsize{34}} & \mc{1}{c}{\scriptsize{-0.221}} & \mc{1}{c}{\scriptsize{-0.370}} & \mc{1}{c}{\scriptsize{-0.671}} & \mc{1}{c}{\scriptsize{-1.125}} & \mc{1}{c}{\scriptsize{-0.901}} & \mc{1}{c}{\scriptsize{-0.120}} & \mc{1}{c}{\scriptsize{-0.033}} & \mc{1}{c}{\scriptsize{-0.363}} \\  

     &  & \mc{1}{c}{\scriptsize{(0.196)}} & \mc{1}{c}{\scriptsize{(0.118)}} & \mc{1}{c}{\scriptsize{\textbf{(0.098)}}} & \mc{1}{c}{\scriptsize{(0.137)}} & \mc{1}{c}{\scriptsize{\textbf{(0.059)}}} & \mc{1}{c}{\scriptsize{(0.294)}} & \mc{1}{c}{\scriptsize{(0.529)}} & \mc{1}{c}{\scriptsize{(0.118)}} \\  

    \mc{1}{l}{\scriptsize{Hostility}} & \mc{1}{c}{\scriptsize{21}} & \mc{1}{c}{\scriptsize{-0.437}} & \mc{1}{c}{\scriptsize{-0.409}} & \mc{1}{c}{\scriptsize{-1.108}} & \mc{1}{c}{\scriptsize{-1.283}} & \mc{1}{c}{\scriptsize{-1.152}} & \mc{1}{c}{\scriptsize{-0.213}} & \mc{1}{c}{\scriptsize{-0.238}} & \mc{1}{c}{\scriptsize{-0.223}} \\  

     &  & \mc{1}{c}{\scriptsize{\textbf{(0.039)}}} & \mc{1}{c}{\scriptsize{\textbf{(0.078)}}} & \mc{1}{c}{\scriptsize{\textbf{(0.020)}}} & \mc{1}{c}{\scriptsize{\textbf{(0.059)}}} & \mc{1}{c}{\scriptsize{\textbf{(0.000)}}} & \mc{1}{c}{\scriptsize{(0.235)}} & \mc{1}{c}{\scriptsize{(0.216)}} & \mc{1}{c}{\scriptsize{(0.196)}} \\  

     & \mc{1}{c}{\scriptsize{34}} & \mc{1}{c}{\scriptsize{-0.118}} & \mc{1}{c}{\scriptsize{-0.115}} & \mc{1}{c}{\scriptsize{-0.400}} & \mc{1}{c}{\scriptsize{-0.886}} & \mc{1}{c}{\scriptsize{-0.635}} & \mc{1}{c}{\scriptsize{-0.056}} & \mc{1}{c}{\scriptsize{0.197}} & \mc{1}{c}{\scriptsize{-0.315}} \\  

     &  & \mc{1}{c}{\scriptsize{(0.275)}} & \mc{1}{c}{\scriptsize{(0.392)}} & \mc{1}{c}{\scriptsize{\textbf{(0.098)}}} & \mc{1}{c}{\scriptsize{\textbf{(0.020)}}} & \mc{1}{c}{\scriptsize{\textbf{(0.039)}}} & \mc{1}{c}{\scriptsize{(0.353)}} & \mc{1}{c}{\scriptsize{(0.784)}} & \mc{1}{c}{\scriptsize{(0.157)}} \\  

    \mc{1}{l}{\scriptsize{Global Severity Index}} & \mc{1}{c}{\scriptsize{21}} & \mc{1}{c}{\scriptsize{-0.331}} & \mc{1}{c}{\scriptsize{-0.324}} & \mc{1}{c}{\scriptsize{-0.653}} & \mc{1}{c}{\scriptsize{-0.804}} & \mc{1}{c}{\scriptsize{-0.724}} & \mc{1}{c}{\scriptsize{-0.224}} & \mc{1}{c}{\scriptsize{-0.159}} & \mc{1}{c}{\scriptsize{-0.165}} \\  

     &  & \mc{1}{c}{\scriptsize{\textbf{(0.000)}}} & \mc{1}{c}{\scriptsize{\textbf{(0.078)}}} & \mc{1}{c}{\scriptsize{\textbf{(0.020)}}} & \mc{1}{c}{\scriptsize{\textbf{(0.020)}}} & \mc{1}{c}{\scriptsize{\textbf{(0.000)}}} & \mc{1}{c}{\scriptsize{(0.118)}} & \mc{1}{c}{\scriptsize{(0.255)}} & \mc{1}{c}{\scriptsize{(0.196)}} \\  

     & \mc{1}{c}{\scriptsize{34}} & \mc{1}{c}{\scriptsize{-1.687}} & \mc{1}{c}{\scriptsize{-4.320}} & \mc{1}{c}{\scriptsize{-11.028}} & \mc{1}{c}{\scriptsize{-19.171}} & \mc{1}{c}{\scriptsize{-14.180}} & \mc{1}{c}{\scriptsize{0.389}} & \mc{1}{c}{\scriptsize{1.583}} & \mc{1}{c}{\scriptsize{-4.342}} \\  

     &  & \mc{1}{c}{\scriptsize{(0.353)}} & \mc{1}{c}{\scriptsize{(0.235)}} & \mc{1}{c}{\scriptsize{(0.137)}} & \mc{1}{c}{\scriptsize{(0.157)}} & \mc{1}{c}{\scriptsize{\textbf{(0.059)}}} & \mc{1}{c}{\scriptsize{(0.529)}} & \mc{1}{c}{\scriptsize{(0.686)}} & \mc{1}{c}{\scriptsize{(0.157)}} \\  

    \mc{1}{l}{\scriptsize{BSI Factor}} & \mc{1}{c}{\scriptsize{21 and 34}} & \mc{1}{c}{\scriptsize{-0.564}} & \mc{1}{c}{\scriptsize{-0.554}} & \mc{1}{c}{\scriptsize{-1.377}} & \mc{1}{c}{\scriptsize{-2.119}} & \mc{1}{c}{\scriptsize{-1.406}} & \mc{1}{c}{\scriptsize{-0.383}} & \mc{1}{c}{\scriptsize{-0.065}} & \mc{1}{c}{\scriptsize{-0.475}} \\  

     &  & \mc{1}{c}{\scriptsize{\textbf{(0.039)}}} & \mc{1}{c}{\scriptsize{\textbf{(0.078)}}} & \mc{1}{c}{\scriptsize{\textbf{(0.039)}}} & \mc{1}{c}{\scriptsize{\textbf{(0.020)}}} & \mc{1}{c}{\scriptsize{\textbf{(0.039)}}} & \mc{1}{c}{\scriptsize{(0.118)}} & \mc{1}{c}{\scriptsize{(0.490)}} & \mc{1}{c}{\scriptsize{\textbf{(0.059)}}} \\ 
    \hline  

    \\[0.1cm]
    \mc{2}{l}{\scriptsize{\% of Pos. TE ($H_0$: $\le$ 25\% $|$ 10\% Significance)}} & \mc{1}{c}{\scriptsize{45}} & \mc{1}{c}{\scriptsize{36}} & \mc{1}{c}{\scriptsize{73}} & \mc{1}{c}{\scriptsize{64}} & \mc{1}{c}{\scriptsize{82}} & \mc{1}{c}{\scriptsize{18}} & \mc{1}{c}{\scriptsize{0}} & \mc{1}{c}{\scriptsize{9}} \\  

     &  & \mc{1}{c}{\scriptsize{(0.196)}} & \mc{1}{c}{\scriptsize{(0.235)}} & \mc{1}{c}{\scriptsize{\textbf{(0.000)}}} & \mc{1}{c}{\scriptsize{(0.196)}} & \mc{1}{c}{\scriptsize{\textbf{(0.000)}}} & \mc{1}{c}{\scriptsize{(0.529)}} & \mc{1}{c}{\scriptsize{(1.000)}} & \mc{1}{c}{\scriptsize{(0.588)}} \\  

    \mc{2}{l}{\scriptsize{\% of Pos. TE ($H_0$: $\le$ 50\% $|$ 10\% Significance)}} & \mc{1}{c}{\scriptsize{45}} & \mc{1}{c}{\scriptsize{36}} & \mc{1}{c}{\scriptsize{73}} & \mc{1}{c}{\scriptsize{64}} & \mc{1}{c}{\scriptsize{82}} & \mc{1}{c}{\scriptsize{18}} & \mc{1}{c}{\scriptsize{0}} & \mc{1}{c}{\scriptsize{9}} \\  

     &  & \mc{1}{c}{\scriptsize{(0.510)}} & \mc{1}{c}{\scriptsize{(0.686)}} & \mc{1}{c}{\scriptsize{(0.235)}} & \mc{1}{c}{\scriptsize{(0.294)}} & \mc{1}{c}{\scriptsize{\textbf{(0.000)}}} & \mc{1}{c}{\scriptsize{(1.000)}} & \mc{1}{c}{\scriptsize{(1.000)}} & \mc{1}{c}{\scriptsize{(1.000)}} \\  

    \mc{2}{l}{\scriptsize{\% of Pos. TE ($H_0$: $\le$ 75\% $|$ 10\% Significance)}} & \mc{1}{c}{\scriptsize{45}} & \mc{1}{c}{\scriptsize{36}} & \mc{1}{c}{\scriptsize{73}} & \mc{1}{c}{\scriptsize{64}} & \mc{1}{c}{\scriptsize{82}} & \mc{1}{c}{\scriptsize{18}} & \mc{1}{c}{\scriptsize{0}} & \mc{1}{c}{\scriptsize{9}} \\  

     &  & \mc{1}{c}{\scriptsize{(0.863)}} & \mc{1}{c}{\scriptsize{(1.000)}} & \mc{1}{c}{\scriptsize{(0.471)}} & \mc{1}{c}{\scriptsize{(0.627)}} & \mc{1}{c}{\scriptsize{(0.471)}} & \mc{1}{c}{\scriptsize{(1.000)}} & \mc{1}{c}{\scriptsize{(1.000)}} & \mc{1}{c}{\scriptsize{(1.000)}} \\  

  \hline\hline
  \end{tabular}
    \begin{tablenotes}
    \scriptsize
    \item 
Note: This table displays various estimates of the treatment effect of ABC's center-based care.
Column (1) displays the ITT, without accounting for any controls.
Column (2) displays the ITT conditioning on vector of controls, $X$, consisting of APGAR scores 1 
minute after birth, an indicator for the subject being born prematurely, and an indicator for the 
father being home at baseline. We also apply IPW weights, $W$, to account for attrition.
Columns (3)--(4) are analogous to columns (1)--(2), but we restrict the control sample to subjects
who did not enroll in any alternative care.
Column (5) displys the matching estimate, where we use the Mahalanobis metric and Epanechnikov kernel
to match on controls $X$ listed above, and restrict the control sample to subjects who did not enroll
in any alternative care. Additionally, we apply IPW weights, $W$.
Columns (6)--(8) are analogous to Columns (3)--(5), except we restrict the control sample to subejcts
who did enroll in alternative care. 
The final three pairs of rows display the proportion of treatment effects in the table that are 
socially positive. The first row in each pair displays the percentage of treatment effects, and the
second row presents the inference.

Numbers in parentheses represent the $p$-value from a single hypothesis test, and are obtained from 
the empirical bootstrap distribution generated by 200 resamples of the original data. 
Bold $p$-values indicate significance at the 10\% level.
Blank point estimates indicate that we are unable to obtain estimates due to a lack of support in the data. 

    \end{tablenotes}
  \end{threeparttable}

\end{table}
\end{center}

\begin{center}
	  \begin{tabular}{cccccccccc}
  \toprule

    \scriptsize{Variable} & \scriptsize{Age} & \scriptsize{(1)} & \scriptsize{(2)} & \scriptsize{(3)} & \scriptsize{(4)} & \scriptsize{(5)} & \scriptsize{(6)} & \scriptsize{(7)} & \scriptsize{(8)} \\ 
    \midrule  

    \mc{1}{l}{\scriptsize{No trouble with spouse family}} & \mc{1}{c}{\scriptsize{30}} & \mc{1}{c}{\scriptsize{0.150}} & \mc{1}{c}{\scriptsize{0.107}} & \mc{1}{c}{\scriptsize{0.233}} & \mc{1}{c}{\scriptsize{0.624}} & \mc{1}{c}{\scriptsize{0.206}} & \mc{1}{c}{\scriptsize{0.133}} & \mc{1}{c}{\scriptsize{0.040}} & \mc{1}{c}{\scriptsize{-0.050}} \\  

     &  & \mc{1}{c}{\scriptsize{(0.197)}} & \mc{1}{c}{\scriptsize{(0.303)}} & \mc{1}{c}{\scriptsize{(0.263)}} & \mc{1}{c}{\scriptsize{\textbf{(0.053)}}} & \mc{1}{c}{\scriptsize{(0.263)}} & \mc{1}{c}{\scriptsize{(0.197)}} & \mc{1}{c}{\scriptsize{(0.434)}} & \mc{1}{c}{\scriptsize{(0.566)}} \\  

    \mc{1}{l}{\scriptsize{Get along well with spouse}} & \mc{1}{c}{\scriptsize{30}} & \mc{1}{c}{\scriptsize{-0.050}} & \mc{1}{c}{\scriptsize{-0.080}} & \mc{1}{c}{\scriptsize{0.033}} & \mc{1}{c}{\scriptsize{0.097}} & \mc{1}{c}{\scriptsize{0.078}} & \mc{1}{c}{\scriptsize{-0.067}} & \mc{1}{c}{\scriptsize{-0.131}} & \mc{1}{c}{\scriptsize{-0.033}} \\  

     &  & \mc{1}{c}{\scriptsize{(0.618)}} & \mc{1}{c}{\scriptsize{(0.658)}} & \mc{1}{c}{\scriptsize{(0.434)}} & \mc{1}{c}{\scriptsize{(0.474)}} & \mc{1}{c}{\scriptsize{(0.329)}} & \mc{1}{c}{\scriptsize{(0.566)}} & \mc{1}{c}{\scriptsize{(0.671)}} & \mc{1}{c}{\scriptsize{(0.579)}} \\  

    \mc{1}{l}{\scriptsize{No disagreement on living arrangement}} & \mc{1}{c}{\scriptsize{30}} & \mc{1}{c}{\scriptsize{0.217}} & \mc{1}{c}{\scriptsize{-0.136}} & \mc{1}{c}{\scriptsize{0.300}} & \mc{1}{c}{\scriptsize{0.238}} & \mc{1}{c}{\scriptsize{0.288}} & \mc{1}{c}{\scriptsize{0.200}} & \mc{1}{c}{\scriptsize{-0.255}} & \mc{1}{c}{\scriptsize{0.004}} \\  

     &  & \mc{1}{c}{\scriptsize{(0.132)}} & \mc{1}{c}{\scriptsize{(0.645)}} & \mc{1}{c}{\scriptsize{(0.250)}} & \mc{1}{c}{\scriptsize{(0.539)}} & \mc{1}{c}{\scriptsize{(0.263)}} & \mc{1}{c}{\scriptsize{(0.145)}} & \mc{1}{c}{\scriptsize{(0.803)}} & \mc{1}{c}{\scriptsize{(0.447)}} \\  

  \bottomrule
  \end{tabular}
\end{center}
\end{document}
%\input{2_abcsa}
%\input{3_abcare}
%\input{4_care}
%\input{5_carefam}
%\input{6_carefam_2sided}

\end{document}
